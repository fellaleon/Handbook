
\chapter*{Preface}

It was the summer of 2010 when I was working on my PhD in Cambridge, UK. In order to compensate the funding deficit, I did the supervision of couple of undergraduate courses. My way of doing it is to write down the whole standard solution of the example sheets and gave it the students just in case that they cannot understand my poor English properly. When I tried to wrap up these solutions and notes, I realized that there were lots of definitions, theorems and propositions missing from the existing work. So I decided to expand it to a handbook for myself and this rolling-snowball project became this book eventually.

Most of the contents are based on the lecture notes, example sheets and past papers of Cambridge University. Meanwhile, lots of friends threw small puzzles to me throughout time. These puzzles turned into the novel problems in the handbook and sometimes I had to write a new chapter to tackle them. Happiness and disappointment were both the theme of this progress as I usually broke the puzzle first and then discovered that it had been solved 50 years ago.

Nevertheless, I also invited other PhD students to join the project and luckily Zexiang Chen gave massive support and made some fabulous work on Algebra and Number Theory. Thanks also goes to Guolong Li and Qianli Xia who eventually escaped from mathematics for finance and astrophysics.

It is mentioned in Prime Obsession that mathematics has branches.
\bit
\item Arithmetic is the study of whole numbers and fractions.
\item Algebra is the use of abstract symbols to represent mathematical objects (numbers, lines, matrices, transformations), and the study of the rules for combining those symbols.
\item Analysis is the study of limits.
\item Geometry is the study of figures in space.
\eit

This handbook will deploy the building blocks in terms of these categories. Unlike other mathematics book, this handbook contains lots of details and hyperlinks which you may find either tedious or rigorous. It is indeed my belief of mathematics.
\begin{center}
There is no trivial thing in mathematics.
\end{center}%

I am a lucky person in different ways. First, I get the greatest parents. When I started this project, my parents asked me for its publication date. I said that it might be infinite, like their eternal love for me as I know they will always be with me. It was also a privilege to have Professor L.C.G. Rogers as my doctorial supervisor who is the most elegant and intellectual person I've ever met. This book was partly motivated by his prominent guidance. This handbook is still growing even though I barely have time to refine it. My wife can definitely claim this credit as she might be the only one can tolerate a useless stubborn lazy mathematician.

This book is the boundary of my perception of mathematics. However, it is not only for myself but also for my son George and other young students who want to explore the maths world. After all, maths is not trivial as it is actually the matter of our lives.

$\qquad$

\rightline{Liang Zhang$\qquad$}

\rightline{Shanghai, CN, 2018}
