\chapter{Rings}

\section{Rings and Polynomials}

\subsection{Definitions and Examples}

\begin{definition}[ring]\label{def:ring_group}
A set $R$ with two operations, $+$ and $\cdot$, forms a ring\index{ring!group} if
\ben
\item [(i)] $(R, +)$ is an abelian group,
\item [(ii)] multiplication is closed and associative and there is a multiplicative identity $1_R \in R$ such that
\be
\forall r \in R,\ 1_R \cdot r = r \cdot 1_R = r,
\ee
\item [(iii)] multiplication is distributive over addition, that is
\be
r_1\cdot (r_2 + r_3) = r_1\cdot r_2 + r_1\cdot r_3,\quad\quad (r_1 + r_2)\cdot r_3 = r_1\cdot r_3 + r_2\cdot r_3.
\ee
\een
\end{definition}

\begin{definition}[commutative ring\index{commutative!ring}]\label{def:commutative_ring}
A rings $R$ is commutative if the multiplication is commutative, i.e.,
\be
r_1\cdot r_2 = r_2\cdot r_1,\ \forall r_1,r_2 \in R.
\ee
\end{definition}

\begin{definition}[subring\index{subring!group} ]
A subset $S$ of $R$ is a subringif it is a ring under the restriction of the two operations. In particular, $1_R \in S$. The notation is $S \leq R$.
\end{definition}

\begin{definition}[field\index{field}]\label{def:field}
A field $\F$ is a commutative ring with an identity element 1, $1 \neq  0$, in which every non-zero element is invertible, i.e., $\forall a\in \F\bs\bra{0}$, $\exists b\in \F$ s.t. $ab = 1$.
%A field is a ring whose non-zero elements form a commutative group under multiplication.

Note that $\F_i$ denotes the field with $i$ elements.
\end{definition}

\begin{example}
\ben
\item [(i)] We have the four nested rings $\Z \leq \Q \leq \R \leq \C$.
\item [(ii)] The Gaussian integers $\Z[i] :=  \bra{a+bi:a,b\in \Z}\leq \C$, $i^2 = -1$ (see Definition \ref{def:gaussian_integer}).
\item [(iii)] $\Q[\sqrt{2}] = \bra{a + b\sqrt{2} : a, b \in \Q} \leq \R$.
\een
\end{example}

\begin{remark}
All fields are rings, e.g. $\Z$ modulo $n$, the set $\{0,\dots , n-1\}$ with addition and multiplication modulo $n$.
\end{remark}


\begin{definition}[unit]
A unit\index{unit!element of ring} of $R$ is an element with a multiplicative inverse in $R$, e.g. 2 is a unit in $\Q$, but not a unit in $\Z$.
\end{definition}

\begin{remark}%There are various conventions.%\ben%\item [(i)] Not everyone insists on a multiplicative identity $1_R$\footnote{need example}.\item [(ii)]
Some people confusingly talk about $1_R$ as the unit of $R$.%\een
\end{remark}

\begin{example}
The zero ring $\{0\}$. In this case the multiplicative identity is the same as the additive one. But in all other rings $0 \neq 1$.
\be
(0 + 0)\cdot r = 0\cdot r,\quad\quad (0 + 0)\cdot r = 0r + 0\cdot r
\ee
so $0r = 0$ for all $r \in R$. Hence 0 is not the multiplicative identity if $R$ has more than one element.
\end{example}

\begin{definition}[direct product of rings]\label{def:direct_product_ring}
Given rings $S$ and $T$, we can construct the direct product\index{direct product!rings} $R := S\times T = \bra{s,t|s\in R,t\in T}$ ($r = (s,t)$) with operation addition and multiplication
\be
(s_1,t_1)+(s_2,t_2) = (s_1+s_2,t_1+t_2),\qquad (s_1,t_1)(s_2,t_2) = (s_1s_2,t_1t_2).
\ee
\end{definition}

\begin{lemma}\label{lem:direct_product_of_rings}
If $S$ and $T$ are rings then $R:= S \times T$ is a also ring.
\end{lemma}

\begin{proof}[\bf Proof]
Let $r_1,r_2,r_3\in R$ with $r_1 = (s_1,t_1),r_2 = (s_2,t_2),r_3=(s_3,t_3)$ where $s_1,s_2,s_3\in R$ and $t_1,t_2,t_3\in T$.
\ben
\item [(i)] $(S\times T,+)$ is an abelian group,
\be
(s_1,t_1) + (s_2,t_2) = (s_1+s_2,t_1+t_2) = (s_2+s_1,t_2+t_1) = (s_2,t_2) + (s_1,t_1).
\ee
and
\be
r_1-r_2 = (s_1-s_2, t_1-t_2) \in R \text{ since }s_1-s_2\in S, t_1-t_2 \in T\text{ (Lemma \ref{lem:subgroup}).}
\ee

%Closed under the given operation, as $h_1h_2 \in H$ and $k_1k_2 \in K$.
\item [(ii)] $r_1r_2 = (s_1,t_1)(s_2,t_2) = (s_1s_2,t_1t_2) \in S\times T $ and $1_R := \bb{1_S,1_T}$.%Associative since multiplication in $H$ and $K$ is associative.
\item [(iii)] Distributivity is implied by %Identity $e = (e_H,e_K)$.%\item [(iv)] Inverse of $(h,k) = (h^{-1},k^{-1})$.
\beast
r_1(r_2 + r_3) & = & (s_1,t_1)(s_2+s_3, t_2+t_3) = (s_1s_2 + s_1s_3, t_1t_2 + t_1 t_3) = (s_1s_2, t_1t_2) + (s_1s_3,t_1t_3)\\
 & = & (s_1,t_1)(s_2,t_2) + (s_1,t_1)(s_3,t_3) = r_1r_2 + r_1r_3.
\eeast
\een
\end{proof}



\begin{definition}[idompotent]\label{def:idompotent_ring}
An element $r$ of a ring $R$ is idempotent if $r^2 = r$.
\end{definition}

\begin{definition}[multiplicative group\index{multiplicative group}]\label{def:multiplicative_group_ring}
The multiplicative group is a group under multiplication of the invertible elements of a field, ring, or other structure $R$ for which one of its operations is referred to as multiplication, written by $R^{\times}$.
\end{definition}


\subsection{Polynomials}

\begin{definition}[polynomial]
Let $R$ be a ring. A polynomial\index{polynomial} $f$ over $R$ is of the form
\be
f(X) = a_nX^n + \dots + a_1X + a_0
\ee
with $a_i \in R$. The degree\index{degree!polynomial} of $f$ is the largest $n$ with $a_n \neq 0$. $f$ is monic\index{monic!polynomial} if $a_n = 1$, where $n$ is the degree of $f$.
\end{definition}

\begin{definition}[polynomial ring\index{polynomial ring}]
$R[X]$ is the (commutative) polynomial ring over $R$ with ring operations
\be
(f + g)(X) = \sum_i (a_i + b_i)X^i,\quad\quad (fg)(X) = \sum_i \bb{\sum^i_{j=0} a_jb_{i-j} } X^i
\ee
where $f(X) = \sum a_iX^i$, $g(X) = \sum b_iX^i$.
\end{definition}

\begin{remark}
$R$ is a subring of $R[X]$ by identifying $R$ with the constant polynomials\index{constant polynomial} $f(X) = \sum a_iX^i$ with $a_i = 0$ for all $i \geq 1$.
\end{remark}

\begin{definition}[ring of formal power series\index{ring of formal power series}]
$R[[X]]$ is the (commutative) ring of formal power series
\be
f(X) = a_0 + a_1X + a_2X^2 + \dots
\ee
with the same addition and multiplication as for $R[X]$ but note that there is no restriction demanding that all but finitely many coefficients be zero (could be infinitely many).
\end{definition}

\begin{definition}[Laurent polynomials\index{Laurent polynomials}]
The (commutative) Laurent polynomials are
\be
f(X) = \sum_{i\in \Z} a_iX^i, \quad\quad (fg)(X) = \sum_i \bb{\sum_j a_jb_{i-j}} X^i
\ee
with the condition that all but finitely many $a_i$ are zero, e.g. $X^{-1} + X$.
\end{definition}

\begin{definition}[Laurent series\index{Laurent series}]
Laurent series are
\be
\sum_{i\in \Z} a_iX^i
\ee
with the condition that all but finitely many of the $a_i$ for $i \leq 0$ are zero. Addition and multiplication are defined in the same way as for Laurent polynomials.
\end{definition}

\begin{example}
The ring of all $R$-valued functions on a set $A$, $f : A \to R$. The ring operations can be defined by pointwise addition and multiplication,
\be
(f + g)(a) = f(a) + g(a),\quad\quad (fg)(a) = f(a)g(a).
\ee
\end{example}

\begin{example}
The set of continuous functions $f : \R \to \R$ forms a subring of the (commutative) ring of all functions $\R \to \R$. The set of continuous functions $\R \to \R$ contains the subring of polynomial functions\index{polynomial functions}, $f : \R \to \R$, $a \mapsto f(a)$, where $f(X) \in \R[X]$.
\end{example}

\begin{remark}
We can similarly consider Laurent polynomial functions, power series functions $\C \to \C$ and Laurent series functions $\C \to \C$.
\end{remark}


\section{Ring Homomorphism, Ideal, Quotient Ring and Isomorphism Theorems}

\subsection{Ring Homomorphism}

\begin{definition}[ring homomorphism]\label{def:ring_homomorphism_isomorphism}
Let $R$ and $S$ be rings (not necessarily commutative). A map $\theta : R \to S$ is a ring homomorphism\index{homomorphism!rings} if
\ben
\item [(i)] $\theta(r_1 + r_2) = \theta(r_1) + \theta(r_2)$,
\item [(ii)] $\theta(r_1r_2) = \theta(r_1)\theta(r_2)$,
\item [(iii)] $\theta(1_R) = 1_S$.
\een

A bijective homomorphism is called an isomorphism\index{isomorphism!rings}.
\end{definition}

\begin{definition}[kernel]\label{def:kernel_ring_homomorphism}
The kernel\index{kernel!ring homomorphism} of $\theta$ is $\ker \theta = \{r \in R : \theta(r) = 0\}$.
\end{definition}

\begin{lemma}\label{lem:injective_kernel_ring}
Let $\theta$ be a ring homomorphism. $\theta$ is injective if and only if $\ker \theta = \{0\}$.
\end{lemma}

\begin{proof}[\bf Proof.]
$\theta$ is a homomorphism of the additive groups (Definition \ref{def:group_homomorphism}). Then we have the required result by Lemma \ref{lem:injective_kernel_group}.
\end{proof}

\subsection{Ideal}

\begin{definition}[ideal\index{ideal}, left ideal\index{ideal!left ideal}, right ideal\index{ideal!right ideal}]\label{def:ideal_ring}
A subset $I$ of $R$ is an ideal, written $I \lhd R$, if
\ben
\item [(i)] $I$ is a subgroup of $R$ under addition,
\item [(ii)] whenever $a \in I$ and $r \in R$ then $ar \in I$ and $ra\in I$.
\een

The second condition is called the strong closure property\index{strong closure property!ring}.

A subset $I$ of $R$ is called a left ideal of $R$ if it is an additive subgroup of $R$ and absorbs multiplication on the left, i.e.,
\ben
\item [(i)] $(I,+)$ is a subgroup of $(R,+)$
\item [(ii)] $\forall a \in I$, $\forall r \in R$, then $ra \in I$.
\een%\footnote{need to check right-ideal and left-ideal}

A subset $I$ of $R$ is called a right ideal of $R$ if it is an additive subgroup of $R$ and absorbs multiplication on the right, i.e.,
\ben
\item [(i)] $(I,+)$ is a subgroup of $(R,+)$
\item [(ii)] $\forall a \in I$, $\forall r \in R$, then $ar \in I$.
\een%\footnote{need to check right-ideal and left-ideal}
\end{definition}

\begin{remark}
An ideal in general is not a subring.
\end{remark}

\begin{lemma}\label{lem:1_belongs_to_ideal_implies_ideal_equals_ring}
Let $I \lhd R$. Then $I = R$ if $1 \in I$.
\end{lemma}

\begin{remark}
The only ideal which is a subring of $R$ is $R$ itself.
\end{remark}

\begin{proof}[\bf Proof]
If $1 \in I$ then $r = 1 r \in I$ for all $r \in R$ and so $I = R$ if $1 \in I$.
\end{proof}


\begin{lemma}\label{lem:ring_field_only_ideal}
A non-zero commutative ring $R$ is a field if and only if its only ideals are $\{0\}$ and $R$.
\end{lemma}

\begin{proof}[\bf Proof]
Suppose that $R$ is a field and take an ideal $I \lhd R$. If $I \neq\{0\}$ then pick $0 \neq a \in I$. Then we can pick $a^{-1} \in R$ such that $1 = aa^{-1} \in I$. Hence $I = R$ by Lemma \ref{lem:1_belongs_to_ideal_implies_ideal_equals_ring}.

Conversely, assume that $\{0\}$ and $R$ are the only ideals. Take $a \in R$ with $a \neq 0$. Then $\bsa{a} \neq \{0\}$ and so $\bsa{a} = R$. Hence $1 \in R = \bsa{a}$ (by Definition \ref{def:ring_group}.(iii)) and there exists $r\in R$ (as $\bsa{a} = aR$) such that $ar = 1$. Thus, $R$ is a field.
\end{proof}


\begin{lemma}\label{lem:kernel_ring_ideal}
The kernel of a ring homomorphism $\theta : R \to S$ is an ideal.
\end{lemma}

\begin{proof}[\bf Proof]
$\theta$ is a homomorphism of the additive groups $(R, +) \to (S, +)$ and so $\ker \theta$ is an additive subgroup of $R$ by first isomorphism theorem of groups (Theorem \ref{thm:isomorphism_1_group}).

If $a \in \ker \theta$, then $\theta(a) = 0$ (by recalling Definition \ref{def:kernel_ring_homomorphism}). Consider $r\in R$,
\be
\theta(ar) = \theta(a)\theta(r) = 0\cdot\theta(r) = 0.
\ee

Thus $ar \in \ker \theta$, so we have the strong closure property.
\end{proof}

\begin{example}
\ben
\item [(i)] In a field $\F$, the only ideals are $\{0\}$ and $\F$. (Multiply any non-trivial element in the ideal by its inverse to see that 1 is in the ideal.)

\item [(ii)] In $\Z$ the ideals are of the form
\be
n\Z = \{\dots,-2n,-n, 0, n, 2n,\dots \}.
\ee

\begin{proof}[\bf Proof]
Certainly each set $n\Z$ is an ideal of $\Z$.

Suppose $I$ is a non-zero ideal. We show that all the non-zero additive subgroups of $\Z$ are of the form $n\Z$.

Let $n$ be the least positive element of $I$ and use Euclidean algorithm\footnote{need quote}. If $b \in I$ then $b = nq + r$ with $0 \leq r < n$ and some $q$. Note that $r = b - nq \in I$. Minimality of $n$ implies that $r = 0$ and hence $b = nq$. Thus $I = n\Z$.
\end{proof}
\een
\end{example}

\begin{proposition}\label{pro:union_ascending_ideal_is_ideal}
Let $R$ be a ring and $I_j \lhd R$ with $I_1 \subseteq I_2 \subseteq I_3 \subseteq \dots$. Then the union $I = \bigcup_j I_j$ is an ideal.
\end{proposition}
\begin{proof}[\bf Proof]
$\forall a,b\in I$, if $a \in I_j$ and $b \in I_k$, $j \leq k$ then $a \in I_k$, $b \in I_k$ implies $a + b \in I_k \subseteq I$.

$\forall a\in I, r\in R$, $a\in I_j$, $ar \in I_j \subseteq I$. Thus, $I \lhd R$.
\end{proof}


\begin{definition}[principal ideal\index{principal ideal}]\label{def:principal_ideal}
Let $R$ be a ring and $a \in R$. Then the ideal generated by $a$ is $aR = \{ar : r \in R\}$. We often use the notation $\bsa{a}$ for $aR$.

Note that this is the smallest ideal of $R$ containing $a$. Such an ideal is called a principal ideal.

More generally, the ideal generated by $a_1,\dots, a_k$ is
\be
\bsa{a_1,\dots , a_k} = a_1R +\dots a_kR = \left\{\sum^k_{i=1} a_ir_i : r_i \in R\right\}.
\ee

Even more generally, the ideal generated by a subset $A$ of $R$ is
\be
\bsa{A} = \left\{ \sum_{a\in A} ar_a :\text{ only finitely many $r_a \in R$ are non-zero }\right\}.
\ee
\end{definition}


\begin{example}
Examples of principal ideals include $n\Z$ in $\Z$ and $\bsa{X}$ in $\C[X]$, where $\bsa{X}$ is the ideal of polynomials with zero constant term.
\end{example}

\subsection{Quotient ring}

\begin{proposition}\label{pro:quotient_ring}
Let $I \lhd R$ be an ideal. Then the quotient ring\index{quotient ring} $R/I = \bra{r+I:r\in R}$ has elements consisting of the cosets $r + I$ for $r\in R$ with the operations
\be
(r_1 + I) + (r_2 + I) = (r_1 + r_2) + I, \quad\quad (r_1 + I)\cdot(r_2 + I) = r_1r_2 + I
\ee
for any $r_1,r_2\in R$. This defines a ring $R/I$.
\end{proposition}

\begin{proof}[\bf Proof]
From Theorem \ref{thm:quotient_group} we have already checked that $R/I$ forms a group under addition as defined. So we need to check that multiplication is well-defined, closed and associative. If
\be
r_1 + I = r'_1 + I,\quad \quad r_2 + I = r'_2 + I
\ee
then $r'_1 = r_1 + a_1$, $r'_2 = r_2 + a_2$ for some $a_1, a_2 \in I$, and therefore
\be
r'_1 r'_2 = (r_1 + a_1)(r_2 + a_2) = r_1r_2 + (a_1r_2 + r_1a_2 + a_1a_2)
\ee
where the second term is contained in $I$ by the strong multiplicative closure property of ideal. Thus $r'_1 r'_2 + I = r_1r_2 + I$ and therefore multiplication is well-defined.

The multiplicative identity is $1 + I$ since $(1 + I)(r + I) = r + I$.

Closure and associativity: $\forall r_1+I,r_2+I\in R/I, r_1,r_2 \in R$, by the closure of ring $R$,
\be
(r_1+I)(r_2 + I) = r_1r_2 + r_1 I + r_2 I + I = \underbrace{r_1r_2}_{\in R} + I \in R/I.
\ee

Thus, $R/I$ is closed. Also, by associativity of ring $R$
\be
(r_1+I)(r_2 + I) = r_1r_2 + r_1 I + r_2 I + I = r_1r_2 + I = r_2r_1 + I = (r_2 + I )(r_1 + I).
\ee

So, $\R/I$ is associative. $\forall r_1+I,r_2+I,r_3 + I\in R/I, r_1,r_2,r_3 \in R$, by distributivity of ring $R$,
\beast
(r_1+I)(r_2 + I + r_3 + I) & = & (r_1 + I)(r_2 + r_3 + I) = r_1(r_2 + r_3) + r_1I + r_2I + r_3I + I \\
& = & r_1r_2 + r_1r_3 + r_1I + r_2I + r_3I + I = r_1r_2 + r_1I + r_2I + I + r_1r_3 + r_1 I + r_3I + I\\
& = & (r_1 +I)(r_2 + I) + (r_1 +I)(r_3 + I).
\eeast

Hence, $R/I$ has distributivity. Therefore $R/I$ is a ring.
\end{proof}


\begin{example}
\ben
\item [(i)] $\Z/n\Z$ is a quotient ring.

$n\Z$ is an ideal of $\Z$. Elements $m + n\Z$ of $\Z/n\Z$ may be expressed as one of
\be
0 + n\Z, 1 + n\Z,\dots , (n - 1) + n\Z.
\ee
Addition and multiplication correspond to arithmetic modulo $n$.

\item [(ii)] Consider the (principal) ideal $I = (X)\lhd C[X]$. Elements $f(X)+I$ of the quotient ring $\C[X]/I$ may be expressed in the form $a + I$ where $a \in \C$ is the constant term of $f(X)$. Addition and multiplication correspond to that in $\C$,
\be
(a + I) + (b + I) = (a + b) + I,\quad\quad (a + I)(b + I) = ab + I.
\ee
Therefore, $\C[X]/I \cong \C$.
\een
\end{example}



\begin{proposition}[Euclidean algorithm for polynomials\index{Euclidean algorithm!polynomials}]\label{pro:euclidean_algorithm_polynomial}
Let $\F$ be a field and consider polynomials $f(X)$, $g(X) \in \F[X]$. Then we can write
\be
f(X) = g(X)q(X) + r(X)
\ee
with $\deg r < \deg g$.
\end{proposition}

\begin{proof}[\bf Proof]
Write
\be
f(X) = \sum^n_{i=0} a_iX^i,\quad\quad g(X) = \sum^m_{i=0} b_iX^i
\ee
with $\deg f(X) = n$ and $\deg g(X) = m$.

If $n < m$ then set $q(X) = 0$ and $r(X) = f(X)$. Now assume $n \geq m$, and argue by induction on $n$. Set $f_1(X) = f(X) - a_nb^{-1}_m X^{n-m}g(X)$ so $\deg f_1(X) < \deg f(X)$. If $m = n$ then $\deg f_1(X) < m$ and we can apply our first case to $f_1(X)$,
\be
f(X) = a_nb^{-1}_m g(X) + f_1(X).
\ee
If $n > m$ then we have by induction
\be
f_1(X) = g(X)q_1(X) + r_1(X)
\ee
for suitable $q_1(X)$ and $r_1(X)$, so
\be
f(X) = g(X)(q_1(X) + a_nb^{-1}_m X^{n-m}) + r_1(X).
\ee
\end{proof}

\begin{remark}
We require $\F$ to be a field as we need $b^{-1}_m$.
\end{remark}


\begin{example}\label{exa:x_square_plus_1_isomorphic_c}
Consider the ideal
\be
I = \bsa{X^2 + 1} \lhd \R[X].
\ee

For any $f(X) \in \R[X]$, we can write $f(X) = (X^2 + 1)q(X) + r(X)$ with $\deg r(X) \leq 1$. Thus $f(X) + I = r(X) + I$. The elements of $\R[X]/I$ are of the form $a + bX + I$.

Addition takes the form
\be
(a_1 + b_1X + I) + (a_2 + b_2X + I) = (a_1 + a_2) + (b_1 + b_2)X + I,
\ee
and similarly multiplication is given by ($XI = I$ since $I$ is ideal)
\beast
(a_1 + b_1X + I)(a_2 + b_2X + I) & = & a_1a_2 + (b_1a_2 + a_1b_2)X + b_1b_2X^2 + I\\
& = & (a_1a_2 - b_1b_2) + (b_1a_2 + a_1b_2)X + I.
\eeast

This corresponds to addition and multiplication in $\C$. In fact,
\beast
\R[X]/\bsa{X^2 + 1} & \cong & \C,\\
a + bX + I & \mapsto & a + bi.
\eeast
\end{example}

\subsection{Isomorphism theorems}


\begin{theorem}[first isomorphism theorem\index{isomorphism theorem!ring}]\label{thm:isomorphism_1_ring}
Let $\theta : R \to S$ be a ring homomorphism. Then $\ker \theta \lhd R$, $\im \theta \leq S$ and $R/ \ker \theta \cong \im \theta $.
\end{theorem}

\begin{proof}[\bf Proof]
Lemma \ref{lem:kernel_ring_ideal} states that $\ker \theta$ is an ideal. Therefore, $R/\ker\theta$ is a subring of $R$ by Proposition \ref{pro:quotient_ring}.

Now we show that $\im \theta$ is a subring of $S$,
\ben
\item [(i)] $\theta$ is a homomorphism of additive groups and so Lemma \ref{lem:image_subgroup} implies that $\im \theta$ is an additive subgroup of $S$.
\item [(ii)] $\theta(r_1)\theta(r_2) = \theta(r_1r_2)\in \im \theta$, so we have closure under multiplication. Associativity is inherited from $S$. Moreover, $\theta(1_R) = 1_S$.
\item [(iii)] $\forall s_1,s_2,s_3 \in S$ where $s_1 = \theta(r_1),s_2 = \theta(r_2),s_3 = \theta(r_3)$, $r_1,r_2,r_3\in R$,
\be
s_1(s_2 + s_3) = \theta(r_1)\bb{\theta(r_2) + \theta(r_3)} = \theta(r_1)\theta(r_2+r_3) = \theta\bb{r_1(r_2 + r_3)}
\ee

Then by distributivity of ring $R$,
\be
\theta\bb{r_1(r_2 + r_3)} = \theta (r_1r_2 + r_1r_3) = \theta (r_1r_2) + \theta(r_1r_3) = \theta (r_1)\theta(r_2) + \theta (r_1)\theta(r_3) = s_1s_2 + s_1 s_3.
\ee

Thus, $s_1(s_2 + s_3) = s_1s_2 + s_1 s_3$ and therefore $\im \theta$ is a subring of $R$.
\een

Let $\Phi: R/I \to \im \theta$, $r + I \mapsto \theta(r)$ and note $I = \ker \theta$. Consider $r_1+ I = r_2+I \in R/I$. We have
\be
r_1 - r_2 \in I = \ker\theta \ \ra\ \theta(r_1 - r_2) = 0 \ \ra\ \theta(r_1) = \theta(r_2)
\ee
which implies that $\Phi$ is well-defined. It is easy to see that $\Phi$ is bijective and thus it is an isomorphism of abelian additive groups from first isomorphism theorem of group (Theorem \ref{thm:isomorphism_1_group}). Note that it is due to the fact that any subgroup of abelian additive groups is normal ($\ker\theta$ is an ideal (subgroup under addition) and thus normal).

So it is left to check that $\Phi$ is a ring homomorphism. Accordingly, 
\beast
\Phi(r_1 + I + r_2 + I) & = & \Phi(r_1 + r_2 + I) = \theta (r_1 +r_2) = \theta (r_1) + \theta(r_2) = \Phi(r_1 + I) + \Phi(r_2 + I)\\
\Phi((r_1 + I)(r_2 + I)) & = & \Phi(r_1r_2 + I) = \theta(r_1r_2) = \theta(r_1)\theta(r_2) = \Phi(r_1 + I)\Phi(r_2 + I),\\
\Phi(1_R + I) & = & \theta(1_R) = 1_S.
\eeast

Thus $\Phi$ is a ring homomorphism.
\end{proof}


\begin{example}
The map
\be
\theta : \R[X] \to \C, \ \sum a_jX^j \mapsto \sum a_j i^j,\quad i^2 = -1
\ee
is a ring homomorphism with $\ker \theta = (X^2 + 1)$ (all the polynomials can be divided by $X^2 + 1$) and $\im \theta = \C$. (Comparing this with Example \ref{exa:x_square_plus_1_isomorphic_c})
\end{example}


\begin{theorem}[second isomorphism theorem\index{isomorphism theorem!ring}]\label{thm:isomorphism_2_ring}
Let $R$ be a subring of $S$ and $J \lhd S$ an ideal. Then $R \cap J\lhd R$ is an ideal,
\be
\bra{r + J : r \in R} = (R + J)/J \leq S/J
\ee
which means that $(R + J)/J $ is a subring of $S/J$ and 
\be
R/(R \cap J) \cong (R + J)/J.
\ee
\end{theorem}

\begin{proof}[\bf Proof]
Let $\theta$ be the map
\be
\theta : R \to S/J,\ r \mapsto  r + J.
\ee

Now $\forall r_1,r_2 \in R$,
\beast
\theta(r_1 + r_2) & = & r_1 + r_2 + J = r_1 + J + r_2 + J = \theta(r_1) + \theta(r_2)\\
\theta(r_1 r_2) & = & r_1 r_2 + J = r_1r_2 + r_1J + r_2J + J = (r_1+J)(r_2 + J) = \theta(r_1)\theta(r_2)\quad (\text{since $J$ is a ideal})\\
\theta(1_R) & = & 1_R + J = 1_S + J.
\eeast

Thus, $\theta$ is a ring homomorphism. By first isomorphism theorem of ring (Theorem \ref{thm:isomorphism_1_ring}), we have that
\beast
\ker \theta & = & \{r \in R : r + J = J\} = R \cap J \lhd R,\\
\im \theta & = & \{r + J : r \in R\} = (R + J)/J \leq S/J.
\eeast
such that $R/ \ker \theta \cong \im \theta$, that is, $R/(R \cap J) \cong (R + J)/J \leq S/J$.
\end{proof}





\begin{theorem}[third isomorphism theorem\index{isomorphism theorem!ring}]\label{thm:isomorphism_3_ring}
Let $I$ and $J$ be ideals of $R$, with $I \subseteq J$. Then $(R/I)/(J/I) \cong R/J$ where $J/I = \{r + I : r \in J\}$.
\end{theorem}

\begin{proof}[\bf Proof]
Let $\theta$ be the map
\be
\theta : R/I \to R/J,\ r + I \mapsto r + J.
\ee

This is well-defined as in Theorem \ref{thm:isomorphism_3_group}. $\forall r_1+I,r_2+I \in R/I$ where $r_1,r_2 \in R$, since $I,J$ are ideals,
\beast
\theta(r_1 + I + r_2 + I) & = & \theta (r_1 + r_2 + I) = r_1 + r_2 + J = r_1 + J + r_2 + J = \theta(r_1) + \theta(r_2)\\
\theta((r_1+I)(r_2+I)) & = & \theta(r_1r_2 + I) = r_1 r_2 + J = r_1r_2 + r_1J + r_2J + J = (r_1+J)(r_2 + J) = \theta(r_1+I)\theta(r_2+I)\\
\theta(1_R+I) & = & 1_R + J.
\eeast
%One can check it is a ring homomorphism and $\theta(1_R + I) = 1_R + J$.

The first isomorphism theorem of ring (Theorem \ref{thm:isomorphism_1_ring}) implies that $R/ \ker \theta \cong \im \theta \leq R/J$. Noting that
\be
\ker \theta = \{r + I : r \in J\} = J/I \lhd R/I,\quad\quad \im \theta = R/J,
\ee

That is, $(R/I)/(J/I) \cong R/J$.
\end{proof}


%\begin{proposition}\label{pro:remark_second_isomorphic_theorem_ring}
%\begin{remark}
%\end{remark}

\begin{theorem}[correspondence theorem of rings]\label{thm:correspondence_subring_ideal_containg_normal_subring_ideal_quotient_ring}
Let $R$ be a ring and $I$ be an ideal of $R$ ($I\lhd R$). Then there is a bijection between the set of all subrings of $R$ containing $I$ and the set of all subrings of $R/I$. That is,
\be
\sub(R:I) \cong \sub(R/I).
\ee

Also, there is a bijection between the set of all ideals of $R$ containing $I$ and the set of all ideals of $R/I$. That is,
\be
\bra{\text{ideals of $R$ containing $I$}}\ \longleftrightarrow \ \bra{\text{ideals of $R/I$}} 
\ee

In other words, for $I\subseteq J\subseteq R$,
\be
J\lhd R \ \lra \ (J/I) \lhd (R/I).
\ee
\end{theorem}

\begin{proof}[\bf Proof]


There is a one-to-one correspondence with maps $\Phi$ and $\Theta$ between additive groups,
\beast
\left\{\text{additive subgroups of $R$ containing ideal $I$}\right\} &  \longleftrightarrow & \left\{\text{additive subgroups of $R/I$}\right\} \\
\Phi: \ S & \longrightarrow & S/I \\
\Theta:\ \bra{r\in R:r+I\in T} & \longleftarrow & T 
\eeast
as in group correspondence theorem (Theorem \ref{thm:correspondence_subgroup_containg_normal_subgroup_quotient_group}) since additive group is abelian (due to definition of ring $R$) and every subgroup is normal. Note that $r+I$ is the equivalent form of $rI$ under addition.

%This correspondence induces\footnote{This needs to be checked.}%Proposition \ref{pro:remark_second_isomorphism_theorem_group}

First, we want to show that there is one-to-one correspondence between
\be
\bra{\text{subrings of $R$ containing ideal $I$}} \ \longleftrightarrow \ \bra{\text{subrings of $R/I$}}
\ee
and it suffices to show that $\Phi$ and $\Theta$ maps subrings to subrings between the rings. Also, since $R$ and $R/I$ are both rings (by Proposition \ref{pro:quotient_ring}), distributive over addition and associative property can be easily inherited from $R$ and $R/I$. Thus, we only check the conditions that multiplication is closed and the multiplicative idenity is contained in the corresponding set.

Let $S\in \sub(R:I)$ and $\Phi$ maps $S$ to $S/I = \bra{s+I: s\in S}$. Then for any $s_1+I,s_2+I\in S/I$ with $s_1,s_2\in S$, we have that $s_1s_2\in S$. Thus, 
\be
(s_1+I)(s_2+I) = s_1s_2 + I \in S/I \ \ra\ S/I \text{ is closed.}
\ee

We know that $1_R+I$ is the identity of $R/I$ since $(r+I) (1_R+I) = r+I = (1_R+I)(r+I)$ for any $r+I \in R/I$. Also, $1_R$ is the identity of $S$ since $S$ is a subring of $R$. Thus, $s1_R = s = 1_Rs$ for any $s\in S$. Thus, for any $s+I \in S/I$, 
\be
(s+I)(1_R+I) = s1_R + I = s + I = 1_R s + I = (1_R+I)(s+I)
\ee
which implies that $1_R+I$ is also the idenity of $S/I$. Thus, $S/I$ is a subring.

Meanwhile, consider the $T \in \sub(R/I)$ and $\Theta$ maps $T$ to $\bra{r\in R: r+I\in T}$. For any $r_1,r_2\in \bra{r\in R: r+I = T}$, we have $r_1+I,r_2+I \in T$ and thus
 Then
\be
r_1r_2 + I = (r_1+I)(r_2+I) \in T \ \ra \ r_1r_2 \in \bra{r\in R: r+I\in T}\ \ra\  \bra{r\in R: r+I\in T}\text{ is closed.}
\ee

Also, $1_R + I$ is the identity of $R/I$ and $T$ since $T$ is a subring of $R/I$. Thus, $1_R\in \bra{r\in R:r+I \in T}$ as $1_R+I \in T$. 

Second, we want to show that there is one-to-one correspondence between
\be
\bra{\text{ideals of $R$ containing  ideal $I$}} \ \longleftrightarrow \ \bra{\text{ideals of $R/I$}}
\ee
and it suffices to show that $\Phi$ and $\Theta$ maps ideals to ideals between the rings. Also, since $R$ and $R/I$ are both rings, we conly check the condition that the multiplication of elements of ideal and ring is in the ideal.

Let $I\lhd J\lhd R$ and $\Phi$ maps $J$ to $J/I = \bra{j+I: j\in J}$. Then for any $j+I \in J/I$ and $r+I \in R/I$ with  $j\in J$ and $r\in R$. Since $J$ is an ideal, we have that $jr,rj\in J$. Therefore,
\be
(j+I)(r+I) = jr + I \in J/I,\ (r+I)(j+I) = rj + I \in J/I
\ee
which implies that $J/I$ is an ideal in $R/I$.

On the other hand, we consider the $T \lhd R/I$ and $\Theta$ maps $T$ to $\bra{r\in R: r+I\in T}$. For any $t\in \bra{r\in R: r+I \in T}$ and $r\in R$, we have $t + I\in T$. Since $T$ is an ideal, we have
\be
(t+I)(r+I) = tr+ I \in T,\ (r+I)(t+I) = rt + I \in T \ \ra\ tr, rt \in \bra{r\in R: r+I \in T}
\ee
which implies that $\bra{r\in R: r+I \in T}$ is an ideal in $R$. %Therefore, we have proved all the required result.
\end{proof}


\subsection{Characteristic of ring}

\begin{definition}[characteristic of ring\index{characteristic!ring}]\label{def:characteristic_ring}
Given a ring $R$, there is a unique ring homomorphism
\be
\phi: \Z \to R,\ 1 \mapsto  1_R, \ m \mapsto  \underbrace{1 + \dots + 1}_{m \text{ times}}
\ee

The first isomorphism theorem (Theorem \ref{thm:isomorphism_1_ring}) implies $\Z/ \ker \phi \cong \im \phi \leq R$. $\im \phi$ is the prime subring\index{prime subring}\footnote{need explanation} of $R$. $\ker \phi \lhd \Z$ is an ideal and so is of the form $n\Z$ for some $n$ and $\im \phi \cong \Z/n\Z$.

This $n$ is the characteristic of ring $R$.
\end{definition}

\begin{remark}
If $R$ is one of $\Z$, $\Q$, $\R$, or $\C$ then the characteristic is 0. If $R = \Z/p\Z$ and $p$ is prime, then the chacteristic is $p$.
\end{remark}

\section{Integral Domain, Associativity, Irreducibility, Primeness and Field of Fractions}

\subsection{Integral domain}

\begin{definition}[zero divisor\index{zero divisor}]\label{def:zero_divisor_ring}
A zero divisor a is non-zero and there exists $b \neq 0$ such that $ab = 0$.
\end{definition}

\begin{definition}[integral domain\index{integral domain}]\label{def:integral_domain_ring}
A commutative ring is an integral domain if $ab = 0$ implies $a = 0$ or $b = 0$ for all $a, b \in R$.
\end{definition}

\begin{remark}
In an integral domain there are no zero divisors.
\end{remark}

\begin{example}
$\Z$ is an integral domain. All fields are integral domains. Subrings of integral domains are integral domains, e.g. $\Z[i] \leq \C$.
\end{example}


\begin{lemma}\label{lem:integral_domain_injective}
Let $R$ be an integral domain and $\phi$ be a map $\phi:\R\to \R$, $r \mapsto ar$ for $a\neq 0$. Then $\phi$ is injective.
\end{lemma}

\begin{proof}[\bf Proof]
If $ar_1 = ar_2$ then $a(r_1 - r_2) = 0$ and so $r_1 - r_2 = 0$, i.e. $r_1 = r_2$.

Note that cancellation is valid in integral domains.
\end{proof}

\begin{lemma}
Let $R$ be a finite integral domain. Then $R$ is a field.
\end{lemma}

\begin{proof}[\bf Proof]
Consider the map $\phi:R \to R$, $r \mapsto ar$, that is, multiplication by $a \neq 0$. Then this map is injective by Lemma \ref{lem:integral_domain_injective}. Since $r\mapsto ar$ is injective, the image is at least as big as the domain. Since $R$ is finite the map is also surjective. So the map is bijective. Therefore there exists $r \in R$ with $ar = 1$. Thus a has a multiplicative inverse. So $R$ is a field.
\end{proof}


\begin{lemma}
If $R$ is an integral domain then $R[X]$ is an integral domain.
\end{lemma}
\begin{proof}[\bf Proof]
If
\be
f(X) = \sum a_iX^i,\quad\quad g(X) = \sum b_iX^i
\ee
with $\deg f = m$ and $\deg g = n$ then $f(X)g(X)$ has degree $m+n$ since $a_mb_n \neq 0$ as $R$ is an integral domain. Therefore,
\be
f(X) \neq 0,\ g(X) \neq 0 \ \ra\ f(X)g(X) \neq 0.%\qquad (\text{converse of the definition }).
\ee
Thus $R[X]$ is an integral domain.
\end{proof}

\begin{remark}
By induction, $R[X_1,\dots ,X_n]$ polynomial ring in indeterminates $X_1,\dots ,X_n$ is an integral domain if $R$ is. This is because $R[X_1,X_2]$ may be regarded as $\bsa{R[X_1]}[X_2]$.
\end{remark}

\begin{theorem}[Lagrange's theorem\index{Lagrange's theorem!ring}]\label{thm:lagrange_ring}
Let $R$ be an integral domain. Then a polynomial in $R[X]$ of degree $d$ can have at most $d$ roots.
\end{theorem}

\begin{proof}[\bf Proof]
Since $R$ is an integral domain, we have that $R[X]$ is an integral domain (see lemma in notes). Let $f(X)\in R[X]$ with $\deg(f(X))=d$ and
\be
f(X) = \sum^d_{i=0} a_i X^i,\quad a_d \neq 0 .
\ee

If $f(X)$ has no roots, we are done. Suppose $a$ is a root of $f(X)$ then
\be
f(X) = (X-a)g(X)
\ee
for some $g(X)$. Suppose that
\be
g(X) = \sum^m_{i = 0} b_i X^i.
\ee

If $m \geq d$, we have the leading term of $f(X) = (X-a)g(X)$ is $b_m X^{m+1} = 0$ since $\deg(f(X)) = d$. Then we have that
\be
b_m X^{m+1} = 0 \ \ra \ b_m = 0.
\ee
since $R[X]$ is an integral domain.

Thus, $m<d$ and thus consider the leading term $f(X)$ which is $b_{d-1}X^d = a_d X^d$, so $b_{d-1} = a_d \neq 0$ since $R[X]$ is integral domain. Thus, $g(X)$ has degree $d-1$. So apply this method inductively, we have that $f(X)$ has at most $d$ roots.
\end{proof}

%\begin{remark}
%By considering $R= \Z/p\Z$ where $p$ is prime, let
%\be
%f(X) = X^{p-1} -1 -\prod^{p-1}_{i=1}(X-i)
%\ee
%we have another proof of Wilson's theorem (Theorem %\ref{thm:wilson_number_theory}) by using Lagrange's theorem. Indeed, $f%$ has degree at most $p-2$ but $X=1,\dots, p-1$ are roots of $f$. Therefore, $f=0$ by Lagrang's theorem. By considering the constant term we have $-1\equiv (p-1)!\lmod{p}$.
%\end{remark}

\begin{example}
Consider $f(X) = X^2 -1$ in $\Z/8\Z[X]$, we have roots $1,3,5,7$ (more than two roots) as we know that $\Z/8\Z$ is not an integral domain. %Give a quadratic polynomial in $(\Z/8\Z)[X]$ that has more than two roots.
\end{example}



\subsection{Associativity, irreducibility, primeness}

%Throughout this section $R$ is assumed to be Let $R$ be an integral domain.

\begin{definition}[associativity, irreducibility, primeness]\label{def:associativity_irreducibility_primeness_integral_domain}
Let $R$ be an integral domain.

An element $a \in R$ is a unit\index{unit!element of ring} if it has a multiplicative inverse. Equivalently, $\bsa{a} = aR = R$.

We say $a$ divides $b$, written $a \mid b$, if there is $c \in R$ such that $b = ac$. Equivalently, $\bsa{b} \subset \bsa{a}$, i.e., $bR \subset aR$.

Elements $a$ and $b$ are associates\index{associates!element of ring} in $R$ if $a = bc$ for some unit $c \in R$. Equivalently, $\bsa{a}= \bsa{b}$.

$r \in R$ is irreducible\index{irreducible!element of ring} in $R$ if it is non-zero, not a unit and whenever $r = ab$ with $a, b \in R$ then $a$ or $b$ is a unit.

$r \in R$ is prime\index{prime!element of ring} in $R$ if it is non-zero, not a unit and if $r \mid ab$ then $r \mid a$ or $r \mid b$.
\end{definition}

\begin{remark}
These definitions do depend on the ring $R$.

2 is prime and irreducible in $\Z$, but not in $\Q$.

$2X$ is irreducible in $\Q[X]$ but not in $\Z[X]$, as $2X = 2 \cdot X$ and 2, $X$ are not units in $\Z[X]$.
\end{remark}



\begin{lemma}\label{lem:prime_implies_irreducible_integral_domain}
Let $R$ be an integral domain. If $r$ is prime in $R$ then $r$ is irreducible in $R$.
\end{lemma}

\begin{proof}[\bf Proof]
Suppose that $r$ is prime in $R$ and $r=ab$ with $a,b\in R$. Then $r\neq 0$ and $r | ab$ and so $r | a$ or $r | b$ by Definition of primeness. Without loss of generality suppose $r | a$. So $a = qr$ for some $q \in R$. So $r = ab = qrb$. Cancellation in integral domains gives $1 = qb$, so $b$ is a unit.
\end{proof}


\begin{definition}[norm\index{norm!element of ring}]\label{def:norm_ring}
A norm of element $z$ of ring over $\C$ is $N(z) = z\bar{z}$. $N$ is multiplicative, that is, $N(z_1z_2) = N(z_1)N(z_2)$.
\end{definition}



\begin{example}\label{exa:z_sqrtroot_minus_5_prime_irreducible}
Let $R$ be an integral domain. To show that the converse of Lemma \ref{lem:prime_implies_irreducible_integral_domain} does not hold in general, consider $R = \Z[\sqrt{-5}] \leq \C$. $R$ is an integral domain since it is a subring of a field. Define a norm,
\be
N(a + b\sqrt{-5}) = a^2 + 5b^2 \in \Z^+\cup\bra{0}.%_{\geq 0}.
\ee

Suppose $z$ is a unit and so there exists $z_1\in R$ with $zz_1 = 1$ so $N(z)N(z_1) = N(zz_1) = N(1) = 1$ hence $N(z)$ is a unit in $\Z$ and also $N(z) \geq 0$. So $N(z) = 1$ is the only possibility. Thus, the units of $R$ are precisely the elements of norm 1, namely $\pm 1$. %(Since for a unit $c \in R$, there exists $d\in R$ such that $cd =1$. So $N(cd) = 1$)
There are no elements in $R$ of norm 2 or 3 (since we cannot solve $a^2 + 5b^2 = 2$ or 3.)

Consider the identity $6 = 2 \cdot 3 = (1 + \sqrt{-5})(1 -\sqrt{-5})$ in $R$.

To see that 2 is irreducible, express $2 = z_1z_2$ and consider norms, $4 = N(z_1)N(z_2)$. Since there are no elements of norm 2 one of the $N(z_j) = 1$ so $z_j$ is a unit.

But 2 is not prime in $R$ since $2 | (1 + \sqrt{-5})(1 - \sqrt{-5})$ but 2 does not divide either $1 \pm \sqrt{-5}$. (Consider norms, $N(2) = 4$, $N(1 \pm
\sqrt{-5}) = 6$ and $4 \nmid 6$.)

Similarly, 3, $1 +\sqrt{-5}$, $1 -\sqrt{-5}$ are irreducible.
\end{example}



\subsection{Field of fractions}

\begin{theorem}[field of fractions\index{field of fractions}]\label{def:field_of_fractions}
Let $R$ be an integral domain. Then there is a field $\F$ with the properties
\ben
\item [(i)] $R \leq \F$ is a subring,%\item [(ii)] $\F$ is a field,
\item [(ii)] every element of $\F$ has the form $ab^{-1}$ where $a \in R$ and $b^{-1}$ is the multiplicative inverse in $\F$ of $b \in R$.
\een
The field $\F$ is called of field of fractions.
\end{theorem}

\begin{example}
With the notation as above, $R = \Z$, $\F = \Q$.
\end{example}



\begin{proof}[\bf Proof]
Consider pairs $(a, b)$ with $a \in R$, $b \in R$ and $b \neq 0$. Define an equivalence relation\footnote{need quote}
\be
(a, b) \sim (c, d) \ \lra \ ad = bc.
\ee
(For transitivity, note that
\be
\ba{l}
(a, b) \sim (c, d) \ \lra \ ad = bc\\
(c, d) \sim (e, f) \ \lra\ cf = de
\ea \ \ra \ adf = bcf = bde
\ee
as cancellation is valid in $R$ (integral domain). So $af = be$ (since multiplication is commutative) and $(a, b) \sim (e, f)$.)

Write $a/b$ or $\frac ab$ for the equivalence class of $(a, b)$.\footnote{Note that `$/$' is a new operation rather than the division here.} We define addition by
\be
\frac ab + \frac cd = \frac{ad + bc}{bd}
\ee
and multiplication by
\be
\frac ab \cdot \frac cd = \frac{ac}{bd}.
\ee

One can easily check that these operations are well-defined\footnote{need to check}.

If we let $\F = \{a/b : a \in R, b \in R, b \neq 0\}$ then $\F$ is a ring under these operations.

$R$ may be identified with the subring of elements of the form $r/1$. The multiplicative identity is $1/1$. $a/b$ has multiplicative inverse $b/a$ if $a \neq 0$,
\be
\frac ab \frac ba = \frac {ab}{ab} = \frac 11.
\ee

Thus, $\F$ is a field. Every element $a/b$ is of the form $(a/1)(b/1)^{-1}$ since $(b/1)^{-1} = 1/b$.
\end{proof}


\section{Modular Rings}

\subsection{Chinese remainder theorem}

\begin{theorem}[Chinese remainder theorem\index{Chinese remainder theorem!rodular ring}]\label{thm:chinese_remainder_modular_ring}
Let $n = \prod^k_{i=1}n_i = n_1 n_2 \dots n_k $ where $n_1,\dots,n_k\in \Z^+$ are pairwise coprime. Then we have a ring isomorphism
\be
\Z/n\Z \cong \Z/n_1\Z \times \dots \times \Z/n_k\Z = \prod^k_{i=1} \Z/n_i\Z.
\ee
\end{theorem}

\begin{remark}
We can introduce the argument in proof of Theorem \ref{thm:chinese_remainder_number_theory} to prove this modular ring version of Chinese remainder theorem.
\end{remark}

\begin{proof}[\bf Proof]
Consider the mapping $\theta$:
\be
\Z/n\Z \to \Z/n_1\Z \times \dots \times \Z/n_k\Z,\quad r \mapsto \bb{r\lmod{n_1},\dots, r\lmod{n_k}}.
\ee

First, we can see the mapping is well-defined. By Lemma \ref{lem:direct_product_of_rings}, we know that $\prod^k_{i=1} \Z/n_i\Z$ is ring. Thus, $r_1,r_2 \in \Z/n\Z$, by basic properties of product ring we have
\beast
\theta(r_1+r_2) & = & \bb{(r_1+r_2)\lmod{n_1},\dots, (r_1+r_2)\lmod{n_k}} \\
& = & \bb{(r_1\lmod{n_1},\dots, r_1\lmod{n_k}} + \bb{(r_2\lmod{n_1},\dots, r_2\lmod{n_k}} \\
& = & \theta(r_1)+ \theta(r_2).
\eeast

\beast
\theta(r_1r_2) & = & \bb{(r_1r_2)\lmod{n_1},\dots, (r_1r_2)\lmod{n_k}} \\
& = & \bb{(r_1\lmod{n_1},\dots, r_1\lmod{n_k}}\cdot \bb{(r_2\lmod{n_1},\dots, r_2\lmod{n_k}} \\
& = & \theta(r_1)\theta(r_2).
\eeast

For $r = 1_{\Z/n\Z}$, we have that
\be
\theta(r) = \bb{r\lmod{n_1},\dots,r\lmod{n_k}} = \bb{1_{\Z/n_1\Z},\dots,1_{\Z/n_k\Z}}
\ee
which is the multiplicative identity of product ring. Thus, $\theta$ is a ring homomorphism.

For $r_1,r_2\in \Z/n\Z$, if $r_1 \lmod{n_i} = r_2 \lmod{n_i}$ for all $i$, then $(r_1-r_2)\mid n_i$ for all $i$ thus $(r_1-r_2)\mid n$ since $n_i$ are pairwise coprime. Hence, $r_1\equiv r_2 \lmod{n}$ implies that $\theta$ is injective.

Also, $\forall c_i\in \Z/n_i\Z$ for all $i$, we can find $r_i\in \Z/n_i\Z$ by Lemma \ref{lem:congruence_equation_soluble_iff_gcd_division}
\be
m_i r_i \equiv c_i \lmod{n_i}
\ee
where $m_i = n/n_i$ since $\hcf(m_i,n_i) =1\mid c_i$. Then we can construct
\be
r = \sum^k_{i=1}m_i r_i \qquad\text{s.t. }r\equiv m_ir_i \equiv c_i \lmod{n_i}\text{ for all }i.
\ee

Hence, $\theta$ is surjective and therefore bijective. Then this mapping is a ring isomorphism.
\end{proof}

In particular, we have the following corollary based on fundamental theorem of arithmetic.

\begin{corollary}\label{cor:chinese_remainder_modular_ring_prime_product}
Let integer $n = \prod^k_{i=1}p_i^{a_i}$ where $p_1,\dots,p_k\in \Z^+$ are distinct primes and $a_i\geq 1$. Then we have a ring isomorphism
\be
\Z/n\Z \cong \Z/p_1^{a_1}\Z \times \dots \times \Z/p_k^{a_k}\Z = \prod^k_{i=1} \Z/p_i^{a_i}\Z.
\ee
\end{corollary}

\begin{remark}
We can also consider this corollary as the decomposition rule of modular rings.
\end{remark}


\subsection{Multiplicative groups of integers modulo $n$}


%In the case of a field $\F$, the group is $\F^{\times} := (\F \bs \bra{0}, \cdot)$, where 0 refers to the zero element of $\F$ and the binary operation $\cdot$ is the field multiplication.

Recalling Definition \ref{def:multiplicative_group_ring} we restate the definition of multiplicative group of integers modulo $n$.

\begin{definition}[multiplicative group of integers modulo $n$\index{multiplicative group!integers modulo $n$}]
The multiplicative group of integers modulo $n$ under multiplication of the invertible elements of the ring $\Z/n\Z$, written by $\bb{\Z/n\Z}^{\times}$.
\end{definition}

For multiplicative groups of modulo $n$

\begin{example}
\ben
\item [(i)] The multiplicative group of quotient ring $\Z/p\Z$ with prime $p$ (actually a filed, see Corollary \ref{cor:znz_field_iff_prime}) is
\be
\bb{\Z/p\Z}^\times = \bb{\Z/p\Z}\bs \bra{0} = \bra{1,2,\dots,p-1}.
\ee

In particular, if $p = 7$,
\be
\bb{\Z/p\Z}^\times = \bb{\Z/7\Z}^\times = \bra{1,2,3,4,5,6}.
\ee

\item [(ii)] The multiplicative group of quotient ring $\Z/8\Z$ is
\be
\bb{\Z/8\Z}^\times = \bra{1,3,5,7}.
\ee
\een
\end{example}

\begin{theorem}\label{thm:multiplicative_group_znz_size_euler_totient}
The multiplicative group of the quotient ring $\Z/n\Z$, written $\bb{\Z/n\Z}^\times$, has size $\phi(n)$ where $\phi$ is Euler's totient function.
\end{theorem}

\begin{proof}[\bf Proof]
We know that $\bb{\Z/n\Z}^\times$ contains all the unit of $\Z/n\Z$. This means $a\in \bb{\Z/n\Z}^\times$ if there exists $ax \equiv 1 \lmod{n}$, which is equivalent to $\hcf(a,n)=1$ by Lemma \ref{lem:coprime_congruence_generator_group}. Thus, the size of $\bb{\Z/n\Z}^\times$ is $\phi(n)$, the size of
\be
\bra{a\in \Z^+: 1\leq a\leq n, \hcf(a,n)=1}.
\ee
\end{proof}

\begin{corollary}\label{cor:znz_field_iff_prime}
$\Z/n\Z$ is a field if and only if $n$ is prime.
\end{corollary}

\begin{proof}[\bf Proof]
$\Z/n\Z$ is a field if and only if every non-zero element is a unit, if and only if $\phi(n) = n-1$ (by Theorem \ref{thm:multiplicative_group_znz_size_euler_totient}), if and only if $n$ is a prime.
\end{proof}

\begin{theorem}
The multiplicative groups of $\Z/2^n\Z$ have the following isomorphism.
\be
\bb{\Z/2\Z}^\times \cong C_1,\qquad \bb{\Z/4\Z}^\times \cong C_2,\qquad \bb{\Z/2^n\Z}^\times \cong C_2\times C_{2^{n-2}},\quad n\geq 2.
\ee
\end{theorem}

\begin{proof}[\bf Proof]
First, we know that the order of $(\Z/2^n\Z)^\times$ is $2^{n-1}$ (containing all odd numbers smaller $2^n$).

For $n=1,2$, we can easily see that
\be
\bb{\Z/2\Z}^\times = \bra{1} \cong C_1,\qquad \bb{\Z/4\Z}^\times = \bra{1,3} \cong C_2.
\ee

For $n=3$, we have
\be
\bb{\Z/8\Z}^\times = \bra{1,3,5,7} \cong C_2\times C_2
\ee
since orders of $3,5,7$ are 2 as
\be
3^2 = 9 \equiv 1 \lmod{8},\qquad 5^2 = 25 \equiv 1 \lmod{8},\qquad 7^2=49 \equiv 1\lmod{8}.
\ee

For $n=4$, we have $\bb{\Z/16\Z}^\times$ is not cyclic and
\be
\bb{\Z/16\Z}^\times = \bra{1,3,5,7,9,11,13,15} \cong C_2\times C_4
\ee
since orders of $3$ are 4 as
\be
3^2 = 9\not\equiv 1 \lmod{16},\qquad 3^4 = 81 \equiv 1 \lmod{16}
\ee
and no element has higher order $8$ as
\beast
5^4 = 25^2 \equiv 9^2 = 81 \equiv 1 \lmod 16,& &  7^4 = 49^2 \equiv 1^2 = 1\lmod{16},\quad 9^4 = 81^2 \equiv 1^2 = 1 \lmod{16}, \\
11^4 = 121^2 \equiv (-7)^2 = 49 \equiv 1 \lmod 16,& & 13^4 = 169^2 \equiv 9^2 = 81\equiv 1 \lmod{16},\quad 15^4 \equiv (-1)^4 = 1 \lmod{16}.\nonumber
\eeast

So we consider the case $n\geq 5$. Picking $\pm 1,2^{n-1}\pm 1 \in (\Z/2^n\Z)^\times$, we can see that these four elements form a subgroup of $(\Z/2^n\Z)^\times$. However, we see that this group is not cyclic as
\be
\bra{\pm 1,2^{n-1}\pm 1} \cong C_2 \times C_2 \not\cong C_4.
\ee

Therefore, $(\Z/2^n\Z)^\times$ is not cyclic by Lemma \ref{lem:subgroup_of_cyclic_group_is_cyclic}. Then we can have that
\be
\bb{\Z/2^n\Z}^\times \cong C_2 \times C_{2^{n-2}}
\ee
by Theorem \ref{thm:finite_abelian_p_group_direct_product_maximal_element} given that there is an element order $2^{n-2}$, indeed, this is 3. This can be done induction. If 3 has order $2^{n-3}$ with respect to modulo $2^{n-1}$ (which is true for $n=1,2,3,4,5$), then $k< n-3$
\be
3^{2^k} \not\equiv 1 \lmod{2^{n-1}} \ \ra\ 3^{2^k} \not\equiv 1 \lmod{2^n}.
\ee

Thus, it suffices to show that for $n\geq 5$,
\beast
& & 3^{2^{n-3}} \not\equiv 1 \lmod{2^n} \qquad (*)\\
& & 3^{2^{n-2}} \equiv 1 \lmod{2^n} \qquad (\dag)
\eeast

%Since $3^{2^{n-3}} \equiv 1 \lmod{2^{n-1}}$, we have
%\be
%3^{2^{n-3}} = 1 + 2^{n-1}x \ \ra\ \ 3^{2^{n-2}} = 9 \bb{1 + 2^{n-1}x} = (1+8) \bb{1 + 2^{n-1}x} \equiv 1 + 2 + 2^{n-1}x \not\equiv 1 \lmod{2^{n}}.
%\ee
%
%Furthermore,
%\be
%3^{2^{n-2}} =
%\ee

Then for $k\geq 2$
\be
3^{2^{k}} = (1+2)^{2^{k}} = 1 + \sum^{2^k}_{i=1} 2^i \binom{2^k}{i} =  1 + \sum^{2^k}_{i=1} \frac{2^i \bb{2^k}!}{i!\bb{2^k-i}!}.
\ee

%For any $i\geq 1$, we have that by fundamental theorem of arithmetic $i! = p^{a_i}z$ for some positive integer $z$. Then $a_i < i$ so $p\mid \frac{p^i}{p^{a_i}}$. Also we have $p^{n-1}\left|\frac{\bb{p^{n-1}}!}{\bb{p^{n-1}-i}!}\right.$. Therefore,
%\be
%p^n \left| \frac{p^i \bb{p^{n-1}}!}{i!\bb{p^{n-1}-i}!}\right.
%\ee
%which implies that

Then by lemma \ref{lem:prime_power_divides_factorial}, the largest $a$ such that $2^a$ divides $i!$, largest $b$ such that $2^b$ divides $\bb{2^k-i}!$ and largest $c$ such that $2^c$ divides $\bb{2^k}!$ can be expressed by
\be
a = \sum^\infty_{j=1}\floor{\frac{i}{2^j}}, \quad b = \sum^\infty_{j=1}\floor{\frac{2^k - i}{2^j}},\quad c = \sum^\infty_{j=1}\floor{\frac{2^k}{2^j}}
\ee

Thus, if $2^j\mid i$ but $2^{j+1}\nmid i$ for $0\leq j\leq k$, we can assume that $i = 2^j y$ with $\hcf(2,y)=1$. Then
\be
c - (a+b) =  k - j
\ee
which implies that the $i$th term is
\be
\frac{2^i \bb{2^k}!}{i!\bb{2^k-i}!} = x 2^{i} 2^{k-j} = x 2^k 2^{2^j y -j}.
\ee

But we know that $2^j y - j \geq 2$ for all $j$ except $i=1,2$. Hence for ($\dag$),
\beast
3^{2^{n-2}} & = & (1+2)^{2^{n-2}} \equiv 1 + 2\cdot 2^{n-2} + 2^2 \cdot \frac{2^{n-2}\bb{2^{n-2}-1}}2 \lmod{2^n} \\
& = & 1 + 2\cdot 2^{n-2} + 2 \cdot 2^{n-2}\bb{2^{n-2}-1} = 1 + 2^{2n-3} \equiv 1 \lmod{2^n}
\eeast
since $n\geq 5$. Similarly, for ($*$) we know that $2^j y - j \geq 3$ for all $j$ except $i=1,2,4$,
\beast
3^{2^{n-3}} & = & (1+2)^{2^{n-3}} \equiv 1 + 2\cdot 2^{n-3} + 2^2 \cdot \frac{2^{n-3}\bb{2^{n-3}-1}}2 + 2^4\cdot \frac{2^{n-3}\bb{2^{n-3}-1}\bb{2^{n-3}-2}\bb{2^{n-3}-3}}{24} \lmod{2^n} \\
& = & 1 + 2\cdot 2^{n-3} + 2 \cdot 2^{n-3}\bb{2^{n-3}-1} + \frac{2^{n-1}\bb{2^{n-3}-1}\bb{2^{n-4}-1}\bb{2^{n-3}-3}}{3} \lmod{2^n} \\
& = & 1 + 2^{2n-5} + x 2^{n-1} \equiv1 + x 2^{n-1}  \not\equiv 1 \lmod{2^n}
\eeast
where $\hcf(x,2) =1$.
\end{proof}

\begin{theorem}[multiplicative group of $\Z/p\Z$]\label{thm:multiplicative_group_pz_cyclic}%$\Z/p\Z$]\label{thm:multiplicative_group_pz_cyclic}%Let $p$ be a prime number and $\F_p$ be field of $p$ element. Then $\F_p^* : = \F_p\bs \bra{0}$ is actually the multiplicative group of $\F_p$, $\F_p^{\times}$. Furthermore, $\F_p^*$ is cyclic.

Let $p$ be a prime number and for field $\Z/p\Z$ (see Example \ref{exa:zpz_field}), its multiplicative group
\be
\bb{\Z/p\Z}^{\times} = \bb{\Z/p\Z}\bs \bra{0} = \bra{1,2,\dots, p-1} \cong C_{p-1}.
\ee%forms $\bb{\Z/p\Z}^{\times}$, the cyclic multiplicative groups of $\Z/p\Z$.%is cyclic.
\end{theorem}

\begin{proof}[\bf Proof]
{\bf Approach 1.} First, it is easy to see that $\bb{\Z/p\Z}\bs \bra{0}$ is a group under multiplication.

%$\forall x\in \bb{\Z/p\Z}^*$. If $x = 1$ or $p-1$, then its inverse is itself since
%\be
%1\cdot 1 \equiv 1 \lmod{p},\qquad (p-1)\cdot (p-1) = p^2 -2p+1 \equiv 1\lmod{p}.
%\ee

%Also, we have
%\be
%x^2 \equiv 1 \lmod{p} \ \ra\ p\mid (x-1)(x+1) \ \ra\ x \equiv \pm 1\lmod{p}
%\ee
%if $x\neq 1$ or $p-1$


Then by structure theorem for finite abelian groups (Theorem \ref{thm:structure_theorem_finite_abelian_group}), we have the multiplicative group
\be
\bb{\Z/p\Z}^{\times} = \bb{\Z/p\Z}\bs \bra{0} \cong C_{d_1} \times C_{d_2} \times \dots \times C_{d_n}\quad \text{with }d_i\mid d_{i+1}
\ee

If $\bb{\Z/p\Z}^{\times}$ is isomorphic to $C_{d_1} \times C_{d_2} \times \dots \times C_{d_n}$ where $n \geq 2$, then there exists $n_a,n_b \in \bra{d_1,d_2,\dots,d_n}$ s.t. $n_a\mid n_b$. Then there is a cyclic subgroup $\bsa{a} \cong C_{n_a}$ and $\bsa{b} \cong C_{n_b}$ generated by $a$ and $b$ with order $n_a$ and $n_b$. Since $n_a \mid n_b$ we have $\bsa{b^{\frac{n_b}{n_a}}} \cong \bsa{a} \cong C_m$ for some $m >1$. Thus, there exists a subgroup $\bsa{b^{\frac{n_b}{n_a}}} \times \bsa{a} \cong C_m \times C_m$ by Lemma \ref{lem:direct_product_of_groups}.

% (That is, $p-1 = 4k$ for some $k\in \Z$, if $k$ is an odd number, by structure theorem for finite abelian groups (Theorem \ref{thm:structure_theorem_finite_abelian_group}) and Lemma \ref{lem:coprime_cycle_cong}, we can have
%\be
%\F_p\bs\bra{0} \cong C_{2k} \times C_2 \cong \left\{ \ba{ll}
%C_2 \times C_2 & k =1\\
%C_{k} \times C_2 \times C_2 \quad\quad & k >1
%\ea\right..x
%\ee

%If not then the structure theorem for finite abelian groups (Theorem \ref{thm:structure_theorem_finite_abelian_group}) implies we can find a subgroup of form $C_m \times C_m$ for some $m > 1$.

%in $\F_p[X]$ which is a UFD. $X^m - 1$ has a unique expression up to ordering and associates as a product of at most $m$ irreducibles in $\F_p[X]$. In particular, there are at most $m$ linear factors $X - a$ and so at most $m$ roots, contradicting that there are at least $m^2$ roots.

Thus there are at least $m^2$ elements satisfying $a^m = 1$. These are all roots of $X^m - 1$. Then by Lagrange theorem (Theorem \ref{thm:lagrange_ring}), there are at most $m$ linear factors $X - a$ and so at most $m$ roots, contradicting that there are at least $m^2$ roots. Thus, the multiplicative group $\bb{\Z/p\Z}^{\times}$ is cyclic.%$\F_p^*$

{\bf Approach 2.} For any $a\in \bb{\Z/p\Z}^\times$ satisfies that $a^{\abs{\bb{\Z/p\Z}^\times}} = a^{p-1} \equiv 1\lmod{p}$ by Theorem \ref{thm:multiplicative_group_znz_size_euler_totient}. So by Lagrange theorem (Theorem \ref{thm:lagrange_group}) the order of $a$ is $d$ for some $d\mid (p-1)$. Let $S_d$ be the set of elements with order $d$.

Suppose $S_d\neq \emptyset$ and let $G_d$ be the subgroup generated by $a$. Then $G_d = \bra{1,a,\dots,a^{d-1}}$. Let $R= \Z/p\Z$ and $f\in R[X]$ with $f(X) = X^d-1$. Then every element of order $d$ is a root of $f$. By Lagrange's theorem (Theorem \ref{thm:lagrange_ring}) $f$ has at most $d$ roots. But each element in $G_d$ is a root of $f$ and so these are all the roots of $f$. Therefore each element of order $d$ is $a^i$ for some $i<d$.

By Proposition \ref{pro:cyclic_group_isomorphism_infinite_finite}, $G_d \cong C_d \cong \bb{\Z/d\Z,\oplus_d}$ via $a^i\mapsto i$. Then by Lemma \ref{lem:coprime_congruence_generator_group}, $\hcf(i,d)=1$. So $\abs{S_d} = \phi(d)$ by definition of Euler's totient function $\phi$. Therefore $\abs{S_d} = 0$ or $\phi(d)$. Recall Theorem \ref{thm:sum_of_factor_totient_equals_n},
\be
\sum_{d\mid p-1} \phi(d) = p-1 = \sum_{d\mid p-1} \abs{S_d}
\ee
as $S_d$ are pairwise disjoint for all $d$. This shows that $\abs{S_d} = \phi(d)$ for all $d$ and so in particular
\be
\abs{S_{p-1}} = \phi(p-1) \geq 1,
\ee
which means that there exists an element of order $p-1$. Thus, $\bb{\Z/p\Z}^\times$ with size $p-1$ is cyclic.
\end{proof}

\begin{example}
$\bb{\Z/23\Z}^{\times}$ is the multiplicative group of $\Z/23\Z$. However, not all its elements generate the multiplicative group. In particular, it's easy to see that the order of each element divides the order of multiplicative group, 22.
\beast
2^{11} & \equiv & 16^2 \cdot 8 \equiv (-7^2)\cdot 8 \equiv 49 \cdot 8 \equiv 3 \cdot 8 \equiv 1 \lmod{23} \\
3^{11} & \equiv & 27^3 \cdot 9 \equiv 4^3 \cdot 9 \equiv -7\cdot 13\equiv -91 \equiv 1  \lmod{23}  \\
4^{11} & \equiv & 256^2 \cdot 64 \equiv 3^2\cdot (-5) \equiv -45 \equiv 1 \lmod{23} \\
22^2 & \equiv & (-1)^2 \equiv 1 \lmod{23} .
\eeast

One of the generator is 5 as
\beast
5^2 & \equiv & 2 \not\equiv 1\lmod{23}\\
5^{11} & \equiv & 25^5 \cdot 5 \equiv 2^5\cdot 5 \equiv 9 \cdot 5 \equiv 45 \equiv -1 \not\equiv 1 \lmod{23} \\
5^{22} & \equiv & 25^{11} \equiv 2^{11} \equiv 256 \cdot 8 \equiv 3\cdot 8 \equiv 24  \equiv 1 \lmod{23}.
\eeast
\end{example}

\begin{lemma}\label{lem:generaor_multiplicative_group_zpnz_iff_generator_multiplicative_group_zpz}
For each integer $n\geq 2$, an element $g$ generates $\bb{\Z/p^n\Z}^{\times}$ if and only if $g$ generates $\bb{\Z/p\Z}^{\times}$ and
\be
g^{p^{n-2}(p-1)}\not\equiv 1\lmod{p^n}.
\ee
\end{lemma}

\begin{proof}[\bf Proof]
Suppose $g$ generates $\bb{\Z/p^n\Z}^\times$, then clearly
\be
g^{p^{n-2}(p-1)} \not\equiv 1 \mod{p^n}
\ee
as $p^{n-2}(p-1) < p^{n-1}(p-1) = p^n - p^{n-1} = \phi\bb{p^n} = \abs{\bb{\Z/p^n\Z}^\times}$ (by Theorem \ref{thm:multiplicative_group_znz_size_euler_totient}). Let $d$ be the order of $g$ in $\bb{\Z/p\Z}^\times$ so
\be
g^d \equiv 1 \lmod{p} \ \lra \ g^d  = 1 + pz
\ee
for some $z\in \Z$. Then
\be
g^{dp^{n-1}} = (1+pz)^{p^{n-1}} = 1 + \sum^{p^{n-1}}_{i=1} (pz)^i \binom{p^{n-1}}{i} =  1 + \sum^{p^{n-1}}_{i=1} z^i \frac{p^i \bb{p^{n-1}}!}{i!\bb{p^{n-1}-i}!}.
\ee

%For any $i\geq 1$, we have that by fundamental theorem of arithmetic $i! = p^{a_i}z$ for some positive integer $z$. Then $a_i < i$ so $p\mid \frac{p^i}{p^{a_i}}$. Also we have $p^{n-1}\left|\frac{\bb{p^{n-1}}!}{\bb{p^{n-1}-i}!}\right.$. Therefore,
%\be
%p^n \left| \frac{p^i \bb{p^{n-1}}!}{i!\bb{p^{n-1}-i}!}\right.
%\ee
%which implies that

Then by lemma \ref{lem:prime_power_divides_factorial}, the largest $a$ such that $p^a$ divides $i!$, largest $b$ such that $p^b$ divides $\bb{p^{n-1}-i}!$ and largest $c$ such that $p^c$ divides $\bb{p^{n-1}}!$ can be expressed by
\be
a = \sum^\infty_{k=1}\floor{\frac{i}{p^k}}, \quad b = \sum^\infty_{k=1}\floor{\frac{p^{n-1} - i}{p^k}},\quad c = \sum^\infty_{k=1}\floor{\frac{p^{n-1}}{p^k}}
\ee

Thus, if $p^k\mid i$ but $p^{k+1}\nmid i$ for $0\leq k\leq n-1$, we can assume that $i = p^k y$ with $\hcf(p,y)=1$. Then
\be
c - (a+b) =  (n-1) - k
\ee
which implies that the $i$th term is
\be
z^i \frac{p^i \bb{p^{n-1}}!}{i!\bb{p^{n-1}-i}!} = z^i x p^{i} p^{n-1-k} = z^i x p^n p^{p^k y -1-k}.
\ee

But we know that $p^k y - 1-k \geq 0$ for all $k$ thus $i$. Hence we conclude that
\be
g^{dp^{n-1}} \equiv 1 \lmod{p^n}.
\ee

Therefore the order of $g$ in $\bb{\Z/p^n\Z}^\times$ is at most $dp^{n-1}$, i.e., $o(g) \leq dp^{n-1}$. However, $o(g)$ should be $\phi\bb{p^n} = p^n - p^{n-1}$ and then $d\geq p-1$ which is the size of $\bb{\Z/p\Z}^\times$. Therefore, $g$ is also a generator in $\bb{\Z/p\Z}^\times$. %Since  which is a contradiction.

Conversely, let $g$ be a generator of $\bb{\Z/p\Z}^\times$ and $g^{p^{n-2}(p-1)} \not\equiv 1 \lmod{p^n}$. Let $d$ be the order of $g$ in $\bb{\Z/p^n\Z}^\times$. Then we have
\be
\phi\bb{p^n} = p^{n-1}(p-1) \ \ra\ d\mid p^{n-1}(p-1).
\ee

Also,
\be
g^d \equiv 1 \lmod{p^n} \ \ra\ g^d \equiv 1 \lmod{p} \ \ra\ p-1 = \phi(p) \mid d \ \ra\ d = p^k(p-1)
\ee
for some integer $k\geq 0$. However, $g^{p^{n-2}(p-1)} \not\equiv 1 \lmod{p^n}$, we have $k>n-2$ so $k=n-1$. Hence, $d = \phi(p^n)$, which is size of $\bb{\Z/p^n\Z}^\times$ and thus $g$ is a generator of $\bb{\Z/p^n\Z}^\times$.
\end{proof}


\begin{lemma}\label{lem:multiplicative_group_zp2z_cyclic}
$\bb{\Z/p^2\Z}^\times$ is cyclic for each odd prime number $p$.
\end{lemma}

\begin{proof}[\bf Proof]%For $a\in \bb{\Z/p\Z}^\times$, we can see that $a\in \bb{\Z/p^2\Z}^\times$ as well.
Consider the natural reduction
\be
\theta: \ \bb{\Z/p^2\Z}^\times \to \bb{\Z/p\Z}^\times,\quad a+ p^2\Z \mapsto a+ p\Z,\qquad p\nmid a.% = 1,\dots p-1.
\ee

%

By considering the sizes of the groups, the kernel of $\theta$ is a normal subgroup of $\bb{\Z/p^2\Z}^\times$ with order $p$. Indeed,
\be
\ker\theta = \bra{1,p+1,2p+1,\dots, p(p-1)+1}.
\ee

So $\ker\theta$ is cyclic by Corollary \ref{cor:prime_order_cyclic}. Then we can find an element $g\in \bb{\Z/p^2\Z}^\times$ with order $p$.

Since $\bb{\Z/p\Z}^\times$ is cyclic (Theorem \ref{thm:multiplicative_group_pz_cyclic}), we can find a generator $a$ of $\bb{\Z/p\Z}^\times$. Then for $1\leq k< p-1$,
\be
a^{k} \not\equiv 1 \lmod{p},\quad a^{p-1} \equiv 1\lmod{p}.
\ee

If $a^{p-1}\equiv 1\lmod{p^2}$, we have that $a$ has order $p-1$ in $\bb{\Z/p^2\Z}^\times$. If $a^{p-1}\not\equiv 1\lmod{p^2}$, we can assume that
\be
a^{p-1} = 1+np
\ee
where $\gcd(n,p)=1$. Then we let the preimage of $a$ be $x := a(1+np) = a + anp \in \bb{\Z/p^2\Z}^\times$. % since $\gcd(n,p)=1$.
Then we have for $1\leq k < p-1$
\be
x^{k} = a^k\bb{1+np}^k \equiv a^k \lmod{p} \ \ra\ x^k \not\equiv 1\lmod{p} \ \ra\ x^k \not\equiv 1 \lmod{p^2}.
\ee

However,
\be
x^{p-1} = a^{p-1}\bb{1+np}^{p-1} = (1+np)^p \equiv 1 \lmod{p^2}.%\equiv (1+np)\bb{1+np(p-1)}\equiv (1+np)(1-np) \equiv 1 \lmod{p^2}.
\ee

Thus in either case, we find an element $h\in \bb{\Z/p^2\Z}^\times$ with order $p-1$. Then $gh$ must has order $p(p-1)$ in $\bb{\Z/p^2\Z}^\times$ since $p$ and $p-1$ are coprime. Thus $gh$ generates $\bb{\Z/p^2\Z}^\times$.



\end{proof}%Let $g = p+1$, then for $n=1,2,\dots p-1$, we have
%\be
%g^2 = (p+1)^2 \equiv 2p+1 \lmod{p^2},\qquad g^3 \equiv (p+1)(2p+1) \equiv 3p + 1 \lmod{p^2}
%\ee

%\footnote{and $\bb{\Z/p^2\Z}^\times/\ker\theta\cong \im\theta = \bb{\Z/p\Z}^\times$ by the first isomorphism theorem (Theorem \ref{thm:isomorphism_1_group}) since $\theta$ is surjective. Therefore the quotient group $\bb{\Z/p^2\Z}^\times/\ker\theta$ (a subgroup of $\bb{\Z/p^2\Z}^\times$) is cyclic since $\bb{\Z/p\Z}^\times$ is cyclic (Theorem \ref{thm:multiplicative_group_pz_cyclic}) Then we can find an element $g$ of order $p-1$ in $\bb{\Z/p^2\Z}^\times$.}%has the same size $p-1$ with $\im\theta$



%Also, we can find a generator $g$ of $\bb{\Z/p\Z}^\times$ with order $p-1$. Clearly, $g\in \bb{\Z/p^2\Z}^\times$. Therefore, $\bb{\Z/p^2\Z}^\times$ have an element $g$ of order $p-1$ and an element of $h$ of order $p$. Then $gh$ must has order $p(p-1)$, which


\begin{theorem}\label{thm:multiplicative_group_zpnz_cyclic}%\label{thm:z_prime_power_n_multiplicative_group_cyclic}%
$\bb{\Z/p^n\Z}^{\times}$ is cyclic for each odd prime number $p$ and positive integer $n$.
\end{theorem}

\begin{example}
Given $p =3$ and $n=2$, we have
\be
\bb{\Z/9\Z}^{\times} = \bra{1,2,4,5,7,8} \cong C_6.%2 \times C_3.
\ee

Indeed, $2$ is the generator of $\bb{\Z/9\Z}^{\times}$ as
\be
2^2 \equiv 4 \lmod{9},\quad 2^3 \equiv 8 \lmod{9},\quad 2^4 \equiv 7 \lmod{9},\quad 2^5 \equiv 5 \lmod{9}, \quad 2^6 \equiv 1 \lmod{9}.
\ee
\end{example}

\begin{proof}[\bf Proof]
By Lemma \ref{lem:multiplicative_group_zp2z_cyclic} let $g$ be a generator of $\bb{\Z/p^2\Z}$. Then by Lemma \ref{lem:generaor_multiplicative_group_zpnz_iff_generator_multiplicative_group_zpz}, $g$ is a generator of $\bb{\Z/p\Z}^\times$ (thus $g^{p-1} \equiv 1 \lmod{p}$) and $g^{p-1} \not\equiv 1 \lmod{p^2}$. So
\be
g^{p-1} = 1 + pz \text{ for some $z$ such that $p\nmid z$}  \ \ra\ \hcf(z,p) = 1.
\ee

Then for each $k$
\be
g^{p^k(p-1)} = \bb{1+pz}^{p^k} = 1 + \sum_{i=1}^{p^k} p^iz^i \binom{p^k}{i} = 1 + \sum_{i=1}^{p^k} \frac{p^iz^i \bb{p^k}!}{i!\bb{p^k-i}!}
\ee

Then by lemma \ref{lem:prime_power_divides_factorial}, the largest $a$ such that $p^a$ divides $i!$, largest $b$ such that $p^b$ divides $\bb{p^{k}-i}!$ and largest $c$ such that $p^c$ divides $\bb{p^k}!$ can be expressed by
\be
a = \sum^\infty_{r=1}\floor{\frac{i}{p^r}}, \quad b = \sum^\infty_{r=1}\floor{\frac{p^{k} - i}{p^r}},\quad c = \sum^\infty_{r=1}\floor{\frac{p^{k}}{p^r}}
\ee

Thus, if $p^r\mid i$ but $p^{r+1}\nmid i$ for $0\leq r\leq k$, we can assume that $i = p^r y$ with $\hcf(p,y)=1$. Then
\be
c - (a+b) =  k - r
\ee
which implies that the $i$th term is
\be
z^i \frac{p^i \bb{p^{k}}!}{i!\bb{p^{k}-i}!} = z^i x p^{i} p^{k-r} = z^i x p^k p^{p^r y -r}.
\ee

But we know that $i=1$ imples that $r=0,y=1$ and thus $p^r y -r = 1$. For $i = 2$, we have $r = 0, y=2$ and thus $p^r y -r = 2$. For $i\geq 3$, it is easy to find
\be
p^r y - r \geq 2.
\ee

Hence we conclude that
\be
g^{p^k (p-1)} \equiv 1 + p^{k+1}z  \lmod{p^{k+2}}.
\ee

In particular, $g^{p^k(p-1)} \not\equiv 1 \lmod{p^{k+2}}$. Let $k= n-2$, we can have
\be
g^{p^{n-2}(p-1)} \not\equiv 1 \lmod{p^n}.
\ee

Thus, by Lemma \ref{lem:generaor_multiplicative_group_zpnz_iff_generator_multiplicative_group_zpz}, we have that $g$ generates $\bb{\Z/p^n\Z}^\times$ and thus $\bb{\Z/p^n\Z}^\times$ is cyclic.
\end{proof}


\begin{remark}
Theorem \ref{thm:multiplicative_group_zpnz_cyclic} is not true for $p=2$. For instance,
\be
\bra{1,3,5,7} = \bb{\Z/2^3\Z}^{\times} \cong \bb{\Z/2\Z}^2.
\ee
\end{remark}

\begin{theorem}[multiplicative group of direct product of modular rings]
Let $n = \prod^k_{i=1}n_i = n_1 n_2 \dots n_k $ where $n_1,\dots,n_k\in \Z^+$ are pairwise coprime. Then we have a group isomorphism
\be
\bb{\Z/n\Z}^\times \cong \bb{\Z/n_1\Z}^\times \times \dots \times \bb{\Z/n_k\Z}^\times = \prod^k_{i=1} \bb{\Z/n_i\Z}^\times.
\ee
\end{theorem}

\begin{remark}
Combining with Theorem \ref{thm:chinese_remainder_modular_ring}, this theorem implies that
\be
\bb{\Z/n_1\Z \times \dots \times \Z/n_k\Z}^\times \cong \bb{\Z/n_1\Z}^\times \times \dots \times \bb{\Z/n_k\Z}^\times.
\ee

We can apply this theorem to prove that Euler's totient function is multiplicative since multiplicative group of integer modulo $n$ has size $\phi(n)$ (see Theorem \ref{thm:multiplicative_group_znz_size_euler_totient}).

See Table \ref{tab:group_structure_multiplicative_group_modular_ring} in Summary for group $\bb{\Z/n_1\Z}^\times$ of all small $n$.
\end{remark}

\begin{proof}[\bf Proof]
Consider the mapping $\theta$:
\be
\bb{\Z/n\Z}^\times \to \bb{\Z/n_1\Z}^\times \times \dots \times \bb{\Z/n_k\Z}^\times,\quad a \mapsto \bb{a\lmod{n_1},\dots, a\lmod{n_k}}.
\ee

Let $a\in \bb{\Z/n\Z}^\times$, then $\hcf(a,n)=1$ which implies that $\hcf(a,n_i) = 1$. Thus, $a\lmod{n_i} \in \bb{\Z/n_i\Z}^\times$ for all $i$ and $\theta$ is well-defined.

Obviously, $\theta$ is a group homomorphism under multiplication by Theorem \ref{thm:chinese_remainder_modular_ring} since $\bb{\Z/n\Z}^\times$ is the subset of $\Z/n\Z$.

For any $a_1,a_2\in \bb{\Z/n\Z}^\times$, if $a_1\equiv a_2\lmod{n_i}$ then $(a_1-a_2)\mid n_i$ for all $i$. Thus, $(a_1-a_2)\mid n$ and $a_1\equiv a_2\lmod{n}$ since $n_i$ are coprime. So $\theta$ is injective.

Furthermore, $\forall c_i\in \bb{\Z/n_i\Z}^\times$, we have $\hcf(c_i,n_i) = 1$ thus we can find $a_i$ by Lemma \ref{lem:congruence_equation_soluble_iff_gcd_division} such that $m_i a_i\equiv c_i \lmod{n_i}$ where $m_i = n/n_i$ since $\hcf(n_i,m_i) =1$. Then let
\be
a= \sum^k_{i=1} m_i a_i \ \ra\ a \equiv c_i \lmod{n_i}.
\ee

Since $\hcf(c_i,n_i)=1$, we have for all $i$,
\be
\hcf(a,n_i) = 1 \ \ra\ \hcf(a,n)=1 \ \ra\ a \in \bb{\Z/n\Z}^\times \ \ra\  \theta \text{ is surjective.}
\ee

Hence, $\theta$ is bijective and thus a group isomorphism.
\end{proof}

\begin{example}
For $n=360 = 2^3\cdot 3^2 \cdot 5$, we have
\be
\bb{\Z/360\Z}^\times \cong \bb{\Z/2^3\Z}^\times \times \bb{\Z/3^2\Z}^\times \times \bb{\Z/5\Z}^\times \cong \bb{C_2 \times C_{2}}\times C_6 \times C_4 \cong C_2 \times C_2 \times C_4 \times C_6.
\ee
\end{example}

%\footnote{need group structure of $\bb{Z/n\Z}^\times$, see wiki for multiplicative group of integers modulo $n$.}


\section{Principal Ideal Domain, Maximal Ideal and Prime Ideal}

\subsection{Principal ideal domain}

\begin{definition}[principal ideal domain\index{principal ideal domain}]\label{def:principal_ideal_domain}
A principal ideal domain (PID\index{PID!principal ideal domain}) is an integral domain in which all ideals are principal ideals, i.e. of the form $\bsa{a} = aR$ for some $a \in R$.
\end{definition}

%\begin{example}
%$\Z$ is a principal ideal domain. We will see that any integral domain where Euclidean algorithm applies is a principal ideal domain.
%\end{example}

\begin{proposition}\label{pro:polynomial_pid_iff_field}
Let $R$ be any commutative ring. Show that the ring $R[X]$ is a PID if and only if $R$ is a field.%$\F[X]$ for a field $\F$ is PID.
\end{proposition}

\begin{proof}[\bf Proof]
($\la$). Suppose $R$ is a field and for any $I \lhd R[X]$, we take $f\in I$ with the least degree, then for any $g\in I$, by Euclidean algorithm for polynomials (Proposition \ref{pro:euclidean_algorithm_polynomial}), we can find $q(X)$ and $r(X)\in I$,
\be
g(X) = q(X)f(X) + r(X)
\ee
with $\deg r < \deg f$. Since $f$ has the least degree, we have $r = 0$ and for any $g(X)\in I$, $f(X)|g(X)$. Thus, $I = \bsa{f(X)}$. Thus, $R[X]$ is PID.

($\ra$). Suppose $R[X]$ is a PID, for any $a\in R$, we construct $I = \bsa{a,X}$ for $a\neq 0 \in R$. Also, we can find $b\in R$ such that $I = \bsa{b}$ since $R[X]$ is PID. Thus, we can find $f(X) = a_0 + a_1 X + \dots \in R[X], a_0,a_1,\dots \in R$ such that
\be
bf = a+X \ \ra \ b(a_0 + a_1X + \dots ) = a+X \ \ra \ ba_1 = 1.
\ee

So $b$ is a unit. So $1\in b R[X] = \bsa{b} = \bsa{a,X}$. Thus, there exist $g(X),h(X) \in R[X]$, $g(X) = c_0 + c_1 X + \dots$, $c_0,c_1,\dots \in R$
\be
a g(X) + Xh(X) = 1 \ \ra \ a c_0 = 1.
\ee

Thus, $a$ is unit. Therefore, this can be done for any $a$, so $R$ is a field.
\end{proof}

\begin{proposition}\label{pro:pid_examples}
$\Z$ and $\Z[i]$ are all PIDs.
\end{proposition}

\begin{proof}[\bf Proof]
\footnote{need details}
\end{proof}

\begin{example}
$\Z[X]$ is not a PID. It is obvious that $\Z$ is not a field, then $\Z[X]$ is not a PID by Proposition \ref{pro:polynomial_pid_iff_field}.

Now we want to prove it by checking the definition of PID.

Consider the ideal $I = \bsa{2,X} = \left\{a_iX^i : a_0 \text{ is even}\right\} \lhd \Z[X]$. Suppose that $I = \bsa{f(X)}$ for some $f(X) \in \Z[X]$.

We have $f(X)\mid 2$ and $f(X)\mid X$ since $2,X \in I$. In particular $2 = f(X)g(X)$ for some $g(X) \in \Z[X]$. We deduce that $f(X)$ has to be a constant polynomial, that is, of degree 0, and must be $\pm 1$ or $\pm 2$. Also $X = f(X)h(X)$ for some $h(X) \in \Z[X]$. So $f(X)$ cannot be $\pm 2$. So $f(X) = \pm 1$. Hence $I = \bsa{f(X)} = \Z[X]$.

This is a contradiction, since $I \neq \Z[X]$ ($I$ has even coefficients).%, e.g. $1 \notin I$.
\end{example}

\begin{example}[minimal polynomials of matrices with entries in a field $\F$]
We recall the notation of matrices and their polynomials (Definition \ref{def:matrix_polynomial}). Let $A \in M_{n}(\F)$ and consider $I = \{m(X) \in \F[X] : m(A) = 0\}$.

We can check that $I \lhd \F[X]$ is an ideal\footnote{need details}. But $\F[X]$ is a PID (by Proposition \ref{pro:polynomial_pid_iff_field}), so $I = \bsa{m(X)}$, and by multiplying $m(X)$ by a
unit in $\F$ we may assume $m(X)$ is monic. This is the minimal polynomial of $A$. (See Theorem \ref{thm:minimal_polynomial_existence_uniqueness} where Euclidean algorithm (Proposition
\ref{pro:euclidean_algorithm_polynomial}) is explicitly used.) %the course Linear Algebra\footnote{check section number}\index{minimal polynomial!matrix}
\end{example}


\begin{lemma}\label{lem:pid_irreducible_prime}
Let $R$ be a PID. If $p$ is an irreducible then it is prime. (This is the converse of Lemma \ref{lem:prime_implies_irreducible_integral_domain} for PIDs.)
\end{lemma}

\begin{proof}[\bf Proof]
Let $p$ be an irreducible and suppose $p | ab$. Without loss of generality, suppose that $p \nmid a$. The ideal $(p, a)$ must be principle in a PID, $\bsa{p, a} = \bsa{d}$, say.

Thus $p = q_1d$ and $a = q_2d$ for some $q_1,q_2 \in R$. $p$ is irreducible so either $q_1$ or $d$ is a unit. But if $q_1$ is a unit, then $d = q_1^{-1} p$ and $a = q_2d = q_2q^{-1}_1 p$, so $p | a$, a contradiction. Hence $d$ is a unit. Thus $\bsa{d} = dR = R$ since $d$ is unit and there exist $r, s \in R$ such that (since $\bsa{p, a} = \bsa{d}$),
\be
1 = rp + sa \ \ra \ b = rpb + sab.
\ee

Since $p \mid rpb$ and $p \mid ab$, $p | b$.
\end{proof}

\begin{remark}
In fact, $d$ is a highest common factor of $p$ and $a$ (see Definition \ref{def:hcf_ring}). Highest common factors are unique up to units (see Proposition \ref{pro:hcf_ufd_unique_unit}).%\footnote{see Example Sheet 3 or in UFD part}
\end{remark}


\subsection{Maximal ideal and prime ideal}

\begin{definition}[proper ideal\index{proper ideal}]\label{def:proper_ideal}
A proper ideal is an ideal $I$ of a ring $R$ which is strictly smaller than the whole ring, i.e., $I\subset R$($I\neq R$).
\end{definition}

\begin{definition}[maximal ideal\index{maximal ideal}]
An proper ideal $I$ in $R$ is maximal in $R$ whenever $I \subseteq J \lhd R$ then $I = J$ or $J = R$.
\end{definition}

\begin{example}
In $\Z$, the maximal ideals are $p\Z$ where $p$ is a prime number.
\end{example}


\begin{definition}[prime ideal\index{prime ideal}]\label{def:prime_ideal}
A proper ideal is a prime ideal in $R$ if whenever $ab \in I$ then $a \in I$ or $b \in I$.
\end{definition}

\begin{remark}
This generalizes the following property of prime numbers: if $p$ is a prime number and if $p$ divides a product $ab$ of two integers, then $p$ divides $a$ or $p$ divides $b$. We can therefore say, a positive integer $n$ is a prime number if and only if the ideal $n\Z$ is a prime ideal in $\Z$.
\end{remark}

\begin{example}
In $\Z$, the prime ideals are $p\Z$ for prime number $p$ and $\bra{0}$.
\end{example}

\begin{lemma}\label{lem:field_maximal_ideal}
Let $R$ be a commutative ring. For an ideal $I \lhd R$, the quotient ring $R/I$ is a field if and only if $I$ is maximal in $R$.
\end{lemma}

\begin{example}\label{exa:zpz_field}
$\Z/p\Z$ is a field with $p$ elements where $p$ is a prime number.
\end{example}

%Since $C_p\cong \Z/p/Z$ under modulo $p$ addition operation, we have can have an isomorphism $\theta$ such that $\theta:\Z/p\Z \to C_p, a\mapsto g^a$ where $g$ is the generator of cyclic group $C_p$ with $C_p = \bsa{g}$ and $g^p = e$. So $\theta$ maps $0\in \Z/p\Z$ to $e\in C_p$. Then we have that $\forall z_1,z_2\in \Z/p/Z$
%\be
%\ee

%Thus, $\theta$ also satisfies
%\be
%\theta (z_1z_2) = \theta\bb{\underbrace{z_1 + \dots + z_1}_{z_2\text{ times}}} = \theta(z_1) + \dots + \theta(z_1) = g^z_1 \dots g^{z_1} = g^{z_1z_2} =
%\ee

%Therefore, for any field with $p$ elements ($p$ is prime), $\F_p$, we have $\F_p \cong \Z/p\Z$.


\begin{proof}[\bf Proof]
$R/I$ is a field if and only if $\bra{0} = I/I$ and $R/I$ are the only ideals in $R/I$ (by Lemma \ref{lem:ring_field_only_ideal} ). By the usual correspondence, this is equivalent to that the only ideals in $R$ containing $I$ are $I$ and $R$, which is the statement that $I$ is maximal in $R$.
\end{proof}

\begin{lemma}\label{lem:integral_domain_prime_ideal}
Let $R$ be a commutative ring. For an ideal $I \lhd R$, the quotient ring $R/I$ is an integral domain if and only if $I$ is a prime ideal in $R$.
\end{lemma}

\begin{proof}[\bf Proof]
Note that zero in $R/I$ is $0 + I = I$.

Suppose $R/I$ is an integral domain and $ab \in I$. Then $0 + I = I = ab+I = (a+I)(b+I)$ and so either $a + I = 0 + I = I$ or $b + I = 0 + I = I$. So $a \in I$ or $b \in I$.

Conversely, suppose $I$ is a prime deal and $(a + I)(b + I) = I$. Then $ab + I = I$ and so $ab \in I$. Primeness implies $a \in I$ or $b \in I$ and so $a + I = I = 0 + I$ or $b + I = 0 + I$.
\end{proof}

\begin{theorem}\label{thm:cr_maximal_ideal_implies_prime}
Let $R$ be a commutative ring. For an ideal $I \lhd R$, if $I$ is a maximal ideal, then it is also a prime ideal.
\end{theorem}

\begin{remark}
This can be seen directly by Lemma \ref{lem:integral_domain_prime_ideal} and Lemma \ref{lem:field_maximal_ideal} as we can see that a field is an integral domain.
\end{remark}

\begin{proof}[\bf Proof]
Assume there exist $a,b$ such that $ab\in I$ but $a,b\in R\bs I$. Now let $J$ be the ideal generated by $I\cup \bra{a}$, i.e.,
\be
J = \bra{x+ar:x\in I,r\in R}.
\ee

Clearly, $I\subset J$ ($I \subsetneq J$) as we can take $r=1$. Thus, since $I$ is maximal, we have $J = R$. Therefore, $1\in J$ and hence there exist $x\in I$ and $r\in R$ such that
\be
1 = x+ar.
\ee

Then we have
\be
b = b\cdot 1 = b(x+ar) = \underbrace{bx}_{\in I} + \underbrace{(ab)r}_{\in I} \in I.
\ee

Contradiction with the assumption that $b\in R\bs I$.
\end{proof}

\begin{theorem}\label{thm:pid_prime_ideal_implies_maximal}
Let $R$ be a PID. For an ideal $I \lhd R$, if $I$ is a non-zero prime ideal, then it is a maximal ideal.
\end{theorem}

\begin{proof}[\bf Proof]
If $I$ is non-zero prime, then we assume that $I\subseteq J\lhd R$. Since $R$ is a PID, we assume that $I = \bsa{a}$ and $J = \bsa{b}$. Thus, $a\in J$.

Then we have $c\in R$ such that $a = bc$. Since $I$ is prime, we have that $a\in I \ \ra \ b\in I\text{ or }c\in I$.

If $b\in I$, we have $I = \bsa{b} = J$. If $c\in I$, we have $d\in R$, such that $c = ad = bcd \ \ra \ 1 = bc$. Thus, $b$ is a unit and by Definition \ref{def:associativity_irreducibility_primeness_integral_domain}, $R = \bsa{b} = J$.

Thus, either $I=J$ or $J=R$. Therefore, $I$ is a maximal ideal.
\end{proof}

\begin{theorem}\label{thm:pid_prime_ideal_equals_maximal}
In PID, maximal ideal and non-zero prime ideal are equivalent.
\end{theorem}

\begin{proof}[\bf Proof]
Direct result by Theorem \ref{thm:cr_maximal_ideal_implies_prime} and Theorem \ref{thm:pid_prime_ideal_implies_maximal}.
\end{proof}

%%%%%%%%%%%%%%%%%%%%%%


\begin{lemma}\label{lem:prime_ideal_prime_ring_element}
Let $R$ be an integral domain. $\bsa{r}$ is a prime ideal of $R$ if and only if $r$ is prime or $r = 0$.
\end{lemma}

\begin{proof}[\bf Proof]
If $r \neq 0$ and $r \mid ab$ then $ab \in \bsa{r}$ and so if $\bsa{r}$ is a prime ideal then $a \in \bsa{r}$ or $b \in \bsa{r}$. Thus $r \mid a$ or $r \mid b$. Thus, $r$ is prime.

If $r$ is unit, by Definition \ref{def:associativity_irreducibility_primeness_integral_domain}, $\bsa{r} = R$. Contradiction to Definition \ref{def:prime_ideal}.

$\bsa{0}$ is a prime ideal in an integral domain.

Conversely, if $r$ is prime and $ab \in \bsa{r}$ then $r \mid ab$ and hence $r \mid a$ or $r \mid b$. Thus $a \in \bsa{r}$ or $b \in \bsa{r}$. Thus, $\bsa{r}$ is a prime ideal of $R$.
\end{proof}


\begin{lemma}
The characteristic of an integral domain $R$ is 0 or a prime number $p$.
\end{lemma}

\begin{proof}[\bf Proof]
Recall that we have a unique ring homomorphism $\phi: \Z \to R$ (Definition \ref{def:characteristic_ring}). The prime subring $\im \phi \cong \Z/n\Z$, where $n$ is the characteristic of $R$.

If $n$ properly factorises as $n = st$ then working modulo $n$ we have $st = 0 \lmod{n}$ and so we have zero divisors. Contradiction to the fact that $R$ is an integral domain (by Definition \ref{def:zero_divisor_ring} and Definition \ref{def:integral_domain_ring}).

So if $R$ is an integral domain, its prime subring is an integral domain and so $n$ is prime or 0.
\end{proof}



\section{Euclidean Domain}%Factorisation in Integral Domains}%\label{sec:factorisation_integral}% -- units, primes, irreducibles

%\subsection{Euclidean domain}

\begin{definition}[Euclidean domain\index{Euclidean domain}]\label{def:euclidean_domain}
An integral domain $R$ is a Euclidean domain (ED\index{ED!Euclidean domain}) if there is a function $\phi: R \bs\bra{0} \to \Z^+$ called Euclidean function\index{Euclidean function} such that
\ben
\item [(i)] $\phi(ab) \geq \phi(a)$ for all $a, b \in R \bs \{0\}$,
\item [(ii)] if $a, b \in R$ with $b \neq 0$ then there exist $q, r \in R$ with $a = qb + r$ with either $r = 0$ or $\phi(r) < \phi(b)$.
\een
\end{definition}

\begin{proposition}
Let $\F$ be a field. Then $\F[X]$ is a Euclidean domain\footnote{This is given in proof of Proposition \ref{pro:polynomial_pid_iff_field}}. %(Subsection \ref{sec:factorisation_integral} gives Euclidean algorithm here.)
\end{proposition}

\begin{proof}[\bf Proof]
Let Euclidean function be $\phi(f(X)) = \deg f$. To check Definition \ref{def:euclidean_domain},
\ben
\item [(i)] It is obvious.
\item [(ii)] It is given by Euclidean algorithm (Proposition \ref{pro:euclidean_algorithm_polynomial}).
\een

Thus, $\F[X]$ is a ED.
\end{proof}

\begin{example}
\ben
\item [(i)] $\Z$ is a Euclidean domain with $\phi(n) = |n|$.
\item [(ii)] $R = \Z[i] = \{a + bi : a, b \in \Z\} \leq \C$ is a Euclidean domain. As with $\Z[\sqrt{-5}]$ we can define a norm via $N(z) = z\bar{z}$ for $z \in R$. Thus $N(a + bi) = a^2 + b^2$. In this case the norm is a Euclidean function. $N$ is multiplicative and so property (i) holds, $N(z_1z_2) = N(z_1)N(z_2) \geq N(z_1)$, since $N(z_2) \geq 1$.

To verify property (ii) take $z_1, z_2 \in R$ with $z_2 \neq 0$. Consider $z_1/z_2 \in \C$. Then in the complex plane it is distance less than $\sqrt{\bb{\frac 12}^2 + \bb{\frac 12}^2} = \sqrt{2}/2$ from the nearest element of $R$.

$z_1/z_2 = q + z_3$ with $q \in R$, $z_3 \in \C$ and $|z_3| < 1$. So $z_1 = qz_2 + z_2z_3$. Set $r = z_2z_3$. Thus $z_1 = qz_2 + r$,
\be
N(r) = |z_2z_3|^2 = |z_2|^2|z_3|^2 < |z_2|^2 = N(z_2).
\ee


\begin{center}
\psset{yunit=2.5cm,xunit=2.5cm}
\begin{pspicture}(0.5,-0.1)(7,1.8)
\rput[lb](0.4,-0.2){$\Z[i]$}
\rput[lb](1.8,-0.2){$\Z[\sqrt{-2}]$}
\rput[lb](3.3,-0.2){$\Z[\sqrt{-3}]$}
\rput[lb](5,-0.2){$\Z[\omega]$, $\omega = (1+ \sqrt{-3})/2$}

\pstGeonode[PointSymbol=*,PointName=none,dotscale=1.5](0.5,0.5){A}(2,0.707){B}(3.5,0.866){C}(6,0.866){D}

\psline(0,0)(1,0)(1,1)(0,1)(0,0)
\psline(0.5,0)(0.5,1)
\psline(0,0.5)(1,0.5)
\psline[linestyle=dashed](0.25,0.25)(0.25,0.75)(0.75,0.75)(0.75,0.25)(0.25,0.25)(0.5,0.5)

\psline(1.5,0)(2.5,0)(2.5,1.414)(1.5,1.414)(1.5,0)
\psline(2,0)(2,1.414)
\psline(1.5,0.707)(2.5,0.707)
\psline[linestyle=dashed](1.75,0.3535)(1.75,1.0605)(2.25,1.0605)(2.25,0.3535)(1.75,0.3535)(2,0.707)

\psline(3,0)(4,0)(4,1.732)(3,1.732)(3,0)
\psline(3.5,0)(3.5,1.732)
\psline(3,0.866)(4,0.866)
\psline[linestyle=dashed](3.25,0.433)(3.25,1.299)(3.75,1.299)(3.75,0.433)(3.25,0.433)(3.5,0.866)

\psline(4.5,0)(6.5,0)(7.5,1.732)(5.5,1.732)(4.5,0)
\psline(5.5,0)(6.5,1.732)
\psline(5,0.866)(7,0.866)
\psline[linestyle=dashed](5.25,0.433)(5.75,1.299)(6.75,1.299)(6.25,0.433)(5.25,0.433)(6,0.866)

\end{pspicture}
\end{center}


\item [(iv)] For $R = \Z[\sqrt{-2}]$, with similar arguments, property (i) holds, for any $z_1/z_2 \in \C$, the longest distance to its nearest point of lattice is $\sqrt{\bb{\frac {\sqrt{2}}2}^2 + \bb{\frac 12}^2} = \sqrt{3}/2 < 1$. Thus $\abs{z_3}<1$ also holds. Hence, $\Z[\sqrt{-2}]$ is a Euclidean domain.

\item [(v)] For $R = \Z[\sqrt{-3}]$, with similar arguments, property (i) holds, for any $z_1/z_2 \in \C$, the longest distance to its nearest point of lattice is $\sqrt{\bb{\frac {\sqrt{3}}2}^2 + \bb{\frac 12}^2} = 1$. So $\abs{z_3}<1$ doesnt't hold. Hence, we cannot say that $\Z[\sqrt{-3}]$ is a Euclidean domain. Actually, we will see later that $\Z[\sqrt{-3}]$ is not a Euclidean domain since it is not UFD (see Example \ref{exa:sqrtroot_3_5_not_ufd}).

\item [(vi)] Similarly, we will see later that $\Z[\sqrt{-5}]$ is not a Euclidean domain since it is not UFD (see Example \ref{exa:sqrtroot_3_5_not_ufd}).

\item [(vii)] For $R = \Z[\omega]$, $\omega = (1+ \sqrt{-3})/2$. With similar arguments, property (i) holds, for any $z_1/z_2 \in \C$, the longest distance to its nearest point of lattice is $\sqrt{\bb{\frac {\sqrt{3}}4}^2 + \bb{\frac 12 + \frac 14}^2} = \sqrt{3}/2 < 1$. Thus $\abs{z_3}<1$ also holds. Hence, $\Z[\sqrt{-2}]$ is a Euclidean domain.
\een
\end{example}

%\begin{remark}
%Similar arguments apply for other subrings of $\C$ and one can sometimes show that the norm is a Euclidean function but one needs that any point in the complex plane is distance less than 1 from a lattice point of $R$. Note for $\Z[\sqrt{-5}]$ this is not true.
%\end{remark}

\begin{proposition}
The subring $\Z[\sqrt{2}]$ of $\R$ is a Euclidean domain. Its units are $\pm(1\pm \sqrt{2})^n$ for $n \geq 0$.
\end{proposition}

\begin{proof}[\bf Proof]
For $a,b \in \Z$, we define a function
\be
\phi:\Z[\sqrt{2}] \to \Z^+,\ a+b\sqrt{2} \mapsto \abs{\bb{a+b\sqrt{2}}\bb{a-b\sqrt{2}}} = \abs{a^2 - 2b^2}.
\ee

Then for $a+b\sqrt{2}, c+ d\sqrt{2} \in \Z[\sqrt{2}]$,
\beast
\phi\bb{\bb{a+b\sqrt{2}}\bb{c+d\sqrt{2}}} & = & \phi \bb{ac + 2bd + (ad + bc)\sqrt{2}} = \abs{(ac + 2bd)^2 - 2(ad + bc)^2} \\
& = & \abs{a^2c^2 + 4b^2 d^2 + 4abcd - 2a^2d^2 - 2b^2c^2 - 4abcd} = \abs{(a^2 - 2b^2)(c^2 - 2d^2)} \\
& = & \abs{a^2 - 2b^2}\abs{c^2 - 2d^2} = \phi(a+b\sqrt{2})\phi(c+d\sqrt{2}).
\eeast

Also, since $a,b\in \Z$
\be
\phi(a+ b\sqrt{2}) = \abs{a^2 - 2b^2} \geq 1 \text{ if }a+ b\sqrt{2}\neq 0.
\ee

So we have for $x,y \in \Z[\sqrt{2}]\bs\bra{0}$,
\be
\phi(xy) \geq \phi(x).
\ee

Now we need to prove that if $a,b\in \Z[\sqrt{2}]\bs \bra{0}$ with $b\neq 0$, $\exists q,r \in \Z[\sqrt{2}]$ s.t.
\be
a= bq + r,\quad \text{with either $r=0$ or }\phi(r) < \phi(b).
\ee

Given $a,b$, if $b\mid a$, i.e. $a = bx$ for some $x\in \Z[\sqrt{2}]$, we have that $r=0$. Otherwise, we can find some $s,t \in \R$ such that
\be
a /b = s + t\sqrt{2}.
\ee

Then we can find $x,y \in \Z$ such that $\abs{x-s} \leq \frac 12$ and $\abs{y -t }\leq \frac 12$. Let
\be
q = x + y \sqrt{2} \ \ra\ a = b(x+ y\sqrt{2}) + \underbrace{b(s-x)}_{\in \Z} + \underbrace{b(t-y)}_{\in \Z}\sqrt{2}
\ee
with $r = b(s-x) + b(t-y) \sqrt{2} \in \Z[\sqrt{2}]$ and
\be
\phi (r) = \phi (b)\abs{ (s-x)^2 - 2 (t-y)^2} \leq \phi(b) \abs{(s-x)^2 + 2(t-y)^2} \leq \phi(b) \abs{\frac 14 + 2 \frac 14} = \frac 34 \phi(b) < \phi(b)
\ee
since $b\neq 0$. Thus, $\phi$ is an Euclidean function and $\Z[\sqrt{2}]$ is an Euclidean domain.

If $x$ is a unit in $\Z[\sqrt{2}]$, then we have that $\exists y \in \Z[\sqrt{2}]$ such that $xy = 1$ and thus $\phi(x)\phi(y) = 1$. Since $\phi(x),\phi(y) \geq 1$, we have that $\phi(x) = 1$.

Since $\sqrt{2}-1 = 1/(1 + \sqrt{2})$, we can consider $1+\sqrt{2}$ only. Now we want to find all the units bigger than 1.

First we prove that $1+\sqrt{2}$ is actually the smallest unit bigger than 1.

Let $a,b\in \N$. If the smallest unit bigger than 1 is of the form $a-b\sqrt{2} > 1$, then $a+b\sqrt{2} > 1$. However,
\be
1 = \phi(a-b\sqrt{2}) = \abs{\bb{a-b\sqrt{2}}\bb{a+b\sqrt{2}}} > \abs{1} = 1. \text{ (Contradiction.)}
\ee

Similarly, If the smallest unit bigger than 1 is of the form $-a+b\sqrt{2} > 1$, then $a+b\sqrt{2} > 1$. This will lead to the contradiction again.

Thus, it has to be of the form  $a+b\sqrt{2}$. Thus, the smallest unit bigger than 1 is $1+\sqrt{2}$.

Clearly, the power of $1+\sqrt{2}$ are units. If there exists any unit between $\bb{1+\sqrt{2}}^n$ and $\bb{1+\sqrt{2}}^{n+1}$ with integer $n$ (positive or negative), say $u$,
\be
\bb{1 + \sqrt{2}}^n < u < \bb{1+\sqrt{2}}^{n+1} \ \ra \ 1 < u \bb{1+\sqrt{2}}^{-n} < 1+\sqrt{2}.
\ee

Obviously, $u \bb{1+\sqrt{2}}^{-n}$ is a unit since $u^{-1}$ and $\bb{1+\sqrt{2}}^{n}$ are units. However, $1+\sqrt{2}$ is the smallest unit bigger than 1. This leads to contradiction. Thus, there is no other unit bigger than 1 besides $(1+\sqrt{2})^n$ with $n\in \N$.
\end{proof}


\begin{proposition}\label{pro:field_implies_ed}
A field $\F$ is also an ED (Euclidean domain).
\end{proposition}

\begin{proof}[\bf Proof]
Let $\phi(x) = 1$ for any non-zero element $x$. Clearly, this function satisfies (i) of Definition \ref{def:euclidean_domain}.

Also for $a,b\in \F$, $1/b\in \F$ and there exists $q = a/b \in \F$ such that $a = qb$ which satisfies (ii) of Definition \ref{def:euclidean_domain}.
\end{proof}

\begin{proposition}\label{pro:ed_implies_pid}
If an integral domain $R$ is an ED (Euclidean domain) then $R$ is a PID (principal ideal domain).
\end{proposition}

\begin{proof}[\bf Proof]%Let $R$ be an integral domain.
Let $R$ be an ED (therefore an integral domain) with Euclidean function $\phi: R \bs \{0\} \to \Z^+$. %_{\geq 0} = \{n \in \Z : n \geq 0\}$.

Let $I /R$ be an ideal and suppose that $I$ is non-zero. Pick $b \in I \{0\}$ with $\phi(b)$ minimal. Then $\forall a \in I$ and use Euclidean algorithm (Proposition \ref{pro:euclidean_algorithm_polynomial}): $a = bq + r$ with $r = 0$ or $\phi(r) < \phi(b)$.

Note that $r = a - bq \in R$, and hence minimality of $\phi(b)$ implies $r = 0$. Thus $a = bq$, so $I = \bsa{b}$. Thus all ideals are prime ideals and therefore $R$ is a PID (by Definition \ref{def:principal_ideal_domain}).
\end{proof}



\section{Ascending Chain Condition and Hilbert's basis theorem}

\subsection{Ascending chain condition}

\begin{definition}[ascending chain condition\index{Ascending chain condition!ring}]\label{def:ascending_chain_condition}
Let $R$ be a (not necessarily commutative) ring.  We say that $R$ satisfies the ascending chain condition\footnote{need general definition, also need DCC} (ACC\index{ACC!ring}) if for $I_j \lhd R$ with $I_1 \subseteq I_2 \subseteq I_3 \subseteq \dots$, there exists some $n \in \N$ such that $I_n = I_{n+i}$ for all $i \geq 0$.
\end{definition}

\begin{definition}[Noetherian ring\index{Noetherian!ring}]\label{def:noetherian_ring}
Rings satisfying the ACC are called Noetherian rings (Named after Emmy Noether).
\end{definition}


\begin{lemma}\label{lem:pid_implies_acc}
Let $R$ be a PID and $I_j \lhd R$ with $I_1 \subseteq I_2 \subseteq I_3 \subseteq \dots$. Then, $R$ satisfies ACC.
\end{lemma}

\begin{proof}[\bf Proof]
By Proposition \ref{pro:union_ascending_ideal_is_ideal}, $I = \bigcup_j I_j \lhd R$. Since $R$ is a PID, $I = \bsa{a}$ for some $a \in I$. We must have $a \in I_n$ for some $n \in \N$. For all $i \geq 0$,
\be
\bsa{a} \subseteq I_n \subseteq I_{n+i} \subseteq \dots \subseteq I \subseteq \bsa{a},
\ee
so we have equality throughout and in particular $I_{n+i} = I_n$.
\end{proof}

%Recall that we showed that a PID satisfies the ascending chain condition (ACC) on ideals.

\begin{lemma}\label{lem:noetherian_iff_ideal_finitely_generated}
A ring $R$ is Noetherian if and only if all ideals in $R$ are finitely generated.
\end{lemma}

\begin{proof}[\bf Proof]
Suppose that all ideals are finitely generated and consider a chain
\be
I_1 \subseteq I_2 \subseteq \dots
\ee
with $I_j \lhd R$. Then the union $\bigcup_j I_j \lhd R$ is an ideal too (by Proposition \ref{pro:union_ascending_ideal_is_ideal}). So by supposition $\bigcup_j I_j$ is finitely generated. There exists $N$ such that all these finitely many generators lie in $I_N$.

If $m \geq N$ then $I_m \subseteq \bigcup_j I_j \subseteq I_N$. Also $I_N \subset I_m$ and so $I_m = I_N$.

Conversely, assume the ACC and let $J \lhd R$. Take $a_1\in J$ with $a_1 \neq 0$. If $J \neq \bsa{a_1}$ (i.e. $\bsa{a_1} \subsetneq J$ since $a_1 r \in J$ for any $r\in R$ with the definition of ideal) pick $a_2 \in J \bs \bsa{a_1}$. If $J \neq \bsa{a_1, a_2}$ (i.e., $\bsa{a_1,a_2} \subsetneq J$ since $a_1 r_1 + a_2 r_2  \in J$ for any $r_1,r_2\in R$ with the definition of ideal) pick $a_3 \in J \bs \bsa{a_1, a_2}$ etc. We are producing a chain
\be
\bsa{a_1} \subsetneq \bsa{a_1, a_2} \subsetneq \bsa{a_1, a_2, a_3} \subsetneq \dots .
\ee

By the ACC this process stops, so $J = \bsa{a_1,\dots , a_N}$ for some $N$.
\end{proof}

\begin{remark}
For a commutative ring to be Noetherian it suffices that every prime ideal of the ring is finitely generated. (The result is due to I. S. Cohen.)\footnote{need details}
\end{remark}

%\begin{definition}
%A ring with these properties is called Noetherian\index{Noetherian}.
%\end{definition}

\subsection{Hilbert's basis theorem}

\begin{theorem}[Hilbert's basis theorem\index{Hilbert's basis theorem}]\label{thm:hilbert_basis_ring}
If the commutative ring $R$ is Noetherian then $R[X]$ is Noetherian.
\end{theorem}

\begin{proof}[\bf Proof]
Let $J \lhd R[X]$ be an ideal. We aim to show that it is finitely generated. Consider
\be
I_j = \left\{ a_j \in R : \exists f(X) \in J, \ f(X) = \sum^j_{i=0} a_iX^i \in J\right\} \cup \{0\},
\ee
the set of leading coefficients of polynomials of degree $j$ in $J$.

$I_j \lhd R$ is an ideal since if $\sum^j_{i=0} a_iX^i \in J$, $\sum^j_{i=0} b_iX^i \in J$ then
\be
\sum^j_{i=0} (a_i + b_i)X^i \in J
\ee
and if $a \in R$ then $\sum^j_{i=0} aa_iX^i \in J$. $I_j \subseteq I_{j+1}$ since if $\sum^j_{i=0} a_iX^i \in J$ then $X \bb{\sum^j_{i=0} a_iX^i} \in J$.

The ACC for $R$ implies that there exists $N$ with $I_m = I_N$ for all $m \geq N$ and $I_N$ is finitely generated by the leading coefficients of $f_1(X),\dots , f_k(X)$, say (by Definition \ref{def:ascending_chain_condition} and \ref{def:noetherian_ring}).

Now take any $f(X) \in J$ of degree $m \geq N$. The leading coefficient of $f(X)$ lies in $I_m = I_N$ and so there exists $r_1,\dots,r_k \in R$ so that $r_1f_1(X) + \dots + r_kf_k(X)$ has the same leading coefficient. So
\be
f(X) - (r_1f_1(X) + \dots + r_kf_k(X))X^{m-N} \in J
\ee
and is of degree less than $m$. Repeating this process yields $q_1(X),\dots , q_k(X) \in R[X]$ such that
\be
f(X) - (q_1(X)f_1(X) + \dots + q_k(X)f_k(X)) \in J
\ee
of degree less than $N$.

Now consider polynomials in $J$ of degree less than $N$. For $j < N$ there is a finite set $S_j$ of polynomials in $J$ of degree $j$ whose leading coefficients generate $I_j$ ($I_j$ is finitely generated since $R$ is Noetherian by Lemma \ref{lem:noetherian_iff_ideal_finitely_generated}). Let $S = \bigcup_{j<N} S_j$, a finite set (union of finitely many finite sets).

A similar argument to the one before shows that any polynomial in $J$ of degree less than $N$ is of the form $\sum g_i(X)h_i(X)$ for $g_i(X) \in S$, $h_i(X) \in R[X]$.

Thus $J$ is generated by $S \cup \{f_1(X),\dots,f_k(X)\}$.
\end{proof}

\begin{corollary}
If $\F$ is a field then $\F[X_1,\dots ,X_n]$ is Noetherian.
\end{corollary}

\begin{proof}[\bf Proof]
$\F$ is field and thus it is PID. So $\F$ is Noetherian by Lemma \ref{lem:pid_implies_acc}. Hence $\F[X]$ is Noetherian by Hilbert's basis theorem (Theorem \ref{thm:hilbert_basis_ring}). Then apply Hilbert's basis theorem (Theorem \ref{thm:hilbert_basis_ring}) repeatedly, we have that $\F[X_1,\dots ,X_n]$ is Noetherian.
\end{proof}

\begin{corollary}
$\Z[X_1,\dots ,X_n]$ is also Noetherian.
\end{corollary}

\begin{proof}[\bf Proof]
$\Z$ is ED. Then $\Z$ is PID and thus $\Z$ is Noetherian by Lemma \ref{lem:pid_implies_acc}. Hence $\Z[X]$ is Noetherian by Hilbert's basis theorem (Theorem \ref{thm:hilbert_basis_ring}). Then apply Hilbert's basis theorem (Theorem \ref{thm:hilbert_basis_ring}) repeatedly, we have that $\Z[X_1,\dots ,X_n]$ is Noetherian.
\end{proof}

\begin{corollary}
Any ring image of $\Z[X_1,\dots ,X_n]$ is Noetherian.
\end{corollary}

\begin{proof}[\bf Proof]
Let $\theta : \Z[X_1,\dots ,X_n] \to S$ be a homomorphism. Then $\theta$ is surjective (since it is image of ring). If $I \lhd S$ is an ideal then
\be
J = \{f(X_1,\dots ,X_n) \in \Z[X_1,\dots ,X_n] : \theta(f(X_1,\dots ,X_n)) \in I\}
\ee
is finitely generated. Then $I$ is generated by the images under $\theta$ of a finite generating set of $J$.
\end{proof}


\section{Unique Factorisation Domain}

\subsection{Unique factorisation domain}

\begin{definition}[unique factorisation domain\index{unique factorisation domain}]\label{def:ufd_integral_domain}
An integral domain is a unique factorisation domain (UFD\index{UFD!unique factorisation domain}) if
\ben
\item [(i)] every non-zero element that is not a unit may be expressed as a product of finitely many irreducibles,
\item [(ii)] whenever $p_1 \dots p_m = q_1 \dots q_n$ for products of irreducibles then $m = n$ and we can reorder so that $p_i$ is an associate of $q_i$. (So factorisation is unique up to ordering and associates).
\een
\end{definition}


\begin{example}\label{exa:sqrtroot_3_5_not_ufd}
$\Z[\sqrt{-5}]$ is not a UFD. Note that
\be
6 = 2 \cdot 3 = (1 + \sqrt{-5})(1 - \sqrt{-5})
\ee
where 2, 3, $1 +\sqrt{-5}$ and $1 -\sqrt{-5}$ are irreducibles (see Example \ref{exa:z_sqrtroot_minus_5_prime_irreducible}). Units are $\pm 1$ in $\Z[\sqrt{-5}]$, so these irreducibles are not associates of each other.
\end{example}

%Our goal is the following proposition.

%We need two lemmas about PIDs to prove Proposition \ref{pro:pid_implies_ufd}.


\begin{proposition}\label{pro:pid_implies_ufd}
If an integral domain $R$ is a PID then $R$ is a UFD.
\end{proposition}


%Proof of (\ref{pro:pid_implies_ufd}).

\begin{proof}[\bf Proof]
\ben
\item [(i)] Let $a \in R$, non-zero and not a unit. Assume it cannot be factorised as a product of finitely many irreducibles. So in particular a itself is not irreducible (otherwise it is a product of finitely many irreducibles), hence $a = a_1b_1$ with $a_1$, $b_1$ non-zero and not units (by definition of irreducible element). We may assume that $a_1$ cannot be factorised as a product of finitely many irreducibles, so write $a_1 = a_2b_2$ and continue. (If both $a_1$, $b_1$ can be factorised into finitely many irreducibles, then so can $a$.)

So we have that $\bsa{a_1} \subseteq \bsa{a_2} \subseteq \bsa{a_3} \subseteq \dots$ in PID. Thus, by Lemma \ref{lem:pid_implies_acc}, there exists $n\in \N$ such that for all $i\geq 0$, $\bsa{a_n} = \bsa{a_{n+i}}$. In particular, $\bsa{a_n} = \bsa{a_{n+1}}$, then $a_n$ and $a_{n+1}$ would be associates and $b_{n+1}$ a unit. Contradiction.

%We obtain $\bsa{a_1} \subsetneq (a_2) \subsetneq (a3) \subsetneq \dots$ with inequality at each stage, for if $(a_i) = (a_{i+1})$ then $a_i$ and $a_{i+1}$ would be associates and $b_{i+1}$ a unit.

\item [(ii)] Suppose we have $p_1 \dots p_m = q_1 \dots q_n$ with products of irreducibles. Then $p_1$ is prime by Lemma \ref{lem:pid_irreducible_prime} and $p_1 | q_1 \dots q_n$, so $p_1$ divides one of the $q_i$ (by Definition \ref{def:associativity_irreducibility_primeness_integral_domain}), $p_1 | q_1$, say, and so $q_1 = p_1u$. But $q_1$ is irreducible and $p_1$ is not a unit, hence $u$ is a unit (by Definition \ref{def:associativity_irreducibility_primeness_integral_domain}). This is to say $p_1$ and $q_1$ are associates. We have (cancellation in $R$)
\be
p_1 \dots p_m = up_1q_2 \dots q_n \ \ra \ p_1 \bb{p_2 \dots p_m - uq_2\dots q_n} = 0 \ \ra \ p_2 \dots p_m = uq_2\dots q_n
\ee
%\be
%p_2 \dots p_m = (uq_2)q_3 \dots q_n.
%\ee

Note that $(uq_2)$ is irreducible. Continuing in this way gives the result.
\een
\end{proof}

\begin{corollary}\label{cor:pid_implies_ufd}
$\Z$, $\F[X]$ for a field $\F$ and $\Z[i]$ are all UFDs.
\end{corollary}
\begin{proof}[\bf Proof]
Direct result from Proposition \ref{pro:pid_examples} and Proposition \ref{pro:pid_implies_ufd}.
\end{proof}


\begin{remark}
The proof of (ii) depends just on Lemma \ref{lem:pid_irreducible_prime}, that irreducibles are prime, and (i) depends on the ACC.
\end{remark}

There are several properties which follow quickly from the definition of a UFD.

\begin{proposition}\label{pro:irreducible_iff_prime_ufd}
Let $R$ be a UFD. An element $p \in R$ is irreducible if and only if it is prime.
\end{proposition}	

\begin{proof}[\bf Proof]
We have already shown the 'if' direction for integral domains in Lemma \ref{lem:prime_implies_irreducible_integral_domain}.

Suppose $p$ is an irreducible and $p \mid ab$. We have $a,b$ are non-zero (otherwise $p$ is zero). If $a$ or $b$ is a unit, wlog, $a$ is, say. Then $cp = ab$ for some $c\in R$. Then $b = a^{-1}cp$ since $a$ is unit, so $p\mid b$. Thus, $p\mid a$ or $p\mid b$ and therefore $p$ is prime.

If $a$ and $b$ are not units, we can express $a$ and $b$ as
\be
a = p_1 \dots p_m,\quad b = q_1 \dots q_n
\ee
with $p_i$, $q_i$ irreducible. Now write $ab = pc$ with $c = r_1 \dots r_s$ where $r_i$ is irreducible. Then
\be
(p_1 \dots p_m)(q_1 \dots q_n) = pr_1 \dots r_s
\ee
and uniqueness of factorisation (Definition \ref{def:ufd_integral_domain}.(ii)) implies that $p$ is an associate of either some $p_i$ or some $q_i$. Thus, $u_p p = p_i$ or $u_qp = q_i$ where $u_p,u_q$ are units. So
\be
a = p_1 \dots p_m = u_p p \cdot (\cdot) \quad \text{ or }\quad b = q_1 \dots q_n = u_q p \cdot (\cdot).
\ee

Hence $p \mid a$ or $p \mid b$ and therefore $p$ is prime.
\end{proof}

\subsection{Highest common factor}

\begin{definition}[Highest common factor\index{highest common factor!rings}]\label{def:hcf_ring}
$d$ is a highest common factor of $a_1,\dots , a_n$, written $\hcf(a_1,\dots , a_n)$, if $d | a_i$ for each $i$ and whenever $d'$ also divides each $a_i$ then $d' | d$.
\end{definition}

\begin{remark}
Note that one often refers to the highest common factor, although certainly multiplying by a unit will give another highest common factor.
\end{remark}

\begin{proposition}\label{pro:hcf_ufd_unique_unit}
Highest common factors exist in UFD and are unique up to multiplication by a unit.
\end{proposition}

\begin{proof}[\bf Proof]
If $a_i$ is zero, $\hcf(a_1,\dots,a_n) = \hcf(a_1,\dots,a_{i-1},a_{i+1},\dots,a_n)$. Thus, we only consider the case that $a_i$ is unit or a product of irreducibles, each $a_i$ is of the form
\be
u_i \prod_j p^{n_{ij}}_j
\ee
where $p_j$ is irreducible and $p_j$ is not associate to $p_k$ whenever $j \neq k$. $u_i$ is a unit in $R$ and $n_{ij} \geq 0$.

The claim is that $\prod_j p^{m_j}_j$ is a highest common factor of $a_1,\dots , a_n$ where $m_j = \min_i\{n_{ij}\}$. It is clear that this is a factor of each $a_i$ i.e., $d\mid a_i$, and if $d' \mid a_i$ for each $i$.

If $a_i$ is a unit, we have $a_i = cd'$ for some $c\in R$, then $\bb{a_i^{-1}c}d' =1$ and therefore $d'$ is a unit. If $a_i$ is not a unit, then each irreducible dividing $d'$ is also an irreducible dividing $a_i$ and so must be one of the $p_j$ (by Proposition \ref{pro:irreducible_iff_prime_ufd}). Thus
\be
d' = u\prod_j p^{t_j}_j
\ee
where $u$ is a unit. But the $t_j$ can be at most $\min_i\{n_{ij}\}$ if $d' | a_i$ for each $i$. Thus
\be
d = \prod_j p^{m_j}_j = p \cdot \prod_j p^{t_j}_j = p u^{-1} d' \ \ra \ d' \mid d
\ee
where $p\in R$.

Suppose $d$ and $d'$ are highest common factors of $a$ and $b$, then $d\mid d'$ and $d' \mid d$. Thus, for $r_1,r_2 \in R$, we have
\be
d' = dr_1, \ d = d'r_2 \ \ra \ d' = d' r_1r_2 \ \ra \ r_1r_2 = 1
\ee
since $R$ is an integral domain. Thus, $r_1,r_2$ are units. So highest common factors are unique up to multiplication by a unit.
\end{proof}

\begin{definition}[Lowest common multiple]
\footnote{need LCM}
\end{definition}

\begin{proposition}%\label{}
Lowest common multiples exist in UFD.
\end{proposition}

\begin{proof}[\bf Proof]
\footnote{need to prove}
\end{proof}
%%%%%%%%%%%%%%%%%%%%%%%%%%%%%%%%%%%%%%%%%%%





%\section{Factorisations in polynomial rings, Gauss' lemma and Eisenstein's criterion}

\section{Gauss' Lemma and Eisenstein's Criterion}

%If $\F$ is a field then $\F[X]$ is a ED, PID and UFD. Every ideal $I \lhd \F[X]$ is principal, $I = (f(X))$ for some $f(X) \in \F[X]$. An element is irreducible if and only if it is prime. $\F[X]/I$ is a field, with elements of the form $r(X)+I$ with $r(X) = 0$ or $\deg r < \deg f$, if and only if $I$ is maximal if and only if $f(X)$ is irreducible in $\F[X]$.

\subsection{Content and primitiveness}

\begin{definition}[coprime\index{coprime!elements of ring}]\label{def:coprime_ring}
Let $R$ be an integral domain and $a_1,\dots,a_n\in R$. Then we say $a_1,\dots,a_n$ are coprime if the highest common factor of $a_1,\dots,a_n$ is a unit.
\end{definition}

\begin{definition}[content\index{content!polynomial}, primitiveness\index{primitive!polynomial}]
Assume that the ring $R$ is a UFD. Let $f(X)\in R[X]$ with $f(X) = a_0 + \dots + a_nX_n$ with $a_n \neq 0$, $\deg f = n$. The content $c(f(X))$ is the highest common factor of $a_0,\dots, a_n$. $f(X)$ is primitive if $c(f(X))$ is a unit, i.e. if the $a_i$ are coprime.%\footnote{need definition}.
\end{definition}


\begin{lemma}\label{lem:primitive_product}
Let the ring $R$ be a UFD. If $f(X)$ and $g(X)$ are primitive in $R[X]$ then so is $f(X)g(X)$.
\end{lemma}

\begin{proof}[\bf Proof]
Let
\be
f(X) = a_0 + a_1X + \dots + a_mX^m,\quad g(X) = b_0 + b_1X + \dots + b_nX^n
\ee
be primitive. Suppose the product $f(X)g(X)$ is not primitive. So $c(f(X)g(X))$ is not a unit and there is a prime $p \in R$ dividing $c(f(X)g(X))$ (by Definition \ref{def:ufd_integral_domain}) but $p \nmid c(f(X))$ and $p \nmid c(g(X))$. (If $p\mid c(f(x)) =u$ where $u$ is a unit since $f(x)$ is primitive, then $pa = u$ for some $a\in R$. Then $u^{-1}a p = 1$, so $p$ is a unit. Contradiction.) % $f(x)$ and $g(x)$ are primitive, i.e).

Let $k$ and $l$ be the first indice such that $p\nmid a_i$, $p\nmid b_j$ (otherwise, if $p\mid a_i$, $p \mid b_j$ for all $i,j$, we have $p\mid c(f(x)))$ and $p\mid c(g(x))$),
\be
p \mid a_0,\ p \mid a_1,\ \dots ,\ p \mid a_{k-1},\ p \nmid a_k,\quad p \mid b_0,\ p \mid b_1,\ \dots ,\ p \mid b_{l-1},\ p \nmid b_l.
\ee

Thus, $p\nmid a_k b_l$ by the definition of primeness. The coefficient $c_{k+l}$ of $X^{k+l}$ in $f(X)g(X)$ is
\be
\dots + a_{k+1}b_{l-1} + a_kb_l + a_{k-1}b_{l+1} + \dots .
\ee
$p$ divides all of these terms apart from $a_kb_l$ which it does not divide. So $p \nmid c_{k+l}$, contradicting that $p | c(f(X)g(X))$. Thus $f(X)g(X)$ is primitive.
\end{proof}


Consider the polynomial ring $R[X]$ and let $\F$ be the field of fractions of $R$.

\begin{corollary}\label{cor:content_associate}
Let the ring $R$ be a UFD. For $f(X), g(X) \in R[X]$, the content $c(f(X)g(X))$ is an associate of $c(f(X))c(g(X))$. (Recall that highest common factors and hence contents are only
defined up to associates.)
\end{corollary}

\begin{proof}[\bf Proof]
As $R$ is a UFD we may write
\be
f(X) = c(f(X))f_1(X),\quad\quad g(X) = c(g(X))g_1(X)
\ee
where $f_1(X), g_1(X)$ are primitive. Then $f_1(X)g_1(X)$ is primitive by Lemma \ref{lem:primitive_product}. Thus,
\be
f(X)g(X) = c(f(X))c(g(X))f_1(X)g_1(X)
\ee

So a highest common factor of the coefficients of $f(X)g(X)$ is $c(f(X))c(g(X))$, i.e.,
\be
c(f(X)g(X)) = c(f(X))c(g(X))c(f_1(X)g_1(X)) = c(f(X))c(g(X)) u.
\ee

where $u$ is a unit. That is, $c(f(X)g(X))$ is an associate of $c(f(X))c(g(X))$
\end{proof}



\begin{proposition}\label{pro:ufd_irreducible_or_primitive}
Let the ring $R$ be a UFD. An irreducible $f(X)$ in $R[X]$ is either in $R$ (and irreducible in $R$) or it is primitive in $R[X]$.
\end{proposition}

\begin{proof}[\bf Proof]
Assume that $f(X)$ is irreducible and write $f(X) = c(f(X))f_1(X)$ with $f_1(X)$ primitive.

If $f_1(X) \in R$, $c(f(X))\in R$ implies $f(X) \in R$.

If $f_1(X) \notin R$ then, since a polynomial of degree at least 1 cannot be a unit, $c(f(X))$ must be a unit by Definition \ref{def:associativity_irreducibility_primeness_integral_domain}. Then $f(X)$ is primitive in $R[X]$.
\end{proof}

\subsection{Gauss' lemma}

Gauss' lemma helps to determine when polynomials are irreducible in $\F[X]$.

\begin{lemma}[Gauss' lemma\index{Gauss' lemma!polynomial}]\label{lem:gauss_polynomial_irreducible}
Let $R$ be a UFD with $\F$ its field of fractions. Suppose $f(X) \in R[X]$ is primitive. Then $f(X)$ is irreducible in $R[X]$ if and only if $f(X)$ is irreducible in $\F[X]$.
\end{lemma}

\begin{remark}
In particular, when $R = \Z$ then $f(X)$ is irreducible in $\Z[X]$ if and only if it is irreducible in $\Q[X]$.
\end{remark}

\begin{proof}[\bf Proof]
($\la$) Take $f(X)$ primitive in $R[X]$. Suppose $f(X)$ is not irreducible in $R[X]$, so it factorises in $R[X]$ as a product of non-units in $R[X]$.

Suppose $f(X) = ab$, $a,b\in R[X]$ and $a,b$ are non-units. If $a\in R$, then $a\mid f(X)$ and $a\mid a_i$ for all coefficients $a_i$ of $f(X)$. Thus, $a\mid a_i$ for all $a_i$. By the definition of highest common factor (Definition \ref{def:hcf_ring}), $a\mid c(f(X))$. Since $f(X)$ is primitive, $c(f(X))$ is a unit. Thus, there exists $b\in R$ such that $ab = u$ where $u$ is a unit. Therefore $1 = a bu^{-1}$ and $a$ is also a unit. Contradiction. Hence, neither of these non-units is in $R$, the factors have degrees greater than 0.

%Since $f(X)$ is primitive in $R[X]$, neither of these non-units is in $R$, the factors have degrees greater than 0.

%by Proposition \ref{pro:ufd_irreducible_or_primitive} $f(X)$ is not in $R$.

So these factors are non-units in $\F[X]$. So $f(X)$ factorises as a product of non-units in $\F[X]$ and therefore $f(X)$ is not irreducible in $\F[X]$. This means that if $f$ is irreducible in $\F[X]$, it must be irreducible in $R[X]$.

($\ra$) Suppose that $f(X)$ is not irreducible with $f(X) = g(X)h(X)$ with $g(X), h(X) \in \F[X]$ non-units and hence not constant. Multiply by $a \in R$ and $b \in R$ repectively such that $ag(X), bh(X) \in R[X]$ (existence is guaranteed by Definition \ref{def:field_of_fractions}). Thus
\be
abf(X) = (ag(X))(bh(X)).
\ee

Write
\be
ag(X) = c(ag(X))g_1(X),\quad\quad bh(X) = c(bh(X))h_1(X)
\ee
where $g_1(X)$, $h_1(X)$ are primitive in $R[X]$. Note that $g_1(X)$, $h_1(X)$ are not constant and hence not units. Now Corollary \ref{cor:content_associate} implies that $ab$ is an associate of $c(ag(X))c(bh(X))$ by considering contents. So
\be
abf(X) = uc(ag(X))c(bh(X))u^{-1}g_1(X)h_1(X)
\ee
where $ab = uc(ag(X))c(bh(X))$ and $u$ a unit in $R$. Cancellation (We know that $ab\neq 0$. Otherwise $f(X) = 0$ and $f(X)$ does not have inverse so it is not primitive.) gives
\be
f(X) = (u^{-1}g_1(X))h_1(X)
\ee
If $u^{-1}g_1(X)$ is a unit then $g_1(X)$ is a unit. Contradiction. Thus $f(X)$ factorises as a product of non-units in $R[X]$.
\end{proof}



\begin{example}
$X^3 + X + 1$ is irreducible in $\Z[X]$, and is primitive in $\Z[X]$ and so Gauss' lemma (Lemma \ref{lem:gauss_polynomial_irreducible}) implies $X^3 +X + 1$ is irreducible in $\Q[X]$.%, and so $\Q[X]/\bsa{X^3 +X + 1}$ is a field\footnote{this need $\bsa{X^3 +X + 1}$ is a maximal ideal}.

Suppose $X^3 +X +1$ is reducible in $\Z[X]$, so $X^3 +X +1 = g(X)h(X)$ for $g(X), h(X) \in \Z[X]$ not units. Primitivity implies that $g(X)$ and $h(X)$ are not constant polynomials, so $\deg g, \deg h \geq 1$ (see the similar argument in proof of Gauss' lemma (Lemma \ref{lem:gauss_polynomial_irreducible})). So one of $g(X)$ and $h(X)$ is of degree 1, write
\be
g(X) = b_0 + b_1X, \quad\quad h(X) = c_0 + c_1X + c_2X^2.
\ee
Considering coefficients of $X^0$ and $X^3$,
\be
b_0c_0 = 1, \quad b_1c_2 = 1.
\ee
So $b_0 = \pm 1$, $b_1 = \pm 1$. This is a contradiction since $\pm 1$ is not a root of $X^3 + X + 1$.
\end{example}



\begin{proposition}
Let $R$ be a UFD with $\F$ its field of fractions. If $f(X) \in R[X]$, not necessarily primitive, and $f(X) = g(X)h(X)$ with $g(X)$ primitive in $R[X], h(X) \in \F[X]$ then
\be
f(X) = g(X)h_0(X)
\ee
with $h_0(X) \in R[X]$.
\end{proposition}

\begin{proof}[\bf Proof]
There exists $b \in R$ with $bh(X) \in R[X]$ (by Definition \ref{def:field_of_fractions}). Consider $bf(X) = g(X)(bh(X))$. Write $bh(X) = c(bh(X))h_1(X)$ with $h_1(X)$ primitive. Consider contents,
\be
bc(h(X)) = c(g(X)(bh(X))) \ \ra \ b \mid c(g(X)bh(X)) \ \ra \ b\mid uc(g(X))c(bh(X))
\ee
by Corollary \ref{cor:content_associate}, where $u$ is a unit. Since $g(X)$ is primitive, we have $b \mid u'c(bh(X))$ where $u'$ is another unit. Thus, we have $b\mid c(bh(X))$ since $u'$ is a unit. Then, let $ab = u'c(bh(X))$ for some $a\in R$,
\be
abf(X) = g(X) c(bh(X)) ah_1(X) = u'c(bh(X)) u'^{-1} a g(X)h_1(X) = ab u'^{-1} a g(X)h_1(X) = ab  g(X) \bb{u'^{-1} ah_1(X) }.
\ee

If $ab = 0$, we have $a=0$ or $b=0$. If $a = 0$, $c(bh(X)) = 0$, $bf(X) = 0$, $b=0$ or $f(X) =0$. If $f(X) = 0$, we have $h(X) = 0$ since $g(X) \neq 0$ ($g(X)$ is primitive). If $b=0$, $bh(X) = 0$ and this is contradiction to Definition \ref{def:field_of_fractions}.(ii).


If $ab \neq 0$, we can do the cancellation and have $f(X) = g(X)\bb{u'^{-1} ah_1(X) } = g(X)h_0(X)$ for some $h_0(X) \in R[X]$.
\end{proof}

Thus, we can have the following proposition.

\begin{proposition}
Let $R$ be a UFD with $\F$ its field of fractions and $g(X)$ be primitive in $R[X]$,
\be
I = g(X)\F[X] \lhd \F[X],\qquad J = g(X)R[X] \lhd R[X] \quad \ra \quad I \cap R[X] = J.
\ee
\end{proposition}


\begin{theorem}\label{thm:r_udf_rx_udf}
If $R$ is a UFD then $R[X]$ is a UFD.
\end{theorem}


\begin{proof}[\bf Proof]
The first step is to show that we need only look at primitives.

Let $f(X) \in R[X]$ be non-zero and a non-unit. Write $f(X) = c(f(X))f_1(X)$ with $f_1(X)$ primitive.

Observe that since R is a UFD, $c(f(X))$ is expressible as a product of irreducibles in $R$ in an essentially unique way. These irreducibles are irreducible in $R[X]$. $p_i \in R$
\be
c(f(X)) = p_1 \dots p_m.\quad (*)
\ee

If $f(X)$ factorises as a product of irreducibles in $R[X]$, then we can collect together the irreducibles in $R$ and the primitive irreducibles. $q_j(X)\in R[X]$ are primitive irreducibles,
\be
f(X) = c(f(X))f_1(X) = p_1 \dots p_m \cdot q_1(X) \dots q_n(X).
\ee

The product of these primitive irreducibles $q_1(X) \dots q_n(X)$ is primitive by Lemma \ref{lem:primitive_product} and so the content of $f(X)$ is an associate of the product of the irreducibles in $R$. That is, by Corollary \ref{cor:content_associate} and primitiveness,
\be
c(f(X)) = c\bb{p_1 \dots p_m \cdot q_1(X) \dots q_n(X)} = u p_1 \dots p_m \cdot c(q_1(X)) \dots c(q_n(X)) = u' p_1 \dots p_m\quad (\dag)
\ee
where $u$, $u'$ are units in $R$. The essentially unique factorisation of $c(f(X))$ means that $(*)$ and $(\dag)$ are essentially the same. Then
\be
f(X) = c(f(X))f_1(X) = u' p_1 \dots p_m u'^{-1}\cdot q_1(X) \dots q_n(X).
\ee

Cancellation implies that the product of the primitives is an associate of $f_1(X)$,
\be
f_1(X) = u'^{-1}\cdot q_1(X) \dots q_n(X).
\ee

Thus we may assume that $f(X)$ is primitive. That is, $c(f(X))$ can be waived since they are in $R$.

We know that $\F[X]$ is a UFD. So we can factorise $f(X)$ in $\F[X]$,
\be
f(X) = p_1(X) \dots p_k(X),
\ee
where $p_i(X)$ is irreducible in $\F[X]$.

There exists $a_i \in R$ with $a_ip_i(X) \in R[X]$ (by Definition \ref{def:field_of_fractions}). We have
\be
a_ip_i(X) = c_iq_i(X)
\ee
with $c_i = c(a_ip_i(X))$, $q_i(X)$ primitive in $R[X]$. $q_i(X)$ is also irreducible in $\F[X]$ since $\F[X]$ is a UFD. Thus, by Gauss' lemma (Lemma \ref{lem:gauss_polynomial_irreducible}), $q_i(X)$ is also irreducible in $R[X]$. So
\be
a_1 \dots a_kf(X) = c_1 \dots c_kq_1(X) \dots q_k(X).
\ee

Considering contents and using the assumption that $f(X)$ is primitive, we have
\be
a_1 \dots a_k = uc_1 \dots c_k
\ee
for some unit $u \in R$ since $q_i(X)$ is primitive by Lemma \ref{lem:primitive_product}, i.e. $q_1(X) \dots q_k(X)$ is primitive.
\be
a_1 \dots a_k f(X) = (uc_1 \dots c_k)(u^{-1}q_1(X))q_2(X) \dots q_k(X).
\ee

Cancellation gives
\be
f(X) = (u^{-1}q_1(X))q_2(X) \dots q_k(X),
\ee
a product of irreducibles in $R[X]$. This shows the existence part. For the uniqueness of the factorisation assume that
\be
f(X) = q_1(X) \dots q_k(X) = r_1(X) \dots r_l(X)
\ee
with $q_i(X)$, $r_i(X)$ irreducible in $R[X]$. We are assuming that $f(X)$ is primitive and so each $r_i(X)$ must be primitive (by Corollary \ref{cor:content_associate}).

By Gauss' lemma (Lemma \ref{lem:gauss_polynomial_irreducible}), $r_i(X)$ is irreducible in $\F[X]$. Uniqueness of factorisation in $\F[X]$ implies that $k = l$, and after reordering we have $q_i(X) = u_ir_i(X)$ with $u_i$ a unit in $\F[X]$ (since $\F[X]$ is a UFD), hence in $\F$.

We can write $u_i = a_i/b_i$ for some $a_i, b_i \in R$ with $b_i \neq 0$. Then $b_iq_i(X) = a_ir_i(X)$. But $q_i(X)$ and $r_i(X)$ are primitive and hence $b_i$ and $a_i$ must be associates being the content of $b_iq_i(X) = a_ir_i(X)$. Cancelling gives that $q_i(X)$ and $r_i(X)$ are associates in $R[X]$. Thus factorisation is essentially unique in $R[X]$.
\end{proof}

\begin{corollary}
Let $R$ be a UFD. Then $R[X_1,\dots ,X_n]$ is a UFD.
\end{corollary}
\begin{proof}[\bf Proof]
We assume that $R$ is a UFD and repeated application of Theorem \ref{thm:r_udf_rx_udf} gives that
\be
R[X_1],\quad R[X_1,X_2] = (R[X_1])[X_2],\quad R[X_1,X_2,X_3] = \bb{(R[X_1])[X_2]}[X_3]\dots.
\ee
Thus, $R[X_1,\dots ,X_n]$ is a UFD.
\end{proof}

\subsection{Eisenstein's criterion}

\begin{proposition}[Eisenstein's criterion\index{Eisenstein's criterion}]\label{pro:eisenstein_criterion}
Let $R$ be a UFD and let
\be
f(X) = a_0+a_1X+ \dots + a_nX^n \in R[X],\ a_n \neq 0,
\ee
be primitive. Assume that for some irreducible $p$ we have $ p \nmid a_n$, $p \mid a_i$ for $0 \leq i < n$ and $p^2 \nmid a_0$. Then $f(X)$ is irreducible in $R[X]$, and hence in $\F[X]$ by Gauss' lemma (Lemma \ref{lem:gauss_polynomial_irreducible}).
\end{proposition}


\begin{proof}[\bf Proof]
Suppose that $f(X) = g(X)h(X)$ with
\be
g(X) = r_kX^k + \dots + r_0,\quad \quad h(X) = s_lX^l + \dots + s_0
\ee
where $r_k, s_l \neq 0$. Note that $k + l = n$. Since $p \mid a_0 = r_0s_0$ and $p^2 \nmid a_0$ we know $p\mid r_0$ or $p\mid s_0$ since $p$ is prime. However, $p\mid r_0$ and $p\mid s_0$ gives that $p^2\mid r_0s_0 = a_0$. Contradiction to the assumption. Thus we may assume that $p | r_0$ and $p \nmid s_0$. $p \nmid a_n = r_ks_l$ implies $p \nmid r_k$, $p \nmid s_l$.

Set $j$ to be such that $p | r_0,\dots , p | r_{j-1}, p \nmid r_j$ (This index $j$ exists since $p\nmid r_k$). Consider the term
\be
a_j = r_0s_j + \dots + r_js_0.
\ee

We have $p \mid r_0s_j, \dots, p\mid r_{j-1}s_1$. If $p\mid r_j s_0$, we have $p\mid r_j$ or $p\mid s_0$ since $p$ is a prime. Contradiction. Thus, $p \nmid a_j$. Thus $j = n$, and hence $k = n$, $l = 0$. As $f(X)$ is primitive,
\be
u = c(f(X)) = u' c(g(X)) c(h(X)) = u's_0 c(g(X)) %\ \ra \ s_0
\ee
where $u,u'$ are units. Thus, $h(X)= s_0$ must be a unit. Thus $f(X)$ cannot be factorised as a product of non-units.
\end{proof}

\begin{example}
Consider $R = \Z$ and let $f(X) = X^n - p \in \Z[X]$, where $p$ is prime.

By Eisenstein's criterion, $f(X)$ is irreducible in $\Z[X]$ and hence in $\Q[X]$ (by Gauss' lemma (Lemma \ref{lem:gauss_polynomial_irreducible}) since $\Q$ is field of fractions of $\Z$). So $f(X)$ does not have any roots in $\Q$, as a root would yield a linear factor of $f(X)$ in $\Q[X]$. Thus $p$ has no nth roots in $\Q$.
\end{example}

\begin{example}
Consider the cyclotomic polynomial
\be
f(X) = X^{p-1} + X^{p-2} + \dots + 1
\ee
where $p$ is prime. The claim is that this is irreducible in $\Z[X]$, and hence in $\Q[X]$. Note that $(X - 1)f(X) = X^p - 1$. We make the substitution $X = 1 + Y$ and get
\be
Y f(1 + Y ) = (1 + Y )^p - 1, \quad\quad f(1 + Y ) = Y^{p-1} + \binom{p}{1} Y^{p-2} + \dots + \binom{p}{p - 1}
\ee
where $\binom{p}{i}$ are binomial coefficients. Eisenstein's criterion does now apply, we have $p |\binom{p}{i}$, $p^2 \nmid\binom{p}{p-1} = p$. We deduce that $f(1 + Y )$ is irreducible and hence $f(X)$ is irreducible in $\Z[X]$, and hence in $\Q[X]$.
\end{example}


\section{$\Z[\cdot]$}

\subsection{Gaussian integers $\Z[i]$}

\begin{definition}[Gaussian integer\index{Gaussian integer}]\label{def:gaussian_integer}
Gaussian integers are the set $\Z[i]=\bra{a+ib :a,b\in \Z }$. The norm (see Definition \ref{def:norm_ring})
\be
N(a + ib) = a^2 + b^2 = (a + ib)(a - ib) = z\bar{z},
\ee
where $z = a + ib$, is multiplicative, that is,
\be
N(z_1z_2) = N(z_1)N(z_2).
\ee

The units have to be of norm 1, hence the only units are $\pm 1$, $\pm i$.

\end{definition}

%We consider the set of Gaussian integers\index{Gaussian integers} as introduced before,
%\be
%\Z[i] = \{a + ib : a, b \in \Z\}.
%\ee


\begin{example}
Recall from an earlier example that $\Z[i]$ is a Euclidean domain and hence a PID and a UFD. Thus the irreducibles and primes are the same, by Lemma \ref{lem:prime_implies_irreducible_integral_domain} and Lemma \ref{lem:pid_irreducible_prime}.
\bit
\item $2 = (1 + i)(1 - i)$ is not irreducible.
\item 3 is irreducible since $N(3) = 9$. If it were to factorise as a product of two non-units then they would have norm 3. But there are no elements of norm 3.
\item $5 = (1 + 2i)(1 - 2i)$ is not irreducible.
\eit
\end{example}

\begin{lemma}\label{lem:square_modulo}
A prime number $p$ in $\Z^+$ is irreducible (and prime) in $\Z[i]$ if and only if $p$ is not of the form $x^2 + y^2$ with $x, y \in \Z \bs \bra{0}$.
\end{lemma}

\begin{proof}[\bf Proof]
$(\ra)$. If $p = x^2 + y^2$ then $p = (x + iy)(x - iy)$ (where $x+iy,x-iy$ are not units) and so $p$ is not irreducible in $\Z[i]$.

$(\la)$. Consider $N(p) = p^2$, since $p$ is not irreducible, it factorises into two non-units, necessarily of norm $p$ ($p = z_1z_2$, $p^2 = N(p) = N(z_1)N(z_2)$, then $N(z_1) = N(z_2) = p$, since $N(z_1),N(z_2)\neq 1$), only if there is $x + iy$ with $N(x + iy) = p$. So $x^2 + y^2 = p$.
\end{proof}


\begin{proposition}\label{pro:irreducible_associate_prime_p}
The irreducibles (and primes) in $\Z[i]$ are up to associates
\ben
\item [(i)] $p \in \Z$ prime with $p \equiv 3 \lmod{4}$,
\item [(ii)] $z$ with $z\bar{z} = p$, for $p$ prime in $\Z$ with $p = 2$ or $p \equiv 1 \lmod{4}$.
\een
\end{proposition}


\begin{proof}[\bf Proof]
Let us first show that these are actually irreducibles (and primes).
\ben
\item [(i)] If $p \equiv 3 \lmod{4}$ then it is not of the form $x^2 + y^2$ since squares modulo 4 are 0 or 1 and so $x^2 + y^2 \equiv 0,\ 1 \text{ or }2 (\bmod 4)$. By Lemma \ref{lem:square_modulo}, $p$ is irreducible (and prime) in $\Z[i]$.

\item [(ii)] \ben
\item [(a)] If $p = 2 = (1 + i)(1 - i)$, % is of the correct form,
$1 \pm  i$ are irreducibles of norm 2.

\item [(b)] If $p \equiv 1 \lmod{4}$, we consider the finite field of $p$ elements $\Z/p\Z = \bra{0, 1,\dots , p - 1}$ (see Example \ref{exa:zpz_field}). Then by Proposition \ref{thm:multiplicative_group_pz_cyclic}, the field $\Z/p\Z$ with $p$ elements has a cyclic multiplicative group $\bb{\Z/p\Z}^{\times} = \bb{\Z/p\Z}\bs\bra{0}$ and thus the order of $\bb{\Z/p\Z}^{\times}$ can be divided by 4. A cyclic group $\bsa{g}$ of order $4n$ has a subgroup of order 4, namely $\bsa{ g^n }$, and a unique element of order 2, namely $g^{2n}$. $p - 1$ is the element of order 2 in $\bb{\Z/p\Z} \bs\bra{0}$, the multiplicative group since %$\F_p^* := \F_p\bs\bra{0}$
%field $\F_p$ with $p$ elements has a cyclic multiplicative group $\F_p^* := \F_p\bs\bra{0}$
\be
(p-1)^2 = p^2 - 2p + 1 \equiv 1 \lmod{p}
\ee

An element of order 4 corresponds to $a \in \Z$ with $a^2$ is order 2. Thus, $a^2 \equiv -1 \lmod{p}$. Thus $p \mid (a^2 + 1) = (a + i)(a - i)$. % where $a\pm i$ are irreducibles.
But $p \nmid a \pm  i$ (since $p$ is a prime number) and so $p$ is not an irreducible (and prime) in $\Z[i]$ (If $p$ is irreducible, it must divides either $a+ i$ or $a-i$ since it is an associate of the irreducible divides $a\pm i$).

Thus $p$ must factorise, $p = z_1z_2$ with $z_i$ non-units in $\Z[i]$. We have $N(z_i) = p$. Writing $z_i = x \pm  iy$, we have $p = x^2 + y^2$ (Lemma \ref{lem:square_modulo}). We also have that $z_i$ are irreducible since if $z_i = y_1y_2$, $p = N(z_i) = N(y_1)N(y_2)$ which will give that one of $y_i$ is a unit ($N(y_i) = 1$).
\een
\een

Now for any irreducible $\alpha \in\Z[i]$, $\bar{\alpha}$ is irreducible (since the complex conjugate of a unit is still a unit). Since $\alpha$ is a irreducible (not unit), we can take prime number $p$ such that $p \mid N(\alpha)$. We use that $N(\alpha) = \alpha\bar{\alpha}$, a product of irreducibles.

If $p \equiv 3 \lmod{4}$ then $p$ is irreducible (and prime) in $\Z[i]$. So the fact that $\Z[i]$ is a UFD and $p \mid \alpha\bar{\alpha}$ implies that $p$ is an associate of either $\alpha$ or $\bar{\alpha}$. So $p$ is an associate of $\alpha$ (since the complex conjugate of a unit is still a unit).

If $p = 2$, or $p \equiv 1 \lmod{4}$ then $p = z\bar{z} | \alpha\bar{\alpha}$. If there exists non-unit $b\in Z[i]$ such that $z\bar{z}b\bar{b} = \alpha \bar{\alpha}$ (since $z\bar{z}, \alpha\bar{\alpha} \in \Z^+$ ), then we $\alpha$ can be expressed by the product of two non-units. Contradiction.

Thus, we have $z\bar{z} = \alpha\bar{\alpha}$. Since $z$ and $\bar{z}$ are irreducibles and $\Z[i]$ is a UFD, we have that $z$ is an associate of $\alpha$ or $\bar{\alpha}$ and so $\alpha$ is an associate of $z$ or $\bar{z}$. %Thus our list is complete.
\end{proof}



\begin{corollary}\label{cor:prime_equals_square_sum}
A prime number $p\in \Z^+$ is a sum of squares ($p= x^2 +y^2$ for $x,y\in \Z\bs\bra{0}$) if and only if $p=2$ or $p \equiv 1\lmod{4}$.
\end{corollary}

\begin{proof}[\bf Proof]
($\ra$). If $p \equiv 3\lmod{4}$, then by Proposition \ref{pro:irreducible_associate_prime_p} we have it is irreducible in $\Z[i]$ and thus it is not of form $x^2+y^2$ by Lemma \ref{lem:square_modulo}.

($\la$). If $p$ is not of form $x^2+y^2$ then it is irreducible in $\Z[i]$ by Lemma \ref{lem:square_modulo}. Thus, it associates $\pm p$ or $\pm pi$ since the units in $\Z[i]$ are $\pm 1$ and $\pm i$. By Proposition \ref{pro:irreducible_associate_prime_p}, if it assocites the form $z$ with $z\ol{z} = p'$ for prime $p'\in \Z$ with $p'=2$ or $p\equiv 1\lmod{4}$. We can have the contradiction that $p^2 = z\ol{z} = p'$. Thus, the only possibility is $p\equiv 3\lmod{4}$ by Proposition \ref{pro:irreducible_associate_prime_p} again.
\end{proof}

\begin{example}
For prime number $p=13$ wtih $13\equiv 1\lmod{4}$, we have $13 = 2^2 + 3^2$.

For prime number $p=23$ with $23 \equiv 3 \lmod{4}$, we can not find such sum of squares.

For prime number $p = 41$ with $41\equiv 1\lmod{4}$, we have $41 = 4^2 + 5^2$.

Note that the corollary is not true for composite number as we cannot find such sum of square for $9\equiv 1\lmod{4}$.
\end{example}

\begin{corollary}\label{cor:square_sum_iff_prime_3_mod_4_even_power}
Let $n = p^{n_1}_1 \dots p^{n_k}_k \in \Z$ be the prime factorisation of $n$, with $p_1,\dots , p_k$ distinct primes in $\Z$. Then $n$ is of the form $x^2 + y^2$ if and only if whenever $p_i \equiv 3 \lmod{4}$ then $n_i$ is even.
\end{corollary}

\begin{proof}[\bf Proof]
($\ra$). Suppose $n = x^2 + y^2 = (x + iy)(x - iy) = z\bar{z}$ with $z = x + iy$. Thus $N(z) = n$.

Express $z$ as a product of irreducibles in $\Z[i]$, $z = \alpha_1 \dots \alpha_s$, say. But we know what the irreducibles are from Proposition \ref{pro:irreducible_associate_prime_p}. Either $N(\alpha_j) = p^2_j$ with $p_j \equiv 3 \lmod{4}$ or $N(\alpha_j) = p_j$ with $p_j = 2$ or $p_j \equiv 1 \lmod {4}$. Thus $n = N(z) = \prod N(\alpha_j)$ is of the required form.

($\la$). Conversely, if $n = p^{n_1}_1 \dots p^{n_k}_k$ with $n_j$ even for $p_j \equiv 3 \lmod {4}$ then we can replace any $p_j = 2$ or $p_j \equiv 1 \lmod {4}$ by $p_j = \alpha_j \bar{\alpha}_j$ and any other primes $p_j = \alpha_j = \bar{\alpha}_j$ and so $p^2_j = \alpha_j \bar{\alpha}_j$ for some irreducible $\alpha_j$ (Proposition \ref{pro:irreducible_associate_prime_p}). Thus $n$ is of the form $z\bar{z}$ for some $z = x + iy$ and so $n = x^2 + y^2$.
\end{proof}

\begin{example}
Let $n = 65 = 5 \times 13$ and note that $5 = (2 + i)(2 - i)$, $13 = (2 + 3i)(2 - 3i)$. The unique factorisation up to reordering and associates is
$n = (2 + i)(2 - i)(2 + 3i)(2 - 3i)$.

We use this to express $n$ as $x^2 + y^2 = z\bar{z}$ with $z = x + iy$
\be
n = [(2 + i)(2 + 3i)][(2 - i)(2 - 3i)] = (1 + 8i)(1 - 8i) = 1^2 + 8^2
\ee
or
\be
n = [(2 + i)(2 - 3i)][(2 - i)(2 + 3i)] = (7 - 4i)(7 + 4i) = 4^2 + 7^2.
\ee
\end{example}


\subsection{$\Z[\alpha]$ with $\alpha$ an algebraic integer}

\begin{definition}[algebraic integer\index{algebraic integer}]\label{def:algebraic_integer}
A complex number $\alpha \in \C$ is an algebraic integer if there is a monic polynomial $f(X) \in \Z[X]$ with $f(\alpha) = 0$.
\end{definition}

\begin{example}
The following are algebraic integers,%For example,
\begin{center}
\begin{tabular}{cc}
$\alpha \in \C$ & \quad Minimal polynomial of $\alpha$ \quad\\
\hline
$i$ & $f(X) = X^2 + 1$\\
$\sqrt{2}$ &  $f(X) = X^2 - 2$\\
$\frac 12(1 +\sqrt{-3})$ & $f(X) = X^2 - X + 1$
\end{tabular}
\end{center}
\end{example}

\begin{definition}\label{def:algebraic_integer_ring}
$\Z[\alpha]$ is the smallest subring of $\C$ containing $\Z$ and $\alpha$. Then,
\be
\Z[\alpha] \cong \Z[X]/I
\ee
where $I$ is the kernel of the map $\theta : \Z[X] \to \Z[\alpha]$, $f(X) \mapsto  f(\alpha)$ (by the first isomorphism theorem (Theorem \ref{thm:isomorphism_1_ring})).% since $\theta$ is surjective by Definition \ref{def:algebraic_integer}).
\end{definition}


\begin{proposition}\label{pro:minimal_polynomial_of_algebraic_integer}
For an algebraic integer $\alpha$, with the notation in Definition \ref{def:algebraic_integer_ring}, the kernel $I$ is a principal ideal generated by a monic irreducible polynomial $f_\alpha(X) \in \Z[X]$, called the minimal polynomial\index{minimal polynomial!algebraic integer} of $\alpha$. That is, $\Z[\alpha] \cong \Z[X]/\bsa{f_\alpha(X)}$.
\end{proposition}

\begin{proof}[\bf Proof]
By Definition \ref{def:algebraic_integer}, $f(\alpha) = 0$ for some monic $f(X) \in \Z[X]$. So we can pick $f_\alpha(X)\in I$ of minimal degree $\geq 0$, and we may assume $f_\alpha(X)$ is primitive in $\Z[X]$. The claim is that this is the required polynomial.

We wish to show that $I = \bsa{f_\alpha(X)}$.

Let $h(X) \in I$. Then, as $\Q$ is a field, we can apply Euclidean algorithm for polynomials (Proposition \ref{pro:euclidean_algorithm_polynomial}) and find $q(X),r(X)\in \Q[X]$ such that% is a Euclidean domain, we can write
\be
h(X) = q(X)f_\alpha(X) + r(X)
\ee
with $r(X) = 0$ or $\deg r(X) < \deg f_\alpha(X)$. Clearing denominators, there exists $a \in \Z\bs\{0\}$ such that
\be
ah(X) = aq(X)f_\alpha(X) + ar(X)
\ee
with $aq(X) \in \Z[X]$ and $ar(X) \in \Z[X]$. But $ar(\alpha) = 0$ so $ar(X) \in I$. Minimality of degree of $f_\alpha(X)$ implies that $ar(X) = 0$ so $r(X) = 0$. So $ah(X) = aq(X)f_\alpha(X)$. Consider the contents of both sides, since $h(X)\in \Z[X]$
\be
a \mid c(ah(X)) = c(aq(X)f_\alpha(X)) = c(aq(X)) u \quad (u \text{ is a unit})
\ee
by the fact that $f_\alpha(X)$ is primitive (Corollary \ref{cor:content_associate}). So $a$ divides all the coefficients of $aq(X)$. So $q(X)\in \Z[X]$. Thus $h(X) \in \bsa{f_\alpha(X)} \lhd \Z[X]$.

We also claim that $f_\alpha(X)$ is prime (and irreducible since $\Z[X]$ is a UFD).

If $f_\alpha(X) = f_1(X)f_2(X)$ then $0 = f_1(\alpha)f_2(\alpha)$ and so $f_i(X) \in I = \bsa{f_\alpha(X)}$ for some $i$. $f_\alpha(X) | f_i(X)$ for some $i$. i.e. $f_\alpha (X) g(X) = f_i(X)$, $g(X) \in \Z[X]$. Thus, $f_\alpha (X)$ is prime.

Finally, we show that $f_\alpha(X)$ is monic.

As in the first line of proof, there are some monic $f(X)\in I = \bsa{f_\alpha(X)}$. Thus, $f_\alpha(X) g(X) = f(X)$ where $g(X) \in \Z[X]$. Checking the leading coefficient of polynomial, we have $a_\alpha a_g = 1$ ($f_\alpha (X) = a_\alpha X^n + \dots$, $g(X) = a_g X^m + \dots $) and then $a_\alpha = \pm 1,\pm i$. But these are all multiplication of 1 and a unit. Thus, $f_\alpha (X)$ is monic.
\end{proof}

\begin{remark}
minimal polynomial of algebraic integer $f_\alpha(X)$ is unique\footnote{need proof}.
\end{remark}

\begin{definition}[rational integer\index{rational integer}]
The elements of $\Z$ are called rational integers.
\end{definition}

\begin{lemma}
If $\alpha$ is an algebraic integer and $\alpha \in \Q$ then $\alpha \in \Z$.
\end{lemma}

\begin{proof}[\bf Proof]
$f_\alpha(X)$ is irreducible in $\Z[X]$ and monic (thus primitive) (by Proposition \ref{pro:minimal_polynomial_of_algebraic_integer}). By Gauss' lemma, Lemma \ref{lem:gauss_polynomial_irreducible} ($\Z$ is a UFD and $\Q$ is the field of fractions of $\Z$), $f_\alpha(X)$ is irreducible in $\Q[X]$.

Since $\alpha$ is a root of $f_\alpha(X)$ in $\Q$, we have $f_\alpha(X) = X - \alpha$ by definition of minimal polynomial of algebraic integer. Also since $f_\alpha \in \Z[X]$, we have $\alpha \in \Z$.
\end{proof}


\begin{example}
Let $p \in \Z$ be prime and consider $(\Z/p\Z)[X]/\bsa{ \bar{f}_\alpha(X)}$, where $\bar{f}_\alpha(X)$ is the polynomial in $(\Z/p\Z)[X]$ obtained from $f_\alpha(X)$ by taking coefficients modulo $p$.%$\Z[X]/\bsa{p, f_\alpha(X)}$ with $\F_p \cong \Z/p\Z$.
\beast
(\Z/p\Z)[X]/\bsa{ \bar{f}_\alpha(X)} & \cong & \Z[X]/\bsa{p, f_\alpha(X)} \quad (\text{the third isomorphism theorem, Theorem \ref{thm:isomorphism_3_ring}})\\
& \cong & \Z[X]/\bsa{f_\alpha(X)}/\bb{\bsa{p, f_\alpha(X)}/\bsa{f_\alpha(X)}} \quad (\text{the third isomorphism theorem, Theorem \ref{thm:isomorphism_3_ring}})\\
& \cong & \Z[\alpha]/\bsa{p} \quad (\text{by Proposition \ref{pro:minimal_polynomial_of_algebraic_integer} and the fact that }\bsa{p, f_\alpha(X)}/\bsa{f_\alpha(X)} = \bsa{p})\\%the first isomorphism theorem, Theorem \ref{thm:isomorphism_1_ring}})\\
& \cong & (\Z/p\Z)[\alpha]. \quad (\text{the third isomorphism theorem, Theorem \ref{thm:isomorphism_3_ring}})
\eeast

For example, $\alpha = i$, $f_\alpha(X) = X^2 + 1$. Then since $(X^2 + 1)\lmod{p} = X^2+1$,
\be
(\Z/p\Z)[X]/\bsa{X^2 + 1} \cong \Z[X]/\bsa{p,X^2 +1}  \cong  \Z[i]/\bsa{p} \cong  (\Z/p\Z)[i].% \cong \F_p[i].
\ee

For $p = 2$ %, we assume $p = 4k+2$, we can pick $2$ and $2k+1$ such that $2(2k+1)\equiv 0 \lmod{p}$. Thus, $\Z/p\Z[i]$ is not an integral domain.
or $p \equiv 1 \lmod {4}$, this is not an integral domain\footnote{need details for $p \equiv 1 \lmod {4}$}.% we assume $p = 4k+1$,\footnote{detail needed.} %we can pick $4k+i$ and $1+4ki$ such that
%\be
%(4k-i)(1+4ki) = 4k + 4k + (16k^2 - 1)i = -2
%\ee
%

For $p \equiv 3 \lmod {4}$ it is an integral domain\footnote{need details for $p \equiv 3 \lmod {4}$}.
\end{example}

%\begin{remark}
%There is quite a lot on algebraic integers in the Part II course Number Fields. For example, quadratic fields which are of the form $\Q(\sqrt{d}) = \{a + b\sqrt{d} : a, b \in \Q\} \leq \C$. In $\Q(\sqrt{d})$ the algebraic integers form a ring $R$. However, $(1 + \sqrt{-3})/2$ is an algebraic integer so $R$ is not necessarily $\Z(\sqrt{d})$.

%$R$ is Euclidean if and only if $d$ is -11, -7, -3, -2, -1, 2, 3, 5, 6, 7, 11, 13, 17, 19, 21, 29, 33, 37, 41, 57 or 73 (21 possibilities).

%$R$ is a UFD for $d < 0$ if and only if $d$ is -1, -2, -3, -7, -11, -19, -43, -67, -163 (9 possibilities, cf. H.M. Stark \textit{An introduction to number theory}). For $d > 0$ there exist 38 possibilities with $2 \leq d < 100$ so that $R$ is a UFD. This question is still open.
%\end{remark}



%%%%%%%%%%%%%%%%%%%%%%%%%%%%%%%%%%%%%%%%%%%%%%%%%%%%%%%%%%%%%%%%%%%%%%%%%%%%%%%%%%



\section{Summary}

Recalling the previous results, we have that

Commutative rings $\supset$ integral domains $\supset$ integrally closed domains\footnote{need details} $\supset$ unique factorization domains $\supset$ principal ideal domains $\supset$ Euclidean
domains $\supset$ fields.

\begin{center}
\begin{tabular}{ccccccc}
\hline
 & \ \ F\ \ \  & ED & PID & UFD & ID & CR\\\hline %Euclidean domain & principal integral domain & unique factorisation domain & integral domain \\ \hline
$\Z$ & $\times$ & $\surd$ & $\surd$ & $\surd$ & $\surd$ & $\surd$\\
$\Z[i]$ & $\times$ & $\surd$ & $\surd$ & $\surd$ & $\surd$ & $\surd$\\
$\Z[\sqrt{-2}]$ & $\times$ & $\surd$ & $\surd$ & $\surd$ & $\surd$ & $\surd$\\
$\Z[\omega]$ & $\times$ & $\surd$ & $\surd$ & $\surd$ & $\surd$ & $\surd$\\
$\Z[\sqrt{-3}]$ & $\times$ & $\times$ & $\times$ & $\times$ & $\surd$ & $\surd$ \\
$\Z[\sqrt{-5}]$ & $\times$ & $\times$ & $\times$ & $\times$ & $\surd$ & $\surd$  \\
$\Z/p\Z$ & $\surd $& $\surd$ & $\surd$ & $\surd$ & $\surd$ & $\surd$\\
$\Z/6\Z$ & $\times$ & $\times$ & $\times$ & $\times$ & $\times$ & $\surd$ \\
$\Q$ & $\surd$ & $\surd$ & $\surd$ & $\surd$ & $\surd$ & $\surd$\\
$\R$ & $\surd$ & $\surd$ & $\surd$ & $\surd$ & $\surd$ & $\surd$\\
$\C$ & $\surd$ & $\surd$ & $\surd$ & $\surd$ & $\surd$ & $\surd$\\
$\Z[X]$ & $\times$ & $\times$ & $\times$ & $\surd$ & $\surd$ & $\surd$\\
$\Q[X]$ & $\times$ & $\surd$ & $\surd$ & $\surd$ & $\surd$ & $\surd$ \\
$\F[X]$ & $\times$ & $\surd$ & $\surd$ & $\surd$ & $\surd$ & $\surd$ \\
\hline
\end{tabular}
\end{center}

where $\omega = (1+\sqrt{-3})/2$, $\F$ is a field and $p$ is a prime number.

A commutative ring $R$ is a field if and only if $R[X]$ is a PID (Proposition \ref{pro:polynomial_pid_iff_field}).

For commutative ring $R$, $R/I$ is a field if and only if $I$ is a maximal ideal, $R/I$ is an integral domain if and only if $I$ is a prime ideal.

A maximal ideal is a prime ideal. In PID, maximal ideal and prime ideal are equivalent.

If $R$ be a ID, then an element $p \in R$ is irreducible if it is prime.

If $R$ is a UFD, then an element $p\in R$ is irreducible if and only if it is prime.


\begin{center}
\begin{longtable}{|ccccccc|ccccccc|}%ccccccc|ccccccc|}%\begin{tabular}
\caption{group structure of $\bb{\Z/n\Z}^\times$} \label{tab:group_structure_multiplicative_group_modular_ring}\\ %\vspace{-2mm}
\hline
& $n$ & & $\bb{\Z/n\Z}^\times$ & & $\phi(n)$ & & & $n$ & & $\bb{\Z/n\Z}^\times$ & & $\phi(n)$ & \\ %& & $n$ & & $\bb{\Z/n\Z}^\times$ & & $\phi(n)$ & & & $n$ & & $\bb{\Z/n\Z}^\times$ & & $\phi(n)$ &  \\
\hline
& 1 & & $C_1$ & & 1 & & & 2 & & $C_1$ & & 1 & \\%&
& 3 & & $C_2$ & & 2 & & & 4 & & $C_2$ & & 2 &  \\
& 5 & & $C_4$ & & 4 & & & 6 & & $C_2$ & & 2 & \\%&
& 7 & & $C_6$ & & 6 & & & 8 & & $C_2\times C_2$ & & 4 &  \\
& 9 & & $C_6$ & & 6 & & & 10 & & $C_4$ & & 4 & \\%&
& 11 & & $C_{10}$ & & 10 & & & 12 & & $C_2\times C_2$ & & 4 &  \\
& 13 & & $C_{12}$ & & 12 & & & 14 & & $C_6$ & & 6 & \\%&
& 15 & & $C_2\times C_4$ & & 8 & & & 16 & & $C_2\times C_4$ & & 8 &  \\
& 17 & & $C_{16}$ & & 16 & & & 18 & & $C_6$ & & 6 & \\%&
& 19 & & $C_{18}$ & & 18 & & & 20 & & $C_2\times C_4$ & & 8 &  \\
& 21 & & $C_{2}\times C_6$ & & 12 & & & 22 & & $C_{10}$ & & 10 & \\%&
& 23 & & $C_{22}$ & & 22 & & & 24 & & $C_2\times C_2\times C_2$ & & 8 &  \\
& 25 & & $C_{20}$ & & 20 & & & 26 & & $C_{12}$ & & 12 & \\%&
& 27 & & $C_{18}$ & & 18 & & & 28 & & $C_2\times C_6$ & & 12 &  \\
& 29 & & $C_{28}$ & & 28 & & & 30 & & $C_2\times C_4$ & & 8 & \\% &
& 31 & & $C_{30}$ & & 30 & & & 32 & & $C_2\times C_8$ & & 16 &  \\
\hline%\end{tabular}%
\end{longtable}
\end{center}


\section{Problems}

\begin{problem}
Let $\omega = \frac 12 (1 + \sqrt{-3})$, let $R = \{a + b\omega : a, b \in \Z\}$, and let $F = \{a + b\omega : a, b \in \Q\}$. Show that $R$ is a subring of $\C$, and that $F$ is a subfield of $\C$. What are the units of $R$?
\end{problem}

\begin{solution}[\bf Solution.]
Let $a_1 + b_1\omega,a_2 + b_2\omega \in R$. Thus, $a_1 + b_1\omega + a_2 + b_2\omega \in R$ since $a_1 + a_2 \in \Z$ and $b_1 + b_2\in \Z$.

The identity of addition is $0 = 0 + 0\omega \in R$, thus the inverse of $a+b\omega$ is $-a-b\omega \in R$.

Also, $(a_1 + b_1\omega)(a_2 + b_2\omega) = a_1a_2 + (a_1b_2 + a_2b_1)\omega + b_1b_2 \omega^2 \in R$. We know that
\be
\omega^2 = \bb{\frac{1+\sqrt{-3}}2}^2 = \frac{1+(-3) + 2\sqrt{-3}}4 = \frac {-1+\sqrt{-3}}2 = \omega -1.
\ee

Thus, $(a_1 + b_1\omega)(a_2 + b_2\omega) = a_1a_2 - b_1b_2 + (a_1b_2 + a_2b_1 + b_1b_2)\omega \in R$.

The multiplicative identity is $1 = 1 + 0\omega \in R$. Also, it can be check that the multiplication is distributive over addition.% $a_1 + b_1\omega,a_2 + b_2\omega, a_3 + b_3\omega \in R$,
%\be
%(a_1 + b_1\omega)\bb{a_2 + b_2\omega + a_3 + b_3\omega} = (a_2 + b_2\omega)\bb{a_2 + a_3+ (b_2 + b_3)\omega} =
%\ee

Thus, $R$ is subring of $\C$.

For $F = \{a + b\omega : a, b \in \Q\}$, we only check the multiplication inverse, let $a+b\omega \in F$
\be
(a+b\omega) (c + d\omega ) = 1 \ \ra \ ac - bd + (ad +bc + bd)\omega = 1 \ \ra \ \left\{\ba{ll} ac - bd = 1\\ ad +bc + bd = 0 \ea\right.
\ee

Thus if $a = 0$, $b\neq 0$, $d = -\frac 1b$, $c = \frac 1b$, the inverse $a+b\omega = b\omega$ is $\frac 1b - \frac 1b \omega \in F$.

IF $a\neq 0$, $c = \frac{1+bd}a$,
\be
ad + b  \frac{1+bd}a + bd = 0  \ \ra \ \bb{a+b + \frac {b^2}a }d = - \frac ba
\ee

If $b = 0$, $d = 0$, $c = \frac 1a$, the inverse $a+b\omega = a$ is $\frac 1c = \frac 1c + 0\omega \in F$.

If $b\neq 0$, $d = \frac b{a^2 + ab + b^2}$, $c = \frac {a^2 + ab + 2b^2}{a(a^2 + ab + b^2)}$, so the inverse $a+b\omega $ is
\be
\frac {a^2 + ab + 2b^2}{a(a^2 + ab + b^2)} + \frac b{a^2 + ab + b^2}\omega \in F.
\ee

Thus, $F$ is subfield of $\C$.

We can define
\be
N\bb{a+b\omega} = N\bb{a+ \frac b2 + \frac{\sqrt{-3}}2b} = (a+\frac b2)^2 + \frac 34 b^2 = a^2 + b^2 + ab \geq 0.
\ee

Given $r_1 = a_1 + b_1\omega \in R$, if it is a unit, there must exist $r_2 \in R$ (its multiplication inverse) such that $r_1 r_2 = 1$. We have $N(r_1r_2) = N(r_1)N(r_2) = 1$. Since $N(r)$ is of form $a^2 + b^2 + ab \in \Z$, $N(r_1) = N(r_2) = 1$. Thus,
\be
1 = N(r_1) = a_1^2 + b_1^2 + a_1b_1 \ \ra \ a_1^2 + a_1 b_1 + b_1^2 -1 = 0 \ \ra \ b_1^2 - 4(b_1^2 -1) \geq 0 \ \ra \ b_1^2 \leq \frac 43.
\ee

If $b = 0$, $a = \pm 1$. If $b= -1$, $a = 0, 1$. If $b=1$, $a = 0,-1$. Thus the units are $\pm 1$, $\pm \omega$, $\pm \omega^2$ ($1-\omega$, $-1 + \omega$).

\end{solution}

\begin{problem}
Show that the ideal $\bsa{2, 1 + \sqrt{-7}}$ in $\Z[\sqrt{-7}]$ is not principal.
\end{problem}

\begin{solution}[\bf Solution.]
If $\bsa{2, 1 + \sqrt{-7}}$ is principal, there must exists $a\in \Z[\sqrt{-7}]$ such that $\bsa{2, 1 + \sqrt{-7}}= \bsa{a}$ and $a\mid 2$ and $a\mid 1 + \sqrt{-7}$. That is, $2 = a b_1$ and $1+ \sqrt{-7} = ab_2$, $b_1,b_2 \in \Z[\sqrt{-7}]$.

Thus, the norm is of the form $c^2 + 7d^2$,
\beast
4 & = & N(2) = N(a)N(b_1) \ \ra \ N(a) = 1,4\\
8 & = & N(1+\sqrt{-7}) = N(a)N(b_2)  \ \ra \ N(a) = 1,8
\eeast

Thus, $N(a) = 1$ and $a$ is unit. Hence $\bsa{a} = \Z[\sqrt{-7}]$. However, $1 \in \Z[\sqrt{-7}]$ and $1 \notin \bsa{2,1+\sqrt{-7}}$. Thus, $\bsa{2,1+\sqrt{-7}}$ is not principal.%But we know that $2$ and $1+\sqrt{-7}$ are irreducible in $\Z[\sqrt{-7}]$, so $a$ is a unit.
\end{solution}


\begin{problem}
Give an element of $\Z[\sqrt{-17}]$ that is a product of two irreducibles and also a product of three irreducibles.
\end{problem}

\begin{solution}[\bf Solution.]
$18 = (1+\sqrt{-17})(1-\sqrt{-17}) = 2 \cdot 3 \cdot 3$.

Now we show that they are all irreducible.

Express 2 with $2= z_1z_2$, $z_1,z_2\in \Z[\sqrt{-17}]$. Thus $4 = N(z_1)N(z_2)$. Since the norm is the form of $a^2 + 17b^2$ and there is no element of norm 2. Thus, one of the $N(z_j) = 1$, so $z_j$ is a unit. Thus, $2$ is irreducible.

Express 3 with $3= z_1z_2$, $z_1,z_2\in \Z[\sqrt{-17}]$. Thus $9 = N(z_1)N(z_2)$. Since the norm is the form of $a^2 + 17b^2$ and there is no element of norm 3. Thus, one of the $N(z_j) = 1$, so $z_j$ is a unit. Thus, $3$ is irreducible.

Express $1 \pm \sqrt{-17}$ with $1 \pm \sqrt{-17} = z_1z_2$, $z_1,z_2\in \Z[\sqrt{-17}]$. Thus $18 = N(z_1)N(z_2)$. Since the norm is the form of $a^2 + 17b^2$ and there is no element of norm 2,3,6. Thus, one of the $N(z_j) = 1$ , so $z_j$ is a unit. Thus, $1 \pm \sqrt{-17}$ is irreducible.
\end{solution}


\begin{problem}\label{que:idempotent}
Recall Definition \ref{def:idompotent_ring},
\ben
\item [(i)] What are the idempotent elements of $\Z/6\Z$? Of $\Z/8\Z$? Of $\Z/24\Z$? Of $\Z/1000\Z$?
\item [(ii)] Show that if $r$ is idempotent then so is $r' = 1 - r$, and $rr' = 0$.
\item [(iii)] Show also that the ideal $\bsa{r}$ is naturally a ring, and that $R$ is isomorphic to $\bsa{r} \times \bsa{r'}$.
\een
\end{problem}

\begin{solution}[\bf Solution.]
\ben
\item [(i)] 0,1,3; 0,1; 0,1,9,16; For $\Z/1000\Z$, we see that
\be
r = \left\{ \ba{l}
10x \\
10x + 1\\
10x + 5\\
10x + 6
\ea\right.
\ee

If $r = 10x$, for some $a$,
\be
100x^2 = 10 x + 1000a \ \ra \ 10x(10x -1) = 1000a \ \ra \ x = 100y \ \ra \ x = 0 \ \ra \ r = 0.
\ee

If $r = 10x + 1$, for some $a$,
\be
100x^2 + 20x + 1 = 10x + 1 + 1000a \ \ra \ 10x(10x + 1) = 1000a \ \ra \ x = 100y \ \ra \ x = 0 \ \ra \ r = 1.
\ee

If $r = 10x + 5$, for some $a$,
\be
100x^2 + 100x + 25 = 10x + 5 + 1000a \ \ra \ x(10x + 9) + 2 = 100a \ \ra \ x = 10y + 2.
\ee

Thus, $r = 100y + 25$, for some $a$,
\be
10000y^2 + 5000y + 625 = 100y + 25 + 1000a \ \ra \ y (100y + 49) + 6 = 10a \ \ra \ y = 10z + 6 \ \ra \ r = 625.
\ee

If $r = 10x + 6$, for some $a$,
\be
100x^2 + 120x + 36 = 10x + 6 + 1000a \ \ra \ x(10x + 11) + 3 = 100a \ \ra \ x = 10y + 7.
\ee

Thus, $r = 100y + 76$, for some $a$,
\be
10000y^2 + 15200y + 5776 = 100y + 76 + 1000a \ \ra \ y (100y + 151) + 57 = 10a \ \ra \ y = 10z + 3 \ \ra \ r = 376.
\ee

Thus, the idempotents are 0,1,376,625.

\item [(ii)] $r^2 = r$, $r'^2 = (1-r)^2 = r^2 - 2r + 1 = r - 2r + 1 = 1-r = r'$ and $rr' = r(1-r) = r - r^2 = 0$.
\item [(iii)] Clearly, $\bsa{r}$ is group under addition. For the multiplication, $\forall rg_1,rg_2 \in \bsa{r}$ for some $g_1,g_2 \in R$,
\be
rg_1 rg_2 = r^2 g_1g_2 = rg_1g_2 \in \bsa{r}.
\ee

Now we need to show that $1\in \bsa{r}$. Claim $r$ is the identity in $\bsa{r}$. It is easy to check that $\forall rg \in \bsa{r}$ for some $g\in R$
\be
r (rg) = r^2 g = rg.
\ee

Thus, $\bsa{r}$ is a ring.

Now let $\phi : R\to \bsa{r}\times \bsa{r'}$ by $\phi(g) = \bb{rg,r'g} = \bb{rg,(1-r)g}$. Then for any $g_1,g_2\in R$,
\be
\phi(g_1 + g_2) = (r(g_1+g_2), (1-r)(g_1 + g_2)) = \bb{rg_1,(1-r)g_1} + \bb{rg_2,(1-r)g_2} = \phi(g_1) + \phi(g_2),
\ee
\be
\phi(g_1g_2) = \bb{rg_1g_2,(1-r)g_1g_2} = \bb{r^2 g_1g_2, (1-r)^2 g_1g_2} = (rg_1,(1-r)g_1)(rg_2,(1-r)g_2) = \phi(g_1)\phi(g_2).
\ee

So $\phi$ is a homomorphism. Then
\be
\ker\phi = \bra{g\in R: rg = 0, (1-r)g = 0} \subseteq \bra{g\in R: rg + (1-r)g = 0} = \bra{0} \ \ra \ \ker\phi = \bra{0}.
\ee

If $\phi(g_1) = \phi(g_2)$,
\be
rg_1 = rg_2,\ (1-r)g_1 = (1-r)g_2 \ \ra \ g_1 = g_2 \ \ra \ \phi \text{ is injective.}
\ee

To show $\phi$ is surjective, $\forall a = rg_1 \in \bsa{r}, b = (1-r)g_2 \in \bsa{r'}$, we need to find $g\in R$ s.t. $\phi(g) = (a,b)$. So
\be
rg = rg_1,\ (1-r)g = (1-r)g_2 \ \ra \ g = rg_1 + (1-r)g_2.
\ee
by the fact $r\cdot r' = 0$. So $\phi$ is surjective (i.e., $\im \phi  = \bsa{r}\times \bsa{r'}$) and then $\phi$ is bijective. (We can also use first isomorphism theorem to show that
\be
R/\ker\phi \cong \im \phi \ \ra \ R = R/\ker\phi  \cong \bsa{r}\times \bsa{r'}.
\ee

Thus, $\phi$ is bijective.)
\een

\end{solution}

\begin{problem}
\ben
\item [(i)] Show that the set $P(S)$ of all subsets of a given set $S$ is a ring with respect to the operations of symmetric difference and intersection (the power-set ring). Note that in this ring $A^2 = A$ for all elements $A$. Describe the principal ideals in this ring. %Describe the ideal $(A,B)$ generated by elements $A,B$.
Show that the ideal $(A,B)$ generated by elements $A$,$B$ is in fact principal. Are there any non-principal ideals?

\item [(ii)] The ring $R$ is Boolean if every element of $R$ is idempotent. Prove that every finite Boolean ring is isomorphic to a power-set ring $P(S)$ for some set $S$. Give an example to show that this need not remain true for infinite Boolean rings.
\een
\end{problem}

\begin{solution}[\bf Solution.]
\ben
\item [(i)] Let $A + B = A\triangle B$ and $AB = A\cap B$. We use indicator function, for symmetric difference (addition),
\be
\ind_{A\triangle B} = \ind_A + \ind_B\lmod{2}.
\ee

Clearly, for $A,B \in P(S)$, $A\triangle B \in P(S)$, $A\triangle B = B \triangle A$, the identity is $\emptyset$ and the inverse of $A$ is $A^c = S\bs A$.

For intersection (multiplication), $A\cap B = B\cap A$ and $A\cap B \in P(S)$. The identity of multiplication is $S$, since $S\cap A = A$.

For distributivity, $\forall A,B,C \in S$,
\beast
\ind_{A\cap (B\triangle C)} & = &  \ind_A \ind_{B\triangle C} = \ind_A \bb{\ind_B + \ind_C \lmod{2}} \\
& = & \ind_A \ind_B \lmod{2} + \ind_A \ind_C \lmod{2} = \ind_{A\cap B} + \ind_{A\cap C} \lmod{2} = \ind_{(A\cap B) \triangle (A\cap C)}.
\eeast

%\beast
%A \cap (B\triangle C) & = & A \cap ((B\cup C) \bs (B \cap C)) = A \cap \bb{(B\cup C) \cap (B \cap C)^c} \\
%& = & (A\cap (B\cup C) ) \cap (A \cap (B \cap C)^c)
%\eeast

Thus, $P(S)$ is a ring. So $A^2 = A\cap A = A$ for all elements $A$. Thus, $\forall A\in P(S)$, the principal ideal is
\be
I = \bsa{A} = \bra{A\cap B:B\in P(S)} = \text{set of all subsets of $A$.}
\ee

Also, the ideal $\bsa{A,B}$ is
\beast
\bra{(A\cap C) \triangle (B\cap D): C,D\in P(S)} & = & \bra{(A\cap C) \triangle (B\cap D): C\subseteq A,D\subseteq B}\\
& = & \bra{C \triangle D: C\subseteq A,D\subseteq B}\\
& = &  \text{set of all subsets of $A\cup B$} = \bsa{A\cup B}.
\eeast

Thus, the ideal $\bsa{A,B}$ is also principal. So all the ideals are principal.

\item [(ii)] Let $(R,\oplus,\otimes)$ be Boolean ring with $r\otimes r = r$ for any $r\in R$. Then $r\oplus r \in R$ by definition of ring. Thus
\be
(r\oplus r) \otimes (r\oplus r) = r\oplus r \ \ra \ r^2 \oplus r^2 \oplus r^2 \oplus r^2 = r \oplus r \ \ra \ r\oplus r\oplus r\oplus r = r\oplus r \ \ra \ r\oplus r = 0
\ee
since $(R,\oplus)$ is a group. A similar proof shows that every Boolean ring is commutative: for any $x,y \in R$
\be
x \oplus y = (x \oplus y)^2 = x^2 \oplus xy \oplus yx \oplus y^2 = x \oplus xy \oplus yx \oplus y
\ee
and this yields $xy \oplus yx = 0$, which means $x\otimes y = y\otimes x$ (using the property above).

The property $x \oplus x = 0$ shows that $R$ has characteristic 2. Thus, $\abs{R}$ (cardinality of $R$) must be $2^n$ for some $n$.
%any Boolean ring is an associative algebra over the field $\F_2$ with two elements

Claim $R \cong \underbrace{\F_2 \times \dots \times \F_2}_{n}$.

If $n=1$. Clearly, $R= \bra{0,1} \cong \F_2$. Suppose it is true for $n-1$. By Problem \ref{que:idempotent}, we have $R \cong \bsa{r}\times \bsa{r'}$. $\bsa{r}, \bsa{r'}$ are both Boolean ring with order $<2^n$ since for any $rg \in \bsa{r}$ with $g\in R$,
\be
(rg)^2 = r^2 g^2 = r g,
\ee
and similar for $r'$. Then by induction, we have $R\cong \F_2 \times \dots \times \F_2$.

Let $P(S)$ be a power set of $n$ elements, so $\abs{P(S)} = 2^n$. Let $S = \bra{a_1,a_2,\dots,a_n}$ and
\be
\phi: \F_2 \times \dots \times \F_2 \to P(S),\quad \phi(0,\dots, 0,\underbrace{1}_{i},0,\dots,0) = A_i = \bra{a_i}.
\ee

$\phi$ can be extended to a homomorphism (inverse of indicator function) with $\ve \in \bra{0,1}^n$
\be
\phi(\ve_A + \ve_B) = A \triangle B = \phi(\ve_A) \triangle \phi(\ve_B), \quad \phi(\ve_A \ve_B) = A\cap B = \phi(\ve_A) \cap \phi(\ve_B).
\ee

Thus, $\ker\phi = \bra{0}^n$ by first isomorphism theorem, we have
\be
\F_2 \times \dots \times \F_2 = \F_2 \times \dots \times \F_2 / \bra{0}^n \cong \im \phi \leq P(S).
\ee

But $\abs{\F_2^n}\abs{\F_2\times \dots \times \F_2} = \abs{P(S)} = 2^n$, $\phi$ is surjective and thus $\im \phi = P(S)$, so
\be
\F_2^n \cong P(S) \ \ra \ R \cong P(S).
\ee

If $S$ is infinite, $P(S)$ is not countable, so there is no bijective $\phi: R\to P(S)$.
\een

\end{solution}


\begin{problem}
Find all ways of writing the following integers as sums of two squares: 221, $209\times 221$, $121\times 221$, $5\times 221$.
\end{problem}

\begin{solution}[\bf Solution.]
\ben
\item [(i)] $221 = 13 \times 17 = (3+2i)(3-2i)(4+i)(4-i)$ which is a product of irreducibles. We use this to express $n$ as $x^2 + y^2 = z\bar{z}$ with $z = x + iy$.
\beast
221 & = &  (10 + 11i)(10-11i) = 10^2 + 11^2,\\
& = & (14 + 5i)(14 - 5i) = 14^2 + 5^2.
\eeast

These are the only ways since we approach it with product of irreducibles.

\item [(ii)] $209\times 221 = 11\times 13 \times 17 \times 19$ with $11 \equiv 3 \lmod{4}$ and $19 \equiv 3 \lmod{4}$. Since the power of 11 and 19 are odd, $209\times 221$ is not of the form $x^2 + y^2$ with $x,y \in \Z$ by the corollary in notes.

\item [(iii)] $121\times 221 = 11^2 \times 13 \times 17 = 11^2 (3+2i)(3-2i)(4+i)(4-i)$ with $11 \equiv 3 \lmod{4}$. Since the power of 11 is even (2), it is of the form $x^2 + y^2$ with $x,y \in \Z$ by the corollary in notes. That is,
\beast
121 \times 221 & = &  (110 + 121i)(110-121i) = 110^2 + 121^2,\\
& = & (154 + 55i)(154 - 55i) = 154^2 + 55^2.
\eeast

\item [(iv)] $5\times 221 = 5 \times 13 \times 17 = (2+i)(2-i)(3+2i)(3-2i)(4+i)(4-i)$.
\beast
5 \times 221 & = &  (9 + 32i)(9-32i) = 9^2 + 32^2,\\
& = & (31 + 12i)(31 - 12i) = 31^2 + 12^2,\\
& = & (23 + 24i)(23 - 24i) = 23^2 + 24^2,\\
& = & (33 + 4i)(33 -4i) = 33^2 + 4^2.
\eeast

These are the only ways since we approach it with product of irreducibles.
\een
\end{solution}



\begin{problem}
By working in $\Z[\sqrt{-2}]$, show that the only integer solutions to $x^2 + 2 = y^3$ are $x = \pm 5$, $y = 3$.
\end{problem}

\begin{solution}[\bf Solution.]
First, the only two units in $\Z[\sqrt{-2}]$ is $\pm 1$. We have
\be
y^3 \equiv 0,1,3\lmod{4},\quad x^2 + 2 \equiv 2,3 \lmod{4} \ \ra \ x^2 + 2 \equiv 3 \lmod{4} \ \ra \ x^2 \equiv 1 \lmod{4} \ \ra \ x = 2z + 1.
\ee

We know that $x^2 + 2 = (x+\sqrt{-2})(x- \sqrt{-2})$. Now we want to show that the hcf of $x+\sqrt{-2}$ and $x- \sqrt{-2})$ is 1.

Then for $x = 0, \pm 1,\pm 2$, there is no solution for $y$, thus we can see that $x \geq 3$ and $N(x+\sqrt{-2}) = x^2 + 2 \geq 11$. Since $\Z[\sqrt{-2}]$ is a ED, we can use Euclidean algorithm to find the hcf,
\be
x+ \sqrt{-2} = x - \sqrt{-2} + 2\sqrt{-2}\quad\quad N(x+\sqrt{-2}) = x^2 + 2 > 8 = N(2\sqrt{-2})
\ee

Rewrite $x = 4z \pm 1$, we have
\beast
4z \pm 1- \sqrt{-2} & = & 2\sqrt{-2} \cdot (-1 - z\sqrt{-2}) + (\pm 1 + \sqrt{-2})\quad\quad N(2\sqrt{-2}) = 8 > 5 = N(\pm 1 +\sqrt{-2})\\
2\sqrt{-2} & = & (1 \pm \sqrt{-2})(\pm 1+\sqrt{-2}) \pm 1\quad\quad N(\pm 1+\sqrt{-2}) = 5 >1 = N(\pm 1)
\eeast

Thus, we can say the hcf of $x+\sqrt{-2}$ and $x- \sqrt{-2})$ is $\pm 1$. Write $y^3 = \prod^n_{i=1} p_i^3$ where $p_i$ are distinct prime (irreducible). Then
\be
(x+\sqrt{-2})(x- \sqrt{-2}) = y^3  \ \ra \ x+\sqrt{-2} = u \prod_{i\in A} p_i^3 \quad \text{for some }A\subseteq \bra{1,\dots,n}, \ u \in \bra{1,-1} \ (\text{unit})
\ee

Thus, $x+\sqrt{-2} = p^3$, for some $p \in \Z[\sqrt{-2}]$. Thus, let $p = a+ b\sqrt{-2}$, we have
\be
x + \sqrt{-2} = a^3 - 6ab^2 + (3a^2 b - 2b^3)\sqrt{-2} \ \ra \ b(3a^2 - 2b^2) = 1 \ \ra \ b = \pm.
\ee
Thus, if $b=1$, $a=\pm 1$. If $b= -1$, there is no solutio for $a$. Thus,
\be
x = a(a^2 - 6b^2) = (\pm 1)(1 - 6) = \pm 5, y^3 = x^2 + 2 = 27 \ \ra \ y = 3.
\ee

Thus, $x=\pm 5$, $y = 3$ is the only solution.
\end{solution}



\begin{problem}
Let $\F$ be a field, and let $R = \F[X, Y]$ be the polynomial ring in two variables.
\ben
\item [(i)] Let $I$ be the principal ideal generated by the element $X - Y$ in $R$. Show that $R/I \cong \F[X]$.
\item [(ii)] What can you say about $R/I$ when $I$ is the principal ideal generated by $X^2 + Y$?
\item [(iii)] What can you say about $R/I$ when $I$ is the principal ideal generated by $X^2 - Y^2$?
\een
\end{problem}

\begin{solution}[\bf Solution.]
\ben
\item [(i)] Let $\phi : R \to \F[X]$ by $\phi(X) =X$, $\phi(Y) =X$ (which maps $Y$ to $X$) and extend it homomorphically. So
\be
\ker\phi = \bra{f(X,Y) : f(X,X) = 0}.
\ee

Clearly, $I = \bsa{X-Y} \in \ker\phi$ when we set $Y=X$. Since $\F[X,Y] = (\F[Y])[X]$ ($\F$ is a field and thus a UFD), then $\forall f(X,Y) \in \ker\phi$, by Euclidean algorithm (Proposition \ref{pro:euclidean_algorithm_polynomial}), we can find $q(X,Y),r(X,Y)\in R = \F[X,Y]$,
\be
f(X,Y) = (X-Y)q(X,Y) + r(X,Y)
\ee
with $\deg(r(X,Y)) < \deg(X-Y)$ for $X$. So $r(X,Y)$ is constant in $X$ and $r(X,Y) = r(Y)$. Thus,
\be
f(X,Y) = (X-Y)q(X,Y) + r(Y) \ \ra \ 0 = f(X,X) = r(Y).
\ee
So
\be
f(X,Y) = (X-Y)q(X,Y) \ \ra \ f(X,Y) \in \bsa{X-Y} = I \ \ra \ \ker \phi \subseteq I.
\ee

Hence, $I = \ker \phi$. For any $g(X) \in \F[X]$, we can find $h(X,Y)\in \F[X,Y]$ such that $g(X) = h(X,X)$. Thus, $\phi$ is surjective, $\im\phi = \F[X]$. Then by the first isomorphism theorem,
\be
R/\ker\phi \cong \im \phi \ \ra \ R/I \cong \F[X].
\ee

\item [(ii)] Let $\phi :R \to \F[X]$ by $\phi(X) =X$, $\phi(Y) = -X^2$ and extend it homomorphically. So
\be
\ker\phi = \bra{f(X,Y) : f(X,-X^2) = 0}.
\ee

Clearly, $I = \bsa{X^2 + Y} \in \ker\phi$ when we set $Y=-X^2$. Since $\F[X,Y] = (\F[X])[Y]$ ($\F$ is a field and thus a UFD), then $\forall f(X,Y) \in \ker\phi$, by Euclidean algorithm, we can find $q(X,Y),r(X,Y)\in R = \F[X,Y]$,
\be
f(X,Y) = (Y+X^2)q(X,Y) + r(X,Y)
\ee
with $\deg(r(X,Y)) < \deg(Y+X^2)$ for $Y$. So $r(X,Y)$ is constant in $Y$ and $r(X,Y) = r(X)$. Thus,
\be
f(X,Y) = (Y+X^2)q(X,Y) + r(X) \ \ra \ 0 = f(X,X) = r(X).
\ee
So
\be
f(X,Y) = (Y+X^2)q(X,Y) \ \ra \ f(X,Y) \in \bsa{X^2 + Y} = I \ \ra \ \ker \phi \subseteq I.
\ee

Hence, $I = \ker \phi$. For any $g(X) \in \F[X]$, we can find $h(X,Y)\in \F[X,Y]$ such that $g(X) = h(X,-X^2)$. Thus, $\phi$ is surjective, $\im\phi = \F[X]$. Then by the first isomorphism theorem,
\be
R/\ker\phi \cong \im \phi \ \ra \ R/I \cong \F[X].
\ee

%Note that $\F[X,Y] = (\F[X])[Y]$.
\item [(iii)] Let $\phi:R\to \F[X]\times \F[Y]$ by $\phi(X) = (X,Y)$ and $\phi(Y) = (-X,Y)$ and extend it homomorphically. So
\be
\ker\phi = \bra{f(X,Y) : \phi(f(X,Y) = (0,0)}.
\ee

In the first coordinate, we replace $Y$ with $-X$, so we need $X+Y$ as a factor. Similarly, for the second coordinate, we need $X-Y$. Thus, we can have
\be
\ker \phi = \bsa{X^2 - Y^2} = I.
\ee

However, the map is not surjective since $f(X)\in \F[X]$ and $g(Y)\in \F[Y]$ must have the same order (by having the same coefficients). Thus,
\be
\im \phi = \bra{(f(X),g(Y)): \deg(f(X)) = \deg(g(Y)), f(X) \in \F[X],g(Y)\in \F[Y]}.
\ee

By the first isomorphism theorem, $R/I \cong \im\phi \leq \F[X]\times \F[Y]$.
\een
\end{solution}


\begin{problem}
Determine whether or not the following rings are fields, EDs, PIDs, UFDs, integral domains:
\be
\text{(i)}\ \Z[X],\quad \text{(ii)}\ \Z[X]/\bsa{X^2 + 1},\quad \text{(iii)}\ (\Z/2\Z)[X]/\bsa{X^2 + 1},\nonumber
\ee
\be
\text{(iv)}\ \Z[X]/\bsa{2,X^2 + X + 1},\quad \text{(v)}\ \Z[X]/\bsa{3,X^2 + 1},\quad \text{(vi)}\ \Z[X]/\bsa{3,X^3 - X + 1}.
\ee
\end{problem}

\begin{solution}[\bf Solution.]
Here we use the following properties:
\ben
\item $R[X]$ is a PID iff $R$ is a field.
\item If $R$ is a UFD, then $R[X]$ is also a UFD.
\item Field $\ra$ ED $\ra$ PID $\ra$ UFD $\ra$ ID.
\een

\ben
\item [(i)] $\Z[X]$. The fact that $\Z$ is not a field implies that $\Z[X]$ is not PID. Since $\Z$ is ED, it is UFD. Then $\Z[X]$ is UFD.

\item [(ii)] $\Z[X]/\bsa{X^2 + 1}$. Let $f(X) \in \Z[X]$ and $I = \bsa{X^2 + 1}$. By Euclidean algorithm for polynomials, we have
\be
f(X) = g(X)(X^2 +1 ) + r(X),\quad g(X),r(X)\in \Z[X], \ \deg(r(X)) < 2.
\ee

Then
\be
f(X) + I = a X + b + g(X)(X^2+ 1) + I = a X + b + I.
\ee

Thus,
\beast
a_1 X + b_1 + I + a_2X + b_2  +I & = & (a_1 + a_2)X + (b_1 + b_2) + I,\\
(a_1 X + b_1 + I)(a_2X + b_2  +I) & = & a_1a_2 X^2 + (a_2b_1 + a_1b_2)X + b_1b_2 + I\\
& = & (a_2b_1 + a_1b_2)X + (b_1b_2 - a_1a_2) + I.
\eeast

Thus, we have (or by the third isomorphism theorem)
\be
\Z[X]/\bsa{X^2 + 1} \cong \Z[i]
\ee
by $b + aX \ \lra \ b + ai$. We already know that $\Z[i]$ is a ED and thus a PID. So $\Z[X]/\bsa{X^2 + 1}$ is a ED and PID, etc. However, we have that $2\in \Z[X]/\bsa{X^2 + 1}$ which doesn't have inverse. Thus, we can say that $\Z[X]/\bsa{X^2 + 1}$ is not a field.

\item [(iii)] $(\Z/2\Z)[X]/\bsa{X^2 + 1}$. With the similar argument as above (or by the third isomorphism theorem), we have
\be
(\Z/2\Z)[X]/\bsa{X^2 + 1} = \F_2[X]/\bsa{X^2 + 1} \cong \F_2[i] = \bra{0,1,i,1+i}.
\ee

We see that $\F_2[i]$ is not an integral domain since $(1+i)^2 \mod{2} = 0$.

It is obvious that $1$ is a unit. Also,
\be
1 \cdot (1+i) = 1+i \neq i, \quad 1 \cdot i = i, \quad (1+i)i = 1 + i \neq i.
\ee

Thus, if $i = ab$, $a,b\in \F_2[i]$, we have $a=1$ or $b=1$ and thus $i$ is irreducible. Since $1+i = i(1+i)$, $1+i$ is not irreducible. However, $1+i \neq i^n$ for any $n\in \Z$. So $\F_2[i]$ is not a UFD.

Thus, $(\Z/2\Z)[X]/\bsa{X^2 + 1}$ is not field, ED, PID, UFD, integral domain. (In this case, $b_0$ is not related to $b_1$).

\item [(iv)] Let $f(X) \in \Z[X]$ and $I = \bsa{2,X^2 + X+ 1}$. By Euclidean algorithm for polynomials, we have
\be
f(X) = g(X)(X^2 + X + 1 ) + r(X),\quad g(X),r(X)\in \Z[X], \ \deg(r(X)) < 2.
\ee

Then
\be
f(X) + I = a X + b + I,\quad a,b \in \bra{0,1}.
\ee

Thus,
\be
\Z[X]/\bsa{2,X^2 + X+ 1} = \bra{I,1+I,X+I,X+1+I}.
\ee

Then, $I(1+I) = I$, $(1+I)(1+I) = 1 + I$, $(X+I)(1+I) = X + I$, $(X+1+I)(1+I) = X+1+I$. $1+I$ is the multiplicative identity. $I$ is the additional identity.
\beast
(X+I)(X+I) & = & X^2 + 2XI + I = (X^2 + X + 1) + X+1 + I = X+1 +I,\\
(X+1+I)(X+1+I) & = & X^2 + 2X + 1 + I = (X^2 + X + 1) + X + I = X+I,\\
(X+I)(X+1+I) & = & X^2 + X + I = X^2 + X + 1 + 1 + I = 1+I.
\eeast

Thus, $X+I$ and $X+1+I$ are multiplicative inverse to each other. Hence, $\Z[X]/\bsa{2,X^2 + X+ 1}$ is a field, ED, etc.

Another way to see it is using $\Z[X]/\bsa{p, f_\alpha(X)} \cong  \F_p[\alpha]$ which is
\be
\Z[X]/\bsa{2, X^2 + X + 1} \cong \F_2[\omega], \quad \omega = \frac {-1\pm \sqrt{-3}}2.
\ee

In particular, take $\omega = \frac {-1 + \sqrt{-3}}2$
\be
\F_2[\omega] = \bra{0,1,\frac {-1 + \sqrt{-3}}2 , \frac {1 + \sqrt{-3}}2}
\ee

We have
\beast
\bb{\frac {-1 + \sqrt{-3}}2}^2 & = & \frac {-2 - 2\sqrt{-3}}4 = \frac {-1 - \sqrt{3}}2 = (-1) \frac {1 + \sqrt{3}}2 = 1 \cdot \frac {1 + \sqrt{3}}2 \lmod{2\omega}\\
\bb{\frac {1 + \sqrt{-3}}2}^2 & = & \frac {-2 + 2\sqrt{-3}}4 = \frac {-1 + \sqrt{3}}2 \\
\bb{\frac {-1 + \sqrt{-3}}2}\bb{\frac {1 + \sqrt{-3}}2} & = & \frac {-3 - 1}4 = -1 = 1 \lmod{2}
\eeast

Thus, $\frac {-1 + \sqrt{-3}}2$ and $\frac {1 + \sqrt{-3}}2$ are multiplicative inverse to each other.


\item [(v)] $\Z[X]/\bsa{3,X^2+ 1}$. With similarly argument, we have
\be
\Z[X]/\bsa{3,X^2 + 1} \cong \F_3[i] = \bra{0,1,2,i,1+i,2+i,2i,1+2i,2+2i}.
\ee

Then we have
\beast
2 \cdot 2 & = & 4 \equiv 1\lmod{3}\quad \ra \quad \text{2 is the inverse of itself}\\
i \cdot 2i & = & -2 \equiv 1 \lmod{3}\quad \ra \quad \text{$i$ is the inverse of $2i$}\\
(1+i)(2+i) & = & 2-1 + 3i \equiv 1 \lmod{3} \quad \ra \quad \text{$1+i$ is the inverse of $2+i$}\\
(1+2i)(2+2i) & = & 2 - 4 + 6i \equiv 1 \lmod{3} \quad \ra \quad \text{$1+2i$ is the inverse of $2+2i$}
\eeast

Thus, $\Z[X]/\bsa{3,X^2+ 1}$ is field, ED, etc.

\item [(vi)] $\Z[X]/\bsa{3,X^3 - X + 1}$. Let $f(X) \in \Z[X]$ and $I = \bsa{3,X^3 - X+ 1}$. By Euclidean algorithm for polynomials, we have
\be
f(X) = g(X)(X^3 - X + 1 ) + r(X),\quad g(X),r(X)\in \Z[X], \ \deg(r(X)) < 3.
\ee

Then
\be
f(X) + I = a X^2 + bX + c + I,\quad a,b,c \in \bra{0,1,-1}.
\ee

Thus,
\beast
\Z[X]/\bsa{3,X^3 - X+ 1} & = & \left\{I,1+I,-1+I,X+I,X+1+I,X-1+I, -X+I,-X+1+I,-X-1+I,\right.\\
& & \left. X^2 + I,X^2 + 1+I,X^2 -1 +I,X^2 + X+I,X^2 +X+1+I,\right.\\
& & \left. X^2 +X -1 +I, X^2 - X+I, X^2 - X+1+I,X^2 - X -1+I,\right.\\
& & \left. -X^2 + I,-X^2 + 1+I,-X^2 - 1+I, -X^2 + X+I,-X^2 +X+1+I,\right.\\
& & \left. -X^2 +X-1+I, -X^2 -X+I, -X^2 - X+1+I,-X^2 - X -1 +I\right\}
\eeast
%\beast
%\Z[X]/\bsa{3,X^3 - X+ 1} & = & \left\{I,1+I,2+I,X+I,X+1+I,X+2+I, 2X+I,2X+1+I,2X+2+I,\right.\\
%& & \left. X^2 + I,X^2 + 1+I,X^2 + 2+I,X^2 + X+I,X^2 +X+1+I,\right.\\
%& & \left. X^2 +X+2+I, X^2 + 2X+I, X^2 + 2X+1+I,X^2 + 2X+2+I,\right.\\
%& & \left. 2X^2 + I,2X^2 + 1+I,2X^2 + 2+I,2X^2 + X+I,2X^2 +X+1+I,\right.\\
%& & \left. 2X^2 +X+2+I, 2X^2 + 2X+I, 2X^2 + 2X+1+I,2X^2 + 2X+2+I\right\}
%\eeast

Then, $1+I$ is the multiplicative identity. $I$ is the additional identity.
\beast
(-1+I)(-1+I) & = & 1 + I,\\
(-X+I)(X^2 - 1 + I) & = & -X^3 + X + I = -(X^3 - X + 1) + 1 + I = 1 +I,\\
(-X-1+I)(X^2 -X + I) & = & -X^3 + X + I = -(X^3 - X + 1) + 1 + I = 1 +I,\\
(-X+1+I)(X^2 + X + I) & = & -X^3 - X + I = -(X^3 - X + 1) + 1 + I = 1 +I,\\
-(X^2+1+I)(X^2 - X + 1 + I) & = & -X^4 + X^3 - 2X^2 + X - 1 + I = X^3 - X - 1 + I = 1+I,\\
-(X^2+I)(X^2 + X - 1 + I) & = & -X^4 - X^3 + X^2 + I = -X^3 + X - 1 + 1 + I = 1+I,\\
-(X^2-X-1+I)(X^2 + X+ 1 + I) & = & -X^4 + X^2 + 2X + 1 + I = 1+I.
\eeast

Thus, $\pm(X+I)$ is the inverse of $\mp(X^2 - 1 + I)$,

$\pm(X+1+I)$ is the inverse of $\mp(X^2 -X + I)$

$\pm(X-1+I)$ is the inverse of $\mp(X^2 + X + I)$

$\pm(X^2+1+I)$ is the inverse of $\mp(X^2 - X + 1 + I)$

$\pm(X^2+I)$ is the inverse of $\mp(X^2 + X - 1 + I)$

$\pm(X^2-X-1+I)$ is the inverse of $\mp(X^2 + X+ 1 + I) $

Hence $\Z[X]/\bsa{3,X^3 - X + 1}$ is a field, ED, etc.%Thus, $X+I$ and $X+1+I$ are multiplicative inverse to each other. Hence, $\Z[X]/\bsa{2,X^2 + X+ 1}$ is a field, ED, etc Again, we have %\be
%\Z[X]/\bsa{3, X^3 - X + 1} \cong \F_3[\omega], \quad \omega = \frac {-1\pm \sqrt{-3}}2.
%\ee
\een
\end{solution}

\begin{problem}
Determine which of the following polynomials are irreducible in $\Q[X]$:
\be
X^4 + 2X + 2,\quad X^4 + 18X^2 + 24, \quad X^3 - 9,\quad X^3 + X^2 + X + 1,\quad X^4 + 1,\quad X^4 + 4.
\ee
\end{problem}

\begin{solution}[\bf Solution.]
%Since all the cases of $f(X)$ are primitive, it suffices to check if $f(X)$ is irreducible in $\Z[X]$.
%Gauss' Lemma. We know that $\Z$ is a UFD and $\Q$ is its field of fractions. Suppose $f(X)\in \Z[X]$ is primitive. Then $f(X)$ is irreducible in $R[X]$ if and only if $f(X)$ is irreducible in $\Q[X]$.
%Eisenstein's criterion. Let $R$ be a UFD and let
%\be
%f(X) = a_0+a_1X+ \dots + a_nX^n \in R[X],\ a_n \neq 0,
%\ee
%be primitive. Assume that for some irreducible $p$ we have $ p \nmid a_n$, $p | a_i$ for $0 \leq i < n$ and $p^2 \nmid a_0$. Then $f(X)$ is irreducible in $R[X]$, and hence in $\F[X]$ by Gauss' lemma.
We apply Eisenstein's criterion (Proposition \ref{pro:eisenstein_criterion}):
\ben
\item [(i)] $X^4 + 2X + 2$. Take $p = 2$, then $p = 2 \nmid 1 = a_4$, $p = 2 \mid 2 = a_1$ and $p = 2\mid 2 = a_0$. Also $p^2 = 4 \mid 2 = a_0$. Thus, $f(X)$ is irreducible in $\Q[X]$.

\item [(ii)] $X^4 + 18X^2 + 24$. Take $p = 3$, then $p = 3 \nmid 1 = a_4$, $p = 3 \mid 18 = a_1$ and $p = 3\mid 24 = a_0$. Also $p^2 = 9 \mid 24 = a_0$.  Thus, $f(X)$ is irreducible in $\Q[X]$. Thus, $f(X)$ is irreducible in $\Q[X]$.

\item [(iii)] $X^3 - 9$. Suppose that $X^3 - 9 = (X^2 + aX + b)(X + c)$ and thus $bc = -9$. Therefore, $c = \pm 1,\pm 3,\pm 9$. However, $\pm 1,\pm 3,\pm 9$ are not roots of $X^3 - 9$. So $f(X)$ is irreducible in $\Z[X]$ and thus irreducible in $\Q[X]$.

\item [(iv)] $X^3 + X^2 + X + 1 = (X^2 + 1)(X+1)$. So $X^3 + X^2 + X + 1$ is not irreducible in $\Z[X]$ and thus not irreducible in $\Q[X]$.

\item [(v)] $X^4 + 1$.

Suppose that $X^4 +1 = (X^3 + aX^2 + bX + c)(X + d)$ and thus $cd = 1$. Therefore, $d = \pm 1$. However, $\pm 1$ are not roots of $X^4 +1$.

Suppose that $X^4 +1 = (X^2 + aX + b)(X^2 + cX + d) = X^4 + (a+c)X^3 + (b+d + ac)X^2 + (bc + ad)X + bd$ and thus $bd = 1$. Therefore, $b,d = \pm 1$ and
\be
a + c = 0, \ \pm 2 + ac = 0 \ \ra \ \pm 2 = a^2
\ee
which is contradiction. Thus, $X^4 + 1$ is irreducible in $\Z[X]$ and thus irreducible in $\Q[X]$.

\item [(vi)] $X^4 + 4 = (X^2 - 2X + 2)(X^2 + 2X + 2)$. Thus, it is not irreducible in $\Z[X]$ and thus not irreducible in $\Q[X]$.
\een
\end{solution}


\begin{problem}
\ben
\item [(i)] Consider the polynomial $f(X, Y) = X^3Y + X^2Y^2 + Y^3 - Y^2 - X - Y + 1$ in $\C[X, Y]$. Write it as an element of $\C[X][Y]$, that is collect together terms in powers of $Y$, and then use Eisenstein's criterion to show that $f$ is prime in $\C[X, Y]$.
\item [(ii)] Let $F$ be any field. Show that the polynomial $f(X, Y) = X^2 + Y^2 - 1$ is irreducible in $F[X, Y]$, unless $F$ has characteristic 2. What happens in that case?
\een
\end{problem}

\begin{solution}[\bf Solution.]
\ben
\item [(i)] We have that $\C[X][Y]$ is UFD since $\C$ is UFD.
\be
f(X,Y) = Y^3 + (X^2-1)Y + (X^3 -X)Y - (X-1).
\ee

Thus, by Eisenstein's criterion
\be
(X-1)\mid (X^2 - 1), \quad (X-1)\mid (X^3 -X),\quad (X-1)\mid -(X-1),\quad (X-1)^2\nmid -(X-1),
\ee
then $f(X,Y)$ is irreducible (and hence prime since $\C[X,Y]$ is a UFD).

\item [(ii)] We have $f(X, Y) = X^2 + Y^2 - 1 = Y^2 + (X-1)(X+1)$. Both of $X-1$ and $X+1$ are irreducible.

If $X+1$ and $X-1$ are not associate, we have $f(X,Y)$ is irreducible. In this case, $1 \neq -1$ which implies that $2 \neq 0$ i.e. $\chara \F \neq 2$.

If $\chara \F=2$, we have
\be
X^2 + Y^2 - 1 = X^2 + Y^2 + 2XY + 2X + 2Y + 2 - 1 = (X+Y+1)(X+Y+1)
\ee
which is not irreducible.
\een
\end{solution}


%%%%%%%%%%%%%%%%%%%%%%%%%%%%%%%%%%%%%%%%%%%%%%%%%%%%%%%%%%%%%%%%%%%%%%%%%%%%%%%%%%

%\begin{problem}
%Show that the subring $\Z[\sqrt{2}]$ of $\R$ is a Euclidean domain. Show that the units are $\pm(1\pm \sqrt{2})^n$ for $n \geq 0$.
%\end{problem}

%\begin{solution}[\bf Solution.]
%For $a,b \in \Z$, we define a function
%\be
%\phi:\Z[\sqrt{2}] \to \Z^+,\ a+b\sqrt{2} \mapsto \abs{\bb{a+b\sqrt{2}}\bb{a-b\sqrt{2}}} = \abs{a^2 - 2b^2}.
%\ee

%Then for $a+b\sqrt{2}, c+ d\sqrt{2} \in \Z[\sqrt{2}]$,
%\beast
%\phi\bb{\bb{a+b\sqrt{2}}\bb{c+d\sqrt{2}}} & = & \phi \bb{ac + 2bd + (ad + bc)\sqrt{2}} = \abs{(ac + 2bd)^2 - 2(ad + bc)^2} \\
%& = & \abs{a^2c^2 + 4b^2 d^2 + 4abcd - 2a^2d^2 - 2b^2c^2 - 4abcd} = \abs{(a^2 - 2b^2)(c^2 - 2d^2)} \\
%& = & \abs{a^2 - 2b^2}\abs{c^2 - 2d^2} = \phi(a+b\sqrt{2})\phi(c+d\sqrt{2}).
%\eeast

%Also, since $a,b\in \Z$
%\be
%\phi(a+ b\sqrt{2}) = \abs{a^2 - 2b^2} \geq 1 \text{ if }a+ b\sqrt{2}\neq 0.
%\ee

%So we have for $x,y \in \Z[\sqrt{2}]\bs\bra{0}$,
%\be
%\phi(xy) \geq \phi(x).
%\ee

%Now we need to prove that if $a,b\in \Z[\sqrt{2}]\bs \bra{0}$ with $b\neq 0$, $\exists q,r \in \Z[\sqrt{2}]$ s.t.
%\be
%a= bq + r,\quad \text{with either $r=0$ or }\phi(r) < \phi(b).
%\ee

%Given $a,b$, if $b\mid a$, i.e. $a = bx$ for some $x\in \Z[\sqrt{2}]$, we have that $r=0$. Otherwise, we can find some $s,t \in \R$ such that
%\be
%a /b = s + t\sqrt{2}.
%\ee

%Then we can find $x,y \in \Z$ such that $\abs{x-s} \leq \frac 12$ and $\abs{y -t }\leq \frac 12$. Let
%\be
%q = x + y \sqrt{2} \ \ra\ a = b(x+ y\sqrt{2}) + \underbrace{b(s-x)}_{\in \Z} + \underbrace{b(t-y)}_{\in \Z}\sqrt{2}
%\ee
%with $r = b(s-x) + b(t-y) \sqrt{2} \in \Z[\sqrt{2}]$ and
%\be
%\phi (r) = \phi (b)\abs{ (s-x)^2 - 2 (t-y)^2} \leq \phi(b) \abs{(s-x)^2 + 2(t-y)^2} \leq \phi(b) \abs{\frac 14 + 2 \frac 14} = \frac 34 \phi(b) < \phi(b)
%\ee
%since $b\neq 0$. Thus, $\phi$ is an Euclidean function and $\Z[\sqrt{2}]$ is an Euclidean domain.

%If $x$ is a unit in $\Z[\sqrt{2}]$, then we have that $\exists y \in \Z[\sqrt{2}]$ such that $xy = 1$ and thus $\phi(x)\phi(y) = 1$. Since $\phi(x),\phi(y) \geq 1$, we have that $\phi(x) = 1$.

%Since $\sqrt{2}-1 = 1/(1 + \sqrt{2})$, we can consider $1+\sqrt{2}$ only. Now we want to find all the units bigger than 1.

%First we prove that $1+\sqrt{2}$ is actually the smallest unit bigger than 1.

%Let $a,b\in \N$. If the smallest unit bigger than 1 is of the form $a-b\sqrt{2} > 1$, then $a+b\sqrt{2} > 1$. However,
%\be
%1 = \phi(a-b\sqrt{2}) = \abs{\bb{a-b\sqrt{2}}\bb{a+b\sqrt{2}}} > \abs{1} = 1. \text{ (Contradiction.)}
%\ee

%Similarly, If the smallest unit bigger than 1 is of the form $-a+b\sqrt{2} > 1$, then $a+b\sqrt{2} > 1$. This will lead to the contradiction again.

%Thus, it has to be of the form  $a+b\sqrt{2}$. Thus, the smallest unit bigger than 1 is $1+\sqrt{2}$.

%Clearly, the power of $1+\sqrt{2}$ are units. If there exists any unit between $\bb{1+\sqrt{2}}^n$ and $\bb{1+\sqrt{2}}^{n+1}$ with integer $n$ (positive or negative), say $u$,
%\be
%\bb{1 + \sqrt{2}}^n < u < \bb{1+\sqrt{2}}^{n+1} \ \ra \ 1 < u \bb{1+\sqrt{2}}^{-n} < 1+\sqrt{2}.
%\ee

%Obviously, $u \bb{1+\sqrt{2}}^{-n}$ is a unit since $u^{-1}$ and $\bb{1+\sqrt{2}}^{n}$ are units. However, $1+\sqrt{2}$ is the smallest unit bigger than 1. This leads to contradiction. Thus, there is no other unit bigger than 1 besides $(1+\sqrt{2})^n$ with $n\in \N$.
%\end{solution}

%%%%%%%%%%%%%%%%%%%%%%%%%%%%%%%%%%%%%%%%%

