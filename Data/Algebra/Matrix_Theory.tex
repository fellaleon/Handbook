\chapter{Matrix Theory}

%\section{Matrix}

\section{Motivation and Linear System}%Definitions of Matrices}

\subsection{History of matrix}

\subsection{Linear equations}

\subsection{Gaussian elimination}

%\subsection{Definition of matrix}

\subsection{Gauss-Jordan method}

\subsection{Row echelon form and reduced row echelon form}

definition of rank (number of pivots)
%\subsection{Consistency of linear systems}

%\circlenode{A}{A}%\circlenode{1}{A}

%%%%%%%%%%%% there is a conflict between pstricks and tikz
%\encircle{1}\encircle{$f_n$}\encircle{Hello}
%\usepackage{tikz}
%\newcommand\encircle[1]{%
%    \tikz[baseline=(X.base)]
%        \node (X) [draw,shape=circle, inner sep=0]{\strut #1};}

%\section{Basic Definitions and Operations of Matrices}%

\section{Basic Definitions and Properties of Matrix}

\subsection{Definitions of matrix}

\begin{definition}[matrix\index{matrix}]\label{def:matrix}
An $m\times n$ matrix over $\F$ is an array $A = (a_{ij})$ having $m$ rows and $n$ columns with entry\index{entry!matrix} $a_{ij} \in \F$ for $1 \leq  i \leq  m$, $1 \leq  j \leq  n$. Write
$M_{m,n}(\F)$ for the set of all $m \times n$ matrices over $\F$. If $m=n$, we can simply write $M_n(\F)$.
\end{definition}

\begin{remark}
As the notation in section \ref{sec:vector_space}, we have that $\F$ will almost always be the real numbers $\R$ or the complex numbers $\C$ under the usual addition and multiplication, but it could be the rational numbers, the integers modulo a specified prime number, or some other field.
\end{remark}

\begin{example}
\be
A = \bepm 1 & 0 \\ 0 & 1 \eepm,\quad B = \bepm 0 & 1\\ 1 & 0 \eepm, \quad C = \bepm 1 & 2 & 3 \\ 4 & 5 & 6\eepm.
\ee
\end{example}


\begin{definition}
We say two matrics $A = (a_{ij})$, $B = (b_{ij})$ are same, $A =B$ if
\be
a_{ij} = b_{ij},\ \forall i,j.
\ee
\end{definition}


\begin{definition}[partitioned matrix\index{partitioned matrix}]\label{def:partitioned_matrix}
For matrix $P \in M_{m+n,p+q}(\F)$ with $m,n,p,q\in \Z^+$, we can denote $P$ as
\be
P = \bepm A & B \\ C & D \eepm
\ee
where $A\in M_{m,p}(\F)$, $B\in M_{m,q}(\F)$, $C\in M_{n,p}(\F)$ and $D\in M_{n,q}(\F)$. $P$ is called partitioned matrix (or block matrix\index{block matrix}).
\end{definition}

\begin{example}\label{exa:partitioned_matrix}
For matrix
\be
P = \bepm 1 & 2 & 3 \\  4 & 5 & 6 \\ 7 & 8 & 9 \eepm := \bepm P_{11} & P_{12} \\ P_{21} & P_{22} \eepm,
\ee

we can have that
\be
P_{11} = \bepm 1 & 2 \\ 4 & 5 \eepm,\quad P_{12} = \bepm 3 \\ 6 \eepm,\quad P_{21} = \bepm 7 & 8 \eepm ,\quad P_{22} = \brb{9}.
\ee
\end{example}


\begin{definition}[submatrix\index{submatrix}]\label{def:submatrix}
Let $A\in M_{m,n}(\F)$. For index sets $\alpha \subseteq \bra{1,\dots,m}$ and $\beta \subseteq \bra{1,\dots,n}$, we denoted the (sub)matrix that lies in the rows of $A$ indexed by $\alpha$ and the
columns indexed by $\beta$ as $A_{\alpha,\beta}$.

Also, the submatrix of deleting the rows indicated by $\alpha$ and the columns indicated by $\beta$ is denoted as $\wh{A}_{\alpha,\beta}$.
\end{definition}

\begin{example}
For $\alpha = \bra{1,3}$ and $\beta = \bra{1,2}$
\be
A = \bepm 1 & 2 & 3\\ 4 & 5 & 6 \\ 7 & 8 & 9 \eepm,\qquad \ra \quad A_{\alpha,\beta} = \bepm  1 & 2 \\ 7 & 8 \eepm.
\ee
\end{example}





\subsection{Basic matrix operations}


\begin{definition}[addition of matrices]
Let $A,B\in M_{m,n}(\F)$. Matrix addition is defined entry-wise for arrays for the same dimensions and is denoted by $A+B$, i.e., $A+B = (a_{ij} + b_{ij})$.
\end{definition}

\begin{remark}
Addition of matrices corresponds to addition of linear transformations (relative to the same basis), and it inherits commutativity and associativity from the scalar field. The zero matrix (all
entries zero) is the identity under addition.
\end{remark}

\begin{example}
\be
A = \bepm 1 & 2 \\ 3 & 4 \eepm,\quad B = \bepm 3 & 4 \\ 5 & 6 \eepm \quad \ra \quad A+B  = \bepm 4 & 6 \\ 8 & 10 \eepm.
\ee
\end{example}

\begin{definition}[scale of matrices]
Let $A\in M_{m,n}(\F)$. The scale of matrix is defined entry-wise for arrays multiplied by the scalar $\lm\in \F$, i.e., $\lm A = \brb{\lm a_{ij}}$.
\end{definition}

\begin{example}
\be
A = \bepm 1 & 2 \\ 3 & 4 \eepm,\quad \lm = 2 \quad \ra \quad \lm A =  \bepm 2 & 4 \\ 6 & 8 \eepm.
\ee
\end{example}



\begin{proposition}\label{pro:matrix_dimension}
$M_{m,n}(\F)$ is a vector space under operations \beast
(a_{ij}) + (b_{ij}) & = & (a_{ij} + b_{ij})\\
\lm(a_{ij}) & = & (\lm a_{ij}),\qquad \lm\in \F \eeast with $\dim_\F M_{m,n}(\F) = mn$.
\end{proposition}

\begin{proof}[\bf Proof]
Obviously, $M_{m,n}(\F)$ is a vector space by Definition \ref{def:vector_space}.

We now prove the dimension claim. For $1 \leq  i \leq  m$, $1 \leq  j \leq  n$ define \be E_{ij} = \left\{\ba{ll}
e_{i'j'} = 1 & (i', j') = (i, j) \\
e_{i'j'} = 0 \quad\quad & (i', j') \neq (i, j). \ea\right. \ee

This is a natural basis and thus the dimension is $mn$.%\footnote{details needed.}.
\end{proof}


\begin{definition}[multiplication of matrices\index{multiplication of matrices}]\label{def:multiplication_matrices}
Let $A\in M_{m,n}(\F)$ and $B\in M_{n,l}(\F)$. Then $AB = C = \brb{c_{ij}} \in M_{m,l}(\F)$ with elements
\be
c_{ij} = \sum^{n}_{k=1} a_{ik}b_{kj},\qquad 1\leq i\leq m, 1\leq j\leq l.
\ee
\end{definition}

\begin{remark}
\ben
\item [(i)] Note that the column number of $A$ and row number of $B$ must be consistent.
\item [(ii)] It is easy to see that $(\lm A)B = \lm (AB) = A (\lm B)$ for constant $\lm\in \F$.
\een
\end{remark}

\begin{example}
\be
\bepm 1 & 0 \\ 0 & 2 \eepm \bepm 1 & 2 \\ 3 & 4 \eepm = \bepm 1 & 2 \\ 6 & 8\eepm
\ee
\end{example}

\begin{proposition}[associativity of multiplication of matrices]\label{pro:associativity_multiplication_matrix}
Let $A\in M_{m,n}(\F)$, $B \in M_{n,p}(\F)$ and $C \in M_{p,q}(\F)$. Then
\be
A(BC) = (AB)C.
\ee
\end{proposition}

\begin{proof}[\bf Proof]
We have
\beast
\brb{A(BC)}_{ij} & = & \sum_k a_{ik}(BC)_{kj} = \sum_k \sum_l a_{ik} b_{kl}c_{lj}\\
\brb{(AB)C}_{ij} & = & \sum_k (AB)_{ik}C_{kj} = \sum_k \sum_l a_{il} b_{lk}c_{kj}
\eeast

Then we have the required result by switching $k$ and $l$.
\end{proof}

\begin{remark}
Note that multiplication of matrices are not, in general, commutative, i.e. $AB \neq BA$, but it can be commutative when restricted to certain subsets of $M_n(\F)$.
\end{remark}

\begin{example}
\be
\bepm 1 & 0 \\ 0 & 2 \eepm \bepm 1 & 2 \\ 3 & 4 \eepm = \bepm 1 & 2 \\ 6 & 8\eepm \neq \bepm 1 & 4 \\ 3 & 8 \eepm =  \bepm 1 & 2 \\ 3 & 4 \eepm\bepm 1 & 0 \\ 0 & 2 \eepm.
\ee
\end{example}

\begin{proposition}[matrix multiplication is distributive over matrix addition]
For $A,D\in M_{m,n}(\F)$ and $B,C\in M_{n,l}(\F)$, $A(B+C) = AB + BC$ and $(A+D)B = AB + DB$.
\end{proposition}

\begin{proof}[\bf Proof]
By the distributive property of field $\F$, we have
\be
\brb{A(B+C)}_{ij} = \sum_k a_{ik}(B+C)_{kj} = \sum_k \brb{a_{ik}b_{kj} + a_{ik}c_{kj}} = \sum_k \brb{a_{ik}b_{kj} + \sum_k a_{ik}c_{kj}} = (AB)_{ij} + (AC)_{ij},
\ee
as required. Similarly, we have $(A+D)B = AB + DB$.
\end{proof}

\begin{proposition}\label{pro:component_wise_conjugate_matrix}
For $A\in M_{m,n}(\F)$ and $B\in M_{n,l}(\F)$, $\ol{AB} = \ol{A}\ \ol{B}$ where $\ol{A}$ is the component-wise conjugate of $A$.
\end{proposition}

\begin{proof}[\bf Proof]
By the conjugate property on $\C$, we have
\be
\brb{\ol{AB}}_{ij} = \ol{\sum_k a_{ik} b_{kj}} = \sum_k \ol{a_{ik}}\cdot \ol{b_{kj}} = \ol{A}\ol{B},
\ee
as required.
\end{proof}

\begin{definition}[singular\index{singular!matrix}, nonsingular\index{nonsingular!matrix}]\label{def:singularity_matrix}
A linear transformation or matrix $A\in M_{m,n}(\F)$ is said to be nonsingular if it produces the output 0 only for the input 0.

Otherwise, it is singular, i.e. \be A \text{ is singular }\ \ra \ Ax = 0 = 0x \quad\text{for some }0 \neq x \in \F^n. \ee
\end{definition}

\begin{remark}
If $A\in M_{m,n}(\F)$ and $m<n$, then $A$ is necessarily singular\footnote{details needed.}%\footnote{This statement is implies by the similar methods in proof of Proposition \ref{pro:invertible_non_singular_equivalent}}. %We can find $(x_1^T,x_2^T)^T\in \F^n$ with $0 = x_1 \in \F^m$ and $ 0\neq x_2 \in \F^{n-m}$ such that be A x = A \bepm x_1 \\ x_2 \eepm\ee
\end{remark}

\begin{definition}[power matrix\index{power matrix}]\label{def:power_matrix}
For matrix $A\in M_n(\F)$ and given integer $k\geq 1$. Then $B = A^k$ is called the $k$th power matrix of $A$.
\end{definition}

\begin{definition}[polynomial of matrix]\label{def:matrix_polynomial}
Let $p(x)$ be a polynomial of $x$ with \be p(x) = a_n x^n + \dots + a_1 x + a_0,\qquad a_0,a_1,\dots,a_n\in \sF. \ee

Then for $A\in M_n(\F)$, we have polynomial of matrix \be p(A) = a_n A^n + \dots + a_1 A + a_0 I. \ee

Note that $p(A)$ is also a square matrix.
\end{definition}




\begin{definition}[root matrix\index{root matrix}]\label{def:root_matrix}
For matrix $A\in M_n(\F)$ and given integer $k\geq 1$. If there exists $B\in M_n(\F)$ such that $B^k = A$, then $B$ is called the $k$th root matrix of $A$.
\end{definition}

By matrix multiplication (Definition \ref{def:multiplication_matrices}), we have the following definition of partitioned matrix multiplication.

\begin{definition}[partitioned matrix multiplication\index{partitioned matrices multiplication}]\label{def:partitioned_matrices_multiplication}
For matrix $A\in M_{m_1+m_2,n_1+n_2}(\F)$ and $B\in M_{n_1+n_2,k_1+k_2}(\F)$ with
\be
A = \bepm A_{11} & A_{12} \\  A_{21} & A_{22} \eepm,\qquad B = \bepm B_{11} & B_{12} \\  B_{21} & B_{22} \eepm
\ee
where
\beast
& & A_{11}\in M_{m_1,n_1}(\F),\quad A_{12}\in M_{m_1,n_2}(\F),\quad A_{21}\in M_{m_2,n_1}(\F),\quad A_{22}\in M_{m_2,n_2}(\F),\\
& & B_{11}\in M_{n_1,k_1}(\F),\quad B_{12}\in M_{n_1,k_2}(\F),\quad B_{21}\in M_{n_2,k_1}(\F),\quad B_{22}\in M_{n_2,k_2}(\F).
\eeast

Thus, for $C\in M_{m_1+m_2,k_1+k_2}(\F)$,
\be
C = AB = \bepm A_{11}B_{11} + A_{12}B_{21} & A_{11}B_{12} + A_{12}B_{22} \\ A_{21}B_{11} + A_{22}B_{21} & A_{21}B_{12} + A_{22}B_{22} \eepm = \bepm C_{11} & C_{12} \\  C_{21} & C_{22} \eepm.
\ee
\end{definition}

\begin{example}
For matrices $A,B\in M_4(\R)$ with $2\times 2$ partitioned matrices,
\be
A = \bepm 1 & 2 & 3 & 4 \\  2 & 3 & 4 & 5 \\ 3 & 4 & 5 & 6 \\ 4 & 5 & 6 & 7\eepm := \bepm A_{11} & A_{12} \\ A_{21} & A_{22} \eepm,  \qquad
B = \bepm 1 & 1 & 2 & 2 \\  1 & 1 & 2 & 2 \\ 3 & 3 & 4 & 4 \\ 3 & 3 & 4 & 4\eepm := \bepm B_{11} & B_{12} \\ B_{21} & B_{22} \eepm,
\ee

we have
\beast
C_{11} & = & A_{11}B_{11} + A_{12}B_{21} = \bepm 1 & 2 \\ 2 & 3 \eepm \bepm 1 & 1 \\ 1 & 1 \eepm +  \bepm 3 & 4 \\ 4 & 5 \eepm \bepm 3 & 3 \\ 3 & 3 \eepm = \bepm 3 & 3 \\ 5 & 5 \eepm +  \bepm 21 & 21 \\ 27 & 27 \eepm = \bepm 24 & 24 \\ 32 & 32 \eepm \\
C_{12} & = & A_{11}B_{12} + A_{12}B_{22} = \bepm 1 & 2 \\ 2 & 3 \eepm \bepm 2 & 2 \\ 2 & 2 \eepm +  \bepm 3 & 4 \\ 4 & 5 \eepm \bepm 4 & 4 \\ 4 & 4 \eepm = \bepm 6 & 6 \\ 10 & 10 \eepm +  \bepm 28 & 28 \\ 36 & 36 \eepm = \bepm 34 & 34 \\ 46 & 46 \eepm \\
C_{21} & = & A_{21}B_{11} + A_{22}B_{21} = \bepm 3 & 4 \\ 4 & 5 \eepm \bepm 1 & 1 \\ 1 & 1 \eepm +  \bepm 5 & 6 \\ 6 & 7 \eepm \bepm 3 & 3 \\ 3 & 3 \eepm = \bepm 7 & 7 \\ 9 & 9 \eepm +  \bepm 33 & 33 \\ 39 & 39 \eepm = \bepm 40 & 40 \\ 48 & 48 \eepm \\
C_{12} & = & A_{21}B_{12} + A_{22}B_{22} = \bepm 3 & 4 \\ 4 & 5 \eepm \bepm 2 & 2 \\ 2 & 2 \eepm +  \bepm 5 & 6 \\ 6 & 7 \eepm \bepm 4 & 4 \\ 4 & 4 \eepm = \bepm 14 & 14 \\ 18 & 18 \eepm +  \bepm 44 &
44 \\ 52 & 52 \eepm = \bepm 58 & 58 \\ 70 & 70 \eepm
\eeast
which is consistent with $C = AB$.
\end{example}






\subsection{Transpose matrix and adjoint matrix}

\begin{definition}[transpose\index{transpose!matrix}]\label{def:transpose_matrix}
If $A \in M_{m,n}(\F)$ with $A = (a_{ij})$, then its transpose $A^T$ is definite to be a $n\times m$ matrix with
\be
A^T = (a_{ji}) \in M_{n,m}(\F).
\ee
\end{definition}

%\begin{remark}\end{remark}

\begin{example}
\ben
\item [(i)]
\be
\bepm 1 & 2\\ 3 & 4 \\ 5 & 6 \eepm^T = \bepm 1 & 3 & 5\\ 2 & 4 & 6 \eepm.
\ee
\item [(ii)] If $x = \bepm x_1\\ x_2 \\ \vdots \\ x_n \eepm$ is column vector, $x^T = \brb{x_1, x_2, \dots , x_n }$ is a row vector. \een
\end{example}



\begin{proposition}\label{pro:matrix_multiple_transpose}
Let $A,C \in M_{m,n}(\F)$ and $B \in M_{n,l}(\F)$. Then
\be
\brb{A^T}^T = A, \qquad (AB)^T = B^T A^T,\qquad (A+C)^T = A^T + C^T.
\ee
\end{proposition}

\begin{proof}[\bf Proof]
$\brb{A^T}^T = A$ and $(A+C)^T = A^T + C^T$ are obvious. Let $A = (a_{ij})$ and $B = (b_{ij})$. Then $A^T = (a'_{ij}) = (a_{ji}) $ and $B^T = (b'_{ij}) = (b_{ji})$. Then
\be
\brb{(AB)^T}_{ij} = (AB)_{ji} = \sum_k a_{jk}b_{ki} = \sum_k b_{ki}a_{jk} =\sum_k b'_{ik}a'_{kj} = (B^TA^T)_{ij}.
\ee

Thus, we have the required result.
\end{proof}

\begin{definition}[column vectors and row vectors of matrices]\label{def:column_row_vectors_of_matrix}
For $A,B\in M_{m,n}(\F)$, we write $A_i$ for the $i$th column of $A$ ($i = 1,\dots,n$) and $B_j$ for the $j$th row of $B$ ($j= 1,\dots,m$) with
\be
A = \brb{A_1,A_2,\dots,A_n},\qquad B = \brb{B_1^T,B_2^T,\dots,B_m^T}^T = \bepm B_1 \\ B_2 \\ \vdots \\ B_m\eepm.
\ee

Then $A_i$ are the column vectors of $A$ and $B_j$ are the row vector of $B$.
\end{definition}

\begin{remark}
If $B_i$ denotes the $i$th column of the matrix $B$, then the $j$th column of the product $AB$ is just $AB_i$.

If $A_j$ denotes the $j$th row of the matrix $A$, then the $j$th row of the product $AB$ is just $A_jB$.

If $A\in M_{m,n}(\F)$ and $x\in \F^n$, then $Ax$ is a linear combination of the columns of $A$.

If $A\in M_{m,n}(\F)$ and $y\in \F^m$, then $y^TA$ is a linear combination of the rows of $A$.
\end{remark}


\begin{definition}[adjoint matrix\index{adjoint matrix}]\label{def:adjoint_matrix}
For $A = \brb{a_{ij}} \in M_{m,n}(\F)$, the (Hermitian) adjoint matrix (or conjugate transpose matrix\index{conjugate transpose matrix}) of $A$ is
\be
A^\adjoint := \brb{\ol{a_{ji}}} = \brb{\ol{A}}^T%,\qquad \text{i.e.,}\  a'_{ij} = .
\ee where $\ol{A}$ is the component-wise conjugate.
\end{definition}

\begin{remark}
Note that $A^* = A^T$ whenever $A$ contains only real entries.
\end{remark}

\begin{example}
\be
{\bepm 1- 4i & i & 2 \\ 3 & 2 +i & 0\eepm}^* = \bepm 1 + 4i & 3 \\ -i & 2-i \\ 2 & 0 \eepm.
\ee
\end{example}

\begin{proposition}\label{pro:matrix_multiple_hermitian}
Let $A \in M_{m,n}(\F)$ and $B \in M_{n,l}(\F)$ ($A$ and $B$ are completx matrices).
\be
\brb{A^\adjoint}^\adjoint = A,\qquad (AB)^* = B^* A^*.
\ee
\end{proposition}

\begin{proof}[\bf Proof]
Similar arguement with Proposition \ref{pro:matrix_multiple_transpose}.
\end{proof}


\begin{remark}
For vectors $x,y\in \C^n$, we have
\be
\brb{y^*x}^* = \ol{y^* x} = x^* y = y^T \ol{x}.
\ee
\end{remark}





\subsection{Identity matrix and zero matrix}

\begin{definition}[identity matrix\index{identity matrix}]\label{def:identity_matrix}
The identity $n\times n$ matrix is defined to be
\be
I = \bepm 1 & 0 & \dots & 0\\ 0 & 1 & \dots & 0 \\ \vdots & \vdots & \ddots & \vdots \\ 0 & 0 & \dots & 1 \eepm,
\ee
i.e., all the elements are 0 except for the diagonal elements that are 1. In particular, we can write $I_n$ for $n\times n$ identity matrix.
\end{definition}

\begin{example}
The $3\times 3$ identity matrix is given by
\be
I = \bepm 1 & 0 & 0 \\ 0 & 1 & 0 \\ 0 & 0 & 1\eepm = \brb{\delta_{ij}}
\ee
where $\delta_{ij}$ is the Kronecker delta.
\end{example}

\begin{proposition}\label{pro:identity_matrix_property}
Let $A,I \in M_n(\F)$ and $I$ is an identity matrix. Then $AI = IA = A$, i.e., identity matrix commutes with all other matrices in $M_n(\F)$.
\end{proposition}

\begin{remark}
Identity matrix is the identity under multiplication.
\end{remark}

\begin{proof}[\bf Proof]
We have
\be
(IA)_{ij} = \sum_k \delta_{ik} a_{kj} = a_{ij} = \sum_k a_{ik}\delta_{kj} = (AI)_{ij},
\ee
i.e., $IA = AI = A$.
\end{proof}

\begin{definition}[zero matrix]
Let $A\in M_{m,n}(\F)$. If each entry of $A$ is zero, then $A$ is called zero matrix.% is the matrix with all entries zero
\end{definition}

\begin{remark}
We note that the symbol 0 is used throughout the denote each of the following: the zero scalar, the zero vector (all components equal to the zero scalar), and the zero matrix (all entries of array
equal to the zero scalar).
\end{remark}

\subsection{Inverible matrices}

\begin{definition}[left inverse\index{left inverse!matrix}, right inverse\index{right inverse!matrix}]\label{def:inverse_left}
Let $A\in M_{m,n}(\F)$. Then $B\in M_{n,m}(\F)$ is a left inverse of $A$ if $BA =I_n$ and $C\in M_{n,m}(\F)$ is a right inverse of $A$ if $AC = I_m$.
\end{definition}

\begin{proposition}[inverse\index{inverse!matrix}, invertible matrix\index{invertible!matrix}]\label{pro:inverse_matrix}
Let $A\in M_n(\F)$. If $B$ is a left inverse of $A$ and $C$ is a right inverse of $A$ then $B=C$ and we write $B=C = A^{-1}$. Then we say that $A$ is invertible (or non-singular\index{non-singular!matrix}). Note that $A^{-1}$ is unique.
\end{proposition}

\begin{proof}[\bf Proof]
By definition of left and right inverse, we have $BA =I$ and $AC = I$. Then
\be
B = BI = B(AC) = (BA)C = IC = C.
\ee
by Proposition \ref{pro:identity_matrix_property}, \ref{pro:associativity_multiplication_matrix}.

If $B$ and $C$ are both inverses of $A$, then $B = BI = BAC = IC = C$. Hence, $A^{-1}$ is unique.
\end{proof}

\begin{remark}
This property is based on the premise that both a left inverse and right inverse exist. In general the existence of a left inverse does not necessarily guarantee the existence of a right inverse or
vice versa.

Equivalently, $A$ is invertible if the linear transformation $A$ is one-to-one, and its inverse transformation (also linear) exists. This is actually bijective.
\end{remark}

However, in the case of a square matrix, the existence of a left inverse does imply the existence of a right inverse, and vice versa. The above property then implies that they are the
same matrix.

\begin{proposition}\label{pro:left_or_right_inverse_of_square_matrix_implies_inverse}
Let $A,B\in M_n(\F)$. We have that $AB = I$ implies $BA=I$. In other words, $B = A^{-1}$.
\end{proposition}

\begin{remark}
Note that the statement is not valid for nonsquare matrices.
\end{remark}

\begin{proof}[\bf Proof]
First, $AB = I$ implies $B$ is nonsingular because if $B$ is singluar (see Definition \ref{def:singularity_matrix}), there is a column vector $x\neq 0$ such that $Bx = 0$, which is contrary to the fact that $x = I x = ABx = A0 = 0$. Then $B$ is invertible.\footnote{see Meyer's book $P_{116}$}

Therefore, $B^{-1}$ exists, we can write that
\be
AB = I \ \ra \ ABB^{-1} = B^{-1} \ \ra \ A = B^{-1} \ \ra \ BA = I.
\ee
\end{proof}


\begin{example}
For matrix $A = \bepm a_{11} & a_{12} \\ a_{21} & a_{22}\eepm $, the inverse of $A$ is
\be
A^{-1} = \frac {1}{a_{11}a_{22} - a_{12}a_{21}} \bepm a_{22} & -a_{12} \\ -a_{21} & a_{11}\eepm
\ee
since
\beast
AA^{-1} & = & \bepm a_{11} & a_{12} \\ a_{21} & a_{22}\eepm\frac {1}{a_{11}a_{22} - a_{12}a_{21}} \bepm a_{22} & -a_{12} \\ -a_{21} & a_{11}\eepm \\
& = & \frac {1}{a_{11}a_{22} - a_{12}a_{21}} \bepm a_{11}a_{22} - a_{12}a_{21} & 0 \\ 0 & a_{11}a_{22} - a_{12}a_{21} \eepm = I
\eeast
and
\beast
A^{-1}A & = & \bepm a_{11} & a_{12} \\ a_{21} & a_{22}\eepm\frac {1}{a_{11}a_{22} - a_{12}a_{21}} \bepm a_{22} & -a_{12} \\ -a_{21} & a_{11}\eepm \bepm a_{11} & a_{12} \\ a_{21} & a_{22}\eepm \\
& = & \frac {1}{a_{11}a_{22} - a_{12}a_{21}} \bepm a_{11}a_{22} - a_{12}a_{21} & 0 \\ 0 & a_{11}a_{22} - a_{12}a_{21} \eepm = I.
\eeast
\end{example}

\begin{example}\label{exa:inverse_matrix_plus_minus_one}
Define $A\in M_n(\F)$ by
\be
A = \bepm 1 & & & & 1 \\
-1 & 1 & & & 1\\
& -1 &  & & 1 \\
& & \ddots & & \vdots\\
& & & 1 & 1\\
& & & -1 & 1\\
\eepm
\ee

Then for $n = 2k, k\in \Z^+$, its inverse matrix is
\be
A^{-1} = \frac 1n \bepm
n-1 & -1 & -1 & & \dots & & -1 &  -1 \\
n-2 & n-2 & -2 & &  \dots & & -2 & -2 \\
& & & \vdots & & & &  \\
k+1 & \dots & k+1 & -(k-1) & -(k-1) & \dots & -(k-1) & -(k-1) \\
k & \dots & k & k & -k & \dots & -k & -k \\
& & & \vdots & & & & \\
1 & \dots & 1 & & & & 1 & -(n-1)\\
1 & \dots & 1 & 1 & \dots & & 1 & 1
\eepm
\ee

For $n = 2k+1, k\in \Z^+$, its inverse matrix is
\be
A^{-1} = \frac 1n \bepm
n-1 & -1 & -1 & & \dots & & -1 &  -1 \\
n-2 & n-2 & -2 & &  \dots & & -2 & -2 \\
& & & \vdots & & & &  \\
k+1 & \dots & k+1 & -k & -k & \dots & -k & -k \\
k & \dots & k & k & -(k+1) & \dots & -(k+1) & -(k+1) \\
& & & \vdots & & & & \\
1 & \dots & 1 & & & & 1 & -(n-1)\\
1 & \dots & 1 & 1 & \dots & & 1 & 1
\eepm
\ee
\end{example}

\begin{proposition}\label{pro:inverse_matrix_property}
If $A,B\in M_n(\F)$ are invertible, then
\be
\text{(i) }\brb{A^{-1}}^{-1} = A,\qquad \text{(ii) }(AB)^{-1} = B^{-1}A^{-1}.
\ee
\end{proposition}

\begin{remark}
This means the product of invertible matrices is still invertible.
\end{remark}

\begin{proof}[\bf Proof]
Let $C = A^{-1}$. Then
\be
\ba{l}
C A = I = CC^{-1} \\
AC = I = C^{-1}C
\ea \quad \ra \quad A = \brb{A^{-1}}^{-1}.
\ee

Also, we have
\beast
B^{-1}A^{-1}(AB) & = & B^{-1}\brb{A^{-1}A}B = B^{-1} I B = B^{-1} B = I,\\
(AB)B^{-1}A^{-1} & = & A\brb{BB^{-1}}A^{-1} = AIA^{-1} = AA^{-1} = I
\eeast
by Proposition \ref{pro:identity_matrix_property}, \ref{pro:inverse_matrix}, \ref{pro:associativity_multiplication_matrix}.
\end{proof}

\begin{proposition}
For invertible matrix $A\in M_n(\F)$, we have
\be
\brb{A^T}^{-1} = \brb{A^{-1}}^T,\qquad \brb{A^*}^{-1} = \brb{A^{-1}}^*.
\ee
\end{proposition}

\begin{proof}[\bf Proof]
For invertible matrix $A$ and its inverse matrix $A^{-1}$, by Proposition \ref{pro:matrix_multiple_hermitian},
\be
\left\{\ba{ll}
I = A A^{-1} \ \ra \ I = I^* = \brb{AA^{-1}}^* = \brb{A^{-1}}^* A^* \\
I = A^{-1} A \ \ra \ I = I^* = \brb{A^{-1}A}^* = A^* \brb{A^{-1}}^*
\ea\right. \quad \ra\quad \brb{A^*}^{-1} = \brb{A^{-1}}^*.
\ee

Similar for the transpose case.
\end{proof}

%%%%%%%%%

\section{Echelon Forms, Elmentary Matrices and Equivalent Matrices}


\subsection{Echelon forms}

%\subsection{Row echelon form and reduced row echelon form}


\begin{definition}[column echolon form]\label{def:column_echolon_form}
$A$ is an $m\times n$ matrix with the following properties is said to be in column echelon form.
\ben
\item [(i)] The highest placed non-zero entry in column $j$ is 1 in row $i_j$ , with $i_1 \leq  i_2 \leq  \dots$.
\item [(ii)] The entry in row $i_j$ and column $k$ with $k < j$ is 0.
\een
\end{definition}

\begin{example}
\be
%\bepm
%1 & 6 & 2 & 4\\
%0 & 1 & 3 & 5\\
%0 & 0 & 1 & 2
%\eepm.
\bepm
1 & 0 & 0 \\
6 & 1 & 0 \\
2 & 3 & 1 \\
4 & 5 & 2
\eepm,\qquad \bepm
0 & 0 & 0 \\
1 & 0 & 0 \\
6 & 1 & 0 \\
2 & 3 & 1
\eepm,\qquad \bepm
0 & 0 & 0 & 0 \\
0 & 1 & 0 & 0 \\
0 & 6 & 1 & 0 \\
0 & 2 & 3 & 1
\eepm
\ee
\end{example}

\subsection{Elementary operations}%Calculations}

\begin{definition}[elementary matrix\index{elementary matrix}]\label{def:elementary_matrix}
The following are the elementary column operations on an $m \times n$ matrix over $\F$.
\ben
\item [(i)] Swap columns $i$ and $j$ (switching or transposition).
\item [(ii)] Replace column $i$ by $\lm$ column $i$, where $\lm \in \F \bs \bra{0}$ (multiplication).
\item [(iii)] Add $\lm$ column $i$ to column $j$, where $i \neq j$ and $\lm \in \F$ (addition).
\een

The corresponding elementary column operation matrices are obtained by applying these operations to $I_n$. Call these $T_{i,j}$, $M_{i,\lm}$ and $C_{i,j,\lm}$. An elementary column operation on $A$ can be performed by postmultiplying $A$ with the corresponding elementary matrix. All these operations are reversible.

Similarly for elementary row operations. We write $E_i$ for elementary (column and row) matrices.

For square matrices, the elementary (column or row) matrices of switching and multiplication are same, given by
\be
T_{i,j} =\bepm 1 & & & & & & \\ & \ddots & & & & & \\ & & 0 & & 1 & & \\ & & & \ddots & & & \\ & & 1 & & 0 & & \\ & & & & & \ddots & \\ & & & & & & 1\eepm ,\quad M_{i,\lm} = \bepm 1 & & & & & & \\ & \ddots & & & & & \\ & & 1 & & & & \\ & & & \lm & & & \\ & & & & 1 & & \\ & & & & & \ddots & \\ & & & & & & 1\eepm.
\ee

The matrices of addition is %are
\be
C_{i,j,\lm} = \bepm 1 & & & & & & & \\ & \ddots & & & & & \\ & & 1 & & \lm & & \\ & & & \ddots & & & \\ & & & & 1 & & \\ & & & & & \ddots & \\ & & & & & & 1\eepm.%, \quad C_{i,j,\lm}^T = \bepm 1 & & & & & & & \\ & \ddots & & & & & \\ & & 1 & & & & \\ & & & \ddots & & & \\ & & \lm & & 1 & & \\ & & & & & \ddots & \\ & & & & & & 1\eepm.
\ee

$AC_{i,j,\lm}$ is adding $\lm$ times column $i$ to column $j$ of $A$ and $C_{i,j,\lm}A$ is adding $\lm$ times row $j$ to row $i$ of $A$ (as $C_{i,j,\lm}^T A$ is adding $\lm$ times row $i$ to row $j$ of $A$).
\end{definition}




\begin{proposition}\label{pro:square_elementary_matrix_invertible}
Let $E \in M_n(\F)$ be elementary matrix. Then it is invertible.
\end{proposition}

\begin{remark}
The square elementary matrices generate the general linear group of invertible matrices\footnote{need details}.
\end{remark}

\begin{proof}[\bf Proof]
It is easy to check that the inverse matrices are
\be
\brb{T_{i,j}}^{-1} = T_{i,j},\quad \brb{M_{i,\lm}}^{-1} = M_{i,1/\lm},\quad \brb{C_{i,j,\lm}}^{-1} = C_{i,j,-\lm}.
\ee
\end{proof}


\begin{lemma}\label{lem:matrix_column_echelon_elementary_column}
Any matrix $A$ can be reduced to a matrix in column echelon form by a sequence of elementary column matrices.
\end{lemma}

\begin{remark}
If $A$ is a square $n\times n$ matrix and is invertible, the equivalent column echelon form is $I_n$. This can be used to find $A^{-1}$.
\be
A \mapsto AE_1E_2 \dots E_k = I_n,\quad I_n \mapsto I_nE_1E_2 \dots E_k = A^{-1}.
\ee
\end{remark}

\begin{proof}[\bf Proof]
\footnote{see \cite{Meyer_2001}.$P_{134}$}
\end{proof}



\begin{lemma}\label{lem:invertible_product_elementary_matrices}
If $A$ is an invertible $n \times n$ matrix then $A$ is a product of elementary matrices.
\end{lemma}

\begin{proof}[\bf Proof]
By the remark of Lemma \ref{lem:matrix_column_echelon_elementary_column}, $A^{-1} = E_1 \dots E_k$ is a product of elementary matrices, hence so is $A = E^{-1}_k \dots E^{-1}_1$ by Proposition \ref{pro:inverse_matrix_property}.
\end{proof}



%\section{Ranks, Determinants and Adjugate Matrices}



\subsection{Equivalent matrices}

\begin{definition}[equivalent matrices\index{equivalent!matrix}]\label{def:equivalent_matrix}
The $m\times n$ matrices $A,B \in M_{m,n}(\F)$ are equivalent if there exist invertible matrices $Q \in M_{m}(\F)$, $P \in M_{n}(\F)$ such that $B = QAP$. This defines an equivalence relation on $M_{m,n}(\F)$.
\end{definition}

\begin{remark}
Equivalent matrices arise as representing the same linear map from a space $U$ to a space $V$ of dimension $m$ and $n$ with respect to different bases.
\end{remark}



\begin{proposition}[equivalence is equivalent relation]\label{pro:equivalence_matrix_is_equivalent_relation}
Equivalence is an equivalent relation on $M_{m,n}(\F)$, i.e., equivalence is for $A,B,C\in M_{m,n}(\F)$,
\ben
\item [(i)] reflexive: $A$ is equivalent to $A$
\item [(ii)] symmetric: $A$ is equivalent to $B$ $\ \ra\ $ $B$ is equivalent to $A$.
\item [(iii)] transitive: $A$ is equivalent to $B$, $B$ is equivalent to $C$ $\ \ra\ $ $A$ is equivalent to $C$.
\een
\end{proposition}

\begin{proof}[\bf Proof]%\footnote{proof needed.}
\ben
\item [(i)] It is obvious that invertible matrices $P = I_n$ and $Q = I_m$ such that $A = I_mAI_n = QAP$.
\item [(ii)] If $A$ is equivalent to $B$, there exist invertible matrices $P$ and $Q$ such that $B = QAP$. Thus, for $P' = P^{-1}$ and $Q' = Q^{-1}$, $A = Q'QAPP'^{-1} = Q'BP$. Hence, $B$ is equivalent to $A$.
\item [(iii)] If $A$ is equivalent to $B$ and $B$ is equivalent to $C$, there exist invertible matrices $P,Q,U,V$ such that $B = QAP$ and $C = VBU$. Thus, for invertible matrices $R = PU$ and $S = VQ$,
\be
C = VBU = VQ A PU = S A R \ \ra \ A \text{ is equivalent to } C.
\ee
\een
\end{proof}

\begin{definition}[column equivalent\index{column equivalent!matrix} and row equivalent\index{row equivalent!matrix} matrices]
Let $A,B \in M_{m,n}(\F)$.

Then $A$ and $B$ are column equivalent if there exists invertible matrix $P \in M_{n}(\F)$ such that $B =AP$.

Also, $A$ and $B$ are row equivalent if there exists invertible matrix $Q \in M_{m}(\F)$, such that $B = QA$.
\end{definition}

\begin{remark}
These define equivalence relations on $M_{m,n}(\F)$ (by Proposition \ref{pro:equivalence_matrix_is_equivalent_relation}).
\end{remark}


%\section{Linear Algebra Approach}

\section{Spaces of Matrices and Ranks}

\subsection{Range space and null space of matrices}

\begin{definition}[range space of matrices\index{range space!matrix}]\label{def:range_space_matrix}
Let $A\in M_{m,n}(\F)$. Then the range of matrix $A$ is defined to the subsapce $\sR(A)$ of $\F^m$ that is generated by the range of $Ax$. That is,
\be
\sR(A) = \bra{Ax :x\in \F^n} \subseteq \F^m.
\ee

Note that $\sR(A)$ is the set of all possible linear combinations of column vectors of $A$. So it is also called the column space\index{column space!matrix} of $A$ as it is the subspace of $\F^m$ generated by the column vectors of $A$, denoted by $\colsp(A)$\index{column space!matrix}.

\be
\sR(A) = \colsp(A) = \bra{\sum^n_{i=1} \alpha_i A_i:\ \alpha_1,\dots,\alpha_n\in \F}.
\ee

It can be easily checked that $\sR(A)$ is a subspace of $\F^m$ and actually
\be
\sR(A) = \linspan\bra{A_1,\dots,A_n}
\ee
by Theorem \ref{thm:span_of_set_is_linear_combination_of_elements_of_set}. %\footnote{Theorem needed. see \cite{Bernstein_2009}. $P_{92}$}.

Because $\sR(A)$ is the set of all images of vectors $x\in \F^m$ under transformation by $A$, it is also called image space of $A$.

Similarly, the range of $A^T$ is the subspace of $\F^n$ defined by
\be
\sR(A^T) = \bra{A^T y :  y\in \F^m} \subseteq \F^n.
\ee

$\sR(A^T) $ is the set of all possible linear combinations of its row vectors. So it is called the row space of $A$ as it is the subspace of $\F^n$ generated by the row vectors of $A$, denoted by $\rowsp(A)$\index{row space!matrix}.
\end{definition}

\begin{remark}
If $\F = \C$, then $\sR(A) = \sR\brb{\ol{A}}$ are not necessarily identical, e.g. $A = (1 \ i )^T$.
\end{remark}

%\begin{definition}[column space\index{column space!matrix}]\label{def:column_space_matrix}
%The column space (sometimes called the range) of a matrix is the set of all possible linear combinations of its column vectors.
%Let $A \in M_{m,n}(\F)$. The column space of $A$ is the subspace of $\F^m$ generated by the column vectors of $A$, denoted by $\colsp(A)$.
%\end{definition}



\begin{definition}[null space\index{null space}]
Let $A\in M_{m,n}(\F)$. Then the null space of $A$ is defined by
\be
\sN(A) = \bra{x\in \F^n:\ Ax = 0}.
\ee

It can be easily checked that $\sN(A)$ is a subspace of $\F^n$.
\end{definition}

\begin{remark}
Note that if $\alpha\in \F$ is non-zero, then by definition
\be
\sR(\alpha A) = \sR(A),\qquad \sN(\alpha A) = \sN(A).
\ee
\end{remark}

\begin{lemma}\label{lem:range_null_space_product}
Let $A\in M_{m,n}(\F), B\in M_{n,p}(\F), C\in M_{k,m}(\F)$. Then
\be
\sR(AB) \subseteq \sR(A),\qquad \sN(A) \subseteq \sN(CA).
\ee
\end{lemma}

\begin{proof}[\bf Proof]
Since $\sR(B)\subseteq \F^n$, it follows that
\be
\sR(AB) = \bra{ABx:x\in\F^p} \subseteq  \bra{A y: y\in \sR(B)} \subseteq \bra{A y: y\in \F^n} = \sR(A).
\ee

%$\sR(AB) =A \sR(B) \subseteq A\sR(B) \subseteq A\F^n = \sR(A)$.

Also, $y\in \sN(A)$ implies that $Ay = 0$ and thus $CAy = 0 \ \ra\ y\in \sN(CA)$ which gives $\sN(A) \subseteq \sN(CA)$.
\end{proof}

\begin{corollary}
Let $A\in M_n(\F)$. Then for $k\in \Z^+$,
\be
\sR\brb{A^k} \subseteq \sR(A),\qquad \sN(A) \subseteq \sN\brb{A^k}.
\ee
\end{corollary}


Although $\sR(AB)\subseteq \sR(A)$ for arbitrary conformable matrices $A,B$, we now show that equality holds for the special case $B = A^*$.

\begin{theorem}\label{thm:adjoint_range_space_null_space_orthogonal_complment}
Let $A\in M_{m,n}(\C)$ and the inner product be the dot product on complex space. Then
\ben
\item [(i)] $\sR(A)^\perp = \sN(A^*)$.
\item [(ii)] $\sR(A) = \sR(AA^*)$.
\item [(iii)] $\sN(A) = \sN(A^*A)$.
\een
\end{theorem}

\begin{remark}
By Proposition \ref{pro:orthogonal_complement_sum_and_twice_complement}.(i) and replacing $A$ by $A^*$, we have
\be
\sR(A) = \brb{\sR(A)^\perp}^\perp = \sN(A^*)^\perp, \qquad \sR(A^*) = \sN(A)^\perp,\qquad  \sN(A) = \brb{\sN(A)^\perp}^\perp = \sR(A^*)^\perp,
\ee
\be
\sR(A^*) = \sR(A^*A),\qquad \sN(A^*) = \sN(AA^*).
\ee

Then we can have
\be
\sR(AA^*A) = A\sR(A^*A) = A\sR(A^*) = \sR(AA^*) = \sR(A).
\ee

Letting $A = (1\ i)$ shows that $\sR(A)$ and $\sR(AA^T)$ may be different as
\be
AA^T = \bepm 1 & i \eepm \bepm 1 \\ i \eepm = 0.
\ee
\end{remark}

\begin{proof}[\bf Proof]
\ben
\item [(i)] First we want to show that $\sR(A)^\perp \subseteq \sN(A^*)$ Let $x\in \sR(A)^\perp$ then $x^*z = 0$ for all $z\in \sR(A)$. Hence,
\be
x^* Ay = 0 \text{ for all }y\in \C^n \ \ra\ \ y^* A^* x = 0 \text{ for all }y\in \C^n.
\ee

Let $y = A^*x$, it follows that
\be
x^*AA^* x = 0 \ \ra\ A^*x = 0  \ \ra\ x\in \sN(A^*).
\ee

Conversely, we should show that $\sN(A^*)\subseteq \sR(A)^\perp$. Let $x\in \sN(A^*)$, it follows that $A^*x = 0$ which implies that
\be
y^* A^* x = 0 \text{ for all }y\in \C^n \ \ra\ x^* A y = 0\text{ for all }y\in \C^n \ \ra\ x^* z = 0 \text{ for all }z\in \sR(A) \ \ra\ x\in \sR(A)^\perp.\nonumber
\ee

\item [(ii)] By Lemma \ref{lem:range_null_space_product} with $B = A^*$ implies that $\sR(AA^*) \subseteq \sR(A)$. Suppose that there exists $x\in \sR(A)$ and $x\notin \sR(AA^*)$. Then by Proposition \ref{pro:orthogonal_complement_sum_and_twice_complement}.(i), $x$ can be written in the following form,
    \be
    x = x_1 + x_2,\qquad x_1\in \sR(AA^*),\ 0\neq x_2\in \sR(AA^*)^\perp \ \ra\ x_2^* AA^* y = 0 \text{ for all }y\in \C^m.
    \ee

Setting $y = x_2$ we have
\be
x_2^* AA^* x_2 = 0 \ \ra\ A^* x_2= 0 \ \ra\ x_2 \in \sN(A^*) = \sR(A)^\perp\qquad \text{by (i).}
\ee

Since $x\in \sR(A)$, it follows that
\be
0 = x_2^* x = x_2^* x_1 + x_2^* x_2 = x_2^* x_2 \ \ra\ x_2 = 0 \ \ra\ \text{contradiction}
\ee
since $x_2^* x_1=0$ by $x_1\in \sR(AA^*),x_2\in \sR(AA^*)^\perp$. This implies that $\sR(A) = \sR(AA^*)$.

\item [(iii)] First, replace $A$ in (ii) by $A^*$ which implies that $\sR(A^*) = \sR(A^*A)$ and thus $\sR(A^*)^\perp = \sR(A^*A)^\perp$. Furthermore, replace $A$ in (i) by $A^*$ yields $\sR(A^*)^\perp = \sN(A)$. Hence, by (i) again,
\be
\sN(A) = \sR(A^*)^\perp = \sR(A^*A)^\perp = \sN\brb{(A^*A)^*} = \sN(A^*A).
\ee
\een
\end{proof}





\subsection{Column rank, row rank and defect}


\begin{definition}[column rank\index{rank!column, matrix}]\label{def:column_rank_matrix}
Let $A\in M_{m,n}(\F)$. The column rank of $A$, denoted $\colrk(A)$ is the dimension of the column space of $A$. That is, $\colrk(A) = \dim\colsp(A)=\dim\sR(A)$.
\end{definition}%C(A) of a matrix (sometimes called the range of a matrix) is the set of all possible linear combinations of its column vectors. The column space of an m × n matrix is a subspace of m-dimensional Euclidean space. The dimension of the column space is called the rank of the matrix.[1]

\begin{remark}
That is, by the definition of dimension, rank of $A$ is the largest number of columns of $A$ that constitute a linearly independent set (see Lemma \ref{lem:finite_basis_linearly_independent_spanning_equivalent}). This set of columns is not, of course, unique, but the cardinality (number of elements (columns in this case)) of this set is unique. This is Theorem \ref{thm:dimension_vector_space}.

For $A\in M_{m,n}(\F)$, $\colrk(A) = n$ iff $A$ is nonsingular\footnote{detail needed.}.
\end{remark}


\begin{definition}[row rank\index{rank!row, matrix}]\label{def:row_rank_matrix}
Let $A\in M_{m,n}(\F)$. The row rank of $A$, denoted $\rowrk(A)$ is the dimension of the column space of $A$. That is, $\rowrk(A) = \dim \rowsp(A) =\dim \sR(A^T) = \colrk(A^T)$.
\end{definition}

\begin{example}
\ben
\item [(i)] Clearly, $\rowrk(I_n) = \colrk(I_n) = n$.
\item [(ii)] For matrix
\be
A = \bepm 1 & 2 \\ 1 & 0 \eepm,
\ee
we have $\colrk(A) = \rowrk(A) = 2$.
\een
\end{example}

\begin{definition}[defect\index{defect!matrix}]\label{def:defect_matrix}
Let $A\in M_{m,n}(\F)$. The The defect of $A$ is $\defect A := \dim \sN(A)$
\end{definition}

%\begin{definition}[row space\index{row space!matrix}]\label{def:row_space_matrix}%{def:range_space_matrix}
%\end{definition}


\begin{lemma}\label{lem:original_conjugate_matrices_have_same_column_rank}
Let $A\in M_{m,n}(\F)$. Then $\colrk(A) = \colrk(\ol{A})$.
\end{lemma}

\begin{proof}[\bf Proof]
If $\colrk(A)= k$, then by Lemma \ref{lem:finite_basis_linearly_independent_spanning_equivalent} there are at most $k$ linearly independent columns $A_1,\dots,A_k$ of $A$ such that
\be
\ol{\alpha_1} A_1 + \dots + \ol{\alpha}_k A_k = 0 \ \ra \ \alpha_1 = \alpha_2 = \dots = \alpha_k = 0.
\ee

This gives that for the corresponding columns $\ol{A}_1,\dots,\ol{A}_k$ of $\ol{A}$,
\be
\alpha_1 \ol{A}_1 + \dots + \alpha_k \ol{A}_k = 0 \ \ra \ \alpha_1 = \alpha_2 = \dots = \alpha_k = 0.
\ee

Thus, there are at most $k$ linearly independent columns in $\ol{A}$. Therefore, $k = \colrk\brb{\ol{A}}$.
\end{proof}


\begin{proposition}\label{pro:colrank_product_smaller_than_individual_colranks}
Let $A\in M_{m,n}(\F)$ and $B \in M_{n,p}(\F)$. Then
\be
\colrk(AB) \leq \min\bra{\colrk(A),\colrk(B)}
\ee
\end{proposition}

\begin{proof}[\bf Proof]
By Lemma \ref{lem:range_null_space_product}, we have $\sR(AB)\subseteq \sR(A)$ which implies that $\colrk(AB) \leq \colrk(A)$ by Proposition \ref{pro:dimension_subspace_property}.

Next, suppose that $\colrk(B) < \colrk(AB)$. Let $\bra{y_1,\dots, y_r}\subseteq \F^n$ be a basis for $\sR(AB)$ with $r = \colrk(AB)$. Since $y_i \in A\sR(B)$ for $i=1,\dots, r$, let $x_i\in \sR(B)$ be such that $y_i = Ax_i$ for all $i=1,\dots,r$. Since $\colrk(B)< r$, it follows that $x_1,\dots,x_r$ are linearly dependent. Hence, there exist $\alpha_1,\dots,\alpha_r\in \F$, not all  zero, such that
\be
\sum^r_{i=1}\alpha_i x_i = 0 \ \ra\ \sum^r_{i=1}\alpha_i A x_i  = \sum^r_{i=1}\alpha_i y_i = 0.
\ee

Thus, $y_1,\dots,y_r$ are linearly dependent, which is a contradiction to the assumption that $\bra{y_1,\dots, y_r}\subseteq \F^n$ is a basis.
\end{proof}


\begin{lemma}\label{lem:colrank_defact_sum_is_column_number}
Let $A\in M_{m,n}(\F)$. Then
\ben
\item [(i)] $\colrk (A^*) + \defect(A) = n$.
\item [(ii)] $\colrk(A) = \colrk(AA^*)$.
\item [(iii)] $\defect(A) = \defect(A^*A)$.
\een
\end{lemma}

\begin{proof}[\bf Proof]
\ben
\item [(i)] Since $\sN(A)$ is a subspace of $\F^n$. Then by Proposition \ref{pro:orthogonal_complement_sum_and_twice_complement}, $\F^n = \sN(A) \oplus \sN(A)^\perp$. Furthermore,
\be
n = \dim_{\F}(\F^n) = \dim\sN(A) + \dim\brb{\sN(A)^\perp}
\ee
by Lemma \ref{lem:direct_sum_vector_space} and Corollary \ref{cor:dimension_sum_intersection_zero_iff_complement_subspaces}. Then by definitions of rank and defect,
\be
\colrk(A^*) = \dim\sR(A^*) = \dim\brb{\sN(A)^\perp} = n - \dim\sN(A) = n - \defect(A).
\ee

\item [(ii)] and (iii) are directly from Theorem \ref{thm:adjoint_range_space_null_space_orthogonal_complment}.
\een
\end{proof}


\subsection{Rank}

\begin{theorem}\label{thm:original_adjoint_matrix_have_same_column_rank}
Let $A\in M_{m,n}(\F)$. Then $\colrk(A) = \colrk(A^*)$.
\end{theorem}

\begin{proof}[\bf Proof]
ByProposition \ref{pro:colrank_product_smaller_than_individual_colranks}, $\colrk(AA^*) \leq \colrk(A^*)$. Also,
we have $\colrk(A) = \colrk(AA^*)$ by Lemma \ref{lem:colrank_defact_sum_is_column_number} which implies that $\colrk(A) \leq \colrk(A^*)$. Then interchanging $A$ and $A^*$ and repeating the same argument, we have $\colrk(A^*) \leq \colrk(A)$. Therefore, we have the required result.
\end{proof}

Then by Lemma \ref{lem:original_conjugate_matrices_have_same_column_rank} and Theorem \ref{thm:original_adjoint_matrix_have_same_column_rank}, we have the following definition.

\begin{definition}[rank\index{rank!matrix}]\label{def:rank_matrix}
Let $A\in M_n(\F)$. Then
\be
\rank(A) := \colrk(A) = \colrk(A^*) = \colrk(A^T) = \rowrk(A).
\ee

In other words, the row rank and column rank are consistent, denoted as the rank of matrix.
\end{definition}

\begin{remark}
Clearly, $\rank(I_n) = n$.
\end{remark}


\begin{proposition}\label{pro:rank_smaller_than_row_column_number}
Let $A\in M_{m,n}(\F)$. Then $\rank(A) \leq \min\bra{m,n}$.
\end{proposition}

\begin{proof}[\bf Proof]
Direct result from Definition \ref{def:rank_matrix} as $\sR(A)\subseteq \F^m$ and $\sR(A^T)\subseteq \F^n$.
\end{proof}


\begin{proposition}\label{pro:rank_of_block_matrix_of_two_blocks}
\ben
\item [(i)] Let $A\in M_{m,n}(\F),B\in M_{m,k}(\F)$. Then
\be
\rank\bepm A & 0_{m\times k} \eepm = \rank (A),\qquad \rank(A),\rank(B) \leq \rank \bepm A & B \eepm = \rank \bepm B & A \eepm.
\ee

\item [(ii)] Let $A\in M_{m,n}(\F),B\in M_{k,n}(\F)$. Then
\be
\rank\bepm A \\ 0_{k\times n} \eepm = \rank (A),\qquad \rank(A),\rank(B) \leq \rank \bepm A \\ B \eepm = \rank \bepm B \\ A \eepm.
\ee
\een
\end{proposition}

\begin{proof}[\bf Proof]
(i) is directly from definitions of range space and column rank. (ii) is from (i) and Theorem \ref{thm:original_adjoint_matrix_have_same_column_rank}.
\end{proof}




\begin{theorem}\label{thm:rank_defect_theorem_matrix}
Let $A\in M_{m,n}(\F)$. Then
\ben
\item [(i)] $\rank (A) + \defect(A) = n$.
\item [(ii)] $\rank(A) = \rank(A^*) = \rank(AA^*) = \rank(A^*A)$.
\item [(iii)] $\defect(A)  =  \defect(A^*A)  = \defect(A^*) + n - m = \defect(AA^*) + n- m$.
\een
\end{theorem}

\begin{remark}
Note that $\rank (A) + \defect(A) = n$ is another version of rank-nullity theorem (Theorem \ref{thm:rank_nullity})

Furthermore, we can have
\be
\rank(A) = \rank(A^*) = \rank(A^*A) = \rank(AA^*) = \rank(AA^*A) = \rank(A^*AA^*)
\ee
from the remark of Theorem \ref{thm:adjoint_range_space_null_space_orthogonal_complment}.
\end{remark}

\begin{proof}[\bf Proof]
Direct result from Lemma \ref{lem:colrank_defact_sum_is_column_number} and Theorem \ref{thm:original_adjoint_matrix_have_same_column_rank}.
\end{proof}


\begin{corollary}
Let $A\in M_n(\F)$. Then for $k\in \Z^+$,
\be
\rank(A^k) \leq \rank(A),\qquad \defect(A^k) \geq \defect(A)
\ee
\end{corollary}

\begin{proof}[\bf Proof]
Direct result from Proposition \ref{pro:colrank_product_smaller_than_individual_colranks} and Theorem \ref{thm:rank_defect_theorem_matrix}.
\end{proof}


\begin{proposition}\label{pro:matrix_sum_rank_leq_sum_of_ranks}
Let $A,B\in M_{m,n}(\F)$. Then
\be
\rank(A+B) \leq \rank(A) + \rank(B).
\ee
\end{proposition}

\begin{proof}[\bf Proof]
By definition of column space of matrix and Theorem \ref{thm:span_of_set_is_linear_combination_of_elements_of_set},
\be
\sR(A+B) = \linspan\bra{A_1+B_1,\dots, A_n + B_n} \subseteq \linspan\bra{A_1,\dots,A_n,B_1,\dots,B_n} = \sR(A) + \sR(B).
\ee

This implies that $\sR(A) + \sR(B)$ is subspace and
\beast
\rank(A+B) = \dim\sR(A+B) & \leq & \dim\brb{\sR(A) + \sR(B)} \\
& = & \dim\sR(A) + \dim\sR(B) - \dim\brb{\sR(A)\cap \sR(B)} \\
 & \leq & \dim\sR(A) + \dim\sR(B) = \rank(A) + \rank(B).
\eeast
by Theorem \ref{thm:modular_equation_vector_space}.
\end{proof}


\begin{proposition}\label{pro:rank_of_block_diagonal_matrix}
Let $A\in M_{m,n}(\F)$ and $B\in M_{p,q}(\F)$.%_{p\times n}
\be
\rank\bepm A & 0 \\ 0 & B \eepm = \rank(A) + \rank(B).
\ee
\end{proposition}

\begin{remark}
With the similar argument, we also have $\rank\bepm 0 & A \\ B & 0 \eepm = \rank(A) + \rank(B)$.

In general, by applying the proposition inductively, we can have for $A_i\in M_{m_i,n_i}(\F)$, $i=1,\dots,k$,
\be
\rank\bepm A_1 & 0 & \dots & 0 \\ 0 & A_2 & \dots & 0 \\ \vdots & \vdots & \ddots & \vdots \\ 0 & 0 & \dots & A_k \eepm = \sum^k_{i}\rank(A_i).
\ee
\end{remark}

\begin{proof}[\bf Proof]
Let for any $v \in \sR\bepm A & 0 \\ 0 & B \eepm$, we can find $u\in \sR\bepm A \\ 0 \eepm$ and $w \in \sR\bepm 0 \\ B \eepm$ such that $v = u+w$. Therefore, we have that
\be
\sR\bepm A & 0 \\ 0 & B \eepm = \sR\bepm A \\ 0_{p\times n} \eepm + \sR\bepm 0_{m\times q} \\ B \eepm.
\ee

Also, we can see that
\be
\sR\bepm A \\ 0_{p\times n} \eepm \cap \sR\bepm 0_{m\times q} \\ B \eepm = \bra{0_{(m+p)\times 1}}
\ee

Then by corollary \ref{cor:dimension_sum_intersection_zero_iff_complement_subspaces} we have that
\beast
\rank\bepm A & 0 \\ 0 & B \eepm & = & \dim\sR\bepm A & 0 \\ 0 & B \eepm = \dim\sR\bepm A \\ 0 \eepm + \dim\sR\bepm 0 \\ B \eepm \\
& = & \rank\bepm A \\ 0 \eepm + \rank\bepm 0 \\ B \eepm = \rank(A) + \rank(B).
\eeast

Note that the last equality is due to Proposition \ref{pro:rank_of_block_matrix_of_two_blocks}.
\end{proof}


\begin{proposition}\label{pro:rank_of_block_triangular_matrix}
Let $A\in M_{m,n}(\F),B\in M_{p,q}(\F),C\in M_{m,q}(\F)$.
\be
\rank(A) + \rank(B) \leq \rank\bepm A & C \\ 0 & B \eepm \leq \begin{cases} \rank(A)+ \rank\bepm C \\ B \eepm \leq \rank(A) + q\\ \rank(B) + \rank\bepm A & C \eepm \leq \rank(B) + m \end{cases} .
\ee
\end{proposition}

\begin{remark}
In general, by applying the proposition inductively, we can have for $A_i\in M_{m_i,n_i}(\F)$, $i=1,\dots,k$,
\be
\sum^k_{i}\rank(A_i) \leq \rank\bepm A_1 & C_{12} & \dots & C_{1,k-1} & C_{1k} \\ 0 & A_2 & \dots & C_{2,k-1} & C_{2k} \\ \vdots & \vdots & \ddots & \vdots & \vdots \\ 0 & 0 & \dots & A_{k-1} & C_{k-1,k} \\ 0 & 0 & \dots & 0 &  A_k \eepm \leq \min_{1\leq i\leq k} \bra{\rank(A_i) + \sum^{i-1}_{j=1} m_j + \sum^{k}_{j=i+1} n_j}.
\ee
\end{remark}

\begin{proof}[\bf Proof]
Clearly, by Proposition \ref{pro:matrix_sum_rank_leq_sum_of_ranks} and Proposition \ref{pro:rank_of_block_matrix_of_two_blocks} and Proposition \ref{pro:rank_smaller_than_row_column_number},
\beast
\rank\bepm A & C \\ 0 & B \eepm & \leq & \rank\bepm A & 0 \\ 0 & 0 \eepm + \rank\bepm 0 & C \\ 0 & B \eepm = \rank\bepm A \\ 0 \eepm + \rank\bepm C \\ B \eepm \\
 & = & \rank(A) + \rank\bepm C \\ B \eepm \leq \rank(A) + q,
\eeast
as required. Similarly, we have $\rank\bepm A & C \\ 0 & B \eepm \leq \rank(B) + m$.

Now let $\rank(A)= a \leq n$ and $\rank(B) = b \leq q$. Then we can find $a$ columns of $\bepm A & C \\ 0 & B \eepm$, say $M_1,\dots,M_a$ corresponding to $a$ linearly independent columns of $A$. That is,
\be
M_k = \bepm A_{i_k} \\ 0 \eepm,\qquad k=1,\dots,a.
\ee
where $A_{i_k}$ is $i_k$th column of $A$. Note that these $a$ columns of linearly independent columns must exists, otherwise the rank of $A$ (dimension of column space $\sR(A)$) is smaller than $a$. Similarly, we can find $b$ columns of $M$, say $M_{a+1},\dots, M_{a+b}$ corresponding to $b$ linearly independent columns of $B$. That is,
\be
M_{a+k} = \bepm C_{j_k} \\ B_{j_k} \eepm,\qquad k=1,\dots, b.
\ee

Thus we can form a set of $a+b$ columns $M_1,\dots, M_{a+b}$ of $\bepm A & C \\ 0 & B \eepm$. If $\alpha_k\in \F$ and
\beast
0 & = &  \sum^{a+b}_{k=1} \alpha_k M_k = \alpha_1 M_1 + \dots + \alpha_a M_a + \alpha_{a+1} M_{a+1} + \dots +  \alpha_{a+b}M_{a+b} \\
\bepm 0_{m\times 1} \\ 0_{p\times 1} \eepm & = & \alpha_1 \bepm A_{i_1} \\ 0 \eepm + \dots + \alpha_a \bepm A_{i_a} \\ 0 \eepm + \alpha_{a+1} \bepm C_{j_1} \\ B_{j_1} \eepm + \dots + \alpha_{a+b} \bepm C_{j_b} \\ B_{j_b} \eepm.
\eeast

Since $B_{j_1},\dots, B_{j_b}$ are linearly independent and $0_{p\times 1} = \alpha_{a+1} B_{j_1} + \dots + \alpha_{a+b} B_{j_b}$, we have
\be
\alpha_{a+1} = \dots = \alpha_{a+b} = 0.
\ee
which implies that
\be
0_{m\times 1} = \alpha_1 A_{i_1} + \dots + \alpha_a A_{i_a}.
\ee

Since $A_{i_1},\dots, A_{i_a}$ are linearly independent, we have that
\be
\alpha_1 = \dots = \alpha_{a} = \alpha_{a+1} = \dots = \alpha_{a+b} = 0 \ \ra\ M_{a+1},\dots, M_{a+b}\text{ are linearly independent.}
\ee

Then by Lemma \ref{lem:finite_basis_linearly_independent_spanning_equivalent},
\be
\dim\sR\bepm A & C \\ 0 & B \eepm \geq a+b \ \ra\ \rank(A)+ \rank(B) \leq \rank\bepm A & C \\ 0 & B \eepm.
\ee
\end{proof}


\begin{proposition}[Sylvester's inequality]\label{pro:sylvester_inequality_rank}
Let $A\in M_{m,n}(\F)$ and $B\in M_{n,p}(\F)$. Then
\be
\rank(A) + \rank(B) \leq \rank(AB) + n.
\ee
\end{proposition}

\begin{proof}[\bf Proof]
{\bf Approach 1.} By Proposition \ref{pro:rank_of_block_diagonal_matrix} and Proposition \ref{pro:rank_of_block_triangular_matrix},
\beast
\rank(A) + \rank(B) = \rank\bepm B & 0 \\ 0 & A \eepm \leq \rank\bepm B & I_n \\ 0 & A \eepm = \rank\bepm 0 &  I_n \\ I_m & A \\ \eepm \bepm -AB & 0  \\ B & I_n  \eepm.
\eeast

Then by Proposition \ref{pro:colrank_product_smaller_than_individual_colranks}, Proposition \ref{pro:matrix_sum_rank_leq_sum_of_ranks} and Proposition \ref{pro:rank_of_block_matrix_of_two_blocks},
\beast
\rank(A) + \rank(B) & \leq &  \rank \bepm -AB & 0  \\ B & I_n  \eepm \leq  \rank \bepm -AB & 0  \\ 0 & 0  \eepm +  \rank \bepm 0 & 0  \\ B & I_n  \eepm \\
& = &\rank \bepm -AB & 0  \eepm  + \rank \bepm B & I_n  \eepm = \rank(AB)  + \rank \bepm I_n & B  \eepm \\
& = & \rank(AB) + n.
\eeast

{\bf Approach 2.} \footnote{Meyer's book, p210, rank of a product, $\rank(AB) = \rank(B) - \dim\brb{\sN(A)\cap \sR(B)}$.}
\end{proof}

\begin{proposition}[Frobenius inequality]
Let $A\in M_{m,n}(\F)$, $B \in M_{n,p}(\F)$ and $C \in M_{p,q}(\F)$. Then
\be
\rank(AB) + \rank(BC) \leq \rank(B) + \rank(ABC)
\ee
\end{proposition}

\begin{proof}[\bf Proof]
Approach 1. By Proposition \ref{pro:rank_of_block_diagonal_matrix}, Proposition \ref{pro:rank_of_block_triangular_matrix} and Proposition \ref{pro:colrank_product_smaller_than_individual_colranks},
\beast
\rank(AB) + \rank(BC) & \leq & \rank\bepm BC & B \\ 0 & AB \eepm = \rank \bepm B & BC \\ AB & 0 \eepm = \rank  \bepm B & 0 \\ AB & -ABC \eepm  \bepm I & C  \\ 0 & I  \eepm \\
& \leq &  \rank  \bepm B & 0 \\ AB & -ABC \eepm =  \rank  \bepm I & 0 \\ A & -I \eepm \bepm B & 0 \\ 0 & ABC \eepm \leq \rank\bepm B & 0 \\ 0 & ABC \eepm\\
& = & \rank(B) + \rank(ABC).
\eeast

Approach 2. Applying the proposition\footnote{prop needed.}, we have
\be
\rank(AB) = \rank(B) - \dim\brb{\sN(A) \cap \sR(B)}, \qquad \rank(ABC) = \rank(BC) - \dim\brb{\sN(A) \cap \sR(BC)}.
\ee

Since $\sR(BC) \subseteq \sR(B)$ by Lemma \ref{lem:range_null_space_product}, we have
\be
\rank(BC) - \rank(ABC) = \dim\brb{\sN(A) \cap \sR(BC)} \leq \dim\brb{\sN(A) \cap \sR(B)} = \rank(B) - \rank(AB)
\ee
as required.
\end{proof}




\begin{proposition}
Let $A \in M_{m,n}(\F)$. Then
\be
\rank\brb{I_m - AA^T} - \rank\brb{I_n - A^TA} = m-n.
\ee
\end{proposition}

\begin{proof}[\bf Proof]
Clearly, the required result is equivalent to
\be
\rank\brb{I_m - AA^T} + n = \rank\brb{I_n - A^TA} + m.
\ee

So we construct block matrix
\be
M = \bepm I_m - AA^T & 0 \\ 0 & I_n \eepm.
\ee%\rank\brb{I_m - AA^T} + n & = &

Thus, by Proposition \ref{pro:rank_of_block_diagonal_matrix} and Proposition \ref{pro:colrank_product_smaller_than_individual_colranks},
\beast
\rank\brb{I_m - AA^T} + \rank(I_n) & = & \rank\bepm I_m - AA^T & 0 \\ 0 & I_n \eepm = \rank \bepm I_m & -A \\ 0 & I_n \eepm \bepm I_m - AA^T & A \\ 0 & I_n \eepm \leq \rank \bepm I_m - AA^T & A \\ 0 & I_n \eepm \\
& = & \rank \bepm I_m & A \\ A^T & I_n \eepm \bepm I_m & 0 \\ -A^T & I_n \eepm \leq \rank \bepm I_m & A \\ A^T & I_n \eepm  = \rank \bepm I_m & 0 \\ A^T & I_n \eepm \bepm I_m & A \\ 0 & \ I_n - A^TA \eepm \\
& \leq &  \rank \bepm I_m & A \\ 0 & \ I_n - A^TA \eepm = \rank \bepm I_m & 0 \\ 0 & \ I_n - A^TA \eepm \bepm I_m & A \\ 0 & \ I_n \eepm \leq \rank \bepm I_m & 0 \\ 0 & \ I_n - A^TA \eepm  \\
& = & \rank(I_m) + \rank(I_n - A^T A) = m + \rank(I_n - A^T A).
\eeast

Similarly, we have $\rank(I_n - A^T A) + m\leq \rank\brb{I_m - AA^T} + n$ which implies that the required result.
\end{proof}





\subsection{Relationship between rank, invertibility and singularity}

\begin{proposition}
Let $A\in M_{m,n}(\F)$. Then the following statements in each session are equivalent.
\ben
\item [(i)] \ben
\item [(a)] The columns of $A$ form a linear independent set.
\item [(b)] $\sN(A) = \bra{0}$.
\item [(c)] $\rank(A) = n$.
\een

\item [(ii)] \ben
\item [(a)] The rows of $A$ form a linear independent set.
\item [(b)] $\sN(A^T) = \bra{0}$.
\item [(c)] $\rank(A) = m$.
\een

\item [(iii)] \ben
\item [(a)] $A$ is nonsingular.
\item [(b)] The columns of $A$ form a linear independent set.
\item [(c)] The rows of $A$ form a linear independent set.
\een
\een
\end{proposition}




\subsection{Alternative approach of rank by equivalent matrices}



\begin{lemma}\label{lem:matrix_equivalent_to_identity}
For any matrix $A\in M_{m,n}(\F)$, it is equivalent to $B = \bepm I_r & 0\\ 0 & 0\eepm$ for some $r$.

$B$ is called rank normal form\index{rank normal form} of $A$.
\end{lemma}

\begin{remark}
The number $r$ is actually the rank (row rank and column rank) of $A$ (see Definition \ref{def:column_rank_matrix}).
\end{remark}

\begin{proof}[\bf Proof]
Let $A \in M_{m,n}(\F)$. Define $\alpha : \F^n \to \F^m$, $x \mapsto Ax$. With respect to the standard bases of $\F^n$, $\F^m$, $\alpha$ has matrix $A$. By Lemma \ref{lem:linear_map_equivalent_to_identity} and Lemma \ref{lem:change_of_matrix_basis}, $A$ is equivalent to $B = \bepm I_r & 0\\ 0 & 0\eepm$ for some $r$.
\end{proof}

\begin{proposition}\label{pro:equivalent_rank}
The matrices $A,A'\in M_{m,n}(\F)$ are equivalent if and only if $\colrk(A) = \colrk(A')$.
\end{proposition}

\begin{proof}[\bf Proof]
($\ra$). Assume $A,A'$ are equivalent such that $A' = Q^{-1}AP$ with $Q, P$ invertible. Let $\alpha : \F^n \to \F^m$, $x \mapsto Ax$. Then $\alpha$ with respect to standard bases $B,C$ has matrix $A$. Let $B'$ be the set of columns of $P$, let $C'$ be the set of columns of $Q'$. Then $[\alpha]_{B',C'} = Q^{-1}AP$ by Lemma \ref{lem:change_of_matrix_basis} since $P$ and $C$ are the change of basis matrices from $B$ to $B'$ and $C$ to $C'$, respectively.

So $[\alpha]_{B',C'} = Q^{-1}AP =A'$ and $[\alpha]_{B,C} = A$. Then by Lemma \ref{lem:rank_linear_map_matrix}, $\colrk(A') = \colrk(\alpha) = \colrk(A)$.

($\la$). We have that $A$ and $A'$ are equivalent to
\be
\bepm I_r & 0\\ 0 & 0\eepm\quad \text{ and }\quad \bepm I_{r'} & 0\\ 0 & 0\eepm
\ee
for some $r$ and $r'$, respectively by Lemma \ref{lem:matrix_equivalent_to_identity}. By the first part, $\colrk(A) = r$, $\colrk(A') = r'$. Since $\colrk(A) = \colrk(A')$ we have $r = r'$ so $A$ and $A'$ are equivalent by transitivity of the equivalence relation.
\end{proof}



\begin{theorem}\label{thm:rank_matrix_transpose}
For $A \in M_{m,n}(\F)$, $\rowrk(A) = \colrk(A)$. That is,  $\colrk(A^T) = \colrk(A)$.
\end{theorem}

\begin{remark}
This is alternative proof of Definition \ref{def:rank_matrix}. Therefore, we can write $\rank$ for $\rowrk$ and $\colrk$.
\end{remark}

\begin{proof}[\bf Proof]
Let $A \in M_{m,n}(\F)$, let $r = \colrk(A)$. Then $A$ is equivalent to $\bepm I_r & 0\\ 0 & 0 \eepm_{m\times n}$ by Lemma \ref{lem:matrix_equivalent_to_identity}, so $\bepm I_r & 0\\ 0 & 0\eepm_{m\times n} = QAP$ for some invertible matrices $Q, P$ by Definition \ref{def:equivalent_matrix}. Consider the transpose, by Proposition \ref{pro:matrix_multiple_transpose},
\be
P^TA^TQ^T = \bepm I_r & 0\\ 0 & 0\eepm_{n\times m} .
\ee

So $\spa(A^T)$ is equivalent to $\bepm I_r & 0\\ 0 & 0\eepm_{n\times m}$. Visibly, $\colrk\bepm I_r & 0\\ 0 & 0\eepm_{n\times m} = r$, so
\be
\rowrk(A) = \colrk(A^T ) = \colrk\bepm I_r & 0\\ 0 & 0\eepm_{n\times m} = r = \colrk(A).
\ee
\end{proof}




%\section{Rank}

%\subsection{Rank}


\subsection{Alternative approach of rank from Gauss elimination, pivot}







%\begin{proposition}\label{pro:invertible_equivalent}
%Let $A\in M_n(\F)$. The following statements are equivalent:
%\ben
%\item [(i)] $A$ is invertible.
%    A is row-equivalent to the n-by-n identity matrix In.
%    A is column-equivalent to the n-by-n identity matrix In.
%    A has n pivot positions.
%    det A ≠ 0. In general, a square matrix over a commutative ring is invertible if and only if its determinant is a unit in that ring.
%    rank A = n.
%    The equation Ax = 0 has only the trivial solution x = 0 (i.e., Null A = {0})
%   The equation Ax = b has exactly one solution for each b in Kn.
%\item [(ii)] The columns of $A$ are linearly independent.
%    The columns of A span Kn (i.e. Col A = Kn).
%\een
%\end{proposition}

%\begin{proof}[\bf Proof]
%\footnote{need proof}
%\end{proof}


\subsection{Properties of rank}



\begin{theorem}\label{thm:invertible_full_rank}
If $A\in M_n(\F)$, then $A$ is invertible iff $\rank(A) = n$.%, i.e., the columns of $A$ are linearly independent.
\end{theorem}

\begin{proof}[\bf Proof]
($\ra$). If $A$ is invertible, then we have invertible matrix $P$ such that $AP = I_n$. This means $A$ is equivalent to $I_n$. Then by Proposition \ref{pro:equivalent_rank}, we have $\rank(A) =\rank(I_n) = n$.

($\la$). Approach 1. If $\rank(A)$, we have $A$ is equivalent to $I_n$ by Lemma \ref{lem:matrix_equivalent_to_identity}. Thus, there exist invertible matrices $P,Q$ such that $Q^{-1}AP = I_n$ by Definition \ref{def:equivalent_matrix}. Therefore, we have $A = QP^{-1}$ and $A^{-1} = PQ^{-1}$ by Proposition \ref{pro:inverse_matrix_property}. Thus, $A$ is invertible as
\be
AA^{-1} = QP^{-1}PQ^{-1} = QQ^{-1} = I = PP^{-1} = PQ^{-1}QP^{-1} = A^{-1}A.
\ee

Approach 2. Since $\rank(A) =n$, we have that the corresponding linear map $\alpha$ is surjective. Then by Corollary \ref{cor:same_dimension_linear_map_equivalent}, we have that the linear map is bijective (since $\dim V = n=m = \dim W$). Thus, the corresponding inverse matrix $A^{-1}$ exists and therefore $A$ is invertible.
\end{proof}

\begin{lemma}\label{lem:full_rank_linearly_independent}
Let $A\in M_n(\F)$. Then $\rank(A) = n$ iff the columns of $A$ are linearly independent.
\end{lemma}

\begin{proof}[\bf Proof]
($\ra$). If $\rank(A) = n$, we have that the columns of $A$ are linearly independent by Lemma \ref{lem:finite_basis_linearly_independent_spanning_equivalent} as the columns span the column space.

($\la$). If the columns of $A$ are linearly independent, then by Lemma \ref{lem:independent_spanning_basis}, the columns of $A$ is a basis. Furthermore, by definition of dimension (Definition \ref{def:dimension_vector_space}), $\rank(A) = n$.
\end{proof}

\begin{corollary}\label{cor:invertible_column_linearly_independent}
If $A\in M_n(\F)$, then $A$ is invertible iff the columns of $A$ are linearly independent.
\end{corollary}

\begin{proof}[\bf Proof]
Direct result from Lemma \ref{thm:invertible_full_rank} and \ref{lem:full_rank_linearly_independent}.
\end{proof}


\begin{proposition}\label{pro:invertible_non_singular_equivalent}
A square matrix $A\in M_n(\F)$ is invertible if and only if $A$ is nonsingular.% (that is, $0\in \sigma(A)$ by Definition \ref{def:singularity_matrix}).
\end{proposition}

\begin{remark}
So singular matrix and invertible matrix are the same term for square matrices from now on.
\end{remark}
%
\begin{proof}[\bf Proof]
Let $A_1,\dots,A_n$ be the column of $A$, then for $0\neq x = (x_1,\dots,x_n)^T\in \F^n$ we have \beast
\text{$A$ is singular} & \lra & A x = 0 \ \lra\ A_1 x_1 + A_2 x_2 + \dots + A_n x_n = 0 \\
& \lra & \text{the columns of $A$ are linearly dependent}\qquad \text{(by Definition \ref{def:linearly_independent_vector})}\\
& \lra & A\text{ is not invertible}\qquad\qquad  \text{(by Corollary \ref{cor:invertible_column_linearly_independent})},
\eeast%\footnote{proof needed. needed definition of singular and the fact singularity equals to invertibility.}
as required.
\end{proof}






%\subsection{Singularity}


\begin{proposition}[rank equalities]\label{pro:rank_equalities}
\ben%\item [(i)] If $A\in M_{m,n}(\F)$, $\rank\brb{A^*} = \rank\brb{A^T} = \rank\brb{\ol{A}} = \rank(A)$.
\item [(i)] If $A\in M_m(\F)$ and $C\in M_n(\F)$ are nonsingular (invertible) and $B\in M_{m,n}(\F)$, then
\be
\rank\brb{AB} = \rank\brb{B} = \rank\brb{BC} = \rank\brb{ABC}.
\ee

That is, rank is unchanged upon left or right multiplication by a nonsingular matrix.

\item [(ii)] If $A,B\in M_{m,n}(\F)$, then $\rank\brb{A} = \rank\brb{B}$ if and only if there exist nonsingular $X\in M_m(\F)$ and $Y\in M_n(\F)$ such that $B = XAY$.

\item [(iii)] If $A\in M_{m,n}(\C)$, $\rank\brb{A^*A} = \rank\brb{A} = \rank\brb{A^*} = \rank\brb{AA^*}$.

\item [(iv)] If $A\in M_{m,n}(\F)$ has rank $k$, then we can write $A = XBY$ where $X\in M_{m,k}(\F)$, $Y \in M_{k,n}(\F)$ and $B\in M_k(\F)$ is nonsingular.

In particular, a matrix $A$ that has rank 1 may always be written in the form $A = xy^T$ for some $x\in \F^m$, $y\in \F^n$.
\een
\end{proposition}

\begin{remark}
Note that $\rank\brb{A^2} \neq \rank\brb{A}$ in general (comparing to (iv)) as we can see that
\be
A = \bepm 0 & 1 \\ 0 & 0 \eepm \ \ra \ A^2 = \bepm 0 & 0 \\ 0 & 0 \eepm \ \ra \ \rank\brb{A^2} = 0 \neq 1 = \rank\brb{A}.
\ee
\end{remark}

%\item [(i)] If $\rank(A)= m$, then (by Lemma \ref{lem:finite_basis_linearly_independent_spanning_equivalent}) there are at most $m$ linearly independent columns $A_1,\dots,A_m$ of $A$ such that
%\be
%\ol{\alpha_1} A_1 + \dots + \ol{\alpha}_m A_m = 0 \ \ra \ \alpha_1 = \alpha_2 = \dots = \alpha_m = 0.
%\ee

%This gives that for the corresponding columns $\ol{A}_1,\dots,\ol{A}_m$ of $\ol{A}$,
%\be
%\alpha_1 \ol{A}_1 + \dots + \alpha_m \ol{A}_m = 0 \ \ra \ \alpha_1 = \alpha_2 = \dots = \alpha_m = 0.
%\ee

%Thus, there are at most $m$ linearly independent columns in $\ol{A}$. Therefore, $m = \rank\brb{\ol{A}}$.


\begin{proof}[\bf Proof]
\ben
\item [(i)] If $\rank (B) = r$, then there exists invertible (nonsingular) matrices $P\in M_m(\F),Q\in M_n(\F)$ such that
\be
B = P\bepm I_r & 0 \\ 0 & 0 \eepm Q = P\Lambda Q
\ee
by Proposition \ref{pro:equivalent_rank}. Thus, $ABC = AP \Lambda QC$ with $AP$ and $QC$ are both invertible (by the fact that the product of invertible matrices is still invertible). Thus, $ABC$ is equivalent to $\Lambda$. By Proposition \ref{pro:equivalent_rank} again, we have $\rank\brb{ABC} = r$.

\item [(ii)] (see \cite{Horn_Johnson_1990}.$P_{13}$)\footnote{proof needed.}
\item [(iii)] \footnote{rank-nullity theorem needed.}
\item [(iv)] \footnote{proof needed.}
\een
\end{proof}


\section{Square Matrices}

\subsection{Basic square matrices}

%, Hermitian matrices and symmetric and anti-symmetic matrices

\begin{definition}[symmetric matrix\index{symmetric!matrix}]\label{def:symmetric_matrix}
A square $n\times n$ matrix $A = (a_{ij})$ is symmetric if
\be
A = A^T,\quad \text{i.e., } a_{ij} = a_{ji}.
\ee
\end{definition}

\begin{definition}[anti-symmetric matrix\index{anti-symmetric!matrix}]\label{def:antisymmetric_matrix}
A square $n\times n$ matrix $A = (a_{ij})$ is anti-symmetric if
\be
A = -A^T,\quad \text{i.e., } a_{ij} = -a_{ji}.
\ee
\end{definition}

\begin{remark}
For an anti-symmetric matrix, $a_{11} = -a_{11}$, i.e., $a_{11} = 0$. Similarly we deduce that all the diagonal elements of anti-symmetric matrices are zero, i.e., $a_{11} = a_{22} = \dots = a_{nn} = 0$.
\end{remark}



\begin{definition}[Hermitian matrix\index{Hermitian matrix}]\label{def:hermitian_matrix}
A square $n\times n$ complex matrix $A = (a_{ij})$ is Hermitian if
\be
A = A^\adjoint,\quad \text{i.e., } a_{ij} = \ol{a_{ji}}.
\ee
\end{definition}

\begin{remark}
Hermitian matrix is also called self-adjoint matrix.
\end{remark}

\begin{definition}[skew-Hermitian matrix\index{skew-Hermitian matrix}]\label{def:skew_hermitian_matrix}
A square $n\times n$ complex matrix $A = (a_{ij})$ is skew-Hermitian if
\be
A = -A^\adjoint,\quad \text{i.e., } a_{ij} = -\ol{a_{ji}}.
\ee
\end{definition}

\begin{remark}
All the diagonal elements of Hermitian and skew-Hermitian matrices are real and pure imaginary respectively.
\end{remark}

\begin{example}
\ben
\item [(i)] A symmetric $3\times 3$ matrix $S$ has the form
\be
S = \bepm a & b & c \\ b & d & e \\ c & e & f \eepm,
\ee
i.e., it has six independent elements.

\item [(ii)] An anti-symmetric $3\times 3$ matrix $A$ has the form
\be
A = \bepm 0 & a & -b \\ -a & 0 & c \\ b & -c & 0 \eepm,
\ee
i.e., it has three independent elements.
\een
\end{example}

%\subsection{Orthogonal and unitary matrices}

\begin{definition}[orthogonal matrix\index{orthogonal matrix}]\label{def:orthogonal_matrix}
$A\in M_n(\F)$ is orthogonal matrix if
\be
AA^T = I = A^TA,
\ee
i.e., if $A$ is invertible and $A^{-1} = A^T$.
\end{definition}

\begin{remark}
Note that $\det A = \pm 1$ for orthogonal matrix $A$.
\end{remark}

\begin{definition}[unitary matrix\index{unitary matrix}]\label{def:unitary_matrix}
$A \in M_n(\C)$ is said to be unitary if its adjoint matrix (or Hermitian conjugate matrix) is equal to its inverse, i.e., if
\be
U^* = U^{-1}.
\ee

In other words,
\be
U^* U = I = UU^*.
\ee
\end{definition}

\begin{remark}
Unitary matrices are to complex matrices what orthogonal matrices are to real matrices. Similar properties to those above for orthogonal matrices also hold for unitary matrices when references to real scalar products are replaced by references to complex scalar products\footnote{need definition}.
\end{remark}

%\subsection{Trace}

\begin{definition}[trace\index{trace!matrix}]\label{def:trace_matrix}
The trace of a square $n\times n$ matrix $A = (a_{ij})$ ($A\in M_n(\F)$) is equal to the sum of the diagonal elements, i.e.,
\be
\tr A = \sum^n_{i=1}a_{ii}.
\ee
\end{definition}

\begin{remark}
Note that $\tr:M_n(\F)\to \F$ is linear.
\end{remark}

\begin{proposition}\label{pro:trace_change_order}
Suppose that $A\in M_{m,n}(\F)$ and $B \in M_{n,m}(\F)$. Then $\tr(AB) = \tr(BA)$.
\end{proposition}

\begin{proof}[\bf Proof]
We have
\be
\tr(AB) = \sum_i \brb{\sum_j a_{ij}b_{ji}} = \sum_j \brb{\sum_i b_{ji}a_{ij}} = \tr(BA).
\ee
\end{proof}


\begin{proposition}
Let $A\in M_{m,n}(\F)$. Then
\be
\tr\brb{A^* A} = 0 \ \lra \ A = 0.
\ee
\end{proposition}

\begin{remark}
Clearly, if $A\in M_{m,n}(\R)$. Then
\be
\tr\brb{A^T A} = 0 \ \lra \ A = 0.
\ee
\end{remark}

\begin{proof}[\bf Proof]
($\la$). It is obvious that $A^*A = 0$ and $\tr\brb{A^* A} = 0$.

($\ra$). If $\tr\brb{A^* A} = 0$ we have that for any $i,j$,
\be
(A^*A)_{ij} = \sum_{k=1}^m (A^*)_{ik}A_{kj} = \sum_{k=1}^m \ol{a_{ki}}a_{kj}
\ee

Then
\be
0 = \tr\brb{A^* A} = \sum^n_{p=1} \brb{A^* A}_{pp} = \sum^n_{p=1}\sum_{k=1}^m \ol{a_{kp}}a_{kp} = \sum_{k,p}\abs{a_{kp}}^2 \ \ra\ \forall k,p,\ \ a_{kp} = 0
\ee
which implies that $A = 0$.
\end{proof}


\begin{definition}[principal submatrix\index{principal submatrix}]\label{def:principal_submatrix}
Recalling the definition of submatrix, the submatrix $A_{\alpha,\alpha}$ of $A\in M_n(\F)$ with $\alpha \subseteq \bra{1,\dots,n}$ is called a principal submatrix of $A$ and is abbreviated $A_\alpha$.

If $\alpha = \bra{1,2,\dots,k}$, we call $A_k := A_\alpha$ the leading principal submatrix\index{principal submatrix!leading}.
\end{definition}




\subsection{Determinant}

\begin{definition}[determinant\index{determinant!matrix}]\label{def:determinant_matrix}
For $A \in M_n(\F)$ define
\be
\det A =\sum_{\sigma \in S_n} \ve(\sigma)a_{1,\sigma(1)} \dots a_{n,\sigma(n)} = \sum_{\sigma \in S_n} \ve(\sigma) \prod^n_{i=1} a_{i,\sigma(i)}
\ee
where $\ve(\sigma)$ is sign of permutation (see Definition \ref{def:sign_permutation}) and it is homomorphism by Theorem \ref{thm:sgn_homomorphism}.

Writing $A_i$ for the $i$th column of a matrix $A$ we have $A = (A_1, \dots,A_n)$, so we can think of $A$ is an $n$-tuple of columns in $\F^n$. Write $\bra{e_1, \dots, e_n}$ for the standard basis of $\F^n$.
\end{definition}

\begin{remark}
It is easy to see that $\det I = 1$.
\end{remark}

%Recall that $S_n$ is the group of all permutations of $\bra{1, \dots, n}$. The elements are permutations, the group operation is composition of permutations $(\sigma \circ \tau )(j) = \sigma (\tau (j))$. Any $\sigma$ can be written as a product of transpositions.
%\be
%\ve(\sigma) =\left\{ \ba{ll}
%+1\quad\quad & \text{if \# permutations is even,}\\
%-1 & \text{if \# permutations is odd.}
%\ea\right.
%\ee
%$\ve : S_n \to \bra{+1,-1}$ is a homomorphism.

\begin{example}
For $2\times 2$ matrix $A = \bepm a_{11} & a_{12} \\ a_{21} & a_{22}\eepm$, we have $\det A = a_{11} a_{22} - a_{12}a_{21}$.

For $3\times 3$ matrix $A = \bepm a_{11} & a_{12} & a_{13} \\ a_{21} & a_{22} & a_{23} \\ a_{31} & a_{32} & a_{33}\eepm$, we have
\be
\det A = a_{11} a_{22}a_{33} + a_{12}a_{12}a_{31} + a_{13}a_{32}a_{21} - _{11}a_{23}a_{32} - a_{12}a_{21}a_{33} - a_{13}a_{22}a_{31}.
\ee
\end{example}

\begin{definition}[volume form\index{volume form}]\label{def:volumn_form}
The function $d : \F^n \times  \dots \times \F^n \to\F$ is a volumne form on $\F^n$ if
\ben
\item [(i)] it is multilinear, i.e.
\beast
d(v_1, \dots, \lm_iv_i, \dots, v_n) & = & \lm d(v_1, \dots, v_i, \dots, v_n)\\
d(v_1, \dots, v_i + v'_i , \dots, v_n) & = & d(v_1, \dots, v_i, \dots, v_n) + d(v_1, \dots, v'_i, \dots, v_n)
\eeast
\item [(ii)] it is alternating, i.e. whenever $i\neq j$ and $v_i = v_j$ then $d(v_1, \dots, v_n) = 0$.
\een
\end{definition}

\begin{definition}[determinant function\index{determinant function}]\label{def:determinant_function}
$d$ is a determinant function if it is normalised volume form, i.e. $d(e_1, \dots, e_n) = 1$.
\end{definition}

\begin{lemma}\label{lem:swap_column_sign}
Swapping columns in a volume form changes the sign. For $i \neq j$,
\be
d(v_1, \dots, v_j , \dots, v_i, \dots, v_n) = -d(v_1, \dots, v_i, \dots, v_j , \dots, v_n).
\ee
\end{lemma}

\begin{proof}[\bf Proof]
By Definition \ref{def:volumn_form}.(ii),
\be
0 = d(v_1, \dots, v_i + v_j , \dots, v_i + v_j , \dots, v_n) = 0 + d(v_1, \dots, v_i, \dots, v_j , \dots, v_n) + d(v_1, \dots, v_j , \dots, v_i, \dots, v_n) + 0
\ee
by Definition \ref{def:volumn_form}.(i).
\end{proof}

\begin{corollary}\label{cor:volume_form_sign_permutation}
If $\sigma \in S_n$ and $d$ is a volume form on $\F^n$ then
\be
d(v_{\sigma(1)}, \dots, v_{\sigma(n)}) = \ve(\sigma)d(v_1, \dots, v_n)
\ee
for $v_1, \dots, v_n \in \F^n$. In particular,
\be
d(e_{\sigma(1)}, \dots, e_{\sigma(n)})= \ve(\sigma)d(e_1, \dots, e_n) = \ve(\sigma)
\ee
if $d$ is a determinant function.
\end{corollary}

\begin{proof}[\bf Proof]
Think about number of transpositions.
\end{proof}

\begin{theorem}\label{thm:volume_form_determinant}
If $d$ is a volume form on $\F^n$ and $A = (a_{ij}) = \brb{A_1, \dots,A_n} \in M_n(\F)$ then $d\brb{A_1, \dots,A_n} = (\det A) d(e_1, \dots, e_n)$.
\end{theorem}

\begin{proof}[\bf Proof]
\beast
d(A_1, \dots,A_n) & = & d\brb{\sum_{j_1} a_{j_1,1}e_{j_1}, A_2, \dots,A_n} = \sum_{j_1} a_{j_1,1}d\brb{e_{j_1} ,A_2, \dots,A_n}\\
& = & \sum_{j_1,j_2} a_{j_1,1} a_{j_2,2} d(e_{j_1} , e_{j_2} ,A_3, \dots,A_n) =  \sum_{j_1\neq j_2} a_{j_1,1} a_{j_2,2} d(e_{j_1} , e_{j_2} ,A_3, \dots,A_n)
\eeast
as $d$ is zero when $j_1 = j_2$ by Definition \ref{def:volumn_form}.(ii). Then for $j_1,\dots,j_n$ forming a permutation
\beast
d(A^{(1)}, \dots,A^{(n)}) = \sum_{j_1,\dots,j_n} a_{j_1,1} \dots a_{j_n,n} d(e_{j_1}, \dots, e_{j_n}) = \sum_{\sigma \in S_n} a_{\sigma(1)1} \dots a_{\sigma(n)n}\ve(\sigma)d(e_1, \dots, e_n) = (\det A)d(e_1, \dots, e_n).
\eeast
by Definition \ref{def:determinant_matrix}.
\end{proof}

\begin{theorem}\label{thm:determinant_determinant_function}
If we define $d : \F^n \times  \dots \times  \F^n \to \F,\ d(A_1, \dots,A_n) = \det A$ for $A = (A_1, \dots,A_n)$ then $d$ is a determinant function.
\end{theorem}

\begin{proof}[\bf Proof]
\ben
\item [(i)] $\prod^n_{j=1} a_{\sigma(j)j}$ is multilinear, hence so is the linear combination $\det A$.
\item [(ii)] We want to show that if $A_k = A_l$ with $k \neq l$ then $\det A = 0$. Write $\tau = (k\ l)$, a transposition in $S_n$.
\be
\det A = \sum_{\sigma \in S_n} \ve(\sigma) \prod_j a_{\sigma (j)j}
\ee

Reorder the sum, take each even permutation $\sigma$ followed by the odd permutation $\sigma \tau$ and note $\ve(\sigma ) = 1$, $\ve(\sigma \tau) = -1$. Thus
\be
\det A = \sum_{\sigma\text{ even}} \brb{\prod_j a_{\sigma (j)j} - \prod_j a_{\sigma \tau (j)j}} = 0
\ee
as each of the summands is zero since $a_{\sigma (k)k} = a_{\sigma (k)l}$, $a_{\sigma (l)k} = a_{\sigma (l)l}$ and
\be
a_{\sigma (1)1} \dots a_{\sigma (k)k} \dots a_{\sigma (l)l} \dots a_{\sigma (n)n} - a_{\sigma (1)1} \dots a_{\sigma (l)k} \dots a_{\sigma (k)l} \dots a_{\sigma (n)n} = 0.
\ee

\item [(iii)] For normal basis $\bra{e_1,\dots,e_n}$, $d(e_1,\dots,e_n) = \sum_{\sigma \in S_n} \ve(\sigma ) \prod^n_{j=1} \delta_{\sigma (j)j} = \ve(\iota) \cdot 1 = 1$.
\een

Thus, $d$ is a determinant function.
\end{proof}

\begin{proposition}\label{pro:determinant_matrix_property}
Let $A\in M_n(\F)$ and $\lm \in \F$. Then
\beast
\text{(i)} & &  \det \brb{A_1,\dots,\lm A_i, \dots, A_n} = \lm \det \brb{A_1,\dots,A_i,\dots,A_n} = \det A,\\
\text{(ii)} & &  \det \brb{A_1,\dots, A_i + A_{i}' , \dots, A_n} = \det\brb{A_1,\dots, A_i, \dots, A_n}  + \det\brb{A_1,\dots,A_{i}', \dots, A_n},\\
\text{(iii)} & &  \det \brb{A_1,\dots, A_i ,\dots, A_j , \dots, A_n} = -\det \brb{A_1,\dots, A_j ,\dots, A_i , \dots, A_n},\\
\text{(iv)} & & \text{If }A_i = A_j,\ \det \brb{A_1,\dots,A_i,\dots,A_j,\dots, A_n} = 0.
\eeast
\end{proposition}

\begin{proof}[\bf Proof]
This is direct result from Theorem \ref{thm:determinant_determinant_function}, Lemma \ref{lem:swap_column_sign}.
\end{proof}

\begin{proposition}\label{pro:determinant_transpose}
$\det A^T = \det A$.
\end{proposition}

\begin{proof}[\bf Proof]
If $\sigma \in S_n$ then $\prod^n_{j=1} a_{\sigma (j)j} = \prod^n_{j=1} a_{j\sigma (j)}$ as they contain the same factors but in a different order. Also, as $\sigma$ runs through $S_n$, so does $\sigma^{-1}$ and $\ve(\sigma^{-1}) = \ve(\sigma)$ (as $\sigma \sigma^{-1} = \iota$). Hence
\be
\det A = \sum_{\sigma \in S_n} \ve(\sigma )\prod^n_{j=1} a_{\sigma (j)j} = \sum_{\sigma \in S_n} \ve(\sigma^{-1} ) \prod^n_{j=1} a_{j\sigma^{-1}(j)} = \sum_{\sigma' \in S_n} \ve(\sigma' ) \prod^n_{j=1} a_{j\sigma' (j)} = \det A^T.
\ee
\end{proof}



%\begin{lemma}
%$\det$ is the unique multilinear alternating form in rows normalised at $I$.
%\end{lemma}

\begin{proposition}\label{pro:determinant_element_matrix}
For elementary matrices, the determinants are
\be
\det T_{ij} =  -1,\quad \det M_{i,\lm} = \lm,\quad \det C_{i,j,\lm} = 1.
\ee
\end{proposition}

\begin{proof}[\bf Proof]
Obvious from Definition \ref{def:elementary_matrix}.
\end{proof}


\begin{lemma}\label{lem:determinant_element_matrix_product}
If $E$ is an elementary matrix, then for any $n \times n$ matrix $A$,
\be
\det(AE) = \det A \det E = \det(EA).
\ee

Performing an elementary column or row operation on $A$ multiplies $\det A$ by the determinant of the corresponding elementary matrix.
\end{lemma}

\begin{proof}[\bf Proof]
By Proposition \ref{pro:determinant_element_matrix}, the determinants of the elementary matrices are %(see Definition \ref{def:elementary_matrix}) we have
\be
\det T_{ij} =  -1,\quad \det M_{i,\lm} = \lm,\quad \det C_{i,j,\lm} = 1
\ee
as we can see that the column and row operation have the same determinant by Proposition \ref{pro:determinant_transpose}.

Performing the corresponding elementary column or row operation multiplies $\det A$ by $-1, \lm, 1$, respectively. Then
\be
\det (AE) = \det A \det E = \det \brb{A^T} \det \brb{E^T} = \det \brb{A^TE^T} = \det\brb{(EA)^T} = \det(EA)
\ee
by Proposition \ref{pro:determinant_transpose}.
\end{proof}

\begin{theorem}\label{thm:matrix_invertible_determinant_non_zero}
Let $A\in M_n(\F)$. Then $A$ is invertible if and only if $\det A \neq 0$.
\end{theorem}

\begin{proof}[\bf Proof]
If $A$ is invertible then $A$ can be written as a product of elementary matrices by Lemma \ref{lem:invertible_product_elementary_matrices}, so $\det A$ is the product of the corresponding determinants, so $\det A \neq 0$.

If $A$ is singular (not invertible), we have that the columns of $A$ are linearly dependent (Corollary \ref{cor:invertible_column_linearly_independent}). Then we can obtain a 0 column as a non-trivial combination of columns of $A$, so using elementary column operations on $A$ we can obtain a matrix with a 0 column. Hence $\det A = 0$ by Lemma \ref{lem:determinant_element_matrix_product} as determinants of elementary matrices are no-zero.
\end{proof}

\begin{theorem}\label{thm:determinant_product}
If $A,B \in M_n(\F)$, then $\det(AB) = \det A \det B = \det(BA)$.
\end{theorem}

\begin{proof}[\bf Approach 1]
First fix $A$. Consider $d_A : \brb{B_1, \dots,B_n} \mapsto \det(AB)$ for $B = \brb{B_1, \dots,B_n}$. Note that $\det(AB) = d_A\brb{B_1, \dots,B_n} = d\brb{(AB)_1, \dots,(AB)_n}$ where $d$ is volume form by Theorem \ref{thm:determinant_determinant_function}. So $d_A$ is multilinear and alternating and $d_A$ is a volume form on $\F^n$. Hence
\be
\det(AB) = d_A\brb{B_1, \dots,B_n }= (\det B) d_A(e_1, \dots, e_n) = \det B \det A,
\ee
using Theorem \ref{thm:volume_form_determinant}.
\end{proof}

\begin{proof}[\bf Approach 2]
Expand by columns and by Corollary \ref{cor:volume_form_sign_permutation},
%as before.
\beast
\det(AB) & = & \det \brb{\sum^n_{j_1=1} b_{j_1,1} A_{j_1}, \dots, \sum^n_{j_n=1} b_{j_n,n}A_{j_n} } = \sum_{\sigma \in S_n} \brb{\prod^n_{j=1} b_{\sigma (j)j}} \det(A_{\sigma (1)}, \dots,A_{\sigma (n)})\\
& = & \sum_{\sigma \in S_n}\brb{\prod^n_{j=1} b_{\sigma (j)j}} \ve(\sigma ) \det(A_1, \dots,A_n) = \sum_{\sigma \in S_n}\brb{\prod^n_{j=1} b_{\sigma (j)j}} \ve(\sigma ) \det A = \det A \det B.
\eeast
\end{proof}

\begin{proof}[\bf Approach 3]
If $B$ is singular, so $\rank(B) <n$ by Theorem \ref{thm:invertible_full_rank} and let $\rank( B) = r$. Then by Lemma \ref{lem:matrix_equivalent_to_identity}, we have for some invertible matrices $P,Q$,
\be
B \sim \bepm I_r & 0 \\ 0 & 0 \eepm \ B = Q\bepm I_r & 0 \\ 0 & 0 \eepm P^{-1} = \bepm C_r & 0 \\ 0 & 0 \eepm\ \ra \ AB = A\bepm C_r & 0 \\ 0 & 0 \eepm = \bepm D_r & 0 \\ 0 & 0 \eepm
\ee
where $C_r$ and $D_r$ are $r\times r$ submatrices. Therefore, we see that $\rank(AB) \leq r \neq n$. So we have that $AB$ is singular as well by Theorem \ref{thm:invertible_full_rank}. %is $AB$ by definition of inverse. %That is if $AB$ is singular, then there exists an invertible matrix $P$ such that \be P(AB) = (AB)P = I \ \ra \ (PA)B =  \ee
Hence $\det B = 0 = \det(AB)$.

So assume $B$ is non-singular (invertible) and write it as a product of elementary matrices, $B = E_1 \dots E_k$ by Lemma \ref{lem:invertible_product_elementary_matrices}. Using Lemma \ref{lem:determinant_element_matrix_product},
\be
\det(AB) = \det(AE_1 \dots E_k) = \det A\det E_1 \dots \det E_k = \det A\det B
\ee
as required.
\end{proof}

\begin{corollary}\label{cor:determinant_inverse}
If $A$ is invertible then $\det A^{-1} = (\det A)^{-1}$.
\end{corollary}

\begin{proof}[\bf Proof]
As $A$ is invertible, $AA^{-1} = I$ so $(\det A)(\det A^{-1}) = \det I = 1$ and hence $\det A^{-1} = (\det A)^{-1}$.
\end{proof}

\begin{proposition}[derivative of a determinant]\label{pro:derivative_of_determinant}
If the entries in $A\in M_n(\F)$ are differentiable functions of $t$, then
\be
\frac{d\brb{\det A}}{dt} = \det D_1 + \det D_2 + \dots + \det D_n
\ee
where $D_i$ is identical to $A$ except that the entries in the $i$th row are replaced by their derivatives.%, i.e., [Di]k∗ =  Ak∗ if i = k, dAk∗/dt if i = k.
\end{proposition}

\begin{proof}[\bf Proof]
This follows directly from the definition of a determinant by writing (from definition of determinant)
\beast
\frac{d\brb{\det A}}{dt} & = & \sum_{\sigma \in S_n} \ve(\sigma) \frac{d\brb{\prod^n_{i=1} a_{i,\sigma(i)}}}{dt} = \sum_{\sigma \in S_n} \ve(\sigma) \sum^n_{j=1} \frac{d a_{j,\sigma(j)}}{dt} \prod_{i\neq j} a_{i,\sigma(i)}
\\
& = & \det D_1 + \det D_2 + \dots + \det D_n
\eeast
where $D_i$ is as assumed.
\end{proof}


Consider the system of linear equations $Ax = b$ with $m$ equations and $n$ unknowns. Here $A$ is an $m\times n$ matrix, $b$ is a column vector in $\F^m$. This has a solution if $\rank (A) = \rank(A| b)$. The solution is unique if and only if $n = \rank A = \rank(A| b)$, then the solution is $x = A^{-1}b$. To solve this equation, use Gaussian elimination\footnote{need details}. In the case $m = n$, there is another method.

\begin{lemma}[Cramer's rule\index{Cramer's rule!matrix}]\label{lem:cramer_rule}
If $A \in M_n(\F)$ is non-singular then $Ax = b$ has $x = (x_1, \dots, x_n)^T$ with $x_i = \frac 1{\det A} \det A_{ib}$ for $i = 1, \dots, n$ as its unique solution, where $A_{ib}$ is the matrix obtained from $A$ by deleting column $i$ and inserting $b$.
\end{lemma}

\begin{proof}[\bf Proof]
Assume $x$ is the solution of $Ax = b$. Then
\beast
\det A_{ib} & = &  \det\brb{A_1, \dots,A_{i-1}, b,A_{i+1}, \dots,A_n} = \det\brb{A_1, \dots,A_{i-1}, Ax ,A_{i+1}, \dots,A_n}  \\
& = & \sum_{j=1} x_j \det(A_1, \dots,A_{i-1},A_j,A_{i+1}, \dots,A_n) = x_i \det(A_1, \dots,A_{i-1},A_i,A_{i+1}, \dots,A_n) = x_i\det A,
\eeast
so $x_i = \frac 1{\det A} \det A_{ib}$ for $i = 1, \dots, n$.
\end{proof}

\begin{corollary}
If $A \in M_n(\Z)$ with $\det A = \pm 1$ and if $b \in \Z^n$ then we can solve $Ax = b$ over $\Z$.
\end{corollary}

\begin{remark}
The solution over $\Z$ is guaranteed since $\det A = \pm 1$.
\end{remark}





\subsection{Adjugate matrices}

\begin{definition}[minor\index{minor!matrix}]\label{def:minor_matrix}
Let $A \in M_n(\F)$. Write $\wh{A}_{ij}$ for the $(n - 1) \times (n - 1)$ matrix obtained from $A$ by deleting row $i$ and column $j$. Let $M_{ij} = \det \wh{A}_{ij}$. We call $M_{ij}$ the
$(i,j)$-minor of $A$.

More generally, if $\alpha, \beta \subseteq \bra{1,\dots,n}$ have the same size, then the $(\alpha,\beta)$-minor of $A$ is the determinant of submatrix $\wh{A}_{\alpha,\beta}$ (see Definition
\ref{def:submatrix}). If $\alpha = \beta$, we call $\det\brb{\wh{A}_\alpha}$ the principal minor\index{principal minor}. By convention, the $n\times n$ principal minor is 1, i.e., $\det
\wh{A}_{1,\dots,n} = \det A_\emptyset = 1$.
\end{definition}




\begin{lemma}[Laplace expansion\index{Laplace expansion}]\label{lem:determinant_minor_matrix}
Let $A\in M_n(\F)$ and $M$ is the minor matrix of $A$.
\ben
\item [(i)] For fixed $j$, $\det A = \sum^n_{i=1} (-1)^{i+j} a_{ij} M_{ij}$.
\item [(ii)] For fixed $i$, $\det A = \sum^n_{j=1} (-1)^{i+j} a_{ij} M_{ij} $.
\een

This is called Laplace expansion, which can be used as a definition of determinant.
\end{lemma}



\begin{proof}[\bf Proof]
By Proposition \ref{pro:determinant_matrix_property},
\beast
\det A & = & \det(A_1, \dots,A_n) = \sum^n_{i=1} \det \brb{A_1,\dots,A_{j-1},A'_j,A_{j+1},\dots,A_n} \\
& = & \sum^n_{i=1} a_{ij} \det \brb{A_1,\dots,A_{j-1}, e_i,A_{j+1},\dots,A_n}
\eeast
where $A_j' = \bepm 0,\dots,0,a_{ij},0,\dots,0\eepm^T$ (the only non-zero element is in row $i$).

Now we can swap row $i$ and row $i-1$, and then swap row $i-1$ and $i-2$,... Finally, we have shift original row $i$ to row 1 and this need $i-1$ transpositions (similarly for column operations which need $j-1$ transpositions, so totally $i+j-2$ transpositions). Thus, by Proposition \ref{pro:determinant_matrix_property}.(iii), % column $j$ and column 1.
\be
\det A = \sum^n_{i=1} a_{ij}(-1)^{i+j-2} \det\bepm 1 & * \\ 0 & \wh{A}_{ij}\eepm = \sum^n_{i=1} (-1)^{i+j} a_{ij} M_{ij},
\ee
using Proposition \ref{pro:block_triangular_matrix_determinant}. (ii) is similar by using determinant of transpose matrix (Proposition \ref{pro:determinant_transpose}).
\end{proof}


\begin{example}\label{exa:determinant_special_tridiagonal_matrices}
Suppose that two matrices in $M_n(\C)$ are
\be
P_n(x) = \bepm
x & 1 & & & & \\ 1 & x & 1 & & & \\ & 1 & x & 1 & & \\ & & & \ddots & & \\ & & &  1 & x & 1 \\ & & & & 1 & x
\eepm,\qquad
R_n(x,a,b) = \bepm
x+a & 1 & & & & \\ 1 & x & 1 & & & \\ & 1 & x & 1 & & \\ & & & \ddots & & \\ & & &  1 & x & 1 \\ & & & & 1 & x +b
\eepm.
\ee

It is easy to see that $P_n(x) = R_n(x,0,0)$. We can define that $\phi_n(x)=\det\brb{P_n(x)}$. By Laplace expansion (Lemma \ref{lem:determinant_minor_matrix}), we have
\be
\phi_n(x) = x \phi_{n-1}(x) + (-1)\cdot 1 \cdot \phi_{n-2}(x),\qquad \phi_1(x) = x,\ \phi_2(x) = x^2 -1.
\ee

It can be seen in \cite{Rutherford_1952} that\footnote{details needed.}
\be
\phi_n(x) = \binom{n}{n}x^n - \binom{n-1}{1}x^{n-2} + \binom{n-2}{2}x^{n-4} -\dots.
\ee

The series is terminated either with a constant term or a term in $x$.

Alternatively, we can use Wolstenholme's idea (see \cite{Muir_1923} and \cite{Rutherford_1946}) that if we write
\be
x = 2\cos \theta = e^{i\theta}+e^{-i\theta},\quad \theta\neq 0,\pi, \theta\in (-\pi,\pi]
\ee
with condition (see Theorem \ref{thm:sine_multiple_angle_formula})
\beast
\phi_1(x) & = & x = 2\cos\theta = \frac{\sin\brb{2\theta}}{\sin\theta},\\
\phi_2(x)& = & x^2 - 1 = 4\cos^2\theta - 1 = \frac{3\sin x-4\brb{\sin x}^3}{\sin\theta} = \frac{\sin\brb{3\theta}}{\sin \theta}.
\eeast

By induction, we assume that $\phi_k(x) = \frac{\sin\brb{(k+1)\theta}}{\sin \theta}$ for all $k\leq n-1$. Therefore,
\beast
\phi_n(x) & = & x \frac{\sin\brb{n\theta}}{\sin \theta} - \frac{\sin\brb{(n-1)\theta}}{\sin \theta} = \frac 1{\sin\theta}\brb{2\cos\theta \sin\brb{n\theta}- \sin\brb{(n-1)\theta}} \\
& = & \frac 1{\sin\theta}\brb{2\cos^2\theta \sin\brb{(n-1)\theta} + \sin\brb{2\theta}\cos\brb{(n-1)\theta} - \sin\brb{(n-1)\theta}} \\
& = & \frac 1{\sin\theta}\brb{\cos(2\theta )\sin\brb{(n-1)\theta} + \sin\brb{2\theta}\cos\brb{(n-1)\theta}} = \frac{\sin\brb{(n+1)\theta}}{\sin \theta}\qquad (*).
\eeast

Also, by Proposition \ref{pro:determinant_matrix_property}.(ii) and Laplace expansion (Lemma \ref{lem:determinant_minor_matrix}), we have
\beast
& & \det\brb{R_n(x,a,b)} \\
& = & \det\bepm
x & 1 & & & \\ 1 & x & 1 & &  \\  & &  \ddots & &  \\ & &  1 & x & 1 \\ & & & 1 & x
\eepm_{n\times n}  + (a+b)\det\bepm
x & 1 & & & \\ 1 & x & 1 & &  \\  & & \ddots & & \\ & &  1 & x & 1 \\ & & & 1 & x
\eepm_{n-1\times n-1} + ab \det\bepm
x & 1 & & & \\ 1 & x & 1 & &  \\  & & \ddots & & \\ & &  1 & x & 1 \\ & & & 1 & x
\eepm_{n-2\times n-2} .
\eeast

That is,
\beast
\det\brb{R_n(x,a,b)} & = & \phi_n(x) + (a+b)\phi_{n-1}(x) + ab\phi_{n-2}(x) \\
& = & \frac 1{\sin\theta} \brb{\sin\brb{(n+1)\theta} + (a+b)\sin\brb{n\theta} + ab\sin\brb{(n-1)\theta}}.
\eeast,

In particular,
\beast
\det\brb{R_n(x,1,1)} & = & \frac 1{\sin\theta} \brb{\sin(n\theta)\cos\theta + \cos (n\theta)\sin \theta + 2\sin(n\theta) + \sin(n\theta)\cos\theta - \cos (n\theta)\sin \theta} \\
& = & \frac 1{\sin\theta} \brb{2\sin(n\theta)\cos\theta + 2\sin(n\theta) } =\frac {2\sin(n\theta)\brb{1+\cos\theta}}{\sin\theta} .
\eeast

\beast
\det\brb{R_n(x,1,0)} = \det\brb{R_n(x,0,1)}& = & \frac 1{\sin\theta} \brb{\sin(n\theta)\cos\theta + \cos (n\theta)\sin \theta + \sin(n\theta)} \\
& = & \frac {2\sin(n\theta)\cos^2\frac{\theta}2 + 2\cos(n\theta)\sin^2\frac{\theta}2\cos^2\frac{\theta}2}{\sin\theta}  \\
& = & \frac{\sin(n\theta)\cos\frac{\theta}2 + \cos(n\theta)\sin\frac{\theta}2}{\sin\frac{\theta}2} = \frac{\sin\brb{\frac{(2n+1)\theta}2}}{\sin\frac{\theta}2}.
\eeast


\beast
\det\brb{R_n(x,-1,1)} & = & \frac 1{\sin\theta} \brb{\sin(n\theta)\cos\theta + \cos (n\theta)\sin \theta - \sin(n\theta)\cos\theta + \cos (n\theta)\sin \theta} \\
& = & 2\frac{\cos(n\theta)\sin\theta}{\sin\theta}= 2\cos(n\theta).
\eeast

It is clear that from $(*)$ that $\det\brb{P_n(x)}$ vanishes when and only when
\be
\theta = \frac {k\pi}{n+1},\qquad k = 1,\dots,n,
\ee
that is, when
\be
x = 2\cos\brb{\frac{k\pi}{n+1}},\qquad k = 1,\dots,n.
\ee

These are in fact the $n$ distinct roots of the equation $\det\brb{P_n(x)} = 0$. Since $\det\brb{P_n(x)}$ is a polynomial in $x$ of degree $n$ and since the coefficient of $x^n$ in this polynomial is 1 we conclude that
\be
\phi_n(x) = \det\brb{P_n(x)} = \prod^n_{k=1} \brb{x - 2\cos\brb{\frac{k\pi}{n+1}}}.
\ee

Similarly, it may be shown that
\beast
\det\brb{R_n(x,1,1)} & = & \prod^n_{k=1} \brb{x - 2\cos\brb{\frac{k\pi}{n}}},\\
\det\brb{R_n(x,1,0)} & = & \prod^n_{k=1} \brb{x - 2\cos\brb{\frac{2k\pi}{2n+1}}}\\
\det\brb{R_n(x,-1,1)} & = & \prod^n_{k=1} \brb{x - 2\cos\brb{\frac{(2k-1)\pi}{2n}}}.
\eeast
\end{example}



\begin{definition}[cofactor matrix\index{cofactor matrix}]\label{def:cofactor_matrix}
Let $A \in M_n(\F)$ and $M_{ij}$ is the $(i,j)$-minor of $A$. Then cofactor matrix of $A$ is $C = (c_{ij})$ where $c_{ij} = (-1)^{i+j} M_{ij}$.
\end{definition}

\begin{definition}[adjugate matrix\index{adjugate matrix}]\label{def:adjugate_matrix}
Let $A \in M_n(\F)$ and $C$ is cofactor matrix of $A$. The adjugate matrix is $\adj A = C^T$, i.e. the $n \times n$ matrix with $(i, j)$ entry equal to $(-1)^{i+j} M_{ji} = (-1)^{i+j} \det \brb{\wh{A}_{ji}}$.
\end{definition}


\begin{example}
As a specific example, we have \be A= \bepm \!-3 & \, 2 & \!-5 \\ \!-1 & \, 0 & \!-2 \\ \, 3 & \!-4 & \, 1 \eepm \ \ra \ \adj A = \bepm \!-8 & \,18 & \!-4 \\ \!-5 & \!12 & \,-1 \\ \, 4 & \!-6 & \, 2
\eepm. \ee

The $-6$ in the third row, second column of the adjugate was computed as follows: \be (-1)^{2+3}\det \bepm \!-3&\,2\\ \,3&\!-4\eepm =-((-3)(-4)-(3)(2))=-6. \ee

Again, the (3,2) entry of the adjugate is the (2,3) cofactor of $A$. Thus, the submatrix $\bepm \!-3&\,\!2\\ \,\!3&\!-4\eepm$ was obtained by deleting the second row and third column of the original
matrix $A$.
\end{example}


\begin{theorem}\label{thm:adjugate_inverse_matrix}
\ben
\item [(i)] $(\adj A)A = (\adj A)A = (\det A)I$.
\item [(ii)] If $A$ is invertible then $A^{-1} = \frac 1{\det A} \adj A$.
\een
\end{theorem}

\begin{proof}[\bf Proof]
\ben
\item [(i)] By Lemma \ref{lem:determinant_minor_matrix}.(i), fix $j$,
\be
\det A = \sum^n_{i=1} (\adj A)_{ji} a_{ij} = ((\adj A) A)_{jj}.
\ee

Also, for fix $j < k$, by Proposition \ref{pro:determinant_matrix_property} and Lemma \ref{lem:determinant_minor_matrix},
\be
0 = \det(A_1, \dots,A_k, \dots,A_k, \dots,A_n) = \sum^n_{i=1} (\adj A)_{ji} a_{ik} = ((\adj A) A)_{jk}.
\ee

Similarly, from Lemma \ref{lem:determinant_minor_matrix}.(ii), we have $(\adj A)A = (\det A)I$.

\item [(ii)] If $A$ is invertible, then $\det A \neq 0$ (by Theorem \ref{thm:matrix_invertible_determinant_non_zero}), so $\frac 1{\det A} (\adj A) A = I$ and hence we deduce $A^{-1} = \frac 1{\det A} \adj A$.
\een
\end{proof}





\begin{proposition}\label{pro:adjugate_matrix_basic_properties}
\ben
\item [(i)] $\adj I = I$ where $I \in M_n(\F)$ is identity matrix.
\item [(ii)] For $A\in M_n(\F)$, $\adj\brb{\lm A} = \lm^{n-1}\adj A$.
\item [(iii)] $\adj \brb{A^T} = \brb{\adj A}^T$.
\item [(iv)] $\adj\brb{\ol{A}} = \ol{\adj A}$.
\item [(v)] Adjugate matrices of diagonal matrices are still diagonal matrices, i.e., if $A = \diag\brb{a_1,a_2,\dots,a_n}$, then
\be
\adj A = \diag\brb{\prod^n_{i=2}a_i,\prod^n_{i=1,i\neq 2}a_i, \prod^n_{i=1,i\neq 3}a_i,\dots, \prod^{n-1}_{i=1}a_i}.
\ee
\item [(vi)] Adjugate matrices of symmetric matrices are still symmetric matrices, i.e., if $A^T = A$, then $\brb{\adj A}^T = \adj A$.
\item [(vii)] If $A\in M_n(\F)$ is antisymmetric, i.e., $A^T = -A$, then
\be
\brb{\adj A}^T = \left\{\ba{ll}
-\adj A \quad\quad & n \text{ is even}\\
\adj A & n \text{ is odd}
\ea\right.
\ee
\item [(viii)] For elementary operations,
\be
\adj T_{i,j} = -T_{i,j},\quad i\neq j, \qquad \adj M_{i,\lm} = \lm M_{i,1/\lm},\quad c\neq 0,\qquad \adj C_{i,j,\lm} = C_{i,j,-\lm}.
\ee
\item [(ix)] Let $Q$ be an othogonal matrix. Then
\be
\adj Q = \left\{\ba{ll}
Q^T & \det Q = 1\\
-Q^T \quad\quad & \det Q = -1
\ea\right.
\ee
\een
\end{proposition}

\begin{proof}[\bf Proof]
\ben
\item [](i)-(iv) are obvious from definition of adjugate matrix (Definition \ref{def:adjugate_matrix}).
\item [(v)] \footnote{need proof}
\een
\end{proof}

\begin{lemma}\label{lem:adjugate_invertible_product}
For any $A\in M_n(\F)$ and invertible matices $P,Q\in M_n(\F)$, $\adj\brb{PAQ} = \adj Q\adj A\adj P$.
\end{lemma}

\begin{proof}[\bf Proof]
Let $E$ be an elementary matrix.% and $B_{ij} = (-1)^{i+j}\det\brb{\wh{A}_{ij}}$. %$P,Q$ are elementary matrices (We only need to consider elementary column matrices as we can take the transpose of them to get elementary row matrices). Thus, we have three cases. %\footnote{need proof}

If $E$ is switching matrix $T_{i,j}$ ($i<j$), then by Proposition \ref{pro:determinant_matrix_property}.(iii),
\beast
\adj\brb{AT_{i,j}} & = & \adj \bepm A_1,\dots, A_j,\dots, A_i,\dots, A_n\eepm  \\
& = & \bepm -(-1)^{1+1}\det\brb{\wh{A}_{11}} & -(-1)^{2+1}\det\brb{\wh{A}_{21}} & \dots & -(-1)^{n+1}\det\brb{\wh{A}_{n1}} \\ \vdots\\ (-1)^{1+i}(-1)^{j-i-1}\det\brb{\wh{A}_{1j}} & (-1)^{2+i}(-1)^{j-i-1}\det\brb{\wh{A}_{2j}} & \dots & (-1)^{n+i}(-1)^{j-i-1}\det\brb{\wh{A}_{nj}} \\ \vdots \\  (-1)^{1+j}(-1)^{j-i-1}\det\brb{\wh{A}_{1i}} & (-1)^{2+j}(-1)^{j-i-1}\det\brb{\wh{A}_{2i}} & \dots & (-1)^{n+j}(-1)^{j-i-1}\det\brb{\wh{A}_{ni}} \\ \vdots \\  -(-1)^{1+n}\det\brb{\wh{A}_{1n}} & -(-1)^{2+n}\det\brb{\wh{A}_{2n}} & \dots & -(-1)^{n+n}\det\brb{\wh{A}_{nn}}  \eepm
\eeast
where $(-1)^{j-i}\det\brb{\wh{A}_{kj}}$ (as we shift $A_i$ to its original position with $j-i-1$ times) is $(k,j)$ minor of $AT_{i,j}$. Thus,
\beast
\adj\brb{AT_{i,j}} & = & \bepm -(-1)^{1+1}\det\brb{\wh{A}_{11}} & -(-1)^{2+1}\det\brb{\wh{A}_{21}} & \dots & -(-1)^{n+1}\det\brb{\wh{A}_{n1}} \\ \vdots\\ -(-1)^{1+j}\det\brb{\wh{A}_{1j}} & -(-1)^{2+j}\det\brb{\wh{A}_{2j}} & \dots & -(-1)^{n+j}\det\brb{\wh{A}_{nj}} \\ \vdots \\ -(-1)^{1+i}\det\brb{\wh{A}_{1i}} & -(-1)^{2+i}\det\brb{\wh{A}_{2i}} & \dots & -(-1)^{n+i}\det\brb{\wh{A}_{ni}} \\ \vdots \\  -(-1)^{1+n}\det\brb{\wh{A}_{1n}} & -(-1)^{2+n}\det\brb{\wh{A}_{2n}} & \dots & -(-1)^{n+n}\det\brb{\wh{A}_{nn}}  \eepm = \adj T_{i,j}\adj A.
\eeast

Also, since $T_{i,j}^T = T_{i,j}$, by Proposition \ref{pro:adjugate_matrix_basic_properties}.(iii) and Proposition \ref{pro:matrix_multiple_transpose}
\be
\adj\brb{T_{i,j}A} = \adj\brb{T_{i,j}^TA} = \adj\brb{\brb{A^TT_{i,j}}^T} = \brb{\adj\brb{A^TT_{i,j}}}^T = \brb{\adj T_{i,j}\adj \brb{A^T}}^T = \adj A \adj T_{i,j}.
\ee

If $E$ is multiplication matrix $M_{i,\lm}$ ($\lm \neq 0$), we have by Theorem \ref{thm:adjugate_inverse_matrix}.(ii) and Proposition \ref{pro:square_elementary_matrix_invertible},
\beast
\adj\brb{AM_{i,\lm}} & = & \adj \bepm A_1,\dots, \lm A_i,\dots, A_n\eepm  \\
& = & \bepm \lm (-1)^{1+1}\det\brb{\wh{A}_{11}} & \lm (-1)^{2+1}\det\brb{\wh{A}_{21}} & \dots & \lm (-1)^{n+1}\det\brb{\wh{A}_{n1}} \\ \vdots\\ (-1)^{1+i}\det\brb{\wh{A}_{1i}} & (-1)^{2+i}\det\brb{\wh{A}_{2i}} & \dots & (-1)^{n+i}\det\brb{\wh{A}_{ni}} \\ \vdots \\  \lm (-1)^{1+n}\det\brb{\wh{A}_{1n}} & \lm(-1)^{2+n}\det\brb{\wh{A}_{2n}} & \dots & \lm (-1)^{n+n}\det\brb{\wh{A}_{nn}}  \eepm \\
& = & \lm M_{i,1/\lm} \adj A = \det\brb{M_{i,\lm}} \brb{M_{i,\lm}}^{-1} \adj A= \adj M_{i,\lm} \adj A.
\eeast

Similarly, since $M_{i,\lm}^T = M_{i,\lm}$, by Proposition \ref{pro:adjugate_matrix_basic_properties}.(iii) and Proposition \ref{pro:matrix_multiple_transpose}
\be
\adj\brb{M_{i,\lm}A} = \adj\brb{M_{i,\lm}^TA} = \adj\brb{\brb{A^TM_{i,\lm}}^T} = \brb{\adj\brb{A^TM_{i,\lm}}}^T = \brb{\adj M_{i,\lm}\adj \brb{A^T}}^T = \adj A \adj M_{i,\lm}.
\ee

If $E$ is addition matrix $C_{i,j,\lm}$ ($i<j$), we have by Proposition \ref{pro:determinant_matrix_property}.(ii,iv),
\beast
\adj\brb{AC_{i,j,\lm}} & = & \adj \bepm A_1,\dots, A_i,\dots, A_j + \lm A_i,\dots, A_n\eepm  \\
& = & \bepm (-1)^{1+1}\det\brb{\wh{A}_{11}} & \dots & \lm (-1)^{n+1}\det\brb{\wh{A}_{n1}} \\ \vdots\\ (-1)^{1+i}\brb{\lm (-1)^{j-i-1}\det\brb{\wh{A}_{1j}}  + \det\brb{\wh{A}_{1i}}} & \dots & (-1)^{n+i}\brb{\lm (-1)^{j-i-1}\det\brb{\wh{A}_{nj}} + \det\brb{\wh{A}_{ni}}}  \\ \vdots \\ (-1)^{1+j}\det\brb{\wh{A}_{1j}} & \dots &  (-1)^{n+j}\det\brb{\wh{A}_{nj}}  \\ \vdots \\   (-1)^{1+n}\det\brb{\wh{A}_{1n}} & \dots &  (-1)^{n+n}\det\brb{\wh{A}_{nn}}  \eepm
\eeast

Thus, by Theorem \ref{thm:adjugate_inverse_matrix}.(ii), Proposition \ref{pro:square_elementary_matrix_invertible} and Proposition \ref{pro:determinant_element_matrix} ($\det\brb{C_{i,j,\lm}} = 1$),
\beast
\adj\brb{AC_{i,j,\lm}} & = & \bepm (-1)^{1+1}\det\brb{\wh{A}_{11}} & \dots & \lm (-1)^{n+1}\det\brb{\wh{A}_{n1}} \\ \vdots\\ -\lm (-1)^{1+j} \det\brb{\wh{A}_{1j}}  + (-1)^{1+i}\det\brb{\wh{A}_{1i}} & \dots & -\lm (-1)^{n+j} \det\brb{\wh{A}_{nj}} + (-1)^{n+i}\det\brb{\wh{A}_{ni}}  \\ \vdots \\ (-1)^{1+j}\det\brb{\wh{A}_{1j}} & \dots &  (-1)^{n+j}\det\brb{\wh{A}_{nj}}  \\ \vdots \\   (-1)^{1+n}\det\brb{\wh{A}_{1n}} & \dots &  (-1)^{n+n}\det\brb{\wh{A}_{nn}}  \eepm \\
& = & C_{i,j,-\lm} \adj A = \det\brb{C_{i,j,\lm}} \brb{C_{i,j,-\lm}}^{-1} \adj A= \adj C_{i,j,\lm} \adj A.
\eeast

With the same argument as above, we have $\adj\brb{C_{i,j,\lm}A} =  \adj A \adj C_{i,j,\lm}$. Therefore, for any elementary matrix $E$, we have
\be
\adj\brb{AE} =  \adj E \adj A,\quad \adj\brb{EA} =  \adj A \adj E.
\ee

Recall Lemma \ref{lem:invertible_product_elementary_matrices}, any invertible matrices $P,Q$ can be written as the products of elementary matrices $P = E_1\dots E_m$, $Q = F_1\dots F_k$. Then
\beast
\adj\brb{PAQ} & = & \adj\brb{E_1\dots E_m A F_1\dots F_k} = \brb{\adj F_k \dots \adj F_1}\adj A \brb{\adj E_m \dots \adj E_1} \\
& = & \adj \brb{F_1\dots F_k} \adj A \adj \brb{E_1\dots E_m} = \adj Q \adj A \adj P.
\eeast
\end{proof}


\begin{theorem}\label{thm:adjugate_matrix_rank}
Let $A\in M_n(\F)$ ($n > 1$). Then
\be
\rank\brb{\adj A} = \left\{\ba{ll}
n \quad\quad & \rank(A) = n\\
1 & \rank(A) = n-1\\
0 & \text{otherwise}
\ea\right.
\ee
\end{theorem}

\begin{proof}[\bf Proof]
If $\rank(A) = n$, $\det A \neq 0$. Then by Theorem \ref{thm:adjugate_inverse_matrix}, $\adj A$ is invertible and hence $\rank\brb{\adj A} = n$ by Theorem \ref{thm:invertible_full_rank}.

If $\rank(A) = n-1$, it is equivalent to matrix $I' = \bepm I_{n-1} & 0 \\ 0 & 0 \eepm$. That is, there exists invertible matrices $P,Q$ such that $A = PI'Q$. By Lemma \ref{lem:adjugate_invertible_product}, we have $\adj A = \adj Q \adj I' \adj P$. Since $P,Q$ are full-rank, by Proposition \ref{pro:matrix_rank_inequalities_product}.(i), we have
\be
\rank\brb{\adj A} \leq \rank\brb{\adj I'} = \rank\bepm 0 & 0 \\ 0 & 1\eepm = 1.
\ee

Also, $I' = P^{-1}A Q^{-1}$, Proposition \ref{pro:matrix_rank_inequalities_product}.(i) gives that $1 = \rank\brb{\adj I'} \leq \rank\brb{\adj A} \ \ra \ \rank\brb{\adj A} = 1$.

With similar argument, we have that $\rank\brb{\adj A} = 0$ if $\rank(A) < n-1$.
\end{proof}

\begin{proposition}
Let $A\in M_n(\F)$. Then the $k$-fold adjugate matrix of $A$, $\adj^{[k]}(A) := \adj\brb{\adj\brb{\dots \adj A \dots}}$ ($k$ times) is
\be
\adj^{[k]}(A) = \left\{\ba{ll}
\brb{\det A}^{\frac{(n-1)^k-(n-1)}{n}}\adj A \quad\quad & \rank(A) = n,\ k \text{ is odd}\\
\brb{\det A}^{\frac{(n-1)^k-1}{n}}A \quad\quad & \rank(A) = n,\ k \text{ is even}\\
0 & \rank(A) = n-1,\ k\geq 2\\
0 & \rank(A) < n-1,\ k\geq 1
\ea\right.
\ee

In particular, $\adj\brb{\adj A} = \brb{\det A}^{n-2}A$.
\end{proposition}

\begin{proof}[\bf Proof]
For $k=1$, by Theorem \ref{thm:adjugate_matrix_rank}, the conclusion holds.

For $k\geq 2$, $\adj^{[k]}(A) = 0$ if $\rank(A)\leq n-1$ by Theorem \ref{thm:adjugate_matrix_rank}. Thus, we only need to check the full-rank situation.

Now assume $\rank(A)=n$. If $k=2$, we have
\be
\adj\brb{\adj A} = \det\brb{\adj A} \brb{\adj A}^{-1} = \det\brb{\det A A^{-1}} \brb{\det A A^{-1}}^{-1} = \brb{\det A}^{n-1} \det\brb{ A^{-1}} A = \brb{\det A}^{n-2}A.
\ee

Now assume the conclusion holds for $k$. If $k$ is odd, $\adj^{[k]}(A) = \brb{\det A}^{\frac{(n-1)^k-(n-1)}{n}}\adj A$. Then by Proposition \ref{pro:adjugate_matrix_basic_properties}.(ii) and the case $k=2$,
\beast
\adj^{[k+1]}(A) & = & \adj\brb{\brb{\det A}^{\frac{(n-1)^k-(n-1)}{n}}\adj A} = \brb{\brb{\det A}^{\frac{(n-1)^k-(n-1)}{n}}}^{n-1} \adj^{[2]}(A)\\
& = & \brb{\det A}^{\frac{(n-1)^{k+1}-(n-1)^2}{n}} \brb{\det A}^{n-2}A = \brb{\det A}^{\frac{(n-1)^{k+1}-1}{n}}A.
\eeast

If $k$ is even, $\adj^{[k]}(A) = \brb{\det A}^{\frac{(n-1)^k-1}{n}} A$. Then by Proposition \ref{pro:adjugate_matrix_basic_properties}.(ii),
\beast
\adj^{[k+1]}(A) & = & \adj\brb{\brb{\det A}^{\frac{(n-1)^k-1}{n}} A} = \brb{\brb{\det A}^{\frac{(n-1)^k-1}{n}}}^{n-1} \adj A = \brb{\det A}^{\frac{(n-1)^{k+1}-(n-1)}{n}} \adj A.
\eeast
\end{proof}


\begin{theorem}\label{thm:adjugate_matrix_product}
For any $A,B\in M_n(\F)$, $\adj\brb{AB} = \adj B \adj A$.
\end{theorem}

\begin{proof}[\bf Proof]
If $A,B$ are invertible, we have $AB$ is also invertible. Then by Theorem \ref{thm:determinant_product}, Proposition \ref{pro:inverse_matrix_property} and Theorem \ref{thm:adjugate_inverse_matrix},
\be
\adj\brb{AB} = \det\brb{AB}\brb{AB}^{-1} = \det A \det B B^{-1}A^{-1} = \adj B \adj A.
\ee

If $\min\bra{\rank(A),\rank(B)} < n-1$ with $\adj A = 0$ or $\adj B = 0$, then by Proposition \ref{pro:matrix_rank_inequalities_product}.(i),%\footnote{need proof}
\be
\rank(AB) \leq \min\bra{\rank(A),\rank(B)} < n-1 \ \ra \ \adj\brb{AB} = 0 = \adj B\adj A
\ee
by Theorem \ref{thm:adjugate_matrix_rank}.

If $\min\bra{\rank(A),\rank(B)} = n-1$ and $\max\bra{\rank(A),\rank(B)} = n$, we can assume $\rank(A) = n$ and $\rank(B) = n-1$ wlog as we can take transpose of $AB$. Then by Lemma \ref{lem:adjugate_invertible_product}, we have $\adj\brb{AB} = \adj B\adj A$.

Therefore, there leaves the last case, $\rank(A) = \rank(B) = n-1$. Assume that (by Lemma \ref{lem:matrix_equivalent_to_identity})
\be
A = P\bepm I_{n-1} & 0 \\ 0 & 0 \eepm Q,\quad B = U\bepm I_{n-1} & 0 \\ 0 & 0 \eepm V
\ee
where $P,Q,U,V$ are invertible matrices and
\be
QU = \bepm C_{(n-1)\times (n-1)} & D_{(n-1)\times 1} \\ E_{1\times (n-1)} & F_{1\times 1} \eepm.
\ee

Therefore, by Lemma \ref{lem:adjugate_invertible_product},
\beast
\adj\brb{AB} & = & \adj\brb{P \bepm I_{n-1} & 0 \\ 0 & 0 \eepm \bepm C_{(n-1)\times (n-1)} & D_{(n-1)\times 1} \\ E_{1\times (n-1)} & F_{1\times 1} \eepm \bepm I_{n-1} & 0 \\ 0 & 0 \eepm V} = \adj\brb{P \bepm C_{(n-1)\times (n-1)} & 0 \\0 & 0 \eepm V} \\
& = & \adj V \adj\bepm C_{(n-1)\times (n-1)} & 0 \\0 & 0 \eepm \adj P = \adj V \bepm 0 & 0 \\0 & \det C_{(n-1)\times (n-1)} \eepm \adj P.
\eeast

But $\adj \bepm C_{(n-1)\times (n-1)} & 0 \\ E_{1\times (n-1)} & 0 \eepm \adj \bepm I_{n-1} & 0 \\ 0 & 0 \eepm = \bepm 0 & 0 \\ \alpha & \det C_{(n-1)\times (n-1)} \eepm \bepm 0 & 0 \\0 & 1 \eepm = \bepm 0 & 0 \\0 & \det C_{(n-1)\times (n-1)} \eepm$. Then we have (by Lemma \ref{lem:adjugate_invertible_product})
\beast
\adj\brb{AB} & = & \adj V \brb{\adj \bepm C_{(n-1)\times (n-1)} & 0 \\ E_{1\times (n-1)} & 0 \eepm \adj \bepm I_{n-1} & 0 \\ 0 & 0 \eepm} \adj P\\
& = & \adj V \brb{\adj \brb{\bepm C_{(n-1)\times (n-1)} & D_{(n-1)\times 1} \\ E_{1\times (n-1)} & F_{1\times 1} \eepm \bepm I_{n-1} & 0 \\ 0 & 0 \eepm} \adj \bepm I_{n-1} & 0 \\ 0 & 0 \eepm} \adj P\\
& = & \adj V \brb{\adj \brb{QU \bepm I_{n-1} & 0 \\ 0 & 0 \eepm} \adj \bepm I_{n-1} & 0 \\ 0 & 0 \eepm} \adj P \\
& = & \adj V \adj \brb{\bepm I_{n-1} & 0 \\ 0 & 0 \eepm}\adj\brb{QU} \adj \bepm I_{n-1} & 0 \\ 0 & 0 \eepm \adj P\\
& = & \adj V \adj \bepm I_{n-1} & 0 \\ 0 & 0 \eepm\adj U \adj Q \adj \bepm I_{n-1} & 0 \\ 0 & 0 \eepm \adj P \\
& = & \adj\brb{U  \bepm I_{n-1} & 0 \\ 0 & 0 \eepm V}\adj \brb{P \bepm I_{n-1} & 0 \\ 0 & 0 \eepm Q} = \adj B \adj A.
\eeast

Therefore, the equation holds for any $A,B\in M_n(\F)$.
\end{proof}


\begin{corollary}\label{cor:adjugate_matrix_product}
For given $k$ and any $A\in M_n(\F)$,
\ben
\item [(i)] $\adj\brb{A_1A_2\dots A_k} = \adj A_k \adj A_{k-1} \dots \adj A_1$.
\item [(ii)] $\adj\brb{A^k} = \brb{\adj A}^k$.
\een
\end{corollary}

\begin{proposition}\label{pro:adjugate_matrix_property}
For given $k \geq 1$,
\ben
\item [(i)] $\adj\brb{A^{-1}} = \brb{\adj A}^{-1}$.
\item [(ii)] $\det\brb{\adj A} = \brb{\det A}^{n-1}$ for $n\geq 2$.
\item [(iii)] $\det\brb{\adj^{[k]}(A)} = \brb{\det A}^{(n-1)^k}$.
\item [(iv)] $\det\brb{\adj\brb{A_1A_2\dots A_k}} = \prod^k_{i=1} \det\brb{\adj A_i} = \brb{\prod^k_{i=1} \det A_i}^{n-1}$.
\item [(v)] $\det\brb{\adj (A^k)} = \brb{\det \brb{\adj A}}^k = \brb{\det A}^{k(n-1)}$.
\een
\end{proposition}

\begin{proof}[\bf Proof]
\ben
\item [(i)] $\adj\brb{A^{-1}} \adj A = \adj \brb{A A^{-1}} = \adj I = I$. $\adj A \adj\brb{A^{-1}} = \adj \brb{A^{-1}A} = \adj I = I$. Thus, $\adj\brb{A^{-1}}$ is the inverse of $\adj A$ by Proposition \ref{pro:inverse_matrix}.
\item [(ii)] If $\rank(A) < n$, we have $\det A = 0$ and thus $\rank\brb{\adj A} < n$ by Theorem \ref{thm:adjugate_matrix_rank}. Thus, $\det\brb{\adj A} = 0$, which satisfies the statement.

If $A$ is full-rank, then we have $\det\brb{\det A A^{-1}} = \brb{\det A}^{n} \det\brb{A^{-1}} = \brb{\det A}^{n-1}$.

\item [(iii)] If $k = 1$, it is (ii). If $\rank(A) < n$, we have $\det A = 0 = \det\brb{\adj^{[k]}(A)}$. Now we assume the conclusion holds for $k$ and $A$ is full-rank (thus $\adj^{[k]}(A)$ is full-rank). Then
\beast
\det\brb{\adj^{[k+1]}(A)} & = & \det \brb{\adj\brb{\adj^{[k]}(A)}} = \det\brb{\det\brb{\adj^{[k]}(A)} \brb{\adj^{[k]}(A)}^{-1}} \\
& = & \brb{\det\brb{\adj^{[k]}(A)}}^{n}\det\brb{\brb{\adj^{[k]}(A)}^{-1}} = \brb{\det\brb{\adj^{[k]}(A)}}^{n-1}\\
& = & \brb{\brb{\det A}^{(n-1)^k}}^{n-1} = \brb{\det A}^{(n-1)^{k+1}}.
\eeast
\een

(iv,v) are direct results from (iii) and Corollary \ref{cor:adjugate_matrix_product}.
\end{proof}




\subsection{Commuting matrices}

\begin{definition}[commuting matrices\index{commuting matrices}]\label{def:commuting_matrices}
For matrices $A,B\in M_n(\F)$, we say that $A,B$ are commuting if $AB = BA$. We can write this by
\be
\bsb{A,B} = 0, \quad \text{ where}\quad \bsb{A,B} := AB - BA.
\ee
\end{definition}

\begin{remark}
Note that the concept of commuting matrices is very important concept in quantum mechanics.
\end{remark}



\begin{proposition}
Let $A,B\in M_n(\F)$ be commuting matrices ($AB = BA$). Then
\be
\rank(AB) + \rank(A+B) \leq \rank(A) + \rank(B).
\ee
\end{proposition}

\begin{proof}[\bf Proof]
By Proposition \ref{pro:rank_of_block_triangular_matrix}, Proposition \ref{pro:rank_of_block_diagonal_matrix} and Proposition \ref{pro:colrank_product_smaller_than_individual_colranks}, we have
\beast
\rank(AB) + \rank(A+B) & \leq & \rank\bepm AB & B \\ 0 & A+B \eepm =  \rank\bepm -AB & B \\ 0 & A+B \eepm  \stackrel{(*)}{=}  \rank\bepm B & B \\ B & A+B \eepm\bepm -(A+B) & 0 \\ B & I_n \eepm \\
& \leq & \rank\bepm B & B \\ B & A+B \eepm = \rank\bepm B & 0 \\ B & A \eepm \bepm I_n & I_n \\ 0 & I_n \eepm = \rank  \bepm I_n & 0 \\ I_n & I_n \eepm \bepm B & 0 \\ 0 & A \eepm \bepm I_n & I_n \\ 0 & I_n \eepm \\
& \leq & \rank \bepm B & 0 \\ 0 & A \eepm = \rank(A) + \rank(B).
\eeast
\end{proof}



\begin{example}
For commuting matrices $A,B\in M_n(\F)$, we cannot have $\rank(A^2)+\rank(B^2) \geq 2\rank(AB)$.

The counterexample is
\be
A = \bepm
0 & 1 & 0 & 0 \\
0 & 0 & 0 & 0 \\
0 & 0 & 0 & 1 \\
0 & 0 & 0 & 0
\eepm, \qquad B = \bepm
0 & 0 & 0 & 0 \\
0 & 0 & 0 & 0 \\
1 & 0 & 0 & 0 \\
0 & 1 & 0 & 0
\eepm
\ee
as
\be
A^2 = B^2  = \bepm
0 & 0 & 0 & 0 \\
0 & 0 & 0 & 0 \\
0 & 0 & 0 & 0 \\
0 & 0 & 0 & 0
\eepm, \qquad AB=BA = \bepm
0 & 0 & 0 & 0 \\
0 & 0 & 0 & 0 \\
0 & 1 & 0 & 0 \\
0 & 0 & 0 & 0
\eepm.
\ee
\end{example}

\subsection{Idempotent matrix}


\begin{definition}[idempotent matrix\index{idempotent matrix}]\label{def:idempotent_matrix}
An idempotent matrix ia a square matrix which, when multiplied by itself, yields itself. That is, $A\in M_n(\F)$ is called idempotent if $A^2 = A$.
\end{definition}


\begin{proposition}
Let $A\in M_n(\F)$. Then
\be
A\text{ is an idempotent matrix }(A^2 = A) \ \lra \ \rank(A) + \rank(I_n - A) =  n.
\ee
\end{proposition}

\begin{proof}[\bf Proof]
Clearly,
\be
A^2 - A  = 0 \ \lra \ A^2 - A = 0  \ \lra \rank(A^2 - A) = 0.
\ee

Then we can construct block matrix
\be
M = \bepm A & 0 \\ 0 & I_n - A \eepm.
\ee

Then by Proposition \ref{pro:rank_of_block_triangular_matrix}, Proposition \ref{pro:rank_of_block_diagonal_matrix} and Proposition \ref{pro:colrank_product_smaller_than_individual_colranks}, we have
\beast
\rank(A) + \rank(I_n - A) & = & \rank\bepm A & 0 \\ 0 & I_n - A \eepm =  \rank\bepm  I_n & 0 \\ -I_n & I_n \eepm \bepm A & 0 \\ A & I_n - A \eepm = \rank\bepm  I_n & 0 \\ -I_n & I_n \eepm \bepm A & A \\ A & I_n \eepm \bepm  I_n & -I_n \\ 0 & I_n \eepm  \\
& \leq & \rank \bepm A & A \\ A & I_n \eepm  = \rank\bepm I_n & A \\ 0 & I_n \eepm \bepm A-A^2 & 0 \\ A & I_n \eepm = \rank  \bepm I_n & A \\ 0 & I_n \eepm \bepm A-A^2 & 0 \\ 0 & I_n \eepm \bepm  I_n & 0 \\ A & I_n \eepm \\
& \leq & \rank  \bepm A-A^2 & 0 \\ 0 & I_n \eepm  = \rank(A^2 - A) + \rank(I_n) = \rank(A^2 - A) + n.
\eeast

Similarly, we have that
\beast
\rank(A^2 - A) + \rank(I_n) & = & \rank  \bepm A-A^2 & 0 \\ 0 & I_n \eepm = \rank \bepm I_n & A \\ 0 & I_n \eepm \bepm A-A^2 & \ -A\ \\ 0 & I_n \eepm \\
& = & \rank \bepm I_n & A \\ 0 & I_n \eepm \bepm A & -A \\ -A & I_n \eepm \bepm I_n & 0 \\ A & I_n \eepm \leq \rank \bepm A & -A \\ -A & I_n \eepm \\
& = & \rank \bepm I_n & 0 \\ -I_n & I_n \eepm \bepm A & -A \\ 0 & I_n - A \eepm = \rank \bepm I_n & 0 \\ -I_n & I_n \eepm \bepm A & 0 \\ 0 & I_n - A \eepm \bepm I_n & -I_n \\ 0 & I_n \eepm \\
& \leq & \rank \bepm A & 0 \\ 0 & I_n - A \eepm  = \rank(A) + \rank(I_n - A).
\eeast% $\rank(A^2 - A) + n \leq \rank(A) + \rank(I_n - A)$.

Therefore, we have that
\be
\rank(A) + \rank(I_n - A) = \rank(A^2 - A) + n
\ee
and thus
\be
A^2 = A  \ \lra\ \rank(A^2 - A) = 0 \ \lra\ \rank(A) + \rank(I_n - A) = n.
\ee
\end{proof}



\subsection{Diagonal matrices and block diagonal matrices}

\begin{definition}[diagonal matrix\index{diagonal matrix}]\label{def:diagonal_matrix}
The matrix $D = (d_{ij}) \in M_n(\F)$ is called diagonal matrix if $d_{ij} = 0$ whenever $i\neq j$, i.e.,
\be
D = \bepm d_{11} & & & \\ & d_{22} & & & \\ & & \ddots & \\ & & & d_{nn} \eepm.
\ee

We denote such a matrix as
\be
D = \diag\brb{d_{11},\dots,d_{nn}}\quad \text{or}\quad D = \diag d,
\ee
where $d$ is the vector of diagonal entries of $D$.

Let $x\in \F^n$ be a vector. Then $\diag(x)\in M_n(\F)$ is the diagonal matrix whose diagonal entries are the components of $x$, denoted by $D_x$.

Let $A\in M_n(\F)$ be a square matrix. Then $\diag(A)\in M_n(\F)$ is the diagonal matrix whose diagonal entries are the corresponding diagonal elements of $A$, $\diag(A)=\bra{a_{11},\dots,a_{nn}}$, denoted by $D_A$.

If all the diagonal entries of a diagonal matrix are positive (non-negative) real numbers, we refer to it as a positive (non-negative) diagonal matrix\index{diagonal matrix!positive, non-negative}.
\end{definition}

\begin{remark}
The identity matrix $I \in M_n(\F)$ is an example of a positive diagonal matrix.
\end{remark}

\begin{definition}[scalar matrix\index{scalar matrix}]\label{def:scalar_matrix}
A diagonal matrix $D\in M_n(\F)$ is called scalar matrix if the diagonal entries of $D$ are equal; that is, $D= \alpha I$ for some $\alpha \in \C$.
\end{definition}

\begin{remark}
Left or right multiplication of a matrix by a scalar matrix has the same effect as multiplying it by the corresponding scalar.
\end{remark}


\begin{proposition}[diagonal matrices properties]\label{pro:diagonal_matrix_property}
\ben
\item [(i)] The determinant of a diagonal matrix is just the product of its diagonal entries: $\det D = \prod^n_{i=1} d_{ii}$.
\item [(ii)] A diagonal matrix is invertible if and only if all its diagonal entries are non-zero.
\item [(iii)] All diagonal matrices commute with each other under multiplication, i.e., $AB = BA$, for diagonal matrices $A,B$.
\item [(iv)] A diagonal matrix $D$ commutes with a given matrix $A = (a_{ij})\in M_n(\F)$ if and only if $a_{ij} = 0$ whenever the $i$th and $j$th diagonal entries of $D$ differ.
\item [(v)] The product of two diagonal matrices is just the diagonal matrix of pairwise products of their respective diagonal entries and similarly for postive integer powers of a single diagonal matrix.
\een
\end{proposition}

\begin{proof}[\bf Proof]
\ben
\item [(i)] This is direct result from Laplace expansion (Lemma \ref{lem:determinant_minor_matrix}).
\item [(ii)] This is the result from (i) and Theorem \ref{thm:matrix_invertible_determinant_non_zero}.
\item [(iii)] We have %$a_{ij} = a_{ji}$ and $b_{ij} = b_{ji}$ for all $i,j$,
\be
(AB)_{ij} = \sum_k a_{ik}b_{kj} = \left\{
\ba{ll}
a_{ii}b_{ii} \quad\quad & i = j \\
0 & i \neq j
\ea\right.  = (BA)_{ij} .
\ee
\item [(iv)] If $a_{ij} = 0$ whenever the $i$th and $j$th diagonal entries of $D$ differ, we have
\be
(DA)_{ij} = \sum d_{ik}a_{kj} = d_{ii}a_{ij} = \left\{
\ba{ll}
d_{jj}a_{ij}\quad\quad & d_{ii} = d_{jj} \\
0 & d_{ii} \neq d_{jj}
\ea\right.
\ee

\be (AD)_{ij} = \sum a_{ik}d_{kj} = a_{ij}d_{jj} = \left\{ \ba{ll}
a_{ij}d_{jj}\quad\quad & d_{ii} = d_{jj} \\
0 & d_{ii} \neq d_{jj} \ea\right. \ee

Therefore $AD = DA$.

If $A$ and $D$ are commute, we have that $d_{ii}a_{ij} = d_{jj}a_{ij}$ by definition. Thus, $(d_{ii} - d_{jj})a_{ij} = 0$ implies that $a_{ij} = 0$ whenever the $i$th and $j$th diagonal entries of
$D$ differ.

\item [(v)] The same argument with (iii).
\een
\end{proof}




\begin{definition}[block diagonal matrix\index{block diagonal matrix}, direct sum\index{direct sum!matrices}]\label{def:block_diagonal_matrix}
A matrix $A = M_{n}(\F)$ of the form
\be
A = \bepm A_{11} & & & 0\\ & A_{22} & & \\ & & \ddots & \\ 0 & & & A_{kk} \eepm
\ee
in which $A_{ii} \in M_{n_i}(\F)$, $i = 1,\dots,k$ and $n = \sum^k_{j=1} n_j$, is called block diagonal matrix. Notationally, such a matrix is often indicated as
\be
A = A_{11} \oplus A_{22} \oplus \dots \oplus A_{kk}, \quad \text{ or }\quad \bigoplus^k_{i=1}A_{ii},
\ee
which is called the direct sum of the matrices $A_{11},\dots,A_{kk}$.
\end{definition}

\begin{proposition}[block diagonal matrices properties]\label{pro:block_diagonal_matrix_property}
For matrix $A = \bigoplus^k_{i=1}A_{ii}$ and $B = \bigoplus^k_{i=1}B_{ii}$, in which each pair $A_{ii}$, $B_{ii}$ has the same size.
\ben
\item [(i)] $\det A = \prod^k_{i=1} \det \brb{A_{ii}}$.
\item [(ii)] $A$ is invertible if and only if each $A_{ii}$, $i=1,\dots,k$ is invertible and the inverse is
\be
A^{-1} = \bepm A_{11}^{-1} & & & 0\\ & A_{22}^{-1} & & \\ & & \ddots & \\ 0 & & & A_{kk}^{-1} \eepm.
\ee
\item [(iii)] $\rank(A) = \sum^k_{i =1} \rank(A_{ii})$.
\item [(iv)] $A,B$ commute if and only if $A_{ii}$ and $B_{ii}$ commute, $i=1,\dots,k$.
\een
\end{proposition}

\begin{proof}[\bf Proof]
\ben
\item [(i)] By definition,
\be
A = \bepm A_{11} & & & 0\\ & A_{22} & & \\ & & \ddots & \\ 0 & & & A_{kk} \eepm = \bepm A_{11} & & & 0\\ & I_{n_2} & & \\ & & \ddots & \\ 0 & & & I_{n_k} \eepm \bepm I_{n_1} & & & 0\\ & A_{22} & & \\ & & \ddots & \\ 0 & & & I_{n_k} \eepm \dots \bepm I_{n_1} & & & 0\\ & I_{n_2} & & \\ & & \ddots & \\ 0 & & & A_{kk} \eepm
\ee

Then by Theorem \ref{thm:determinant_product} and Laplace expansion (Lemma \ref{lem:determinant_minor_matrix}) we have
\beast
\det A & = & \det\bepm A_{11} & & & 0\\ & I_{n_2} & & \\ & & \ddots & \\ 0 & & & I_{n_k} \eepm \det\bepm I_{n_1} & & & 0\\ & A_{22} & & \\ & & \ddots & \\ 0 & & & I_{n_k} \eepm \dots \det\bepm I_{n_1} & & & 0\\ & I_{n_2} & & \\ & & \ddots & \\ 0 & & & A_{kk} \eepm\\
& = & \det A_{11} \det A_{22} \dots \det A_{kk}.
\eeast

Alternatively, we can have the direct result from Proposition \ref{pro:block_triangular_matrix_determinant}.

\item [(ii)] Direct result from (i) and Theorem \ref{thm:matrix_invertible_determinant_non_zero}.

\item [(iii)] Approach 1. Direct result from Proposition \ref{pro:rank_of_block_diagonal_matrix}.

Approach 2. Assume $\rank (A_{ii}) = m_i$ for $i=1,\dots,k$. Then $A_{ii}$ is equivalent to a matrix $B_i\in M_{n_i}(\F)$ such that %\footnote{proof needed.}
\be
B_i = \bepm I_{m_i} & 0 \\ 0 & 0\eepm,\qquad B_{i} = Q_i A_{ii}P_i
\ee
with invertible matrices $P_i,Q_i\in M_{n_i}(\F)$. Then we can define
\be
P= \bigoplus^k_{i=1}P_{i},\qquad Q = \bigoplus^k_{i=1}Q_{i}.
\ee

We can see that $P$ and $Q$ are invertible by (i) and Theorem \ref{thm:matrix_invertible_determinant_non_zero} as all fields are integral domain (see Definition \ref{def:integral_domain_ring}). Therefore, by multiplication of block matrices, we have
\beast
B & := & PAQ = \bepm P_1 & & \\ & \ddots & \\ & & P_k \eepm \bepm A_{11} & & 0\\  & \ddots & \\ 0 &  & A_{kk} \eepm \bepm Q_1 & & \\ & \ddots & \\ & & Q_k \eepm \\
& = & \bepm B_1 & & \\ & \ddots & \\ & & B_k \eepm = \bepm \bepm I_{m_1} & 0 \\ 0 & 0\eepm & & \\ & \ddots & \\ & & \bepm I_{m_k} & 0 \\ 0 & 0\eepm \eepm \nonumber
\eeast
which means that $A$ is equivalent to a matrix $B$ having rank $m_1 + \dots + m_k = \sum^k_{i=1}m_i$. Since equivalent matrices have the same rank (Proposition \ref{pro:equivalent_rank}), we can see that $\rank (A) = \sum^k_{i =1} \rank(A_{ii})$.

\item [(iv)] $AB = BA$ if and only if $A_{ii} B_{ii} = B_{ii} A_{ii} $ for any $i$th block where $i=1,\dots,k$. Therefore, the conclusion holds.
\een
\end{proof}


\subsection{Triangular matrices and block triangular matrices}

\begin{definition}[triangular matrix\index{triangular matrix}]\label{def:triangular_matrix}
$A\in M_n(\F)$ is an upper triangular matrix if $a_{ij} = 0$ whenever $i > j$. If $a_{ij} = 0$ whenever $i\geq j$, then $A$ is said to be strictly upper triangular.

$A$ is said to be lower triangular (or strictly upper triangular) if its transpose is upper triangular (or strictly upper triangular).
\end{definition}

\begin{example}
\be
A = \bepm 1 & 2 & 3\\ 0 & 4 & 5 \\ 0 & 0 & 6 \eepm \qquad \text{is upper triangular.}
\ee
\end{example}

\begin{proposition}
The product of two upper triangular matrices is upper triangular.
\end{proposition}

\begin{proof}[\bf Proof]
Let $A,B\in M_n(\F)$ be two upper triangular matrices. Then
\be
a_{ij} = 0, \ b_{ij} = 0,\qquad \forall i>j.
\ee

Thus, $AB = C = \brb{c_{ij}}$ and for any $i>j$
\be
c_{ij} = \sum^n_{k=1}a_{ik}b_{kj} = \sum^n_{k=1} a_{ik} \ind_{i<k} b_{kj}\ind_{k < j} = \sum^n_{k=1} a_{ik} b_{kj}\ind_{i<k < j} = 0.
\ee

Hence $C$ is upper triangular.
\end{proof}

Similarly, we have the following proposition.

\begin{proposition}
The product of two similarly partitioned block upper triangular matrices is block upper triangular.
\end{proposition}



\begin{lemma}\label{lem:upper_triangular_matrix_determinant}
If $A$ is an upper triangular matrix, then $\det A = a_{11} \dots a_{nn}$.
\end{lemma}

\begin{proof}[\bf Proof]
From the definition the determinant (Definition \ref{def:determinant_matrix}),
\be
\det A = \sum_{\sigma \in S_n} \ve(\sigma )a_{1,\sigma (1)} \dots a_{n,\sigma (n)}.
\ee

For a product to contribute, we must have $\sigma (i) \leq  i$ for all $i = 1, \dots, n$. Hence $\sigma (1) = 1$, $\sigma (2) = 2$, $\dots$, $\sigma (n) = n$, so $\sigma  = \iota$ and hence $\det A = a_{11} \dots a_{nn}$.
\end{proof}


\begin{definition}[block triangular matrices]
A matrix $A\in M_n(\F)$ of the form \be A = \bepm A_{11} & & & * \\ & A_{22}  & & \\ & & \ddots & \\ 0 & & & A_{kk} \eepm \ee in which $A_{ii}\in M_{n_i}(\F)$, $i = 1,\dots,k$, $\sum^k_{i=1}n_i = n$ and
``$*$'' denotes any entry, is called block upper triangular. Block lower triangular, strictly block lower triangular and strictly block upper triangular ($A_{ii} = 0$ for $i=1,\dots,k$) may be defined similarly.
\end{definition}


\begin{proposition}\label{pro:block_triangular_matrix_determinant}
If $A \in M_m(\F)$, $B \in M_n(\F)$ and $C \in M_{m,n}(\F)$. Then $\det \bepm A & C\\ 0 & B \eepm = \det A \det B$.
\end{proposition}

\begin{proof}[\bf Proof]
{\bf Approach 1.} Fix $B,C$. Then $d_{B,C} : A \mapsto \det\bepm A & C\\ 0 & B\eepm$ is a volume form on the column space $\F^m$ by Proposition \ref{pro:determinant_matrix_property}. Hence by Theorem \ref{thm:volume_form_determinant}, $d_{B,C}(A) = \det A \det \bepm I & C\\ 0 & B\eepm$.

Now keep $C$ fixed. The map $B \mapsto \det\bepm I & C\\ 0 & B\eepm$ is a volume form on the row space $\F^n$ by Proposition \ref{pro:determinant_matrix_property}. Hence $\det\bepm I & C\\ 0 & B \eepm= \det B\bepm I & C\\ 0 & I \eepm$ by Theorem \ref{thm:volume_form_determinant}.

Now $\det\bepm I & C\\ 0 & I\eepm = 1$ as $\bepm I & C\\ 0 & I\eepm$ is upper triangular matrix. So $\det\bepm A & C\\ 0 & B \eepm = \det A\det B \cdot 1 = \det A\det B$ by Lemma \ref{lem:upper_triangular_matrix_determinant}.

{\bf Approach 2.} Write $X =\bepm A & C\\ 0 & B\eepm$ and expand the expression for the determinant.
\be
\det \bepm A & C\\ 0 & B \eepm = \sum_{\sigma \in S_{m+n}} \ve(\sigma ) \prod^{m+n}_{j=1} x_{\sigma (j)j}
\ee

Note $x_{\sigma (j)j} = 0$ if $j \leq  m$, $\sigma (j) > m$, so we only sum over $\sigma$ with the following properties.
\ben
\item [(i)] For $j\in [1,m]$, $\sigma (j) \in [1,m]$. Here $x_{\sigma (j)j} = a_{\sigma_1(j)j}$ where $\sigma_1\in S_m$ is the restriction of $\sigma$ to $[1,m]$.
\item [(ii)] For $j \in [m + 1,m + n]$, $\sigma (j) \in [m + 1,m + n]$. Here, writing $k = j - m$, we have $x_{\sigma (j)j} = b_{\sigma_2(k)k}$ where $\sigma_2(k) = \sigma (m + k) - m$.
\een

Noting also that $\ve(\sigma ) = \ve(\sigma_1)\ve(\sigma_2)$ for such $\sigma$, we obtain
\be
\det\bepm A & C\\ 0 & B\eepm = \brb{\sum_{\sigma_1\in S_m} \ve(\sigma_1)\prod^m_{j=1} a_{j,\sigma_1(j)}} \brb{\sum_{\sigma_2\in S_k} \ve(\sigma_2) \prod^k_{l=1} a_{j,\sigma_2(j)}} = \det A\det B.
\ee

{\bf Approach 3.} Use $QR$ factorization (see Example \ref{exa:block_triangular_matrix_determinant_qr_factorization}).
\end{proof}

\begin{proposition}
Let $A$ be block upper triangular matrix with diagonal blocks $A_{11},\dots,A_{kk}$. Thus,
\ben
\item [(i)] $\det A = \prod^k_{i=1} \det A_{ii}$.
\item [(ii)] $\rank (A) \geq \sum^k_{i=1} \rank\brb{A_{ii}}$.
\een
\end{proposition}

\begin{proof}[\bf Proof]
\ben
\item [(i)] Direct result from Proposition \ref{pro:block_triangular_matrix_determinant}.

\item [(ii)] Direct result from Proposition \ref{pro:rank_of_block_triangular_matrix}
\een
\end{proof}

\subsection{Permutation matrices and block permutation matrices}

\begin{definition}[permutation matrix\index{permutation matrix}]\label{def:permutation_matrix}
A matrix $P\in M_n(\F)$ is called a permutation matrix if exactly one entry in each row and column is equal to 1, and all other entries are 0.
\end{definition}

\begin{remark}
Multiplication by permutation matrices effects a permutation of the rows or columns of the object multiplied.

In general, left multiplication of a matrix $A\in M_{m,n}(\F)$ by a permutation matrix $P\in M_m(\F)$ permutes the rows of $A$, while right multiplication of a matrix $A_{m,n}(\F)$ by permutation
matrix $P\in M_n(\F)$ permutes the columns of $A$.

Note that elementary matrix $T_{i,j}$ (see Definition \ref{def:elementary_matrix}) is a special case of permutation matrix, called a transposition (similar to the definition in groups).
\end{remark}

\begin{example}
\be
P = \bepm 0 & 1 & 0 \\ 1 & 0 & 0 \\ 0 & 0 & 1 \eepm \in M_3(\F)
\ee
is permutation matrix and
\be
P \bepm 1 \\ 2 \\ 3 \eepm = \bepm 2 \\ 1 \\ 3 \eepm
\ee
is a permutation of the rows (components) of the vector $\brb{1,2,3}^T$.
\end{example}

\begin{proposition}\label{pro:permutation_matrix_determinant_inverse}
For a permutation matrix $P\in M_n(\F)$, we have that $\det P = \pm 1$, which implies that $P$ is necessarily nonsingular. Furthermore, $P^T = P^{-1}$.
\end{proposition}

\begin{proof}[\bf Proof]
Direct result from Laplace expansion (Lemma \ref{lem:determinant_minor_matrix}).

Also, we have that $A := P^TP$ and
\be
a_{ij} = \sum^n_{k=1}p_{ki}p_{kj} = \delta_{ij}
\ee
which implies that $A= I$. Similarly, we have $PP^T = I$ and therefore, $P^{-1} = P^T$.
\end{proof}

Although permutation matrices do not, in general, commute under multiplication, we have the following proposition.

\begin{proposition}
The product of two permutation matrices is again a permutation matrix.
\end{proposition}

\begin{proof}[\bf Proof]
\footnote{proof needed.}
\end{proof}

\begin{remark}
The permutation matrices constitutes a subgroup of $GL(n,\F)$, the group of nonsingular matrices in $M_n(\F)$, which has finite cardinality $n!$. In fact, any permutation matrix is a product of
transpositions.\footnote{details needed.}
\end{remark}

\begin{definition}[block transposition matrix\index{block transposition matrix}]\label{def:block_transposition_matrix}
A matrix $P\in M_n(\F)$ is called a block permutation matrix if it is of the form of
\be
P = \bepm I_{n_1} & & & & & & 0 \\ & \ddots & & & & & \\ & & 0_{n_i\times n_j} & & I_{n_i} &  &  \\ & & & \ddots & & & \\ & & I_{n_j} & & 0_{n_j \times n_i} & & \\  &  & & & & \ddots & \\ 0 & & & & & & I_{n_k} \eepm
\ee
where $n_1 + \dots + n_k = n$ and $i,j = 1,\dots, k$. This matrix is the corresponding matrix of switching blocks $n_i$ and $n_j$ in either row or column sense.
\end{definition}

\begin{remark}
Similar to the transposition matrix in Definition \ref{def:elementary_matrix}, we have that left multiplication of a matrix $A\in M_{m,n}(\F)$ by a block transposition matrix $P\in M_m(\F)$ permutes
the block rows of $A$, while right multiplication of a matrix $A_{m,n}(\F)$ by block transposition matrix $P\in M_n(\F)$ permutes the block columns of $A$.
\end{remark}

Accordingly, we have the definition of block permutation matrix.

\begin{definition}[block permutation matrix\index{block permutation matrix}]\label{def:block_permutation_matrix}
A matrix $P\in M_n(\F)$ is called a block permutation matrix if its block form has exactly one identity matrix (may be of different sizes) in each block row and block column, and all other entries
are 0.
\end{definition}

\begin{remark}
Accordingly, we have every block permutation matrix is a product of block transposition matrices as the block matrix multiplicity has the same rule with matrix multiplicity.
\end{remark}



\subsection{Circulant matrices}

\begin{definition}[circulant matrix\index{circulant matrix}]\label{def:circulant_matrix}
A matrix $A\in M_n(\F)$ of the form
\be
A = \bepm a_1 & a_2 & \dots & \dots & a_n \\ a_n & a_1 & a_2 & \dots & a_{n-1} \\ a_{n-1} & a_n & a_1 & \dots & a_{n-2} \\ \vdots & \vdots & \ddots & \ddots & \vdots \\ a_2 & a_3 & \dots & a_n & a_1 \eepm
\ee
is called a circulant matrix. Each row is just the previous row cycled forward one step, so the entries in each row are just a cyclic permutation of those in the first.
\end{definition}

\begin{definition}[basic circulant permutation matrix\index{basic circulant permutation matrix}]\label{def:basic_circulant_permutation_matrix}
The permutation matrix $C\in M_n(\F)$,
\be
C = \bepm 0 & 1 & 0 & \dots & 0\\ 0 & 0 & 1 & \dots & 0 \\ 0 & 0 & 0 & \dots & 0 \\ \vdots & \vdots & \ddots & \ddots & 1 \\ 1 & 0 & \dots & 0 & 0 \eepm
\ee
is called the basic circulant permutation matrix. A matrix $A\in M_n(\F)$ can be written in the form
\be
A = \sum^{n-1}_{k=0} a_{k+1} C^k
\ee
if and only if $A$ is a circulant where $C^0 = C^n = I_n$ and coefficients $a_1,a_2,\dots,a_n$ are just the entries of the first row of $A$.
\end{definition}

\begin{proposition}\label{pro:product_of_circulant_matrices_is_circulant}
The product of two circulant matrices is again a circulant matrix.
\end{proposition}

\begin{remark}
This proposition can be generalized as the rows are cycled forward (or backward) a fixed number of steps that is greater than 1.
\end{remark}

\begin{proof}[\bf Proof]
By Definition \ref{def:basic_circulant_permutation_matrix}, we can write two circulant matrices by
\be
A = \sum^{n-1}_{k=0} a_{k+1} C^k,\qquad B = \sum^{n-1}_{k=0} b_{k+1} C^k.
\ee

Thus, we have $\brb{C^k}_{ij} = 1$ if $(j-i)\bmod n = k$. Then for $X = AB$,
\beast
X_{ij} & = & \sum^n_{k=1} A_{ik}B_{kj} =  \sum^n_{k=1} \brb{\sum^{n-1}_{s=0} a_{s+1} C^s_{ik}}\brb{\sum^{n-1}_{t=0} b_{t+1} C^t_{kj}}
= \sum^n_{k=1} \brb{\sum^{n-1}_{s=0} a_{s+1} \delta_{\brb{(k-i)\bmod n}, s}}\brb{\sum^{n-1}_{t=0} b_{t+1} \delta_{\brb{(j-k)\bmod n},t}}\\
& = & \sum^n_{k=1} a_{\brb{(k-i)\bmod n}+1} b_{\brb{(j-k)\bmod n} + 1}.
\eeast

Thus, for fixed $l\in \bra{0,1,\dots,n-1}$, all $(i,j)$ entries with $(j-i)\bmod n = l$ have the same value. Then we can set this value
\beast
c_{l+1} & = & \sum^n_{k=1} a_{\brb{(k-i)\bmod n}+1} b_{\brb{(j-k)\bmod n} + 1} = \sum^{n}_{k=1} a_{\brb{(k-i)\bmod n}+1} b_{\brb{(l+i-k)\bmod n} + 1} \\
& = & \sum^{n}_{k=1} a_{\brb{k\bmod n}+1} b_{\brb{(l-k)\bmod n} + 1} = \sum^{n-1}_{k=0} a_{\brb{k\bmod n}+1} b_{\brb{(l-k)\bmod n} + 1}.   \qquad (*)
\eeast

Therefore, we can write
\be
X =  \sum^{n-1}_{l=0} c_{l+1} C^l.
\ee

Thus, $AB$ is also a circulant matrix. %\be%a_{ij} = \sum^{n-1}_{k=0} a_{k+1} \brb{C^k}_{ij} = %%AB = \sum^{n-1}_{k=0} a_{k+1} C^k \sum^{n-1}_{l=0} b_{l+1} C^l = \sum^{n-1}_{w=0} \sum^{w}_{k=0} a_{k+1}b_{w-k+1} C^kC^{w-k}  +  \sum^{2n-2}_{w=n} \sum^{n-1}_{k= w-n+1} a_{k+1}b_{w-k+1} C^kC^{w-k}%\ee
\end{proof}

\begin{example}
Let
\be
A = \bepm a_1 & a_2 & a_3 \\ a_3 & a_1 & a_2 \\ a_2 & a_3 & a_1 \eepm ,\qquad B = \bepm b_1 & b_2 & b_3 \\ b_3 & b_1 & b_2 \\ b_2 & b_3 & b_1 \eepm.
\ee

Then
\be
C = AB = \bepm a_1b_1 + a_2b_3 + a_3b_2 & a_1 b_2 + a_2b_1 + a_3b_3 & a_1b_3 + a_2b_2 + a_3b_1\\ a_1b_3 + a_2b_2 + a_3b_1 & a_1b_1 + a_2b_3 + a_3b_2 & a_1 b_2 + a_2b_1 + a_3b_3 \\ a_1 b_2 + a_2b_1 + a_3b_3 & a_1b_3 + a_2b_2 + a_3b_1 & a_1b_1 + a_2b_3 + a_3b_2 \eepm
\ee
which is consistent with the result ($*$) in proof of Proposition \ref{pro:product_of_circulant_matrices_is_circulant}.
\end{example}

\begin{proposition}
Two circulant matrices of the same size commute.
\end{proposition}

\begin{proof}[\bf Proof]
Direct result from ($*$) in proof of Proposition \ref{pro:product_of_circulant_matrices_is_circulant} by switching $a$ and $b$ and substituting $k$ by $(l-k)\bmod n$.
\end{proof}

\subsection{Toeplitz matrices and Hankel matrices}

\begin{definition}[Toeplitz\footnote{Otto Toeplitz (1881–1940) was a professor in Bonn, Germany, but because of his Jewish back- ground he was dismissed from his chair by the Nazis in 1933. In addition to the matrix that bears his name, Toeplitz is known for his general theory of infinite-dimensional spaces devel- oped in the 1930s.} matrix\index{Toeplitz matrix}]\label{def:toeplitz_matrix}
A matrix $A\in M_{n+1}(\F)$ of the form
\be
A = \bepm a_0 & a_1 & a_2 & \dots & a_n \\ a_{-1} & a_0 & a_1 & \dots & a_{n-1} \\ a_{-2} & a_{-1} & a_0 & \dots & \vdots \\ \vdots & \vdots & \ddots & \ddots & a_1 \\ a_{-n} & a_{-n+1} & \dots & a_{-1} & a_0 \eepm
\ee
is called a Toeplitz matrix.

The general term $a_{ij} = a_{j-i}$ for some given sequence $a_{-n},a_{-n+1},\dots,a_{-1},a_0,a_1,a_2,\dots, a_{n-1},a_n\in \F$. The entries of $A$ are constant down the diagonals parallel to the
main diagonal.

The Toeplitz matrices
\be
B = \bepm 0 & 1 & \dots & 0 \\ 0 & 0 & & \vdots \\ \vdots & & \ddots  & 1 \\ 0 & & & 0 \eepm,\qquad F = \bepm 0 & 0 & \dots & 0 \\ 1 & 0 & & \vdots \\ \vdots & & \ddots  &  \\ 0 & \dots & 1 & 0 \eepm
\ee
are called the ``backward shift'' and ``forward shift'' because of their effect on the elements of the standard basis $\bra{e_1,\dots,e_{n+1}}$ in $\F^{n+1}$.

A matrix $A\in M_{n+1}(\F)$ can be written in the form
\be
A = \sum^n_{k=1}a_{-k}F^k + \sum^n_{k=0}a_k B^k
\ee
if and only if $A$ is Toeplitz matrix.
\end{definition}

\begin{remark}
Toeplitz matrices arise naturally in problems involving trigonometric moments.\footnote{details needed.}
\end{remark}

%\subsection{}

\begin{definition}[Hankel matrix\index{Hankel matrix}]\label{def:hankel_matrix}
A matrix $A\in M_{n+1}(\F)$ of the form
\be
A = \bepm a_0 & a_1 & a_2 & \dots & a_n \\ a_{1} & a_2 & a_3 & \dots & a_{n+1} \\ a_{2} & a_{3} & a_4 & \dots & a_{n+2} \\ \vdots & \vdots & \ddots & \ddots & \vdots \\ a_{n} & a_{n+1} & \dots & \dots & a_{2n} \eepm
\ee
is called a Hankel matrix.

The general term $a_{ij} = a_{i+j-2}$ for some given sequence $a_0,a_1,\dots,a_{2n-1},a_{2n}$. The entries of $A$ are constant along the diagonals perpendicular tot he main diagonal.
\end{definition}

\begin{remark}
Hankel matrices arise naturally in problems involving power moments.\footnote{details needed.}
\end{remark}

\begin{proposition}
For the backward identity permutation
\be
P = \bepm 0 & & & & 1 \\ & & & 1 & \\ & & \iddots & & \\ & 1 & & & \\ 1 & & & & 0\eepm,
\ee

$PT$ is a Hankel matrix for any Toeplitz matrix $T$ and $PH$ is a Toeplitz matrix for any Hankel matrix $H$.
\end{proposition}

\begin{remark}
It can be seen that any Toeplitz matrix is a product of two symmetric matrices ($P$ and a Hankel matrix).
\end{remark}

\begin{proof}[\bf Proof]
This is direct result from matrix multiplication and the fact $P = P^{T} = P^{-1}$ for permutation matrix $P$ (Proposition \ref{pro:permutation_matrix_determinant_inverse}).
\end{proof}



\subsection{Hessenberg matrices and tridiagonal matrices}

\begin{definition}[Hessenberg matrix]\label{def:hessenberg_matrix}
The matrix $A = \brb{a_{ij}} \in M_n(\F)$ is said to be in upper Hessenberg form or to be an upper Hessenbery matrix if $a_{ij} = 0$ for $i>j+1$,
\be
A = \bepm
a_{11} & a_{12} & & \dots & & a_{1n} \\
a_{21} & a_{22} & &\dots  & & a_{2n} \\
0 & a_{32} & & & & a_{3n} \\
0 & 0 & a_{43} & \ddots & & \vdots \\
\vdots & \vdots & & & \ddots &  \\
0 & 0 & \dots & 0 & a_{n,n-1} & a_{nn}
\eepm
\ee

The matrix $A\in M_n(\F)$ is called lower Hessenberg if $A^T$ is upper Hessenberg.
\end{definition}

\begin{definition}
A matrix $A = \brb{a_{ij}} \in M_n(\F)$ that is both upper and lower Hessenberg is called tridiagonal, that is, $A$ is tridiagonal if $a_{ij} = 0$, whenever $\abs{i-j} >1$,
\be
A =
\bepm
a_{11} & a_{12} & 0 & 0 & \dots & 0 \\
a_{21} & a_{22} & a_{32} & 0 & \dots & 0 \\
0 & a_{32} & a_{33} & a_{34} & \dots & 0 \\
0 & 0 & \ddots & \ddots & \ddots & \vdots \\
\vdots & \vdots & & \ddots & \ddots & a_{n-1,n}\\
0 & 0 & \dots & 0 & a_{n,n-1} & a_{nn}
\eepm
\ee
\end{definition}

\begin{proposition}
For a tridiagonal matrix $A\in M_n(\F)$ and the set $\alpha_k = \bra{1,\dots,k}$,
\be
\det\brb{A_{\alpha_{k+1}}} = a_{k+1,k+1} \det\brb{A_{\alpha_k}} - a_{k+1,k}a_{k,k+1}\det\brb{A_{\alpha_{k-1}}},\qquad k=2,\dots,n-1.
\ee
\end{proposition}

\begin{proof}[\bf Proof]
Direct result from Laplace expansion (Lemma \ref{lem:determinant_minor_matrix}).
\end{proof}



\subsection{Vandermonde matrices and Lagrange interpolating polynomials}


\begin{definition}[Vandermonde matrix\index{Vandermonde matrix}]\label{def:vandermonde_matrix}
A Vandermonde matrix $V\in M_{m,n}(\F)$ is a matrix of the form
\be
V = \bepm
1 & x_1 & x_1^2 & x_1^3 & \dots & x_1^{n-1} \\
1 & x_2 & x_2^2 & x_2^3 & \dots & x_2^{n-1} \\
\vdots & \vdots & \vdots & \vdots &  & \vdots \\
1 & x_m & x_m^2 & x_m^3 & \dots & x_m^{n-1} \\
\eepm
\ee
where $x_1,\dots,x_m\in \F$. That is, $v_{ij} = x_i^{j-1}$.
\end{definition}

\begin{proposition}\label{pro:vandermonde_determinant}
The determinant of a square Vandermonde matrix (where $m = n$) can be expressed as:
\be
\det V = \prod_{1\le i<j\le n} (x_j-x_i).
\ee
\end{proposition}

\begin{remark}
This means that square Vandermonde matrix is nonsingular if and only if the $n$ parameters $x_1,\dots,x_n$ are distinct.
\end{remark}

\begin{proof}[\bf Proof]%If $x_i = x_j$ for some $i\neq j$, then $\det V = 0$ by Proposition \ref{pro:determinant_matrix_property}.(iv).assume that $x_i$ are distinct and
First replace the $i$th row of $V$ by
\be
\bepm 1 & t & t^2 & \dots & t^{n-1} \eepm.
\ee

Then, take the determinant. The determinant is a function of $t$ is called $V_i(t)$. It is a polynomial in $t$ of degree $n-1$ at most. Hence, it has $n-1$ roots\footnote{theorem needed here. The fundamental theorem of algebra combined with the factor theorem states that the polynomial of degree $n$ has $n$ roots in the complex plane, if they are counted with their multiplicities.}.
Furthermore, $V_i(x_j)$ for any $j$ between 1 and $n$ except $i$ makes the determinant 0 by Proposition \ref{pro:determinant_matrix_property}.(iv). Hence, $x_j$ with $i\neq j$ is a root of the
polynomial $V_i(t)$. Thus, $(t - x_j)$ is a factor in the expansion of $V_i(t)$. Since $t$ is of degree $n-1$ at most, we have that %Repeating this with each of the i rows tells us that that
\be
\det V = V_i(x_i) = C_n \prod_{1\leq i<j\leq n} (x_j - x_i),
\ee
where $C_n$ is a constant. The fact that $C_n=1$ follows from induction.

It is easy to check that $\det V_{2\times 2} = x_2 -x_1$ satisfies the condition. Now assume the conclusion holds for $k$ such that $C_k = 1$ and
\be
\det V_{k\times k} = C_k \prod_{1\leq i<j\leq k} (x_j - x_i).
\ee

Thus, For $V_{(k+1)\times(k+1)}$, we can use Laplace expansion (Lemma \ref{lem:determinant_minor_matrix}) to have that the coefficient of $x_{k+1}^{k}$, the bottom right entry of
$V_{(k+1)\times(k+1)}$, in its determinant is
\be
(-1)^{k+1+k+1} \det V_{k\times k} = C_k \prod_{1\leq i<j\leq k} (x_j - x_i) = \prod_{1\leq i<j\leq k} (x_j - x_i)
\ee
which implies that $C_{k+1} = 1$. Therefore, we get the required result.
\end{proof}


\begin{proposition}
Let $V$ be a Vandermonde matrix with $x_i$ and $x_i\neq x_j$ for all $i\neq j$. Then the columns of $V$ form a linearly independent set whenever $n\leq m$ (proof in \cite{Meyer_2001}.$P_{185}$).
\end{proposition}

\begin{proof}[\bf Proof]
If
\be
\bepm
1 & x_1 & x_1^2 & x_1^3 & \dots & x_1^{n-1} \\
1 & x_2 & x_2^2 & x_2^3 & \dots & x_2^{n-1} \\
\vdots & \vdots & \vdots & \vdots &  & \vdots \\
1 & x_m & x_m^2 & x_m^3 & \dots & x_m^{n-1} \\
\eepm \bepm a_0\\ a_1 \\ \vdots \\ a_{n-1} \eepm = \bepm 0 \\ 0\\ \vdots \\ 0 \eepm,
\ee
then for each $i = 1,2,\dots,m$,
\be
a_0 + a_1 x_i + a_2 x_i^2 + \dots + a_{n-1}x_i^{n-1} = 0.
\ee
has $m$ distinct roots, namely, the $x_i$'s.

However, the degree of $p(x)$ is smaller than $n-1$ and the fundamental theorem of algebra\footnote{theorem needed.} guarantees that if $p(x)$ is not the zero polynomial, then
$p(x)$ can have at most $n-1$ distinct roots. Therefore, $a_0=a_1=\dots = a_{n-1}$, which means that the column of $V$ form a linearly independent set.
\end{proof}



\begin{definition}[Lagrange interpolation polynomial\index{Lagrange interpolation polynomial}]\label{def:lagrange_interpolation_polynomial}
Let $x_1,\dots,x_n,y_1,\dots,y_n\in \F$ with distinct $x_1,\dots,x_n$ and $X = \brb{x_1,\dots,x_n}^T$, $Y = \brb{y_1,\dots,y_n}^T$. We can define Lagrange interpolation polynomial of degree $n-1$ by
\be
L(t,X,Y) = \sum^n_{i=1} \brb{y_i\ \frac{\prod^n_{j\neq i}\brb{t - x_j}}{\prod^n_{j\neq i}\brb{x_i - x_j}}}.
\ee
\end{definition}

\begin{proposition}[interpolation problem (see \cite{Meyer_2001}.$P_{186}$)]
Given a set of $n$ points $S = \bra{(x_1,y_1),(x_2,y_2),\dots,(x_n,y_n)}$ in which the $x_i$'s are distinct, then Lagrange interpolation polynomial, $L(t,X,Y)$ with $X = \brb{x_1,\dots,x_n}^T$ and $Y
= \brb{y_1,\dots,y_n}^T$, is the unique polynomial of degree less than $n-1$ that passes through each point of $S$.
\end{proposition}

\begin{proof}[\bf Proof]
The polynomial $p(x) = a_0 + a_1 x + \dots + a_{n-1}x^{n-1}$ must satisfy the equations
\beast
y_1 & = & p(x_1) = a_0 + a_1 x_1 + \dots + a_{n-1}x_1^{n-1}\\
& \vdots & \\
y_n & = & p(x_n) = a_0 + a_1 x_n + \dots + a_{n-1}x_n^{n-1}
\eeast

Writing this $n\times n$ system as
\be
\bepm
1 & x_1 & x_1^2 & x_1^3 & \dots & x_1^{n-1} \\
1 & x_2 & x_2^2 & x_2^3 & \dots & x_2^{n-1} \\
\vdots & \vdots & \vdots & \vdots &  & \vdots \\
1 & x_n & x_n^2 & x_n^3 & \dots & x_n^{n-1} \\
\eepm \bepm a_0\\ a_1 \\ \vdots \\ a_{n-1} \eepm = \bepm y_1 \\ y_2\\ \vdots \\ y_n\eepm
\ee

reveals that the coefficient matrix is a square Vandermonde matrix $V$, so Proposition \ref{pro:vandermonde_determinant} guarantees that it is nonsingular. Consequently, the system has a unique
solution, and thus there is one and only one possible set of coefficients for polynomial $p(x)$. Thus, by the fact $a = V^{-1} Y$ and Theorem \ref{thm:adjugate_inverse_matrix}.(ii), we have the
Lagrange interpolation polynomial $L(t,X,Y)$ is this only polynomial of degree less than $n-1$.\footnote{Note that coefficient $a_i$ could be zero.}
\end{proof}

%\section{Matrix Anaylsis}

%\section{}

% Another use of Schur's result is to make it clear that every matrix is `almost' diagonalizable in two possible interpretations of the phrase. The first says that arbitarily close to a given matrix
% there is a diagonalizable matrix, and the second says that any given matrix is similar to an upper triangular matrix whose off-diagonal entries are arbitarily small\cite{}.

\subsection{Hadamard matrices and Sylvester matrices}


\begin{definition}[Hadamard matrix\index{Hadamard matrix}]\label{def:hadamard_matrix}
A Hadamard matrix, named after the French mathematician Jacques Hadamard, is a square matrix is called a Hadamard matrix if each entry is 1 or -1 and the rows are mutually orthogonal.
\end{definition}


\begin{definition}[Sylvester matrices\index{Sylvester matrices}]\label{def:sylvester_matrix}
Examplesof Hadamard matrices were actually first constructed by James Joseph Sylvester in 1867. Let $H$ be Hadamard matrix of order $n$. Then the partitioned matrix $\bepm H & H \\ H & -H \eepm$ is a Hadamard matrix of order $2n$. This observation can be applied repeatedly and leads to the following sequence of matrices, also called Walsh matrices.
\be
H_1 = \bepm 1 \eepm,\qquad H_2 = \bepm 1 & 1 \\ 1 & -1 \eepm,\qquad H_{2^k} = \bepm H_{2^{k-1}} & H_{2^{k-1}} \\ H_{2^{k-1}} & - H_{2^{k-1}} \eepm = H_2 \otimes H_{2^{k-1}},
\ee
for $2\leq k\in \N$, where $\otimes$ denotes the Kronecker product (see Section \ref{sec:kronecker_product}).

In this manner, Sylvester constructed Hadamard matrices of order $2^k$ for every non-negative integer $k$.
\end{definition}

\begin{remark}
Sylvester matrices have a number of special properties. They are symmetric and, when $k\geq 1$,have trace zero. The elements in the first column and the first row are all positive. The elements in all the other rows and columns are evenly divided between positive and negative. Sylvester matrices are closely connected with Walsh functions.
\end{remark}

%If we map the elements of the Hadamard matrix using the group homomorphism $\bra{1,-1,\times}\mapsto \bra{0,1,\oplus}$, we can describe an alternative construction of Sylvester
%
%\begin{definition}[alternative construction of Hadamard matrix]
%
%\end{definition}

\section{Eigenvalues and Eigenvectors}

\subsection{The eigenvalue-eigenvector equation}

\begin{definition}[eigenvalue\index{eigenvalue}, eigenvector\index{eigenvector}]\label{def:eigenvalue_eigenvector}
If $A\in M_n(\F)$ ($\F = \R$ or $\C$) and $x\in \C^n$, we consider the equation
\be
A x = \lm x,\qquad x\neq 0
\ee
where $\lm\in \C$ is a scalar. If a scalar $\lm$ and a non-zero vector $x$ happen to satisfy this equation, then $\lm$ is called eigenvalue of $A$ and $x$ is called eigenvector of $A$ associated with $\lm$.
Also, the equation is called eigenvalue-eigenvector equation\index{eigenvalue-eigenvector equation}.
\end{definition}

\begin{remark}
Note that the eigenvalue and eigenvector occur inextricably as a pair, and that an eigenvector cannot be the zero vector.

Also, if $x$ is an eigenvector associated with the eigenvalue $\lm$ of $A$, any non-zero scalar multiple of $x$ is an eigenvector as well.
\end{remark}



\begin{definition}[spectrum of matrix\index{spectrum!matrix}]
The set of all $\lm\in \C$ that are eigenvalues of $A\in M_n(\F)$ is called the spectrum of $A$ and is denoted by $\sigma(A)$.

The spectral radius\index{spectral radius!matrix} of $A$ is the non-negative real number
\be
\rho(A) = \max\bra{\abs{\lm}:\lm \in \sigma(A)}.
\ee

It is just the radius of the smallest disc centered at the origin in the complex plane that includes all the eigenvalues of $A$.
\end{definition}

\begin{example}
Consider the matrix
\be
A = \bepm 7 & -2 \\ 4 & 1 \eepm \in M_2(\R).
\ee

Then we have $3\in \sigma(A)$ with $(1,2)^T$ as an associated eigenvector since
\be
A \bepm 1 \\ 2 \eepm = \bepm 3\\ 6 \eepm = 3\bepm 1\\ 2\eepm.
\ee

Also, $5\in \sigma(A)$ with associated eigenvector $(1,1)^T$.
\end{example}

\begin{proposition}\label{pro:zero_eigenvalue_singular_equivalent}
$0\in \sigma(A)$ iff $A$ is singular.
\end{proposition}

\begin{proof}[\bf Proof]
Direct result from Definition \ref{def:singularity_matrix}.
\end{proof}







\begin{proposition}\label{pro:polynomial_matrix_eigenvalue_eigenvector}
Let $p(A)$ be a polynomial of matrix $A\in M_n(\F)$. If $\lm$ is an eigenvalue of $A$, while $x$ is an associated eigenvector, then $p(\lm)$ is an eigenvalue of the matrix $p(A)$ and $x$ is an
eigenvector of $p(A)$ associated with $p(\lm)$.
\end{proposition}

\begin{proof}[\bf Proof]
Consider $p(A)x$. First,
\be
p(A)x = a_n A^n x + \dots + a_1 A x + a_0 x.
\ee

Second, $A^n x = A^{n-1}A x = A^{n-1}\lm x = \lm A^{n-1} x = \dots =\lm^n x$ by repeated application of the eigenvalue-eigenvector equation. Thus,
\be
p(A)x = a_n \lm^n x + \dots + a_1 \lm x + a_0 x = \brb{a_n\lm^n + \dots + a_1 \lm + a_0}x = p(\lm) x
\ee
which is the required result by definition of eigenvalue and eigenvector.
\end{proof}

\begin{example}
If $\sigma(A) = \bra{-1,2}$ for $A\in M_2(\F)$, then $\sigma(A^2) = \bra{1,4}$ by Proposition \ref{pro:polynomial_matrix_eigenvalue_eigenvector}.
\end{example}


\begin{lemma}\label{lem:distinct_eigenvalues_implies_linearly_independent_eigenvectors}
Suppose that $\lm_1,\dots,\lm_k$ are distinct eigenvalues of $A\in M_n(\F)$, and suppose that $x^{(i)}$ is an eigenvector associated with $\lm_i$, $i=1,\dots,k$. Then $\bra{x^{(1)},\dots,x^{(k)}}$ is a
linearly independent set.
\end{lemma}

\begin{proof}[\bf Proof]
The proof is essentially by contradiction (see \cite{Horn_Johnson_1990}.$P_{47}$).

Suppose that $x^{(1)},\dots,x^{(k)}$ is actually a linearly independent set. Then there is a nontrivial linear transformation which produces the zero vector, and in fact there is such a linear combination
with the fewest non-zero coefficients. Suppose that such a minimal linear dependence relation is
\be
a_1 x^{(1)} + \dots + a_r x^{(r)} = 0,\qquad r\leq k.\qquad (*)
\ee

We have $r>1$ because all $x^i \neq 0$ for $i = 1,\dots,r$. We may assume for convenience (renumber if necessary) that it involves the first $r$ vectors. We also have
\be
A \brb{a_1 x^{(1)} + \dots + a_r x^{(r)}} = a_1 A x^{(1)} + \dots + a_r A x^{(r)} = a_1 \lm_1 x^{(1)} + \dots + a_r \lm_r x^{(r)} = 0\qquad (\dag)
\ee
by $(*)$, which is another dependence relation. Now we have $(\dag) - (*)\times \lm_r$ is
\be
a_1 (\lm_1 - \lm_r) x^{(1)} + \dots + a_{r-1}(\lm_{r-1}-\lm_r) x^{(r-1)} = 0, \qquad (**)
\ee
which has fewer non-zero coefficients than ($*$). Note that $(**)$ is nontrivial since $\lm_i \neq \lm_r$ for $i = 1,\dots, r-1$. This contradicts the minimality assumption for ($*$).
\end{proof}


\subsection{Characteristic polynomial}

%The may be rewritten equivalently as
%\be
%(A - \lm I ) x = 0,\qquad x\neq 0.
%\ee
%
%Thus, $\lm \in \sigma(A)$ if and only if $A - \lm I$ is a singular matrix. By Proposition \ref{pro:invertible_non_singular_equivalent} and Theorem \ref{thm:matrix_invertible_determinant_non_zero}, we have
%\be
%\det(A - \lm I) = 0.
%\ee

\begin{definition}[characteristic polynomial of matrix\index{characteristic polynomial!matrix}]\label{def:characteristic_polynomial_matrix}
The characteristic polynomial of $A\in M_n(\F)$ is defined by
\be
p_A(t) = \det\brb{tI - A}.
\ee
\end{definition}

\begin{example}
For $A = \bepm 1 & 2 \\ 3 & 4 \eepm$, we have
\be
p_A(t) = \det \bepm t-1 & -2 \\ -3 & t-4 \eepm = (t-1)(t-4) - 2\times 3 = t^2 -5t - 2.
\ee
\end{example}

\begin{proposition}\label{pro:characteristic_polynomial_root_coincide_spectrum}
If $A\in M_n(\F)$, the characteristic polynomial $p_A(t)$ has degree $n$ and the set of roots of $p_A(t) = 0$ coincides with $\sigma(A)$.
\end{proposition}

\begin{remark}
It means that the matrix has $n$ distinct eigenvalues at most\footnote{How can we say polynomial with degree $n$ has $n$ roots at most}.
\end{remark}

\begin{proof}[\bf Proof]
That $p_A(t)$ has degree $n$ follows inductively from Laplace expansion of $\det(tI -A)$: each row of $tI -A$ contributes one and only one power of $t$ as the determinant is expanded.

By Proposition \ref{pro:invertible_non_singular_equivalent} and Theorem \ref{thm:matrix_invertible_determinant_non_zero}, we have
\be
0 = p_A(\lm) = \det\brb{\lm I - A} \ \lra \ A - \lm I \text{ is singular matrix}\ \lra \ \lm \in \sigma(A) \ \lra \  (A - \lm I ) x = 0,\quad x\neq 0,
\ee
which is the equivalent form of eigenvalue-eigenvector equation (see Definition \ref{def:eigenvalue_eigenvector}) .
\end{proof}

\begin{definition}[elementary symmetric function\index{elementary symmetric function}]\label{def:elementary_symmetric_function}
Let $A\in M_n(\F)$ with eigenvalues $\lm_1,\dots,\lm_n$ by repeating eigenvalues according to multiplicity. The $k$th ($k\leq n$) elementary symmetric function of the eigenvalues is
\be
S_k(A) := S_k(\lm_1,\dots,\lm_n) = \sum_{1\leq i_1\leq \dots \leq i_k\leq n}\brb{ \prod^k_{j=1} \lm_{i_j}},
\ee
the sum of all $\binom{n}{k}$ $k$-fold products of distinct items from $\lm_1,\dots,\lm_n$. Note that $S_0(\lm_1,\dots,\lm_n) = 1$.
\end{definition}

\begin{remark}
Note that elementary symmetric function can be expressed in the form of elementary symmetric polynomials\footnote{details needed.}.
\end{remark}

\begin{example}
If $n= 4$, then
\beast
S_1(\lm_1,\lm_2,\lm_3,\lm_4) & = & \lm_1 + \lm_2 + \lm_3 + \lm_4,\\
S_2(\lm_1,\lm_2,\lm_3,\lm_4) & = & \lm_1 \lm_2 + \lm_1 \lm_3 + \lm_1 \lm_4 + \lm_2 \lm_3 + \lm_2 \lm_4 + \lm_3 \lm_4,\\
S_3(\lm_1,\lm_2,\lm_3,\lm_4) & = & \lm_1 \lm_2 \lm_3 + \lm_1 \lm_2 \lm_4 + \lm_1 \lm_3 \lm_4 + \lm_2 \lm_3 \lm_4,\\
S_4(\lm_1,\lm_2,\lm_3,\lm_4) & = & \lm_1 \lm_2 \lm_3 \lm_4.
\eeast
\end{example}

\begin{definition}\label{def:sum_of_determinants_of_principal_submatrices}
Recalling Definition \ref{def:principal_submatrix}, there are $\binom{n}{k}$ different $k\times k$ principal submatrices of $A\in M_n(\F)$ and the sum of their determinants is denoted by $E_k(A)$.

In particular, $E_1(A) = \sum^n_{i = 1}a_{ii}$ is the trace of $A$, $\tr A$. Also, $E_n(A) = \det A$.

By convention, we have $\det(A_\emptyset) = 1$ and thus, $E_0(A) = 1$.
\end{definition}


\begin{theorem}\label{thm:elementary_symmetric_function_sum_of_determinant_principal_submatrix_equivalent}
If $\lm_1,\dots,\lm_n$ are the eigenvalues of $A\in M_n(\F)$, then
\be
S_k(A) = S_k(\lm_1,\dots,\lm_n) = E_k(A).
\ee

That is, the $k$th elementary symmetric function of eigenvalues of $A$ is the sum of the $k\times k$ determinants of principal submatrix of $A$ (see \cite{Meyer_2001}.$P_{495}$). In particular,
\be
\tr A = \sum^n_{i=1} \lm_i,\qquad \det A = \prod^n_{i=1} \lm_i. \ee
\end{theorem}

\begin{remark}
Note that Theorem \ref{thm:elementary_symmetric_function_sum_of_determinant_principal_submatrix_equivalent} implies that
\be
0\in \sigma(A) \ \ra \ \det A = 0 \ \ra \ A\text{ is singular (not invertible)}.
\ee
\end{remark}

\begin{proof}[\bf Proof]
By Proposition \ref{pro:characteristic_polynomial_root_coincide_spectrum}, we have
\be
p_A(t) = \prod^n_{i=1} (t-\lm_i) = t^n - S_1(\lm_1,\dots,\lm_n)t^{n-1} + \dots \pm S_n(\lm_1,\dots,\lm_n) = \sum^n_{k=0} (-1)^{k} S_k(\lm_1,\dots,\lm_n)t^{n-k}
\ee
which can be verified directly by picking out the coefficient of $t^k$ in the product.

Taking $k$ differentiation of $p_A(t)$ with respect to $t$, we have
\be
(n-k)!(-1)^{k} S_k(\lm_1,\dots,\lm_n) = p^{(n-k)}_A(0)\qquad (*).
\ee

By Proposition \ref{pro:derivative_of_determinant} (as $p_A(t)$ is a determinant), we have for $0\leq r\leq n$,
\be
p^{(r)}_A(t) = \sum_{i_1,\dots,i_r\text{ are distinct}} D_{i_1,\dots,i_r}(t),
\ee
where $D_{i_1,\dots,i_r}(t)$ is the determinant of the matrix identical to $tI -A$ expect that rows $i_1,\dots,i_r$ have been replaced by $e^T_{i_1},\dots, e^T_{i_r}$, respectively\footnote{If $i_j
= i_k$ for some $j\neq k$, we have that the determinant should be zero by Proposition \ref{pro:derivative_of_determinant}}. It follows that
\be
D_{i_1,\dots,i_r}(0) = (-1)^{n-r}\det\brb{A_{i_1,\dots,i_r}},
\ee
where $A_{i_1,\dots,i_r}$ is idential to $A$ except that $i_1,\dots,i_r$ have been replaced by $e^T_{i_1},\dots,e^T_{i_r}$, respectively.

Furthermore, $\det\brb{A_{i_1,\dots,i_r}}$ is the determinant of $(n-r)\times (n-r)$ principal submatrix obtained by deleting rows and columns $i_1,\dots,i_r$ from $A$. Thus,
\be
p^{(r)}_A(0) = \sum_{i_1,\dots,i_r\text{ are distinct}} D_{i_1,\dots,i_r}(0) = (-1)^{n-r} r! E_{n-r}(A)\qquad (\dag).
\ee

Note that $r!$ appears because each of the $r!$ permutations of the subscripts on $A_{i_1,\dots,i_r}$ describes the same matrix. Hence, let $k = n-r$ and combine $(*)$ and $(\dag)$, we have
\be
(n-k)!(-1)^{k} S_k(\lm_1,\dots,\lm_n) = p^{(n-k)}_A(0) = (-1)^{k} (n-k)! E_{n-r}(A) \ \ra\ S_k(\lm_1,\dots,\lm_n) = E_{n-r}(A),
\ee
as required.
\end{proof}

\begin{example}
If $A\in M_2(\F)$, then
\be
p_A(t) = t^2 - t \cdot \tr A  + \det A
\ee
and that
\be
\sum_{\lm_i \in \sigma(A)} \lm_i = \tr A,\qquad \prod_{\lm_i \in \sigma(A)} \lm_i = \det A.
\ee
\end{example}



\begin{theorem}\label{thm:product_matrices_change_order_have_the_same_eigenvalues}
Let $A,B\in M_n(\F)$. Then $AB$ and $BA$ have the same eigenvalues.
\end{theorem}

\begin{remark}
This means the transformation of $A$ and $B$ will give the same scaling effect.
\end{remark}

\begin{proof}[\bf Proof]
Recall Newton's identities by which the $n$ elementary symmetric polynomials in $n$ variables are expressed in terms of the $n$ sums of powers (see Proposition \ref{pro:newton_identities}). It suffices to prove that, for $1\leq m\leq n$,
\be
\lm_1^m(AB) + \dots + \lm_n^m(AB) = \lm_1^m(BA) + \dots + \lm_n^m(BA)\qquad (*)
\ee
where $\lm^m_i(C)$ is the $i$th eigenvalue of matrix $C$. Then by Newton's identities we can get the same elementary symmetric polynomials (function of $x_1,\dots, x_n$), $S_m(x_1,\dots,x_n)$ for $AB$ and $BA$. Then with the relation
\be
p_C(t) = \prod^n_{i=1} (t-\lm_i(C)) = \sum^n_{m=0} (-1)^{m} S_m(\lm_1,\dots,\lm_n)t^{n-m},
\ee
we get the $p_{AB}(t) = p_{BA}(t)$ and this implies the same eigenvalues of $AB$ and $BA$.

Note that the eigenvalues of $C^m$ are the $m$th powers of the eigenvalues of $C$ (by Proposition \ref{pro:polynomial_matrix_eigenvalue_eigenvector}).  Thus,
\be
\sum^n_i \brb{\lm_i(C)}^m = \sum_i^n \lm_i(C^m) = \tr\brb{C^m}
\ee
by Theorem \ref{thm:elementary_symmetric_function_sum_of_determinant_principal_submatrix_equivalent}. Thus, the statement $(*)$ is equivalent to
\be
\tr\brb{(AB)^m} = \tr\brb{(BA)^m}
\ee

By Proposition \ref{pro:trace_change_order}, we have $\tr(AB) = \tr(BA)$. Therefore,
\be
\tr\brb{(AB)^m} = \tr\brb{ABAB\dots AB} = \tr\brb{BABA\dots BA} = \tr\brb{(BA)^m}
\ee
which will give the required conclusion $(*)$.

Note that this can be apparoached by alternative proof.\footnote{For alternative proof, see Eigenvalues of $AB$ and $BA$, Rajendra Bhatia, Department Classroom, Resonance, January 2002, Vol 7, Issue 1, pp 88-93.}
\end{proof}


\subsection{Eigenvalues and eigenvectors of special matrices}




\begin{proposition}
An idempotent matrix is always diagonalizable and its eigenvalues are either 0 or 1.
\end{proposition}

\begin{proof}[\bf Proof]
\footnote{see Horn Roger, p148.}
\end{proof}



\begin{proposition}\label{pro:eigenvalue_tridiagonal_toeplitz_matrix}
Let $A$ be Toeplitz matrix with
\be
A = \bepm
b & a & & & \\
c & b & a & &  \\
& \ddots & \ddots & \ddots & \\
& & c & b & a \\
& & & c & b
\eepm, \qquad a\neq 0\neq c.
\ee

Then the eigenvalues and eigenvectors of $A$ are given by
\be
\lm_k = b+ 2\sqrt{ac}\cos\brb{\frac{k\pi}{n+1}},\qquad x_k = \bepm
\brb{\frac ca}^{1/2}\sin\brb{\frac{k\pi}{n+1}} \\
\brb{\frac ca}^{2/2}\sin\brb{\frac{2k\pi}{n+1}} \\
\vdots\\
\brb{\frac ca}^{n/2}\sin\brb{\frac{nk\pi}{n+1}} \\
\eepm
\ee
for $k = 1,\dots,n$ and therefore conclude that $A$ is diagonalizable.
\end{proposition}

\begin{proof}[\bf Proof]
For an eigenpair $(\lm, x)$, the components in $(A - \lm I) x = 0$ are
\be
c x_{k-1}+(b-\lm)x_k+a x_{k+1} = 0,\qquad k = 1,\dots,n,\qquad x_0 = x_{n+1} = 0
\ee
or, equivalently,
\be
x_{k+2}+ \frac{b-\lm}a x_{k+1}+ \frac ca x_k =0,\qquad k=0,...,n-1,\qquad x_0 =x_{n+1} =0.
\ee

These are second-order homogeneous difference equations, and solving them is similar to solving analogous differential equations. The technique is to seek solutions of the form $x_k = \xi r^k$ for constants $\xi$ and $r$. This produces the quadratic equation
\be
r^2 + (b - \lm)r/a + c/a = 0
\ee
with roots $r_1$ and $r_2$, and it can be argued that the general solution of
\be
x_{k+2} + \frac{b -\lm}a x_{k+1} + \frac ca x_k = 0
\ee
is
\be
x_k = \left\{\ba{ll}
\alpha r_1^k + \beta r_2^k & r_1 \neq r_2 \\
\alpha \rho^k + \beta k \rho^k \quad\quad & r_1 = r_2 = \rho
\ea\right.
\ee
where $\alpha$ and $\beta$ are arbitrary constants.

For the eigenvalue problem at hand, $r_1$ and $r_2$ must be distinc - otherwise
\be
x_k = \alpha \rho^k + \beta k\rho^k, \quad x_0 = x_{n+1} = 0
\ee
implies each $x_k = 0$, which is impossible because $x$ is an eigenvector. Hence
\be
x_k = \alpha r_1^k + \beta r_2^k, \quad x_0 = x_{n+1} = 0
\ee
yields
\be
\left\{\ba{l}
0 = \alpha + \beta \\
0 = \alpha r_1^{n+1} + \beta r_2^{n+1}
\ea\right\} \quad \ra\quad
\brb{\frac{r_1}{r_2}}^{n+1} = \frac{-\beta}{\alpha}=1 \quad \ra\quad \frac{r_1}{r_2} = e^{\frac{2k\pi i}{n+1}},
\ee

so $r_1 = r_2e^{\frac{2k\pi i}{n+1}}$ for some $1 \leq j \leq n$. Couple this with
\be
r^2 + \frac{(b-\lm)r}c + \frac ca =  (r - r_1)(r -r_2)  \quad \ra \quad  \left\{\ba{l}
r_1r_2 = \frac ca\\
r_1 + r_2 = -\frac{b - \lm}a
\ea\right.
\ee
to conclude that
\be
r_1 = \sqrt{\frac ca} e^{\frac{k\pi i}{n+1}},\quad  r_2 = \sqrt{\frac ca} e^{-\frac{k\pi i}{n+1}},
\ee
and
\be
\lm = b+ \sqrt{ac} \brb{e^{\frac{k\pi i}{n+1}} + e^{-\frac{k\pi i}{n+1}}} =b+2\sqrt{ac}\cos \brb{\frac{k\pi}{n+1}}.
\ee

Therefore, the eigenvalues of $A$ must be given by
\be
\lm_k =b+2\sqrt{ac}\cos\brb{\frac{k\pi}{n+1}} ,\quad  k=1,2,\dots,n.
\ee

Since these $\lm_k$'s are all distinct ($\cos \theta$ is a strictly decreasing function of $\theta$ on $(0, \pi)$, and $a \neq 0\neq c$), $A$ must be diagonalizable (see Theorem \ref{thm:distinct_eigenvalues_implies_diagonalizable}).

Finally, the $j$th component of any eigenvector associated with $\lm_k$ satisfies
\be
x_j = \alpha r_1^j + \beta r_2^j,\quad  \alpha + \beta=0,
\ee
so
\be
x_j = \alpha \brb{\frac ca}^{j/2}\brb{e^{\frac{jk\pi i}{n+1}} - e^{-\frac{jk\pi i}{n+1}}} = 2i\alpha \brb{\frac ca}^{j/2}\sin\brb{\frac{jk\pi}{n+1}}.
\ee

Setting $\alpha = \frac 1{2i}$ yields a particular eigenvector associated with $\lm_k$ as
\be
\lm_k = b+ 2\sqrt{ac}\cos\brb{\frac{k\pi}{n+1}},\qquad x_k = \bepm
\brb{\frac ca}^{1/2}\sin\brb{\frac{k\pi}{n+1}} \\
\brb{\frac ca}^{2/2}\sin\brb{\frac{2k\pi}{n+1}} \\
\vdots\\
\brb{\frac ca}^{n/2}\sin\brb{\frac{nk\pi}{n+1}} \\
\eepm
\ee

Because the $\lm_k$s are distinct, $\brb{x_1 , x_2 , \dots , x_n }$ is a complete linearly independent set\footnote{explanation needed. see (7.2.3) in \cite{Meyer_2001}} .
\end{proof}

\begin{example}
Recall Example \ref{exa:determinant_special_tridiagonal_matrices}. We can have that the determinant of matrix
\be
A = \bepm
x & 1 & & & & \\ 1 & x & 1 & & & \\ & 1 & x & 1 & & \\ & & & \ddots & & \\ & & &  1 & x & 1 \\ & & & & 1 & x
\eepm
\ee
is $\prod^n_{k=1} \brb{x - 2\cos\brb{\frac {k\pi}{n+1}}}$. Thus, the eigenvalues of $A$ are ($x' =x-\lm$)
\be
\lm = x - 2\cos\brb{\frac {k\pi}{n+1}} = x + 2\cos\brb{\frac {k\pi}{n+1}},\qquad k = 1,\dots,n
\ee
which is consistent with Proposition \ref{pro:eigenvalue_tridiagonal_toeplitz_matrix} when $b=x$, $a=c=1$.
\end{example}

\begin{theorem}\label{thm:inverse_of_toeplitz_matrix_plus_ab}
Recall Example \ref{exa:determinant_special_tridiagonal_matrices} and define %$R_n(x,a,b)\in M_c(\C)$ with
\be
R_n(x,a,b) = \bepm
x+a & 1 & & & & \\ 1 & x & 1 & & & \\ & 1 & x & 1 & & \\ & & & \ddots & & \\ & & &  1 & x & 1 \\ & & & & 1 & x +b
\eepm,\qquad
\phi_n(x) = \det\brb{P_n(x)} = \det\bepm
x & 1 & & & & \\ 1 & x & 1 & & & \\ & 1 & x & 1 & & \\ & & & \ddots & & \\ & & &  1 & x & 1 \\ & & & & 1 & x
\eepm
\ee

Then we can let $x=2\cos \theta$ and extend the definition of $\phi_n(x)$ (see ($*$) in Example \ref{exa:determinant_special_tridiagonal_matrices}) so that, even when $n$ is not integer,
\beast
\phi_n\brb{2\cos\theta} & = & \frac{\sin\brb{(n+1)\theta}}{\sin\theta},\quad \theta\neq 0,\pi,\\
\phi_n(2) & = & n+1,\\
\phi_n(-2) & = & (-1)^n(n+1).
\eeast

Then we can define that
\be
R := R_n\brb{x, -\frac{\phi_{\alpha-1}(x)}{\phi_\alpha(x)}, -\frac{\phi_{\beta-1}(x)}{\phi_\beta(x)}}.
\ee
and the element $c_{ij}$ of $C = R^{-1}$ is therefore
\be
c_{ij} = c_{ji} = \frac{(-1)^{i+j}\phi_{\alpha+i-1}(x)\phi_{\beta+n-j}(x)}{\phi_{\alpha+\beta+n}(x)},\qquad i\leq j.\qquad (*)
\ee
\end{theorem}

\begin{proof}[\bf Proof]
First, it is obvious that $C$ is symmetric as $R$ is symmetric, i.e., $c_{ij} = c_{ji}$. Then we consider $CR$ and get for $ c_{1k}$, $k =1,\dots,n$
\beast
1 & = & c_{11}(x+a) + c_{12} ,\\
0 & = & c_{11} + xc_{12} + c_{13} ,\\
& \vdots & \\
0 & = & c_{1,n-2} + xc_{1,n-1} + c_{1,n} ,\\
0 & = & c_{1,n-1} + (x+b)c_{1,n}.
\eeast

Thus we can convert this to the problem
\be
0 = c_{1,k-1} +  x c_{1,k} + c_{1,k+1},\quad k=1,\dots,n
\ee
with boundary condition $c_{1,0} = ac_{11}-1$ and $c_{1,n+1} = bc_{1,n}$. Therefore, the auxillary equation is
\be
z^2 + xz + 1 = 0 \ \ra\ z = \frac{-x\pm \sqrt{x^2 - 4}}{2}.
\ee

If $x=2$, we have $z=-1$
\be
c_{1,k} = A(-1)^k + Bk(-1)^k,\quad k = 1,\dots,n
\ee
with boundary condition
\be
\ba{l}
A = -a(A+B)-1\\
A + B(n+1)= -b\brb{A+Bn}
\ea \ \ra\ c_{1,k} = \frac{\brb{(b+1)(n-k)+1}(-1)^{k+1}}{n(a+1)(b+1) + 1-ab}.
\ee

Let
\beast
a =-\frac{\phi_{\alpha-1}(2)}{\phi_\alpha(2)} = -\frac{\alpha}{\alpha+1},\quad
b = -\frac{\phi_{\beta-1}(2)}{\phi_\beta(2)} = -\frac{\beta}{\beta+1}.
\eeast

Then
\be
c_{1,k}= \frac{(-1)^{k+1}(\alpha +1)\brb{n+\beta + 1-k}}{n+ \alpha + \beta +1} = \frac{(-1)^{k+1}\phi_{\alpha + 1-1}(2)\phi_{n+\beta-k}(2)}{\phi_{n+\alpha+\beta}(2)}.
\ee

Then by induction, assume that the formula $(*)$ holds for $c_{i-1,j-1} = c_{j-1,i-1} $ where $i\leq j$. We can use $c_{i,k}$, $k=i,\dots,n$ with
\beast
1 & = & c_{i,i-1} + xc_{i,i} + c_{i,i+1}\\
0 & = & c_{i,i} + xc_{i,i+1} + c_{i,i+2} \\
& \vdots & \\
0 & = & c_{i,n-2} + xc_{i,n-1} + c_{i,n} \\
0 & = & c_{i,n-1} + (x+b)c_{i,n}.
\eeast

Similarly, we have that
\be
c_{i,k} = A(-1)^k + Bk(-1)^k,\quad k = i,\dots,n
\ee
with boundary condition
\be
\ba{l}
c_{i,i-1}-1 = (A+B(i-1))(-1)^{i-1}\\
A + B(n+1)= -b\brb{A+Bn}
\ea \ \ra\ A = \frac{\brb{c_{i,i-1}-1}(-1)^{i-1}\brb{-n - \frac 1{b+1}}}{i-1-n-\frac 1{b+1}},\ B = \frac{\brb{c_{i,i-1}-1}(-1)^{i-1}}{i-1-n-\frac 1{b+1}}
\ee

Thus, for $x=2$, $\frac 1{b+1} = \beta + 1$,
\be
c_{i,k} = \frac{\brb{c_{i,i-1}-1}(-1)^{i+k-1}\brb{k-n-\frac 1{b+1}}}{i-1-n-\frac 1{b+1}} =  \frac{\brb{c_{i,i-1}-1}(-1)^{i+k-1}\brb{k-n-\beta -1}}{i-1-n-\beta - 1},\quad k=i,\dots,n.
\ee

Accroding to $(*)$ we have that (note that $(*)$ holds for $i\leq j$)
\be
c_{i,i-1} = c_{i-1,i} =\frac{(-1)^{2i-1}\phi_{\alpha+i-2}(2)\phi_{\beta+n-i}(2)}{\phi_{\alpha+\beta+n}(2)} = -\frac{\brb{\alpha+i-1}\brb{\beta + n -i +1}}{\alpha+\beta+n+1}.
\ee

Thus,
\beast
c_{i,k} & = & (-1)^{i+k}\brb{\frac{\brb{\alpha+i-1}\brb{\beta + n -i +1}}{\alpha+\beta+n+1} + 1}\frac{n+\beta + 1 - k}{n+\beta + 2-i} \\
& = &  (-1)^{i+k}\frac{\brb{\alpha+i}\brb{\beta + n -i +2}}{\alpha+\beta+n+1}\frac{n+\beta + 1 - k}{n+\beta + 2-i} = (-1)^{i+k}\frac{\brb{\alpha+i}\brb{\beta + n -k +1}}{\alpha+\beta+n+1} \\
& = & \frac{(-1)^{i+k}\phi_{\alpha+i-1}(2)\phi_{\beta+n-k}(2)}{\phi_{\alpha+\beta+n}(2)}
\eeast
which gives the required results. For $x=-2$, we have $z=1$ and the similar result.

Now we assume that $x=2\cos\theta$ and $\theta \neq 0,\pi$. With the similar argument, we have
\be
z = -e^{\pm i\theta}, \qquad c_{1,k} = (-1)^k\brb{A\cos(k\theta) + B\sin(k\theta)},\qquad k=1,\dots, n
\ee
with boundary condition  $c_{1,0} = ac_{11}-1$ and $c_{1,n+1} = bc_{1,n}$, i.e.,
\be
\ba{l}
A = -a(A\cos\theta + B\sin\theta) -1 \\
A\cos\brb{(n+1)\theta} + B\sin\brb{(n+1)\theta} = -b\brb{A\cos\brb{n\theta} + B\sin\brb{n\theta}}
\ea
\ee
which implies that
\beast
c_{1,k} & = & (-1)^{k+1}\frac{b\sin((n-k)\theta) + \sin\brb{(n+1-k)\theta}}{\sin\brb{(n+1)\theta}+(a+b)\sin(n\theta) + ab\sin\brb{(n-1)\theta}}
\eeast

%\beast
%c_{1,k} & = & \frac{\brb{b\sin(n\theta) - \sin\brb{(n+1)\theta}}\cos(k\theta) + \brb{-b\cos(n\theta) + \cos\brb{(n+1)\theta}}\sin(k\theta)}{\sin\brb{(n+1)\theta}-(a+b)\sin(n\theta) + ab\sin\brb{(n-1)\theta}} \\
%& = & \frac{b\sin((n-k)\theta) - \sin\brb{(n+1-k)\theta}}{\sin\brb{(n+1)\theta}-(a+b)\sin(n\theta) + ab\sin\brb{(n-1)\theta}}
%\eeast

Let
\beast
a =-\frac{\phi_{\alpha-1}(x)}{\phi_\alpha(x)} = -\frac{\sin\brb{\alpha\theta}}{\sin\brb{(\alpha+1)\theta}},\quad
b = -\frac{\phi_{\beta-1}(x)}{\phi_\beta(x)} = -\frac{\sin\brb{\beta\theta}}{\sin\brb{(\beta+1)\theta}}.
\eeast

Then $c_{1,k}$ is
\beast
\frac{(-1)^{k+1}\sin((\beta+1)\theta) \brb{-\sin(\beta\theta)\sin((n-k)\theta) + \sin((\beta+1)\theta)\sin\brb{(n+1-k)\theta}}}{\sin((\alpha+1)\theta)\sin((\beta+1)\theta)\sin\brb{(n+1)\theta}-\brb{\sin(\alpha\theta)\sin((\beta+1)\theta) + \sin((\alpha+1)\theta)\sin(\beta\theta)} \sin(n\theta) + \sin(\alpha\theta)\sin(\beta\theta)\sin\brb{(n-1)\theta}}
\eeast

By Proposition \ref{pro:trigonometric_product_difference}, we have
\beast
\sin((\alpha+1)\theta)\sin((\beta+1)\theta)\sin\brb{(n+1)\theta} - \sin((\alpha+1)\theta)\sin(\beta\theta)\sin(n\theta) & = & \sin((\alpha+1)\theta)\sin((\beta+n+1)\theta) \sin\theta \\
\sin(\alpha\theta)\sin((\beta+1)\theta) \sin(n\theta)  - \sin(\alpha\theta)\sin(\beta\theta)\sin\brb{(n-1)\theta}& = & \sin(\alpha\theta)\sin((\beta+n)\theta)\sin\theta.
\eeast

Therefore, we can have
\beast
\sin((\alpha+1)\theta)\sin((\beta+n+1)\theta) \sin\theta - \sin(\alpha\theta)\sin((\beta+n)\theta)\sin\theta & = & \sin((\alpha+\beta+n+1)\theta)\brb{\sin\theta}^2 \\
\sin((\beta+1)\theta)\sin\brb{(n+1-k)\theta}-\sin(\beta\theta)\sin((n-k)\theta) & = & \sin\brb{(\beta+n+1-k)\theta} \sin\theta.
\eeast

Hence,
\be
c_{1,k} = \frac{(-1)^{k+1}\sin((\beta+1)\theta) \sin\brb{(\beta+n+1-k)\theta} }{\sin((\alpha+\beta+n+1)\theta)\sin\theta} = \frac{(-1)^{k+1}\phi_{\alpha+1-1}(x)\phi_{n+\beta-k}(x)}{\phi_{\alpha+\beta+n}}.
\ee

Then by induction, assume that the formula $(*)$ holds for $c_{i-1,j-1} = c_{j-1,i-1} $ where $i\leq j$. We can use $c_{i,k}$, $k=i,\dots,n$ with
\beast
1 & = & c_{i,i-1} + xc_{i,i} + c_{i,i+1}\\
0 & = & c_{i,i} + xc_{i,i+1} + c_{i,i+2} \\
& \vdots & \\
0 & = & c_{i,n-2} + xc_{i,n-1} + c_{i,n} \\
0 & = & c_{i,n-1} + (x+b)c_{i,n}.
\eeast

Similarly, we have that
\be
c_{i,k} = (-1)^k \brb{A\cos(k\theta) + B\sin(k\theta)},\quad k = i,\dots,n
\ee
with boundary condition
\beast
c_{i,i-1}-1 & = & (-1)^{i-1}\brb{A\cos\brb{(i-1)\theta}+ B\sin\brb{(i-1)\theta}}\\
A\cos((n+1)\theta) + B\sin((n+1)\theta) & = &  - b\brb{A\cos(n\theta) + B\sin(n\theta)}
\eeast

Then we have
\be
A = \frac{\brb{c_{i,i-1}-1}(-1)^{i-1}\sin\brb{(n+\beta+1)\theta}}{\sin\brb{(n+\beta+2-i)\theta}},\ B = -\frac{\brb{c_{i,i-1}-1}(-1)^{i-1}\cos\brb{(n+\beta+1)\theta}}{\sin\brb{(n+\beta+2-i)\theta}}
\ee

Thus, for $x=2$, $\frac 1{b+1} = \beta + 1$,
\be
c_{i,k} = \frac{\brb{c_{i,i-1}-1}(-1)^{i+k-1}\sin\brb{(n+\beta+1-k)\theta}}{\sin\brb{(n+\beta+2-i)\theta}},\quad k=i,\dots,n.
\ee

Accroding to $(*)$ we have that (note that $(*)$ holds for $i\leq j$)
\be
c_{i,i-1} = c_{i-1,i} =\frac{(-1)^{2i-1}\phi_{\alpha+i-2}(x)\phi_{\beta+n-i}(x)}{\phi_{\alpha+\beta+n}(x)} = -\frac{\sin\brb{(\alpha+i-1)\theta}\sin\brb{(\beta + n -i +1)\theta}}{\sin\brb{(\alpha+\beta+n+1)\theta}\sin\theta}.
\ee

%We now that
%\beast
%\cos\brb{(\alpha+\beta+n)\theta} & = & \cos\brb{(\alpha+\beta+n+1-1)\theta} = \cos\brb{(\alpha+\beta+n+1)\theta}\cos\theta + \sin \brb{(\alpha+\beta+n+1)\theta} \sin\theta  \\
%& = & \cos\brb{(\alpha+i-1 + \beta+n+1-i)\theta} \\
%& = & \cos\brb{(\alpha+i-1)\theta}\cos\brb{(\beta+n+1-i)\theta} - \sin\brb{(\alpha+i-1)\theta}\sin\brb{(\beta+n+1-i)\theta}.
%\eeast

Thus, by Proposition \ref{pro:trigonometric_product_difference},
\beast
c_{i,k} & = & \frac{(-1)^{i+k}\brb{\sin\brb{(\alpha+i-1)\theta}\sin\brb{(\beta + n -i +1)\theta} + \sin\brb{(\alpha+\beta+n+1)\theta}\sin\theta}}{\sin\brb{(\alpha+\beta+n+1)\theta}\sin\theta} \frac{\sin\brb{(n+\beta+1-k)\theta}}{\sin\brb{(n+\beta+2-i)\theta}} \\
& = & \frac{(-1)^{i+k}\sin\brb{(\alpha+i)\theta}\sin\brb{(\beta + n -i+2 )\theta}} {\sin\brb{(\alpha+\beta+n+1)\theta}\sin\theta} \frac{\sin\brb{(n+\beta+1-k)\theta}}{\sin\brb{(n+\beta+2-i)\theta}} \\
& = & \frac{(-1)^{i+k}\sin\brb{(\alpha+i)\theta}\sin\brb{(n+\beta+1-k)\theta} }{\sin\brb{(\alpha+\beta+n+1)\theta}\sin\theta} = \frac{(-1)^{i+j}\phi_{\alpha+i-1}(x)\phi_{\beta+n-k}(x)}{\phi_{\alpha+\beta+n}(x)}
\eeast
as required.
\end{proof}

%

\begin{example}\label{exa:inverse_of_toeplitz_matrix_plus_10}
If $x=-2$, $\alpha = \infty$ and $\beta = 0$, then $a = 1$ and $b=0$. Therefore, by Theorem \ref{thm:inverse_of_toeplitz_matrix_plus_ab} we have for $i\leq j$,
\beast
c_{ij} = c_{ji} & = & \frac{(-1)^{i+j}\phi_{\alpha+i-1}(-2)\phi_{\beta+n-j}(-2)}{\phi_{\alpha+\beta+n}(-2)}\\
& = & \frac{(-1)^{i+j}(-1)^{\alpha+i-1}(\alpha+i)(-1)^{\beta+n-j}(\beta+n+1-j)}{(-1)^{\alpha+\beta+n}(\alpha+\beta+n+1)} \\
& = & -\brb{n+1-j}
\eeast

This implies that
\be
\brb{-R_n(-2, 1,0)}^{-1} = \bepm
1 & -1 & & & & \\
-1 & 2 & -1 & & & \\
& -1 & 2 & -1 & & \\
& & & \ddots & & \\
& & & -1 & 2 & -1 \\
& & & & -1 & 2
\eepm^{-1} = \bepm
n & n-1 & n-2 & \cdots & 2 & 1 \\
n-1 & n-1 & n-2 & \cdots & 2 & 1 \\
n-2 & n-2 & n-2 & \cdots & 2 & 1 \\
\vdots & \vdots & \vdots & \ddots & \vdots & \vdots\\
2 & 2 & 2 & \cdots & 2 & 1 \\
1 & 1 & 1 & \cdots & 1 & 1
\eepm := C.
\ee

From Example \ref{exa:determinant_special_tridiagonal_matrices}, we know that
\be
\det\brb{R_n(x,1,0)} = \prod^n_{k=1} \brb{x - 2\cos\brb{\frac{2k\pi}{2n+1}}}.
\ee

Thus, we can set $x=-2-\lm$ to get the eigenvalues of $R_n\brb{-2,1,0}$, i.e.,
\be
\lm = -2\brb{1+\cos\brb{\frac{2k\pi}{2n+1}}} = -4\cos^2\brb{\frac{k\pi}{2n+1}},\qquad k=1,\dots,n.
\ee

Therefore, the eigenvalues of $C$ are
\be
\frac 14\sec^2\brb{\frac{k\pi}{2n+1}},\qquad k=1,\dots,n
\ee
which are the inverses of eigenvalues of $-R_n(-2,1,0)$ (This can be seen in \cite{Rutherford_1946}).

Similarly, we have
\be
\bepm
2 & -1 & & & & \\
-1 & 2 & -1 & & & \\
& -1 & 2 & -1 & & \\
& & & \ddots & & \\
& & & -1 & 2 & -1 \\
& & & & -1 & 1
\eepm^{-1} = \bepm
1 & 1 & 1 & \cdots & 1 & 1 \\
1 & 2 & 2 & \cdots & 2 & 2 \\
1 & 2 & 3 & \cdots & 3 & 3 \\
\vdots & \vdots & \vdots & \ddots & \vdots & \vdots\\
1 & 2 & 3 & \cdots & n-1 & n-1 \\
1 & 2 & 3 & \cdots & n-1 & n
\eepm
\ee
\end{example}

\begin{example}
From Example \ref{exa:determinant_special_tridiagonal_matrices}, we can see that
\be
\det\brb{R_n(x,-1,1)} = \prod^n_{k=1} \brb{x - 2\cos\brb{\frac{(2k-1)\pi}{2n}}}.
\ee

If $x=-2$, we can have that the eigenvalues of $ R_n(-2,-1,1)$ is
\be
\lm = -2\brb{1+\cos\brb{\frac{(2k-1)\pi}{2n}}} = -4\cos^2\brb{\frac{(2k-1)\pi}{4n}},\qquad k=1,\dots,n.
\ee

Therefore, the eigenvalues of $-\frac 12 R_n(-2,-1,1)$ are
\be
\lm = 2\cos^2\brb{\frac{(2k-1)\pi}{4n}},\qquad k=1,\dots,n.
\ee

Thus, if $\alpha = -1/2$ and $\beta = \infty$, then $a = -1$ and $b=1$. Then by Theorem \ref{thm:inverse_of_toeplitz_matrix_plus_ab} we have for $i\leq j$ the inverse of $-R_n(-2,-1,1)$
\beast
c_{ij} = c_{ji} & = & \frac{(-1)^{i+j}\phi_{\alpha+i-1}(-2)\phi_{\beta+n-j}(-2)}{\phi_{\alpha+\beta+n}(-2)}\\
& = & (-1)^{i+j}(-1)^{\alpha +i-1}(-1)^{\beta+n-j}(-1)^{\alpha+\beta+n} (\alpha+i)\\
& = & -\brb{\alpha+i} = -\frac 12\brb{2i-1}.
\eeast

Thus,
\beast
\brb{-R_n(-2,-1,1)}^{-1} = \frac 12\bepm
1 & 1 & 1 & \cdots & 1 & 1 \\
1 & 3 & 3 & \cdots & 3 & 3 \\
1 & 3 & 5 & \cdots & 5 & 5 \\
\vdots & \vdots & \vdots & \ddots & \vdots & \vdots\\
1 & 3 & 5 & \cdots & 2n-3 & 2n-3 \\
1 & 3 & 5 & \cdots & 2n-3 & 2n-1
\eepm \ \ra \ \brb{-\frac 12 R_n(-2,-1,1)}^{-1} = \bepm
1 & 1 & 1 & \cdots & 1 & 1 \\
1 & 3 & 3 & \cdots & 3 & 3 \\
1 & 3 & 5 & \cdots & 5 & 5 \\
\vdots & \vdots & \vdots & \ddots & \vdots & \vdots\\
1 & 3 & 5 & \cdots & 2n-3 & 2n-3 \\
1 & 3 & 5 & \cdots & 2n-3 & 2n-1
\eepm
\eeast

Therefore, the eigenvalues of
\be
C := \bepm
1 & 1 & 1 & \cdots & 1 & 1 \\
1 & 3 & 3 & \cdots & 3 & 3 \\
1 & 3 & 5 & \cdots & 5 & 5 \\
\vdots & \vdots & \vdots & \ddots & \vdots & \vdots\\
1 & 3 & 5 & \cdots & 2n-3 & 2n-3 \\
1 & 3 & 5 & \cdots & 2n-3 & 2n-1
\eepm
\ee
are the inverses of eigenvalues of $-\frac 12 R_n(-2,-1,1)$
\be
\lm = \frac 12\sec^2\brb{\frac{(2k-1)\pi}{4n}},\qquad k=1,\dots,n.
\ee

This result can be seen in \cite{Rutherford_1952}. This result was also mentioned in Dundan, W. J. 1950, ``Dependence of Errors in the Natural Frequencies of the Numbers of Elements in a Segmental Representation of a Uniform Shaft'', College of Aeronautics, Cranfield, Report. The original paper can not be found.

Now we want to check the eigenvalues of $C$ by other method. For $x\in \C^n$ with $x\neq 0$,
\be
\bepm
1-\lm & 1 & 1 & \cdots & 1 & 1 \\
1 & 3-\lm & 3 & \cdots & 3 & 3 \\
1 & 3 & 5-\lm & \cdots & 5 & 5 \\
\vdots & \vdots & \vdots & \ddots & \vdots & \vdots\\
1 & 3 & 5 & \cdots & 2n-3-\lm & 2n-3 \\
1 & 3 & 5 & \cdots & 2n-3 & 2n-1-\lm
\eepm x =0 \ \ra\ \left\{\ba{l}
(1-\lm)x_1 + x_2 + x_3 + \dots + x_n = 0\\
x_1 + (3-\lm)x_2 + 3x_3 + \dots + 3x_n = 0\\
\qquad\quad \vdots\\
x_1 + 3x_2 + 5x_3 + \dots + (2n-1-\lm)x_n = 0
\ea\right.
\ee

Then we have
\be
 \left\{\ba{r}
2(1-\lm)x_1 + 2x_2 + 2x_3 + \dots + x_n = 0\\
\lm x_1 + (2-\lm)x_2 + 2x_3 + \dots + 2x_n = 0\\
\qquad\quad \vdots\\
 \lm x_{n-1}  + (2-\lm)x_n = 0
\ea\right. \ \ra\ \lm x_{k-1} + 2(1-\lm)x_k + \lm x_{k+1} 0,\quad k=1,\dots,n
\ee
with boundary condition $x_0 = -x_1$ and $x_n = x_{n+1}$. Thus, the auxiliary equation is
\be
\lm x^2 + 2(1-\lm)x + \lm = 0 \ \ra \ x =\frac{2(\lm-1)\pm 2\sqrt{1-2\lm}}{2\lm} = \frac{\lm-1\pm\sqrt{1-2\lm}}{\lm}
\ee

If $\lm = 1/2$ we have that
\be
x_k = A(-1)^k + Bk(-1)^k,\qquad k=0,1,\dots,n+1
\ee
satisfying
\beast
-A & = &  -(A+B), \\
(A+Bn) & = & -(A+B(n+1)).
\eeast

This gives that $x_k = 0$, which is a contradiction. Thus, we have
\be
x_k = A r_1^k + B r_2^k,\qquad k=0,1,\dots,n,n+1
\ee
where $r_1+r_2 = 2(\lm-1)/\lm = 2-2/\lm$  and $r_1r_2 = 1$. Therefore
\be
\left\{\ba{l}
A+B = -(Ar_1 + Br_2),\\
Ar_1^n + Br_2^n = Ar_1^{n+1} + Br_2^{n+1}
\ea\right. \ \ra\ \left\{\ba{l}
A+B r_1^{-1} = 0,\\
Ar_1^n - Br_1^{-(n+1)} = 0
\ea\right.
\ee

Thus,
\be
B \brb{r_1^{2n}+1} = 0 \ \ra\ r_1 = e^{\frac{(2k-1)\pi i}{2n}},\quad k=1,\dots,n.
\ee

Thus,
\be
2-\frac 2{\lm} = r_1 + r_2 = 2\cos\brb{\frac{(2k-1)\pi }{2n}} \ \ra\ \frac 1{\lm} = 1- \cos\brb{\frac{(2k-1)\pi }{2n}} = 2\sin^2\brb{\frac{(2k-1)\pi }{4n}} ,\quad k=1,\dots,n.
\ee

Therefore,
\be
\frac 1{\lm} = 2\cos^2\brb{\frac {\pi}2 - \frac{(2k-1)}{4n}} = 2\cos^2\brb{\frac{(2n-2k+1)\pi}{4n}} = 2\cos^2\brb{\frac{2k-1)\pi}{4n}},\qquad k=1,\dots,n.
\ee

Hence,
\be
\lm_k = \frac 12\sec^2\brb{\frac{2k-1)\pi}{4n}},\qquad k=1,\dots,n
\ee
which is consistent with the previous result.
\end{example}

\begin{example}\label{exa:eigenvalue_one_matrix_with_zero_diagonal}
Define $A_n,B_n\in M_n(\F)$ by
\be
A_n = \bepm
0 & 1 & 1 & & \dots & & 1 & 1\\
1 & 0 & 1 & & \dots & & 1 & 1\\
1 & 1 & 0 & & \dots & & 1 & 1 \\
1 & 1 & & & \ddots & & 1 & 1 \\
& & & & & \ddots & &  \\
1 & 1 & 1 & & \dots & & 1 & 0
\eepm,\qquad B_n = \bepm
-1 & -1 & -1 & & \dots & & -1 & -1\\
-1 & \lm & -1 & & \dots & & -1 & -1\\
-1 & -1 & \lm & & \dots & & -1 & -1 \\
-1 & -1 & -1 & & \ddots & & -1 & -1 \\
& & & & & \ddots & &  \\
-1 & -1 & -1 & & \dots & & -1 & \lm
\eepm.
\ee

Also, we can define $f(n) = \det\brb{\lm I_n -A_n}$ and $g(n) = \det B_n$. Then its eigenvalues are $n-1$ and -1 ($n-1$ times). It can be verified by the following arguments. %For $n=2k,k\in\Z$,
\be
f(n) = \det\bepm
\lm & -1 & -1 & & \dots & & -1 & -1\\
-1 & \lm & -1 & & \dots & & -1 & -1\\
-1 & -1 & \lm & & \dots & & -1 & -1 \\
-1 & -1 & -1 & & \ddots & & -1 & -1 \\
& & & & & \ddots & &  \\
-1 & -1 & -1 & & \dots & & -1 & \lm
\eepm = \lm f(n-1) + (-1)^2(n-1) g(n-1)
\ee

Also,
\be
g(n) = \det \bepm
-1 & -1 & -1 & & \dots & & -1 & -1\\
-1 & \lm & -1 & & \dots & & -1 & -1\\
-1 & -1 & \lm & & \dots & & -1 & -1 \\
-1 & -1 & -1 & & \ddots & & -1 & -1 \\
& & & & & \ddots & &  \\
-1 & -1 & -1 & & \dots & & -1 & \lm
\eepm = (-1)f(n-1) + (-1)^2(n-1)g(n-1)
\ee
with boundary condition $f(1) = \lm$ and $g(1) = -1$. Thus,
\be
f(2) = \lm\cdot f(1) + 1\cdot g(1) = \lm^2 -1 = (\lm-1)(\lm+1), \qquad g(2) = -f(1) + 1\cdot g(1) = -(\lm +1).
\ee
\be
f(3) = \lm\cdot f(2) + 2\cdot g(2) = \lm(\lm^2-1) - 2(\lm+1) = (\lm+1)\brb{\lm^2-\lm-2} = (\lm-2)(\lm+1)^2
\ee
\be
g(3) = (-1)\cdot f(2) + 2\cdot g(2) = -(\lm^2 -1) - 2(\lm+1) = -(\lm+1)^2
\ee

Thus we can guess that $f(k) = (\lm-k+1)(\lm+1)^{k-1}$ and $g(k) = -(\lm+1)^{k-1}$ for $k\in \Z$. Then
\beast
f(k+1) & = & \lm (\lm-k+1)(\lm+1)^{k-1} - k(\lm+1)^{k-1} = \brb{\lm(\lm-k+1) - k}(\lm+1)^{k-1} \\
& = & (\lm-k)(\lm+1)(\lm+1)^{k-1} = (\lm-k)(\lm+1)^{k},
\eeast
and
\be
g(k+1) = -(\lm-k+1)(\lm+1)^{k-1} - k(\lm+1)^{k-1} = -(\lm-k+1+k) (\lm+1)^{k-1} = -(\lm+1)^{k}
\ee
which are what we guess. Let $x=\brb{x_1,\dots,x_n}^T$ be the corresponding eigenvector of eigenvalue $n-1$. Thus,
\be
Ax = (n-1)x \ \ra\ x_1 = x_2 = \dots = x_n.
\ee

Then we can let $x = (1,\dots,1)^T$. For eigenvalue $-1$, we assume the corresponding eigenvectors are $x=\brb{x_1,\dots,x_n}^T$. Then
\be
Ax = -1\cdot x \ \ra\ x_1 + \dots + x_n = 0.
\ee

Therefore, we can let $x = (\dots,1,-1,\dots)^T$. Thus, the corresponding eigenvalue and eigenvector matrix for $A$ are
\be
\Lambda = \bepm
-1 & & & & \\
& -1 & & & \\
& & \ddots & & \\
& & & - 1 & \\
& & & & n-1
\eepm
,\qquad P = \bepm 1 & & & & 1 \\
-1 & 1 & & & 1\\
& -1 &  & & 1 \\
& & \ddots & & \vdots\\
& & & 1 & 1\\
& & & -1 & 1\\
\eepm
\ee
which give $AP = P\Lambda$ ($A = P\Lambda P^{-1})$.
\end{example}

\section{Similar Matrices}

\subsection{Similarity}

\begin{definition}[similar matrices\index{similar!matrix}]\label{def:similar_matrix}
Two matrices $A,B\in M_n(\F)$ are similar if $B = P^{-1}AP$ for some invertible matrix $P\in M_n(\F)$. $P$ is called similarity matrix\index{similarity matrix}.

The transformation $A \to P^{-1}AP$ is called a similarity transformation\index{similarity transformation} by the similarity matrix $P$. The relation `$A$ is similar to $B$' is sometimes abbreviated $A \sim B$.
\end{definition}

\begin{proposition}\label{pro:similarity_implies_equivalent_matrices}
Let $A,B\in M_n(\F)$ with $A\sim B$. Then $A$ is equivalent to $B$.
\end{proposition}

\begin{proof}[\bf Proof]
Direct result from definitions of similar matrices (Definition \ref{def:similar_matrix}) and equivalent matrices (Definition \ref{def:equivalent_matrix}) as we can let $P = Q$.
\end{proof}

\begin{proposition}[similarity is equivalent relation]
Similarity is an equivalent relation on $M_n(\F)$, i.e., similarity is for $A,B,C\in M_n(\F)$,
\ben
\item [(i)] reflexive: $A\sim A$
\item [(ii)] symmetric: $A\sim B \ \ra\ B\sim A$.
\item [(iii)] transitive: $A\sim B, B\sim C \ \ra\ A\sim C$.
\een
\end{proposition}

\begin{proof}[\bf Proof]%\footnote{proof needed.}
\ben
\item [(i)] It is obvious that invertible matrix $P = I$ such that $A = IAI = P^{-1}AP$.
\item [(ii)] If $A\sim B$, there exists invertible matrix $P$ such that $B = P^{-1}AP$. Thus, for $Q = P^{-1}$, $B = QAQ^{-1} \ \lra \ Q^{-1}BQ = A$. Hence, $B\sim A$.
\item [(iii)] If $A\sim B$ and $B\sim C$, there exist invertible matrices $P$ and $Q$ such that $B = P^{-1}AP$ and $C = Q^{-1}BQ$. Thus, for invertible matrix $R = PQ$,
\be
C = Q^{-1}BQ = Q^{-1}P^{-1}A PQ = R^{-1}AR \ \ra \ A \sim C,
\ee
as $(PQ)^{-1} = Q^{-1}P^{-1}$ by Proposition \ref{pro:inverse_matrix_property}.(ii).
\een
\end{proof}


\begin{proposition}
Similar matrices have the same determinant.
\end{proposition}

\begin{proof}[\bf Proof]
If $A$ and $B$ are similar, then there exists invertible matrix $P$ such that $B= P^{-1}AP$. Then by Theorem \ref{thm:determinant_product} and Corollary \ref{cor:determinant_inverse},
\be
\det B = \det\brb{P^{-1}AP} = \det \brb{P^{-1}} \det A\det P = (\det A)(\det P)(\det P)^{-1} = \det A.
\ee
\end{proof}


\begin{proposition}
Similar matrices have the same trace, i.e., for $A,B,P\in M_n(\F)$, if $B = P^{-1}AP$ for some invertible matrix $P$, then $\tr A = \tr B$.
\end{proposition}

\begin{proof}[\bf Proof]
By Proposition \ref{pro:trace_change_order},
\be
\tr B = \tr\brb{P^{-1}AP} = \tr\brb{APP^{-1}} = \tr\brb{AI} = \tr A.
\ee
\end{proof}

\begin{theorem}\label{thm:similar_matrices_have_same_characteristic_polynomial}
Let $A,B\in M_n(\F)$. If $B$ is similar to $A$, then the characteristic polynomial of $B$ is the same as that of $A$. That is, $p_A(t) = p_B(t)$.
\end{theorem}

\begin{proof}[\bf Proof]
For any $t$ we have
\beast
p_B(t) & = & \det\brb{tI - B} = \det\brb{tP^{-1}P - P^{-1}AP} = \det\brb{P^{-1}(tI -A)P} \\
& = &  \det P^{-1} \det(tI -A) \det P \qquad (\text{Theorem \ref{thm:determinant_product}}) \\
& = &  \det(tI -A) \det P \det P^{-1} = \det(tI -A) = p_A(t)
\eeast
as required.
\end{proof}

\begin{corollary}\label{cor:similarity_same_eigenvalues}
If $A,B\in M_n(\F)$ and if $A$ and $B$ are similar, then they have the same eigenvalues, counting multiplicity.
\end{corollary}

\begin{proof}[\bf Proof]
Direct result from Theorem \ref{thm:similar_matrices_have_same_characteristic_polynomial} and Proposition \ref{pro:characteristic_polynomial_root_coincide_spectrum}.
\end{proof}

\begin{example}\label{exa:same_eigenvalues_do_not_imply_similarity}
Having the same eigenvalues is a necessary but not sufficient condition for similarity. Consider the matrices
\be
A = \bepm 0 & 1 \\ 0 & 0 \eepm, \qquad B = \bepm 0 & 0 \\ 0 & 0 \eepm.
\ee

Each has the eigenvalue 0 with multiplicity 2, but they are not similar.
\end{example}

\subsection{Application of similar matrices}

\begin{example}
A flower is passed on among $n$ persons. How many ways are there if the flower returns to the original person after $m$ steps? We can have $A\in M_n(\F)$ defined by
\be
A = \bepm
0 & 1 & 1 & & \dots & & 1 & 1\\
1 & 0 & 1 & & \dots & & 1 & 1\\
1 & 1 & 0 & & \dots & & 1 & 1 \\
1 & 1 & 1 & & \ddots & & 1 & 1 \\
& & & & & \ddots & &  \\
1 & 1 & 1 & & \dots & & 1 & 0
\eepm.
\ee

Then what we want is $A^m$. We can use the result in Example \ref{exa:eigenvalue_one_matrix_with_zero_diagonal} and have that $A = P\Lambda P^{-1}$ where
\be
\Lambda = \bepm
-1 & & & & \\
& -1 & & & \\
& & \ddots & & \\
& & & - 1 & \\
& & & & n-1
\eepm
,\qquad P = \bepm 1 & & & & 1 \\
-1 & 1 & & & 1\\
& -1 &  & & 1 \\
& & \ddots & & \vdots\\
& & & 1 & 1\\
& & & -1 & 1\\
\eepm.
\ee

So we can calculate $A^m$ according to the result in Example \ref{exa:inverse_matrix_plus_minus_one} with respect to different cases of $m$.

For $m= 2k,k\in \Z^+$, we have
\beast
& & A^m = P\Lambda^m P^{-1} \\
& = &  \frac 1n \bepm 1 & & & & 1 \\
-1 & 1 & & & 1\\
& -1 &  & & 1 \\
& & \ddots & & \vdots\\
& & & 1 & 1\\
& & & -1 & 1\\
\eepm\bepm
1 & & & & \\
& 1 & & & \\
& & \ddots & & \\
& & & 1 & \\
& & & & (n-1)^m
\eepm\bepm
n-1 & -1 & -1 & \dots & -1 &  -1 \\
n-2 & n-2 & -2 &  \dots & -2 & -2 \\
& & & \vdots  & &  \\
k+1 & \dots & k+1 &  \dots & -(k-1) & -(k-1) \\
k & \dots & k & \dots & -k & -k \\
& & & \vdots & & \\
1 & \dots & 1 & & 1 & -(n-1)\\
1 & \dots & 1 &\dots & 1 & 1
\eepm
\eeast
which gives
\be
A^m = \frac 1n\bepm
1 &  &  & \dots & & (n-1)^m \\
-1 & 1 &  & \dots & &  (n-1)^m \\
& -1 & 1 & \dots & & (n-1)^m\\
& & & \ddots & & (n-1)^m \\
& & & -1 & & (n-1)^m
\eepm\bepm
n-1 & -1 & -1 & \dots & -1 &  -1 \\
n-2 & n-2 & -2 &  \dots & -2 & -2 \\
& & & \vdots  & &  \\
k+1 & \dots & k+1 &  \dots & -(k-1) & -(k-1) \\
k & \dots & k & \dots & -k & -k \\
& & & \vdots & & \\
1 & \dots & 1 & & 1 & -(n-1)\\
1 & \dots & 1 &\dots & 1 & 1
\eepm
\ee

Thus, we can have
\beast
A^m = \frac 1n \bepm
(n-1)^m+(n-1) & & (n-1)^m - 1 & & (n-1)^m - 1& \dots & (n-1)^m -1 & &  (n-1)^m-1 \\
(n-1)^m - 1 & & (n-1)^m + (n-1) & & (n-1)^m - 1& \dots & (n-1)^m -1 & & (n-1)^m -1 \\
& & & & &\ddots &  & & \\
(n-1)^m -1 & & (n-1)^m -1 & &  (n-1)^m -1 & \dots & (n-1)^m -1 & & (n-1)^m+(n-1)
\eepm
\eeast

Similar, for $m = 2k+1, k\in \Z^+$, we have
\beast
A^m = \frac 1n \bepm
(n-1)^m-(n-1) & & (n-1)^m + 1 & & (n-1)^m +1& \dots & (n-1)^m +1 & &  (n-1)^m+1 \\
(n-1)^m + 1 & & (n-1)^m - (n-1) & & (n-1)^m + 1& \dots & (n-1)^m +1 & & (n-1)^m +1 \\
& & & & &\ddots &  & & \\
(n-1)^m +1 & & (n-1)^m +1 & &  (n-1)^m +1 & \dots & (n-1)^m +1 & & (n-1)^m-(n-1)
\eepm
\eeast

Thus, we have that the number ways of returning after $m$ steps is
\be
\begin{cases}
\frac {n-1}n \brb{(n-1)^{m-1}+1} \quad\quad & m \text{ is even} \\
\frac {n-1}n \brb{(n-1)^{m-1}-1} \quad\quad & m \text{ is odd}
\end{cases}
\ee

For example, for $n=4$ and $m=6$, we have
\be
A = \bepm 0 & 1 & 1 & 1\\
1 & 0 & 1 & 1 \\
1 & 1 & 0 & 1 \\
1 & 1 & 1 & 0
\eepm \ \ra \
A^6 = \bepm 183 & 182 & 182 & 182 \\
182 & 183 & 182 & 182 \\
182 & 182 & 183 & 182 \\
182 & 182 & 182 & 183 \\
\eepm
\ee
where $183 = \frac{3}{4}\brb{3^5+1}$.
\end{example}

\subsection{Diagonalizable matrices}

\begin{definition}[diagonalizable matrix\index{diagonalizable matrix}]\label{def:diagonalizable_matrix}
If the matrix $A\in M_n(\F)$ is similar to a diagonal matrix, then $A$ is said to be diagonalizable.
\end{definition}

\begin{theorem}\label{thm:diagonalizable_linearly_independent_eigenvector_equivalent}
Let $A\in M_n(\F)$. Then $A$ is diagonalizable if and only if there is a set of $n$ linearly independent vectors, each of which is an eigenvector of $A$.
\end{theorem}

\begin{proof}[\bf Proof]
If $A$ has $n$ linearly independent eigenvectors $x^1,\dots, x^n$, form a nonsingular matrix $P$ with them as columns (by Corollary \ref{cor:invertible_column_linearly_independent}) and calculate
\beast
P^{-1} A P & = & P^{-1}A \brb{x^1, \dots , x^n } = P^{-1} \brb{ Ax^1 , \dots , Ax^n} = P^{-1} \brb{ \lm_1 x^1 , \dots , \lm_n x^n}   \\
& = & P^{-1} \brb{x^1,\dots,x^n} \Lambda = P^{-1}P \Lambda = \Lambda
\eeast
where
\be
\Lambda := \diag\brb{\lm_1,\dots,\lm_n} = \bepm \lm_1 & & \\ & \ddots & \\ & & \lm_n \eepm
\ee
and $\lm_1,\dots,\lm_n$ are the corresponding eigenvalues of $A$.

Conversely, suppose that there is a similarity matrix $P$ such that $P^{-1}AP = \Lambda$, then $AP = P\Lambda$. This means that $A$ times the $i$th column of $P$ (i.e., the $i$th column of $AP$) is
the $i$th diagonal entry of $\Lambda$ times the $i$th column of $P$ (i.e., the $i$th column of $P\Lambda$), or that the $i$th column of $P$ is an eigenvector of $A$ associated with the $i$th
diagonal entry of $\Lambda$. Since $P$ is invertible, there are $n$ linearly independent eigenvectors by Corollary \ref{cor:invertible_column_linearly_independent}.
\end{proof}




\begin{theorem}\label{thm:distinct_eigenvalues_implies_diagonalizable}
If $A\in M_n(\F)$ has $n$ distinct eigenvalues, then $A$ is diagonalizable.
\end{theorem}

\begin{proof}[\bf Proof]
Direct result from Theorem \ref{thm:diagonalizable_linearly_independent_eigenvector_equivalent} and Lemma \ref{lem:distinct_eigenvalues_implies_linearly_independent_eigenvectors}
\end{proof}

\begin{example}
If $A$ is diagonalizable, it is not necessary to have distinct eigenvalues. For instance, $\bepm 0 & 0 \\ 0 & 0 \eepm$.
\end{example}

\begin{proposition}
Let $A\in M_n(\F)$ and $B\in M_m(\F)$ be given matrices and let
\be
C = \bepm A & 0 \\ 0 & B \eepm \in M_{m+n}(\F)
\ee
be the direct sum (see Definition \ref{def:block_diagonal_matrix}) of $A$ and $B$. Then $C$ is diagonalizable if and only if both $A$ and $B$ are diagonalizable.
\end{proposition}

\begin{proof}[\bf Proof]
If $A$ and $B$ are diagonalizable, there are invertible matrices $P\in M_n(\F),Q \in M_m(\F)$ such that
\be
P^{-1}AP,\ Q^{-1}BQ\ \text{ are diagonal matrices.}
\ee

Then it is easy to check that $R^{-1}C R$ is diagonal if $R = \bepm P & 0 \\ 0 & Q \eepm$.

Conversely, if $C$ is diagonalizable, there is an invertible matrix $R\in M_{m+n}(\F)$ such that
\be
R^{-1}C R = \Lambda = \diag\bra{\lm_1,\dots,\lm_{m+n}} \text{ is diagonal.}
\ee

We can write that $R = \brb{R_1,\dots,R_{m+n}}$ with
\be
R_i = \bepm P_i \\ Q_i \eepm \in \C^{m+n},\quad P_i \in \C^n,\quad Q_i \in \C^m,\qquad i = 1,2,\dots,m+n.
\ee

Then $C R_i = \lm_i R_i$ implies that
\be
A P_i = \lm_i P_i,\ B Q_i = \lm_i Q_i \quad \text{for }i = 1,2,\dots,m+n.
\ee

If there were fewer than $n$ independent vector in the set $\bra{P_1,\dots,P_{m+n}}$, then the column rank (and hence the row rank) of the matrix
\be
\bepm P_1 & \dots & P_{m+n} \eepm \in M_{n,m+n}(\F)
\ee
would be less than $n$ (by Lemma \ref{lem:finite_basis_linearly_independent_spanning_equivalent} and definitions of rank (Definition \ref{def:column_rank_matrix}) and dimension
(Theorem \ref{thm:dimension_vector_space})). By the same reasoning, if there were fewer than $m$ independent vectors in the set $\bra{Q_1,\dots,Q_{m+n}}$, then the column rank (and hence the row rank) of matrix
\be
\bepm Q_1 & \dots & Q_{m+n} \eepm \in M_{m,m+n}(\F)
\ee
would be less than $m$. In either event (or both), the matrix would have row rank (and hence rank) less than $m+n$, which is impossible since $R$ is invertible. %\footnote{theorem needed here.}

Thus, there are exactly $n$ linearly independent vectors in the set $\bra{P_1,\dots,P_{m+n}}$. Let these vectors be $\bra{P'_1,\dots,P'_n}$ and $P' = \brb{P'_1,\dots,P'_n}$ is invertible by Corollary
\ref{cor:invertible_column_linearly_independent}. Since each of these $n$ vectors is an eigenvector of $A$, we have that
\be
AP'_i = \lm_i' P'_i \ \ra \ A P' = P' \Lambda'
\ee
where $\lm'_i$ is the corresponding eigenvalue of $P'_i$ and $\Lambda' = \diag\bra{\lm'_1,\dots,\lm'_n}$. Since $P'$ is invertible, we have $P^{-1}AP = \Lambda'$ and thus the matrix $A$ must be
diagonalizable. The same argument shows that the matrix $B$ is diagonalizable.
\end{proof}

\subsection{Simultaneously diagonalizable matrices}

\begin{definition}[simultaneously diagonalizable matrices\index{simultaneously diagonalizable!matrices}]\label{def:simultaneously_diagonalizable_matrices}
Two diagonalizable matrices $A,B\in M_n(\F)$ are said to be simultaneously diagonalizable if there is a single similarity matrix $S\in M_n(\F)$ such that $S^{-1}AS$ and  $S^{-1}BS$ are both
diagonal, i.e., if there is a single basis in which the representations of both linear transformations are diagonal.
\end{definition}

\begin{remark}
Clearly, two diagonal matrices commute.
\end{remark}

\begin{lemma}\label{lem:scalar_matrix_and_any_diagonalizable_matrix_are_simultaneously_diagonalizable}
Let $A\in M_n(\F)$ be diagonalizable and $\lm I\in M_n(\F)$ is a scalar matrix. Then $A$ and $\lm I$ are simultaneously diagonalizable.
\end{lemma}

\begin{proof}[\bf Proof]
First, we can find invertible matrix $S$ such that $S^{-1}AS = \Pi$ where $\Pi$ is a diagonal matrix. Also, we can have that $\lm I$ is diagonalizable (with similarity matrix $I$) and
\be
S^{-1} \lm I S = \lm S^{-1}IS = \lm I.
\ee
as required.
\end{proof}

\begin{theorem}\label{thm:diagonalizable_matrices_commute_iff_simultaneously diagonalizable}
Let $A,B\in M_n(\F)$ be diagonalizable. Then $A$ and $B$ commute ($AB = BA$) if and only if they are simultaneously diagonalizable.
\end{theorem}

\begin{proof}[\bf Proof]
($\la$). If $A$ and $B$ are simultaneously diagonalizable. We can find a similarity matrix $S$ such that
\be
S^{-1}AS = \Lambda,\quad S^{-1}BS = \Pi
\ee
where $\Lambda$ and $\Pi$ are diagonal matrices. Since diagonal matrices commute, we have that $\Lambda \Pi = \Pi \Lambda$ and
\be
AB = S\Lambda S^{-1} S\Pi S^{-1} = S\Lambda \Pi S^{-1} = S\Pi \Lambda S^{-1} = S\Pi S^{-1} S\Lambda S^{-1} = BA.
\ee

($\ra$). Now assume diagonalizable matrices $A$ and $B$ commute with $AB = BA$. Let $P$ be the similarity matrix of $A$. Thus, we can have $P^{-1}AP = \Lambda$ where $\Lambda$ is a diagonal matrix with eigenvalues $\lm_i,i=1,\dots,n$. So
\beast
AB & = & BA \\
P^{-1} APP^{-1} BP & = & P^{-1} B P P^{-1} A P \\
\Lambda C & = & C \Lambda\qquad\qquad (*)
\eeast
where $C := P^{-1}BP$. Note that ($*$) can be written in other form as
\be
\lm_i c_{ij} = c_{ij} \lm_j,\qquad i,j=1,\dots,n.
\ee

Thus, we have $c_{ij}= 0$ if $\lm_i \neq \lm_j$ which implies that $C$ is a block diagonal matrix if we find the proper similarity matrix $P$. Thus, $\Lambda$ has $k$ distinct eigenvalues and consists of $k$ scalar matrices. That is,
\be
C = \bepm C_{11} & & & 0\\ & C_{22} & & \\ & & \ddots & \\ 0 & & & C_{kk} \eepm,\qquad P^{-1}AP = \Lambda = \bepm \lm_1 I & & & 0\\ & \lm_2 I & & \\ & & \ddots & \\ 0 & & & \lm_k I \eepm
\ee

Also, since $B$ is diagonalizable, we have that $C = P^{-1}BP$ is also diagonalizable. Thus, we $C_{ii},i=1,\dots,k$ are diagonalizable. For any $C_{ii},i=1,\dots,k$, we can find the corresponding similarity matrix $T_i$ such that
\be
T_i^{-1}C_{ii}T_i = \Pi_i,\qquad i=1,\dots,k
\ee
where $\Pi_i$ is a diagonal matrix. Then we can construct a matrix
\be
T = \bepm T_{1} & & & 0\\ & T_{2} & & \\ & & \ddots & \\ 0 & & & T_{k} \eepm,\qquad
T^{-1} = \bepm T_{1}^{-1} & & & 0\\ & T_{2}^{-1}  & & \\ & & \ddots & \\ 0 & & & T_{k}^{-1}  \eepm
\ee
(implied by Proposition \ref{pro:block_diagonal_matrix_property}) such that
\be
T^{-1} C T = \Pi = \bepm \Pi_{1} & & & 0\\ & \Pi_{2} & & \\ & & \ddots & \\ 0 & & & \Pi_{k} \eepm
\ee
where $\Pi$ is also a diagonal matrix. Therefore, we have $T^{-1} P^{-1} B P T = \Pi$ which is a diagonal matrix.

Meanwhile, we have that
\beast
T^{-1} P^{-1}A P T & = &  T^{-1} \Lambda T =  \bepm T_{1}^{-1} & & & 0\\ & T_{2}^{-1}  & & \\ & & \ddots & \\ 0 & & & T_{k}^{-1}  \eepm
\bepm \lm_1 I & & & 0\\ & \lm_2 I & & \\ & & \ddots & \\ 0 & & & \lm_k I \eepm
\bepm T_{1} & & & 0\\ & T_{2} & & \\ & & \ddots & \\ 0 & & & T_{k} \eepm \\
& = & \bepm \lm_1 T_{1}^{-1} & & & 0\\ & \lm_2 T_{2}^{-1}  & & \\ & & \ddots & \\ 0 & & & \lm_k T_{k}^{-1}  \eepm
\bepm T_{1} & & & 0\\ & T_{2} & & \\ & & \ddots & \\ 0 & & & T_{k} \eepm = \bepm \lm_1 I & & & 0\\ & \lm_2 I & & \\ & & \ddots & \\ 0 & & & \lm_k I \eepm = \Lambda
\eeast
which is also a diagonal matrix. Thus, $A$ and $B$ commute.
\end{proof}




\section{Unitary Equivalence}

We use $\F = \C$ in this section.

\subsection{Unitary matrices}



\begin{theorem}\label{thm:unitary_matrix_property}
If $U\in M_n(\C)$, the following are equivalent:
\ben
\item [(i)] $U$ is unitary.
\item [(ii)] $U$ is invertible (nonsingular) and $U^* = U^{-1}$.
\item [(iii)] $UU^* = I$.
\item [(iv)] $U^*$ is unitary.
\item [(v)] The columns of $U$ form an orthonormal set (see Definition \ref{def:orthonormal_set}).
\item [(vi)] The rows of $U$ form an orthonormal set.
\item [(vii)] For all $x\in \C^n$, the Euclidean length of $y = Ux$ is the same as that of $x$, i.e., $y^*y = x^*x$.
\item [(viii)] $U$ is an isometry with respect to the usual norm.
\item [(ix)] $U$ is a normal matrix with eigenvalues lying on the unit circle.
\een
\end{theorem}

\begin{proof}[\bf Proof]
\footnote{proof needed.}
\end{proof}

\subsection{Unitary equivalence}

\begin{definition}[unitarily equivalent\index{unitarily equivalent!matrix}]\label{def:unitarily_equivalent_matrix}
Let $A,B\in M_n(\C)$ and $B$ is said to be unitarily equivalent (unitarily similar) to $A$ if there is a unitary matrix $U\in M_n(\C)$ such that
\be
B = U^* A U.
\ee

If $U$ can be taken to be real (and hence is real orthogonal), then $B$ is said to be (real) orthogonally equivalent (orthogonally similar) to $A$.
\end{definition}



\begin{proposition}[unitary equivalence is equivalent relation]\label{pro:unitary_equivalence_implies_equivalent_matrices}
Unitary equivalence is an equivalent relation on $M_n(\C)$, i.e., unitary equivalence is for $A,B,C\in M_n(\C)$
\ben
\item [(i)] reflexive: $A$ is unitarily equivalent to $A$
\item [(ii)] symmetric: $A$ unitarily equivalent to $B$ $\ \ra\ $ $B$ unitarily equivalent to $A$.
\item [(iii)] transitive: $A$ unitarily equivalent to $B$, $B$ unitarily equivalent to $C$ $\ \ra\ $ $A$ unitarily equivalent to $C$.
\een
\end{proposition}


\begin{proof}[\bf Proof]%\footnote{proof needed.}
\ben
\item [(i)] It is obvious that invertible matrix $U = I$ such that $A = IAI = U^*AU$.
\item [(ii)] If $A$ unitarily equivalent to $B$, there exists unitary matrix $U$ such that $B = U^*AU$. Thus, for unitary matrix $V = U^{-1}$, $A = U B U^{-1} = V^{-1}BV = V^*BV $. Hence, $B$
    unitarily equivalent to $A$.

\item [(iii)] If $A$ unitarily equivalent to $B$ and $B$ unitarily equivalent to $C$, there exist unitary matrices $U$ and $V$ such that $B = U^*AU$ and $C = V^*BV$. Thus, for unitary matrix $R = UV$,
\be
C = V^*BV = V^*U^* A UV = R^* A R \ \ra \ A \text{ unitarily equivalent to } C,
\ee
as $(UV)^* = V^*U^*$ by Proposition \ref{pro:matrix_multiple_hermitian}.
\een
\end{proof}

\begin{proposition}[unitary equivalence implies similarity]\label{pro:unitary_equivalence_implies_similarity}
Let $A,B\in M_n(\C)$ with $A$ is unitarily equivalent to $B$. Then $A\sim B$.
\end{proposition}

\begin{proof}[\bf Proof]
Direct result from the definition by letting $P = U$ as $U^* = U^{-1}$.
\end{proof}

\begin{example}
The converse is not true as
\be
\bepm 3 & 1 \\ -2 & 0 \eepm\quad \text{and}\quad \bepm 1 & 1 \\ 0 & 2 \eepm
\ee
are similar but are not unitarily equivalent.
\end{example}

This can be checked by the following theorem.

\begin{theorem}\label{thm:unitary_equivalence_elements_square_sum}
If $A,B\in M_n(\C)$ are unitarily equivalent, then
\be
\sum^n_{i,j =1}\abs{a_{ij}}^2 = \sum^n_{i,j =1}\abs{b_{ij}}^2.
\ee
\end{theorem}

\begin{proof}[\bf Proof]
Observe that
\be
\tr\brb{A^* A} = \sum^n_{i,j =1}\abs{a_{ij}}^2,\qquad \tr\brb{B^* B} = \sum^n_{i,j =1}\abs{b_{ij}}^2
\ee
by carrying out the matrix multiplication. Thus, it suffices to show that $\tr(A^*A) = \tr (B^*B)$. But if $B= U^*A U$ for unitary matrix $U$,
\be
\tr\brb{B^*B} = \tr\brb{\brb{U^*AU}^*U^*AU} = \tr\brb{U^*A^*UU^*AU} = \tr\brb{U^*A^*AU} = \tr\brb{A^*A UU^*} =  \tr\brb{A^*A}
\ee
by Proposition \ref{pro:matrix_multiple_hermitian} and Proposition \ref{pro:trace_change_order}.
\end{proof}

\subsection{Schur's unitary triangularization theorem}

Perhaps the most fundamentally useful fact of elementary matrix theory is that any matrix $A\in M_n(\C)$ is unitarily equivalent to an upper triangular matrix $T$ (and also to a lower triangular
matrix). The diagonal entries of $T$ are, of course, the eigenvalues of $A$. Although this form is far from unique, it represents the simplest form achievable under unitary equivalence.

\begin{theorem}[Schur's unitary triangularization theorem\index{Schur's unitary triangularization theorem}]\label{thm:schur_unitary_triangularization}
Given $A\in M_n(\C)$ with eigenvalues $\lm_1,\dots,\lm_n$ in any prescribed order, there a unitary matrix $U\in M_n(\C)$ such that
\be
U^* A U = T = \brb{t_{ij}}
\ee
is upper triangular, with diagonal entries $t_{ii} = \lm_i$, $i=1,\dots,n$. That is, every square matrix $A$ is unitarily equivalent to a triangular matrix whose diagonal entries are the eigenvalues
of $A$ in a prescribed order.

Furthermore, if $A\in M_n(\R)$ and if all the eigenvalues of $A$ are real, then $U$ may be chosen to real and orthogonal.
\end{theorem}

\begin{remark}
This theorem is also called Schur's decomposition theorem.

Note that ``upper triangular'' could be replaced by ``lower triangular'' in the statement of the theorem with, of course, a different unitary equivalence $U$.
\end{remark}

\begin{proof}[\bf Proof]
The proof is algorithmic and proceeds by a sequence of reductions of like type. Let $x^{(1)}$ be a normalized eigenvector (i.e., ${x^{(1)}}^* x^{(1)} = 1$) of $A$ associated with eigenvalue $\lm_1$.
$x^{(1)}$ may be extended to a basis (by Lemma \ref{lem:linearly_independent_extend_basis})
\be
x^{(1)},\ y^{(2)},\ y^{(3)},\ \dots\ ,\ y^{(n)}
\ee
of $\C^n$. Apply the Gram-Schmidt orthonormalization (Algorithm \ref{alg:gram_schmidt_orthonormalization}) to this basis to produce an orthonormal basis
\be
x^{(1)},\ z^{(2)},\ z^{(3)},\ \dots\ ,\ z^{(n)}
\ee
of $\C^n$. Array these orthonormal vectors left to right as the columns of a unitary matrix $U_1$ (by Theorem \ref{thm:unitary_matrix_property}). Since the first columns of $AU_1$ is $\lm_1 x^{(1)}$, a
calculation reveals that $U^*_1(AU_1)$ has the form
\be
U^*_1(AU_1) = \bepm \lm_1 & * \\ 0 & A_1 \eepm.
\ee

The matrix $A_1 \in M_{n-1}(\C)$ has eigenvalues $\lm_2,\dots,\lm_n$ as $U^*_1AU_1$ has the same eigenvalues with $A$ (This is implied by
\be
\det\brb{tI -A} = \det \brb{U_1 tIU_1^* - U_1U_1^*A} = \det \brb{\brb{tIU_1^* - U_1^*A}U_1 } =  \det \brb{tI - U_1^*AU_1}% = \det\brb{U_1^*(tI)U_1 - U_1AU_1}
\ee
by Theorem \ref{thm:determinant_product}) by Proposition \ref{pro:characteristic_polynomial_root_coincide_spectrum}. Let $x^{(2)}\in \C^{n-1}$ be a normalized eigenvector of $A_1$ corresponding
to $\lm_2$, and do it all over again. Determine a unitary $U_2 \in M_{n-1}(\C)$ such that
\be
U^*_2 A_1U_2  = \bepm \lm_2 & * \\ 0 & A_2 \eepm
\ee
and let
\be
V_2 = \bepm 1 & 0 \\ 0 & U_2 \eepm.
\ee

The matrices $V_2$ and $U_1V_2$ are then unitary (by definition of unitary matrix), and $V_2^*U_1^*AU_1V_2$ has the form
\be
V_2^*U_1^*AU_1V_2 = \brb{\ba{c;{2pt/2pt}c}\ba{cc} \lm_1 & * \\ 0 & \lm_2 \ea & \text{\Large $*$}  \\ \hdashline[2pt/2pt] \ba{c}\text{\Large $\stackrel{\qquad}{0}$} \ea & A_2 \ea}
\ee

%\be
%V_2^*U_1^*AU_1V_2 =
%\begin{array}{c@{\hspace{-5pt}}l}
%\left(\begin{array}{cc;{2pt/2pt}c}
%\lm_1 & * &  \multicolumn{1}{c}{\Large *} \\
%0 & \lm_2 & \\
%\hdashline[2pt/2pt]
%\multicolumn{2}{c;{2pt/2pt}}{\raisebox{1ex}{\large 0}} & A_2
%\end{array}\right)
%\end{array}
%\ee

Continue this reduction to produce unitary matrices $U_i\in M_{n-i+1}(\C)$, $i=1,\dots,n-1$ and unitary matrices $V_i\in M_n(\C)$, $i=2,\dots,n-1$. The matrix
\be
U = U_1 V_2 V_3 \dots V_{n-1}
\ee
is unitary and $U^*AU$ yields the desired form.

If all eigenvalues of $A\in M_n(\R)$ happen to be real, then the corresponding eigenvectors can be chosen to be real and all the above steps may be carried out in real arithmetic, verifying the final assertion.
\end{proof}

\begin{example}
Neither the unitary matrix $U$ nor the triangular matrix $T$ of Schur's unitary triangularization theorem is unique. Not only may the diagonal entries of $T$ (the eigenvalues of $A$) appear in any
order, but unitarily equivalent upper triangular matrices may appear very different above the diagonal. For example,
\be
T_1 = \bepm 1 & 1 & 4 \\ 0 & 2 & 2 \\ 0 & 0 & 3 \eepm \qquad \text{and}\qquad T_2 = \bepm 2 & -1 & 3\sqrt{2} \\ 0 & 1 & \sqrt{2} \\ 0 & 0 & 3 \eepm
\ee
are unitarily equivalent via
\be
U = \frac 1{\sqrt{2}}\bepm 1 & 1 & 0 \\ 1 & -1 & 0 \\ 0 & 0 & \sqrt{2} \eepm.
\ee

In general, many different upper triangular matrices can be in the same unitary equivalence class.
\end{example}

\begin{example}
For any $A\in M_n(\C)$, by Theorem \ref{thm:schur_unitary_triangularization} we have that $A = U^*TU$ where $U$ is a unitary matrix and $T$ is an upper triangular matrix with $A$'s eigenvalues. Thus, for $m\in \Z^+$,
\be
A^m = \brb{U^*TU }^m = U^* T^m U.
\ee

Thus, the eigenvalues of $A^m$ are $m$th powers of the eigenvalues of $A$. This is consistent with Proposition \ref{pro:polynomial_matrix_eigenvalue_eigenvector}.
\end{example}


A strictly real version of Schur's unitary triangularization theorem (Theorem \ref{thm:schur_unitary_triangularization}) is contained in the following theorem.

\begin{theorem}
If $A\in M_n(\R)$, there is a real orthogonal matrix $Q\in M_n(\R)$ such that
\be
Q^TAQ = \bepm \ba{cc} A_1 & \\ & A_2 \ea & \text{\Large $*$} \\ \text{\Large 0} & \ba{cc} \ddots & \\ & A_k \ea \eepm\in M_n(\R),\quad 1\leq k\leq n
\ee
where each $A_i$ is a real $1\times 1$ matrix, or a real $2\times 2$ matrix with a non-real pair of complex conjugate eigenvalues. The diagonal blocks $A_i$ may be arranged in any prescribed order.
\end{theorem}

\begin{proof}[\bf Proof]
We can modify the argument for Theorem \ref{thm:schur_unitary_triangularization} to prove this theorem.

If $\lm$ is a real eigenvalue of the real matrix $A$, then there is a corresponding real eigenvector that can be used to deflate $A$ as in Theorem \ref{thm:schur_unitary_triangularization}.

Now let $\lm = \alpha + i\beta$ with $\beta \neq 0$ is a non-real eigenvalue of $A$ and $Ax = \lm x$, $x = u + iv\neq 0$, $u,v\in \R^n$. By Proposition \ref{pro:component_wise_conjugate_matrix}, we have $A\ol{x} =
\ol{\lm} \ol{x}$ (and thus $\ol{\lm}$ is also an eigenvalue associated with eigenvector $\ol{x}$). If $x$ and $\ol{x}$ are linearly dependent, then there exists $a\in \C$ such that $ax = \ol{x}$. Then we have
\be
\left\{\ba{l}
A ax = A\ol{x} \ \ra \ a\lm x = \ol{\lm}\ol{x} \\
a\lm x = \lm \ol{x} \ea\right.  \ \ra \ 0 = \brb{\lm - \ol{\lm}}\ol{x} \ \ra \ \lm = \ol{\lm} \quad \text{ since }\ \ol{x} \neq 0,
\ee
which is a contradiction to $\beta \neq 0$. Thus, we have that $x$ and $\ol{x}$ are linearly independent. Therefore, we can apply Gram-Schmidt orthonormalization (Algorithm
\ref{alg:gram_schmidt_orthonormalization}) with respect to inner product (dot product) to $\bra{u,v}$ ($\bra{x,\ol{x}}$) to obtain a real orthonormal set $\bra{w,z}$ with
\be
w = \frac{u}{\sqrt{\inner{u}{u}}},\quad y = v - \inner{v}{w}w,\quad z = \frac{y}{\sqrt{\inner{y}{y}}}.   %= \frac{v - \inner{v}{w}w}{\sqrt{\inner{v}{v} - \ol{\inner{v}{w}}\inner{v}{w} - \inner{v}{w}\inner{w}{v} - \inner{v}{w}\ol{\inner{v}{w}} \inner{w}{w}}}  .
\ee

Let $Q_1$ be the real orthogonal matrix whose first two columns are $w$ and $z$ by Theorem \ref{thm:unitary_matrix_property}. Thus, by the orthogonormality of $w,z$ and other columns of $Q_1$,
\beast
Q_1^T A Q_1 & = &  \bepm w^T \\ z^T \\ * \eepm A \bepm w & z & * \eepm = \bepm w^T \\ z^T \\ * \eepm \bepm Aw & Az & * \eepm  =  \bepm w^T \\ z^T \\ * \eepm \bepm \lm w & \frac 1{\sqrt{\inner{y}{y}}}\brb{\ol{\lm} v - \lm\inner{v}{w}w} & * \eepm
\\
& = & \bepm w^T \\ z^T \\ * \eepm \bepm \lm w & \frac 1{\sqrt{\inner{y}{y}}}\brb{\ol{\lm} y + \brb{\ol{\lm}-\lm}\inner{v}{w}w} & * \eepm
=  \bepm w^T \\ z^T \\ * \eepm \bepm \lm w & \ol{\lm} z + \frac 1{\sqrt{\inner{y}{y}}}\brb{\ol{\lm}-\lm}\inner{v}{w}w & * \eepm\\
& = &  \bepm w^T \lm w & w^T \brb{\ol{\lm} z + \frac 1{\sqrt{\inner{y}{y}}}\brb{\ol{\lm}-\lm}\inner{v}{w}w} & * \\ z^T \lm w & z^T \brb{\ol{\lm} z + \frac 1{\sqrt{\inner{y}{y}}}\brb{\ol{\lm}-\lm}\inner{v}{w}w} & * \\ 0 & 0 & * \eepm
=  \bepm \lm & * & * \\ 0 & \ol{\lm} & * \\ 0 & 0 & \wt{A} \eepm
\eeast
so that $A$ may be deflated two columns at a time in this case. Note that blocks $A_i$ corresponding to each real eigenvalue and each pair of complex conjugate eigenvalues can be arranged in any prescribed order.

Hence, we can repeat the similar steps in proof of Theorem \ref{thm:schur_unitary_triangularization}.
\end{proof}

\begin{lemma}\label{lem:product_of_triangular_matrices}
Suppose that $R = \brb{r_{ij}}$ and $T = \brb{t_{ij}} \in M_n(\F)$ are upper triangular and that $r_{ij} = 0$, $1\leq i,j\leq k \leq n$ and $t_{k+1,k+1} = 0$. Let $S = (s_{ij}) = RT$. Then
\be
s_{ij} = 0,\qquad 1\leq i,j\leq k+1.
\ee
\end{lemma}

\begin{proof}[\bf Proof]
Since $R\brb{\bra{1,2,\dots,k}} = 0$ and $t_{k+1,k+1} =0$, $R$ and $T$ have the form
\be
R = \brb{\ba{c;{2pt/2pt}c} \text{\large 0} & \text{\large $*$} \\ \hdashline[2pt/2pt] \text{\large 0} & \ba{ccc} * & & * \\ & \ddots & \\ 0 & & * \ea \ea} ,\qquad
T = \brb{\ba{c;{2pt/2pt}c} \ba{ccc} * & & * \\ & \ddots & \\ 0 & & * \ea & \text{\large $*$} \\ \hdashline[2pt/2pt] \text{\large 0} & \ba{cccc} 0 & & & * \\ & * & & \\ & & \ddots & \\ 0 & & & * \ea  \ea}
\ee
where both upper left blocks in the partitions are $k\times k$. The upper left $k\times k$ block of $S$ is clearly 0 by partitioned multiplication. Also, inspection reveals that the first $k+1$ rows
of $R$ have 0's in all non-zero positions of column $k+1$ of $T$, and that the first $k+1$ columns of $T$ have 0's in all non-zero positions of row $k+1$ of $R$. Matrix multiplication then shows that $S$ has the form
\be
S = \brb{\ba{c;{2pt/2pt}c;{2pt/2pt}c} \text{\Large 0}  & \ba{c} 0 \\ \vdots \\ 0 \ea & \text{\Large $*$} \\ \hdashline[2pt/2pt] 0 & 0 & * \\ \hdashline[2pt/2pt] \text{\Large 0} & \ba{c} 0 \\  \vdots \\ 0 \ea & \ba{cccc} * & & * \\ & \ddots & \\ 0 & & * \ea  \ea}
\ee
and $S\brb{\bra{1,2,\dots,k+1}} = 0$, as was to be shown.
\end{proof}

\begin{definition}[annihilation polynomial\index{annihilation polynomia}l]\label{def:annihilation_polynomial}
A polynomial $p(t)$ whose value is the zero matrix at $A$ is said to annihilate $A$. That is, $p(A) = 0$.
\end{definition}

\begin{theorem}[Cayley-Hamilton theorem\index{Cayley-Hamilton theorem}]\label{thm:cayley_hamilton}
Let $p_A(t)$ be the characteristic polynomial of $A\in M_n(\C)$. Then $p_A(t)$ annihilate $A$, i.e.,
\be
p_A(A) = 0.
\ee
\end{theorem}

\begin{remark}
\ben
\item [(i)] Note that we can not use the following proof
\be
p_A(t) = \det\brb{tI -A}\ \ra\ p_A(A) = \det\brb{AI - A} = \det 0 = 0
\ee
since $p_A(A)$ is a matrix and $\det\brb{AI - A}$ is a number.

\item [(ii)] Since we have the Cayley-Hamilton theorem for matrices with complex entries, and hence it must hold for matrices whose entries come from any subfield of the complex numbers (the reals
    or the rationals).

    In fact, the Cayley-Hamilton theorem is a completely formal result that holds for matrices whose entries come from any field or, more generally, any commutative ring\footnote{checking needed.}.
    \een
\end{remark}

\begin{proof}[\bf Proof]
Since $p_A(t)$ is of degree $n$ with leading coefficient 1 and the roots of $p_A(t) = 0$ are precisely the eigenvalues $\lm_1,\dots,\lm_n$ of $A$ (by Proposition
\ref{pro:characteristic_polynomial_root_coincide_spectrum}), counting multiplicity, we may factor $p_A(t)$ as
\be
p_A(t) = (t-\lm_1)(t-\lm_2)\dots (t-\lm_n).
\ee

By Schur's unitary triangularization theorem (Theorem \ref{thm:schur_unitary_triangularization}), we can write $A$ as $A = UTU^*$ where $T$ is upper triangular with $\lm_i$ in the $i$th diagonal
position, $i=1,\dots,n$. Now compute
\beast
p_A(A) & = & p_A\brb{UTU^*} = \brb{UTU^* - \lm_1 I}\brb{UTU^* - \lm_2 I} \dots \brb{UTU^* - \lm_n I} \\
& = &  \brb{U(T - \lm_1 I)U^*}\brb{U(T - \lm_2 I)U^*} \dots \brb{U(T - \lm_n I)U^*} \\
& = &  U(T - \lm_1 I)(T - \lm_2 I) \dots (T - \lm_n I)U^* = U p_A(T)U^*
\eeast
and notice that $p_A(A) = 0$ if and only if $p_A(T) = 0$. However, Lemma \ref{lem:product_of_triangular_matrices} allows us to conclude that $p_A(T) =0$. The upper left $1\times 1$ block of $T-\lm_1
I$ is 0 and $(2,2)$ entry of $T-\lm_2 I$ is 0; since both are upper triangular, the upper left $2\times 2$ block of $(T-\lm_1 I)(T-\lm_2 I)$ is 0. Inductively, since the upper left $k\times k$ block
of $(T-\lm_1 I)(T-\lm_2 I)\dots (T-\lm_k I)$ and the $(k+1,k+1)$ of $(T-\lm_{k+1} I)$ are 0 and both are upper triangular, the upper left $(k+1)\times (k+1)$ block of $(T-\lm_1 I)\dots(T-\lm_{k+1}
I)$ is 0. Continuation until $n$ is reached allows us to conclude the product $p_A(T) = (T-\lm_1 I)\dots(T-\lm_n I) = 0$, which completes the proof.
\end{proof}

One important use of the Cayley-Hamilton theorem is to write powers $A^k$ of $A\in M_n(\C)$, for $k\geq n$, as linear combinations of $I,A,A^2,\dots,A^{n-1}$. By a linear dependence argument, it is
easy to show (since the dimension of $M_n(\C)$, considered as a vector space over the complex numbers, is $n^2$) that powers $A^{n^2}$ and beyond can be expressed as linear combinations of lower
powers, but the Cayley-Hamilton theorem provides a notable improvement.

\begin{example}
Let $A = \bepm 3 & 1 \\ -2 & 0 \eepm$. Then $p_A(t) = t^2 - 3t + 2$, and $A^2 - 3A + 2I = 0$. Thus,
\beast
A^2 & = & 3A - 2I \\
A^3 & = & A(A^2) = 3A^2 - 2A = 3\brb{3A - 2I} = 7A - 6I\\
A^4 & = & 7A^2 - 6A = 15A - 14I,
\eeast
and so on. Also, since the constant term in $p_A(t)$, the determinant of $A$, is non-zero, $A$ is nonsingular, and we may write $A^{-1}$ as a polynomial in $A$. Again from $p_A(A) = A^2 - 3A + 2I =
0$, we get $2I = -A^2 + 3A = A\brb{-A + 3I}$, or $I = \frac 12 A\brb{-A + 3I}$. This means that
\be
A^{-1} = -\frac 12 A + \frac 32 I = \bepm 0 & -1/2 \\ 1 & 3/2 \eepm.
\ee
\end{example}

\begin{corollary}
If $A\in M_n(\C)$ is nonsingular, then there is a polynomial $q(t)$ (whose coefficients depend upon $A$), of degree at $n-1$, such that $A^{-1} =  q(A)$.
\end{corollary}

\begin{proof}[\bf Proof]
By Cayley-Hamilton theorem, we have that $p_A(A) =0$. If $p_A(t) = a_n t^n + \dots + a_1 t + a_0$, we can have
\be
0 = a_n A^n + \dots + a_1 A + a_0 I.
\ee

If $a_0 \neq 0$, we can have that
\be
-a_0 I = A \brb{a_n A^{n-1} + \dots + a_1I} \ \ra \ A^{-1} = - \frac 1{a_0} \brb{a_n A^{n-1} + \dots + a_1I}
\ee
where $- \frac 1{a_0} \brb{a_n t^{n-1} + \dots + a_2 t + a_1}$ is a polynomial $q(t)$. Otherwise, if $a_0 = 0$, we can use the nonsingularity of $A$ and get
\be
0 = a_n A^{n-1} + \dots + a_2 A + a_1 I.
\ee

Then we can repeat the above steps to get the required conclusion.
\end{proof}

Another use of Schur's unitary triangularization theorem is to make it clear that every matrix is `almost' diagonalizable in two possible interpretations of the phrase. The first says that
arbitrarily close to a given matrix there is a diagonalizable matrix, and the second says that any given matrix is similar to an upper triangular matrix whose off-diagonal entries are arbitrarily
small.

\begin{theorem}
Let $A = \brb{a_{ij}} \in M_n(\C)$. For every $\ve >0$, there exists a matrix $A\brb{\ve} = \brb{a_{ij}\brb{\ve}} \in M_n(\C)$ that has $n$ distinct eigenvalues (and is therefore diagonalizable by
Theorem \ref{thm:distinct_eigenvalues_implies_diagonalizable}) and is such that

\be
\sum^n_{i,j = 1}\abs{a_{ij} - a_{ij}(\ve)}^2 < \ve.
\ee
\end{theorem}

\begin{proof}[\bf Proof]
Let $U\in M_n(\C)$ be unitary and such that $U^*AU = T$ is upper triangular. Let $E = \diag\brb{e_1,e_2,\dots,e_n}$ in which $e_1,\dots,e_n$ are numbers chosen so that
\be
\abs{e_i} < \brb{\frac{\ve}{n}}^{1/2}
\ee
and so the numbers $t_{11} + e_1,\dots,t_{nn} + e_n$ are distinct\footnote{theorem needed.}. Then $T+E$ has $n$ distinct eigenvalues: $t_{11}+e_1,\dots,t_{nn}+e_n$, and
so does $A + UEU^*$, which is similar to $T+E$ by Corollary \ref{cor:similarity_same_eigenvalues}. Let $A(\ve) = A + UEU^*$, so that $A - A(\ve) = -UEU^*$ (therefore $A(\ve)- A$ and $E$ are
unitarily equivalent). By Theorem \ref{thm:unitary_equivalence_elements_square_sum}, we have
\be
\sum^n_{i,j = 1}\abs{a_{ij} - a_{ij}(\ve)}^2 = \sum^n_{i=1} \abs{e_i}^2 < n \brb{\frac {\ve}n} = \ve.
\ee

Therefore, $A(\ve)$ satisfies the assertion of the theorem.
\end{proof}


\begin{theorem}
Let $A \in M_n(\C)$. For every $\ve >0$ there is a nonsingular matrix $S_\ve\in M_n(\C)$ such that
\be
S^{-1}_\ve A S_\ve = T(\ve) = \brb{t_{ij}(\ve)}
\ee
is upper triangular and $\abs{t_{ij}(\ve)} < \ve$ for $1\leq i < j\leq n$.
\end{theorem}

\begin{proof}[\bf Proof]
First apply Schur's theorem (Theorem \ref{thm:schur_unitary_triangularization}) to produce a unitary matrix $U\in M_n(\C)$ and an upper triangular matrix $T\in M_n(\C)$ such that
\be
U^* AU = T.
\ee

Define $D_\alpha = \diag\bra{1,\alpha,\alpha^2,\dots,\alpha^{n-1}}$ for a non-zero scalar $\alpha\in \R$ and set $t = \max_{i<j}\abs{t_{ij}}$. Assume that $\ve < 1$, since it certainly suffices to
prove the statement in this case.

If $t\leq 1$, let $S_\ve = UD_\ve$, and, if $t>1$, let $S_\ve = UD_{1/t}D_\ve$. In either case, the appropriate $S_\ve$ substantiates the claim of the theorem.

If $t\leq 1$, a simple calculation reveals that
\be
t_{ij}(\ve) = \brb{S^{-1}_\ve A S_\ve}_{ij} = \brb{D_{1/\ve} T D_\ve}_{ij} =  t_{ij}\ve^{-i}\ve^j = t_{ij}\ve^{j-i},
\ee
whose absolute value is no more than $\ve^{j-i}$, which is, in turn, no more than $\ve$ if $i<j$.

If $t>1$, on the other hand, the similarity by $D_{1/t}$,
\be
T(\ve) = S^{-1}_\ve A S_\ve = D_{1/\ve} D_{t} TD_{1/t}D_\ve.
\ee

It's obvious that $D_{t} TD_{1/t}$ is upper triangular and for any $i<j$,
\be
\brb{D_{t} TD_{1/t}}_{ij} = t_{ij}t^it^{-j} \leq t_{ij} t^{-1} \leq 1.
\ee

Therefore, all off-diagonal entries are no more than 1 in absolute value and $\abs{t(\ve)_{ij}} \leq \abs{\brb{D_{1/\ve}D_{\ve}}_{ij}} = \ve^{j-i}$, which is, in turn, no more than $\ve$ if $i<j$.%\footnote{proof needed.}
\end{proof}


An extension of Schur's theorem, easily proved from it, is an important step toward the Jordan canonical form (Theorem \ref{thm:jordan_canonical_form}).

\begin{theorem}\label{thm:every_complex_matrix_similar_to_diagonal_block_upper_triangular}
Suppose that $A\in M_n(\C)$ has eigenvalues $\lm_i$ with multiplicity $n_i$, $i=1,\dots,k$, and that $\lm_1,\dots,\lm_k$ are distinct. Then $A$ is similar to a matrix of the form
\be
\bepm \ba{cc} T_1 & \\  & T_2 \ea &  \text{\Large 0} \\ \text{\Large 0} & \ba{cc} \ddots & \\  & T_k \ea \eepm
\ee
where $T_i\in M_{n_i}(\C)$ is upper triangular with all diagonal entries equal to $\lm_i$, $i=1,\dots,k$. If $A\in M_n(\R)$ and if all the eigenvalues of $A$ are real, then the same result holds,
and the similarity matrix may be taken to be real.
\end{theorem}

\begin{proof}[\bf Proof]%we can swap the rows and columns of $T$ by elementary matrices ($T_{i,j}$ in Definition \ref{def:elementary_matrix})
First apply Schur's theorem (Theorem \ref{thm:schur_unitary_triangularization}) to exhibit unitary equivalence to an upper triangular matrix $T = (t_{ij})$, and suppose that we have arranged that
all the $\lm_1$ terms come first (as the order can be prescribed in Schur's theorem again), the $\lm_2$ terms next, and so on, on the diagonal of $T$, i.e., $U^*AU = T$.

We next perform a sequence of simple similarities (non-unitary equivalence) on $T$ that produce the desired above-diagonal 0's, without changing the diagonal or the upper triangular structure of
$T$.

Let $E_{ij}$ be the matrix in $M_n(\C)$ with a 1 in the $(i,j)$ position and 0's elsewhere (which is a natural basis of $M_n(\C)$). Note that for $i\neq j$ and $\alpha$ any scalar, $I + \alpha
E_{ij}$ is nonsingular (which is actually an elementary matrix $C_{i,j,\alpha}$, see Definition \ref{def:elementary_matrix}) and
\be
\brb{I + \alpha E_{ij}}^{-1} = I - \alpha E_{ij},\qquad (\text{Proposition \ref{pro:square_elementary_matrix_invertible}}).
\ee

Furthermore, straightforward calculation reveals the similarity by $I + \alpha E_{ij}$ for $i<j$,
\be
\brb{I + \alpha E_{ij}}^{-1} T \brb{I + \alpha E_{ij}} = \brb{I - \alpha E_{ij}}  T \brb{I + \alpha E_{ij}}
\ee
changes entries of $T$ by adding $i$th row by $j$th row times $-\alpha$ and then adding $j$th column by $i$th column times $\alpha$. %only in the $i$th row, to the right of column $j$, and in the $j$th column above the $i$th row.
Thus, it only changes $t_{ij}$ replaces it by
\be
t_{ij} + \alpha (t_{ii} -t_{jj}).
\ee

Thus, if $t_{ii} \neq t_{jj}$, the $(i,j)$ entry may be made 0 by choosing
\be
\alpha_{ij} = -\frac{t_{ij}}{t_{ii} - t_{jj}}
\ee
without otherwise altering the relevant structure.

Now consider the sequnce of positions in $T$:
\be
(n-1,n),\quad (n-2,n-1),\quad (n-2,n),\quad (n-3,n-2) ,\quad (n-3,n-1),\quad (n-3,n),\quad (n-3,n-3)\dots.
\ee

Make each of these 0, in turn, via a similarity of the indicated sort, if $t_{ii} \neq t_{jj}$, and notice that no previously created 0 entry will be distributed. Let
\be
P = U \prod^n_{i<j}\brb{I + \alpha_{ij}E_{ij}}.
\ee

Then we have for the invertible matrix $P$,
\beast
P^{-1}AP &=& \brb{\prod^n_{i<j}\brb{I + \alpha_{ij}E_{ij}}}^{-1}U^{-1}AU \brb{\prod^n_{i<j}\brb{I + \alpha_{ij}E_{ij}}} = \brb{\prod^n_{i<j}\brb{I + \alpha_{ij}E_{ij}}}^{-1}U^*AU \brb{\prod^n_{i<j}\brb{I + \alpha_{ij}E_{ij}}} \\
& = & \brb{\prod^n_{i<j}\brb{I + \alpha_{ij}E_{ij}}}^{-1}T \brb{\prod^n_{i<j}\brb{I + \alpha_{ij}E_{ij}}} := S.
\eeast

The resulting matrix will be similar to $A$ and will have the desired form.%\footnote{proof needed.}
\end{proof}





\subsection{Normal matrices}

\begin{definition}[normal matrix\index{normal!matrix}]\label{def:normal_matrix}
Let $A\in M_n(\C)$. It is said to be normal if $A^*A = AA^*$, i.e., if $A$ commutes with its (Hermitian) adjoint.
\end{definition}

\begin{example}
\ben
\item [(i)] Since $U^*U = I = UU^*$ if $U$ is unitary, all unitary matrices are normal.
\item [(ii)] Since $A^*A = AA^*$ trivially if $A^* = A$, all Hermitian matrices are normal.
\item [(iii)] If $A \in M_n(\C)$ is such that $A^* = -A$ ($A$ is skew-Hermitian), then $A^*A = -A^2 = AA^*$. So all skew-Hermitian matrices are also normal.
\item [(iv)] $A = \bepm 1 & -1 \\ 1 & 1 \eepm$ is normal, but it does not fall into any of the above categories.
\be
A^*A = \bepm 1 & 1 \\ -1 & 1 \eepm \bepm 1 & -1 \\ 1 & 1 \eepm = \bepm 2 & 0 \\ 0 & 2 \eepm = \bepm 1 & -1 \\ 1 & 1 \eepm \bepm 1 & 1 \\ -1 & 1 \eepm = AA^*
\ee

\item [(v)] Similarly, the matrix $A = \bepm 1 & 1 & 0 \\ 0 & 1 & 1 \\ 1 & 0 & 1 \eepm$ is not the case that all normal matrices are either unitary or (skew-)Hermitian.
\be
A^*A = \bepm 2 & 1 & 1 \\ 1 & 2 & 1 \\ 1 & 1 & 2 \eepm = AA^*.
\ee
\een
\end{example}



\begin{proposition}[unitary equivalence preserves normality]\label{pro:unitary_equivalence_preserves_normality}
$A\in M_n(\C)$ is normal if and only if every matrix that is unitarily equivalent to $A$ is normal.
\end{proposition}

\begin{remark}
The class of normal matrices is closed under unitary equivalence.
\end{remark}

\begin{proof}[\bf Proof]%By Schur's theorem (Theorem \ref{thm:schur_unitary_triangularization}), for any $A\in M_n(\C)$ we have a unitary matrix $U$ such that $U^*A U = T$ where $T$ is upper triangular matrix.
If $A$ and $B$ are unitarily equivalent, then there exists a unitary matrix $U$ such that $B = U^*A U$. Then %Thus, we have $T = U^*V^* B VU$.
\be
B^*B = U^*A*U U^*AU = U^*A^*AU = U^*AA^*U = U^*AUU^*A^*U = BB^* \ \lra \ A\text{ is normal, i.e. }\ A^*A = AA^*,
\ee %\footnote{proof needed.}
as required.
\end{proof}

\begin{proposition}[class of Hermitian matrices is closed under unitary equivalence]\label{pro:hermitian_matrices_closed_under_unitary_equivalence}
$A\in M_n(\C)$ is Hermitian if and only if every matrix that is unitarily equivalent to $A$ is Hermitian.
\end{proposition}

\begin{proof}[\bf Proof]
If $A$ and $B$ are unitarily equivalent, then there exists a unitary matrix $U$ such that $B = U^*A U$. Then by Proposition \ref{pro:matrix_multiple_hermitian}%Thus, we have $T = U^*V^* B VU$.
\be B^* = U^*A*U = \brb{U^*AU} = B \ \lra \ A\text{ is Hermitian, i.e. }\ A = A^*,
\ee %\footnote{proof needed.}
as required.
\end{proof}



\begin{definition}[unitarily diagonalizable matrices\index{unitarily diagonalizable matrices}]
If $A\in M_n(\C)$ is unitarily equivalent to a diagonal matrix, $A$ is said to be unitarily diagonalizable, with a similar definition for orthogonally diagonalizable with respect to $M_n(\R)$.
\end{definition}

\begin{proposition}
Unitarily (or orthogonally diagonalizable) implies diagonalizable (but not conversely).
\end{proposition}

\begin{proof}[\bf Proof]
Direct result from definitions of Unitary diagonalizability and diagonalizability.
\end{proof}

\begin{lemma}\label{lem:normal_triangular_matrix_is_diagonal}
A normal triangular matrix $T\in M_n(\C)$ must be diagonal.
\end{lemma}

\begin{proof}[\bf Proof]
First, since $T$ is normal matrix, we have that $TT^* = T^*T$. Then (1,1) entry of $T^*T$, $\brb{T^*T}_{11}$ is equal to $\brb{TT^*}_{11}$, (1,1) entry of $TT^*$. This is,
\be
\ol{t_{11}}t_{11} = \sum^n_{i=1} t_{1i}\ol{t_{1i}} = \ol{t_{11}}t_{11} + \sum^n_{i=2}\abs{t_{1i}}^2 \ \ra \ \sum^n_{i=2}\abs{t_{1i}}^2 = 0 \ \ra \ t_{1i} = 0,\quad i = 2,\dots,n.
\ee

Similarly, the (2,2) entries of $T^*T$ and $TT^*$ are equivalent, i.e.,
\be
\sum^2_{i=1} \ol{t_{i2}}t_{i2} = \sum^n_{i=1} t_{2i}\ol{t_{2i}}.
\ee

Since $t_{1i}$ and $t_{i1}$ are all zero for $i=2,\dots,n$, we have
\be
\ol{t_{22}}t_{22} = \ol{t_{22}}t_{22} + \sum^n_{i=3} t_{2i}\ol{t_{2i}} = \ol{t_{22}}t_{22} + \sum^n_{i=3} \abs{t_{2i}}^2 \ \ra \ t_{2i} = 0,\quad i = 3,\dots, n.
\ee

In the same manner, assuming we have verified that $t_{ij} = 0$ and $t_{ji} = 0$ for $j>i$ and $i=1,\dots,k-1$, we may conclude that
\be
t_{ij} = 0,\quad j>i,\ i = 1,\dots,k.
\ee

Thus, we can have $t_{ij} = 0$ for all $j>i$ and therefore it implies that $T$ is diagonal.
\end{proof}


\begin{theorem}[spectral theorem for normal matrices]\label{thm:spectral_normal_matrices}
If $A = \brb{a_{ij}}\in M_n(\C)$ has eigenvalues $\lm_1,\dots,\lm_n$, the following statements are equivalent:
\ben
\item [(i)] $A$ is normal.
\item [(ii)] $A$ is unitarily diagonalizable.
\item [(iii)] $\sum^n_{i,j=1} \abs{a_{ij}}^2 = \sum^n_{i=1} \abs{\lm_i}^2$.
\item [(iv)] There is an orthonormal set of $n$ eigenvectors of $A$.
\een
\end{theorem}

\begin{proof}[\bf Proof]%\footnote{proof needed.}
We suppose that throughout that $T = \brb{t_{ij}} \in M_n(\C)$ is an upper triangular matrix which is unitarily equivalent to $A$, as guaranteed by Schur's theorem (Theorem
\ref{thm:schur_unitary_triangularization}). That is, $T = U^*AU$ for some unitary matrix $U\in M_n(\C)$.

(i) is equivalent to the statement that $T$ is normal by Proposition \ref{pro:unitary_equivalence_preserves_normality}.

(i) $\ra$ (ii). If $A$ is normal, then $T$ is normal. Then by Lemma \ref{lem:normal_triangular_matrix_is_diagonal}, we know that $T$ is diagonal and thus $A$ is unitarily diagonalizable.

(i) $\ra$ (ii). Since diagonal matrices are clearly normal and unitary equivalence preserves normality (Proposition \ref{pro:unitary_equivalence_preserves_normality}), we can have the required.

(ii) $\ra$ (iii). Since the diagonal entries of any diagonalization of $A$ are eigenvalues $\lm_1,\dots,\lm_n$ (in some order) by Corollary \ref{cor:similarity_same_eigenvalues}, we can have (iii)
from (ii) by Theorem \ref{thm:unitary_equivalence_elements_square_sum}.

(iii) $\ra$ (ii). Since $\lm_i$, $i=1,\dots,n$, are diagonal entries of $T$ (in some order), by Theorem \ref{thm:unitary_equivalence_elements_square_sum}, we have
\be
\sum^n_{i,j=1} \abs{a_{ij}}^2 = \sum^n_{i=1} \abs{\lm_{i}}^2 + \sum^n_{i<j} \abs{t_{ij}}^2 \ \ra \  \sum^n_{i<j} \abs{t_{ij}}^2 = 0 \qquad \text{from (iii)} \ \ra\  t_{ij} = 0,\quad i<j
\ee
and thus $A$ is unitarily diagonalizable.

(ii) $\ra$ (iv). If $A$ is unitarily diagonalizable, there exists unitary matrix $U$ such that $U^*AU = \Lambda$ where $\Lambda$ is a diagonal matrix whose eigenvalues are consistent with $A$. Thus,
\be
AU = \brb{U^{*}}^{-1}\Lambda = U\Lambda \qquad (*).
\ee

Thus, the $i$th column of $U\Lambda$ is $\lm_i U_i$ where $U_i$ is the $ith$ column of $U$. Also, the $i$th column of $AU$, $(AU)_i$, is $AU_i$. Therefore,
\be
A U_i = \lm_i U_i \ \ra \ U_i \text{ is eigenvector associated with eigenvalue }\lm_i.
\ee

Thus, since U is unitary, we have that $U_i$, $i=1,\dots,n$ form an orthonormal set by Theorem \ref{thm:unitary_matrix_property}.

(ii) $\ra$ (iv). Assume we have an orthonormal set $U_1,\dots,U_n$ of $n$ eigenvectors of $A$. Then by using ($*$), it is easy to show that $U = \brb{U_1,\dots,U_n}$ is the unitary matrix such that
$U^*AU = \Lambda$ where $\Lambda$ is diagonal matrix formed by eigenvalues (in some order) of $A$ .
\end{proof}

\begin{theorem}[spectral theorem for Hermitian matrices]\label{thm:spectral_hermitian_matrices}
If $A \in M_n(\C)$ is Hermitian, then
\ben
\item [(i)] All eigenvalues of $A$ are real.
\item [(ii)] $A$ is unitarily diagonalizable.
\een

If $A\in M_n(\R)$ is real (and thus symmetric), then $A$ is real orthogonally diagonalizable.
\end{theorem}

\begin{proof}[\bf Proof]%\footnote{proof needed.}
(ii) follows from Theorem \ref{thm:spectral_normal_matrices} as Hermitian is normal.

Since (ii) and the fact that the set of Hermitian matrices is closed under unitary equivalence (by Proposition \ref{pro:hermitian_matrices_closed_under_unitary_equivalence}), the corresponding
diagonal matrix must be Hermitian. As a diagonal Hermitian matrix must have real diagonal entries, we can have that all the eigenvalues of $A$ are real by Corollary
\ref{cor:similarity_same_eigenvalues}.

If $A\in M_n(\R)$ is symmetric, then it is Hermitian, and thus all the eigenvalues of $A$ are real. Then by Schur's theorem (Theorem \ref{thm:schur_unitary_triangularization}), there exists
orthogonal matrix $U$ such that $U^T AU = T$ where $T$ is an upper triangular matrix. Since $T$ is also symmetric ($T^T = U^TA^T U = U^TA U = T$), we have that $T$ is diagonal (or we can get this by
Lemma \ref{lem:normal_triangular_matrix_is_diagonal}). Thus, $A$ is real orthogonally diagonalziable.
\end{proof}

\begin{theorem}
Let $A\in M_n(\R)$. Then $A$ is normal if and only if there is a real orthogonal matrix $Q\in M_n(\R)$ such that
\be Q^TAQ =
\bepm
\ba{cc} A_1 & \\ & A_2 \ea & \text{\Large $*$} \\ \text{\Large 0} & \ba{cc} \ddots & \\
& A_k \ea \eepm\in M_n(\R),\quad 1\leq k\leq n
\ee
where each $A_i$ is either a real $1\times 1$ matrix or a real $2\times 2$ matrix of the form
\be
A_i = \bepm \alpha_i & \beta_i \\ -\beta_i & \alpha_i \eepm.
\ee
\end{theorem}

\begin{proof}[\bf Proof]
\footnote{proof needed.}
\end{proof}


\subsection{$QR$ factorization and algorithm}

A particular means for calculating a specific unitary Schur's upper triangularization of a given matrix $A\in M_n(\C)$, and a popular numerical method for calculating eigenvalues (under some
assumptions) is called the $QR$ algorithm. It is based on the so-called $QR$ factorization of a general matrix $A\in M_{m,n}(\C)$.

\begin{theorem}[$QR$ factorization\index{QR factorization@$QR$ factorization}]\label{thm:qr_factorization}
If $A \in M_{m,n}(\C)$ and $m\geq n$, there is a matrix $Q \in M_{m,n}(\C)$ with orthonormal columns and an upper triangular matrix $R\in M_{n}(\C)$ such that $A = QR$.

If $m=n$, $Q$ is unitary; if in addition $A$ is nonsingular, then $R$ may be chosen so that all its diagonal entries are positive, and in this case, the factors $Q$ and $R$ are both unique.

If $A\in M_{m,n}(\R)$, then $Q$ and $R$ may be taken to be real.
\end{theorem}

\begin{proof}[\bf Proof]%\footnote{proof needed.}
If $A\in M_{m,n}(\C)$ and $\rank (A) = n$, the $QR$ factorization of $A$ is just a description, in matrix notation, of the result of applying the Gram-Schmidt orthonormalization (Algorithm
\ref{alg:gram_schmidt_orthonormalization}) to the columns of $A$, which comprise an independent set in $\C^m$.

A natural extension of the Gram-Schmidt algorithm permits the same description to apply to the general case in which the columns of $A$ may be dependent (see the remark of Algorithm
\ref{alg:gram_schmidt_orthonormalization}).

Let $A = \brb{a_1,\dots,a_n}$ be written in partitioned form in terms of its columns $a_i\in \C^m$. If $a_1 = 0$, set $q_1 = 0$; otherwise set $q_1 = a_1 /\brb{a_1^*a_1}^{1/2}$. For each
$k=2,3,\dots,n$, compute
\be
y_k = a_k - \sum^{k-1}_{i=1}\brb{q_i^* a_k}q_i
\ee
just as in the ordinary Gram-Schmidt orthonormalization. If $y_k = 0$ (which happens if and only if $a_k$ is a linear combination of $a_1,a_2,\dots,a_{k-1}$), set $q_k = 0$; otherwise set $q_k = y_k
/ \brb{y_k^*y_k}^{1/2}$. The vectors $q_1,\dots,q_n$ are then orthogonal set (in $\C$), each element of which is either a unit vector or the zero vector. Each vector $q_i$ is a linear combination of
$a_1,\dots,a_i$, and the construction ensures that, conversely, each column $a_i$ is linear combination of $q_1,\dots,q_i$. Thus, scalars $r_{ki}$ exist such that
\be
a_i = \sum^i_{k=1}r_{ki}q_i,\qquad i = 1,2,\dots, n.\qquad (*)
\ee

If we set $r_{ki}=0$ for all $k>i$ and set $r_{ji} = 0$ for all $i = 1,2,\dots,n$ for each $j$ such that $q_j = 0$, the upper triangular matrix $R = (r_{ij})\in M_n(\C)$ and the vectors
$q_1,q_2,\dots,q_n$ are determined, via the outlined procedure, by $a_1,\dots,a_n$. The matrix $Q := \brb{q_1,\dots,q_n}\in M_{m,n}(\C)$ has orthogonal columns (in $\C$ and some of which may be
zero), and ($*$) says that $A = QR$.

If $\rank(A) = n$, $Q$ has orthonormal columns, and hence a factorization of the desired form has been achieved.

If the columns of $A$ are not independent ($\rank(A) < n$), take the (orthonormal) set of non-zero columns of $Q$ and extend it to an orthonormal basis of $\C^m$ by Steinitz exchange theorem
(Theorem \ref{thm:steinitz_exchange}). Denote the new vectors obtained in this way by $z_1,\dots,z_p$. Now replace the first zero column of $Q$ by $z_1$, replace the second zero column by $z_2$, and
so on until all zero columns have been replaced in this way. Denote the resulting matrix by $Q'$. Then $Q'$ has orthonormal columns and $QR = Q'R$ because the new columns of $Q'$ are matched by
zero rows of $R$. Then $A = Q'R$ is a factorization of the desired form.

In particular, if $m=n$ and $A$ is nonsingular, then $Q$ must be unitary by Theorem \ref{thm:unitary_matrix_property} and the diagonal entries of the nonsingular matrix $R = Q^*A$ must be non-zero.
In this event, because $R$ is required to be upper triangular, $q_1$ is a scalar multiple of $a_1$, and for $i=2,3,\dots,n$, $q_i$ lies in the one-dimensional space that is the orthogonal complement
of the span $a_1,\dots,a_{i-1}$ with respect to the span of $a_1,\dots,a_i$. Therefore, each $q_i$ is uniquely determined up to a factor of scale of absolute value 1. Thus, replacement of $R$ by
\be
R' := \diag\brb{\frac{\abs{r_{11}}}{r_{11}},\dots, \frac{\abs{r_{nn}}}{r_{nn}}} R
\ee
and replacement of $Q$ by
\be
Q' := Q\diag\brb{\frac{r_{11}}{\abs{r_{11}}},\dots, \frac{r_{nn}}{\abs{r_{nn}}}}
\ee
gives the unique factorization $A = Q'R'$ promised in the statement of the theorem.


If $A$ is real, notice that all the necessary operations may be carried out in real arithmetic, so the factors obtained are real.
\end{proof}

\begin{example}\label{exa:block_triangular_matrix_determinant_qr_factorization}
We can apply $QR$ factorization in alternative proof for Proposition \ref{pro:block_triangular_matrix_determinant}. That is, for $A\in M_m(\F),B\in M_n(\F),C\in M_{m,n}(\F)$, $\det\bepm A & C \\ 0 & B \eepm = \det A \det B$.

We can use $QR$ factorization (Theorem \ref{thm:qr_factorization}) by letting
\be
A = Q_A R_A,\quad B = Q_BR_B
\ee
where $Q_A,Q_B$ are unitary and $R_A,R_B$ are upper triangular. Then by Theorem \ref{thm:determinant_product}
\beast
\det \bepm A & C \\ 0 & B \eepm & = & \det\bepm Q_AR_A & Q_AQ_A^*C \\ 0 & Q_BR_B \eepm = \det \brb{\bepm Q_A & \\ & Q_B \eepm \bepm R_A & Q_A^*C \\ 0 & R_B\eepm} \\
& = & \det \bepm Q_A & \\ & Q_B \eepm \det\bepm R_A & Q_A^*C \\ 0 & R_B\eepm = \det Q \det R
\eeast
where
\be
Q:=  \bepm Q_A & \\ & Q_B \eepm,\quad R := \bepm R_A & Q_A^*C \\ 0 & R_B\eepm.
\ee

Notice that $R$ is upper triangular, so its determinant is equal to the product of its diagonal elements, so
\be
\det R = \det \bepm R_A & Q_A^*C \\ 0 & R_B\eepm = \bepm R_A & 0 \\ 0 & R_B\eepm.
\ee

Then by formula for the determinant of block diagonal matrix (Proposition \ref{pro:block_diagonal_matrix_property}.(i)) and Theorem \ref{thm:determinant_product} again
\beast
\det \bepm A & C \\ 0 & B \eepm & = & \det Q\det R = \det\bepm Q_A & \\ & Q_B \eepm\det\bepm R_A & 0 \\ 0 & R_B\eepm = \det Q_A \det Q_B \det R_A \det R_B \\
& = & \det Q_A \det R_A \det Q_B  \det R_B = \det (Q_AR_A)\det (Q_BR_B) = \det A\det B.
\eeast
\end{example}

We next state the $QR$ algorithm for eigenvalue calculation and briefly indicate some of its features.

\begin{algorithm}
Let $A_0\in M_n(\C)$ be given. Write $A_0 = Q_0R_0$, where $Q_0$ and $R_0$ are as guaranteed in $QR$ factorization (Theorem \ref{thm:qr_factorization}), and define $A_1 = R_0 Q_0$. Again, write $A_1
= Q_1 R_1$, with $Q_1$ unitary and $R_1$ upper triangular, and continue.

In general, factor $A_k = Q_kR_k$ and define $A_{k+1} = R_kQ_k$.
\end{algorithm}

\begin{remark}
Since $A_{k+1} = R_kQ_k$, we have
\be
Q_k A_{k+1} Q_k^* = Q_k R_k Q_k Q_k^* = Q_k R_k = A_k \ \ra\ A_k, A_{k+1}\ \text{ are unitarily equivalent.}
\ee

Thus, we can have each $A_k$ is unitarily equivalent to $A_0$.

Under certain circumstance (for example, if all the eigenvalues of $A_0$ have distinct absolute values), the $QR$ iterates $A_k$ will converge to an upper triangular matrix as
$k\to\infty$.\footnote{proof needed.} Since this upper triangular matrix is unitarily equivalent to $A_0$, the eigenvalues of $A_0$ are revealed.

If $A_0$ is real, then the $QR$ iterates $A_k$ may be calculated using real arithmetic. If $A_0$ has any non-real eigenvalues, there is no hope that the $QR$ iterates will converges to an upper
triangular matrix, since this upper triangular limit must be real (by real case of $QR$ factorization (Theorem \ref{thm:qr_factorization})).

Under certain circumstances, however, the iterates $A_k$ may be chosen so that they converge to a real block upper triangular matrix with $1\times 1$ and $2\times 2$ main diagonal
blocks.\footnote{proof needed.} A sufficient condition for this to occur is that all the eigenvalues of $A_0$ have distinct moduli, expect for the two eigenvalues in each non-real complex conjugate
pair, which have the same modulus.

Since the eigenvalues of a block triangular matrix are the union of the sets of eigenvalues of the diagonal blocks, the eigenvalues of $A_0$ are revealed as the $1\times 1$ diagonal entries of the
block triangular limit of the $QR$ iterates $A_k$, together with the (complex conjugate pairs of) eigenvalues of the $2\times 2$ diagonal blocks of the limit, which may be calculated easily using
the real arithmetic and the quadratic formula.
\end{remark}

\section{Canonical Forms}

\subsection{The Jordan canonical form}

Since two matrices that look very different can still be similar, one approach to determining whether two given matrices are similar is to have some set of `simple' matrices of prescribed form and
then see if both given matrices can be reduced by similarity to one of these `simple' forms.

Every complex $A$ is unitarily equivalent (similar) to an upper triangular matrix whose diagonal entries (the eigenvalues of $A$) may be arranged in any given order (by Schur's theorem (Theorem
\ref{thm:schur_unitary_triangularization})), so two matrices are similar if they are similar to the same upper triangular matrix.

However, two upper triangular matrices with the same main diagonal entries and different off-diagonal entries can still be similar. Thus, if we have succeeded in reducing the two given matrices to
two unequal upper triangular matrices with the same main diagonal, we cannot conclude that the matrices are not similar.

As the class triangular matrices is too large for our purposes, we turn to the class diagonal matrices. Unfortunately, not every complex matrix is similar to a diagonal matrix.

\begin{example}
Consider the matrices in Example \ref{exa:same_eigenvalues_do_not_imply_similarity},
\be
A = \bepm 0 & 1 \\ 0 & 0 \eepm,\qquad B = \bepm 0 & 0 \\ 0 & 0 \eepm.
\ee

We can seet that they have the same trace, determinant, characteristic polynomial and eigenvalues ($\bra{0,0}$) and $B$ is diagonal. However, $A$ is not diagonalizable. If it is, we have that the
diagonal matrix $\Lambda$ should be $B$ and
\be
A = P\Lambda P^{-1} = PB P^{-1} = \bepm 0 & 0 \\ 0 & 0 \eepm \qquad \ra \quad \text{Contradiction.}
\ee
\end{example}

To trade off between feasibility and complexity, we want to search for an upper triangular form that is as nearly diagonal as possible but is still attainable by similarity for every matrix.

The result is Jordan canonical form. Once one knows the Jordan canonical form of a matrix, all the linear algebraic information about the given matrix (i.e., linear transformation) is known at a
glance.

\begin{definition}[Jordan block\index{Jordan block}]\label{def:jordan_block}
A Jordan block $J_k(\lm)$ is a $k\times k$ upper triangular matrix of the form
\be
J_k(\lm) = \bepm \ba{ccc} \lm & 1 & \\ & \lm & 1 \\ & & \lm \ea & \text{\Large 0} \\ \text{\Large 0} & \ba{ccc} \ddots & \ddots & \\ & \lm  & 1  \\ & & \lm \ea\eepm.
\ee

There are $k-1$ terms `+1' in the superdiagonal; the scalar $\lm$ appears $k$ times on the main diagonal. All other entries are zero, and $J_1(\lm) = \brb{\lm}$.
\end{definition}


\begin{definition}[Jordan matrix\index{Jordan matrix}]\label{def:jordan_matrix}
A Jordan matrix $J\in M_n(\C)$ is direct sum (see Definition \ref{def:block_diagonal_matrix}) of Jordan blocks
\be
J = \bepm \ba{cc} J_{n_1}(\lm_1) & \\ & J_{n_2}(\lm_2) \ea & \text{\Large 0} \\ \text{\Large 0} & \ba{cc} \ddots & \\ & J_{n_k}(\lm_k) \ea\eepm,\qquad n_1 + n_2 + \dots + n_k = n
\ee
in which the order $n_i$ may not be distinct and the values $\lm_i$ need not be distinct.
\end{definition}

\begin{remark}\label{rem:jordan_matrix_diagonalizable}
Note that if each Jordan block $J_{n_i}(\lm_i)$ is one-dimensional, that is, all $n_i =1$ and $k=n$, then the Jordan matrix $J$ is diagonal.

If any Jordan block $J_m(\lm)$ has $m>1$, then $J$ is not only not diagonal, it is not even diagonalizable. If $J_m(\lm)$ is diagonalizable such that $J_m(\lm) = P\Lambda P^{-1}$ with $\Lambda$
diagonal, then necessarily $\Lambda = \diag\brb{\lm,\lm,\dots,\lm} = \lm I$. Thus,
\be
J_m(\lm) - \lm I = P\Lambda P^{-1} - \lm I = P \lm I P^{-1} - \lm I = \lm I - \lm I = 0,
\ee
which is not the case if $m>1$.
\end{remark}

\begin{lemma}\label{lem:jordan_block_property}
Let $k\geq 1$ be given, and consider the Jordan block
\be
J_k(0) = \bepm \ba{ccc} 0 & 1 & \\ & 0 & 1 \\ & & 0 \ea & \text{\Large 0} \\ \text{\Large 0} & \ba{ccc} \ddots & 1 & \\ & \ddots & 1 \\ & & 0 \ea\eepm.
\ee

Then
\be
J_k^T(0) J_k(0) = \bepm 0 & 0 \\ 0 & I_{k-1}\eepm,\qquad J_k(0)^p = 0,\quad p \geq k.
\ee

Moreover,
\be
J_k(0)e_{i+1} = e_i,\quad i = 1,2,\dots,k-1
\ee
and
\be
\brb{I - J_k^T(0) J_k(0)}x = \brb{x^T e_1}e_1
\ee
where $I_{k-1}\in M_{k-1}(\C)$ is the identity matrix, $e_i$ is the $i$th standard unit basis vector in $\C^k$, and $x\in \C^k$.
\end{lemma}

\begin{proof}[\bf Proof]
Let $W = J_k^T(0) J_k(0)$, then for $J_k(0) := \brb{x_{ij}}$,
\be
w_{ij} = \sum^k_{m=1} x_{mi}x_{mj}.
\ee

If $i=1$ or $j=1$, then $x_{mi} = x_{mj} = 0$. For $i,j\geq 2$,
\be
w_{ij} = x_{mi}x_{mj}\delta_{ij} = \delta_{ij}
\ee
as required. Also, we can have
\be
J_k(0)^2 = \bepm \ba{ccc} 0 & 1 & \\ & 0 & 1 \\ & & 0 \ea & \text{\Large 0} \\ \text{\Large 0} & \ba{ccc} \ddots & 1 & \\ & \ddots & 1 \\ & & 0 \ea\eepm
\bepm \ba{ccc} 0 & 1 & \\ & 0 & 1 \\ & & 0 \ea & \text{\Large 0} \\ \text{\Large 0} & \ba{ccc} \ddots & 1 & \\ & \ddots & 1 \\ & & 0 \ea\eepm.
= \bepm \ba{ccc} 0 & 0 & 1 \\ & 0 & 0 \\ & & 0 \ea & \text{\Large 0} \\ \text{\Large 0} & \ba{ccc} \ddots & \ddots & 1 \\ \ddots & 0 & 0 \\ & 0 & 0 \ea\eepm
\ee
which has `+1' entries only at $(i,j)$ with $j - i = 2$. Similarly, we can have $J_k(0)^{k-1}$ has `+1' entries only at $(i,j)$ with $j - i = k-1$, which is the top right element.
Accordingly, it is easy to see that $J_k(0)^p = 0$ for any $p\geq k$.

Moreover, for $i=1,2,\dots,k-1$, $j$th entry of vector $J_k(0)e_{i+1}$ is
\be
\sum^k_{m=1}\brb{J_k(0)}_{jm}\brb{e_{i+1}}_m = \brb{J_k(0)}_{j,i+1} = \delta_{ij} \ \ra \ J_k(0)e_{i+1} = e_i.
\ee

Also, for $x = \brb{x_1,\dots,x_k}^T\in \C^n$,
\be
\brb{I - J_k^T(0) J_k(0)}x = \bepm \ba{cc} 1 & 0 \\ 0 & 0 \ea & \text{\Large 0} \\ \text{\Large 0} &  \ba{cc} 0 & 0 \\ 0 & 0 \ea \eepm \bepm x_1 \\ x_2 \\ \vdots \\ x_k \eepm
=  \bepm x_1 \\ 0 \\ \vdots \\ 0 \eepm = x_1 \bepm 1 \\ 0 \\ \vdots \\ 0 \eepm =  \brb{x^T e_1}e_1
\ee
as required.
\end{proof}

Now we give the reduction of a strictly upper triangular matrix (which is a triangular matrix with zero entries on the main diagonal).

\begin{theorem}\label{thm:strictly_upper_triangular_jordan_canonical_form}
Let $A\in M_n\brb{\C}$ be strictly upper triangular. Then there is a nonsingular $P \in M_n(\C)$ and integers $n_1 \geq n_2 \geq \dots \geq n_m \geq 1$ and $n_1 + n_2 + \dots + n_m = n$ such that
\be
A = P \bepm \ba{cc} J_{n_1}(0) & \\ & J_{n_2}(0) \ea & \text{\Large 0} \\ \text{\Large 0} &  \ba{cc} \ddots &  \\ & J_{n_m}(0) \ea \eepm P^{-1}\qquad (*)
\ee
where $J_{n_i}(0)$, $i = 1,\dots,m$ is the Jordan block with scalar 0.

If $A$ is real, the matrix $P$ may be chosen to be real.
\end{theorem}

\begin{proof}[\bf Proof]
If $n=1$, $A = \brb{0}$ and the result is trivial. We now proceed by induction on $n$ and assume that $n>1$ and that the result has been proved for all strictly upper triangular matrices of order
less than $n$. Then we can partition $A$ as
\be
A = \bepm 0 & a^T \\ 0 & A_1 \eepm
\ee
where $a\in \C^{n-1}$ and $A_1\in M_{n-1}(\C)$ is strictly upper triangular. By the induction hypothesis, there is a nonsingular matrix $P_1\in M_{n-1}(\C)$ such that $P_1^{-1}A_1P_1$ has the desired form ($*$), that is
\be
P_1^{-1}A_1P_1 = \bepm \ba{cc} J_{k_1}(0) & \\ & J_{k_2}(0) \ea & \text{\Large 0} \\ \text{\Large 0} &  \ba{cc} \ddots &  \\ & J_{k_s}(0) \ea \eepm = \bepm J_{k_1}(0) & 0 \\ 0 & J \eepm
\ee
where $k_1 \geq k_2 \geq \dots \geq k_s \geq 1$, $k_1 + k_2 + \dots + k_s = n-1$, and
\be
J := \bepm \ba{cc} J_{k_2}(0) & \\ & \ddots \ea & \text{\Large 0} \\ \text{\Large 0} &  \ba{cc} \ddots &  \\ & J_{k_s}(0) \ea \eepm \in M_{n-k_1 -1}(\C).
\ee

Note that no diagonal Jordan block in $J$ has order greater than $k_1$ ($k_1 \geq k_2,\dots,k_m$), so $J^{k_1} = 0$ by Lemma \ref{lem:jordan_block_property}. A simple computation shows that
\be
\bepm 1 & 0 \\ 0 & P_1^{-1} \eepm A \bepm 1 & 0 \\ 0 & P_1 \eepm = \bepm 1 & 0 \\ 0 & P_1^{-1} \eepm  \bepm 0 & a^T \\ 0 & A_1 \eepm \bepm 1 & 0 \\ 0 & P_1 \eepm
= \bepm 0 & a^T \\ 0 & P_1^{-1}A_1 \eepm \bepm 1 & 0 \\ 0 & P_1 \eepm = \bepm 0 & a^TP_1 \\ 0 & P_1^{-1}A_1 P_1\eepm.
\ee

Partition $a^T P_1 = \brb{a_1^T,a_2^T}$ where $a_1\in \C^{k_1}$ and $a_2 \in \C^{n-k_1-1}$ such that
\be
\bepm 1 & 0 \\ 0 & P_1^{-1} \eepm A \bepm 1 & 0 \\ 0 & P_1 \eepm = \bepm 0 & a_1^T & a_2^T \\ 0 & J_{k_1}(0) & 0 \\ 0 & 0 & J \eepm.
\ee

Now consider the following similarly of this matrix,
\beast
& & \bepm 1 & -a_1^TJ^T_{k_1}(0) & 0 \\ 0 & I_{k_1} & 0 \\ 0 & 0 & I_{n-k_1 -1} \eepm \bepm 0 & a_1^T & a_2^T \\ 0 & J_{k_1}(0) & 0 \\ 0 & 0 & J \eepm \bepm 1 & a_1^TJ^T_{k_1}(0) & 0 \\ 0 & I_{k_1} & 0 \\ 0 & 0 & I_{n-k_1 -1} \eepm
\\
& = &  \bepm 0 & a_1^T\brb{I_{k_1}- J^T_{k_1}(0)J_{k_1}(0)} & a_2^T \\ 0 & J_{k_1}(0) & 0 \\ 0 & 0 & J \eepm \bepm 1 & a_1^TJ^T_{k_1}(0) & 0 \\ 0 & I_{k_1} & 0 \\ 0 & 0 & I_{n-k_1 -1} \eepm
=  \bepm 0 & a_1^T\brb{I_{k_1}- J^T_{k_1}(0)J_{k_1}(0)} & a_2^T \\ 0 & J_{k_1}(0) & 0 \\ 0 & 0 & J \eepm .
\eeast

By Lemma \ref{lem:jordan_block_property}, we have it is
\be
\bepm 0 & \brb{a_1^T e_1}e_1^T & a_2^T \\ 0 & J_{k_1}(0) & 0 \\ 0 & 0 & J \eepm   \qquad (\dag)
\ee
where $e_i$ is the $i$th standard unit basis vector in $\C^{k_1}$. Then we have the following two cases.

{\bf Case 1.} If $a_1^T e_1 \neq 0$, then the similarity of this matrix \beast & & \bepm 1/\brb{a_1^T e_1} & 0 & 0 \\ 0 & I_{k_1} & 0 \\ 0 & 0 & 1/\brb{a_1^T e_1} I_{n-k_1 -1} \eepm \bepm 0 &
\brb{a_1^T e_1}e_1^T & a_2^T \\ 0 & J_{k_1}(0) & 0 \\ 0 & 0 & J \eepm  \bepm a_1^T e_1 & 0 & 0 \\ 0 & I_{k_1} & 0 \\ 0 & 0 & \brb{a_1^T e_1} I_{n-k_1 -1} \eepm
\\
& = & \bepm 0 & e_1^T & a_2^T/\brb{a_1^T e_1} \\ 0 & J_{k_1}(0) & 0 \\ 0 & 0 & J/\brb{a_1^T e_1} \eepm  \bepm a_1^T e_1 & 0 & 0 \\ 0 & I_{k_1} & 0 \\ 0 & 0 & \brb{a_1^T e_1} I_{n-k_1 -1} \eepm
= \bepm 0 & e_1^T & a_2^T \\ 0 & J_{k_1}(0) & 0 \\ 0 & 0 & J \eepm := \bepm \wt{J} & e_1'a_2^T \\ 0 & J \eepm
\eeast
where
\be
\wt{J} = \bepm 0 & e_1^T \\ 0 & J_{k_1}(0) \eepm = J_{k_1 + 1}(0)
\ee
and $e_i'$ is the $i$th standard unit basis vector in $\C^{k_1+1}$. Using the property that $\wt{J}e'_{i+1} = e'_{i}$ for $i=1,2,\dots,k_1$, we can easily show that the similarity of this matrix
\beast
& & \bepm I_{k_1+1} & e_2'a_2^T \\ 0 & I_{n-k_1 -1} \eepm  \bepm \wt{J} & e_1'a_2^T \\ 0 & J \eepm \bepm I_{k_1+1} & -e_2'a_2^T \\ 0 & I_{n-k_1 -1} \eepm \\
& = & \bepm \wt{J} & e_1'a_2^T + e_2'a_2^TJ \\ 0 & J \eepm \bepm I_{k_1+1} & -e_2'a_2^T \\ 0 & I_{n-k_1 -1} \eepm  = \bepm \wt{J} & - \wt{J}e_2'a_2^T + e_1'a_2^T + e_2'a_2^TJ \\ 0 & J \eepm
= \bepm \wt{J} & e_2'a_2^TJ \\ 0 & J \eepm.
\eeast

Similarly, we can have the similarity of these matrices, for $i = 1,2,\dots$,
\beast
& & \bepm I_{k_1+1} & e_{i+1}'a_2^T J^{i-1} \\ 0 & I_{n-k_1 -1} \eepm  \bepm \wt{J} & e_i'a_2^TJ^{i-1} \\ 0 & J \eepm \bepm I_{k_1+1} & -e_{i+1}'a_2^T J^{i-1} \\ 0 & I_{n-k_1 -1} \eepm \\
& = & \bepm \wt{J} & e_{i}'a_2^T J^{i-1} + e_{i+1}'a_2^TJ^i \\ 0 & J \eepm \bepm I_{k_1+1} & -e_{i+1}'a_2^T J^{i-1} \\ 0 & I_{n-k_1 -1} \eepm  \\
& = & \bepm \wt{J} & - \wt{J}e_{i+1}'a_2^T J^{i-1} + e_i 'a_2^TJ^{i-1} + e_{i+1}'a_2^TJ^i \\ 0 & J \eepm = \bepm \wt{J} & e_{i+1}'a_2^TJ^i \\ 0 & J \eepm.
\eeast

Since $J^{k_1} =0$, we see that after $k_1$ steps in this series of similarities, the off-diagonal term will finally vanish. We can then conclude that $A$ is similar to the matrix
\be
\bepm \wt{J} & 0 \\ 0 & J \eepm
\ee
which is strictly upper triangular Jordan matrix of the required form.

{\bf Case 2.} If $a_1^Te_1 = 0$, then ($\dag$) shows that $A$ is similar to the matrix
\be
\bepm 0 & 0 & a_2^T \\ 0 & J_{k_1}(0) & 0 \\ 0 & 0 & J \eepm
\ee

which is permutation similar to the matrix

\beast
& & \brb{\ba{c;{2pt/2pt}c} \ba{ccccc} 0 & 1 & & & \\ & 0 & 1 & &  \\ & & \ddots & 1 &  \\ 0 & & & \ddots & 1 \\ 1 & 0 & \dots & &  0 \ea & \text{\Large 0} \\ \hdashline[2pt/2pt] & \\ \text{\Large 0} & I_{n-k_1 -1} \ea}
\bepm 0 & 0 & a_2^T \\ 0 & J_{k_1}(0) & 0 \\ 0 & 0 & J \eepm
\brb{\ba{c;{2pt/2pt}c} \ba{ccccc} 0 & & & 0 & 1 \\ 1 & 0 &  & & 0 \\ & 1 & \ddots & &  \\ & & & \ddots &  \\ 0 & 0 & \dots & 1 &  0 \ea & \text{\Large 0} \\ \hdashline[2pt/2pt] & \\ \text{\Large 0} & I_{n-k_1 -1} \ea}
\\
& = & \bepm 0 & J_{k_1}(0) & 0 \\ 0 & 0 & a_2^T \\  0 & 0 & J \eepm \brb{\ba{c;{2pt/2pt}c} \ba{ccccc} 0 & & & 0 & 1 \\ 1 & 0 &  & & 0 \\ & 1 & \ddots & &  \\ & & & \ddots &  \\ 0 & 0 & \dots & 1 &  0
\ea & \text{\Large 0} \\ \hdashline[2pt/2pt] & \\ \text{\Large 0} & I_{n-k_1 -1} \ea}
= \bepm J_{k_1}(0) & 0 & 0  \\ 0 & 0 & a_2^T \\  0 & 0 & J \eepm.      \qquad (\dag\dag)
\eeast

By the induction hypothesis, there is a nonsingular $P_2\in M_{n-k_1}(\C)$ such that
\be
P_2^{-1}\bepm 0 & a_2^T \\ 0 & J \eepm P_2 = \wh{J} \in M_{n-k_1}(\C)
\ee
is a Jordan matrix with zero main diagonal. Thus,
\be
\bepm I_{k_1} & 0 \\ 0 & P_2^{-1} \eepm  \bepm J_{k_1}(0) & 0 & 0  \\ 0 & 0 & a_2^T \\  0 & 0 & J \eepm  \bepm I_{k_1} & 0 \\ 0 & P_2 \eepm  = \bepm J_{k_1}(0) & 0 \\ 0 & \wh{J} \eepm
\ee

which is a Jordan matrix of the required form, except that the diagonal Jordan blocks might not be arranged in non-increasing order. A block permutation similarity (see Definition
\ref{def:block_permutation_matrix}, similar with ($\dag\dag$)), if necessary, will produce the required form.

Finally, observe that if $A$ is real then all the similarities used in this proof can be chosen to be real (as we can find invertible $P_1\in M_{n-1}(\R),P_2\in M_{n-k_1}(\R)$), so $A$ is similar
via a real similarity to the required Jordan matrix.%\brb{\ba{c;{2pt/2pt}c} \ba{ccc} * & & * \\ & \ddots & \\ 0 & & * \ea & \text{\large $*$} \\ \hdashline[2pt/2pt]
\end{proof}


\begin{corollary}\label{cor:upper_triangular_jordan_canonical_form}
For an upper triangular matrix with all diagonal entries equal to $\lm$,
\be
A = \bepm \ba{cc} \lm & \\ & \lm \ea & \text{\Large $*$} \\ \text{\Large 0} &  \ba{cc} \ddots &  \\ & \lm \ea \eepm ,
\ee

there exists nonsingular matrix $P$ such that
\be
A = P \bepm \ba{cc} J_{n_1}(\lm) & \\ & J_{n_2}(\lm) \ea & \text{\Large 0} \\ \text{\Large 0} &  \ba{cc} \ddots &  \\ & J_{n_m}(\lm) \ea \eepm P^{-1}
\ee
where $J_{n_i}(\lm)$ are Jordan blocks with scalar $\lm$.

If $A$ is real, the matrix $P$ may be chosen to be real.
\end{corollary}

\begin{proof}[\bf Proof]
By assumption, we know that $A_0 = A - \lm I$ is a strictly upper triangular matrix, thus by Theorem \ref{thm:strictly_upper_triangular_jordan_canonical_form} we can find a nonsingular matrix $P$ such that
\be
A_0 = P \bepm \ba{cc} J_{n_1}(0) & \\ & J_{n_2}(0) \ea & \text{\Large 0} \\ \text{\Large 0} &  \ba{cc} \ddots &  \\ & J_{n_m}(0) \ea \eepm P^{-1} = PJP^{-1}.
\ee

Also, we can have that
\be
A = A_0 + \lm I = PJP^{-1} + \lm I PP^{-1} = P\brb{J+\lm I}P^{-1} = P \bepm \ba{cc} J_{n_1}(\lm) & \\ & J_{n_2}(\lm) \ea & \text{\Large 0} \\ \text{\Large 0} &  \ba{cc} \ddots &  \\ & J_{n_m}(\lm) \ea \eepm P^{-1},
\ee
as required. The case for real $A$ is also the direct result from Theorem \ref{thm:strictly_upper_triangular_jordan_canonical_form}.
\end{proof}

\begin{theorem}[Jordan canonical form\index{Jordan canonical form}]\label{thm:jordan_canonical_form}
Let $A\in M_n(\C)$ be a given complex matrix. Then there is a nonsingular matrix $P\in M_n(\C)$ such that
\be
A = P \bepm \ba{cc} J_{n_1}(\lm_1) & \\ & J_{n_2}(\lm_2) \ea & \text{\Large 0} \\ \text{\Large 0} &  \ba{cc} \ddots &  \\ & J_{n_k}(\lm_k) \ea \eepm P^{-1}  = PJP^{-1}
\ee
and $n_1+n_2 + \dots + n_k = n$.

The Jordan matrix $J$ of $A$ is unique up to permutations of the diagonal Jordan blocks.

The eigenvalues $\lm_i$, $i= 1,\dots,k$ are not necessarily distinct.

If $A$ is real matrix with only real eigenvalues $\lm_1,\dots,\lm_k$, then the similarity matrix $P$ can be taken to be real.
\end{theorem}

\begin{proof}[\bf Proof]
We shall proceed to this conclusion in three steps:
\ben
\item [(i)] Observe that every complex matrix is similar to an upper triangular matrix whose eigenvalues appear on the main diagonal in a prescribed order; this is the Schur's unitary triangular
    theorem (Theorem \ref{thm:schur_unitary_triangularization} ).

\item [(ii)] Then we have that an upper triangular matrix can be transformed by similarity into a block diagonal matrix in which each individual diagonal block has all its diagonal entries equal;
    this is Theorem \ref{thm:every_complex_matrix_similar_to_diagonal_block_upper_triangular}.

\item [(iii)] Finally, we can show that an upper triangular matrix whose main diagonal entries are all equal is similar to a direct sum of Jordan blocks; this is Corollary
    \ref{cor:upper_triangular_jordan_canonical_form}.
\een

Also, if $A$ is real with only real eigenvalues, then the similarity matrix $P$ can be taken to be real. This is obvious by combining Theorem
\ref{thm:every_complex_matrix_similar_to_diagonal_block_upper_triangular} and Corollary \ref{cor:upper_triangular_jordan_canonical_form}.

Hence, we have proved everything except the uniqueness assertion.

If $A,B\in M_n(\F)$ are similar, then for any scalar $\lm \in \F$ and any exponent $m=1,2,\dots$, the matrices $\brb{A - \lm I}^m$ and $\brb{B - \lm I}^m$ are also similar. This is, if $B =
Q^{-1}AQ$ for nonsingular matrix $Q$,
\be
\brb{B - \lm I}^m = \brb{Q^{-1}AQ - \lm I}^m = \brb{Q^{-1}\brb{A - \lm I}Q}^m = Q^{-1}\brb{A - \lm I}^m Q.
\ee

In particular, $\brb{A - \lm I}^m$ and $\brb{B - \lm I}^m$ have the same rank by Proposition \ref{pro:equivalent_rank}. Thus, it suffices to show that the set of Jordan blocks (including repetitions)
lying on the diagonal of a Jordan matrix $J\in M_n(\F)$ is uniquely determined by the finitely many integer $\rank\brb{\brb{J-\lm I}^m}$ (equivalently, by $\rank\brb{\brb{A-\lm I}^m}$ as $A$ is already
known), $m = 1,2,\dots$, $\lm \in \sigma(J)$.

First, we consider a Jordan block $J_k(\lm) \in M_k(\F)$ of Definition \ref{def:jordan_block} for some $\lm \in \F$ and $m\geq 1$.

If $\lm \neq 0$, then $\rank\brb{J_k(\lm)^m} = \rank\brb{J_k(\lm)^{m+1}} = k$, so $\rank\brb{J_k(\lm)^m} - \rank\brb{J_k(\lm)^{m+1}} = 0$.

If $\lm = 0$ and $m\geq k$, then $\rank\brb{J_k(0)^m} = \rank\brb{J_k(0)^{m+1}} = 0$, so $\rank\brb{J_k(\lm)^m} - \rank\brb{J_k(\lm)^{m+1}} = 0$ again.

If $\lm = 0$ and $m\leq k-1$, then $\rank\brb{J_k(0)^m} - \rank\brb{J_k(0)^{m+1}} = 1$ as $J_k(0)^0 = I$.

Now let $J \in M_n(\F)$ be a Jordan matrix of Definition \ref{def:jordan_matrix}. Let $\lm\in \sigma(J)$. Define
\be
r_m(\lm) := \rank\brb{\brb{J-\lm I}^m},\qquad m=1,2,\dots
\ee
and set $r_0(\lm) = n$. By Proposition \ref{pro:block_diagonal_matrix_property}.(iii), it follows from the preceding analysis of the case of one block that the difference
\be
d_m(\lm) = r_{m-1}(\lm) - r_m(\lm)
\ee
is equal to the total number of Jordan blocks $J_k(\lm)$ in $J$ of sizes $k\geq m$ and $d_m(\lm) = 0$ for all $m\geq n$. Thus, the number of Jordan blocks in $J$ with exact size $k=m$ is therefore equal to
\be
d_m(\lm) - d_{m+1}(\lm) = r_{m-1}(\lm) - 2r_m(\lm) + r_{m+1}(\lm) ,\qquad m=1,2,\dots,n.
\ee

Hence, the set of Jordan blocks are uniquely determined by $\rank\brb{\brb{J-\lm I}^m}$, for $\lm\in \sigma(J)$ and $m=1,2,\dots,n$.
\end{proof}


\begin{corollary}
Let $A\in M_n(\C)$ be a given complex matrix, and let $\ve >0$ be given. Then there exists a nonsingular matrix $P=P(\ve)\in M_n(\C)$ such that
\be
A = P \bepm \ba{cc} J_{n_1}(\lm_1,\ve) & \\ & J_{n_2}(\lm_2,\ve) \ea & \text{\Large 0} \\ \text{\Large 0} &  \ba{cc} \ddots &  \\ & J_{n_k}(\lm_k,\ve) \ea \eepm P^{-1} ,\qquad (*)
\ee
with $n_1 + n_2 + \dots + n_k = n$ and
\be
J_m(\lm,\ve) = \bepm \ba{ccc} \lm & \ve & \\ & \lm & \ve \\ & & \lm \ea & \text{\Large 0} \\ \text{\Large 0} & \ba{ccc} \ddots & \ddots & \\ & \lm  & \ve  \\ & & \lm \ea\eepm.
\ee

If $A$ is real with real eigenvalues, then $P$ may be taken to be real.
\end{corollary}

\begin{remark}
This means that the off-diagonal entries can be chosen to be arbitrarily small and the form is very `close' to a diagonal matrix.
\end{remark}

\begin{proof}[\bf Proof]
First, by Theorem \ref{thm:jordan_canonical_form}, we find a nonsingular matrix $P_1\in M_n(\C)$ such that $P_1^{-1}AP_1$ is in Jordan canonical form (with a real $P_1$ if $A$ is real and has real eigenvalues). Then take
\be
D_\ve = \diag\brb{1,\ve,\ve^2,\dots,\ve^{n-1}}
\ee
and compute $X = D^{-1}_\ve\brb{P_1^{-1}AP_1}D_\ve = D^{-1}_\ve J D_\ve$. For the first Jordan block, we have
\be
\diag\brb{1,1/\ve,\dots,1/\ve^{n_1-1}} J_{n_1}(\lm_1)\diag\brb{1,\ve,\dots,\ve^{n_1-1}}.
\ee

Similarly, for the $i$th Jordan block, $i =1,\dots,k$,
\be
\diag\brb{\ve^{-\sum^{k-1}_{i=1} n_i},\ve^{-\sum^{k-1}_{i=1} n_i -1 },\dots,\ve^{-\sum^{k}_{i=1} n_i+1}}  J_{n_i}(\lm_i)\diag\brb{\ve^{\sum^{k-1}_{i=1} n_i},\ve^{\sum^{k-1}_{i=1} n_i +1 },\dots,\ve^{\sum^{k}_{i=1} n_i-1}}.
\ee
which is
\beast
& & \bepm \ve^{-\sum^{k-1}_{i=1} n_i} & & & \\ & \ve^{-\sum^{k-1}_{i=1} n_i -1 } & & \\ & & \ddots & \\ & & & \ve^{-\sum^{k}_{i=1} n_i+1} \eepm
\bepm \lm_i \ve^{\sum^{k-1}_{i=1} n_i} & \ve^{\sum^{k-1}_{i=1} n_i +1} & & \\ & \lm_i\ve^{\sum^{k-1}_{i=1} n_i +1 } & \ve^{\sum^{k-1}_{i=1} n_i + 2} & \\ & & \ddots & \\ & & & \lm_i\ve^{\sum^{k}_{i=1} n_i-1}\eepm
\\
& = &  \bepm \lm_i  & \ve & & \\ & \lm_i  & \ve & \\ & & \ddots & \ve \\ & & & \lm_i \eepm .
\eeast

Thus, the matrix $D^{-1}_\ve J D_\ve$ is of the form ($*$). Thus, $P = P(\ve) = P_1D_\ve$ meets the requirements.
\end{proof}

%\subsection{Applications of Jordan canonical form}

\subsection{Minimal polynomial and companion matrix}

The vital role of the characteristic polynomial has already been observed, but there are other polynomials associated with a square matrix. One of these is the minimal polynomial.

Cayley-Hamilton theorem (Theorem \ref{thm:cayley_hamilton}) guarantees that for each $A\in M_n(\C)$, the characteristic polynomial at $A$, $p_A(A) = 0$. Now we want to find a polynomial of smaller
degree to annihilates $A$, which is actually the minimal polynomial of $A$.

\begin{theorem}[minimal polynomial\index{minimal polynomial!matrix}]\label{thm:minimal_polynomial_existence_uniqueness}
Let $A\in M_n(\C)$ be given. There exists a unique monic polynomial $q_A(t)$ of minimum degree that annihilates $A$ (i.e., $q_A(A) = 0$). The degree of this polynomial is at most $n$.

If $p(t)$ is any polynomial such that $p(A) = 0$, then $q_A(t)$ divides $p(t)$.

$q_A(t)$ is called minimal polynomial of matrix $A$.
\end{theorem}

\begin{proof}[\bf Proof]
The characteristic polynomial is an example of a polynomial of degree $n$ that annihilates $A$, so there is a minimum positive integer $m\leq n$ such that there exists a monic polynomial $q(t)$ of
degree $m$ with $q(A) = 0$.

If there is a polynomial $p(t)$ annihilates $A$, and if $q(t)$ is a monic polynomial of minimum degree that annihilates $A$, then the degree of $q(t)$ must be less than or equal to the degree of
$p(t)$. By Euclidean algorithm for polynomial (Proposition \ref{pro:euclidean_algorithm_polynomial}), there exists a polynomial $h(t)$ and a polynomial $r(t)$ of degree less than that of $q(t)$ such
that
\be
p(t) = q(t)h(t) + r(t).
\ee

But
\be
0 = p(A) = q(A)h(A) + r(A) = 0 h(A) + r(A) \ \ra \ r(A) = 0.
\ee

If $r(t)\neq 0$, we could normalize it and obtain a monic polynomial of degree less than that of $q(t)$ that annihilates $A$. Since this would contradict the minimal property of $q(t)$, we conclude
that $r(t) \equiv 0$, and hence $q(t)$ divides $p(t)$ with quotient $h(t)$.

If there are two monic polynomials of minimum degree that annihilates $A$, this argument shows that each divides the other;since the degrees are the same, one must be a scalar multiple of the other.
But since both are monic, the scalar factor must be $+1$ and they are identical.
\end{proof}

\begin{corollary}\label{cor:similar_matrices_have_same_minimal_polynomial}
Similar matrices have the same minimal polynomial. That is, if $A\sim B$, $q_A(t) = q_B(t)$.
\end{corollary}

\begin{proof}[\bf Proof]%\footnote{proof needed.}
If $A,B,P\in M_n(\C)$ and $A$ and $B$ are similar with $A = PBP^{-1}$ for nonsingular $P$, then the minimal polynomial of $B$,
\be
q_B(A) = q_B\brb{PBP^{-1}} = P q_B(B)P^{-1} = 0
\ee
by Cayley-Hamilton theorem (Theorem \ref{thm:cayley_hamilton}). So the degree of $q_B(t)$ is not less than the degree of $q_A(t)$. Similarly, we have that the degree of $q_A(t)$ is not less than the degree of $q_B(t)$.

Thus, these two monic minimal polynomials have the same degree and both annihilate $A$, so they must be identical by Theorem \ref{thm:minimal_polynomial_existence_uniqueness}.
\end{proof}



\begin{corollary}\label{cor:characteristic_polynomial_minimal_polynomial_same_roots}
For every $A\in M_n(\C)$, the minimal polynomial $q_A(t)$ divides the characteristic polynomial $p_A(t)$.

Moreover, $q_A(\lm) = 0$ if and only if $p_A(\lm) = 0$. That is, every root of $p_A(t) = 0$ (every eigenvalue of $A$) is a root of $q_A(t) = 0$.
\end{corollary}



\begin{proof}[\bf Proof]
Since $p_A(A) = 0$ by Cayley-Hamilton theorem (Theorem \ref{thm:cayley_hamilton}), the fact that there is a polynomial $h(t)$ such that $p_A(t) = h(t)q_A(t)$ follows from Theorem \ref{thm:minimal_polynomial_existence_uniqueness}.

This factorization makes it clear that every root of $q_A(t) = 0$ is a root of $p_A(t) = 0$, and hence every root of $q_A(t) = 0$ is an eigenvalue of $A$ .

If $\lm$ is an eigenvalue of $A$ (such that $p_A(\lm) = 0$), and $x\neq 0$ is an associated eigenvector, then $Ax = \lm x$ and
\be
0 = q_A(A) x = q_A(\lm) x
\ee
by Proposition \ref{pro:polynomial_matrix_eigenvalue_eigenvector}. Since $x\neq 0$, we have that $q_A(\lm) = 0$ and thus every root of $p_A(t) = 0$ is a root of $q_A(t) = 0$.
\end{proof}

\begin{remark}
The above theorem and corollaries show that if the characteristic polynomial $p_A(t)$ has been completely factors as
\be
p_A(t) = \prod^m_{i=1}\brb{t-\lm_i}^{s_i},\qquad 1\leq s_i\leq n,\quad s_1 + s_2 + \dots + s_m = n\qquad (*)
\ee
with $\lm_1,\lm_2,\dots,\lm_m$ distinct, then the minimal polynomial $q_A(t)$ must have the form
\be
q_A(t) = \prod^m_{i=1}\brb{t-\lm_i}^{r_i},\qquad 1\leq r_i\leq s_i. \qquad (\dag)
\ee

In principle, this gives an algorithm for finding the minimal polynomial of a given matrix $A$:
\ben
\item [(i)] First compute the eigenvalues of $A$, together with their algebraic multiplicities, perhaps by finding the characteristic polynomial and factoring it completely. By some means, determine the factorization ($*$).

\item [(ii)] There are finitely many polynomials of the form of the product in ($\dag$). Starting with the product in which all $r_i = 1$, determine by explicit calculation the one of minimal degree that annihilates $A$. This will be the minimal polynomial.
\een

Numerically, this is not a good algorithm if it involves factoring the characteristic polynomial of a large matrix, but it can be very effective for handy calculations involving small matrices of simple form.

Another approach to computing the minimal polynomial that does not involve knowing either the characteristic polynomial or the eigenvalues is outlined in \cite{Horn_Johnson_1990}.$P_{148}\footnote{details needed.}$.
\end{remark}

The following theorem reveals an intimate connection between the Jordan canonical form of $A$ and the minimal polynomial of $A$.

\begin{theorem}\label{thm:jordan_block_order_minimal_polynomial}
Let $A\in M_n(\C)$ be a given matrix whose distinct eigenvalues are $\lm_1,\lm_2,\dots,\lm_m$. The minimal polynomial of $A$ is
\be
q_A(t) = \prod^m_{i=1}\brb{t-\lm_i}^{r_i},
\ee
where $r_i$ is the order of the largest Jordan block of $A$ corresponding to the eigenvalue $\lm_i$.
\end{theorem}

\begin{proof}[\bf Proof]%\footnote{proof needed.}
Suppose $A = PJP^{-1}$ is the Jordan canonical form of $A$ (by Theorem \ref{thm:jordan_canonical_form}) and suppose first that
\be
J = \bepm \ba{ccc} \lm & 1 & \\ & \lm & 1 \\ & & \lm \ea & \text{\Large 0} \\ \text{\Large 0} & \ba{ccc} \ddots & \ddots & \\ & \lm  & 1  \\ & & \lm \ea\eepm \in M_n(\C).
\ee
is the only a single Jordan block. The characteristic polynomial of $J$ is $(t-\lm)^n$ and thus the minimal polynomial of $J$ is of the form $(t-\lm)^m$ for some $m\leq n$ by Corollary \ref{cor:characteristic_polynomial_minimal_polynomial_same_roots}.
Since $(J-\lm I)^k \neq 0$ if $k< n$, the minimal polynomial of J (and of $A$) is also $(t-\lm)^n$.

If
\be
J = \bepm J_{n_1}(\lm) & 0 \\ 0 & J_{n_2}(\lm)\eepm \in M_n(\C)
\ee
with $n_1\geq n_2$, then the characteristic polynomial of $J$ is still $(t-\lm)^n$, but now $(J-\lm I)^{n_1} = 0$ and no lower power vanishes. The minimal polynomial is therefore $(t-\lm)^{n_1}$.

If there are more blocks, the result is the same: The minimal polynomial of $J$ is $(t-\lm)^r$, where $r$ is the order of the largest Jordan block corresponding to $\lm$.

If $J$ is a general Jordan matrix, the minimal polynomial of $J$ must contain a factor $(t-\lm_i)^{r_i}$ for each distinct eigenvalue $\lm_i$, and $r_i$ must be the order of the largest Jordan block
corresponding to $\lm_i$. For each $i$, $\brb{J - \lm_i}^{r_i}$ will only annihilate the Jordan blocks corresponding to $\lm_i$. No smaller power will annihilate all the Jordan blocks corresponding to
$\lm_i$, and no greater power is needed.

Thus, we need $q(t) = \prod^m_{i=1} (t-\lm_i)^{r_i}$ to annihilate all the Jordan blocks for $m$ distinct eigenvalues and this is the minimal polynomial of $J$.

Then by Corollary \ref{cor:similar_matrices_have_same_minimal_polynomial}, similar matrices have the same minimal polynomial, i.e., $q_J(t) = q_A(t)$.
\end{proof}


\begin{remark}
In practice, this result is not very helpful in computing the minimal polynomial since it is usually harder to determine the Jordan canonical form of a matrix than it is to determine its minimal polynomial.

After all, if only the eigenvalues of a matrix are known, its minimal polynomial can be determined by simple trial and error.
\end{remark}


\begin{corollary}\label{cor:distinct_eigenvalue_minimal_polynomial}
Let $A\in M_n(\C)$ be a given matrix whose distinct eigenvalues are $\lm_1,\lm_2,\dots,\lm_m$. Then $A$ is diagonalizable if and only if $q(A) = 0$, where
\be
q(t) = (t-\lm_1)(t-\lm_2)\dots (t-\lm_m).
\ee
\end{corollary}

\begin{proof}[\bf Proof]
Since a matrix is diagonalizable if and only if all its Jordan blocks have order 1 (Jordan block with order $k\geq 2$ is not diagonalizable by Remark \ref{rem:jordan_matrix_diagonalizable}.), a
necessary and sufficient condition for diagonalizability it that all $r_i =1$ in Theorem \ref{thm:jordan_block_order_minimal_polynomial}.
\end{proof}

\begin{remark}
Note that this corollary gives the necessary and sufficient condition for diagonalizability. Recall Theorem \ref{thm:distinct_eigenvalues_implies_diagonalizable}, which gives the sufficient
condition for diagonalizability, we can see that distinct eigenvalues case is the special case of the statement in Corollary \ref{cor:distinct_eigenvalue_minimal_polynomial}.
\end{remark}

It is sometimes useful to have this result formulated in several equivalent ways (without proof):

\begin{corollary}
Let $A\in M_n(\C)$ be given. Each of the following is necessary and sufficient condition for $A$ to be diagonalizable:
\ben
\item [(i)] The minimal polynomial $q_A(t)$ has distinct linear factors, $(t-\lm_i)$.
\item [(ii)] Every root of $q_A(t) = 0$ has multiplicity 1.
\item [(iii)] For all $t$ such that $q_A(t) = 0$, the derivative $q_A'(t) \neq 0$.
\een
\end{corollary}

We have been considering the problem of finding, for a given matrix $A\in M_n(\C)$, a monic polynomial of minimum degree (minimal polynomial) that annihilates $A$. But what about the converse? Given
a monic polynomial, is there a matrix $A$ for which $p(t)$ is the minimal polynomial? If so, the size of $A$ must be at least $n\times n$. Actually, it is not hard to find such a matrix $A\in
M_n(\C)$.

\begin{definition}[companion matrix\index{companion matrix}]\label{def:companion_matrix}
Given monic polynomial
\be
p(t) = t^n + a_{n-1}t^{n-1} + a_{n-2}t^{n-2} + \dots + a_1 t + a_0,
\ee

The companion matrix of the polynomial $p(t)$ is defined by
\be
A = \bepm 0 & & & 0 & -a_0 \\ 1 & 0 & & & -a_1 \\ & 1 & \ddots & & \vdots \\ & & \ddots & 0 & -a_{n-2} \\ 0 & & & 1 & -a_{n-1} \eepm \in M_n(\C).
\ee
\end{definition}

\begin{theorem}\label{thm:characteristic_minimal_polynomial_identical_for_companion_matrix}
Every monic polynomial is both the minimal polynomial and the characteristic polynomial of its companion matrix.
\end{theorem}

\begin{proof}[\bf Proof]
Let $e_i$ be the $i$th of natural basis of $\C^n$, then
\beast
I e_1 & = &  e_1 = A^0 e_1 \\
A e_1 & = & e_2 = A^1 e_1 \\
A e_2 & = & e_3 = A^2 e_1 \\
& \vdots & \\
A e_{n-1} & = & e_n = A^{n-1}e_1 \\
A e_n & = & -a_{n-1}e_n - a_{n-2}e_{n-1} - \dots -a_1 e_2 - a_0 e_1 \\
& = & -a_{n-1}A^{n-1}e_1 - a_{n-2}A^{n-2}e_1 - \dots -a_1 A e_1 - a_0 e_1 = A^n e_1.
\eeast

But we know that $p(A)e_1 = A^n e_1 + a_{n-1}A^{n-1}e_1 + a_{n-2}A^{n-2}e_1 + \dots + a_1 A e_1 + a_0 e_1$, so $p(A)e_1 = 0$. Furthermore, for any $k=1,2,\dots,n$,
\be
p(A)e_k = p(A)A^{k-1} e_1 = A^{k-1}p(A)e_1 = 0 \ \ra \ p(A) I = 0 \ \ra \ p(A) = 0.
\ee
by combining all the columns. Thus, $p(t)$ is a monic polynomial of degree $n$ that annihilates $A$.

%Thus, \be p(A) e_1 = \brb{a_0 e_1 + a_1 A e_1 + a_2 A^2 e_1 + \dots + a_{n-1}A^{n-1}e_1} + A^n e_1 = \ee

If there were a polynomial
\be
q(t) = t^m + b_{m-1} t^{m-1} + \dots + b_1t + b_0
\ee
of lower degree $m<n$ that annihilates $A$, then
\be
0 = q(A)e_1 = A^m e_1 + b_{m-1} A^{m-1} e_1 + \dots + b_1 A e_1 + b_0 e_1 = e_{m+1} + b_{m-1}e_m + \dots + b_1 e_2 + b_0 e_1
\ee
which would imply that the basis vector $e_{m+1}$ is linearly dependent on the basis vectors $e_1,\dots,e_m$. Since this is impossible, we conclude that $p(t)$ is the unique monic polynomial of minimum order that annihilates $A$.

Moreover, since $p(t)$ has degree $n$, $A\in M_n(\C)$, and the characteristic polynomial $p_A(t)$ is a monic polynomial of degree $n$ that also annihilates $A$, $p(t)$ must be the characteristic
polynomial of $A$ (since minimal polynomial divides characteristic polynomial by Corollary \ref{cor:characteristic_polynomial_minimal_polynomial_same_roots}).
\end{proof}

\begin{theorem}
A matrix $A\in M_n(\C)$ is similar to the companion matrix $A'$ of its characteristic polynomial $p_A(t)$ if and only if the minimal and characteristic polynomial of $A$ are identical.
\end{theorem}

\begin{proof}[\bf Proof]
($\ra$). By Theorem \ref{thm:characteristic_minimal_polynomial_identical_for_companion_matrix}, we let monic polynomial $p_{A'}(t)$ be both the minimal and characteristic polynomial of its companion
matrix $A'$, i.e., $p_{A'}(t) = q_{A'}(t)$. Since $A$ is similar to $A'$, they have the same characteristic and minimal polynomials by Theorem
\ref{thm:similar_matrices_have_same_characteristic_polynomial} and Corollary \ref{cor:similar_matrices_have_same_minimal_polynomial}. Therefore, $p_A(t) = q_A(t)$ as required.

($\la$). If for matrix $A$, its characteristic polynomial $p_A(t)$ and minimal polynomial $q_A(t)$ are identical, then the Jordan canonical form $J$ of $A$ must contain exactly one Jordan block for each
distinct eigenvalue (by Theorem \ref{thm:jordan_block_order_minimal_polynomial}). The size of each Jordan block is equal to the multiplicity of the corresponding eigenvalue as a zero of
characteristic (minimal) polynomial of $A$. That is, the characteristic (minimal) polynomial $J$ ($A$) has the form
\be
q_J(t) = p_J(t) = p_A(t) = \prod^m_{i=1}(t-\lm_i)^{r_i}.
\ee

where $\lm_1,\dots,\lm_m$ are distinct eigenvalues and $r_i$, $i=1,\dots,m$ are the corresponding sizes of Jordan blocks. But the Jordan canonical form $J'$ of the companion matrix $A'$ has the same
characteristic and minimal polynomials by Theorem \ref{thm:similar_matrices_have_same_characteristic_polynomial} and Corollary \ref{cor:similar_matrices_have_same_minimal_polynomial}, i.e.,
\be
p_{J'}(t) = p_{A'}(t),\qquad q_{J'}(t) = q_{A'}(t).
\ee

By Theorem \ref{thm:characteristic_minimal_polynomial_identical_for_companion_matrix}, $p_A(t)$ is both characteristic and minimal polynomials of $A'$. Therefore,
\be
p_{J'}(t) = q_{J'}(t) = p_{A'}(t) = q_{A'}(t) = p_A(t) = \prod^m_{i=1}(t-\lm_i)^{r_i}.
\ee

By Theorem \ref{thm:jordan_block_order_minimal_polynomial}, $J$ has the same Jordan block structure with $J'$. Thus, $A$ and $A'$ are similar.
\end{proof}


\subsection{Triangular factorizations}

If a linear system $Ax =b$ has a nonsingular triangular coefficient matrix $A\in M_n$, computation of the unique solution $x$ is remarkably easy.
If, for example, $A$ is upper triangular
\be
A = \bepm a_{11} & \dots & \dots & a_{1n} \\ & a_{22} & & \vdots \\ & & \ddots & \\ 0 &  & & a_{nn} \eepm,
\ee
then $\det A = a_{11}a_{22}\dots a_{nn} \neq 0$ and back substitution is used: $a_{nn}x_n = b_n$ determines $x_n$; $a_{n-1,n-1}x_{n-1} + a_{n-1,n}x_n = b_{n-1}$
is then one equation in one unknown, which determines $x_{n-1}$; and, in general, each of the sequence of equaitons
\be
\sum^n_{j=i} a_{ij}x_j = b_i,\qquad i = n,n-1,\dots, 2,1
\ee
is one equation in one known (once $x_{i+1}, \dots, x_n$ have been determined), which determines $x_i$.




\section{Positive Definite and Positive Semi-definite Matrices}

\subsection{Definitions and basic properties}

\begin{definition}[positive definite matrix\index{positive definite matrix}, positive semi-definite matrix\index{positive semi-definite matrix}]\label{def:positive_definite_matrix}%$A \succ 0$
A Hermitian matrix $A\in M_n(\C)$ is said to be positive definite, denoted as $A > 0$,  if
\be
x^*A x >0,\qquad \forall\ x\in \C^n,\ x \neq 0. \qquad\qquad (*)
\ee

If the strict inequality is weakened to $x^*A x \geq 0$, then $A$ is said to be positive semi-definite (or non-negative definite) and denoted as $A \geq 0$. %$A \succcurlyeq 0$.
\end{definition}

\begin{remark}
\ben
\item [(i)] Since $A$ is Hermitian, the left-hand side of ($*$) is always a real number.
\item [(ii)] Of course, if $A$ is positive definite, then it is also positive semi-definite.
\item [(iii)] When $n=1$, positive definite matrices are positive real numbers and positive semi-definite matrices are non-negative real numbers.
\een
\end{remark}

\begin{example}
Identity matrix $I$ is positive definite.
\end{example}

\begin{proposition}\label{pro:positive_definite_principal_submatrix_is_positive_definite}
Any principal submatrix (see Definition \ref{def:principal_submatrix}) of a positive definite matrix is positive definite.

Similarly, any principal submatrix of a positive semi-definite matrix is positive semi-definite.
\end{proposition}

\begin{remark}
It is easy to see that the diagonal entries of a positive definite matrix are positive real numbers by the above proposition.
\end{remark}

\begin{proof}[\bf Proof]
Let $\alpha$ be a proper subset of $\bra{1,2,\dots,n}$ and $A_{\alpha}$ be the corresponding principal submatrix.

Let $x\in\C^n$ be a non-zero vector with arbitrary entries in the components indicated by $\alpha$ and zero entries elsewhere.Let $x_\alpha$ denote the vector obtained from $x$ by deleting the
(zero) components complementary to $\alpha$, and observe that
\be
x_\alpha^* A_\alpha x_\alpha = x^*A x >0.
\ee

Since $x_\alpha \neq 0$ is arbitrary, the means that $A_\alpha$ is positive definite.
\end{proof}

\begin{proposition}
The sum of any two positive definite matrices of the same size is positive definite.

More generally, any positive linear combination of positive definite matrices is positive definite and any
non-negative linear combination of positive semi-definite matrices is positive semi-definite.
\end{proposition}

\begin{remark}
This means that the set of positive definite matrices is a positive cone in the vector space of all matrices.
\end{remark}

\begin{proof}[\bf Proof]
Let $A$ and $B$ be positive semi-definite and $a,b\geq 0$, and observe that
\be
x^*(aA+bB)x = ax^*A x + bx^*Bx \geq 0,\qquad \forall x\in \C^n.
\ee

The case of more than two summands is treated in the same way. If the coefficients are positive, $A$ and $B$ are positive definite, and the vector $x$ is non-zero, then every term in the sum is
positive, so a positive linear combination of positive definite matrices is positive definite.
\end{proof}

\begin{proposition}\label{pro:positive_definite_matrix_positive_eigenvalue}
Each eigenvalue of a positive definite matrix is a positive real number. %Each eigenvalue of a positive semi-definite matrix is a non-negative real number.

It is similar for positive semi-definite matrices as they have non-negative eigenvalues.
\end{proposition}

\begin{proof}[\bf Proof]
Let $A$ be positive definite and $\lm \in \sigma(A)$. Let $x$ be an eigenvector of $A$ associated with $\lm$ (such that $x^*A x >0$) and get
\be
x^*A x = x^*\lm x = \lm x^*x \ \ra \ \lm = \brb{x^*A x}/\brb{x^*x} > 0,
\ee
which is a ratio of two positive numbers as $x\neq 0$.
\end{proof}



\begin{proposition}
The trace, determinant, and all principal submatrices' determiants of a positive definite matrix are positive.
\end{proposition}

\begin{proof}[\bf Proof]
The trace and determinant are just the sum and product of the eigenvalues by Theorem \ref{thm:elementary_symmetric_function_sum_of_determinant_principal_submatrix_equivalent}. The rest follows from Proposition \ref{pro:positive_definite_principal_submatrix_is_positive_definite}
\end{proof}


\begin{proposition}\label{pro:product_rank_positive_definite}
Let $A\in M_m(\C)$ be positive definite. If $C\in M_{m,n}(\C)$, then $C^*AC$ is positive semi-definite.

Furthermore, $\rank\brb{C^*AC} = \rank(C)$, so that $C^*AC$ is positive definite if and only if $C$ has rank $n$.
\end{proposition}


\begin{proof}[\bf Proof]    %\footnote{proof needed.}
First note that $C^*AC$ is Hermitian. For any $x\in \C^n$, we have
\be
x^*C^*A C x = y^* A y \geq 0,
\ee
where $y = Cx$ and the inequality follows from the positive definiteness of $A$. Thus, $C^*AC$ is positive semi-definite.

Furthermore, note that $x^*C^*ACx >0$ if and only if $Cx \neq 0$ because $A$ is positive definite. The statement about rank (and thus about the positive definiteness of $C^*AC$) would follow if we knew that
\be
C^*AC x = 0 \ \lra \ Cx = 0
\ee
because this would mean that $C^*AC$ and $C$ have the same null space (and hence they also have the same rank by rank-nullity theorem\footnote{matrix version needed.} (Theorem \ref{thm:rank_nullity})).

If $Cx = 0$, then obviously $C^*ACx = 0$. Conversely, if $C^*ACx = 0$, then $x^*C^*AC x = 0$ and we conclude that $Cx = 0$ by using the positive definiteness of $A$.
\end{proof}


\subsection{Characterizations}

\begin{theorem}\label{thm:positive_definite_matrix_iff_positive_eigenvalue}
A Hermitian matrix $A\in M_n(\C)$ is positive semi-definite if and only if all of its eigenvalues are non-negative.

It is positive definite if and only if all of its eigenvalues are positive.
\end{theorem}

\begin{proof}[\bf Proof]
If each eigenvalue of $A$ is positive, then for any non-zero $x\in \C^n$, we have (by spectral theorem for Hermitian matrices (Theorem \ref{thm:spectral_hermitian_matrices}))
\be
x^*A x = x^*U^* DU x = y^* D y = \sum^n_{i=1}d_{i}\abs{y_i}^2 >0
\ee
where $D = \diag\brb{d_1,\dots,d_n}$ is the diagonal matrix of eigenvalues of $A$ (by Corollary \ref{cor:similarity_same_eigenvalues}) and $y = Ux$ with unitary $U$.

The reverse implication is contained in Proposition \ref{pro:positive_definite_matrix_positive_eigenvalue} and the positive semi-definite case is similar.
\end{proof}

\begin{corollary}\label{cor:power_of_positive_semidefinite_matrix}
If $A\in M_n(\C)$ is positive semi-definite, then so are all the powers $A^k$, $k=1,2,\dots$.

Similarly, the conclusion holds for positive definite matrix $A$, respectively,
\end{corollary}

\begin{proof}[\bf Proof]
Since $A$ is positive semi-definite, all the eigenvalues of $A$ are non-negative by Theorem \ref{thm:positive_definite_matrix_iff_positive_eigenvalue}. Since $A$ is Hermitian, there exists a unitary matrix $U$ such that $U^*AU = \Lambda$ where $\Lambda$ is the diagonal matrix $\diag\brb{\lm_1,\dots,\lm_n}$. Thus,
\be
U^*A^k U = \underbrace{U^* A U U^* A U \dots U^*A U}_{k\text{-fold}} = \Lambda^k = \diag\brb{\lm_1^k,\dots,\lm_n^k}.
\ee

Thus, all the eigenvalues of $A^k$ are non-negative. Therefore, we can say $A^k$ is positive semi-definite by Theorem \ref{thm:positive_definite_matrix_iff_positive_eigenvalue}.
\end{proof}



\begin{proposition}\label{pro:positive_definite_matrix_inverse}
Let $A\in M_n(\C)$ be a positive definite. Then $A^{-1}$ is also a positive definite matrix.
\end{proposition}

\begin{proof}[\bf Proof]
Since all eigenvalues $\lm_1,\dots,\lm_n$ of $A$ are positive by Theorem \ref{thm:positive_definite_matrix_iff_positive_eigenvalue}, we have that for unitary matrix $U$ and $\Lambda := \diag\brb{\lm_1,\dots,\lm_n}$,
\be
\det A = \det\brb{U\Lambda U^*} = \det \Lambda \det \brb{U^*U} = \det\Lambda >0.
\ee

Thus, $A^{-1}$ is well-defined. We can then define the Hermitian matrix $A^{-1} := U\Lambda^{-1}U^*$ such that
\be
A^{-1}A =  U\Lambda^{-1}U^* U\Lambda U^* = UU^* = I = UU^* = U\Lambda\Lambda^{-1}U^* = U\Lambda U^*U\Lambda^{-1}U = AA^{-1}.
\ee

For any $x\neq 0$, we have that
\beast
x^* A x & = & x^* U\Lambda{-1}U^*x = x^*U\Lambda^{-1}\Lambda \Lambda^{-1} U^*x \\
& = & x^*U\Lambda^{-1} U^* U\Lambda U^*U \Lambda^{-1} U^*x = \brb{U \Lambda^{-1} U^*x}^* A \brb{U \Lambda^{-1} U^*x} >0
\eeast
since $A$ is positive definite and $U \Lambda^{-1} U^*x \neq 0$. Thus, $A^{-1}$ is positive definite.
\end{proof}





%%%%%%%%%%%


\begin{corollary}
If $A = \brb{a_{ij}} \in M_n(\C)$ is Hermitian and strictly diagonally dominant\footnote{definition needed.} and if $a_{ii} >0$ for all $i=1,2,\dots,n$, then $A$ is positive definite.
\end{corollary}

\begin{proof}[\bf Proof]
\footnote{proof needed.}
\end{proof}


\begin{corollary}
Let $A\in M_n(\C)$ be Hermitian, and let
\be
p_A(t) = t^n + a_{n-1}t^{n-1} + \dots + a_{n-m}t^{n-m}
\ee
be the characteristic polynomial of $A$. Suppose that $0\leq m\leq n$ and $a_{n-m} \neq 0$.

Then $A$ is positive semi-definite if and only if $a_k \neq 0$ for all $n-m\leq k\leq n$ and $a_k a_{k+1} <0$ for $k=n-m,\dots,n-1$.
\end{corollary}

\begin{proof}[\bf Proof]
\footnote{proof needed.}
\end{proof}


%%%%%%%%




\subsection{$k$th root of positive semi-definite matrix}

Every positive real number has a unique positive $k$th root for all $k=1,2,\dots$. A similar result holds for positive definite matrices.

\begin{theorem}[existence and uniqueness of $k$th root matrix of positive semi-definite matrix]\label{thm:existence_uniqueness_kth_root_matrix_of_positive_semi-definite_matrix}
Let $A\in M_n(\C)$ be positive semi-definite and let $k\geq 1$ be a given integer. Then there exists a unique positive semi-definite Hermitian matrix $B$ such that $B^k = A$, i.e., $B$ is the $k$th
root matrix of $A$. We also have
\ben
\item [(i)] $BA = AB$ and there is a polynomial $p(t)$ such that $B = p(A)$.
\item [(ii)] $\rank(B) = \rank(A)$, so $B$ is positive definite if $A$ is.
\item [(iii)] $B$ is real if $A$ is real.
\een
\end{theorem}

\begin{proof}[\bf Proof]
We know that the Hermitian matrix $A$ can be unitarily diagonalizable (by spectral theorem for Hermitian matrices (Theorem \ref{thm:spectral_hermitian_matrices})) as $A = U\Lambda U^*$ with $\Lambda
= \diag\brb{\lm_1,\dots,\lm_n}$ and all $\lm_i \geq 0$ as $A$ is positive semi-definite.

We define $B = U\Lambda^{1/k} U^*$ where $\Lambda^{1/k} := \diag\brb{\lm_1^{1/k},\dots,\lm_n^{1/k}}$, and the unique non-negative $k$th root is taken in each case.

Clearly, $B^k = U\Lambda^{1/k}U^* \dots U\Lambda^{1/k}U^* = U\Lambda U^* = A$ and $B$ is Hermitian and positive semi-definite, i.e.,
\be
B^* = \brb{U\Lambda^{1/k} U^*}^* = U\Lambda^{1/k} U^* = B\qquad \text{(by Proposition \ref{pro:matrix_multiple_hermitian})}
\ee
and
\be
x^* B x = x^* U\Lambda^{1/k}U^* x = \brb{\Lambda^{1/(2k)}U^*x}^* \brb{\Lambda^{1/(2k)}U^*x} \geq 0,\qquad \forall x\neq 0.
\ee

Also,
\be
AB = U\Lambda U^* U\Lambda^{1/k} U^* = U\Lambda^{1/k}\Lambda U^* = U\Lambda^{1/k} U^*U\Lambda U^* = BA.
\ee

The rank of $B$ is just the number of non-zero $\lm_i$ terms, which is also the rank of $A$ by Proposition \ref{pro:rank_equalities}.(ii). Also, if $A$ is positive definite and thus, all its
eigenvalues $\lm_i>0$ by Theorem \ref{thm:positive_definite_matrix_iff_positive_eigenvalue}, so all eigenvalues of $B$, $\lm_i^{1/k} >0$ and thus $B$ is positive definite by Theorem
\ref{thm:positive_definite_matrix_iff_positive_eigenvalue}.

If $A$ is real and positive definite, we know that $U$ can be chosen to be a real orthogonal matrix (by Theorem \ref{thm:spectral_hermitian_matrices}). So it is clear that $B$ can be chosen to be
real in case.

It remains only to consider the question of uniqueness of $B$ and to find a polynomial such that $p(A) = B$. Recalling Lagrange interpolation polynomial (Definition
\ref{def:lagrange_interpolation_polynomial}), we have
\be
L(t,X,Y) = \sum^m_{i=1} \brb{y_i\ \frac{\prod^m_{j\neq i}\brb{t - x_j}}{\prod^m_{j\neq i}\brb{x_i - x_j}}}
\ee
where $X = \brb{x_1,\dots,x_m}$ and $Y= \brb{x_1^{1/k},\dots,x_m^{1/k}}$ for $m$ distinct non-zero eigenvalues $x_1,\dots,x_m$ among all $n$ eigenvalues (as $m\leq n$). Thus, we define for any $i$, $i=1,\dots,m$,
\be
p_i(A) := \frac{\prod^m_{j\neq i}\brb{A - x_jI}}{\prod^m_{j\neq i}\brb{x_i - x_j}} = \brb{\prod^m_{j\neq i}\brb{x_i - x_j}}^{-1}  \prod^m_{j\neq i}\brb{A - x_jI}
\ee

Therefore,
\beast
p_i(\Lambda) & = & \brb{\prod^m_{j\neq i}\brb{x_i - x_j}}^{-1}  \prod^m_{j\neq i}\brb{\Lambda - x_jI} \\
& = &  \brb{\prod^m_{j\neq i}\brb{x_i - x_j}}^{-1} \bepm 0 & 0 & \dots & 0 \\ 0 & \prod^m_{i\neq j}\brb{x_i-x_j} & & \\ & & \ddots & \\ 0 & 0 & \dots & 0 \eepm
=  \bepm 0 & 0 & \dots & 0 \\ 0 & 1 & & \\ & & \ddots & \\ 0 & 0 & \dots & 0 \eepm
\eeast
with entries 1 in the diagonal if $\lm_{jj} = x_i$ for $j=1,\dots,n$. Thus, we can have that $L(t,X,Y) = p(t)$ and
\be
p(\Lambda) =  \sum^m_{i=1} \brb{y_i p_i(\Lambda)} = \sum^m_{i=1} \brb{x_i^{1/k} p_i(\Lambda)} = \Lambda^{1/k}.
\ee

Thus, by property of unitary matrix,
\be
p(A) = p\brb{U\Lambda U^*} = Up(\Lambda) U^* = U\Lambda^{1/k} U^* = B.
\ee

If $C$ is any positive semi-definite matrix such that $C^k = A$, we have
\be
B = p(A) = p\brb{C^k}  \ \ra \ CB = Cp\brb{C^k} = p\brb{C^k}C = BC.
\ee%{thm:commute_iff_simultaneously diagonalizable}

Since $B$ and $C$ are commuting Hermitian matrices (thus diagonalizable), they are simultaneously diagonalizable (by Theorem \ref{thm:diagonalizable_matrices_commute_iff_simultaneously diagonalizable}). That is, there is
some unitary matrix $V$ and diagonal matrices $\Lambda_1$ and $\Lambda_2$ with non-negative diagonal entries such that $B = V\Lambda_1 V^*$ and $C = V\Lambda_2 V^*$.

Then from the fact that $B^k = A = C^k$ we deduce that $\Lambda_1^k = \Lambda_2^k$. But since the non-negative $k$th root of a non-negative number is unique, we conclude that $\Lambda_1 =
\brb{\Lambda_1^k}^{1/k} = \brb{\Lambda_2^k}^{1/k} = \Lambda_2$ and $B =C$.
\end{proof}

\begin{example}%Theorem \ref{thm:positive_definite_iff_leading_principal_submatrices_determinants}.
Let $A = \bepm 5 & 3 \\ 3 & 2 \eepm$. It is easy to check that $A$ is positive definite by Theorem \ref{thm:positive_definite_matrix_iff_positive_eigenvalue}. Thus, we can have
unique positive definite matrix $B = A^{1/2}$. The eigenvalues of $A$ are given by
\be
\det \brb{A - \lm I} = 0 \ \ra\ \lm^2  - 7\lm + 1 = 0 \ \ra \ \lm = \frac 12 \brb{7 \pm 3\sqrt{5}}
\ee
and their roots are $\frac 12 \brb{3\pm \sqrt{5}}$. So we can use Lagrange interpolation polynomial to evaluate $B$,
\be
p(t) = \frac 12 \brb{3 + \sqrt{5}} \frac{t-\frac 12 \brb{7 - 3\sqrt{5}}}{3\sqrt{5}}  - \frac 12 \brb{3 - \sqrt{5}} \frac{t-\frac 12 \brb{7 + 3\sqrt{5}}}{3\sqrt{5}} = \frac 13 (t + 1).
\ee

Thus,
\be
B = p(A) = \frac 13 \brb{A + I} = \frac 13 \bepm 6 & 3 \\ 3 & 3 \eepm = \bepm 2 & 1 \\ 1 & 1 \eepm.
\ee

Also, it is easy to check that
\be
B^2 = \bepm 2 & 1 \\ 1 & 1 \eepm \bepm 2 & 1 \\ 1 & 1 \eepm = \bepm 5 & 3 \\ 3 & 2 \eepm = A.
\ee
\end{example}

\begin{proposition}
If $A$ is positive definite, then for integer $k\geq 1$, $\brb{A^{1/k}}^{-1} = \brb{A^{-1}}^{1/k}$ and it can be written by $A^{-1/k}$, which is also a positive definite.
\end{proposition}

\begin{proof}[\bf Proof]
We can uniquely define positive definite matrix $B = A^{1/k}$ by Theorem \ref{thm:existence_uniqueness_kth_root_matrix_of_positive_semi-definite_matrix}. Also, since all eigenvalues of $A$ and $B$
are positive, $A$ and $B$ are invertible. Therefore, $B^{-1}$ is well-defined and $A^{-1}$ is also a positive definite matrix by Proposition \ref{pro:positive_definite_matrix_inverse}. Let $C :=
\brb{A^{-1}}^{1/k}$ and it is uniquely determined by Theorem \ref{thm:existence_uniqueness_kth_root_matrix_of_positive_semi-definite_matrix}.

Since $A$ is Hermitian, we can have $A = U\Lambda U^*$ by spectral theorem for Hermitian matrices (Theorem \ref{thm:spectral_hermitian_matrices}) where $U$ is unitary and $\Lambda$ is diagonal
matrix of eigenvalues of $A$. Therefore, by the same step in proof of Proposition \ref{pro:positive_definite_matrix_inverse} and Theorem
\ref{thm:existence_uniqueness_kth_root_matrix_of_positive_semi-definite_matrix}, we have
\be
B = U\Lambda^{1/k} U^*,\ A^{-1} = U \Lambda^{-1} U^*,\ C = U \Lambda^{-1/k} U^* \ \ra\ BC = I,\ CB = I.
\ee

%\be
%\left\{\ba{l}
%C^k B^k = A^{-1} A = I\\
%B^k C^k = A A^{-1} = I \ea\right.
%\ \ra \
%\left\{\ba{l}
%C^k B^k = A^{-1} A = I\\
%B^k C^k = A A^{-1} = I \ea\right.
%\ee
%

Thus, $C = B^{-1}$. Furthermore, $A^{-1/k}$ is also a positive definite by Theorem \ref{thm:existence_uniqueness_kth_root_matrix_of_positive_semi-definite_matrix} as $A^{-1}$ is positive definite.
\end{proof}



\subsection{Product of positive definite matrices}

If $A,B>0$, $AB$, $BA$ and $AB+BA$ are not necessarily positive definite. For instance,
\be
A = \bepm 2 & 1 \\ 1 & 1 \eepm, \qquad B = \bepm 6 & 0 \\ 0 & 1 \eepm.
\ee

We have that
\be
AB = \bepm 12 & 1 \\6 & 1 \eepm,\qquad BA = \bepm 12 & 6 \\ 1 & 1 \eepm
\ee
which are not even Hermitian and
\be
AB + BA = \bepm 24 & 7 \\ 7 & 2 \eepm
\ee
whose determinant is -1. This implies that $AB + BA$ is not positive definite.

However, we can have the following theorem to guarantee the positive definite feature of $AB$.

\begin{theorem}
Let $A,B\in M_n(\C)$ and $A,B>0$ (resp. $A,B\geq 0$). If $AB$ is normal ($AB(AB)^* = (AB)^*AB$), then $AB>0$ (resp. $AB\geq 0$).
\end{theorem}

\begin{remark}
The special case is that $AB$ is Hermitian (or equivalently $AB = BA$) as we know that all Hermitian matrices are normal.
\end{remark}

\begin{proof}[\bf Proof]
Since $A,B>0$ ($A,B\geq 0$), we have that $A^{1/2}$ exists and unique by Theorem \ref{thm:existence_uniqueness_kth_root_matrix_of_positive_semi-definite_matrix}. Also, we have that $A^{1/2}BA^{1/2}$ and $A^{1/2}\brb{A^{1/2}B} = AB$ have the same eigenvalues by Theorem \ref{thm:product_matrices_change_order_have_the_same_eigenvalues}. Since $A^{1/2}BA^{1/2}> 0$ (resp. $A^{1/2}BA^{1/2}\geq 0$), we have that all the eigenvalues of $AB$ are positive (resp. non-negative).

Also, since $AB$ is normal, we have $AB$ is unitarily diagonalizable by spectral theorem for normal matrices (Theorem \ref{thm:spectral_normal_matrices}). That is, we can find a unitary matrix $U$ such that $AB = U^*\Lambda U$ where $\Lambda = \diag\bra{\lm_1,\dots,\lm_n}$ is a diagonal matrix of the eigenvalues of $AB$ with $\lm_i>0$ ($\lm_i \geq 0$). Meanwhile, $(AB)^* = U^*\Lambda^* U = U^*\Lambda U = AB$ which implies that $AB$ is a Hermitian matrix.

Hence, for any $x\neq 0$, we have that
\beast
x^* AB x & = & x^* U^* \Lambda U x = \sum^n_{i=1}\ol{(U x)_i} \lm_i (U x)_i = \sum^n_{i=1}\lm_i \ol{(U x)_i} (U x)_i = \sum^n_{i=1}\lm_i \ol{y_i} y_i > 0 \qquad(\text{resp. }\geq 0)
\eeast
where $Ux = y \neq 0$ and $\ol{y_i}y_i >0$ for some $i\in \bra{1,\dots,n}$. Thus, we can see that $AB$ is positive definite (resp. positive semi-definite).
\end{proof}



\begin{theorem}\label{thm:symmetrized_product_and_one_matrix_positive_definite_implies_positive_definite_of_other_matrix}
Let $A,B\in M_n(\C)$ with $A > 0$ (resp. $A \geq 0$) and $B$ be Hermitian. If $AB + BA > 0$ (resp. $AB + BA \geq 0$) then $B > 0$ (resp. $B \geq 0$).
\end{theorem}%Similarly, if $A \geq 0$ and $B$ be Hermitian, then $AB + BA \geq 0$ implies $B \geq 0$.

\begin{remark}
This proof is from p8 of Bhatia, Rajendra (2007). Positive Definite Matrices. Princeton, New Jersey: Princeton University Press. p. 8. ISBN 978-0-691-12918-1.
\end{remark}

\begin{proof}[\bf Proof]
First we can easily see that $AB + BA$ is Hermitian. By spectral theorem for Hermitian matrices (Theorem \ref{thm:spectral_hermitian_matrices}), we can find a unitary matrix $U$ such that $B = U^* \Lambda U$. Then $AB + BA = AU^*\Lambda U + U^*\Lambda U A$. Then the diagonal entries of $AU^*\Lambda U$ are
\be
\sum_{j=1}^n a_{ij} \brb{U^*\Lambda U}_{ji} = \sum_{j=1}^n a_{ij} U_j^* \brb{\Lambda U}_{i} = \sum_{j=1}^n a_{ij} U_j^* \lm_{ii} U_i = \lm_{ii}a_{ii}.
\ee

Similarly, the diagonal entries of $U^*\Lambda U A$ are
\be
\sum_{j=1}^n \brb{U^*\Lambda U}_{ij}a_{ji}  = \sum_{j=1}^n U_i^* \brb{\Lambda U}_{j}a_{ij} = \sum_{j=1}^n  U_i^* \lm_{jj} U_j a_{ji} = \lm_{ii}a_{ii}.
\ee

Thus, the diagonal entries of $AB+BA$ are $2\lm_{ii}a_{ii}$. Since we know $AB+BA>0$ (resp., $AB + BA \geq 0$), its diagonal entries must be positive (resp. non-negative). Therefore, $\lm_{ii}a_{ii} > 0$ (resp. $\lm_{ii}a_{ii}\geq 0$) which implies that $\lm_{ii} > 0$ (resp. $\lm_{ii}\geq 0$). Therefore, $B>0$ (resp. $B\geq 0$).
\end{proof}

\begin{corollary}
Let $A,B\in M_n(\C)$ with $A > 0$ (resp. $A\geq 0$) and $B$ be Hermitian. If $AB > 0$ (resp. $AB \geq 0$), then $B>0$ (resp. $B\geq 0$).
\end{corollary}

\begin{proof}[\bf Proof]
Clearly, $BA = B^*A^* = (AB)^*= AB > 0$ (resp. $BA \geq 0$). Thus, $AB + BA > 0$ ($AB + BA \geq 0$) and we can apply Theorem \ref{thm:symmetrized_product_and_one_matrix_positive_definite_implies_positive_definite_of_other_matrix}.
\end{proof}


\begin{proposition}
Let $A,B\in M_n(\C)$, $A>0$ and $B\geq 0$ with $A > B$, then $A^{1/2} > B^{1/2}$.

Similarly, if $A,B\geq 0$ with $A\geq B$, then $A^{1/2}\geq B^{1/2}$.
\end{proposition}

\begin{proof}[\bf Proof]
Let $X = A^{1/2} + B^{1/2}$ and $Y = A^{1/2} - B^{1/2}$. Clearly, $X$ and $Y$ are Hermitian and $X>0$ (resp. $X\geq 0$). Then
\beast
XY + YX & = & \brb{A^{1/2} + B^{1/2}}\brb{A^{1/2} - B^{1/2}} + \brb{A^{1/2} - B^{1/2}}\brb{A^{1/2} + B^{1/2}} \\
& = & A + B^{1/2}A^{1/2} - A^{1/2}B^{1/2} - B + A -  B^{1/2}A^{1/2} + A^{1/2}B^{1/2} - B = 2(A-B)
\eeast
which implies that $XY + YX > 0$ (resp. $XY + YX \geq 0$). By Theorem \ref{thm:symmetrized_product_and_one_matrix_positive_definite_implies_positive_definite_of_other_matrix}, we have that $Y > 0$ (resp, $Y\geq 0$), as required.
\end{proof}

\subsection{Other properties of positive definite matrix}

\begin{theorem}\label{thm:positive_definite_semidefinite_sum_determinant}
Let $A\in M_n(\F)$ be positive definite matrix and $B\in M_n(\F)$ be positive semi-definite. Then
\be
\det\brb{A+B} \geq \det A
\ee
with equality if and only if $B = 0$.
\end{theorem}

\begin{proof}[\bf Proof]
Let $\Lambda$ be a positive definite diagonal matrix such that $U^*A U = \Lambda$ for unitary matrix $U$. Then $UU^* = I$
\be
A + B = U\Lambda^{1/2} \brb{I + \Lambda^{-1/2}U^* B U\Lambda^{-1/2}} \Lambda^{1/2} U^*.
\ee

Hence, by  Theorem \ref{thm:determinant_product}%\footnote{determinant product needed.}
\beast
\det\brb{A+B} & = & \det\brb{U\Lambda^{1/2}} \det\brb{I + \Lambda^{-1/2}U^* B U\Lambda^{-1/2}}\det\brb{\Lambda^{1/2} U^*} \\
& = & \det\brb{U\Lambda^{1/2}} \det\brb{\Lambda^{1/2} U^*} \det\brb{I + \Lambda^{-1/2}U^* B U\Lambda^{-1/2}}\\
& = & \det\brb{U\Lambda^{1/2}\Lambda^{1/2} U^*} \det\brb{I + \Lambda^{-1/2}U^* B U\Lambda^{-1/2}}\\
& = & \det A \det\brb{I + \Lambda^{-1/2}U^* B U\Lambda^{-1/2}}.
\eeast

If $B = 0$ then $\det\brb{A+B} = \det A$. If $B \neq 0$, then the matrix $\Lambda^{-1/2}U^* B U\Lambda^{-1/2}$ will be positive semi-definite with at least one positive eigenvalue. Let $\lm_i$ be the eigenvalues of $\Lambda^{-1/2}U^* B U\Lambda^{-1/2}$ and thus $\diag\brb{\lm_1,\dots,\lm_n} = \Sigma = X^{-1}\Lambda^{-1/2}U^* B U\Lambda^{-1/2} X$. Then %$\Sigma$ be the eigenvalue of $\Lambda^{-1/2}U^* B U\Lambda^{-1/2}$ and $X$ its corresponding matrix consisting of eigenvectors, then
\be
\Lambda^{-1/2}U^* B U\Lambda^{-1/2} X = X \Sigma \ \ra\ \ \brb{I + \Lambda^{-1/2}U^* B U\Lambda^{-1/2}}X = X(I+\Sigma)
\ee
which implies that $1+\lm_i$ is the eigenvalue of $I + \Lambda^{-1/2}U^* B U\Lambda^{-1/2}$. Therefore, we have
\be
\det\brb{I + \Lambda^{-1/2}U^* B U\Lambda^{-1/2}} = \prod^n_{i} (1+\lm_i) > 1 \ \ra\ \det\brb{A+B} > \det A.
\ee
\end{proof}


\begin{corollary}
Let $A,B\in M_n(\F)$ be positive definite matrices such that $A\geq B$. Then $\det A \geq \det B$ with equality if and only if $A=B$.
\end{corollary}

\begin{proof}[\bf Proof]%Proposition \ref{pro:positive_definite_diff_reverse_diff_positive_definite}
Let $C=A-B$. Then $C$ is positive semi-definite. By Theorem \ref{thm:positive_definite_semidefinite_sum_determinant},
\be
\det\brb{B+C} \geq \det\brb{B}
\ee
with equality if and only if $C=0$. That is, $\det A \geq \det B$ with equality if and only if $A=B$.
\end{proof}

A useful special case of the above corollary is the following proposition.

\begin{proposition}
Let $A$ be positive definite with $\det A =1$. If $I-A$ is also positive semi-definite, then $A=I$.
\end{proposition}


\begin{proposition}\label{pro:positive_definite_hermitian_decomposition_common_factor}
Let $A\in M_n(\F)$ be positive definite and $B\in M_n(\F)$ be Hermitian matrix. Then there exists a non-singular matrix $P$ and a diagonal matrix $\Lambda$ such that
\be
A = PP^*,\qquad B = P\Lambda P^*.
\ee
\end{proposition}


\begin{proof}[\bf Proof]
Let $C = A^{-1/2}BA^{-1/2}$. Since $C$ is Hermitian, there exists unitary matrix $U$ and a diagonal matrix $\Lambda$ such that
\be
U^* C U = \Lambda, \qquad U^*U = I.
\ee

Then define $P = A^{1/2}U$ and we have
\be
PP^* = A^{1/2}U U^* A^{1/2} = A^{1/2} A^{1/2} = A.
\ee

Also,
\be
P\Lambda P^* = A^{1/2}U U^* C U U^* A^{1/2} = A^{1/2} C A^{1/2} = B
\ee
as required.
\end{proof}

\begin{definition}
Let $A,B\in M_n(\F)$ be two Hermitian matrices. Then we write $A\geq B$ (or $B\leq A$) if $A-B$ is positive semi-definite. Furthermore, $A>B$ (or $B < A$) if $A-B$ is positive definite.
\end{definition}

\begin{remark}
The definition also holds for real version. That is, $A,B$ are symmetric.
\end{remark}

\begin{proposition}\label{pro:positive_definite_diff_reverse_diff_positive_definite}
Let $A,B\in M_n(\F)$ be positive definite matrices. Then
\be
A>B \ \lra \ B^{-1} > A^{-1}.
\ee
\end{proposition}

\begin{proof}[\bf Proof]
By Proposition \ref{pro:positive_definite_hermitian_decomposition_common_factor}, there exists a non-singular matrix $P$ and a positive definite diagonal matrix $\Lambda = \diag\brb{\lm_1,\dots,\lm_n}$ such that
\be
A = PP^*,\qquad B = P\Lambda P^* \ \ra\ A-B = P(I - \Lambda)P^*,\qquad B^{-1} - A^{-1} = \brb{P^*}^{-1} \brb{\Lambda^{-1} -I}P^{-1}.
\ee

If $A-B$ is positive definite, then $I-\Lambda$ is positive definite and hence $0<\lm_i<1$ ($i=1,\dots,n$). This implies that $\Lambda^{-1} -I$ is positive definite and hence that $B^{-1}-A^{-1}$ is positive definite. Similar argument for reverse.
\end{proof}

\subsection{Principal submatrix}

\begin{proposition}\label{pro:positive_definite_iff_diagonal_element}
Let $A\in M_n(\F)$ be positive definite matrix and $B$ be $M_{n+1}(\F)$
\be
B  = \bepm
A & b \\ b^* & a
\eepm
\ee
where $a\in \R$. Then
\ben
\item [(i)] $\abs{B} \leq a\abs{A}$ with equality if and only if $b = 0$.
\item [(ii)] $B$ is positive definite if and only if $\abs{B}>0$.
\een
\end{proposition}

\begin{proof}[\bf Proof]
Define matrix
\be
P = \bepm
I_n & -A^{-1}b \\ 0 & 1
\eepm
\ee
since $A^{-1}$ exists for positive definite matrix.
\ben
\item [(i)] By $A^* = A$
\be
P^*BP = \bepm
I_n & 0 \\ -b^* \brb{A^*}^{-1} & 1
\eepm \bepm
A & b \\ b^* & a
\eepm \bepm
I_n & -A^{-1}b \\ 0 & 1
\eepm = \bepm
A & b \\ 0 & a - b^*A^{-1}b
\eepm \bepm
I_n & -A^{-1}b \\ 0 & 1
\eepm  = \bepm
A & 0 \\ 0 & a - b^*A^{-1}b
\eepm\nonumber
\ee
so that by Theorem \ref{thm:determinant_product}
\be
\abs{B} = \abs{P^*B P} = \abs{A} \brb{a - b^*A^{-1}b} \leq \abs{A}a
\ee
since $A^{-1}$ is also positive definite by Proposition \ref {pro:positive_definite_matrix_inverse}. The equality holds when $b=0$.

\item [(ii)] Then we can have $\abs{B}>0$ if and only if $a - b^*A^{-1}b>0$, which is the case if only if $P^*BP$ is positive definite. Since
\be
\bepm
I_n & A^{-1}b \\ 0 & 1
\eepm \bepm
I_n & -A^{-1}b \\ 0 & 1
\eepm = QP = PQ = I \ \ra\ Q = P^{-1} .
\ee
Thus, we have the required result by Proposition \ref{pro:product_rank_positive_definite}.
\een
\end{proof}

Then by Proposition \ref{pro:positive_definite_iff_diagonal_element} we have the following property.

\begin{proposition}\label{pro:positive_definite_determinant_smaller_than_product_diagonal_elements}
Let $A\in M_n(\F)$ be a positive semi-definite matrix. Then
\be
\det A \leq a_{11}\dots a_{nn} = \prod^n_{i=1}a_{ii}.
\ee

Furthermore, if $A$ is positive definite, then equality holds in the above inequality if and only if $A$ is diagonal matrix.
\end{proposition}

\begin{remark}
By spectral theorem (Theorem \ref{thm:spectral_hermitian_matrices}), all the eigenvalues are real, we can then see that $\det A$ is real by Theorem \ref{thm:elementary_symmetric_function_sum_of_determinant_principal_submatrix_equivalent}. Also, we have $a_{ii}\geq 0$ for all $i$ since $A$ is semi-definite by Proposition \ref{pro:positive_definite_principal_submatrix_is_positive_definite}. So the these terms are comparable.

If the matrix $A$ is not positive defnite and merely symmetric, the above equality is no longer sufficient for the diagonality of $A$. For example, the matrix
\be
A = \bepm
2 & 3 & 3 \\
3 & 2 & 3 \\
3 & 3 & 2 \\
\eepm
\ee
has determinant $\abs{A} = 8$ (its eigenvalues are $-1,-1,8$), thus satisfying the equality, but $A$ is not diagonal. \footnote{proposition needed. A real symmetric  matrix is diagonal if and only if its eigenvalues and its diagonal elements coincide. see matrix calculus p27.}
\end{remark}

\begin{proof}[\bf Proof]
If $A$ is singular, then $\det A = 0$ by Theorem \ref{thm:matrix_invertible_determinant_non_zero}. whereas $a_{ii}\geq 0$ for all $i$, and the result is trivial.

So suppose $A$ is nonsingular. Then each $a_{ii} > 0$. Let $D = \diag\brb{\sqrt{a_{11}},\dots,\sqrt{a_{nn}}}$ and let $B = D^{-1}AD^{-1}$. Then $B$ is positive semi-definite and $b_{ii} = 1$ for each $i$. Let $\lm_1,\dots,\lm_n$ be the eigenvalues of $B$. By the arithmetric-geometric mean inequality\footnote{theorem needed.},
\be
\frac 1n \sum^n_{i=1}\lm_i \geq \brb{\prod^n_{i=1} \lm_i}^{1/n}.
\ee

Since $\sum^n_{i=1}\lm_i = \tr(B) = n$, and $\det B = \prod^n_{i=1}\lm_i$, we get $\det B \leq 1$. Therefore, by
\be
\det A \prod^n_{i=1} a_{ii}^{-1}  = \det D^{-2} \det A = \det\brb{D^{-1}AD^{-1}}  \leq 1
\ee
which gives the required result.

If $A$ is positive definite and if equality holds in the arithmetic-geometric mean inequality in the proof above. But then $\lm_1,\dots,\lm_n$ are all equal, and it follows by the spectral theorem (Theorem \ref{thm:spectral_hermitian_matrices}) that $B$ is scalar multiple of the identity matrix. Then $A$ must be diagonal.
\end{proof}



\begin{corollary}\label{cor:positive_definite_determinant_smaller_than_n_power_n}
Let $X\in M_n(\C)$ and suppose $\abs{x_{ij}}\leq 1$ for all $i,j$. Then
\be
\det\brb{X^*X} \leq n^n.
\ee
\end{corollary}

\begin{proof}[\bf Proof]
For $A = X^*X$, we have
\be
a_{ii} = \sum^n_{j=1} \ol{x_{ji}} x_{ji} \leq n.
\ee

Since $X^*X$ is positive semi-definite, by Proposition \ref{pro:positive_definite_determinant_smaller_than_product_diagonal_elements}, we have the required result.
\end{proof}

The following proposition gives a necessary and sufficient condition for the diagonality of a Hermitian matrix.

\begin{proposition}
Let $A\in M_n(\F)$ be Hermitian matrix. Then $A$ is diagonal if and only if its eigenvalues and its diagonal elements coincide.
\end{proposition}

\begin{proof}[\bf Proof]
($\ra$). If $A$ is diagonal, then we can write $A=\diag\brb{a_{11},\dots,a_{nn}}$. Then for $e_i = \brb{0,\dots,0,\underbrace{1}_{\text{position }i},0,\dots,0}^T$,
\be
Ae_i = a_{ii}e_i \ \ra\ a_{ii} \text{ is eigenvalue and its vector is }e_i.
\ee

($\la$). Let $a_{ii}$ be the eigenvalues of $A$, for $i=1,\dots,n$. Then consider the matrix $B = A + kI$ where $k>0$ such that $B$ is positive definite. That is, the eigenvalues of $B$ are $\lm_i = a_{ii}+k = b_{ii}$ and their corresponding eigenvectors are the ones with respect to eigenvalues $a_{ii}$ of $A$. Note that $k$ can be large enough to guarantee that all $\lm_i$ are positive so $B$ is positive definite by Theorem \ref{thm:positive_definite_matrix_iff_positive_eigenvalue}.

Hence, by Theorem \ref{thm:elementary_symmetric_function_sum_of_determinant_principal_submatrix_equivalent},
\be
\det B = \prod^n_{i=1}\lm_i = \prod^n_{i=1} b_{ii} \ \ra\ B\text{ is diagonal}
\ee
by Proposition \ref{pro:positive_definite_determinant_smaller_than_product_diagonal_elements}. This implies that $A = B- kI$ is diagonal as well.
\end{proof}



\begin{theorem}\label{thm:positive_definite_iff_leading_principal_submatrices_determinants}
Let $A\in M_n(\C)$ be Hermitian and $A_k$ be its leading principal submatrix of first $k$ rows and columns.

Then $A$ is positive definite if and only if $\det A_k > 0$ for $k=1,2,\dots,n$.

More generally, the positivity of any nested sequence of $n$ principal submatrices' determinants of $A$ (not just the leading principal submatrix determinants) is necessary and sufficient for $A$ to be positive definite.
\end{theorem}

\begin{proof}[\bf Proof]
($\ra$). Let $P_k = \bepm I_k & 0 \eepm$ be a $k\times n$ matrix in $M_n(\F)$, so that $A_k = P_k A P_k^*$. Let $x$ be an arbitrary non-zero $k\times 1$ vector and then
\be
x^* A_k x = \brb{P_k^* x}^* A \brb{P_k^* x} >0
\ee
since $P_k^* x \neq 0$ and $A$ is positive definite. Hence $A_k$ is positive definite and in particular, $\det A_k>0$.

($\la$). For leading principal submatrices, it is obvious, $A_1$ is positive definite $\det A_1 >0$. Then by induction, we assume $A_k$ is positive definite for $k=1,\dots,n-1$ and
\be
A_{k+1} = \bepm A_k & b_k \\ b_k^* & a_{k+1,k+1} \eepm
\ee

Then by Proposition \ref{pro:positive_definite_iff_diagonal_element}, we have that $A_{k+1}$ is positive definite since $\det A_{k+1} >0$ by assumption. Since $A_n = A$, we can conclude that $A$ is positive definite.
\end{proof}


\subsection{Cholesky decomposition}


\begin{theorem}\label{thm:positive_definite_product_nonsingular_matrix}
A matrix $B\in M_n(\C)$ is positive definite if and only if there is a nonsingluar matrix $C\in M_n(\C)$ such that $B = C^*C$.
\end{theorem}

\begin{proof}[\bf Proof]
If $B$ can be so written, we have $B = C^*C = C^*I C$ is positive definite by Proposition \ref{pro:product_rank_positive_definite} since rank of $C$ is $n$ (by Theorem \ref{thm:invertible_full_rank} as $C$ is nonsingular).

If $B$ is positive definite, let $U^*B U = \Lambda$ by spectral theorem for Hermitian matrices (Theorem \ref{thm:spectral_hermitian_matrices}) where $\Lambda = \diag\brb{\lm_1,\dots,
\lm_n}$ with $\lm_1,\dots,\lm_n >0$. Thus, we can define
\be
C := U \Lambda^{1/2}U^* \ \ra \ C^*C = U \Lambda^{1/2}U^*U \Lambda^{1/2}U^* = U \Lambda U^* = B.
\ee

Obviously, $\det C = \det U\brb{\det \Lambda}^{1/2}\det \brb{U^*} \neq 0$ as $\det U$, $\det \brb{U^*}$ and $\det \Lambda$ are non-zero. This gives the fact that $C$ is invertible by Theorem \ref{thm:matrix_invertible_determinant_non_zero}.
\end{proof}

\begin{example}\label{exa:toeplitz_matrix_product_nonsingular_matrix}
\be
B = \bepm
n & n-1 & n-2 & \cdots & 2 & 1 \\
n-1 & n-1 & n-2 & \cdots & 2 & 1 \\
n-2 & n-2 & n-2 & \cdots & 2 & 1 \\
\vdots & \vdots & \vdots & \ddots & \vdots & \vdots\\
2 & 2 & 2 & \cdots & 2 & 1 \\
1 & 1 & 1 & \cdots & 1 & 1
\eepm = \bepm
1 & 1 & 1 & \cdots & 1 & 1 \\
 & 1 & 1 & \cdots & 1 & 1 \\
 &  & 1 & \cdots & 1 & 1 \\
& &  & \ddots & \vdots & \vdots\\
&  & & & 1 & 1 \\
&  & &  &  & 1
\eepm \bepm
1 &  &  &  &  &  \\
1 & 1 &  &  &  &  \\
1 & 1 & 1 &  &  &  \\
\vdots & \vdots & \vdots & \ddots & & \\
1 & 1 & 1 & \cdots & 1 &  \\
1 & 1 & 1 & \cdots & 1 & 1
\eepm = C^TC
\ee
where
\be
C = \bepm
1 &  &  &  &  &  \\
1 & 1 &  &  &  &  \\
1 & 1 & 1 &  &  &  \\
\vdots & \vdots & \vdots & \ddots & & \\
1 & 1 & 1 & \cdots & 1 &  \\
1 & 1 & 1 & \cdots & 1 & 1
\eepm .
\ee

By Example \ref{exa:inverse_of_toeplitz_matrix_plus_10} and Theorem \ref{thm:determinant_product} ($\det B = \det \brb{C^T}\det C = 1$ as $\det C = 1$) we can have that
\be
4^n = \prod^n_{k=1}\sec^2\brb{\frac {k\pi}{2n+1}}.
\ee

Furthermore,
\be
\brb{\frac 12}^n = \prod^n_{k=1}\cos\brb{\frac {k\pi}{2n+1}}.
\ee

For $n=2$, by Example \ref{exa:sin_pi_divided_by_5}, we have
\be
\sin\brb{\pi/5} = \frac 14\sqrt{10-2\sqrt{5}}\quad \ra\quad \cos\brb{\pi/5} = \frac 14\sqrt{6+2\sqrt{5}},\quad \cos\brb{2\pi/5} = \frac 14\sqrt{6-2\sqrt{5}}.
\ee

Thus,
\be
\brb{\frac 12}^2 = \frac 14 = \frac 14\sqrt{6+2\sqrt{5}} \ \cdot \ \frac 14\sqrt{6-2\sqrt{5}} = \cos\brb{\pi/5} \cos\brb{2\pi/5}.
\ee
\end{example}


\begin{corollary}
A Hermitian matrix $A$ is positive definite if and only if it is $*$-congruent\footnote{definition needed.} to the identity.
\end{corollary}

\begin{proof}[\bf Proof]
\footnote{proof needed.}
\end{proof}


\begin{corollary}[Cholesky decomposition\index{Cholesky decomposition}]\label{cor:cholesky_decomposition}
A matrix $A\in M_n(\C)$ is positive definite if and only if there exists an invertible lower triangular matrix $L \in M_n(\C)$ with positive diagonal entries such that $A = LL^*$.

If $A$ is real, $L$ may be taken to be real.

Note that $L$ is not unique.\footnote{example needed.} %as we can take $L' = L\diag\brb{1,\dots,-1,\dots,1}$ such that $L'L'^* = L I L^* = LL^* = A$.
\end{corollary}

\begin{remark}
Cholesky decomposition is discovered by Andr\'e-Louis Cholesky for real matrices and is an example of a square root of a matrix. When it is applicable, the Cholesky decomposition is roughly twice as efficient as the $LU$ decomposition for solving systems of linear equations.
\end{remark}

\begin{proof}[\bf Proof]%\footnote{proof needed.}
According to Theorem \ref{thm:positive_definite_product_nonsingular_matrix}, we can have that $A$ is positive definite if and only if there is a nonsingular matrix $C$ such that $B =C^*C$. By $QR$
factorization (Theorem \ref{thm:qr_factorization}), we have $C = QR$ where $Q$ is unitary and $R$ is an upper triangular matrix with positive diagonal entries (and thus nonsingular). Then
\be
A = C^*C = \brb{QR}^*QR = R^* Q^*Q R = R^*R.
\ee

Define $L := R^*$, we can have the required form.

Now assume $A$ is real.

If there exists such lower triangular matrix $L$, we can take real $Q$ and $R$ to form nonsingular real $C$ (by $QR$ factorization (Theorem \ref{thm:qr_factorization})), which will imply the
positive definiteness of $A$.

Conversely, if $A$ is positive definite, we can replicate the method in the proof of Theorem \ref{thm:positive_definite_product_nonsingular_matrix} by applying spectral theorem for Hermitian
matrices (Theorem \ref{thm:spectral_hermitian_matrices}) in the real case to get the real $C$. Accordingly, the lower triangular matrix $L$ can be given by applying the real case of $QR$
factorization (Theorem \ref{thm:qr_factorization}).
\end{proof}

\begin{example}
We have
\be
\bepm
4 & 12 & -16 \\
12 & 37 & -43 \\
-16 & -43 & 98
\eepm = \bepm
2 & & \\6 & 1 & \\ -8 & \ 5\ & 3
\eepm \bepm
2 & \ 6 \ & -8 \\
& 1 & 5 \\
& & 3
\eepm.
\ee

We can also decompose it by
\be
\bepm
4 & 12 & -16 \\
12 & 37 & -43 \\
-16 & -43 & 98
\eepm = \bepm
1 & & \\3 & 1 & \\ -4 & \ 5\ & 1
\eepm \bepm
4 & & \\
& 1 & \\
& & 9
\eepm\bepm
1 & \ 3 \ & -4 \\
& 1 & 5 \\
& & 1
\eepm.
\ee
\end{example}

\begin{example}
\be
\bepm
1 & 1 & 1 & \cdots & 1 & 1 \\
1 & 2 & 2 & \cdots & 2 & 2 \\
1 & 2 & 3 & \cdots & 3 & 3 \\
\vdots & \vdots & \vdots & \ddots & \vdots & \vdots\\
1 & 2 & 3 & \cdots & n-1 & n-1 \\
1 & 2 & 3 & \cdots & n-1 & n
\eepm =  \bepm
1 &  &  &  &  &  \\
1 & 1 &  &  &  &  \\
1 & 1 & 1 &  &  &  \\
\vdots & \vdots & \vdots & \ddots & & \\
1 & 1 & 1 & \cdots & 1 &  \\
1 & 1 & 1 & \cdots & 1 & 1
\eepm \bepm
1 & 1 & 1 & \cdots & 1 & 1 \\
 & 1 & 1 & \cdots & 1 & 1 \\
 &  & 1 & \cdots & 1 & 1 \\
& &  & \ddots & \vdots & \vdots\\
&  & & & 1 & 1 \\
&  & &  &  & 1
\eepm = LL^T
\ee
where
\be
L = \bepm
1 &  &  &  &  &  \\
1 & 1 &  &  &  &  \\
1 & 1 & 1 &  &  &  \\
\vdots & \vdots & \vdots & \ddots & & \\
1 & 1 & 1 & \cdots & 1 &  \\
1 & 1 & 1 & \cdots & 1 & 1
\eepm .
\ee
\end{example}


%\begin{remark}
%This is true for real and complex $A$.
%\end{remark}

%\section{PoNon-negative Definite Matrices}

\begin{algorithm}[Cholesky algorithm, modified version of Gaussian elimination\index{Cholesky algorithm!modified version of Gaussian elimination}]
Given a complex positive definite matrix $A\in M_n(\C)$.

The recursive algorithm starts with $i =1$ and $A^{(1)} :=A$. At step $i$, the matrix $A^{(i)}$ has the following form
\be
A^{(i)} = \bepm
I_{i-1} & 0 & 0\\
0 & a_{ii} & b_i^* \\
0 & b_i & B^{(i)}_{n-i}
\eepm
\ee
where $I_{i-1}$ denotes the identity matrix of dimension $i-1$ and $B^{(i)}_{n-i}$ is the lower right corner block (with size $(n-i)\times (n-i)$) of $A$. That is
\be
B^{(i)}_{n-i} = \bepm
a_{i+1,i+1}& b_{i+1}^* \\
b_{i+1} & B^{(i+1)}_{n-i-1}
\eepm.
\ee

If we now define the matrix $L_i$ by
\be
L_i := \bepm
I_{i-1} & 0 & 0 \\
0 & \sqrt{a_{ii}} & 0 \\
0 & \frac 1{\sqrt{a_{ii}}}b_i & I_{n-i}
\eepm.
\ee

Note that $L_i$ and its inverse are well-defined as $\sqrt{a_{ii}}$ is positive real number by Proposition \ref{pro:positive_definite_principal_submatrix_is_positive_definite}. Then we can write $A^{(i)}$ as
\beast
 L_i A^{(i+1)} L_i^* & =&
\bepm
I_{i-1} & 0 & 0 \\
0 & \sqrt{a_{ii}} & 0 \\
0 & \frac 1{\sqrt{a_{ii}}}b_i & I_{n-i}
\eepm
\bepm
I_{i} & 0 & 0\\
0 & a_{i+1,i+1} & b_{i+1}^* \\
0 & b_{i+1} & B^{(i+1)}_{n-i-1}
\eepm
\bepm
I_{i-1} & 0 & 0 \\
0 & \sqrt{a_{ii}} & \frac 1{\sqrt{a_{ii}}}b_i^* \\
0 & 0 & I_{n-i}
\eepm \\
& =&
\bepm
I_{i-1} & 0 & 0 \\
0 & \sqrt{a_{ii}} & 0 \\
0 & \frac 1{\sqrt{a_{ii}}}b_i & I_{n-i}
\eepm
\bepm
I_{i} & 0 \\
0 & B^{(i)}_{n-i}
\eepm
\bepm
I_{i-1} & 0 & 0 \\
0 & \sqrt{a_{ii}} & \frac 1{\sqrt{a_{ii}}}b_i^* \\
0 & 0 & I_{n-i}
\eepm \\
& = &  \bepm
I_{i-1} & 0 & 0 \\
0 & \sqrt{a_{ii}} & 0 \\
0 & \frac 1{\sqrt{a_{ii}}}b_i & B^{(i)}_{n-i}
\eepm
\bepm
I_{i-1} & 0 & 0 \\
0 & \sqrt{a_{ii}} & \frac 1{\sqrt{a_{ii}}}b_i^* \\
0 & 0 & I_{n-i}
\eepm = \bepm
I_{i-1} & 0 & 0\\
0 & a_{ii} & b_i^* \\
0 & b_i & B^{(i)}_{n-i}
\eepm = A^{(i)}.
\eeast

Thus, we have
\be
A = A^{(1)} = L_1 A^{(2)}L_1^* = \dots = L_1 L_2 \dots L_n A^{(n+1)} L_n^* \dots L_2^* L_1^* = L_1 L_2 \dots L_n L_n^* \dots L_2^* L_1^* = LL^*
\ee
where $L = L_1 L_2 \dots L_n$. Here we apply the fact that $A^{(n+1)} = I$. Then $L$ is the lower triangular matrix of Cholesky decomposition.
\end{algorithm}

\begin{algorithm}[Cholesky-Banachiewicz algorithm\index{Cholesky algorithm, Banachiewicz}, Cholesky-Crout algorithm\index{Cholesky algorithm, Crout}]

Given $A\in M_n(\C)$, we write $A = LL^*$ can have%Cholesky-Crout algorithm (column by column)
\be
A = LL^* = \bepm
l_{11} & 0 & \dots & 0 \\
l_{21} & l_{22} & \dots & 0 \\
\vdots & & \ddots & \vdots \\
l_{n1} & l_{n2} & \dots & l_{nn}
\eepm\bepm
l_{11} & \ol{l_{21}} & \dots & \ol{l_{n1}} \\
0 & l_{22} & \dots & \ol{l_{n2}} \\
\vdots & & \ddots & \vdots \\
0 & 0 & \dots & l_{nn}
\eepm.
\ee

Note that $l_{ii}$ is positive ($i=1,\dots,n$) by Corollary \ref{cor:cholesky_decomposition}. Thus, we have for $j = 1,\dots,n$
\be
a_{jj} = \sum^j_{k=1} l_{jk}\ol{l_{jk}}\qquad (*)
\ee

For $i > j$, $j=1,\dots,n-1$,
\be
a_{ij} = \sum^{j}_{k=1} l_{ik} \ol{l_{jk}},\qquad a_{ji} = \sum^{j}_{k=1} l_{jk} \ol{l_{ik}}\qquad (\dag)
\ee
as $A$ is Hermitian matrix.

Thus, Cholesky-Banachiewicz algorithm starts from the upper left corner of the matrix $L$ and proceeds to calculate the matrix row by row.

For row 1, by $(*)$,
\be
l_{11} = \sqrt{a_{11}},\qquad l_{1i}= 0,\quad i\geq 2.
\ee

For row $i>j$ and $i\geq 2$, by $(\dag)$ we have
\be
l_{ij}= \ol{\frac 1{l_{jj}}\brb{a_{ji}-\sum^{j-1}_{k=1} l_{jk} \ol{l_{ik}} }} = \frac 1{l_{jj}}\brb{a_{ij}-\sum^{j-1}_{k=1} l_{ik} \ol{l_{jk}} },\qquad l_{ii} = \sqrt{a_{ii} - \sum^{i-1}_{k=1} l_{ik}\ol{l_{ik}}}
\ee
given rows $1,\dots,i-1$.

Cholesky-Crout algorithm starts from the upper left corner of the matrix $L$ and proceeds to calculate the matrix column by column.

For column 1,
\be
l_{11} = \sqrt{a_{11}},\qquad l_{i1}= \frac {a_{i1}}{l_{11}},\quad i\geq 2.
\ee

For column $j<i$ and $j\geq 2$
\be
l_{jj} = \sqrt{a_{jj} - \sum^{j-1}_{k=1} l_{jk}\ol{l_{jk}}},\qquad l_{ij}= \frac 1{l_{jj}}\brb{a_{ij}-\sum^{j-1}_{k=1} l_{ik} \ol{l_{jk}} }, \qquad j = 2,\dots,n.
\ee
given all columns $1,\dots,j-1$.
\end{algorithm}




\subsection{The polar form and the singular value decomposition}

\begin{lemma}\label{lem:any_matrix_product_of_unitary_diagonal_orthonormal}
Let $A\in M_{m,n}(\C)$ with $m\leq n$ and $\rank (A) = k\leq m$.

Then there exists a unitary matrix $X\in M_m(\C)$, a diagonal matrix $\Lambda \in M_m(\C)$ with non-negative diagonal entries $\lm_1 \geq \lm_2 \geq \dots \geq \lm_k > \lm_{k+1} = \dots = \lm_m =
0$, and a matrix $Y\in M_{m,n}(\C)$ with orthonormal rows (and thus $YY^* = I_m$) such that
\be
A = X\Lambda Y.
\ee

The matrix $\Lambda = \diag\brb{\lm_1,\dots,\lm_n}$ is always uniquely determined and $\bra{\lm_1^2,\dots,\lm_n^2}$ are the eigenvalues of $AA^*$. The columns of the matrix $X$ are eigenvectors of
$AA^*$.

If $AA^*$ has distinct eigenvalues, then $X$ is determined up to a right diagonal factor $D = \diag\brb{e^{i\theta_1},\dots,e^{i\theta_n}}$ with all $\theta_i\in \R$. That is, if $A = X'\Lambda Y' =
X''\Lambda Y''$, then $X'' = X'D$.

Given $X$, the matrix $Y$ is uniquely determined if $\rank(A) = m$.

If $A$ is real, then $X$ and $Y$ can be taken to be real.
\end{lemma}

\begin{proof}[\bf Proof]
First we consider Hermitian matrix $AA^*$. It is obvious that $AA^*$ is positive semi-definite. Thus, by spectral theorem for Hermitian matrices (Theorem \ref{thm:spectral_hermitian_matrices}) we
can find a unitary matrix $X$ such that
\be
AA^* = X\Lambda^2 X^*
\ee
where $\Lambda^2 = \diag\brb{\lm_1^2,\dots,\lm_n^2}$ with eigenvalues of $AA^*$ as all its eigenvalues $\lm_1^2,\dots,\lm_n^2$ are non-negative. Because the diagonal entries of $\Lambda$ are to be
non-negative and are to be arranged in non-increasing order, $\Lambda$ is uniquely determined by $AA^*$. Also, we have
\be
AA^* X_i = \lm_i^2 X_i,\quad i = 1,2,\dots,m
\ee
where $X_i$ is the $i$th column of $X$.

Note that we have $\rank\brb{\Lambda^2} = \rank\brb{AA^*} = \rank (A^*) = k$ by Proposition \ref{pro:rank_equalities}.(ii),(iv). Thus, we can see that $\lm_i^2 > 0$ for $i\leq k$ and $\lm_i^2 = 0$ for $i>k$, $i=1,\dots,m$.

If the numbers $\bra{\lm_i^2}$ are distinct, the corresponding normalized eigenvectors of $AA^*$ are each determined up to a complex scalar factor of modulus
1. This is because that if there exists two eigenvectors of $AA^*$ associated with the same $\lm_i^2$, denoted as $X_i$ and $X_i'$, then each of them can form a basis by combining the other
eigenvectors associated with other eigenvalues. This means that $X_1,\dots,X_i,\dots,X_m$ and $X_1,\dots,X_i',\dots,X_m$ are both bases. Thus, $X_i$ can be expressed by the linear combination of
$X_1,\dots,X_i,\dots,X_m$, i.e.,
\be
X_i' = a_1 X_1 + \dots + a_m X_m \ \ra\ \inner{X_i'}{X_j} = a_j\inner{X_j}{X_j} = a_j.
\ee

Thus, $a_j = 0$ for any $i\neq j$ and $X_i' = a_i X_i$ with $\abs{a_i} = 1$. Therefore, if $X'$ and $X''$ are unitary matrices whose columns are eigenvectors of $AA^*$, we must have $X'' = X'D$
with $D = \diag\brb{d_1,\dots,d_m}$ and all $\abs{d_i} = 1$ for $i = 1,\dots,m$. Eigenvectors of $AA^*$ corresponding to a multiple eigenvalue are not uniquely determined.%but once they are chosen and orthonormalized so that

Now assume $X$ is given. If $A$ is full rank, i.e., $k = \rank(A) = m$. Then $\rank\brb{AA^*} = \rank (A) =m$ by Proposition \ref{pro:rank_equalities}.(i)\&(iv). Then $\rank(\Lambda) =
\rank\brb{\Lambda^2} = m$ by Proposition \ref{pro:rank_equalities}.(ii)\&(iv). Thus $\Lambda$ is nonsingular by Proposition \ref{pro:invertible_non_singular_equivalent}. Then $Y$ is uniquely determined by
\be
Y = \Lambda^{-1}X^*A.
\ee

It is easy to check that
\be
YY^* = \Lambda^{-1}X^*A A^* X \Lambda^{-1} = \Lambda^{-1} X^*X\Lambda^2 \Lambda^{-1} =  \Lambda^{-1} X^*X\Lambda = I_m,
\ee
so this matrix $Y$ has orthonormal rows.

It remains only to handle the case in which $\rank(A) = k<m$ ($\lm_i>0$ for $i\leq k$ and $\lm_i =0$ for $i>k$). %Since we want $Y = \Lambda^{-1} X^* A$
We define the $i$th row vector $Y_i^*$ where
\be
Y_i := \lm_i^{-1}\brb{A^*X_i},\qquad i = 1,\dots,k.
\ee

Then for $i,j = 1,\dots,k$,
\be
Y_i^* Y_j = \brb{ \lm_i^{-1}\brb{A^*X_i}}* \lm_j^{-1}\brb{A^*X_j} = \frac{1}{\lm_i\lm_j} X_i^* AA^* X_j = \frac{1}{\lm_i\lm_j} X_i^* \lm_j^2 X_j = \frac{\lm_j}{\lm_i} X_i^* X_j
\ee
which is 0 if $i\neq j$ and is $1$ if $i=j$ since the vectors $\brb{X_i}$ are orthonormal. Thus, $\brb{Y_1,\dots,Y_k}$ are orthonormal set in $\C^n$, and $n\geq m> k$, so there exists $m-k$ additional
(but not uniquely determined) orthonormal vectors $Y_{k+1},\dots,Y_m$ (by Theorem \ref{thm:steinitz_exchange}) such that the matrix $Y^* := \brb{Y_1,\dots,Y_k,Y_{k+1},\dots,Y_m} \in M_{n,m}(\C)$ has $m$ orthonormal columns.

Now notice that $X^*A = \Lambda Y$. The first $k$ rows of both sides of this identity are equal by construction of the vector $Y_i$ for $i=1,\dots,k$. The last $m-k$ rows are all 0 on the right
because the last $m-k$ diagonal entries $\Lambda$ are 0. If $AA^* X_i = 0 = \lm_i^2 X_i$ for $i=k+1,\dots,m$ ($X_i$ is the eigenvector associated with $\lm_i^2$), then $\brb{X_{k+1},\dots,X_m}$ are
the last $m-k$ columns of $X$. Thus, we have for $i= k+1,\dots,m$,
\be
AA^* X_i = 0 \ \ra\ \ X_i^* AA^*X_i = 0 \ \ra\ \brb{A^*X_i}^*A^*X_i = 0 \ \ra \ A^*X_i = 0 \ \ra \ X_i^*A = 0,
\ee
which means the last $m-k$ rows are all 0 on the left. Thus, $X^*A = \Lambda Y$ gives that the statement that
\be
A = \brb{X^*}^{-1} \Lambda Y = X \Lambda Y.
\ee

Finally, if $A$ is real, then $AA^*$ is real and has real eigenvalues, and hence the eigenvectors $X$ can be taken to be real (by spectral theorem for Hermitian matrices (Theorem
\ref{thm:spectral_hermitian_matrices})). The first $k$ rows of $Y$ which are determined by $X$, are real by construnction, the $m-k$ orthonormal vectors that are added may be taken to be real. Thus
all the factors can be taken to be real if $A$ is real.
\end{proof}


Every non-zero complex number $z$ has a unique ``polar representation'',
\be
z = pu,
\ee
where $p$ is a positive real number and $u$ is complex number of modulus 1. Now we generalize this property to the follow polar decomposition theorem.

\begin{theorem}[polar decomposition\index{polar decomposition}]\label{thm:polar_decomposition}
Let $A \in M_{m,n}(\C)$ with $m\leq n$. Then $A$ can be written as
\be
A = PU
\ee
where $P\in M_m(\C)$, called polar matrix\index{polar matrix}, is positive semi-definite, $\rank (P) = \rank (A)$, and $U\in M_{m,n}(\C)$ has orthonormal rows (i.e., $UU^* = I_m$).

The matrix $P$ is always uniquely determined as $P = \brb{AA^*}^{1/2}$, and $U$ is uniquely determined when $A$ has rank $m$.

If $A$ is real, then both $P$ and $U$ can be taken to be real.
\end{theorem}

\begin{remark}
Note that both factors are unique if $A$ has full rank.
\end{remark}

\begin{proof}[\bf Proof]
By Lemma \ref{lem:any_matrix_product_of_unitary_diagonal_orthonormal}, we can write
\be
A = X\Lambda Y = X\Lambda X^*X Y = PU
\ee
where $P := X\Lambda X^*$ and $U = XY$. Then $P$ is positive semi-definite since
\be
x^* P x = \brb{\Lambda^{1/2} X^*x}^*  \Lambda^{1/2} X^*x \geq 0,\qquad \forall x\neq 0.
\ee

Also, since $Y$ has orthonormal rows,
\be
UU^* = XY (XY)^* = XYY^* X^* = X I_m X^* = I_m,
\ee
so $U$ has orthonormal rows as well. By the construction in Lemma \ref{lem:any_matrix_product_of_unitary_diagonal_orthonormal}, we have
\be
P^2 = \brb{X\Lambda X^*}^2 = X\Lambda X^* X\Lambda X^* = X\Lambda X^* = AA^*.
\ee

Therefore, $P = \brb{AA^*}^{1/2}$ is the unique positive semi-definite square root matrix of $AA^*$ by Theorem \ref{thm:existence_uniqueness_kth_root_matrix_of_positive_semi-definite_matrix}.

If $A$ has rank $m$, then $P$ is full rank and thus nonsingular ($\rank\brb{AA^*} = \rank (A) =m$ by Proposition \ref{pro:rank_equalities}.(i)\&(iv), $\rank(P) = \rank\brb{AA^*}$ by Theorem
\ref{thm:existence_uniqueness_kth_root_matrix_of_positive_semi-definite_matrix}). Thus, %Thus, $Y$ is uniquely determined by Lemma \ref{lem:any_matrix_product_of_unitary_diagonal_orthonormal} and
\be
A = PU \ \ra \ U = P^{-1}A,
\ee
which means $U$ is uniquely determined as well.

However, if $\rank (A) < m$, then the row of $Y$ corresponding to the 0 eigenvalues of $P$ are not uniquely determined, so $U=XY$ need not be uniquely determined when $\rank(A) < m$.
\end{proof}


\begin{corollary}
If $A\in M_n(\C)$, then it can be written in the form
\be
A = PU
\ee
where $P$ is positive semi-definite and $U$ is unitary. The matrix $P$ is always uniquely determined as $P := \brb{AA^*}^{1/2}$.

If $A$ is nonsingular, then $U$ is uniquely determined as $U = P^{-1}A$.

If $A$ is real, then $P$ and $U$ can be taken to be real.
\end{corollary}

\begin{proof}[\bf Proof]
Direct result by letting $m=n$ from Theorem \ref{thm:polar_decomposition}.
\end{proof}


\begin{theorem}
Let $A\in M_n(\C)$ and $A = PU$ be a polar decomposition where $P\in M_n(\C)$ is the positive semi-definite matrix and $U\in M_n(\C)$ is unitary matrix. Then $A$ is normal if and only if $P$ and $U$ commute, i.e., $PU = UP$.
\end{theorem}

\begin{proof}[\bf Proof]%\footnote{proof needed.}
If $P$ and $U$ commute, then
\be
AA^* = PUU^*P^* = PP^* = P^2,\quad A^*A = U^*P^*PU = U^* P^2 U = U^*U P^2 = P^2 \ \ra \ A\text{ is normal.}
\ee

If $A$ is normal, then
\be
P^2 = PUU^*P^* = AA^* = A^*A = U^*P^*PU = U^* P^2 U.
\ee

Since $P^2$ and $U^*P^2 U$ are positive semi-definite by Corollary \ref{cor:power_of_positive_semidefinite_matrix}, both of them have the unique positive semi-definite square root matrices by
Theorem \ref{thm:existence_uniqueness_kth_root_matrix_of_positive_semi-definite_matrix}.
Thus,
\be
P = \brb{P^2}^{1/2} = \brb{U^*P^2 U}^{1/2} = U^*P U  \ \ra\ UP = PU,
\ee
as required.
\end{proof}

The next goal is to deduce the singular value decomposition (SVD) of an arbitrary (not necessarily square) matrix.

\begin{theorem}[singular value decomposition\index{singular value decomposition}]\label{thm:singular_value_decomposition}
If $A\in M_{m,n}(\C)$ has rank $k$, then it may be written in the form
\be
A = V\Sigma W^*
\ee
where $V\in M_m(\C)$ and $W\in M_n(\C)$ are unitary. The matrix $\Sigma = \brb{\sigma_{ij}} \in M_{m,n}(\C)$ has $\sigma_{ij} =0$ for all $i\neq j$, and
\be
\sigma_{11} \geq \sigma_{22} \geq \dots \geq \sigma_{kk} > \sigma_{k+1,k+1} = \dots = \sigma_{qq} = 0, \qquad  q = \min\bra{m,n}.
\ee

The numbers $\bra{\sigma_{ii}} := \bra{\sigma_i}$, called singular values of $A$, are the non-negative square roots of the eigenvalues of $AA^*$, and hence are uniquely determined.

The columns of $V$, called left singular vectors of $A$, are eigenvectors of $AA^*$. The columns of $W$, called right singular vectors of $A$, are eigenvectors of $A^*A$ (arranged in the same order
as the corresponding eigenvalues $\sigma^2_i$).

If $m\leq n$ and if $AA^*$ has distinct eigenvalues, then $V$ is determined up to a right diagonal factor $D = \diag\brb{e^{i\theta_1},\dots,e^{i\theta_n}}$ with all $\theta_i \in \R$. That is, if $A
= V_1\Sigma W_1^* = V_2 \Sigma W_2^*$, then $V_2 = V_1 D$. If $m<n$, then $W$ is never uniquely determined. If $n=m=k$ and $V$ is given, then $W$ is uniquely determined.

If $n\leq m$, the uniqueness of $V$ and $W$ is determined by considering $A^*$.

If $A$ is real, then $V$, $\Sigma$ and $W$ may all be taken to be real.
\end{theorem}

\begin{remark}
Note that the singular value decomposition is a natural generalization to arbitrary matrices of the unitary diagonalization of normal matrices. For this reason, it is often the case that facts about
eigenvalues of normal matrices generalize to statements about singular values of general matrices.
\end{remark}

\begin{proof}[\bf Proof]%\footnote{proof needed.}
We assume without loss of generality that $m\leq n$ (otherwise, we can replace $A$ by $A^*$).

Note that $AA^*$ and $A^*A$ have the same non-zero eigenvalues, so the eigenvalues are still the same when we replace $A$ by $A^*$.

If $\lm$ is a non-zero eigenvalue of $AA^*$, then
\be
AA^*x = \lm x \neq 0 \quad\text{for some }x\neq 0 \ \ra\ A^*AA^*x = \lm A^*x.
\ee

Since $A^*x \neq 0$ (otherwise, $AA^*x = 0 = \lm x$ implies a contradiction), we have that $\lm$ is also an eigenvalue of $A^*A$.

Applying Lemma \ref{lem:any_matrix_product_of_unitary_diagonal_orthonormal}, $A = X\Lambda Y$ with $X,\Lambda\in M_m(\C)$ and $Y \in M_{m,n}(\C)$. Set
\be
V:=X,\qquad \Sigma := \brb{ \Lambda | 0 }\in M_{m,n}(\C),\qquad W:= \brb{Y^*| S^*}\in M_n(\C),
\ee
by requiring that the columns of $W$ be an orthonormal set in $\C^n$. The columns of $Y^*$ are already orthonormal by Lemma \ref{lem:any_matrix_product_of_unitary_diagonal_orthonormal}. So if $m<n$,
the columns of $S^* \in M_{n,n-m}(\C)$ may be chosen (but not uniquely) to make $W$ be unitary.

%Also, by Lemma \ref{lem:any_matrix_product_of_unitary_diagonal_orthonormal}, if $AA^*$ has distinct eigenvalues, $V$ is determined up to a right diagonal factor $D = \diag\brb{e^{i\theta_1},\dots,e^{i\theta_n}}$ with all $\theta_i \in \R$.

It is immediate that
\be
V \Sigma W^* = X \bepm \Lambda & 0 \eepm \bepm Y \\ S \eepm = X\Lambda Y = A.
\ee

The statements about uniqueness follow from Lemma \ref{lem:any_matrix_product_of_unitary_diagonal_orthonormal}.
Also, $V$, $\Sigma$ and $W$ may all be taken to be real by Lemma \ref{lem:any_matrix_product_of_unitary_diagonal_orthonormal}.
\end{proof}

\subsection{Optimal properties of positive definite matrices}

\begin{proposition}\label{pro:hermitian_max_largest_eigenvalue}
Let $A\in M_n(\F)$ be a Hermitian matrix. Then
\be
\max_{x\neq 0, x\in \F^n} \frac{x^*A x}{x^*x} = \lm_1
\ee
where $\lm_1$ is the largest eigenvalue of $A$ and the maximum is attained at any eigenvector of $A$ corresponding to $\lm_1$.
\end{proposition}

\begin{proof}[\bf Proof]
By spectral theorem of Hermitian (Theorem \ref{thm:spectral_hermitian_matrices}) we can write $A = U^* \Lambda U$, where $U$ is unitary matrix and $\Lambda = \diag\brb{\lm_1,\dots,\lm_n}$, as usual, $\lm_1 \geq \dots \geq \lm_n$ since all eigenvalues are real. Then for $x\neq 0$ for $x\in \F^n$,
\beast
\frac{x^* A x}{x^*x} & = & \frac{x^*U^* \Lambda U x}{x^*U^* Ux} = \frac{y^*\Lambda y}{y^*y} = \frac{\lm_1 \abs{y_1}^2 + \dots + \lm_n \abs{y_n}^2}{\abs{y_1}^2 + \dots + \abs{y_n}^2} \\
& \leq & \frac{\lm_1 \abs{y_1}^2 + \dots + \lm_1 \abs{y_n}^2}{\abs{y_1}^2 + \dots + \abs{y_n}^2} = \lm_1
\eeast
where $y = Ux$. Therefore,
\be
\max_{x\neq 0, x\in \F^n} \frac{x^*A x}{x^*x} = \lm_1.
\ee

Clearly, if $x$ is an eigenvector corresponding to $\lm_1$, then $\frac{x^*Ax}{x^*x} = \lm_1$ and the result is proved.
\end{proof}


\begin{proposition}\label{pro:hermitian_positive_definite_ratio_max_largest_eigenvalue}
Let $A\in M_n(\F)$ be a Hermitian matrix and $B \in M_n(\F)$ be positive definite. Then
\be
\max_{x\neq 0, x\in \F^n}\frac{x^*A x}{x^* B x} = \mu
\ee
where $\mu$ is the largest eigenvalue of $AB^{-1}$.
\end{proposition}

\begin{proof}[\bf Proof]
Since $B$ is positive definite, its inverse exists and we can find invertible matrix $C$ such that $B = C^*C$ by Theorem \ref{thm:positive_definite_product_nonsingular_matrix}. We have that
\be
\max_{x\neq 0, x\in \F^n}\frac{x^*A x}{x^* B x} = \max_{x\neq 0, x\in \F^n}\frac{x^*A x}{x^* C^* C x}.
\ee

Let $y = Cx$ and then $x = C^{-1}y$. Therefore,
\be
\max_{x\neq 0, x\in \F^n}\frac{x^*A x}{x^* C^* C x} = \max_{x\neq 0, x\in \F^n}\frac{y^*\brb{C^*}^{-1} A C^{-1}y}{y^* y}.
\ee

Thus, by Proposition \ref{pro:hermitian_max_largest_eigenvalue}, the last expression equals the maximum eigenvalue of $\brb{C^*}^{-1} A C^{-1}$. However, $\brb{C^*}^{-1} A C^{-1}$ and $A C^{-1}\brb{C^*}^{-1} = A\brb{C^*C}^{-1} = AB^{-1}$ have the same eigenvalues (Theorem \ref{thm:product_matrices_change_order_have_the_same_eigenvalues} and the proof is complete.

Note that the maximum is attained at $C^{-1}y$ where $y$ is any eigenvector of $\brb{C^*}^{-1} A C^{-1}$ corresponding to the eigenvalue $\mu$.
\end{proof}


\begin{proposition}
Let $B\in M_n(\F)$ be a positive definite matrix and $y\in \F^n$. Then
\be
\max_{x\neq 0,x\in \F^n} \frac{\brb{x^* y}^2}{x^* B x} = y^* B^{-1} y.
\ee
\end{proposition}

\begin{remark}
Note that the statement can be written as
\be
\brb{x^*y}^2 \leq \brb{x^* B x}\brb{y^* B^{-1}y}
\ee
for any positive definite matrix $B$. An alternative proof of above inequality can be given by using Cauchy-Schwartz inequality:
\be
\brb{\sum_{i} u_i v_i}^2 \leq \brb{\sum_i u_i^2} \brb{\sum_i v_i^2}.
\ee
\end{remark}

\begin{proof}[\bf Proof]
We have
\be
\max_{x\neq 0,x\in \F^n} \frac{\brb{x^* y}^2}{x^* B x} = \max_{x\neq 0,x\in \F^n} \frac{x^* y y^* x}{x^* B x}
\ee
which is the largest eigenvalue of $yy^* B^{-1}$ by Proposition \ref{pro:hermitian_positive_definite_ratio_max_largest_eigenvalue}. Again, the eigenvalues of $yy^* B^{-1}$ and $\brb{C^*}^{-1} yy^* C^{-1}$ are equal with the same argument in Proposition \ref{pro:hermitian_positive_definite_ratio_max_largest_eigenvalue}. Since the latter matrix is symmetric and has rank 1, it has only one non-zero eigenvalue, counting multiplicity (see Proposition \ref{pro:matrix_rank_inequalities_product}.(i)). Thus, by Proposition \ref{pro:trace_change_order} the eigenvalue equals
\be
\tr\brb{\brb{C^*}^{-1} yy^* C^{-1}} = \tr\brb{\brb{y^* C^{-1}C^*}^{-1} y} = \tr\brb{y^* B^{-1} y}.
\ee
which is a positive real number as $B^{-1}$ is also positive definite (see Proposition \ref{pro:positive_definite_matrix_inverse}).\footnote{Note that $y$ can be 0.} Thus, the proof is complete.
\end{proof}


\section{Advanced Matrix Operations}


\subsection{Kronecker Product}\label{sec:kronecker_product}

%\subsection{Motivation}

%\subsection{Kroneckor product}

\begin{definition}[Kronecker product\index{Kronecker product}]\label{def:kronecker_product}
Let $A\in M_{m,n}(\F)$ and $B \in M_{p,q}(\F)$. Then the Kronecker product $A\times B$ is the $mp \times nq$ block matrix,
\be
A \otimes B = \bepm
a_{11}B & \dots & a_{1n}B \\
\vdots & \ddots & \vdots \\
a_{m1}B & \dots & a_{mn}B
\eepm,
\ee
more explicitly
\be
A \otimes B = \bepm
a_{11}b_{11} & a_{11}b_{12} & \dots & a_{11}b_{1q} & \dots & \dots & a_{1n}b_{11} & a_{1n}b_{12} & \dots & a_{1n}b_{1q}\\
a_{11}b_{21} & a_{11}b_{22} & \dots & a_{11}b_{2q} & \dots & \dots & a_{1n}b_{21} & a_{1n}b_{22} & \dots & a_{1n}b_{2q}\\
\vdots & \vdots & \ddots & \vdots  & & & \vdots & \vdots & \ddots & \vdots \\
a_{11}b_{p1} & a_{11}b_{p2} & \dots & a_{11}b_{pq} & \dots & \dots & a_{1n}b_{p1} & a_{1n}b_{p2} & \dots & a_{1n}b_{pq}\\
\vdots & \vdots &  & \vdots  & \ddots & & \vdots & \vdots & \ddots & \vdots \\
\vdots & \vdots &  & \vdots  & & \ddots & \vdots & \vdots & \ddots & \vdots \\
a_{m1}b_{11} & a_{m1}b_{12} & \dots & a_{m1}b_{1q} & \dots & \dots & a_{mn}b_{11} & a_{mn}b_{12} & \dots & a_{mn}b_{1q}\\
a_{m1}b_{21} & a_{m1}b_{22} & \dots & a_{m1}b_{2q} & \dots & \dots & a_{mn}b_{21} & a_{mn}b_{22} & \dots & a_{mn}b_{2q}\\
\vdots & \vdots & \ddots & \vdots  & & & \vdots & \vdots & \ddots & \vdots \\
a_{m1}b_{p1} & a_{m1}b_{p2} & \dots & a_{m1}b_{pq} & \dots & \dots & a_{mn}b_{p1} & a_{mn}b_{p2} & \dots & a_{mn}b_{pq}\\
\eepm
\ee
\end{definition}

\begin{remark}
Obviously,
\be
I_m \otimes I_n = I_{mn}.
\ee
\end{remark}

\begin{example}
\be
\bepm 1 & 2 \\ 3 & 4 \eepm \otimes \bepm 0 & 5 \\ 6 & 7 \eepm = \bepm 0 & 5 & 0 & 10 \\ 6 & 7 & 12 & 14 \\ 0 & 15 & 0 & 20 \\ 18 & 21 & 24 & 28 \eepm.
\ee
\end{example}

The Kronecker prodcut is a special case of the tensor product, so it is bilinear and associative.

\begin{proposition}[Bilinearity and associativity of Kronecker product]\label{pro:bilinearity_associativity_kronecker_product}% between Kronecker product and other matrix operations]
Suppose that $A,B,C$ are matrices (having same size according to the corresponding case) and $\alpha$ is a scalar.
\ben
\item [(i)] $A\otimes (B+C) = A\otimes B + A \otimes C$.
\item [(ii)] $(A+B)\otimes C  =  A \otimes C + B \otimes C $.
\item [(iii)] $\alpha \otimes A = \alpha A = A\alpha = A \otimes \alpha$.
\item [(iv)] $(\alpha A)\otimes B = A \otimes (\alpha B) = \alpha(A\otimes B)$.
\item [(v)] $(A\otimes B)\otimes C = A\otimes (B \otimes C)$.
\item [(vi)] If $A$ is a partitioned matrix
\be
A = \bepm A_{11} & A_{12} \\ A_{21} & A_{22} \eepm,
\ee
then $A\otimes B$ takes the form
\be
A\otimes B = \bepm A_{11} \otimes B & A_{12} \otimes B \\ A_{21} \otimes B & A_{22} \otimes B \eepm.
\ee
\een
\end{proposition}

\begin{proof}[\bf Proof]
\footnote{proof needed.}
\end{proof}

\begin{proposition}[mixed-product of Kronecker product]\label{pro:kronecker_product_mixed_product}
Let $A \in M_{m,n}(\F),B\in M_{p,q}(\F),C\in M_{n,k}(\F), D \in M_{q,r}(\F)$. Then %If $A,B,C,D$ are matrices of such size that one can form the matrix products $AC$ and $BD$, then
\be
\brb{A\otimes B} \brb{C\otimes D} = (AC) \otimes (BD).
\ee
\end{proposition}

\begin{proof}[\bf Proof]
The $((i-1)p + s, (j-1)r + t)$ element of $(AC)\otimes(BD)$ is
\be
(AC)_{i,j}BD_{s,t} = \sum^{n}_{u=1} a_{iu}c_{uj} \sum^{q}_{v=1} b_{sv}d_{vt}.
\ee

Also, the $((i-1)p + s, (j-1)r + t)$ element of $(A\otimes B)(C\otimes D)$ is
\beast
\sum^{nq}_{w=1} (A\otimes B)_{(i-1)p + s,w} (C\otimes D)_{w,(j-1)r + t} & = & \sum_{1\leq u\leq n,1\leq v\leq q} (A\otimes B)_{(i-1)p + s,(u-1)q+v} (C\otimes D)_{(u-1)q+v,(j-1)r + t} \\
& = &  \sum^n_{u=1}\sum^q_{v=1} a_{iu} b_{sv} c_{uj}d_{vt} = \sum^{n}_{u=1} a_{iu}c_{uj} \sum^{q}_{v=1} b_{sv}d_{vt}
\eeast
as required.
\end{proof}

\begin{proposition}[inverse of Kronecker product]\label{pro:kronecker_product_inverse}
Let $A,B$ be two square matrices. Then their Kronecker product $A\otimes B$ is invertible if and only if both $A$ and $B$ are invertible, in which case the inverse is given by
\be
\brb{A\otimes B}^{-1} = A^{-1} \otimes B^{-1}.
\ee
\end{proposition}

\begin{proof}[\bf Proof]
By Definition of matrix inverse (Proposition \ref{pro:inverse_matrix})
\beast
\brb{A\otimes B} \brb{A^{-1} \otimes B^{-1}} & = &  AA^{-1} \otimes BB^{-1} = I_n\otimes I_m = I_{mn}  \\
\brb{A^{-1} \otimes B^{-1}} \brb{A\otimes B} & = &  A^{-1} A \otimes B^{-1} B = I_n \otimes I_m = I_{mn} .
\eeast
\end{proof}


\begin{proposition}[transpose and adjoint matrices of Kronecker product]\label{pro:kronecker_product_transpose_adjoint}
The transpose matrix and adjoint matrix are distributive over the Kronecker product
\be
\brb{A\otimes B}^T = A^T \otimes B^T,\qquad \brb{A\otimes B}^* = A^* \otimes B^*.
\ee
\end{proposition}

\begin{proof}[\bf Proof]
By definition,
\be
(A \otimes B)^T = \bepm
a_{11}B & \dots & a_{1n}B \\
\vdots & \ddots & \vdots \\
a_{m1}B & \dots & a_{mn}B
\eepm^T = \bepm
a_{11}B^T & \dots & a_{m1}B^T \\
\vdots & \ddots & \vdots \\
a_{1n}B^T & \dots & a_{mn}B^T
\eepm = A^T\otimes B^T,
\ee
as required. It is similar for adjoint matrix.
\end{proof}

\subsection{Eiegenvalues of Kronecker product}

\begin{theorem}\label{thm:kronecker_product_eigenvalue}
Let $A\in M_n(\F)$ with eigenvalues $\lm_1,\dots,\lm_n$ and $B\in M_m(\F)$ with eigenvalues $\mu_1,\dots,\mu_m$. Then the $mn$ eigenvalues of $A\otimes B$ are $
\lm_i\mu_j$ ($i = 1,\dots,n,j = 1,\dots,m$).
\end{theorem}

\begin{proof}[\bf Proof]
By Schur's unitary triangularization theorem \ref{thm:schur_unitary_triangularization} there exist unitary matrices $U,V$ such that
\be
U^*AU = S,\quad V^*BV = T,
\ee
where $S,T$ are upper triangular matrices whose diagonal elements are the eigenvalues of $A$ and $B$ respectively. Thus, by Proposition \ref{pro:kronecker_product_mixed_product}, \ref{pro:kronecker_product_inverse}
\beast
S \otimes T & = & \brb{U^*AU}\otimes \brb{V^*BV} = \brb{U^*\otimes V^*}\brb{A\otimes B}\brb{U\otimes V} \\
& = & \brb{U^{-1}\otimes V^{-1}}\brb{A\otimes B}\brb{U\otimes V} = \brb{U\otimes V}^{-1}\brb{A\otimes B}\brb{U\otimes V}.
\eeast

Therefore, the similar matrices $S \otimes T$ and $A\otimes B$ have the same eigenvalues by Corollary \ref{cor:similarity_same_eigenvalues}. Then we can have $S\otimes T$ is an upper triangular matrix since $S$ and $T$ are upper triangular. Hence, its eigenvalues are its diagonal elements $\lm_i\mu_j$.
\end{proof}

\begin{example}
If $x$ is an eigenvector of $A$ and $y$ is an eigenvector of $B$, then $x\otimes y$ is clearly an eigenvector of $A\otimes B$. It is not generally true, however, that every eigenvector of $A\otimes B$ is the Kronecker product of an eigenvector of $A$ and an eigenvector of $B$. For example,
\be
A = B = \bepm 0 & 1 \\ 0 & 0 \eepm,\qquad e_1 = \bepm 1\\ 0 \eepm,\qquad e_2 = \bepm 0 \\ 1\eepm.
\ee

Both eigenvalues of $A$ (and $B$) are zero and the only eigenvector is $e_1$. The four eigenvalues of $A\otimes B$ are all zeros by Theorem \ref{thm:kronecker_product_eigenvalue}, but the eigenvectors of $A\otimes B$ are not just $e_1 \otimes e_1$, but also $e_1\otimes e_2$ and $e_2\otimes e_2$.
\end{example}


\begin{corollary}\label{cor:kronecker_product_positive_definite_matrices_is_positive_definite}
Let $A\in M_n(\F)$ and $B\in M_m(\F)$ be two positive semi-definite (definite, respectively) matrices, then $A\otimes B$ is also positive semi-definite (definite, respectively).
\end{corollary}

\begin{proof}[\bf Proof]
First, note that $A\otimes B$ is Hermitian
\be
\brb{A\otimes B}^* = A^* \otimes B^* = A\otimes B.
\ee
by Proposition \ref{pro:kronecker_product_transpose_adjoint}. Then we can have required result by Theorem \ref{thm:positive_definite_matrix_iff_positive_eigenvalue} and \ref{thm:kronecker_product_eigenvalue}.%footnote{positive semi-definite eigenvalues non-negative.}
\end{proof}

\begin{corollary}[determinant of Kronecker product]
Let $A\in M_n(\F)$ and $B\in M_m(\F)$. Then the determinant of their Kronecker product is
\be
\det (A\otimes B) = \brb{\det A}^m \brb{\det B}^n.
\ee
\end{corollary}

\begin{proof}[\bf Proof]
By Theorem \ref{thm:elementary_symmetric_function_sum_of_determinant_principal_submatrix_equivalent}, the determinant of $A\otimes B$ is equal to the product of its eigenvalues. Then we have the required result by Theorem \ref{thm:kronecker_product_eigenvalue}.
\end{proof}



\begin{corollary}
Let $A\in M_n(\F)$ and $B\in M_m(\F)$. Then the trace of $A\otimes B$ is the product of traces of $A$ and $B$. That is,
\be
\tr(A\otimes B) = \tr A \tr B.
\ee
\end{corollary}

\begin{proof}[\bf Proof]
By Theorem \ref{thm:kronecker_product_eigenvalue}, the eigenvalues of $A\otimes B$ are $\lm_i\mu_j$ where $\lm_i$ and $\mu_j$ are eigenvalues of $A$ and $B$ respectively. Then by Theorem \ref{thm:elementary_symmetric_function_sum_of_determinant_principal_submatrix_equivalent},
\be
\tr(A\otimes B) = \sum^n_{i=1}\sum^m_{j=1}\lm_i\mu_j = \sum^n_{i=1}\lm_i\sum^m_{j=1}\mu_j = \tr A \tr B.
\ee% Then we can have required result by Theorem \footnote{positive semi-definite eigenvalues non-negative.}
\end{proof}



\begin{corollary}
Let $A\in M_n(\F)$ and $B\in M_m(\F)$. Then the rank of their Kronecker product is the product of their ranks. That is,
\be
\rank\brb{A\otimes B} = \rank(A)\rank(B).
\ee
\end{corollary}

\begin{proof}[\bf Proof]
By Proposition \ref{pro:rank_equalities}.(iv),%\footnote{rank prop, r(AA*) = r(A)},
\be
\rank\brb{A\otimes B} = \rank\brb{\brb{A\otimes B}\brb{A\otimes B}^*} = \rank\brb{\brb{A\otimes B}\brb{A^*\otimes B^*}} = \rank\brb{AA^* \otimes BB^*}
\ee

Thus, the rank of $AA^* \otimes BB^*$ (positive semi-definite matrix) equals the number of positive eigenvalues by Proposition \ref{pro:rank_equalities}.(ii). According to Theorem \ref{thm:kronecker_product_eigenvalue}, the eigenvalues of $AA^*\otimes BB^*$ are $\lm_i\mu_j$, where $\lm_i$ are the eigenvalues of $AA^*$ and $\mu_j$ are the eigenvalues of $BB^*$. Then $\lm_i\mu_j$ is non-zero if and only if both $\lm_i$ and $\mu_j$ are non-zero. Hence, the number of non-zero eigenvalues of $AA^*\otimes BB^*$ is the product of the number of non-zero eigenvalues of $AA^*$ and the number of non-zero eigenvalues of $BB^*$. Thus the rank of $AA^*\otimes BB^*$ is the product of the ranks of $AA^*$ and $BB^*$. Then
\be
\rank\brb{A\otimes B} = \rank\brb{\brb{A\otimes B}\brb{A\otimes B}^*} = \rank\brb{AA^*}\rank\brb{BB^*} = \rank\brb{A}\rank\brb{B}
\ee
by Proposition \ref{pro:rank_equalities}.(iv).%\footnote{rank prop, r(AA*) = r(A)}.
\end{proof}



%\begin{proposition}
%Let $A\in M_m(\F)$ and $B\in M_n(\F)$. Then the determinant of their Kronecker product is
%\be
%\det\brb{A\otimes B} = \brb{\det A}^n \brb{\det B}^m.
%\ee
%\end{proposition}

%\begin{proof}[\bf Proof]
%\footnote{proof needed.}
%\end{proof}


\subsection{Hadamard Product}

%\subsection{Definition and basic properties}

\begin{definition}[Hadamard (or entrywise) product of matrices]
Let $A,B\in M_{m,n}(\F)$ be two matrices of the same size. Then Hadamard product\index{Hadamard product} $C$ of $A$ and $B$ is defined by
\be
c_{ij} = a_{ij}b_{ij},\qquad i=1,\dots,m,\ j = 1,\dots,n.
\ee

It is also called Schur product\index{Schur product} and entrywise product\index{entrywise product}, denoted by $C= A\odot B$.
\end{definition}

The following perperties are immediate consequences of the definition.

\begin{proposition}\label{pro:hadamard_product_basic_properties}
For $A,B,C,D\in M_{m,n}(\C)$,
\ben
\item [(i)] $A\odot B = B \odot A$.
\item [(ii)] $\brb{A\odot B}^* = A^* \odot B^*$.
\item [(iii)] $(A\odot B)\odot C = A\odot (B\odot C)$.
\item [(iv)] $(A+B)\odot (C+D) = A\odot C+A\odot D+B\odot C+B\odot D$.
\item [(v)] $A\odot I = \diag(A)$.
\een
\end{proposition}


\begin{lemma}\label{lem:hadamard_product_commute}
Let $A,B\in M_{m,n}(\F)$. Also, $C\in M_m(\F)$ and $D \in M_n(\F)$ are diagonal. Then
\be
C\brb{A\odot B} D = (CAD)\odot B = (CA)\odot(BD) = (AD)\odot (CB) = A\odot (CBD).
\ee
\end{lemma}

\begin{proof}[\bf Proof]
The $(i,j)$-entry of $C\brb{A\odot B} D$ is
\be
c_{ii}\brb{\brb{A\odot B}D}_{ij} = c_{ii}\brb{A\odot B}_{ij}d_{jj} = c_{ii} a_{ij} b_{ij}d_{jj}.
\ee

The $(i,j)$-entry of $(CAD)\odot B$ is
\be
(CAD)_{ij}b_{ij} = c_{ii} (AD)_{ij}b_{ij} = c_{ii}a_{ij}d_{jj} b_{ij}.
\ee

The $(i,j)$-entry of $(CA)\odot(BD)$ is
\be
(CA)_{ij}(BD)_{ij} = c_{ii}a_{ij}b_{ij} d_{jj}.
\ee

The $(i,j)$-entry of $(AD)\odot (CB)$ is
\be
(AD)_{ij}(CB)_{ij} = a_{ij}d_{jj}c_{ii} b_{ij}.
\ee

The $(i,j)$-entry of $A\odot (CBD)$ is
\be
a_{ij}(CBD)_{ij} = a_{ij} c_{ii}b_{ij} d_{jj}.
\ee

Thus, we have the required results.
\end{proof}


%\subsection{Schur product theorem}

\begin{lemma}\label{lem:hadamard_product_diagonal_entry_equals_vector_entry}
Let $A,B\in M_{m,n}(\F)$ and $x\in \F^n$. Then the $i$th diagonal entry of the matrix $AD_xB^T$ coincides with the $i$th entry of the vector $(A\odot B)x$ for $i = 1,\dots,m$.
\end{lemma}

\begin{proof}[\bf Proof]
We have that for $i=1,\dots,m$,
\be
\brb{AD_xB^T}_{ii} = \sum^n_{j=1} a_{ij} \brb{D_x B^T}_{ji} = \sum^n_{j=1} a_{ij} \brb{D_x}_{jj}\brb{B^T}_{ji} = \sum^n_{j=1} a_{ij} x_j b_{ij} = \sum^n_{j=1} a_{ij}b_{ij} x_j = \brb{(A\odot B)x}_i.
\ee
\end{proof}


\begin{lemma}\label{lem:hadamard_product_with_vectors}
Let $A,B\in M_{m,n}(\F)$. Also, $x\in \F^n$ and $y\in \F^m$. Then
\be
y^* (A\odot B)x = \tr\brb{\ol{D_y} A D_x B^T} = \tr\brb{\ol{D_y} B D_x A^T} .
\ee
\end{lemma}

\begin{proof}[\bf Proof]
First, we have
\be
y^* (A\odot B)x = \onevec_{1\times m} \ol{D_y} (A\odot B)x = \onevec_{1\times m} \brb{(\ol{D_y}A)\odot B}x
\ee
by Lemma \ref{lem:hadamard_product_commute}. Then, by Lemma \ref{lem:hadamard_product_diagonal_entry_equals_vector_entry},
\be
\onevec_{1\times m} \brb{(\ol{D_y}A)\odot B}x = \sum^m_{i} \brb{\brb{(\ol{D_y}A)\odot B}x}_i =  \sum^m_{i} \brb{\ol{D_y}A D_x B^T}_{ii} = \tr\brb{\ol{D_y}A D_x B^T }.
\ee

We can have the second equality by switching the order of $A$ and $B$.
\end{proof}


\begin{theorem}[Schur product theorem\index{Schur product theorem}]\label{thm:schur_product}%{thm:entrywise_product_of_positive_semidefinite_is_positive_semidefinite}
Hadamard (or entrywise) product of positive definite (semi-definite) matrices is also positive definite (semi-definite).

That is, let $A,B\in M_{m,n}(\C)$ be two positive definite (semi-definite) matrices. Then $A\odot B$ is also a positive definite (semi-definite).
\end{theorem}

\begin{proof}[\bf Proof]
{\bf Approach 1.} By Lemma \ref{lem:hadamard_product_with_vectors}, for any matriice $A,B\in M_{m,n}(\C)$, $x\in \C^n$ and $y = \C^m$,
\be
y^*\brb{A\cdot B}x = \tr\brb{\ol{D}_y B D_x A^T} = \tr\brb{A^T\ol{D}_y B D_x }.
\ee

The last equality is due to Proposition \ref{pro:trace_change_order}. Since $A$ and $B$ are positive definite (semi-definite), $A^{1/2}$ and $B^{1/2}$ exist and are unique and Hermitian by Theorem \ref{thm:existence_uniqueness_kth_root_matrix_of_positive_semi-definite_matrix}. Thus,
\beast
\tr\brb{A^T\ol{D}_y B D_x } & = & \tr\brb{\ol{A}\ol{D}_y B D_x }  = \tr\brb{\ol{A}^{1/2}\ol{A}^{1/2}\ol{D}_y B^{1/2}B^{1/2} D_x } = \tr\brb{\ol{A}^{1/2}\ol{D}_y B^{1/2}B^{1/2} D_x \ol{A}^{1/2}} \\
& = & \tr\brb{\ol{A^*}^{1/2}D_{\ol{y}} \brb{B*}^{1/2}B^{1/2} D_x \ol{A}^{1/2}} =  \tr\brb{\brb{\ol{A}^{1/2}}^* \brb{D_{y}}^* \brb{\brb{B}^{1/2}}^*B^{1/2} D_x \ol{A}^{1/2}}
\eeast
by Proposition \ref{pro:trace_change_order} and the fact that $\brb{M^*}^{1/k} = \brb{M^{1/k}}^*$ for positive definite (semi-definite) matrix $M$. Then for $x= y$, this is
\be
\tr\brb{C^*C},\qquad \text{where } C = B^{1/2} D_x \ol{A}^{1/2}
\ee
which is strictly positive (positive for semi-definite case) for $C\neq 0$, which occurs if and only if $X\neq 0$. This shows that $\brb{A\odot B}$ is positive definite (semi-definite).

{\bf Approach 2.} Kronecker product of positive definite (semi-definite matrices is still positive definite (semi-definite) matrix by Corollary \ref{cor:kronecker_product_positive_definite_matrices_is_positive_definite}. Also, any principal submatrix of positive definite (semi-definite) matrix is still positive definite (semi-definite) by Proposition \ref{pro:positive_definite_principal_submatrix_is_positive_definite}.

Then we can conclude that Hadamard product of positive definite (semi-definite matrices is the principal submatrix of Kronecker product of these matrices.
\end{proof}


\subsection{Vectorization Operator}


\begin{definition}[vectorization operator\index{vectorization operator}]
Let $A\in M_{m,n}(\F)$ and $A_i$ its $i$th column. Then we define the vectorization operator $\vec: M_{m,n}(\F) \to M_{mn,1}(\F)$ by
\be
\vec A = \bepm A_1 \\ A_2 \\ \vdots \\ A_n \eepm
\ee
which is $mn\times 1$ vector.
\end{definition}



\begin{proposition}\label{pro:vectorization_operator_basic_properties}
\ben
\item [(i)] Let $A,B\in M_{m,n}(\F)$. Then
\be
\vec\brb{A+B} = \vec A + \vec B.
\ee

%\item [(i)] Let $A,B\in M_{m,n}(\F)$ and $C\in M_{n,p}(\F)$. Then
%\be
%\vec\brb{(A+B)C} = \vec AC + \vec B.
%\ee

\item [(ii)] A very simple but often useful property is
\be
\vec a^T = \vec a = a
\ee
for any column vector $a$.

\item [(iii)] The basic connection between the vectorization operator and the Kronecker product is
\be
\vec \brb{ab^T} = b\otimes a
\ee
for any two column vectors $a$ and $b$ (not necessarily of the same order).

\item [(iv)] The basic connection between the vectorization operator and the trace is
\be
\brb{\vec A}^T \vec B = \tr \brb{A^T B} = \sum_i\sum_j a_{ij}b_{ij}
\ee
where $A$ and $B$ are matrices of the same order.
\een
\end{proposition}

\begin{theorem}\label{thm:vectorization_product_expression}
Let $A\in M_{m,n}(\F), B \in M_{n,p}(\F), C \in M_{p,q}(\F)$. Then
\be
\vec\brb{ABC} = \brb{C^T \otimes A}\vec B
\ee
which has size $M_{mq\times 1} = M_{mq\times np} M_{np\times 1}$.
\end{theorem}

\begin{remark}
The special case is $AB = IAB =AIB = ABI$ for $A\in M_{m,n}(\F), B\in M_{n,p}(\F)$, which gives
\be
\vec(AB) = (B^T \otimes I_m)\vec A = (B^T \otimes A) \vec I_n = (I_p \otimes A)\vec B.
\ee
\end{remark}

\begin{proof}[\bf Proof]
Assume that $B$ has $p$ columns denoted $B_1,\dots,B_p$. Similarly let $e_1,\dots,e_p$ denote the columns of identity matrix $I_p$. Thus,
\be
B = \sum^p_{k=1} B_k e_k^T.
\ee%since $\vec\brb{ab^T} = b\otimes a$ for column vectors $a,b$,

Then by Proposition \ref{pro:vectorization_operator_basic_properties} and \ref{pro:kronecker_product_mixed_product} we have
\beast
\vec \brb{ABC} & = & \vec\brb{A\sum^p_{k=1} B_k e_k^T C} = \sum^p_{k=1} \vec\brb{ A B_k e_k^T C} = \sum^p_{k=1}\brb{e_k^T C}^T \otimes \brb{ A B_k } = \sum^p_{k=1}\brb{C^Te_k} \otimes \brb{ A B_k } \\
& = & \sum^p_{k=1}\brb{C^T \otimes A}\brb{e_k \otimes B_k } = \brb{C^T \otimes A} \sum^p_{k=1}\brb{e_k \otimes B_k } = \brb{C^T \otimes A} \sum^p_{k=1}\vec\brb{B_k e_k^T }= \brb{C^T \otimes A}\vec B.
\eeast
\end{proof}

\begin{proposition}
Let $A\in M_{m,n}(\F), B \in M_{n,p}(\F), C \in M_{p,q}(\F), D\in M_{q,m}(\F)$ with a square product $ABCD$. Then
\be
\tr\brb{ABCD} = \brb{\vec (D^T)}^T \brb{C^T \otimes A}\vec B = \brb{\vec (D)}^T \brb{A \otimes C^T }\vec (B^T)
\ee
which has size $M_{1\times 1} = M_{1\times qm} M_{qm\times pn}M_{pn \times 1}$.
\end{proposition}

\begin{proof}[\bf Proof]
By Proposition \ref{pro:trace_change_order}, Proposition \ref{pro:vectorization_operator_basic_properties}.(iv) and Theorem \ref{thm:vectorization_product_expression},
\be
\tr\brb{ABCD} = \tr\brb{DABC} = \brb{\vec (D^T)}^T \vec(ABC) = bb{\vec (D^T)}^T \brb{C^T \otimes A}\vec B .
\ee

The second equality is proved in the same way starting from $\tr\brb{ABCD} = \tr\brb{D^T\brb{C^TB^TA^T}}$.
\end{proof}

\begin{example}
For $A,B,V\in M_n(\F)$,
\be
\brb{\vec V}^T \brb{A\otimes B} \vec V = \tr\brb{BVA^TV^T} = \tr\brb{VAV^TB^T} = \tr\brb{AV^TB^TV} = \brb{\vec V}^T(A \otimes B) \vec V.
\ee
\end{example}

\subsection{Moore-Penrose generalized inverse}

\begin{definition}[Moore-Penrose generalized inverse\index{Moore-Penrose generalized inverse}]\label{def:moore_penrose_generalized_inverse}
Let $A\in M_{m,n}(\F)$ and $X\in M_{n,m}(\F)$. Then $X$ is called the Moore-Penrose (MP) generalized inverse of $A$ if
\ben
\item [(i)] $AXA = A$.
\item [(ii)] $XAX = X$.
\item [(iii)] $(AX)^* = AX$.
\item [(iv)] $(XA)^* = XA$.
\een

We shall denote MP pseudoinverse as $A^\mpi$.
\end{definition}



\begin{theorem}
For each $A\in M_{m,n}(\F)$, there exists a unique Moore-Penrose generalized inverse $A^\mpi$ .
\end{theorem}

\begin{proof}[\bf Proof]
{\bf Uniqueness}. Assume that two matrices $B$ and $C$ both satisfy the condition of definition of Moore-Penrose generalized inverse. Then
\be
AB = (AB)^* = B^* A^* = B^* (ACA)^* = B^* A^* C^* A^* = (AB)^*(AC)^* = ABAC = AC.
\ee

Similarly,
\be
BA = (BA)^* = A^* B^* = (ACA)^* B^* = A^* C^* A^* B^* = (CA)^* (BA)^* = CABA = CA.
\ee

Hence,
\be
B = BAB = BAC = CAC = C.
\ee

{\bf Existence}. Let $\rank(A) = r$. If $r=0$, then $A = 0$ and $A^\mpi = 0$ satisfies all four conditions of Moore-Penrose generalized inverse. Thus we consider the case $r>0$.

According to singular value decomposition (Theorem \ref{thm:singular_value_decomposition}) %\footnote{theorem needed.},
\be
A = U \Sigma V^*
\ee
where $U\in M_m(\C),V\in M_n(\C)$ are unitary matrices and $\Sigma = (\sigma_{ij})\in M_{m,n}(\C)$ with $\sigma_{ij}=0$ for all $i\neq j$ and $\sigma_{11}\geq \sigma_{22} \geq \dots \geq \sigma_{rr} > \sigma_{r+1,r+1} \dots = 0$. $A$ can also be expressed by
\be
A = U\bepm \Lambda & 0_{r\times (n-r)} \\ 0_{(m-r)\times r} & 0_{(m-r)\times (n-r)} \eepm V^* = U\bepm I_r  \\ 0_{(m-r)\times r} \eepm \Lambda \bepm I_r \ & 0_{r\times (n-r)} \eepm  V^*.
\ee

Then letting
\be
S = U\bepm I_r  \\ 0_{(m-r)\times r} \eepm,\quad T = V\bepm I_r \\  0_{(n-r)\times r} \eepm \quad \ra\quad S^*S = T^*T = I_r
\ee

So we have $A = S \Lambda T^*$. Then we can define
\be
B := T\Lambda^{-1} S^*.
\ee

Therefore,
\beast
ABA & = & S \Lambda T^* T\Lambda^{-1} S^* S \Lambda T^* = S \Lambda T^* = A,\\
BAB & = & T\Lambda^{-1} S^* S \Lambda T^* T\Lambda^{-1} S^* = T\Lambda^{-1} S^* = B,\\
(AB)^* & = & B^* A^* = S\Lambda^{-1} T^* T \Lambda S^* = SS^* \text{ (Hermitian)}, \\
(BA)^* & = & A^* B^* = T \Lambda S^*S\Lambda^{-1} T^* = TT^* \text{ (Hermitian)}.
\eeast

Hence, $B$ is the unique Moore-Penrose generalized inverse of $A$.
\end{proof}

\begin{proposition}\label{pro:mp_inverse_basic_properties}
Let $A\in M_{m,n}(\C)$. Then
\ben
\item [(i)] $A^\mpi = A^{-1}$ for nonsingular $A$.
\item [(ii)] $\brb{A^\mpi}^\mpi = A$.
\item [(iii)] $\brb{A^*}^\mpi = \brb{A^\mpi}^*$.
\item [(iv)] $A^\mpi =A$ if $A$ is Hermitian and idempotent.
\item [(v)] $AA^\mpi$ and $A^\mpi A$ are idempotent.
\item [(vi)] $\rank(A) = \rank(A^\mpi) = \rank(AA^\mpi) = \rank(A^\mpi A)$.
\item [(vii)] $A^*AA^\mpi = A^* = A^\mpi AA^*$.
\item [(viii)] $A^*\brb{A^\mpi}^* A^\mpi = A^\mpi = A^\mpi \brb{A^\mpi}^* A^*$.
\item [(ix)] $\brb{A^*A}^\mpi = A^\mpi \brb{A^\mpi}^*$, $\brb{AA^*}^\mpi = \brb{A^\mpi}^*A^\mpi $.
\item [(x)] $A\brb{A^*A}^\mpi A^*A = A = AA^*\brb{AA^*}^\mpi A$.
\item [(xi)] $A^\mpi = \brb{A^*A}^\mpi A^* = A^* \brb{AA^*}^\mpi$.
\item [(xii)] $A^\mpi = \brb{A^*A}^{-1} A^*$ if $A$ has full column rank.
\item [(xiii)] $A^\mpi = A^* \brb{AA^*}^{-1}$ if $A$ has full row rank.
\item [(xiv)] $A= 0 \ \lra \ A^\mpi = 0$.
\item [(xv)] $AB = 0 \ \lra \ B^\mpi A^\mpi = 0$.
\item [(xvi)] $A^\mpi B = 0 \ \lra\ A^* B = 0$.
\item [(xvii)] $\brb{A\otimes B}^\mpi = A^\mpi \otimes B^\mpi$.
\een
\end{proposition}

\begin{proof}[\bf Proof]
\ben
\item [(i)] By definition of Moore-Penrose generalized inverse,
\be
AA^\mpi A = A \ \ra\ A^{\mpi} = A^{-1}AA^\mpi AA^{-1} = A^{-1} A A^{-1} = A^{-1}.
\ee

\item [(ii)] Direct result from definition of Moore-Penrose generalized inverse.

\item [(iii)] Let $B:= A^*$
\beast
B\brb{A^\mpi}^*B & = & \brb{AA^\mpi A}^* = A^* = B \\
\brb{A^\mpi}^*B \brb{A^\mpi}^*& = & \brb{A^\mpi AA^\mpi}^* = \brb{A^\mpi}^*  \\
\brb{\brb{A^\mpi}^*B}^* & = & AA^\mpi = \brb{AA^\mpi}^* = \brb{A^\mpi}^* B \\
\brb{B\brb{A^\mpi}^*}^* & = & A^\mpi A = \brb{A^\mpi A}^* = B\brb{A^\mpi}^*
\eeast

\item [(iv)] We only need to check the following two equalities.
\beast
AAA & = & AA = A \\
\brb{AA}^* & = & A^* = A = AA
\eeast

\item [(v)] By definition of Moore-Penrose generalized inverse,
\beast
\brb{AA^\mpi}^2 & = & AA^\mpi AA^\mpi = (AA^\mpi A)A^\mpi = AA^\mpi,\\
\brb{A^\mpi A}^2 & = & A^\mpi AA^\mpi A= (A^\mpi A A^\mpi)A = A^\mpi A.
\eeast

\item [(vi)] By Proposition \ref{pro:matrix_rank_inequalities_product}, $\rank\brb{AA^\mpi} \leq \min\bra{\rank(A),\rank(A^\mpi)}$. But
\be
\rank(A) = \rank\brb{AA^\mpi A} \leq \min\bra{\rank(AA^\mpi),\rank(A)} \leq \rank(AA^\mpi) \ \ra\ \rank(A) = \rank(AA^\mpi).
\ee

Similarly, we can have the other equalities.

\item [(vii)] By definition of Moore-Penrose generalized inverse,
\be
A^*AA^\mpi = A^*\brb{AA^\mpi}^* = \brb{AA^\mpi A}^* = A^* = \brb{A A^\mpi A}^* =\brb{A^\mpi A}^*A^* = A^\mpi A A^*.
\ee

\item [(viii)] By definition of Moore-Penrose generalized inverse,
\be
A^*\brb{A^\mpi}^* A^\mpi = \brb{A^\mpi A}^* A^\mpi = \brb{A^\mpi A} A^\mpi = A^\mpi = A^\mpi \brb{A A^\mpi} =A^\mpi \brb{A A^\mpi }^* = A^\mpi \brb{A^\mpi}^* A^*.
\ee

\item [(ix)] By (vii) and (viii), we have
\beast
A^*A \brb{A^\mpi \brb{A^\mpi}^*} A^*A & = & A^*A \brb{A^\mpi \brb{A^\mpi}^* A^*}A =  A^*A A^\mpi A= A^* A \\
\brb{A^\mpi \brb{A^\mpi}^*} A^*A \brb{A^\mpi \brb{A^\mpi}^*} & = & \brb{A^\mpi \brb{A^\mpi}^* A^*}A A^\mpi \brb{A^\mpi}^* =  A^\mpi A A^\mpi \brb{A^\mpi}^* = A^\mpi \brb{A^\mpi}^* \\
\brb{A^*A \brb{A^\mpi \brb{A^\mpi}^*} }^* & = & A^\mpi \brb{A^\mpi}^* A^*A  = \brb{A^\mpi \brb{A^\mpi}^* A^*}A = \brb{A^*\brb{A^\mpi}^* A^\mpi}A =\brb{A^\mpi A}^* A^\mpi A\text{  (Hermitian)} \\
\brb{\brb{A^\mpi \brb{A^\mpi}^*} A^*A  }^* & = & A^*A  A^\mpi \brb{A^\mpi}^*  = \brb{A^\mpi A A^*} \brb{A^\mpi}^* = A^\mpi A \brb{A^\mpi A}^* \text{  (Hermitian)}
\eeast
and
\beast
AA^* \brb{\brb{A^\mpi}^*A^\mpi } A A^* & = & A\brb{ A^* \brb{A^\mpi}^*A^\mpi } A A^* = A A^\mpi  A A^* = A A^* \\ % ^*A \brb{A^\mpi \brb{A^\mpi}^* A^*}A =  A^*A A^\mpi A= A* A \\
\brb{\brb{A^\mpi}^*A^\mpi } A A^*\brb{\brb{A^\mpi}^*A^\mpi } & = & \brb{A^\mpi}^*A^\mpi  A \brb{ A^*\brb{A^\mpi}^*A^\mpi } = \brb{A^\mpi}^*A^\mpi  A A^\mpi  =  \brb{A^\mpi}^*A^\mpi \\ % \brb{A^\mpi \brb{A^\mpi}^* A^*}A A^\mpi \brb{A^\mpi}^* =  A^\mpi A A^\mpi \brb{A^\mpi}^* = A^\mpi \brb{A^\mpi}^* \\
\brb{A A^*\brb{\brb{A^\mpi}^*A^\mpi }}^* & = & \brb{A^\mpi}^* A^\mpi  A A^* = \brb{A^\mpi}^* \brb{A^\mpi  A A^*} = \brb{A^\mpi}^* \brb{A^*AA^\mpi} =\brb{AA^\mpi }^* A A^\mpi \text{  (Hermitian)} \\
\brb{\brb{\brb{A^\mpi}^*A^\mpi }A A^* }^* & = & A  A^* \brb{A^\mpi}^*A^\mpi   =  A \brb{A^\mpi \brb{A^\mpi}^* A^*} = A A^\mpi  \brb{ A A^\mpi}^* \text{  (Hermitian)}
\eeast%$\brb{AA^*}^\mpi = \brb{A^\mpi}^*A^\mpi $.

\item [(x)] By (viii) and (ix), $A\brb{A^*A}^\mpi A^*A = A A^\mpi \brb{A^\mpi}^* A^*A =A A^*\brb{A^\mpi}^* A^\mpi A= A A^* \brb{A A^*}^\mpi A$. %= \brb{A A^*\brb{A^\mpi}^* A^\mpi}^* A

Also, by (vii), $A\brb{A^*A}^\mpi A^*A = A A^\mpi \brb{A^\mpi}^* A^*A =A A^\mpi AA^* \brb{A^\mpi}^* = A A^* \brb{A^\mpi}^* = \brb{A^\mpi A A^*}^* = \brb{A^*}^* = A$.%\ee%$A\brb{A^*A}^\mpi A^*A = A = AA^*\brb{AA^*}^\mpi A$.

\item [(xi)] Direct result from (viii) and (ix). %$A^\mpi = \brb{A^*A}^\mpi A^* = A^* \brb{AA^*}^\mpi$.

\item [(xii)] and (xiii) follow from (i) and (xi).

\item [(xiv)] Direct result from the definition of Moore-Penrose generalized inverse.

\item [(xv)] If $AB =0$, then by (xi),
\be
B^\mpi A^\mpi = \brb{B^*B}^\mpi B^* A^* \brb{AA^*}^\mpi =  \brb{B^*B}^\mpi (AB)^* \brb{AA^*}^\mpi = 0.
\ee

Thus, we can use similar argument to get the reverse.

\item [(xvi)] By (x) and (xi),
\be
A^\mpi B = 0 \ \lra\ \brb{A^*A}^\mpi A^* B = 0 \ \lra \ A^*A \brb{A^*A}^\mpi A^* B = 0 \ \lra\ A^* B = 0.
\ee

\item [(xvii)] By Proposition \ref{pro:kronecker_product_mixed_product}, \ref{pro:kronecker_product_transpose_adjoint}
\beast
\brb{A\otimes B} \brb{A^\mpi \otimes B^\mpi} \brb{A\otimes B} & = &  AA^\mpi A \otimes BB^\mpi B =  A\otimes B  \\
\brb{A^\mpi \otimes B^\mpi} \brb{A\otimes B} \brb{A^\mpi \otimes B^\mpi} & = &  A^\mpi A A^\mpi \otimes B^\mpi B B^\mpi=  A^\mpi\otimes B^\mpi  \\
\brb{\brb{A\otimes B} \brb{A^\mpi \otimes B^\mpi}}^* & = & \brb{ AA^\mpi \otimes BB^\mpi}^* = \brb{AA^\mpi}^*\otimes \brb{BB^\mpi}^* = AA^\mpi\otimes BB^\mpi = \brb{A\otimes B} \brb{A^\mpi \otimes B^\mpi}\\
\brb{\brb{A^\mpi \otimes B^\mpi}\brb{A\otimes B} }^* & = & \brb{ A^\mpi A \otimes B^\mpi B}^* = \brb{A^\mpi A}^*\otimes \brb{B^\mpi B}^* = A^\mpi A\otimes B^\mpi B= \brb{A^\mpi \otimes B^\mpi} \brb{A\otimes B}.
\eeast
\een
\end{proof}

\subsection{Further properties of MP generalized inverse}

\begin{proposition}
Let $A\in M_{m,n}(\C), B\in M_{n,p}(\C), C \in M_{m,p}(\C)$. Then
\be
A^* AB = A^* C \ \lra\ AB = AA^\mpi C.
\ee
\end{proposition}

\begin{proof}[\bf Proof]
If $AB = AA^\mpi C$, then
\be
A^*AB = A^*AA^\mpi C = A^* C
\ee
by Proposition \ref{pro:mp_inverse_basic_properties}.(vii). Conversely, if $A^* AB = A^* C$, then
\be
AA^\mpi C = A(A^*A)^\mpi A^* C = A(A^*A)^\mpi A^*AB = AB,
\ee
using Proposition \ref{pro:mp_inverse_basic_properties}.(xi),(x).
\end{proof}

\begin{proposition}\label{pro:full_rank_matrix_times_matrix_mpi_is_identity}
Let $A\in M_{m,n}(\C)$.
\ben
\item [(i)] If $A$ has full row rank, then $AA^\mpi = I_m$.
\item [(ii)] If $A$ has full column rank, then $A^\mpi A = I_n$.
\een
\end{proposition}

\begin{proof}[\bf Proof]
If $A$ has full column rank, we have by Proposition \ref{pro:mp_inverse_basic_properties}.(xii)
\be
A^\mpi = \brb{A^*A}^{-1} A^* \ \ra\ A^\mpi A = \brb{A^*A}^{-1} A^*A = I_n.
\ee

If $A$ has full column rank, we have by Proposition \ref{pro:mp_inverse_basic_properties}.(xiii)
\be
A^\mpi = A^* \brb{AA^*}^{-1} \ \ra\ A A^\mpi = A A^* \brb{AA^*}^{-1} = I_m.
\ee
\end{proof}

\begin{proposition}
Let $A\in M_{m,n}(\C), B\in M_{n,p}(\C)$. If $\det\brb{BB^*}\neq 0$, then $AB(AB)^\mpi = AA^\mpi$.
\end{proposition}

\begin{proof}[\bf Proof]
Since $\det\brb{BB^*}\neq 0$, $B$ has full row rank and $BB^\mpi = I$ by Proposition \ref{pro:full_rank_matrix_times_matrix_mpi_is_identity}. Then since $AB (AB)^\mpi$ is Hermitian (condition of definition MP generalized inverse),
\beast
AB (AB)^\mpi & = & \brb{AB (AB)^\mpi}^* = \brb{(AB)^\mpi}^*B^*A^* = \brb{(AB)^\mpi}^*B^*A^*AA^\mpi = \brb{(AB)^\mpi}^*(AB)^*AA^\mpi \\
& = & AB(AB)^\mpi AA^\mpi = AB(AB)^\mpi ABB^\mpi A^\mpi = ABB^\mpi A^\mpi = AA^\mpi
\eeast
by Proposition \ref{pro:mp_inverse_basic_properties}.(vii) and definition of MP generalized inverse.
\end{proof}

\begin{proposition}\label{pro:hermitian_idempotent_rank_difference_product_matrix_and_its_mp_inverse}
Let $A\in M_n(\C),B\in M_{n,m}(\C)$ with $A = A^* = A^2$ and $AB = B$. Then $A- BB^\mpi$ is Hermitian and idempotent with rank $\rank(A)-\rank(B)$. In particular, if $\rank(A) = \rank(B)$, then $A = BB^\mpi$.
\end{proposition}

\begin{proof}[\bf Proof]
Let $C = A - BB^\mpi$. Then
\be
C^* = A^* - \brb{BB^\mpi}^* = A - BB^\mpi = C
\ee
and
\be
CB = AB - BB^\mpi B = AB - B = 0.
\ee

Thus, by definition of MP generalized inverse.
\beast
C^2 & = & C\brb{A-BB^\mpi} = CA - CBB^\mpi = CA = A^2 - BB^\mpi A = A - BB^\mpi A \\
& = & A - (BB^\mpi)^*A^* = A - \brb{ABB^\mpi}^* = A - \brb{BB^\mpi}^* = A - BB^\mpi = C.
\eeast

Hence, $C$ is idempotent. Its rank is\footnote{proposition needed.}
\be
\rank(C) = \tr C = \tr A - \tr\brb{BB^\mpi} = \rank(A) - \rank\brb{BB^\mpi} = \rank(A) - \rank\brb{BB^\mpi} = \rank(A) - \rank\brb{B}
\ee
by Proposition \ref{pro:mp_inverse_basic_properties}.(vi). Clearly, if $\rank(A) - \rank\brb{B}$, then $C = 0$.
\end{proof}

\begin{proposition}
Let $A\in M_{n}(\C)$ be a Hermitian idempotent matrix and let $AB = 0$ for $B\in M_{n,m}(\C)$. If $\rank(A) + \rank(B) = n$, then $A = I_n - BB^\mpi$.
\end{proposition}

\begin{proof}[\bf Proof]
Let $C = I_n - A$. Then $C$ is Hermitian idempotent since $C^2 = I_n^2 - 2A + A^2 = I_n - A = C$. Also,
\be
CB = (I_n - A)B = B - AB = B - 0 = B.
\ee

Furthermore, since $A,C$ are idempotent, \footnote{idempotent matrix eigenvalue proposition needed.}
\be
\rank(C) = n - \rank(A) = \rank(B) \ \ra\ C = BB^\mpi \ \ra\ A = I_n - C = I_n - BB^\mpi
\ee
by Proposition \ref{pro:hermitian_idempotent_rank_difference_product_matrix_and_its_mp_inverse}.
\end{proof}

\begin{theorem}
Let $A\in M_n(\F)$ ($n>1$). Then
\ben
\item [(i)] If $\rank(A)=n$, then
\be
\adj A = \brb{\det A} A^{-1}.
\ee
\item [(ii)] If $\rank(A) = n-1$, then
\be
\adj A = (-1)^{k+1} \mu(A) \frac{xy^T}{y^T\brb{A^{k-1}}^\dag x}
\ee
where $k$ denotes the multiplicity of the zero eigenvalue of $A$ ($1\leq k\leq  n$), $\mu(A)$ is the product of the $n-k$ non-zero eigenvalues of $A$ (if $k=n$, we put $\mu(A)=1$), and $x$ and $y$ are $n\times 1$ vectors satisfying $Ax = A^Ty = 0$.

\item [(iii)] If $\rank(A)\leq n-2$, then $\adj A = 0$.
\een
\end{theorem}

\begin{remark}
Note that this theorem can imply Theorem \ref{thm:adjugate_matrix_rank}.
\end{remark}

\begin{proof}[\bf Proof]
\footnote{proof needed.}
\end{proof}

\section{Schur Complement}

\subsection{Schur complement}

\begin{definition}\label{def:schur_complement}
Let $M\in M_{n}(\F)$ be partitioned as
\be
M = \bepm A & B \\ C & D \eepm.
\ee

If $A$ is nonsingular, then Schur complement of $A$ in $M$ is denoted by
\be
M/A = D - CA^{-1}B.
\ee

If $D$ is nonsingular, then Schur complement of $D$ in $M$ is denoted by
\be
M/D = A - BD^{-1}C.
\ee
\end{definition}

\begin{theorem}\label{thm:schur_inverse_a}
Let $M\in M_{n}(\F)$ be partitioned as
\be
M = \bepm A & B \\ C & D \eepm
\ee
and both $M$ and $A$ are nonsingular. Then
\be
M^{-1} = \bepm A^{-1} + A^{-1}B(M/A)^{-1}CA^{-1} & \ -A^{-1}B(M/A)^{-1} \ \\ -(M/A)^{-1}CA^{-1}& (M/A)^{-1}\eepm.
\ee
\end{theorem}

\begin{proof}[\bf Proof]
\footnote{proof needed.}
\end{proof}




\section{Summary}

\subsection{General matrice}

\subsubsection{Matrix operations}

\ben
\item [(i)] $(AB)^T = B^TA^T$.
\item [(ii)] $(AB)^* = B^*A^*$.
\een



\subsection{Square matrices}


\subsubsection{Matrix operations}

\ben
\item [(i)] $(AB)^{-1} = B^{-1}A^{-1}$ for invertible matrices $A,B\in M_n(\F)$.
\item [(ii)] $\adj(AB) = \adj(B)\adj(A)$.
\een

\subsection{Invertible matrices}

We can combine Propositions and Theorems (see \cite{Horn_Johnson_1990}.$P_{14}$)\footnote{more equivalent forms needed.} to get that for $A\in M_n(\F)$ the following statements are equivalent: \ben
\item [(i)] $A$ is nonsingular.
\item [(ii)] $0\notin \sigma(A)$ (i.e., 0 is not an eigenvalue of $A$). \hspace{2cm} (Proposition \ref{pro:zero_eigenvalue_singular_equivalent})
\item [(iii)] $A$ is invertible, i.e. $A^{-1}$ exists. \hspace{3cm} (Proposition \ref{pro:invertible_non_singular_equivalent})
\item [(iv)] $\det A \neq 0$. \hspace{6cm} (Theorem \ref{thm:matrix_invertible_determinant_non_zero})
\item [(v)] The columns of $A$ are linearly independent. \hspace{1cm} (Corollary \ref{cor:invertible_column_linearly_independent})
\item [(vi)] $r(A) = n$, i.e., $A$ is full-rank. \hspace{3cm} (Lemma \ref{lem:full_rank_linearly_independent})
\item [(vii)] The rows of $A$ are linearly independent. \hspace{2cm} (Theorem \ref{thm:rank_matrix_transpose})
\een

\subsection{Rank}

(product rank inequalities). For $A\in M_{m,n}(\F)$ and $B\in M_{n,k}(\F)$,
\ben
\item [(i)] $\rank(AB) \leq \min\bra{\rank(A), \rank(B)}$ (Proposition \ref{pro:colrank_product_smaller_than_individual_colranks}).
\item [(ii)] $\rank(A) + \rank(B) \leq n + \rank(AB)$ (Sylvester's inequality, Proposition \ref{pro:sylvester_inequality_rank}).
\een

%\begin{proposition}[rank inequalities, product]\label{pro:matrix_rank_inequalities_product}%{pro:matrix_rank_properties}
%Let $A\in M_{m,n}(\F)$ and $B\in M_{n,k}(\F)$. Then
%\ben
%\item [(i)] $\rank(AB) \leq \min\bra{\rank(A), \rank(B)}$.
%\item [(ii)] $\rank(A) + \rank(B) \leq n + \rank(AB)$.
%\een
%\end{proposition}

%\begin{proof}[\bf Proof]
%\ben
%\item [(i)] We know that $\colsp(AB)$ is generated by the column vectors of $AB$ and it is also generated by the column vectors of $A$. That is, $\forall X\in \colsp(AB)$, we can find $C = \bepm c_1,\dots,c_k\eepm^T \in \F^k$ such that $X = c_1 (AB)_1 + \dots + c_k (AB)_k = (AB) C = A(BC) \in \colsp(A)$. Thus, $\colsp(AB)\subseteq \colsp(A)$. Then by the definition of rank (Definition \ref{def:column_rank_matrix}), we have $\rank(AB) \leq \rank(A)$.   %\footnote{prove $\rank(AB) \leq \rank(A)$ and then use rank of matrix transpose}.

%Also, we have $\rank(B^TA^T) \leq \rank(B^T)$. Then by Theorem \ref{thm:rank_matrix_transpose} and Proposition \ref{pro:matrix_multiple_transpose},
%\be
%\rank(AB) = \rank\brb{(AB)^T} = \rank(B^TA^T) \leq \rank(B^T) = \rank(B) \ \ra \ \rank(AB) \leq \min\bra{\rank(A), \rank(B)}.
%\ee

%\item [(ii)] \footnote{need proof}
%\een
%\end{proof}

(subadditivity of rank). For $A,B\in M_{m,n}(\F)$, $\rank(A+B) \leq \rank(A) + \rank(B)$ (Proposition \ref{pro:matrix_sum_rank_leq_sum_of_ranks}).

%\begin{proposition}[subadditivity of rank]\label{pro:subadditivity_matrix_rank}%{pro:matrix_rank_inequalities_sum}
%Let $A\in M_{m,n}(\F)$ and $B\in M_{m,n}(\F)$. Then
%\be
%\rank(A+B) \leq \rank(A) + \rank(B).
%\ee
%\end{proposition}

%\begin{proof}[\bf Proof]
%\footnote{proof needed.}
%\end{proof}

%\subsection{Positive definite matrices}


%\begin{problem}\label{exe:hermitian_positive_definite_product_sum_implies_positive_semidefinite}
%Let $X,Y\in M_n(\F)$ be Hermitian matrices such that $X$ is positive definite and $XY + YX$ is positive semi-definite. Show that $Y$ must be positive semi-definite.
%\end{problem}

%\begin{solution}[\bf Solution.]
%\footnote{solution needed.}%By spectral theorem\footnote{theorem needed.} we may assume that $X = U^* \Lambda U$ where $\Lambda$ is a diagonal matrix with positive elements.
%\end{solution}

%\begin{problem}
%Let $A,B\in M_n(\F)$ such that $A\geq B$, $A-B$ is positive semi-definite. Show that $A^{1/2}\geq B^{1/2}$.
%\end{problem}

%\begin{solution}[\bf Solution.]
%\footnote{solution needed.}%Problem \ref{exe:hermitian_positive_definite_product_sum_implies_positive_semidefinite}
%\end{solution}



\subsubsection{Order change of multiplication}

For $A,B\in M_n(\F)$, we have that
\be
AB\text{ and }BA\text{ have the same }\left\{\ba{ll}
\text{determinant} \quad\quad & \\ \text{trace} & \\ \text{eigenvalues} &
\ea
\right.
\ee

\subsection{Kronecker product}

\subsubsection{Basic properties}

\ben
\item [(i)] $\brb{A\otimes B}^T =  A^T\otimes B^T$.
\item [(Ii)] $\brb{A\otimes B}^* =  A^*\otimes B^*$.
\item [(iii)] $\brb{A\otimes B}^{-1} =  A^{-1}\otimes B^{-1}$.
\item [(iv)] $\brb{A\otimes B}\brb{C\otimes D} = (AC) \otimes (BD)$.
\een

\subsection{Relation of matrices}

\subsubsection{Equivalent matrices, similar matrices and unitarily equivalent}

\begin{center}
\begin{tabular}{ccllll}
\hline
 & matrix  & rank & determinant & trace & eigenvalue \\\hline
equivalence & $m \times n$ & $\surd$ (Proposition \ref{pro:equivalent_rank}) & - & - & -  \\
similarity & $n\times n$ & $\surd$ (Proposition \ref{pro:similarity_implies_equivalent_matrices}) & $\surd$ & $\surd$  & $\surd$ (Corollary \ref{cor:similarity_same_eigenvalues})  \\
unitary equivalence & $n\times n$ & $\surd$ (Proposition \ref{pro:unitary_equivalence_implies_similarity}) & $\surd$ & $\surd$ & $\surd$  \\
\hline
\end{tabular}
\end{center}


\begin{center}
\begin{tabular}{ccllllll}
\hline
 & matrix  & characteristic  & Jordan canonical & minimal & & &  \\
 &  & polynomial & form & polynomial & & &  \\
 \hline
similarity & $n\times n$ & $\surd$ (Theorem \ref{thm:similar_matrices_have_same_characteristic_polynomial}) & $\surd$ (Theorem \ref{thm:jordan_canonical_form} ) & $\surd$ (Corollary \ref{cor:similar_matrices_have_same_minimal_polynomial}) & & & \\
unitary equivalence & $n\times n$ & $\surd$ & $\surd$ & $\surd$ & & & \\
\hline
\end{tabular}
\end{center}

\beast
\text{unitary equivalence} & \subset & \text{similarity} \qquad (\text{Proposition \ref{pro:unitary_equivalence_implies_similarity}})\\
& \subset & \text{equivalence} \qquad (\text{Proposition \ref{pro:similarity_implies_equivalent_matrices}})  \\
& \subset & \text{equivalent relation} \qquad (\text{Proposition \ref{pro:equivalence_matrix_is_equivalent_relation}})
\eeast

Class of Hermitian matrices is closed under unitary equivalence. (Proposition \ref{pro:hermitian_matrices_closed_under_unitary_equivalence}).

\subsubsection{Diagonalizable matrices}

unitary diagonalizability $\subset$ diagonalizability $\subset$ similarity


\subsubsection{Normal matrices}

unitary equivalence preserves normality (Proposition \ref{pro:unitary_equivalence_preserves_normality}).

Hermitian $\subset$ normal

skew-Hermitian $\subset$ normal

unitary $\subset$ normal

%\subsection{Subsets of matrices}


\section{Problems}

\subsection{Basic properties}

\begin{problem}
Let $A\in M_n(\R)$. Then the following statements are equivalent:
\ben
\item [(i)] $A$ is symmetric, i.e. $A = A^T$.
\item [(ii)] $A^2 = A^TA$.
\item [(iii)] $A^2 = AA^T$.
\item [(iv)] $\tr(A^2) = \tr(A^TA)$.
\item [(v)] $\tr(A^2) = \tr(AA^T)$.
\een
\end{problem}

\begin{solution}[\bf Solution.]
It is easy to show that (i) $\ra$ (ii) $\ra$ (iv) and (i) $\ra$ (iii) $\ra$ (v). So is suffices to show that (iv) $\ra$ (i) and (v) $\ra$ (i).

Recalling the definition of trace, we have
\beast
0 & = & \tr(A^TA) - \tr(A^2) = \sum_{i=1}^n \sum^n_{j=1} a_{ji}^2 - \sum_{i=1}^n \sum^n_{j=1} a_{ij}a_{ji} \\
& = & \sum_{i<j} a_{ij}^2 + a_{ji}^2 + \sum^n_{i=1}a_{ii}^2 - \brb{2\sum_{i<j} a_{ij}a_{ji} + \sum^n_{i=1}a_{ii}^2} = \sum_{i<j} \brb{a_{ij} - a_{ji}}^2
\eeast
which implies that $a_{ij} = a_{ji}$ for all $i<j$ since the matrix is real. This is actually the statement that $A =A^T$. Similarly, (v) implies (i) as well.
\end{solution}



