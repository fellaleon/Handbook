\chapter{Statistics}

%\section{Estimation of Parameters}

\section{Estimators and Bias}

\begin{definition}[estimator]
An estimator is a rule for calculating an estimate of a given quantity based on observed data.
\end{definition}

\begin{definition}[bias\index{bias}, unbiased estimator\index{unbiased estimator}, unbiased estimator\index{biased estimator}]\label{def:unbiased_biased_estimator}
The bias (or bias function) of an estimator is the difference between this estimator's expected value and the true value of the parameter being estimated. That is, for an estimator $\wh{\theta}$ of a parameter $\theta$, the bias is $\E\wh{\theta} - \theta$.

An estimator or decision rule with zero bias is called unbiased. Otherwise the estimator is called to be biased.
\end{definition}

\begin{remark}
It is usually impossible to have both low variance and low bias at the same time.

A trivial estimator $\wh{\theta} = \theta_0$ for some constant $\theta_0$ has variance 0 but may have a very large bias if $\theta_0$ is very different from $\theta$.
\end{remark}


\begin{example}
The most popular estimators are the estimators of normal random variable $X_1,\dots,X_n\sim \sN(\mu,\sigma^2)$.
\be
\wh{\mu} = \frac 1n\sum^n_{i=1} X_i = \ol{X},\qquad \wh{\sigma}^2 = \frac 1{n-1} \sum^n_{i=1} \brb{X_i - \ol{X}}^2.
\ee

Note that both of above estimators are unbiased.
\end{example}


%\section{Statistic}
%\section{Parameters Estimation}
%\subsection{Unbiased estimation}

\section{Fisher information}

\subsection{Fisher information}

\begin{definition}[regularity conditions of density function\index{regularity conditions!density function}]\label{def:regularity_conditions_density_function}
Let $f$ be a probability density function. Then the regularity conditions are
\ben
\item Identificability: If $\theta \neq \theta'$, then $f(x;\theta) = f(x;\theta')$ for all $x$.
\item probability density functions $f$ have common support for all $\theta$.
\item The true value $\theta_0$ is an interior point in $\Omega$.
\een
\end{definition}

\begin{definition}[Fisher information\index{Fisher information}]\label{def:fisher_information}
Let probability density function $f $satisfies regularity conditions and two additional conditions:
\ben
\item [(i)] The probability density function $f(x;\theta)$ is differentiable as a function of $\theta$.
\item [(ii)] The integral $\int f(x;\theta) dx$ can be differentiated under the integral sign as a function of $\theta$.
\een

Then the Fisher information can be defined by
\be
\sI(\theta )=\E\left[\left.\left({\frac {\partial \log f(X;\theta )}{\partial \theta }}\right)^{2}\right|\theta \right] = \int \left({\frac {\partial \log f(x;\theta )}{\partial \theta }}\right)^{2}f(x;\theta )\,dx.
\ee

Note that $0\leq \sI(\theta) <\infty$.
\end{definition}

\begin{example}
Let $X$ random variable with density function $f(x|\theta) = \theta x^{\theta -1}$ within $(0,1)$ where $\theta >0$. Thus, $X \sim \betad(\theta,1)$. Then its Fisher information is
\beast
\sI_X(\theta) & = & \E\bsb{\left.\brb{\fp{\log \theta X^{\theta-1}}{\theta} }^2\right|\theta} = \E\bsb{\left.\brb{\fp{}{\theta}\brb{\log \theta + \brb{\theta-1}\log X }}^2\right|\theta} = \E\brb{\frac 1{\theta} + \log X }^2.
\eeast

Let $Y := -\log X$ and thus $e^{-Y} = X$. Then density function of $Y$ is
\be
f(y|\theta) = \theta e^{-(\theta-1)y} e^{-y} = \theta e^{-\theta y}
\ee
which is an exponential whose mean and variance are $1/\theta$ and $1/\theta^2$. So
\be
\sI_X(\theta) =  \E\brb{\frac 1{\theta} - Y }^2 = \frac 1{\theta^2} - \frac 2{\theta}\E Y + \E Y^2 =  \frac 1{\theta^2} -  \frac 2{\theta^2} +  \frac 2{\theta^2} = \frac 1{\theta^2}.
\ee
\end{example}





\begin{proposition}[Fisher information of Bernoulli distribution]
Let $X\sim \berd(p)$. Then its Fisher information is
\be
\sI_X(p) = \frac 1{p(1-p)}.
\ee
\end{proposition}

\begin{proof}[\bf Proof]
By definition, we have
\beast
\sI_X(p) & = & -\E\brb{\left.\frac{\partial^2}{\partial p^2} \log\brb{p^X(1-p)^{1-X}} \right|p} = -\E\brb{\left.\frac{\partial^2}{\partial p^2} X\log p + (1-X)\log(1-p) \right|p} \\
& = & \E\brb{\left. \frac X{p^2} + \frac{1-X}{(1-p)^2}\right|p} = \frac{p}{p^2} + \frac{1-p}{(1-p)^2} = \frac 1{p(1-p)}.
\eeast
\end{proof}


\begin{proposition}\label{pro:fisher_information_variance_expression}
Let $f$ be probability density function satisfies regularity conditions and its Fisher information is well-defined (the first derivative of $f$ as a function of $\theta$ exists and the integral $\int f(x|\theta) dx$ can be differentiated under the integral sign as a function of $\theta$.). Then we have
\be
\E\bsb{\left.\fp{}{\theta}\log f(X|\theta)\right|\theta} = 0.
\ee

The function $\fp{}{\theta}\log f(X|\theta)$ is called the score function $S(\theta)$. Consequently, the Fisher information can be expressed by
\be
\sI_X(\theta) = \var\bsb{\left.\fp{}{\theta}\log f(X|\theta)\right|\theta}.
\ee
\end{proposition}

\begin{proof}[\bf Proof]
By the definition of expectation, we have
\beast
\E\bsb{\left.\fp{}{\theta}\log f(X|\theta)\right|\theta} & = & \int \brb{\fp{}{\theta}\log f(X|\theta)}f(x|\theta) dx = \int \frac{\fp{}{\theta} f(X|\theta)}{f(x|\theta)}f(x|\theta) dx \\
& = & \int \fp{}{\theta} f(X|\theta) dx = \fp{}{\theta} \int  f(X|\theta) dx = \fp{1}{\theta}= 0
\eeast
as required.
\end{proof}

\begin{proposition}
Let $X = \brb{X_1,\dots,X_n}$ be independent random variables with density function $f_i(x|\theta)$ and their Fisher information are well-defined. Then
\be
\sI_X(\theta) = \sI_{X_1}(\theta) + \dots + \sI_{X_n}(\theta).
\ee

Furthermore, if $X_1,\dots,X_n$ are i.i.d., we have that $\sI_X(\theta) = n\sI_{X_1}(\theta)$.
\end{proposition}

\begin{proof}[\bf Proof]
Since $X_1,\dots,X_n$ are independent, we have
\be
f(X|\theta) = \prod^n_{i=1} f_i(X_i|\theta) \ \ra\ \fp{}{\theta} \log f(X|\theta) = \sum^n_{i=1}\fp{}{\theta} \log f_i(X_i|\theta)
\ee
by Theorem \ref{thm:characteristic_function_independence}. Then the random variables in the sum are independent by Theorem \ref{thm:random_variable_function_indenpence}. Thus by Proposition \ref{pro:expectation_of_independent_product},
\beast
\sI_X(\theta) & = & \E\brb{\left. \brb{\fp{}{\theta} \log f(X|\theta)}^2 \right|\theta} = \E\brb{\left. \brb{ \sum^n_{i=1}\fp{}{\theta} \log f_i(X_i|\theta) }^2 \right|\theta} \\
& = & \sum^n_{i=1} \E\brb{\left. \brb{ \fp{}{\theta} \log f_i(X_i|\theta) }^2 \right|\theta} = \sum^n_{i=1} \sI_{X_i}(\theta).
\eeast
\end{proof}



\begin{proposition}\label{pro:fisher_information_second_order_derivative_expression}
Let $f$ be probability density function and the Fisher information is well-defined. If probability density function $f $satisfies
\ben
\item [(i)] The probability density function $f(x;\theta)$ is twice differentiable as a function of $\theta$.
\item [(ii)] The integral $\int f(x;\theta) dx$ can be differentiated twice under the integral sign as a function of $\theta$.
\een

Then the Fisher information can be expressed by
\be
\sI(\theta )=\E\bsb{\left.\brb{\frac {\partial \log f(X|\theta )}{\partial \theta }}^{2}\right|\theta } = - \E\bsb{\left.\frac {\partial^2 \log f(X;\theta ) }{\partial \theta^2 }\right|\theta }.
\ee
\end{proposition}


\begin{proof}[\bf Proof]
Since $1 = \int f(x|\theta) dx$, we can take differentiation on both sides,
\be
0 = \fp{}{\theta}\int f(x|\theta) dx = \int \fp{}{\theta} f(x|\theta) dx = \int \frac{ \fp{f(x|\theta)}{\theta}}{f(x|\theta)} f(x|\theta) dx = \int \brb{ \fp{}{\theta}\log f(x|\theta)} f(x|\theta) dx
\ee

Taking differentiation again,
\beast
0 & = & \fp{}{\theta} \int \brb{ \fp{}{\theta}\log f(x|\theta)} f(x|\theta) dx = \int \fp{}{\theta}\bsb{\brb{ \fp{}{\theta}\log f(x|\theta)} f(x|\theta)} dx \\
& = & \int \frac{\partial^2}{\partial \theta^2}\brb{\log f(x|\theta)} f(x|\theta) dx + \int \brb{\fp{}{\theta}\log f(x|\theta)} \fp{}{\theta} f(x|\theta) dx \\
& = & \int \frac{\partial^2}{\partial \theta^2}\brb{\log f(x|\theta)} f(x|\theta) dx + \int \brb{\fp{}{\theta}\log f(x|\theta)}^2 f(x|\theta) dx \\
& = & \int \frac{\partial^2}{\partial \theta^2}\brb{\log f(x|\theta)} f(x|\theta) dx + \sI_X(\theta)
\eeast
as required.
\end{proof}



\begin{example}
Let $X$ be a Poisson distributed random variable with density function $f(x|\lm) = \frac{\lm^x e^{-\lm}}{x!}$. Thus, we have
\beast
f(x|\lm) & = & \frac{\lm^x e^{-\lm}}{x!}, \\
\log f(x|\lm) & = & x \log \lm - \lm - \log (x!) ,\\
\fp{}{\lm}\log f(x|\lm) & = & \frac{x}{\lm}  -1 ,\\
-\fpp{}{\lm}\log f(x|\lm) & = & \frac{x}{\lm^2}  .
\eeast

Also, by Proposition \ref{pro:moments_poisson}, $\E X  = \lm$ and $\var X = \lm$. Therefore, we have three approaches to calculate the Fisher information.
\ben
\item [(i)] Using Definition \ref{def:fisher_information},
\beast
\sI_X(\lm) = \E\bsb{\left.\brb{\fp{}{\lm}\log f(x|\lm) }^2\right|\lm} = \E\brb{\frac{X}{\lm}  -1}^2 = \var\brb{\frac{X}{\lm}} = \frac 1{\lm^2}\var X = \frac 1{\lm}.
\eeast

\item [(ii)] By Proposition \ref{pro:fisher_information_variance_expression},
\beast
\sI_X(\lm) = \var\bsb{\left.\fp{}{\lm}\log f(x|\lm) \right|\lm} = \var\brb{\frac{X}{\lm}  -1} = \var\brb{\frac{X}{\lm}} = \frac 1{\lm}.
\eeast

\item [(iii)] By Proposition \ref{pro:fisher_information_second_order_derivative_expression},
\beast
\sI_X(\lm) = \E\bsb{\left.-\fpp{}{\lm}\log f(x|\lm)\right|\lm} = \E\brb{\frac{X}{\lm^2}} = \frac 1{\lm^2} \E X = \frac 1{\lm}.
\eeast
\een
\end{example}



\subsection{Fisher information matrix}


\begin{definition}[Fisher information matrix\index{Fisher information matrix}]\label{def:fisher_information_matrix}
Let $X$ be a random variable with $n$ parameters. Then the Fisher information matrix of $X$ is defined by
\be
\bsb{\sI_X(\theta)}_{i,j} : = \E\brb{\left. \brb{\fp{\log f(X|\theta)}{\theta_i}}\brb{\fp{\log f(X|\theta)}{\theta_j}}\right|\theta}
\ee
for $1\leq i,j\leq n$.
\end{definition}

\begin{remark}
The Fisher information matrix is $n\times n$ positive semi-definite matrix.
\end{remark}


\begin{proposition}[Fisher information matrix for normal random variable]
Let $X\sim \sN(\mu,\sigma^2)$. Then the Fisher information matrix is
\be
\sI_X(\mu,\sigma^2) = \bepm \frac 1{\sigma^2} & 0 \\ 0 & \frac 1{2\sigma^4} \eepm.
\ee
\end{proposition}


\begin{proof}[\bf Proof]
By definition, we have
\be
f_X(x|\theta) = \frac 1{\sqrt{2\pi\sigma^2}}\exp\brb{-\frac{(x-\mu)^2}{2\sigma^2}} \ \ra\ \log f(X|\theta) = -\frac 12 \log(2\pi \sigma^2) - \frac{(x-\mu)^2}{2\sigma^2}.
\ee

Thus,
\be
\fp{\log f(X)}{\mu} = \frac 1{\sigma^2} (X - \mu), \qquad \fp{\log f(X)}{\sigma^2} = -\frac 1{2\sigma^2} + \frac{(X-\mu)^2}{2\sigma^4}.
\ee

%\beast
%\sI_X(p) & = & -\E\brb{\left.\frac{\partial^2}{\partial p^2} \log\brb{p^X(1-p)^{1-X}} \right|p} = -\E\brb{\left.\frac{\partial^2}{\partial p^2} X\log p + (1-X)\log(1-p) \right|p} \\
%& = & \E\brb{\left. \frac X{p^2} + \frac{1-X}{(1-p)^2}\right|p} = \frac{p}{p^2} + \frac{1-p}{(1-p)^2} = \frac 1{p(1-p)}.
%\eeast

Then by the properties of normal distribution,
\be
\sI_X(\theta)_{1,1} = \frac 1{\sigma^4} \E (X - \mu)^2 = \frac 1{\sigma^2}.
\ee

\be
\sI_X(\theta)_{1,2} = \sI_X(\theta)_{2,1} = \frac 1{\sigma^2} \E\brb{(X - \mu)\brb{-\frac 1{2\sigma^2} + \frac{(X-\mu)^2}{2\sigma^4}} } = 0.
\ee

\be
\sI_X(\theta)_{2,2} = \E\brb{\frac 1{2\sigma^2} - \frac{(X-\mu)^2}{2\sigma^4}}^2 = \frac 1{4\sigma^4} - \frac {\sigma^2}{2\sigma^6} + \frac {3\sigma^4}{4\sigma^8} = \frac 1{2\sigma^4} .
\ee
\end{proof}

\begin{proposition}[Fisher information matrix for Cauchy random variable]
Let $X\sim \sC(\mu,d)$. Then the Fisher information matrix is
\be
\sI_X(\mu,d) = \bepm \frac 1{2d^2} & 0 \\ 0 & \frac 1{2d^2} \eepm.
\ee
\end{proposition}

\begin{proof}[\bf Proof]
By definition, we have
\be
f_X(x|\theta) = \frac{d}{\pi\brb{d^2 + (x-\mu)^2}}.
\ee

Then
\be
\fp{}{\mu} \log f(X|\theta) = \frac{\fp{f(X|\theta)}{\mu}}{f(X|\theta)} = \frac{\frac{2d(X-\mu)}{\pi\brb{d^2 + (X-\mu)^2}}}{\frac{d}{\pi\brb{d^2 + (X-\mu)^2}}} = \frac{2(X-\mu)}{d^2 + (X-\mu)^2}.
\ee

\be
\fp{}{d} \log f(X|\theta) = \frac{\fp{f(X|\theta)}{d}}{f(X|\theta)} = \frac{\frac{d^2 + (X-\mu)^2 - 2d^2}{\pi\brb{d^2 + (X-\mu)^2}}}{\frac{d}{\pi\brb{d^2 + (X-\mu)^2}}} = \frac{(X-\mu)^2 - d^2}{d\brb{d^2 + (X-\mu)^2}}.
\ee

Therefore,
\beast
\sI(\theta)_{1,1} & = & \E\brb{\frac{2(X-\mu)}{d^2 + (X-\mu)^2}}^2 = 4\int^\infty_{-\infty} \frac{u^2}{\brb{d^2 + u^2}^2} \frac{d}{\pi(d^2+ u^2)} du  = \frac{4d}{\pi}\int^\infty_{-\infty} \frac{u^2}{\brb{d^2 + u^2}^3} du \qquad (*) \\
& = & \frac{8d}{\pi}\int^\infty_0 \frac{u^2}{\brb{d^2 + u^2}^3} du.
\eeast

Let $x = 1/(d^2 + u^2)$, then $u = \brb{1/x - d^2}^{1/2}$ and
\be
du = \frac{\frac 12 \brb{-\frac 1{x^2}}}{\sqrt{1/x - d^2 }} dx.
\ee

Thus,
\beast
\sI(\theta)_{1,1} & = & \frac{4d}{\pi} \int^{1/d^2}_0 \brb{\frac 1x  -d^2}x^3 \frac 1{x^2}\brb{\frac 1x - d^2}^{-1/2} dx = \frac{4d}{\pi} \int^{1/d^2}_0 x^{1/2}\brb{1- xd^2}^{1/2} dx \\
& = & \frac{4d}{\pi} \int^{1}_0 \frac 1d y^{1/2}\brb{1- y}^{1/2} d\frac y{d^2} = \frac{4}{\pi d^2} \int^1_0 y^{1/2}(1-y)^{1/2} dy \\
& = & \frac{4}{\pi d^2} B\brb{\frac 32,\frac 32} = \frac{4}{\pi d^2}\frac{\Gamma(3/2)\Gamma(3/2)}{\Gamma(3)} = \frac{4}{\pi d^2} \frac{\brb{\frac 12 \sqrt{\pi}}^2}{2!} = \frac{1}{2d^2}.
\eeast

Additionally,
\beast
\sI(\theta)_{1,2} = \sI(\theta)_{2,1}  & = & \E\bsb{\frac{2(X-\mu)}{d^2 + (X-\mu)^2}\frac{(X-\mu)^2 - d^2}{d\brb{d^2 + (X-\mu)^2}}} = 2\int^\infty_{-\infty} \frac{u(u^2 -d^2)}{\brb{d^2 + u^2}^2} \frac{d}{\pi(d^2+ u^2)} du \\
& = & \frac{2d}{\pi}\int^\infty_{-\infty} \frac{u^3 -u d^2}{\brb{d^2 + u^2}^3}  du = 0
\eeast
as the integrand is odd function. Finally,
\beast
\sI(\theta)_{2,2} & = & \E\brb{\frac{(X-\mu)^2 - d^2}{d\brb{d^2 + (X-\mu)^2}}}^2 = \int^\infty_{-\infty} \frac{\brb{u^2 + d^2}^2 - 4u^2d^2 }{d^2\brb{d^2 + u^2}^2} \frac{d}{\pi(d^2+ u^2)} du \\
&  = & \frac{2}{\pi d} \int^\infty_0 \frac{1}{d^2 + u^2} du -  \frac{8d}{\pi }  \int^\infty_0 \frac{u^2}{\brb{d^2 + u^2}^3} du = \frac{2}{\pi d^2} \int^\infty_0 \frac{1}{1 + (u/d)^2} d(u/d) - \frac 1{2d^2} \\
& = & \frac{2}{\pi d^2} \frac {\pi}2  - \frac 1{2d^2} =  \frac 1{2d^2}
\eeast
as required. Note that the integral $(*)$ can also be solved by Cauchy residue theorem\footnote{theorem needed.} as
\be
\int^\infty_{-\infty} \frac{u^2}{(d^2 + u^2)^3} du = 2\pi i \res\bsb{\frac{u^2}{(d^2 + u^2)^3}, +id} = \frac{2\pi i}{(3-1)!}\lim_{u\to id} \frac{d^2}{du^2}\brb{\frac {u^2}{\brb{u + id}^3}} = \frac{\pi}{8d^3}.
\ee
\end{proof}

\subsection{Cram\'er-Rao lower bound}

\begin{theorem}[Cram\'er-Rao lower bound\index{Cram\'er-Rao lower bound}]\label{thm:cramer_rao_lower_bound}
\end{theorem}

\begin{proof}[\bf Proof]
\footnote{proof needed.}
\end{proof}

\section{Maximum Likelihood Estimation}

\subsection{Likelihood function}

\begin{definition}[likelihood function\index{likelihood function}, log-likelihood function\index{log-likelihood function}]\label{def:likelihood_function}
Let $f(x|\theta)$ be probability density function with parameters $\theta$ in $\Omega$. Then assume observations $X_1,\dots,X_n$ are independent and have identical density function $f(x|\theta)$. Then the likelihood function is defined by % (and their corresponding the observations are $x_1,\dots,x_n$).
\be
L(\theta|X_1,\dots,X_n) = \prod^n_{i=1} f(X_i|\theta)
\ee

Also, the log-likelihood function is
\be
\sL\brb{\theta|X_1,\dots,X_n} = \log L(\theta|X_1,\dots,X_n) = \sum^n_{i=1} \log f(X_i|\theta).
\ee
\end{definition}


\subsection{Estimation of distributions}

\begin{example}[MLE for Bernoulli distribution]
Let $X_i \sim \berd(p)$. The likelihood function is
\be
L(p|X_1,\dots,X_n) = p^{\sum^n_{i=1}X_i} (1-p)^{n - \sum^n_{i=1}X_i}
\ee

Then taking the differentiation of $L$ with respect to $p$, we have
\be
\fp{L(p|X_1,\dots,X_n)}{p} = 0 \ \ra\ 0 =\brb{(1-p) \sum^n_{i=1}X_i  - p\brb{n - \sum^n_{i=1}X_i} } p^{\sum^n_{i=1}X_i-1} (1-p)^{n - \sum^n_{i=1}X_i-1}
\ee

Therefore, we have
\be
\wh{p} = \frac 1n\sum^n_{i=1}X_i = \ol{X}
\ee
which is a rather natural estimator.
\end{example}

\begin{example}[MLE for uniform distribution]
Given the observations $X_i\in (0,\infty)$, $i=1,\dots,n$, the maximum likelihood estimation of uniform distribution $\sU(0,\theta)$ is \be \wh{\theta} = \max_{1\leq i\leq n} X_i .\ee%Furthermore, $\wh{\mu}$ is an unbiased estimator, while $\wh{\sigma^2}$ is a biased one.
\end{example}

\begin{proof}[\bf Proof]
We have the likelihood function is
\be
L\brb{\theta|X_1,\dots,X_n} = \prod^n_{i = 1} \frac 1{\theta} \ind_{[0,\theta]}(X_i) = \frac 1{\theta^n} \ind_{[0,\theta]}\brb{\max_{1\leq i\leq n}X_i} \ind_{[0,\theta]}\brb{\min_{1\leq
i\leq n}X_i}.
\ee


\begin{center}
\psset{yunit=3cm,xunit=3cm}
\begin{pspicture}(-0.2,-0.2)(3,1.2)%[showgrid](-3,-1.5)(3,4)
\psaxes[labels=none,ticks=none]{->}(0,0)(-0.2,-0.2)(3,1.2)%Dx=0.25,Dy=0.25%dx =1,dy=1,
\psset{algebraic}
\psplot{1}{2.8}{3/3^x}
%\psplot{1}{1}{x-0.5}

\rput[lb](0.8,-0.17){$\max_{1\leq i\leq n}X_i$}
\rput[lb](-0.2,0.9){$\frac 1{\theta^n}$}
\rput[lb](2.7,-0.15){$\theta$}
\rput[lb](0.05,1.1){$L$}

\psline[linestyle=dashed](1,0)(1,1)(0,1)

\end{pspicture}
\end{center}


So the likelihood function is 0 for $\theta < \max_{1\leq i\leq n}X_i$ and positive and decreasing for $\theta \geq \max_{1\leq i\leq n}X_i$. Thus, $\wh{\theta} = \max_{1\leq i\leq n} X_i$.
\end{proof}


%\subsection{Estimation of exponential distribution}

\begin{example}[MLE for exponential distribution]
Given the observations $X_i\in (0,\infty)$, $i=1,\dots,n$, the maximum likelihood estimation of exponential distribution $\sE(\lm)$ is \be \wh{\lm} = \brb{\frac 1n \sum^n_{i = 1} X_i}^{-1} := 1/\ol{X}.\ee%Furthermore, $\wh{\mu}$ is an unbiased estimator, while $\wh{\sigma^2}$ is a biased one.
\end{example}

\begin{remark}
Although $\wh{\lm}$ is not an unbiased estimator of $\lm$, $\ol{X}$ is an unbiased estimator of $1/\lm$.% = \beta, where \beta is the scale parameter defined in the 'Alternative parameterization' section above.
\end{remark}

\begin{proof}[\bf Proof]
We have the likelihood function is \be \sL\brb{\lm|X_1,\dots,X_n} = \sum^n_{i = 1} \log \lm - \lm X_i. \ee

Thus, taking the differentiation with respect to $\lm$, we have \be \left.\fp{\sL}{\lm}\right|_{\wh{\lm}} = 0 \ \ra \ \sum^n_{i = 1} \frac 1{\wh{\lm}} - X_i = 0 \ \ra\ \wh{\lm} = \brb{\frac 1n \sum^n_{i = 1} X_i}^{-1} :=
1/\ol{X}. \ee %By Theorem \ref{thm:normal_rv_sample_mean_variance}, we know that \be \wh{\mu} \sim \sN\brb{\mu, \frac 1n \sigma^2},\qquad \wh{\sigma^2} \sim \frac 1n \sigma^2 \chi^2_{n-1}. \ee %Thus, we have \be \E\wh{\lm} = \mu,\quad \E\wh{\sigma^2} = \frac {n-1}n\sigma^2 < \sigma^2 \ee which implies that $\wh{\mu}$ is an unbiased estimator and $\wh{\sigma^2}$ is a biased one.

Recall the mean of exponential random variable (Proposition \ref{pro:moments_exponential}), we have that $\ol{X}$ is an unbiased estimator of $1/\lm$.
\end{proof}

%\subsection{Estimation of Gaussian distribution}

Now we try to estimate the parameters of Gaussian distribution by its samples.

\begin{example}[MLE for normal distribution]
Given the observations $X_i\in \R$, $i=1,\dots,n$, the maximum likelihood estimation of Gaussian distribution $\sN(\mu,\sigma^2)$ is \be \wh{\mu} = \frac 1n \sum^n_{i = 1} X_i := \ol{X},\qquad \wh{\sigma^2} =   \frac
1n\sum^n_{i=1} \brb{X_i- \ol{X}}^2.\ee

Furthermore, $\wh{\mu}$ is an unbiased estimator, while $\wh{\sigma^2}$ is a biased one.
\end{example}

\begin{proof}[\bf Proof]
We have the likelihood function is \be \sL\brb{\mu,\sigma^2|X_1,\dots,X_n} = \sum^n_{i = 1} -\frac 12\log (2\pi\sigma^2) - \frac 1{2\sigma^2}\brb{X_i-\mu}^2. \ee

Thus, taking the differentiation with respect to $\mu$ and $\sigma^2$, we have\footnote{It is straightforward to check this stationary point is indeed a maximum by examining the Hessian matrix.} \be
\left.\fp{\sL}{\mu}\right|_{\wh{\mu},\wh{\sigma^2}} = 0 \ \ra \ \sum^n_{i = 1} (X_i - \wh{\mu}) = 0 \ \ra\ \wh{\mu} = \frac 1n \sum^n_{i = 1} X_i := \ol{X}. \ee

\be \left.\fp{\sL}{\sigma^2}\right|_{\wh{\mu},\wh{\sigma^2}} = 0 \ \ra \ 0 = \sum^n_{i = 1} -\frac 1{2\wh{\sigma^2}} + \frac 1{2\wh{\sigma^2}^2}\brb{X_i- \ol{X}}^2 \ \ra\ \wh{\sigma^2} = \frac 1n\sum^n_{i=1} \brb{X_i-
\ol{X}}^2.  \ee

By Theorem \ref{thm:normal_rv_sample_mean_variance}, we know that \be \wh{\mu} \sim \sN\brb{\mu, \frac 1n \sigma^2},\qquad \wh{\sigma^2} \sim \frac 1n \sigma^2 \chi^2_{n-1}. \ee

Thus, we have \be \E\wh{\mu} = \mu,\quad \E\wh{\sigma^2} = \frac {n-1}n\sigma^2 < \sigma^2 \ee which implies that $\wh{\mu}$ is an unbiased estimator and $\wh{\sigma^2}$ is a biased one.
\end{proof}

%\subsection{Estimation of gamma distribution}

\begin{example}[MLE for gamma distribution]
Given the observations $X_i\in (0,\infty)$, $i=1,\dots,n$, the maximum likelihood estimation of gamma distribution $\Gamma(k,\lm)$ is \be \wh{\lm} = \brb{\frac 1{kn} \sum^n_{i = 1} X_i}^{-1}, \qquad \log \wh{k}
-\psi\brb{\wh{k}}  =  \log\brb{\frac 1n\sum^n_{i = 1} X_i} - \frac 1n \sum^n_{i=1}
\log X_i.\ee
where $\psi(x) = \frac{d}{dx}\ln\Gamma(x) = \frac{\Gamma'(x)}{\Gamma(x)}$ is the digamma function\footnote{definition needed.}.%Furthermore, $\wh{\mu}$ is an unbiased estimator, while $\wh{\sigma^2}$ is a biased one.
\end{example}

\begin{remark}
There is no closed-form solution for $k$. The function is numerically very well behaved, so if a numerical solution is desired, it can be found using, for example, Newton's method\footnote{method needed.}. An initial value
of $k$ can be found either using the method of moments, or using the approximation

\be \log\wh{k} - \psi\brb{\wh{k}} \approx \frac{1}{2\wh{k}}\brb{1 + \frac{1}{6\wh{k} + 1}}.\ee

If we let
\be
s = \log\brb{\frac{1}{n}\sum_{i=1}^n X_i} - \frac{1}{n}\sum_{i=1}^n\log X_i,
\ee
then $\wh{k}$ is approximately \be \wh{k} \approx \frac{3 - s + \sqrt{(s - 3)^2 + 24s}}{12s} \ee which is within 1.5\% of the correct value.%Although $\wh{\lm}$ is not an unbiased estimator of $\lm$, $\ol{X}$ is an unbiased estimator of $1/\lm$.% = \beta, where \beta is the scale parameter defined in the 'Alternative parameterization' section above.
\end{remark}

\begin{proof}[\bf Proof]
We have the likelihood function is
\be
\sL\brb{k,\lm|X_1,\dots,X_n} = \sum^n_{i = 1} - \log \brb{\Gamma(k)} + k\log \lm + (k-1)\log X_i - \lm X_i.
\ee

Thus, taking the differentiation with respect to $k$ and $\lm$, we have
\be
\left.\fp{\sL}{\lm}\right|_{\wh{k},\wh{\lm}} = 0 \ \ra \ \sum^n_{i = 1} \frac k{\wh{\lm}} - X_i = 0 \ \ra\ \wh{\lm} = \brb{\frac 1{kn} \sum^n_{i = 1} X_i}^{-1},
\ee

\be
\left.\fp{\sL}{k}\right|_{\wh{k},\wh{\lm}} = 0 \ \ra \ \sum^n_{i = 1} -\psi\brb{\wh{k}} + \log \wh{\lm} + \log X_i = 0 \ \ra\ \log \wh{k} -\psi\brb{\wh{k}}  =  \log\brb{\frac 1n\sum^n_{i = 1} X_i} - \frac 1n \sum^n_{i=1} \log X_i.
\ee                       %By Theorem \ref{thm:normal_rv_sample_mean_variance}, we know that \be \wh{\mu} \sim \sN\brb{\mu, \frac 1n \sigma^2},\qquad \wh{\sigma^2} \sim \frac 1n \sigma^2 \chi^2_{n-1}. \ee %Thus, we have \be \E\wh{\lm} = \mu,\quad \E\wh{\sigma^2} = \frac {n-1}n\sigma^2 < \sigma^2 \ee which implies that $\wh{\mu}$ is an unbiased estimator and $\wh{\sigma^2}$ is a biased one.
\end{proof}


\begin{example}
Let $X_1,\dots,X_n$ are i.i.d. with density function $f(x|\theta) = \theta x^{\theta -1}$ within $(0,1)$ where $\theta >0$. Thus, $X_i \sim \betad(\theta,1)$. Then its log-likelihood function
\be
\sL(\theta|X_1,\dots,X_n) = n \log \theta + \brb{\theta-1}\sum^n_{i=1}\log X_i .
\ee

Taking differentiation we have
\be
0 = \fp{\sL(\theta|X_1,\dots,X_n)}{\theta}= \frac n{\theta} + \sum^n_{i=1}\log X_i  \ \ra\ \wh{\theta} = - \frac{n}{\sum^n_{i=1}\log X_i}.
\ee

Let $Y_i := -\log X_i$ and thus $e^{-Y_i} = X_i$. Then density function of $Y_i$ is
\be
f(y|\theta) = \theta e^{-(\theta-1)y} e^{-y} = \theta e^{-\theta y}
\ee
which is an exponential whose mean and variance are $1/\theta$ and $1/\theta^2$. Thus, $- \sum^n_{i=1}\log X_i$ is a gamma distributed random variable $\Gamma\brb{n,\theta}$. So $\brb{-\sum^n_{i=1}\log X_i}^{-1}$ is inverse gamma distributed ($\Gamma^{-1}(n,\theta)$). Then by Proposition \ref{pro:moments_inverse_gamma},
\be
\E\wh{\theta} = n\E\brb{- \frac{1}{\sum^n_{i=1}\log X_i}} = n\frac{\theta}{n-1} = \frac{n\theta}{n-1}
\ee
and
\be
\var\wh{\theta} = n^2\var\brb{- \frac{1}{\sum^n_{i=1}\log X_i}} = n^2\frac{\theta^2}{(n-1)^2(n-2)} = \frac{n^2\theta^2}{(n-1)^2(n-2)}.
\ee

\end{example}








\section{Linear Regression Model}


\subsection{Linear model}

We assume $Y_1,\dots,Y_n$ are independent with each $Y_i\sim \sN(\mu_i,\sigma^2)$ and $\mu_i = X_i \cdot \beta$ where $X_i = \brb{X_{i1},\dots,X_{im}}$ ($i = 1,\dots,n$) is a known vector explanatory variables and $\beta \in \R^m$ is an unknown parameter with $m< n$.

Usually, we assume that $X^TX$ has rank $m<n$ ($\rank(X^TX) = \rank(X)$ by Proposition \ref{pro:rank_equalities}.(i)) with $X\in M_{n,m}(\F)$. However, this can be extended to $r<m$ such that $X^TX$ is not of full rank. In this case, we take g-inverse $\brb{X^TX}^-$ to replace $\brb{X^TX}^{-1}$ in the corresponding theorems.

This is usually written in vector notation as $Y = X\beta + \ve$ where
\be
Y = \bepm
Y_1\\
\vdots\\
Y_n
\eepm,\qquad
X = \bepm
X_1\\
\vdots\\
X_n
\eepm,\qquad
\ve = \bepm
\ve_1\\
\vdots\\
\ve_n
\eepm
\ee
and $\ve \sim \sN(0,\sigma^2I_n)$. $X$ is called the design matrix\index{design matrix} and has size $n\times m$.

It is standard to assume (and easy to check in practice) that $X$ has full rank $m$. This means in particular that $X^TX$ is positive definite, because if $z$ is a non-zero vector in $\R^m$ then $z^TX^TXz = \dabs{Xz}^2 >0$ ($\dabs{\cdot}$ is Euclidean norm and $X_i$ are independent $X$ and $z^TX^TXz$ is strictly greater than zero as $X$ has full rank $m$). Note also that the error variance is assumed to be constant in each component; the model is said to be homoscedastic.



\begin{example}
Let $X_i$ be the average height of the $i$th set of parents, and let $Y_i$ be the height of their oldest child on reaching adulthood. A simple model is that $Y_1,\dots,Y_n$ are independent realisation of random variable with $Y_i = a+ bX_i + \ve_1$, where $\ve_1,\dots,\ve_n$ are i.i.d. $\sN(0,\sigma^2)$ random variables. This is a linear model with
\be
Y = \bepm Y_1\\ \vdots \\ Y_n\eepm,\qquad
X = \bepm 1 & X_1 \\ \vdots & \vdots \\ 1 & X_n\eepm,\qquad
\beta = \bepm a \\ b\eepm.
\ee
\end{example}

\begin{example}[one-way analysis of variance]
For $i\in \bra{1,\dots,m}$ and $j \in \bra{1,\dots, n_i}$, let $Y_{ij}$ denote the tripos merit mark score of the $j$th individual at the $i$th college. A simple model may be $Y_{ij} = \mu_i + \ve_{ij}$, where the $\ve_{ij}$ are i.i.d. $\sN(0,\sigma^2)$ random variables. This is a linear model since it has the form $Y = X\beta + \ve$ where
\be
Y = \bepm Y_{1,1}\\ \vdots \\ Y_{1,n_1}\\ \vdots \\ Y_{m,1}\\ \vdots \\ Y_{m,n_m} \eepm,\qquad
X = \bepm 1 & 0 & \dots & 0 \\ \vdots & \vdots & \ddots & \vdots \\ 1 & 0 & \dots & 0 \\ \dots & \dots & \dots & \dots \\ 0 & \dots & 0 & 1 \\ \vdots & \ddots & \vdots & \vdots \\ 0 & \dots & 0 & 1 \eepm,\qquad
\beta = \bepm \mu_1 \\ \vdots \\ \mu_m\eepm
\ee
where $X$ has size $(n_1+\dots + n_m)\times m$.
\end{example}

\subsection{Cochran's theorem}

In order to understand the distribution theory in the linear model, we require the following theorem.

\begin{theorem}[Cochran's theorem\index{Cochran's theorem}]\label{thm:cochran_independent_chi_square}
Let $Y \sim \sN(0,\sigma^2 I_n)$ and let $A_1,\dots, A_k$ be $n\times n$ symmetric matrices in $M_n(\R)$ with $\rank (A_i) = r_i$ and suppose
\be
A_i A_j = \left\{\ba{ll}
A_i \quad\quad & i = j\\
0& i\neq j
\ea\right.
\ee

Then $Y^TA_i Y \sim \sigma^2 \chi^2_{r_i}$ and $\bra{Y^TA_1Y,\dots,Y^TA_k Y}$ are independent.
\end{theorem}

\begin{remark}
The theorem is usually stated with the condition $A_1 + \dots + A_k = I$ instead of the condition on $A_iA_j$. However, in the presence of the other
hypotheses, these conditions are equivalent.%\footnote{see wiki for alternative form of Cochran's theorem}.
\end{remark}

\begin{proof}[\bf Proof]
Fix $i\in \bra{1,\dots,k}$. If $\lm$ is an eigenvalue of $A_i$ with eigenvector $x$, then $\lm x = A_i x = A_i^2 x = \lm^2 x$, so $\lm = 0$ or $\lm =1$. So $A_i$ is a projection matrix\footnote{definition needed}. Then we can find an orthogonal matrix $Q$ such that $Q^T A_i Q = D_i$ (by Theorem \ref{thm:spectral_hermitian_matrices} as $A_i$ is symmetric), a diagonal matrix such that the first $r_i$ diagonal entries of $D_i$ are 1 and the rest 0.

Set $Z = Q^T Y$. Then $Z\sim \sN\brb{0,\sigma^2 I}$ by Theorem \ref{thm:multivariate_gaussian_rv_property}, so
\be
Y^T A_i Y = Z^TQ^T A_i Q Z = Z^TD_i Z \sim \sigma^2\chi^2_{r_i}.
\ee

Let $M(\theta_1,\dots,\theta_k)$ be the moment generating function of $\frac 1{\sigma^2}\brb{Y^TA_1Y,\dots,Y^TA_k Y}^T$ and let $M_i(\theta_i)$ denote the moment generating function of $\frac 1{\sigma^2}Y^TA_iY$. Note that for $\abs{\theta_1},\dots,\abs{\theta_k}$ sufficiently small, the matrix $I - 2\theta_1 A_1 - \dots - 2\theta_kA_k$ and each matrix $I - 2\theta_iA_i$ are positive definite\footnote{details needed. Note the second last line of the following equations need more details, e.g. the reason to omit the higher order products}. For such $\abs{\theta_1},\dots,\abs{\theta_k}$,
\beast
M(\theta_1,\dots,\theta_k) & = & \int_{\R^n}\frac 1{(2\pi\sigma^2)^{n/2}}\exp\brb{-\frac 1{2\sigma^2}Y^T Y}\exp\brb{\theta_1 \frac{Y^TA_1Y}{\sigma^2} + \dots + \theta_k \frac{Y^TA_kY}{\sigma^2}} dY \\
& = & \int_{\R^n}\frac 1{(2\pi\sigma^2)^{n/2}}\exp\brb{-\frac 1{2\sigma^2}Y^T \brb{I - 2\theta_1 A_1 - \dots - 2\theta_k A_k} Y} dY \\
& = & \det\brb{\brb{I - 2\theta_1 A_1 - \dots - 2\theta_k A_k}^{-1}}^{1/2} \quad (\text{see Definition \ref{def:non_degenerate_multivariate_gaussian}})\\
& = & \det\brb{I - 2\theta_1 A_1 - \dots - 2\theta_k A_k}^{-1/2} \quad (\text{by Corollary \ref{cor:determinant_inverse}})\\
& & \qquad\qquad (\text{$I - 2\theta_1 A_1 - \dots - 2\theta_k A_k$ is invertible with $\abs{\theta_1},\dots,\abs{\theta_k}$ are sufficiently small \footnote{theorem needed}})\\
& = & \det\brb{\brb{I - 2\theta_1 A_1}\dots \brb{I - 2\theta_k A_k}}^{-1/2} \quad (\text{by assumption}) \\
& = & \det\brb{I - 2\theta_1 A_1}^{-1/2} \dots\det \brb{I - 2\theta_k A_k}^{-1/2} \quad (\text{by Theorem \ref{thm:determinant_product}}) \\
& = & M_1(\theta_1)\dots M_k(\theta_k).
\eeast

Observe also that by Theorem \ref{thm:determinant_product}
\be
\det\brb{I-2\theta_i A_i}^{-1/2} = \det\brb{Q^T(I - 2\theta_i A_i)Q}^{-1/2} = \det \brb{I - 2\theta_i D_i}^{-1/2} = (1-2\theta_i)^{-r_i/2}
\ee
which is the moment generating function of a $\chi^2_{r_i}$ distribution for $\theta_i < \frac 12$ (see Proposition \ref{pro:mgf_chi_squared}). Hence, by Proposition \ref{pro:independent_mgf}, we have $\bra{\frac 1{\sigma^2}Y^T A_1Y, \dots, \frac 1{\sigma^2}Y^TA_kY}$ are independent random variables, so $\bra{Y^TA_1Y, \dots, Y^TA_kY}$ are as well.
\end{proof}

\begin{theorem}[Cochran's theorem]\label{thm:cochran_equivalence}
Suppose that $Y\sim \sN(0,\sigma^2 I)$ is a standard multivariate Gaussian random variable and if $A_1,\dots,A_k$ are all $n\times n$ symmetric matrices with $\sum^k_{i=1}A_i = I_n$. Then, on defining $r_i = \rank(A_i)$, any one of the following conditions implies the other two:
\ben
\item [(i)] $\sum^k_{i=1}r_i = n$.
\item [(ii)] $A_i$ are idempotent, i.e. $A_i^2 = A_i$ for $i = 1,\dots, k$.
\item [(iii)] $A_iA_j =0$, $i\neq j$.
\een
\end{theorem}

\begin{proof}[\bf Proof]
(i) $\ra$ (ii). Since $\sum^k_{i=1}A_i = I$, we have
\be
Y^T Y = Y^T \brb{\sum^k_{i=1}A_i} Y = \sum^k_{i=1} Y^T A_i Y.
\ee

Let $B_i = \sum_{i\neq j}A_j = I - A_i$. Thus, by Proposition \ref{pro:matrix_rank_inequalities_sum}.(i) we have
\be
\rank(B_i) \leq \sum_{i\neq j} \rank(A_j) = n - r_i,\qquad n = \rank(I) \leq \rank(A_i) + \rank(B_i)
\ee
which implies that $\rank(B_i) = n-r_i$.

Note that $A_i$ and $B_i$ are diagonalizable (by  spectral theorem (Theorem \ref{thm:spectral_hermitian_matrices}) as they are symmetric matrix). Then $A_i$ and $B_i$  simultaneously diagonalizable by Theorem \ref{thm:commute_iff_simultaneously diagonalizable} since they commute ($A_iB_i = A_i(I-A_i) =(I-A_i)A_i = B_iA_i$). That is, there exists an invertible matrix $P_i$ such that $P_i^{-1}A_i P_i$ and $P_i^{-1}B_i P_i$ are both diagonal.

Thus, we know that we can have that $P_i^{-1} A_i P_i$ has the form of (as we can shift all the eigenvalue to the top left block by elementary operation (multiplying elementary matrices)
\be
\bepm
\lm_1 & 0 & \dots & & \dots & & 0 \\
0 & \lm_2 & \dots & & \dots &  & 0 \\
0 & 0 & \ddots & & & & \\
 &  & & \lm_{r_i} & & & \\
\vdots &  & &  & 0 & & \\
 &  & &  & & \ddots & \\
0 & 0 & \dots & & & & 0 \\
\eepm
\ee

Thus,
\be
P_i^{-1}B_i P_i = P_i^{-1}\brb{I - A_i}P_i = I - P_i^{-1} A_i P_i
\ee

Thus, the last $n-r_i$ diagonal elements are all +1. Since the rank of $B_i$ is $n-r_i$, the other diagonal elements of $P_i^{-1}B_i P_i$ are all zeros. Therefore, we can see that $\lm_1=\lm_2 = \dots = \lm_{r_i} =1$. Therefore, eigenvalues of $A_i$ should be $r_i$ of +1 and $n-r_i$ of zeros. Therefore,
\be
A_i^2 = \brb{P_i^{-1}\bepm I_{r_i} & 0 \\ 0 & 0 \eepm P_i}^2 = P_i^{-1}\bepm I_{r_i} & 0 \\ 0 & 0 \eepm^2 P_i = P_i^{-1}\bepm I_{r_i} & 0 \\ 0 & 0 \eepm P_i = A_i.
\ee

(i) $\ra$ (iii). Let $A_i = B_iC_i$ be a rank factorization\footnote{theorem needed.} with $B_i \in M_{n,r}(\R)$ and $C_i \in M_{r,n}(\R)$, $i=1,\dots,k$. Then
\be
I = \sum^k_{i=1}A_i = \sum^k_{i=1}B_iC_i = \bepm B_1 & \dots & B_k \eepm \bepm C_1 \\ \vdots \\ C_k\eepm .
\ee

Since $\sum^k_{i=1}\rank(A_i) = n$, then $B = \bepm B_1 & \dots & B_k \eepm$ and $C = \bepm C_1 \\ \vdots \\ C_k\eepm$ are square matrices. Then by Proposition \ref{pro:matrix_rank_inequalities_sum}.(i)
\be
\rank\brb{I} \leq \min\bra{\rank(B),\rank(C)} \qquad \ra \quad r(B) = r(C) = n
\ee
which means that $B$ and $C$ are of full rank and thus invertible. Thus
\be
BC = I \ \ra \ C = B^{-1} \ \ra\ CB = I.
\ee

Hence, $C_iB_j = 0$ for $i\neq j$. It follows that for $i\neq j$
\be
A_i A_j = B_iC_i B_jC_j = 0.
\ee

(iii) $\ra$ (ii). Since $\sum^k_{i=1}A_i = I$, then
\be
A_j \brb{\sum^k_{i=1}A_i} = A_j, \qquad j=1,\dots,k.
\ee

It follows that $A_j^2 = A_j$.

(ii) $\ra$ (i). Since $A_i$ is idempotent, $\rank(A_i) = \tr(A_i)$ as (idempotent only have eigenvalues of +1 and 0). Then
\be
\sum^k_{i=1}\rank(A_i) = \sum^k_{i=1}\tr(A_i) = \tr\brb{\sum^k_{i=1}A_i} = \tr(I) = n.
\ee
\end{proof}

\subsection{MLE for linear model}

We can now state the joint distribution of $\wh{\beta}$ and $\wh{\sigma^2}$ of linear model.

\begin{theorem}\label{thm:estimator_linear_model_mle}
For $Y\in \R^n$ and $X\in M_{n,m}(\R)$ with $m<n$ and $\rank(X) = m$, if $Y = X\beta + \ve$ with $\beta\in \R^m$ and $\ve\sim \sN\brb{0,\sigma^2 I_n}$, then
\ben
\item [(i)] the MLE $\R^m$ estimator $\wh{\beta}\sim \sN\brb{\beta,\sigma^2 \brb{X^TX}^{-1}}$.
\item [(ii)] the MLE estimator $\wh{\sigma^2} \sim \frac 1n\sigma^2 \chi^2_{n-m}$.
\item [(iii)] $\wh{\beta}$ and $\wh{\sigma^2}$ are independent.
\een
\end{theorem}

\begin{proof}[\bf Proof]
\ben
\item [(i)] For Gaussian random variable $\ve$, we have that the log-likelihood for the unknown parameter $\theta = \bepm \beta \\ \sigma^2 \eepm$ is
\be
\ell\brb{\beta,\sigma^2} = -\frac 12 n\log \sigma^2 - \frac 1{2\sigma^2} \sum^n_{i=1}(Y_i - X_i \beta)^2 = -\frac 12 n\log \sigma^2 - \frac 1{2\sigma^2} \sum^n_{i=1}\brb{Y_i - \sum^m_{j=1}X_{ij} \beta_j}^2.
\ee

A MLE $\brb{\wh{\beta}, \wh{\sigma^2}}$ satisfies
\be
0 = \left.\fp{\ell}{\beta_j}\right|_{\wh{\beta}, \wh{\sigma^2}} = \frac 1{\wh{\sigma^2}} \sum^n_{i=1} X_{ij}\brb{Y_i - \sum^m_{k=1}X_{ik} \wh{\beta}_k}.
\ee

So $\brb{X^T Y}_j = \brb{X^TX \wh{\beta}}_j$ since $X_i \wh{\beta} = \brb{X\wh{\beta}}_i$. By assumption, $X^TX$ is positive definite and thus invertible\footnote{details needed}, so we deduce
\be
\wh{\beta} = \brb{X^T X}^{-1} X^T Y.
\ee

Thus, $\wh{\beta}$ is a linear transformation of $Y$, but we know that $Y\sim \sN(X\beta,\sigma^2 I)$, hence by Theorem \ref{thm:multivariate_gaussian_rv_property},
\beast
\wh{\beta} & \sim & \sN \brb{\brb{X^T X}^{-1} X^TX\beta, \brb{X^T X}^{-1} X^T \sigma^2 I \brb{\brb{X^T X}^{-1} X^T}^T} \\
& \sim & \sN\brb{\beta, \sigma^2 \brb{X^T X}^{-1} X^T X \brb{X^T X}^{-T}}\\
& \sim & \sN\brb{\beta,\sigma^2 \brb{X^TX}^{-1}}.
\eeast

\item [(ii)] Moreover,
\be
0 =  \left.\fp{\ell}{\sigma^2}\right|_{\wh{\beta}, \wh{\sigma^2}} = - \frac n{2\wh{\sigma^2}} + \frac{1}{2\wh{\sigma^2}^2}\dabs{Y - X\wh{\beta}}^2
\ee
where $\dabs{\cdot}$ is Euclidean norm. So
\be
\wh{\sigma^2} = \frac 1n\dabs{Y -X\wh{\beta}}^2.
\ee

Observe that the fitted values $\wh{Y} = X\wh{\beta}$ can be written as $\wh{Y} = PY$, where
\be
P = X\brb{X^T X}^{-1}X^T \ \ra \ P^2 = P,\quad P^T = P,
\ee
so $P$ is an orthogonal projection\footnote{definition needed} onto the subspace of $\R^n$ spanned by the columns of $X$. We can write
\beast
\wh{\sigma^2} & = & \frac 1n\dabs{Y-X\wh{\beta}}^2 = \frac 1n\dabs{Y-X\brb{X^T X}^{-1} X^T Y}^2 = \frac 1n\dabs{Y - PY}^2\\
& = & \frac 1n Y^T \brb{I - P^T}(I-P)Y = \frac 1n Y^T(I-P)Y.
\eeast

We note that $I = P + (I-P)$ and $\lm$ is an eigenvalue of $P$ with eigenvector $Z$, then
\be
\lm Z = PZ = P^2Z = \lm^2Z \ \ra \ \lm \in \bra{0,1}.
\ee

Hence, $\rank(P) = \tr P$\footnote{Theorem needed here. Any projection $P = P^2$ on a vector space of dimension $d$ over a field is a diagonalizable matrix, since its minimal polynomial is $x^2 - x$, which splits into distinct linear factors. Or we can use the theorem that real symmetric matrices are diagonalizable by orthogonal matrices} and by Proposition \ref{pro:trace_change_order},
\be
\tr P = \tr\brb{X\brb{X^T X}^{-1}X^T} = \tr\brb{X^TX\brb{X^T X}^{-1}} = \tr I_m = m.
\ee

Similarly, the eigenvalues of $I-P$ are 0 or 1 only, so
\be
\rank(I - P) = \tr(I_n-P) = \tr I_n - tr P = n-m.
\ee

Note that
\beast
\brb{Y-X\beta}^T (I-P)(Y-X\beta) & = & Y^T(I-P)Y - \beta^TX^T(I-P) Y - Y^T(I-P)X\beta + \beta^TX^T(I-P)X\beta \\
& = & Y^T(I-P)Y.
\eeast

Thus,
\be
\wh{\sigma^2} = \frac 1n Y^T(I-P)Y = \frac 1n(Y-X\beta)^T(I-P)(Y-X\beta) = \frac 1n\ve^T(I-P)\ve \sim \frac 1n \sigma^2 \chi^2_{n-m}
\ee
by Theorem \ref{thm:cochran_independent_chi_square}.

\item [(iii)] By spectral theorem (Theorem \ref{thm:spectral_hermitian_matrices}), we can find an orthogonal matrix $Q$ such that
\be
Q^T(I-P)Q = \bepm I_{n-m} & 0 \\ 0 & 0 \eepm.
\ee

Set an $n\times (n-m)$ matrix with rank $n-m$ (by Proposition \ref{pro:rank_equalities}.(ii)), $L = Q\bepm I_{n-m}\\ 0 \eepm$ and observe that
\be
LL^T = Q\bepm I_{n-m}\\ 0 \eepm \bepm I_{n-m} & 0 \eepm Q^T = Q\bepm I_{n-m} & 0 \\ 0 & 0 \eepm Q^T = I - P
\ee
and
\be
L^TL = \bepm I_{n-m} & 0 \eepm Q^TQ\bepm I_{n-m}\\ 0 \eepm = I_{n-m}.
\ee

Also note that if $B = (X^TX)^{-1}X^T$ (which is an $m\times n$ matrix), then
\be
BL = BLL^TL = (X^TX)^{-1}X^T \brb{I - P}L = (X^TX)^{-1}X^T \brb{I - X\brb{X^T X}^{-1}X^T}L = 0.
\ee

So it follows that $\bepm B \\ L^T \eepm Y$ is a linear transformation of $Y$, hence is normally distributed with covariance matrix (by Theorem \ref{thm:multivariate_gaussian_rv_property}.(i))
\be
\sigma^2 \bepm B\\ L^T \eepm \bepm B^T & L\eepm = \sigma^2 \bepm \brb{X^TX}^{-1} & 0 \\ 0 & I_{n-m}\eepm.
\ee

Thus, $\wh{\beta} = BY$ and
\be
\wh{\sigma^2} = \frac 1n\dabs{Y - PY}^2 = \frac 1n Y^T (I-P)^T(I-P)Y = \frac 1n Y^T (I-P)Y = \frac 1n Y^T LL^T Y = \frac 1n \dabs{L^TY}^2
\ee
are independent by Theorem \ref{thm:multivariate_gaussian_rv_property}.(v) and Theorem \ref{thm:random_variable_function_indenpence}.
\een
\end{proof}



\subsection{Ordinary least squares}

\begin{theorem}[estimator of OLS]
The estimator $\wh{\beta}$ to minimize the sum of squared error is
\be
\wh{\beta} = \argmin_{\beta}\brb{Y-X\beta}^T \brb{Y-X\beta}  = (X^TX)^{-1}X^T Y.
\ee
\end{theorem}

\begin{proof}[\bf Proof]
\footnote{matrix calculus needed.}
\end{proof}

\begin{proposition}
The minimum of $\brb{Y-X\beta}^T \brb{Y-X\beta}$ is attained at OLS estimator $\wh{\beta}$.
\end{proposition}


\begin{proof}[\bf Proof]
We have
\beast
\brb{Y-X\beta}^T \brb{Y-X\beta} & = & \brb{Y-X\wh{\beta} + X\wh{\beta} - X\beta}^T \brb{Y-X\wh{\beta} + X\wh{\beta} - X\beta} \\
& = & \brb{Y-X\wh{\beta} }^T \brb{Y-X\wh{\beta}}+ \brb{\wh{\beta} - \beta}^T X^T X\brb{\wh{\beta} - \beta}
\eeast
since
\be
X^T\brb{Y - X\wh{\beta}} = X^T\brb{Y - X(X^TX)^{-1}X^T Y} = \brb{X^T - X^TX(X^TX)^{-1}X^T}Y = 0.
\ee

It follows that
\be
\brb{Y-X\beta}^T \brb{Y-X\beta} \geq \brb{Y-X\wh{\beta} }^T \brb{Y-X\wh{\beta}}
\ee
as claimed.
\end{proof}

\begin{example}
Suppose four objects are to be weighted using an ordinary chemical balance (without bias) with two pans. We are allowed four weighings. In each weighing we may put some of the objects in the right pan and some in the left pan. Any procedure that specifies this allocation is called a weighing design.

Let $\beta_1,\beta_2,\beta_3,\beta_4$ be the true weights of the objects. Define $x_{ij} =1$ or $-1$ depending upon whether we put the $j$th object in the right pan or in the left pan in the $i$th weighing. Let $Y_i$ denote the weight needed to achieve balance in the $i$th weighing. If the sign of $Y_i$ is positive, then the weight is required in the left pan, otherwise in the right pan.

Then we have the model $\E Y = X\beta$ where $X = \brb{x_{ij}}$ is real, $Y$ is the $4\times 1$ vector with components $Y_i$, and $\beta$ is the $4\times 1$ vector with components $\beta_i$. As usual, we make the assumption that the $Y_i$s are uncorrelated with common variance $\sigma^2$. The dispersion matrix of $\wh{\beta}$ is $\sigma^2 \brb{X^TX}^{-1}$, assuming $X^TX$ to be nonsingular.

Thus to get more precision we must make the matrix $X^TX$ `large'. One measure of largeness of a positive semi-definite matrix is the determinant. This is the D-optimality criterion\footnote{details needed.}. %, which we will encounter again in the context of block designs)
The matrix
\be
X = \bepm 1 & 1 & 1 & 1 \\ 1 & -1 & 1& -1 \\ 1 & 1 & -1 & -1 \\ 1 & -1 & -1 & 1 \eepm
\ee
satisfies $\det\brb{X^TX} = 4^4$, and by Corollary \ref{cor:positive_definite_determinant_smaller_than_n_power_n}, this is the maximum determinant possible. Note that the matrix $X$ is a Hadamard matrix (see Definition \ref{def:hadamard_matrix}). That is for $H_2 = \bepm 1 & 1 \\ 1 & -1 \eepm$,
\be
X = H_2 \otimes H_2.
\ee

Note that the matrix $X$ to attain the maximum determinant is not unique and it can be
\be
X = \bepm 1 & 1 & 1 & 1 \\ 1 & 1 & -1 & -1  \\ 1 & -1 & -1 & 1 \\  1 & -1 & 1& -1 \eepm
\ee
as well.
\end{example}



\begin{definition}[residual sum of squares, explained sum of squares, total sum of squares]
Suppose we estimate the linear model parameters with OLS. Then the residual sum of squares (RSS\index{residual sum of squares, RSS}) is defined to be
\beast
RSS & = & (Y-\wh{Y})^T(Y-\wh{Y}) = (Y-X\wh{\beta})^T(Y-X\wh{\beta}) \\
& = & Y^TY - 2\wh{\beta}^TX^T Y + \wh{\beta}^T X^T X \wh{\beta} \\
& = & Y^TY - 2Y^T X (X^TX)^{-1} X^T Y + Y^T X (X^TX)^{-1} X^T X (X^TX)^{-1}X^T Y \\
& = & Y^TY - Y^T X (X^TX)^{-1} X^T Y = Y^T \brb{I - P}Y.
\eeast

The total sum of squares (TSS\index{total sum of squares, TSS}) is defined to be
\be
TSS = (Y-\ol{Y}\onevec_{n\times 1})^T(Y-\ol{Y}\onevec_{n\times 1}) = Y^TY - 2 \ol{Y} Y^T \onevec_{n\times 1} + \ol{Y}^2\onevec_{n\times 1}^T \onevec_{n\times 1}.
\ee
where $\ol{Y} = \frac 1n \onevec_{1\times n} Y$.

The explained sum of squares (ESS\index{explained sum of squares, ESS}) is defined to be
\be
ESS = (\wh{Y}-\ol{Y}\onevec_{n\times 1})^T(\wh{Y}-\ol{Y}\onevec_{n\times 1}) = \wh{Y}^T\wh{Y} - 2\ol{Y} \wh{Y}^T  \onevec_{n\times 1} +  \ol{Y}^2\onevec_{n\times 1}^T \onevec_{n\times 1}.
\ee
\end{definition}


\begin{proposition}\label{pro:tss_equals_rss_ess_iff_mean_times_est_error_zero}
With the linear model setup $Y = X\beta +\ve$, we have
\be
TSS = RSS + ESS
\ee
iff $\ol{Y}(Y- \wh{Y})^T\onevec_{n\times 1} = 0$.
\end{proposition}

\begin{proof}[\bf Proof]
First, we have
\beast
TSS & = & (Y-\ol{Y}\onevec_{n\times 1})^T(Y-\ol{Y}\onevec_{n\times 1}) = (Y-\wh{Y}+ \wh{Y} - \ol{Y}\onevec_{n\times 1})^T(Y-\wh{Y}+ \wh{Y} -\ol{Y}\onevec_{n\times 1}) \\
& = & (Y-\wh{Y})^T(Y-\wh{Y}) + (\wh{Y} - \ol{Y}\onevec_{n\times 1})^T(\wh{Y} -\ol{Y}\onevec_{n\times 1}) + 2(Y-\wh{Y})^T(\wh{Y} -\ol{Y}\onevec_{n\times 1})\\
& = & RSS + ESS + 2(Y-\wh{Y})^T\wh{Y} - 2\ol{Y} (Y- \wh{Y})^T\onevec_{n\times 1}.
\eeast

In particular,
\be
(Y-\wh{Y})^T\wh{Y} = Y^T(I-X(X^TX)^{-1}X^T)X(X^TX)^{-1}X^TY = Y^T\brb{X(X^TX)^{-1}X^T - X(X^TX)^{-1}X^T}Y = 0.
\ee

Thus, $TSS = RSS + ESS$ iff $\ol{Y}(Y- \wh{Y})^T\onevec_{n\times 1} = 0$.
\end{proof}

\begin{theorem}\label{thm:tss_rss_ess_constant_x}
With the linear model setup $Y = X\beta +\ve$, we have $TSS = RSS + ESS$ if $X$ has a non-zero constant column.
\end{theorem}

\begin{proof}[\bf Proof]
According to Proposition \ref{pro:tss_equals_rss_ess_iff_mean_times_est_error_zero}, it suffices to show that
\be
0 = \onevec_{1\times n}(Y- \wh{Y}) = \onevec_{1\times n} \brb{I - X(X^TX)^{-1}X^T}Y.
\ee

If $X$ has a non-zero constant column, it is equivalent to consider $X = \brb{\onevec_{n\times 1},Z_{n\times (n-1)}}$ with $\rank(Z) = n-1$. Therefore,
\be
X^TX = \bepm \onevec_{1\times n} \\ Z^T \eepm \bepm \onevec_{n\times 1} & Z\eepm = \bepm n & \onevec_{1\times n} Z \\ Z^T\onevec_{n\times 1} & Z^TZ \eepm.
\ee

Then by Schur's theorem (Theorem \ref{thm:schur_inverse_a}), Schur complement of $n$ in $X^TX$ is $W = (X^TX)/\brb{n}$. Thus,
\be
(X^TX)^{-1} = \bepm \frac 1n + \frac 1{n^2}\onevec_{1\times n}Z W^{-1}Z^T\onevec_{n\times 1} & \ - \frac 1n \onevec_{1\times n}ZW^{-1}\ \\ - \frac 1n W^{-1}Z^T\onevec_{n\times 1} & W^{-1} \eepm
\ee

Furthermore,
\beast
X(X^TX)^{-1}X^T & = &  \bepm \onevec_{n\times 1} & Z\eepm \bepm \frac 1n + \frac 1{n^2}\onevec_{1\times n}Z W^{-1}Z^T\onevec_{n\times 1} & \ - \frac 1n \onevec_{1\times n}ZW^{-1}\ \\ - \frac 1n W^{-1}Z^T\onevec_{n\times 1} & W^{-1} \eepm   \bepm \onevec_{1\times n} \\ Z^T \eepm \\
& = &  \bepm \frac 1n \onevec_{n\times 1} + \frac 1{n^2} \onevec_{n\times 1} \onevec_{1\times n}Z W^{-1}Z^T\onevec_{n\times 1} - \frac 1n Z W^{-1}Z^T\onevec_{n\times 1}  & \ - \frac 1n \onevec_{n\times 1} \onevec_{1\times n}ZW^{-1} +  Z W^{-1} \eepm   \bepm \onevec_{1\times n} \\ Z^T \eepm \\
& = & \frac 1n \onevec_{n\times n} + \frac 1{n^2} \onevec_{n\times n}Z W^{-1}Z^T\onevec_{n\times n} - \frac 1n Z W^{-1}Z^T\onevec_{n\times n }  - \frac 1n \onevec_{n\times n}ZW^{-1} Z^T +  Z W^{-1}Z^T.
\eeast

Then,
\beast
& & \onevec_{1\times n} \brb{I - X(X^TX)^{-1}X^T} \\
& = & \onevec_{1\times n}\brb{I - \frac 1n \onevec_{n\times n} - \frac 1{n^2} \onevec_{n\times n}Z W^{-1}Z^T\onevec_{n\times n} + \frac 1n Z W^{-1}Z^T\onevec_{n\times n }  + \frac 1n \onevec_{n\times n}ZW^{-1} Z^T -  Z W^{-1}Z^T} \\
& = & \onevec_{1\times n} - \frac 1n n\onevec_{1\times n} - \frac 1{n^2}n \onevec_{1\times n} Z W^{-1}Z^T\onevec_{n\times n} + \frac 1n \onevec_{1\times n} Z W^{-1}Z^T\onevec_{n\times n }  + \frac 1n n \onevec_{1\times n} ZW^{-1} Z^T - \onevec_{1\times n} Z W^{-1}Z^T \\
& = & 0.
\eeast

Thus, $\onevec_{1\times n}(Y- \wh{Y}) = 0$ as claimed.
\end{proof}

\begin{definition}[$R$ square]
For linear model with constant $X$ column, we define the $R$ square by
\be
R^2 = \frac{ESS}{TSS} = 1 - \frac{RSS}{TSS}.
\ee
\end{definition}

\begin{remark}
$R^2$ is between 0 and 1 and this is guaranteed by Theorem \ref{thm:tss_rss_ess_constant_x}.

Note that the larger $R^2$, the better the linear model fits.
\end{remark}


\begin{theorem}\label{thm:estimators_ordinary_least_square}
The estimators $\wh{\beta} = (X^TX)^{-1}X^TY$ and $s^2 = \wh{\sigma^2} = \frac{RSS}{n-m} = \frac 1{n-m}(Y-X\wh{\beta})^T(Y-X\wh{\beta})$ are independently distributed with
\be
\wh{\beta} \sim \sN\brb{\beta, \sigma^2\brb{X^TX}^{-1}},\qquad \wh{\sigma^2} \sim \frac{\sigma^2}{n-m} \chi^2_{n-m}.
\ee
\end{theorem}

\begin{remark}
Clearly, $\E s^2 = \sigma^2$ implies that $s^2$ is unbiased estimator of $\sigma^2$.
\end{remark}

\begin{proof}[\bf Proof]
This is from Theorem \ref{thm:estimator_linear_model_mle}.
\end{proof}

\begin{proposition}[$t$-test for coefficient $\beta_i$]
Suppose the null hypothesis is that $\beta_i$, the $i$th element of $\beta$, is equal to some particular value $\beta_i^0.$ Then OLS $t$-statistic for testing this null hypothesis is given by
\be
t = \frac{\wh{\beta_i} - \beta_i^0}{s\sqrt{\xi_{ii}}} := \frac{\wh{\beta_i} - \beta_i^0}{se_i}
\ee
where $\xi_{ii}$ denotes the row $i$, column $i$ element of $(X^TX)^{-1}$ and $se_i = s\sqrt{\xi_{ii}} = \sqrt{\frac{RSS\cdot \xi_{ii}}{n-m}}$ is the standard error of the OLS estimate of the $i$th coefficient.

Under null hypothesis, the magnitude $t$ has an exact $t$-distribution with $n-m$ degrees of freedom so long as $X$ is deterministic and $\ve$ is i.i.d. Gaussian.
\end{proposition}

\begin{remark}
Usually, $\beta_i^0$ is taken to be 0. Thus, if the null hypothesis is rejected, then we can say that $\beta_i$ is significant (from 0).
\end{remark}

\begin{proof}[\bf Proof]
By Theorem \ref{thm:estimators_ordinary_least_square},
\be
\brb{\wh{\beta}_i - \beta_i}/\brb{\sigma \sqrt{\xi_{ii}}} \sim \sN(0,1),\qquad s^2/\sigma^2 \sim \chi^2_{n-m}/\brb{n-m}.
\ee

Then by Proposition \ref{pro:normal_chi_square_t} as these two random variables are independent,
\be
\frac{\brb{\wh{\beta}_i - \beta_i}/\brb{\sigma \sqrt{\xi_{ii}}} }{s/\sigma} \sim T\brb{n-m}\ \ra \ \frac{\wh{\beta}_i - \beta_i}{ s\sqrt{\xi_{ii}}} \sim T\brb{n-m}
\ee
\end{proof}

\subsection{Ordinary least square with constraints}

\begin{theorem}[estimator of OLS with constraints]
The estimator $\wt{\beta}$ to minimize the sum of squared error with constraint $R\beta =Z$ where  $\rank(X)=m$, $R\in M_{k,m}(\F)$ with $\rank(R) = k$ and $Z \in M_{k,1}(\F)$ is
\beast
\wt{\beta} & = & \argmin_{\beta}\brb{Y-X\beta}^T \brb{Y-X\beta} \\
& = & (X^TX)^{-1}X^T Y - (X^TX)^{-1}R^T\brb{R(X^TX)^{-1}R^T}^{-1}\brb{R(X^TX)^{-1}X^T Y-Z} \\
& = &  \wh{\beta} - (X^TX)^{-1}R^T\brb{R(X^TX)^{-1}R^T}^{-1}\brb{R \wh{\beta}-Z}.
\eeast
\end{theorem}

\begin{proof}[\bf Proof]
\footnote{matrix calculus needed. see Hamilton, chinese version, p269.}
\end{proof}

\begin{proposition}
The minimum of $\brb{Y-X\beta}^T \brb{Y-X\beta}$ with constraint  $R\beta =Z$ where $\rank(X)=m$, $R\in M_{k,m}(\F)$ with $\rank(R) = k$ and $Z \in M_{k,1}(\F)$ is attained at OLS estimator
\beast
\wt{\beta} & = & (X^TX)^{-1}X^T Y - (X^TX)^{-1}R^T\brb{R(X^TX)^{-1}R^T}^{-1}\brb{R(X^TX)^{-1}X^T Y-Z}\\
& = & \wh{\beta} - (X^TX)^{-1}R^T\brb{R(X^TX)^{-1}R^T}^{-1}\brb{R \wh{\beta}-Z}.
\eeast
\end{proposition}


\begin{proof}[\bf Proof]
Since $\sR(R^T) \subseteq \sR(X^T)$ and since $\rank(X) = \rank(X^TX)$\footnote{theorem needed.}, $R = WX^TX$ for some $W$,\footnote{theorem needed.} Now let $S = WX^T$, $R = SX$
\be
R(X^TX)^{-1}R^T = WX^TX (X^TX)^{-1} X^TX W^T = WX^TXW^T = SS^T.
\ee

For some $V$, $RV = Z$, we have
\beast
R(X^TX)^{-1}R^T \brb{R(X^TX)^{-1}R^T}^{-1}Z & = &  R(X^TX)^{-1}R^T \brb{R(X^TX)^{-1}R^T}^{-1}RV \\
& = & SS^T \brb{SS^T}^{-1}SXV  = SXV = RV = Z.
\eeast

Similarly,
\beast
R(X^TX)^{-1}R^T \brb{R(X^TX)^{-1}R^T}^{-1}R\wh{\beta} & = & SS^T (SS^T)^{-1} WX^TX (X^TX)^{-1}X^TY \\
& = & WX^TY = SY
\eeast
and
\be
R\wh{\beta} = WX^TX(X^TX)^{-1}X^TY = WX^TY = SY.
\ee

Then we can have that $R\wt{\beta} = Z$.

Thus, we have
\beast
\brb{Y-X\beta}^T \brb{Y-X\beta} & = & \brb{Y-X\wt{\beta} + X\wt{\beta} - X\beta}^T \brb{Y-X\wt{\beta} + X\wt{\beta} - X\beta} \\
& = & \brb{Y-X\wt{\beta} }^T \brb{Y-X\wt{\beta}}+ \brb{\wt{\beta} - \beta}^T X^T X\brb{\wt{\beta} - \beta}
\eeast
since we can show that $(\wt{\beta} -\beta)^T X^T(Y-X\wt{\beta}) = 0$ as follows. We have
\beast
X^TX\wt{\beta} & = & X^TX \wh{\beta} - X^TX(X^TX)^{-1} R^T(R(X^X)^{-1}R^T)^{-1}\brb{R\wh{\beta}-Z} \\
& = & X^TY - X^TX(X^TX)^{-1} R^T(R(X^X)^{-1}R^T)^{-1}\brb{R\wh{\beta}-Z} \\
& = & X^TY - R^T(R(X^X)^{-1}R^T)^{-1}\brb{R\wh{\beta}-Z} .
\eeast

Thus, $X^T(Y-X\wt{\beta}) = R^T(R(X^X)^{-1}R^T)^{-1}\brb{R\wh{\beta}-Z}$. Also, since $R\wt{\beta} = R\beta = 0$,
\beast
(\wt{\beta} -\beta)^T X^T(Y-X\wt{\beta}) & = & (\wt{\beta} -\beta)^T R^T(R(X^X)^{-1}R^T)^{-1}\brb{R\wh{\beta}-Z} \\
& = &  \brb{R(\wt{\beta} -\beta)}^T(R(X^X)^{-1}R^T)^{-1}\brb{R\wh{\beta}-Z}  = 0.
\eeast

It follows that
\be
\brb{Y-X\beta}^T \brb{Y-X\beta} \geq \brb{Y-X\wt{\beta} }^T \brb{Y-X\wt{\beta}}
\ee
as claimed.
\end{proof}


\begin{proposition}[$F$-test for coefficient $\beta$]
Suppose the null hypothesis contains $k$ different linear restrictions about $\beta$, as represented by:
\be
H_0:\ R\beta = Z\quad \text{against}\quad H_1:\ R\beta \neq Z
\ee
where $R\in M_{k,m}(\F)$ with $\rank(R) = k$ and $Z \in M_{k,1}(\F)$. Note that it is natural that $\sR(R^T) \subseteq \sR(X^T)$ as $X$ is full rank ($\rank(X) = m$)\footnote{We will discuss the case that $X$ is not full rank later, which needs $\sR(R^T) \subseteq \sR(X^T)$ indeed.}. Then OLS $F$-statistic for testing this null hypothesis is given by Wald form
\be
F = (R\wh{\beta} - Z)^T \brb{s^2 R(X^TX)^{-1}R^T}^{-1}(R\wh{\beta} - Z)/k.
\ee
where
\be
s^2 = \frac{RSS}{n-m} = \frac 1{n-m}(Y-X\wh{\beta})^T(Y-X\wh{\beta})
\ee
and $\wh{\beta} = (X^TX)^{-1}X^T Y$ is estimator of OLS without constraint. Equivalently, we have
\be
F = \frac{(RSS_H - RSS)/k}{RSS/(n-m)}.
\ee
where
\be
\wt{\beta} = \wh{\beta} - (X^TX)^{-1}R^T\brb{R(X^TX)^{-1}R^T}^{-1}\brb{R\wh{\beta}-Z}
\ee
is estimator of OLS with constraints and
\be
RSS = (Y-X\wh{\beta})^T(Y-X\wh{\beta}),\qquad RSS_H = (Y-X\wt{\beta})^T(Y-X\wt{\beta}).
\ee

Under null hypothesis, the magnitude $F$ has an exact $F$-distribution with parameters $k$ and $n-m$ so long as $X$ is deterministic and $\ve$ is i.i.d. Gaussian. That is,
\be
F \sim F(k,n-m).
\ee
\end{proposition}

\begin{proof}[\bf Proof]
By Theorem \ref{thm:estimators_ordinary_least_square} and Theorem  \ref{thm:multivariate_gaussian_rv_property}.(i), we have that
\be
R\wh{\beta} \sim \sN\brb{R\beta, \sigma^2R(X^TX)^{-1}R^T} \sim  \sN\brb{Z, \sigma^2R(X^TX)^{-1}R^T}
\ee
under hull hypothesis. Then by Proposition \ref{pro:multivariate_normal_convert_to_chi_square},
\be
L = (R\wh{\beta}-Z)^T \brb{\sigma^2R(X^TX)^{-1}R^T}^{-1}(R\wh{\beta}-Z) \sim \chi^2_k.
\ee

Also, $s^2 = RSS/(n-m) \sim \frac{\sigma^2}{n-m}\chi^2_{n-m}$ and $\wh{\beta}$ are independnent by Theorem \ref{thm:estimators_ordinary_least_square}, Thus, $L$ and $s^2$ are independent.
\beast
F & = & (R\wh{\beta} - Z)^T \brb{s^2 R(X^TX)^{-1}R^T}^{-1}(R\wh{\beta} - Z)/k  \\
& = & \frac 1{s^2} (R\wh{\beta} - Z)^T \brb{\sigma^2 R(X^TX)^{-1}R^T}^{-1}(R\wh{\beta} - Z)/k  = \frac{L/k}{s^2} \sim F(k,n-m).
\eeast

We can let $R = WX^TX$, then
\be
\wt{\beta} = \wh{\beta} - W^T(WX^TXW^T)^{-1}(WX^TY -Z).
\ee

\beast
RSS_H - RSS & = & 2Y^TX(\wh{\beta} - \wt{\beta}) + \wt{\beta}^T X^TX\wt{\beta} -  \wh{\beta}^T X^TX\wh{\beta} \\
& = & 2 Y^TX W^T(WX^TXW^T)^{-1}(R\wh{\beta} - Z) - 2\wh{\beta}^TX^TXW^T(WX^TXW^T)^{-1}(R\wh{\beta} - Z) \\
& & \qquad + \brb{W^T(WX^TXW^T)^{-1}(R\wh{\beta} - Z)}^T \ X^TX\ W^T(WX^TXW^T)^{-1}(R\wh{\beta} - Z)\\
& = & \brb{(R\wh{\beta} - Z)}^T (WX^TXW^T)^{-1} W X^TX W^T(WX^TXW^T)^{-1}(R\wh{\beta} - Z)\\
& = & (R\wh{\beta} - Z)^T \brb{R(X^TX)^{-1}R^T}^{-1}(R\wh{\beta} - Z)
\eeast
which is the required result.\footnote{See the alternative proof in \cite{Bapat_2012}.$P_{90}$}
\end{proof}

%\section{Analysis of Variance (ANOVA)}

\subsection{One-way ANOVA}

\begin{theorem}[one-way ANOVA]
For model\footnote{see Linear Algebra and Linear Model, R.B.Bapat, 2011, Springer.}
\be
Y_{ij} = \mu + \alpha_i + \ve_{ij}, \qquad i = 1,\dots,k,\quad j = 1,\dots,n_i
\ee
where we now assume that $\ve_{ij}$ are independent $\sN(0,\sigma^2)$.\footnote{Note that $n_i$ might be different for different $i$.} Also, we have
\be
\ol{Y}_{\cdot\cdot} = \frac 1n\sum^k_{i=1} \sum^{n_i}_{j=1} Y_{ij},\qquad \ol{Y}_{i\cdot} = \frac 1{n_i}\sum^{n_i}_{j=1} Y_{ij} .
\ee

Suppose we wish to test the hypothesis
\be
H_0:\ \alpha_1 = \alpha_2 = \dots = \alpha_k.
\ee

Then we calculate the following
\begin{center}%\begin{table}
\begin{tabular}{lllllllll}
\hline
Source& & degrees of freedom & & Sum of squares & & Mean sum of squares & & $F$-statistic\\
\hline
treatments & & $k-1$ & &  $SSA = \sum^k_{i=1} n_i \brb{\ol{Y}_{i\cdot} - \ol{Y}_{\cdot\cdot}}^2$ & &   $MSA = SSA/(k-1)$ & &   \\
error & &  $n-k$ & &  $SSE = \sum^k_{i=1} \sum^{n_i}_{j=1}\brb{Y_{ij} - \ol{Y}_{i\cdot}}^2$ & &  $MSE = SSE/(n-k)$ & &  $MSA/MSE$ \\
total & &  $n-1$ & &  $SST = \sum^k_{i=1} \sum^{n_i}_{j=1} \brb{Y_{ij} - \ol{Y}_{\cdot\cdot}}^2$ & & & &  \\
\hline
\end{tabular}%\end{table}
\end{center}
and have that
\be
\frac{MSA}{MSE} = \frac{\sum^k_{i=1} n_i \brb{\ol{Y}_{i\cdot} - \ol{Y}_{\cdot\cdot}}^2/(k-1)}{\sum^k_{i=1}\sum^{n_i}_{j=1}\brb{Y_{ij} - \ol{Y}_{i\cdot}}^2/(n-k)} \sim F(k-1,n-k).
\ee
under $H_0$. Note that we use $SST$, $SSE$ and $SSA$ for total sum of squares, error sum of squares and treatment sum of squares, respectively.
\end{theorem}

\begin{proof}[\bf Proof]
Let
\be
z_{ij} = \frac{Y_{ij}-\mu -\alpha}{\sigma}
\ee
which is standard normal if $H_0$ is true and where $\alpha$ denotes the common value of $\alpha_1,\dots,\alpha_k$. Let $z$ be the vector
\be
\brb{z_{11},\dots,z_{1n_1},z_{21},\dots,z_{2n_2},\dots,z_{k1},\dots,z_{kn_k}}^T.
\ee

Then
\beast
\sum^k_{i=1}\sum^{n_i}_{j=1} z_{ij}^2 & = & \sum^k_{i=1}\sum^{n_i}_{j=1} \brb{z_{ij} - \ol{z}_{i\cdot} +  \ol{z}_{i\cdot}  - \ol{z}_{\cdot\cdot} +  \ol{z}_{\cdot\cdot} }^2 \\
& = & \sum^k_{i=1}\sum^{n_i}_{j=1} \brb{z_{ij} - \ol{z}_{i\cdot}}^2 + \sum^k_{i=1} n_i\brb{ \ol{z}_{i\cdot}  - \ol{z}_{\cdot\cdot}}^2  +  n\ol{z}_{\cdot\cdot}^2 \qquad (*)
\eeast
since the cross-product terms equal zero. That is,
\be
\sum^k_{i=1}\sum^{n_i}_{j=1} \brb{z_{ij} - \ol{z}_{i\cdot}}\brb{ \ol{z}_{i\cdot}  - \ol{z}_{\cdot\cdot}} = \sum^k_{i=1} \brb{ \ol{z}_{i\cdot}  - \ol{z}_{\cdot\cdot}}\sum^{n_i}_{j=1}\brb{z_{ij} - \ol{z}_{i\cdot}} = 0,
\ee
\be
\ol{z}_{\cdot\cdot} \sum^k_{i=1}\sum^{n_i}_{j=1} \brb{z_{ij} - \ol{z}_{i\cdot}} = \ol{z}_{\cdot\cdot} \sum^k_{i=1} 0 = 0
\ee
\be
\ol{z}_{\cdot\cdot} \sum^k_{i=1}\sum^{n_i}_{j=1} \brb{ \ol{z}_{i\cdot}  - \ol{z}_{\cdot\cdot}} = \ol{z}_{\cdot\cdot} \sum^k_{i=1} \brb{ \ol{z}_{i\cdot}  - \ol{z}_{\cdot\cdot}}\sum^{n_i}_{j=1} 1 = \ol{z}_{\cdot\cdot} \sum^k_{i=1} n_i\brb{ \ol{z}_{i\cdot}  - \ol{z}_{\cdot\cdot}} = 0.
\ee

Let $A_1,A_2,A_3\in M_n(\F)$ be symmetric matrices such that the quadratic forms
\be
z^TA_1 z,\quad z^TA_2 z,\quad z^TA_3z
\ee
are the three forms of $(*)$ repectively. Since each form is a sum of squares, $A_i$ are in fact, positive semi-definite.

Note that $\sN(A_3)$ is the orthogonal complement\footnote{details needed.} of $\onevec_n$, the vector of all ones, and hence $\dim(\sN(A_3)) = n-1$, $\rank(A_3) = 1$. Furthermore, $\sN(A_1)$ is spanned by
\be
\bepm
\onevec_{n_1}\\ 0 \\ \vdots \\ 0 \eepm,\quad \bepm 0 \\ \onevec_{n_2}\\ \vdots \\ 0 \eepm ,\quad \dots\quad , \bepm 0 \\ 0\\ \vdots \\ \onevec_{n_k}\eepm
\ee
and so $\dim(\sN(A_1)) = k$, $\rank(A_1) = n-k$.

Now, since $A_1 + A_2 + A_3 = I$, then $\sum^3_{i=1}\rank(A_i) \geq n$,\footnote{inequality needed.} and hence,
\be
\rank(A_2) \geq n - \rank(A_1) - \rank(A_3) = n -(n-k) - 1 = k-1.\qquad (\dag)
\ee

We next observe that for vectors $x_1\perp \onevec_{n_1},\dots,x_k\perp \onevec_{n_k}$,
\be
\bepm x_1\\ 0 \\ \vdots \\ 0 \eepm,\quad \bepm 0 \\ x_2\\ \vdots \\ 0 \eepm ,\quad \dots\quad , \bepm 0 \\ 0\\ \vdots \\ x_k \eepm
\ee
are linearly independent vectors in $\sN(A_2)$.\footnote{details needed.} Thus, we can generate $(n_1 -1) + \dots (n_k-1) = n-k$ linearly independent vectors in $\sN(A_2)$. These $n-k$ vectors, along with $\onevec_{n}$, give $n-k+1$ linear independent vectors in $\sN(A_2)$. This is true as $\bepm \onevec_{n_1}^T & \onevec_{n_2}^T & \dots & \onevec_{n_k}^T \eepm = \onevec_n^T$, and $\bepm x_{n_1}^T & 0 \dots & 0 \eepm^T \perp \onevec_{n}$, etc. Thus by rank-nullity theorem\footnote{theorem needed.}
\be
\dim(\sN(A_2)) \geq n-k+1 \ \ra \ \rank(A_2) \leq k-1.
\ee

Combining this observation and $(\dag)$, we see that $\rank(A_2) = k-1$. Hence $\sum^3_{i=1}\rank(A_i) = n$. We conclude by Cochran's theorem (Theorem \ref{thm:cochran_independent_chi_square} and Theorem \ref{thm:cochran_equivalence}) that $z^TA_1z$, $z^TA_2z$ are independent chi-square variables. It remains to find the degrees of freedom. These are $\rank(A_1)$, $\rank(A_2)$ respectively, which we have already seen to be $n-k$ and $k-1$, respectively. Therefore, under $H_0$, by Proposition \ref{pro:independent_chi_square_ratio_implies_f},
\be
\frac{\sum^k_{i=1} n_i \brb{\ol{z}_{i\cdot} - \ol{z}_{\cdot\cdot}}^2/(k-1)}{\sum^k_{i=1}\sum^{n_i}_{j=1}\brb{z_{ij} - \ol{z}_{i\cdot}}^2/(n-k)} \sim F(k-1,n-k).
\ee

In terms of $Y_{ij}$ we write this as the required result.
\end{proof}

\begin{example}
\footnote{example needed.}
\end{example}


\subsection{Two-way ANOVA}



\subsection{Generalized least squares}



\section{Sample Statistics}

\subsection{Distribution of sample variance}

Let $X_1,\dots,X_n$ be i.i.d. Gaussian random variables with variance $\sigma^2$. Then by Theorem \ref{thm:normal_rv_sample_mean_variance}, we know that
\be
(n-1)\wh{\sigma}^2 := \sum^n_{i=1}\brb{X_i - \ol{X}}^2 \sim \sigma^2 \chi^2_{n-1},\qquad \ol{X} = \frac 1n\sum^n_{i=1} X_i.
\ee

Note that $\wh{\sigma}^2$ is an unbiased estimator of $\sigma^2$. Define
\be
\chi^2_n\brb{\alpha} = F^{-1}_X\brb{1-\alpha} ,\quad X\sim \chi^2_n.
\ee

Therefore, the two-sided critical values of confidence level $1-\alpha$ is
\be
\chi^2_{n-1}\brb{1-\alpha/2} = \frac{(n-1)\wh{\sigma}^2}{\sigma^2} \leq \chi^2_{n-1}\brb{\alpha/2}  .
\ee

If $\alpha = 0.05$ and $n= 27$, we have $\chi^2_{n-1}\brb{0.975} = 13.844$ and $\chi^2_{n-1}\brb{0.025} = 41.923$. Thus,
\be
13.844/26 = 0.532 \leq \brb{\frac{\wh{\sigma}}{\sigma}}^2 \leq 1.612 = 41.923/26.
\ee

For large $n$, we have\footnote{theorem needed.}
\be
\chi^2_{k}(\alpha) = \frac 12 \brb{z_{\alpha} + \sqrt{2k-1}}^2.
\ee
where $z_{\alpha}$ is the critical value of standard gaussian random variable of confidence level $1-\alpha$.

If $\alpha = 0.05$ and $n = 10000$, we have that $z_{0.975} = -1.96$ and $z_{0.025} = 1.96$. Then
\be
\frac {\brb{z_{0.975} + \sqrt{2n-3}}^2}{2(n-1)} \leq \brb{\frac{\wh{\sigma}}{\sigma}}^2 \leq \frac {\brb{z_{0.025} + \sqrt{2n-3}}^2}{2(n-1)}
\ee
which implies that
\be
0.9724 \leq \brb{\frac{\wh{\sigma}}{\sigma}}^2 \leq 1.0279 \ \ra\ 0.9861 \leq \frac{\wh{\sigma}}{\sigma} \leq 1.0138.
\ee

\section{Other Theorems}

\subsection{EM algorithm}


\section{Principal Component Analysis (PCA)}

Principal component analysis (PCA) is a statistical procedure that uses an orthogonal transformation to convert a set of observations of possibly correlated variables into a set of values of linearly uncorrelated variables called principal components (or sometimes, principal modes of variation). The number of principal components is less than or equal to the smaller of the number of original variables or the number of observations. This transformation is defined in such a way that the first principal component has the largest possible variance (that is, accounts for as much of the variability in the data as possible), and each succeeding component in turn has the highest variance possible under the constraint that it is orthogonal to the preceding components.

The principal components transformation can be associated with singular value decomposition (SVD).

\begin{definition}[principal components\index{principal components}]\label{def:principal_components_svd}
Let $X$ be $n\times d$ real matrix having $n$ observations of $d$ dimensional variables with column-wise zero mean (the sample mean of each column has been shifted to zero).

The transformation is defined by a set of $d$-dimensional vectors of weights or loadings $V = \brb{v_1,\dots,v_m}$ where $V$ is $d\times m$ and $m\leq d$. Then we can define the principal components by
\be
T = XV, \qquad T = \brb{T_1,\dots,T_m}
\ee
where $T$ is $n\times m$. In order to maximize variance, the first loading vector $v_1$ has to satisfy
\beast
v_1 & = & \underset{\dabs{v}=1}{\arg\max} \sum_{k=1}^n(T_1)_k^2= \underset{\dabs{v}=1}{\arg\max} \brb{T_1^T T_1} = \underset{\dabs{v}=1}{\arg\max} \brb{\brb{Xv}^T Xv} \\
& = & \underset{\dabs{v}=1}{\arg\max} \brb{v^T X^T Xv} = \underset{v\neq 0, x\in \R^d}{\arg\max} \frac{v^T X^T Xv}{v^Tv}% = {\arg\max}_{v\neq 0,v\in\R^d}
\eeast
by Proposition \ref{pro:hermitian_max_largest_eigenvalue} as $\underset{v\neq 0, x\in \R^d}{\max} \frac{v^T Av}{v^Tv} = \lm_1$ where $\lm_1$ is he largest eigenvalue of symmetric matrix $A$. Thus, $v_1$ is eigenvector of $X^TX$ corresponding to $\lm_1$. Furthermore, let
\be
\wh{T}_i = T - X \wh{V}_i,\qquad \wh{V}_i = \brb{v_1,\dots,v_{i-1},0,\dots,0}
\ee
and thus $\wh{X}_i V = XV - X \wh{V}_i$. Then %,\qquad \wh{V}_i = \brb{v_1,\dots,v_{i-1},0,\dots,0}
\be
\wh{X}_i = X - X \wh{V}_iV^T = X - X\sum^{i-1}_{k=1} v_k v_k^T
\ee

Thus, to find the $i$th principal component, we have
\beast
v_i & = & \underset{\dabs{v}=1}{\arg\max} \sum_{k=1}^n(T_i)_k^2 = \underset{\dabs{v}=1}{\arg\max} \brb{T_i^T T_i} = \underset{\dabs{v}=1}{\arg\max} \brb{\brb{\wh{X}_iv}^T \wh{X}_iv} \\
& = & \underset{\dabs{v}=1}{\arg\max} \brb{v^T \wh{X}_i^T \wh{X}_i v} = \underset{v\neq 0, x\in \R^d}{\arg\max} \frac{v^T \wh{X}_i^T \wh{X}_i v}{v^Tv}.% = {\arg\max}_{v\neq 0,v\in\R^d}
\eeast

Then this gives $T_i = X v_i$, $i = 1,\dots,m$.

Alternatively, given the singular value decompostion, $X = U\Sigma W^T$, we define the principal components are
\be
T = X W = U\Sigma W^TW = U\Sigma I = U\Sigma
\ee
where $T$ is a $n\times d$ real matrix with $i$th column for $i$th principal component.
\end{definition}

\begin{remark}
For simulation of $X$, we can simulate the $1\times d$ principal components ($T^s$) first and then use the linear transformation
\be
X^s = T^s W^T
\ee
since $W^T = W^{-1}$. If components of $T$ are correlated, we can apply Cholesky decomposition (Corollary \ref{cor:cholesky_decomposition}) to the covariance matrix $S$ of principal components. That is,
\be
S = LL^T
\ee
where $L$ is a real lower triangular matrix. Then we can simulate $d\times 1$ independent standard random variables vector $Z$ and let $T^s = (LZ)^T = Z^T L^T$ such that $\E((T^s)^TT^s) = \E\brb{LZZ^TL^T}= L I L^T = LL^T = S$. Then the simulated observation is
\be % where $T^s$ is $1\times d$
X^s = (LZ)^T W^T = (WLZ)^T.
\ee
\end{remark}

%\begin{proof}[\bf Proof]
%
%\end{proof}
