\chapter{Options}

\section{Black-Scholes Model}

Black-Scholes formula (see \cite{Black_Scholes_1973}).

\section{Greeks}

\section{Barrier Options}

\cite{Carr_Bowie_1994}

\section{Lookback Options}

\cite{Carr_Bowie_1994}

\begin{definition}[European lookback option\index{lookback option}]
A European lookback call option, $C_{\min}$ entitles the holder to buy one unit of stock at the expiry time $T$ at the lowest price reached by the stock during the life of the option. Thus, if it is purchased at time 0, at
time $T$ it pays off the amount $S_T - \inf_{0\leq t\leq T} S_t$.

In contrary, a European lookback put option, $P_{\max}$ entitles the holder to sell one unit of stock at the expiry time $T$ at the highest price reached by the stock during the life of the option. Thus, if it is purchased
at time 0, at time $T$ it pays off the amount $\sup_{0\leq t\leq T} S_t - S_T$.
\end{definition}
%,  P_{\max}

\begin{proposition}
In the Black-Scholes model, the prices at time 0 of European lookback call and put option are\footnote{checking of $r=0$ case is needed.}
\be
C_{\min}= \left\{ \ba{ll} S_0\brb{\brb{\frac{2r + \sigma^2}{2r}}\Phi\brb{\frac {\brb{2r + \sigma^2}\sqrt{T}}{2\sigma}} - e^{-rT} \brb{\frac {2r - \sigma^2}{2r}} \Phi\brb{\frac{\brb{2r - \sigma^2}\sqrt{T}}{2\sigma}} - \frac
{\sigma^2}{2r}} \quad\quad & r \neq 0 \\
S_0 \brb{\frac{\sigma \sqrt{T}}{\sqrt{2\pi}}\exp\brb{- \frac{\sigma^2 T}8} - \brb{2+\frac{\sigma^2 T}2}\Phi\brb{-\frac{\sigma\sqrt{T}}2}+1} & r = 0 \ea \right. ,
\ee

\be
P_{\max}= \left\{ \ba{ll} S_0\brb{\brb{\frac{2r + \sigma^2}{2r}}\Phi\brb{\frac {\brb{2r + \sigma^2}\sqrt{T}}{2\sigma}} - e^{-rT} \brb{\frac {2r - \sigma^2}{2r}} \Phi\brb{-\frac{\brb{2r - \sigma^2}\sqrt{T}}{2\sigma}} - 1} \quad\quad & r \neq 0 \\
S_0 \brb{\frac{\sigma \sqrt{T}}{\sqrt{2\pi}}\exp\brb{- \frac{\sigma^2 T}8} + \brb{2+\frac{\sigma^2 T}2}\Phi\brb{\frac{\sigma\sqrt{T}}2}-1} & r = 0 \ea \right. .
\ee
%\be P_{\max} = \left\{ \ba{ll} S_0\brb{\brb{\frac{2r + \sigma^2}{2r}}\Phi\brb{\frac {\brb{2r + \sigma^2}\sqrt{T}}{2\sigma}} - e^{-rT} \brb{\frac {2r - \sigma^2}{2r}} \Phi\brb{\frac{\brb{2r - \sigma^2}\sqrt{T}}{2\sigma}} - \frac {\sigma^2}{2r}} \quad\quad & r \neq 0 \\
%S_0 \brb{1 + \frac{\sigma \sqrt{T}}{\sqrt{2\pi}}\exp\brb{- \frac{\sigma^2 T}8} - \brb{2+\frac{\sigma^2 T}2}\Phi\brb{-\frac{\sigma\sqrt{T}}2}} & r = 0 \ea \right. . \ee
\end{proposition}

\begin{remark}
Note that (e.g. for lookback call option) we can not use $r\to 0$ and get $e^{-rT} \to 1$ and get the price
\be
S_0 \brb{1-2\Phi\brb{-\frac{\sigma\sqrt{T}}2}}
\ee
as $e^{-rT}\brb{\frac {2r - \sigma^2}{2r}} $ is not equal to $1- \frac{\sigma^2}{2r}$ as $r\to 0$.
\end{remark}

\begin{proof}[\bf Proof]
Let $X_t=at + \sigma W_t$ and $Y_T=\inf_{0\leq t\leq T}X_t$, where $W$ is a standard Brownian motion. For $x>y$, $y<0$, by Proposition \ref{pro:pairwise_joint_density_bm_maximum_minimum_current} and
Girsanov, Cameron-Martin theorem (Corollary \ref{cor:girsanov_drift_brownian_motion}), we know the joint density $f$ of ($X_T,Y_T$) is
\be
f(x,y)=\frac{2(x-2y)}{\sigma^3\sqrt{2\pi T^3}}\exp\brb{\frac{a}{\sigma}\frac{x}{\sigma}-\frac{a^2T}{2\sigma^2}-\frac{(x-2y)^2}{2\sigma^2T}},\qquad x>y,y<0.
\ee

For Black-Scholes model we have the underlying price is $S_t=S_0\exp\brb{\brb{r-\frac{\sigma^2}2}t+\sigma W_t}$. Hence, the price of the option is given by
\be
e^{-rT}\E\brb{S_T-\inf_{0\leq t\leq T}S_t}= S_0 - e^{-rT}S_0 \int^0_{-\infty}\int^\infty_y e^{y}f(x,y)dxdy.
\ee
where $f$ is given above with the coefficient $a=r-\sigma^2/2$. Thus, % = r/\sigma-\sigma/2$. Here are the details:
\be
\int^0_{-\infty}\int^\infty_y e^{y}f(x,y)dxdy  = \frac{2\exp\brb{-\frac{a^2T}{2\sigma^2}}}{\sigma^3\sqrt{2\pi T^3}}\int^0_{-\infty}\int^\infty_y  (x-2y)\exp\brb{y+\frac{ax}{\sigma^2}-\frac{(x-2y)^2}{2\sigma^2T}}dxdy .\qquad (*)
\ee

Make the change of variables
\be
\ba{ccl}
x & = & \sigma\sqrt{T}u \\
y & = & \frac{1}{2}\sigma\sqrt{T}(u-v)
\ea\la
\ba{ccl}
u & = & \frac{1}{\sigma\sqrt{T}}x \\
v & = & \frac{1}{\sigma\sqrt{T}}(x-2y)
\ea
\ee
in the integral, remembering to put in the Jacobian factor $\sigma^2 T/2$:
\be
\int^0_{-\infty}\int^\infty_y  (x-2y)\exp\brb{y+\frac{ax}{\sigma^2}-\frac{(x-2y)^2}{2\sigma^2T}}dxdy =
\frac{\sigma^3T^{3/2}}{2}\int^\infty_{0}\int_{-v}^v v\exp\brb{\frac{1}{2}\sigma\sqrt{T}(u-v) + \frac{a\sqrt{T}u}{\sigma}-\frac{v^2}2}dudv
\ee

Now, recalling that $\sigma^2/2+a=r$, we do the integration with respect to $u$,
\beast
& & \int^\infty_{0}\int_{-v}^v v\exp\brb{\frac{r\sqrt{T}u}{\sigma} - \frac{\sigma\sqrt{T}v}{2}-\frac{v^2}2}dudv \qquad (**) \\
& = & \frac{\sigma}{r\sqrt{T}}\int_0^\infty v\brb{\exp\brb{\frac{r\sqrt{T}v}{\sigma}}-\exp\brb{-\frac{r\sqrt{T}v}{\sigma}}}\exp\brb{-\frac{\sigma\sqrt{T}v}2-\frac{v^2}2}dv
\eeast

Now, do the integrals separately for $c$:
\beast
\int_0^\infty v\exp\brb{c\sqrt{T}v-v^2/2}dv & = & \int^\infty_0 v\exp\brb{(c^2T/2-(v-c\sqrt{T})^2/2}dv = e^{c^2T/2}\int^\infty_{-c\sqrt{T}} \brb{s+c\sqrt{T}}e^{-s^2/2}ds \\
& = & e^{c^2T/2}\brb{ e^{-c^2T/2} + c\sqrt{2\pi T}\brb{1-\Phi\brb{-c\sqrt{T}}} } = 1 + c\sqrt{2\pi T} \exp\brb{\frac{c^2T}2}\Phi\brb{c\sqrt{T}}
\eeast

Similiarly, letting $c=a/\sigma$ or $c = b/\sigma$ where $b:= -r -\sigma^2/2$, ($*$) is
%
%\be
%\int_{v>0} ve^{-(r/\sigma+\sigma/2)\sqrt{T}v-v^2/2}dv = \int_{v>0} ve^{(b^2T/2-(v+b\sqrt{T})^2/2}dv = 1 - b\sqrt{2\pi T} e^{b^2T/2}\Phi(-b\sqrt{T})
%\ee
\beast
& & \frac{2\exp\brb{-\frac{a^2T}{2\sigma^2}}}{\sigma^3\sqrt{2\pi T^3}} \frac{\sigma^3T^{3/2}}{2} \frac{\sigma}{r\sqrt{T}} \brb{\frac{a}{\sigma}\sqrt{2\pi T}
\exp\brb{\frac{a^2T}{2\sigma^2}}\Phi\brb{\frac{a\sqrt{T}}{\sigma}}- \frac{b}{\sigma}\sqrt{2\pi T} \exp\brb{\frac{b^2T}{2\sigma^2}}\Phi\brb{\frac{b\sqrt{T}}{\sigma}}} \\
& = & \frac ar \Phi\brb{\frac{a\sqrt{T}}{\sigma}} - \frac br\exp\brb{\frac{2r\sigma^2 T}{2\sigma^2}}\Phi\brb{\frac{b\sqrt{T}}{\sigma}} \\
& = & \frac {2r-\sigma^2}{2r} \Phi\brb{\frac{\brb{2r-\sigma^2}\sqrt{T}}{2\sigma}} + \frac {2r+\sigma^2 }{2r} e^{rT }\Phi\brb{\frac{\brb{-2r-\sigma^2}\sqrt{T}}{2\sigma}} \\
& = & \frac {2r-\sigma^2}{2r} \Phi\brb{\frac{\brb{2r-\sigma^2}\sqrt{T}}{2\sigma}} + \frac {2r+\sigma^2 }{2r} e^{rT } \brb{1-\Phi\brb{\frac{\brb{2r+\sigma^2}\sqrt{T}}{2\sigma}}}
\eeast

Putting it all together yields
\beast
e^{-rT}\E\brb{S_T-\inf_{0\leq t\leq T}S_t} & = & S_0 - e^{-rT}S_0 \int^0_{-\infty}\int^\infty_y e^{y}f(x,y)dxdy \\
& = & S_0\brb{\brb{\frac{2r + \sigma^2}{2r}}\Phi\brb{\frac {\brb{2r + \sigma^2}\sqrt{T}}{2\sigma}} - e^{-rT} \brb{\frac {2r - \sigma^2}{2r}} \Phi\brb{\frac{\brb{2r - \sigma^2}\sqrt{T}}{2\sigma}} - \frac {\sigma^2}{2r}}
\eeast
as required. %But $-a^2+b^2=-(\sigma/r-\sigma/2)^2+(\sigma/r+\sigma/2)^2=2r$, so that everything equals
%\begin{equation}
%S_0\left[1- e^{-rT}(\sigma/r)a\Phi(a\sqrt{T}) - (\sigma/r)b + (\sigma/r)b\Phi(b\sqrt{T}) \right]
%\end{equation}
%We're done now since $-(\sigma/r)b+1=-(\sigma/r)(r/\sigma+\sigma/2)+1=-\sigma^2/(2r)$.

If $r=0$, we can have that (from ($**$))
\beast
\int^0_{-\infty}\int^\infty_y e^{y}f(x,y)dxdy  & = & I\int^\infty_{0}  2v^2\exp\brb{- \frac{\sigma\sqrt{T}v}{2}-\frac{v^2}2}dv = 2I\exp\brb{\frac{\sigma^2 T}{8}}\int^\infty_{0}  v^2\exp\brb{- \frac{\brb{v +
\sigma\sqrt{T}/2}^2}2}dv \\
& = & 2I\exp\brb{\frac{\sigma^2 T}{8}}\brb{\int^\infty_{0}  \brb{v^2 - \frac{\sigma^2 T}4}\exp\brb{- \frac{\brb{v + \sigma\sqrt{T}/2}^2}2}dv + \sqrt{2\pi}\frac{\sigma^2 T}4\Phi\brb{-\sigma\sqrt{T}/2}} \\
& = & 2I\exp\brb{\frac{\sigma^2 T}{8}}\brb{-\frac{\sigma \sqrt{T}}2\exp\brb{- \frac{\sigma^2 T}8} + \sqrt{2\pi}\brb{1+\frac{\sigma^2 T}4}\Phi\brb{-\sigma\sqrt{T}/2}}
\eeast
where
\be
I = \frac{2\exp\brb{-\frac{a^2T}{2\sigma^2}}}{\sigma^3\sqrt{2\pi T^3}} \frac{\sigma^3T^{3/2}}{2} = \frac 1{\sqrt{2\pi}}\exp\brb{-\frac{\sigma^2 T}{8}}.
\ee

Thus,
\be
e^{-rT}\E\brb{S_T-\inf_{0\leq t\leq T}S_t} = S_0 \brb{1 + \frac{\sigma \sqrt{T}}{\sqrt{2\pi}}\exp\brb{- \frac{\sigma^2 T}8} - \brb{2+\frac{\sigma^2 T}2}\Phi\brb{-\sigma\sqrt{T}/2}}.
\ee

Note that $1\geq 2\Phi\brb{-\sigma\sqrt{T}/2}$ and \be \frac{\sigma \sqrt{T}}{\sqrt{2\pi}}\exp\brb{- \frac{\sigma^2 T}8}\geq \frac{\sigma^2 T}2 \Phi\brb{-\sigma\sqrt{T}/2} \ee by Proposition \ref{pro:bound_of_gaussian_law}.
Thus, we can have the price is greater than zero.

Now let $Y_T=\sup_{0\leq t\leq T}X_t$ for $y>x$, $y>0$, by Proposition \ref{pro:pairwise_joint_density_bm_maximum_minimum_current} and Girsanov's theorem\footnote{theorem needed, or Cameron-Martin
theorem}, we know the joint density $f$ of ($X_T,Y_T$) is
\be
f(x,y)=\frac{2(2y-x)}{\sigma^3\sqrt{2\pi T^3}}\exp\brb{\frac{a}{\sigma}\frac{x}{\sigma}-\frac{a^2T}{2\sigma^2}-\frac{(2y-x)^2}{2\sigma^2T}},\qquad x<y,y>0.
\ee

For Black-Scholes model we have the underlying price is $S_t=S_0\exp\brb{\brb{r-\frac{\sigma^2}2}t+\sigma W_t}$. Hence, the price of the option is given by
\be
e^{-rT}\E\brb{\sup_{0\leq t\leq T}S_t-S_T}= e^{-rT}S_0 \int_0^\infty \int^y_{-\infty} e^{y}f(x,y)dxdy - S_0.
\ee

Thus, % = r/\sigma-\sigma/2$. Here are the details:
\be
\int_0^\infty \int^y_{-\infty} e^{y}f(x,y)dxdy  = \frac{2\exp\brb{-\frac{a^2T}{2\sigma^2}}}{\sigma^3\sqrt{2\pi T^3}}\int_0^\infty \int^y_{-\infty}  (2y-x)\exp\brb{y+\frac{ax}{\sigma^2}-\frac{(2y-x)^2}{2\sigma^2T}}dxdy
\qquad (\dag) .
\ee

Make the change of variables \be
\ba{ccl}
x & = & \sigma\sqrt{T}u \\
y & = & \frac{1}{2}\sigma\sqrt{T}(u+v)
\ea \la
\ba{ccl}
u & = & \frac{1}{\sigma\sqrt{T}}x \\
v & = & \frac{1}{\sigma\sqrt{T}}(2y-x)
\ea
\ee
in the integral, remembering to put in the Jacobian factor $\sigma^2 T/2$:
\be
\int_0^\infty \int^y_{-\infty}  (2y -x)\exp\brb{y+\frac{ax}{\sigma^2}-\frac{(2y-x)^2}{2\sigma^2T}}dxdy = \frac{\sigma^3T^{3/2}}{2}\int^\infty_{0}\int_{-v}^v v\exp\brb{\frac{1}{2}\sigma\sqrt{T}(u+v) + \frac{a\sqrt{T}u}{\sigma}-\frac{v^2}2}dudv
\ee

Now, recalling that $\sigma^2/2+a=r$, we do the integration with respect to $u$,
\beast
& & \int^\infty_{0}\int_{-v}^v v\exp\brb{\frac{r\sqrt{T}u}{\sigma} + \frac{\sigma\sqrt{T}v}{2}-\frac{v^2}2}dudv \qquad (**) \\
& = & \frac{\sigma}{r\sqrt{T}}\int_0^\infty v\brb{\exp\brb{\frac{r\sqrt{T}v}{\sigma}}-\exp\brb{-\frac{r\sqrt{T}v}{\sigma}}}\exp\brb{\frac{\sigma\sqrt{T}v}2-\frac{v^2}2}dv
\eeast

So letting $a' = r + \frac {\sigma^2}2$ and $b' = -r + \frac {\sigma^2}2$, ($\dag$) is
\beast
& & \frac{2\exp\brb{-\frac{a^2T}{2\sigma^2}}}{\sigma^3\sqrt{2\pi T^3}} \frac{\sigma^3T^{3/2}}{2} \frac{\sigma}{r\sqrt{T}} \brb{\frac{a'}{\sigma}\sqrt{2\pi T} \exp\brb{\frac{a'^2T}{2\sigma^2}}\Phi\brb{\frac{a'\sqrt{T}}{\sigma}}- \frac{b'}{\sigma}\sqrt{2\pi T} \exp\brb{\frac{b'^2T}{2\sigma^2}}\Phi\brb{\frac{b'\sqrt{T}}{\sigma}}} \\
& = & \frac {a'}r e^{rT}\Phi\brb{\frac{a'\sqrt{T}}{\sigma}} - \frac {b'}r\Phi\brb{\frac{b'\sqrt{T}}{\sigma}} \\
& = & \frac {2r+\sigma^2}{2r}e^{rT} \Phi\brb{\frac{\brb{2r+\sigma^2}\sqrt{T}}{2\sigma}} + \frac {2r-\sigma^2 }{2r} \Phi\brb{-\frac{\brb{2r-\sigma^2}\sqrt{T}}{2\sigma}}
\eeast

Putting it all together yields
\beast
e^{-rT}\E\brb{\sup_{0\leq t\leq T}S_t-S_T} & = & e^{-rT}S_0 \int^0_{-\infty}\int^\infty_y e^{y}f(x,y)dxdy - S_0 \\
& = & S_0\brb{\brb{\frac{2r + \sigma^2}{2r}}\Phi\brb{\frac {\brb{2r + \sigma^2}\sqrt{T}}{2\sigma}} - e^{-rT} \brb{\frac {2r - \sigma^2}{2r}} \Phi\brb{-\frac{\brb{2r - \sigma^2}\sqrt{T}}{2\sigma}} - 1}
\eeast
as required. Also, it is similar for the case $r=0$.
\end{proof}


\section{Hedging}

Discretely adjusted hedging (see \cite{Boyle_Emanuel_1980}) with transaction costs (see \cite{Leland_1985})

\subsection{Put-call Symmetry}

First metioned in \cite{Carr_Bowie_1994}, proved in the appendix of \cite{Carr_Ellis_Gupta_1998}.

\section{American options}



\section{Path Dependent options}

\cite{Goldman_Sosin_Gatto_1979_1},\cite{Goldman_Sosin_Gatto_1979_2}


\section{Heston Model}

The Heston model is first given in \cite{Heston_1993}.
