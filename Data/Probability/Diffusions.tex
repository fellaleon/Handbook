\chapter{Diffusion Processes}


\section{Ornstein-Uhlenbeck Process}

\subsection{Ornstein-Uhlenbeck process}

\begin{definition}[Ornstein-Uhlenbeck (OU) process\index{Ornstein-Uhlenbeck process}\index{OU process}]\label{def:ornstein_uhlenbeck_process_sde}
SDE of Ornstein-Uhlenbeck (OU) Process is defined by
\be
dX_t = \theta(\mu - X_t) dt + \sigma dW_t.
\ee
where $\mu$, $\theta$, $\sigma$ are constants with $\theta,\sigma >0$.
\end{definition}

\begin{theorem}
Given the initial point $x_0$, the transition density of OU process at time $t$ is
\be
f(x) = \sqrt{\frac{\theta}{\pi\sigma^2\brb{1-e^{-2\theta t}}}} \exp\brb{-\frac{\theta}{\sigma^2}\brb{\frac{\brb{x - \mu - (x_0 -\mu)e^{-\theta t} }^2}{1-e^{-2\theta t}}}}
\ee
which is Gaussian distribution with mean and variance
\be
\mu + (x_0 - \mu)e^{-\theta t}, \qquad \frac{\sigma^2}{2\theta }\brb{1-e^{-2\theta t}}.
\ee

The limit for time tending to infinity is a Gaussian distribution with mean $\mu$ and variance $\frac{\sigma^2}{2\theta}$,
\be
f(x) = \sqrt{\frac{\theta}{\pi\sigma^2}} \exp\brb{-\frac{\theta(x-\mu)^2}{\sigma^2}}.
\ee
\end{theorem}

\begin{proof}[\bf Proof]
Let $f(t,x) = e^{\theta t}(x-\mu) $. Then by It\^o lemma (Corollary \ref{cor:ito_formula_semimartingale_time}), we have
\beast
df(t,X_t) & = & \theta f(t,X_t) dt + e^{\theta t} dX_t + \frac 12 \cdot 0 \cdot \brb{dX_t}^2 \\
& = & \theta  e^{\theta t}(X_t-\mu) dt + e^{\theta t}\brb{\theta(\mu -X_t)}dt + \sigma e^{\theta t} dB_t \\
& = & \sigma e^{\theta t} dB_t.
\eeast

Then the stochastic integral implies that
\be
e^{\theta t}(X_t-\mu) - (X_0-\mu)  = \int^{t}_0 df(s,X_s) = \sigma \int^{t}_0  e^{\theta s} dB_s
\ee
which can be converted to
\be
X_t = X_0 e^{-\theta t} + \mu(1- e^{-\theta t}) + \sigma e^{-\theta t}\int^t_0 e^{\theta s} dB_s
\ee

It can be proved by It\^o isometry\footnote{theorem needed.} that the stochastic integral $\int^t_0 e^{\theta s} dB_s$ is a Gaussian random variable with zero mean and variance
\be
\E\brb{\int^t_0 e^{\theta s} dB_s}^2 = \bsb{\int^t_0  e^{\theta s} dB_s} = \int^t_0 e^{2\theta s}\bsb{B_s} = \int^t_0 e^{2\theta s}s = \frac 1{2\theta} \brb{e^{2\theta t}-1}
\ee

Therefore,
\be
X_t =  X_0 e^{-\theta t} + \mu\brb{1 - e^{-\theta t}} + \sigma\sqrt{\frac{1-e^{-2\theta t}}{2\theta}} Z
\ee
where $Z \sim \sN(0,1)$. Hence, the transition density of OU process at time $t$ is
\be
f(x) = \sqrt{\frac{\theta}{\pi\sigma^2\brb{1-e^{-2\theta t}}}}\exp\brb{-\frac{\theta\brb{x-\mu - (x_0 -\mu)e^{-\theta t}}^2}{\sigma^2\brb{1-e^{-2\theta t}}}}.
\ee

Accordingly, the asymptotic distribution is $\sN\brb{\mu, \frac{\sigma^2}{2\theta}}$.
\end{proof}



\subsection{Parameters estimation of OU process} % OU Process calibration, least squares regression

Recall the result of stochastic integral in subsection\footnote{subsection needed.}, we have
\be
e^{\theta (t+\delta)}(X_{t+\delta}-\mu) - e^{\theta t}(X_{t}-\mu)  = \int^{t+\delta}_t df(s,X_s) = \sigma \int^{t+\delta}_t  e^{\theta s} dB_s
\ee

If $X_t = S_{i-1}$ and $X_{t+\delta} = S_{i}$,
\be
S_{i} =  S_{i-1} e^{-\theta \delta} + \mu\brb{1 - e^{-\theta \delta}} + \sigma e^{-\theta (t+\delta)} \int^{t+\delta}_t  e^{\theta s} dB_s
\ee


It can be proved by It\^o isometry\footnote{theorem needed.} that the stochastic integral $\int^{t+\delta}_t e^{\theta s} dB_s$ is a Gaussian random variable with zero mean and variance
\be
\E\brb{\int^{t+\delta}_t e^{\theta s} dB_s}^2 = \bsb{\int^{t+\delta}_t  e^{\theta s} dB_s} = \int^{t+\delta}_t e^{2\theta s}\bsb{B_s} = \int^{t+\delta}_t e^{2\theta s}s = \frac 1{2\theta} \brb{e^{2\theta (t+\delta)}-e^{2\theta t}}
\ee

Then given the previous observation $S_{i-1}$,
\be
S_{i} =  S_{i-1} e^{-\theta \delta} + \mu\brb{1 - e^{-\theta\delta}} + \sigma\sqrt{\frac{1-e^{-2\theta \delta}}{2\theta}} Z
\ee
where $Z \sim \sN(0,1)$. The conditional probability density of $S_{i}$ given $S_{i-1}$ is
\be
f(x|S_{i-1}) = \sqrt{\frac {\theta}{\pi\sigma^2\brb{1-e^{-2\theta \delta}}}}\exp\brb{- \frac{\theta\brb{x- S_{i-1}e^{-\theta \delta} - \mu\brb{1-e^{-\theta\delta}}}^2}{\sigma^2\brb{1-e^{-2\theta\delta}}}}.
\ee

For simplicity, we define
\be
S_x = \sum^n_{i=1} S_{i-1}, \quad S_y = \sum^n_{i=1} S_{i}, \quad S_{xx} = \sum^n_{i=1} S^2_{i-1}, \quad S_{xy} = \sum^n_{i=1} S_{i-1}S_i, \quad S_{yy} = \sum^n_{i=1} S^2_{i}.
\ee

{\bf Approach 1: least squares regression} We consider the linear model
\be
Y = aX + b + \ve \ \lra \ S_{i} = aS_{i-1} + b + \ve,\quad i = 1,\dots,n
\ee
where
\be
Y = \bepm S_1 \\ \vdots \\ S_n \eepm,\qquad X = \bepm S_0 \\ \vdots \\ S_{n-1} \eepm, \qquad \ve = \bepm \ve_1 \\ \vdots \\ \ve_n \eepm, \qquad \ve_i \sim \sN\brb{0, \frac{\sigma^2\brb{1-e^{-2\theta \delta}}}{2\theta}}.
\ee

We can also write it in more general form of linear model we have $Y = W\beta + \ve$ where
\be
W = \bepm 1 & S_0 \\ \vdots & \vdots \\ 1 & S_{n-1} \eepm,\qquad  \beta = \bepm b \\ a \eepm.
\ee


Assume $S_i$ are not all the same thus $W$ has rank 2. Hence, the estimator of $\beta$ is
\beast
\wh{\beta} & = & \brb{W^TW}^{-1}W^TY = \bepm n & S_x\\ S_x & S_{xx}\eepm^{-1} \bepm S_y \\ S_{xy} \eepm \\
& = & \frac{1}{nS_{xx} - S_x^2} \bepm S_{xx} & -S_x\\ -S_x & n \eepm \bepm S_y \\ S_{xy} \eepm = \frac{1}{nS_{xx} - S_x^2} \bepm S_{xx}S_y - S_x S_{xy}\\ nS_{xy} - S_x S_y\eepm
\eeast

Thus, we have
\be
e^{-\theta \delta} = a = \frac{nS_{xy} - S_x S_y}{nS_{xx} - S_x^2} \ \ra\ \wh{\theta} = -\frac 1{\delta} \ln a = -\frac 1{\delta} \ln\brb{\frac{nS_{xy} - S_x S_y}{nS_{xx} - S_x^2}}.\qquad (*)
\ee

Also,
\be
\mu\brb{1 - e^{-\theta\delta}} = b = \frac{S_{xx}S_y - S_x S_{xy}}{nS_{xx} - S_x^2} = \frac{nS_{xx}S_y - S_yS_x^2 + S_yS_x^2 - n S_x S_{xy}}{n\brb{nS_{xx} - S_x^2}} = \frac {S_y - a S_x}n
\ee
which implies
\be
\wh{\mu} = \frac{b}{1-a} = \frac{\frac{S_{xx}S_y - S_x S_{xy}}{nS_{xx} - S_x^2}}{1 - \frac{nS_{xy} - S_x S_y}{nS_{xx} - S_x^2}} = \frac{S_{xx}S_y - S_x S_{xy}}{n\brb{S_{xx} - S_{xy}}- S_x\brb{S_x - S_y}} = \frac{S_y - a S_x}{n(1-a)}.\qquad (\dag)
\ee

Furthermore, we have the unbiased estimator of $\frac{\sigma^2\brb{1-e^{-2\theta \delta}}}{2\theta}$ ($m=2$ for general linear model)
\beast
\wt{\sigma}^2 & = & \frac 1{n-2} \dabs{Y - W\wh{\beta}}^2 = \frac 1{n-2} \brb{Y - W \brb{W^TW}^{-1}W^TY }^T \brb{Y - W \brb{W^TW}^{-1}W^TY } \\
& = & \frac 1{n-2} Y^T\brb{I - W \brb{W^TW}^{-1}W^T }^T \brb{I - W \brb{W^TW}^{-1}W^T}Y = \frac 1{n-2} Y^T\brb{I - W \brb{W^TW}^{-1}W^T }Y \\
& = & \frac 1{n-2} \brb{Y^TY - Y^TW \brb{W^TW}^{-1}W^TY } = \frac 1{n-2} \brb{S_{yy} -\bepm S_y & S_{xy} \eepm \brb{W^TW}^{-1}\bepm S_y \\ S_{xy} \eepm } \\
& = & \frac 1{(n-2)\brb{nS_{xx} - S_x^2}} \brb{S_{yy}\brb{nS_{xx} - S_x^2} -\bepm S_y & S_{xy} \eepm \bepm S_{xx}S_y - S_x S_{xy}\\ nS_{xy} - S_x S_y\eepm } \\
& = & \frac 1{(n-2)\brb{nS_{xx} - S_x^2}} \brb{n \brb{S_{xx}S_{yy} - S^2_{xy}} - S_x^2S_{yy} - S_y^2S_{xx} + 2S_xS_y S_{xy} } \qquad (\ddag)
\eeast

Alternatively,
\beast
\wt{\sigma}^2 & = & \frac 1{n(n-2)\brb{nS_{xx} - S_x^2}} \brb{n^2 \brb{S_{xx}S_{yy} - S^2_{xy}} - nS_x^2S_{yy} - nS_y^2S_{xx} + 2nS_xS_y S_{xy} } \\
& = & \frac 1{n(n-2)\brb{nS_{xx} - S_x^2}} \brb{n^2 S_{xx}S_{yy} - nS_x^2S_{yy} - nS_y^2S_{xx} + S_y^2 S_x^2 -  S_y^2 S_x^2 - n^2 S^2_{xy}  + 2nS_xS_y S_{xy} } \\
& = & \frac 1{n(n-2)} \brb{\frac{n^2 S_{xx}S_{yy} - nS_x^2S_{yy} - nS_y^2S_{xx} + S_y^2 S_x^2}{\brb{nS_{xx} - S_x^2}} -  \frac{S_y^2 S_x^2 + n^2 S^2_{xy} -2nS_xS_y S_{xy}}{\brb{nS_{xx} - S_x^2}} } \\
& = & \frac 1{n(n-2)} \brb{n S_{yy} - S^2_{y} - a\brb{n S_{xy} - S_xS_y}}
\eeast

Therefore,
\beast
\wh{\sigma}^2 =  \frac{2\wt{\sigma}^2 \wh{\theta}}{\brb{1-e^{-2\wh{\theta} \delta}}}  & = & \frac{-\frac 2{\delta} \ln a }{1-a^2} \frac 1{n(n-2)} \brb{n S_{yy} - S^2_{y} - a\brb{n S_{xy} - S_xS_y}} \\
& = & \frac{-\frac 2{\delta} \ln a }{n(n-2)\brb{1-a^2}} \brb{n S_{yy} - S^2_{y} - a\brb{n S_{xy} - S_xS_y}}
\eeast

%\item OU Process calibration, MLE

{\bf Approach 2: MLE}

%\be
%\sqrt{\frac {\theta}{\pi\sigma^2\brb{1-e^{-2\theta \delta}}}}\exp\brb{- \frac{\theta\brb{x- S_{i-1}e^{-\theta \delta} - \mu\brb{1-e^{-\theta\delta}}}^2}{\sigma^2\brb{1-e^{-2\theta\delta}}}}.
%\ee

The log-likelihood function of a set of observation $\brb{S_0,S_1,\dots,S_n}$ can be derived from the condition density function
\beast
\sL\brb{\mu,\theta,\sigma} & = & \sum^n_{i=1} \log f\brb{S_i|S_{i-1}} \\
& = & \frac n2 \log\brb{\frac {\theta}{\pi\brb{1-e^{-2\theta \delta}}}} - n\log\sigma  - \frac{\theta}{\sigma^2\brb{1-e^{-2\theta\delta}}} \sum^n_{i=1}\brb{S_i- S_{i-1}e^{-\theta \delta} - \mu\brb{1-e^{-\theta\delta}}}^2 \\
\sL\brb{\mu,\theta,\rho} & = & -\frac n2\ln(2\pi) - n\log\rho - \frac 1{2\rho^2}\sum^n_{i=1} \brb{S_i- S_{i-1}e^{-\theta \delta} - \mu\brb{1-e^{-\theta\delta}}}^2
\eeast
where $\rho^2 = \frac{\sigma^2\brb{1-e^{-2\theta \delta}}}{2\theta}$. Then we take the differentiation of the log-likelihood function
\beast
0 & = & \fp{\sL(\mu,\theta,\rho)}{\mu} = \frac{\brb{1-e^{-\theta\delta}}}{\rho^2} \sum^n_{i=1}\brb{S_i- S_{i-1}e^{-\theta \delta} - \mu\brb{1-e^{-\theta\delta}}} \\
\ra \ \mu & = & \frac 1{n\brb{1-e^{-\theta\delta}}} \sum^n_{i=1}\brb{S_i- S_{i-1}e^{-\theta \delta} } = \frac {S_y - S_x e^{-\theta \delta} }{n\brb{1-e^{-\theta\delta}}}
\eeast

\beast
0 =\fp{\sL(\mu,\theta,\rho)}{\theta} & = & - \frac{\delta e^{-\theta\delta}}{\rho^2}  \sum^n_{i=1}\brb{S_i- \mu -  \brb{S_{i-1} - \mu}e^{-\theta\delta}}\brb{S_{i-1} - \mu} \\
& = & - \frac{\delta e^{-\theta\delta}}{\rho^2}  \sum^n_{i=1}\brb{\brb{S_i- \mu}\brb{S_{i-1} - \mu}  -  \brb{S_{i-1} - \mu}^2e^{-\theta\delta}} \\% & & \qquad - \frac{2\theta\delta  e^{-\theta\delta}}{\sigma^2\brb{1-e^{-2\theta\delta}}} \sum^n_{i=1}\brb{S_i- \mu -  \brb{S_{i-1} - \mu}e^{-\theta\delta}}\brb{S_{i-1} - \mu} \\
\ra \theta & = & -\frac 1{\delta}\log\brb{\frac{\sum^n_{i=1}\brb{S_i- \mu}\brb{S_{i-1} - \mu}}{\sum^n_{i=1}\brb{S_{i-1} - \mu}^2}} =  -\frac 1{\delta}\log\brb{\frac{S_{xy} - \mu S_x - \mu S_y + n\mu^2}{S_{xx} - 2\mu S_x + n\mu^2}} %\frac 1{n\brb{1-e^{-\theta\delta}}} \sum^n_{i=1}\brb{S_i- S_{i-1}e^{-\theta \delta} }^2
\eeast

\beast
0 =\fp{\sL(\mu,\theta,\rho)}{\rho} & = & - \frac n{\rho} + \frac{1}{\rho^3}  \sum^n_{i=1}\brb{S_i- S_{i-1}e^{-\theta \delta} - \mu\brb{1-e^{-\theta\delta}}}^2 \\%& = & - \frac{\delta e^{-\theta\delta}}{\rho^2}  \sum^n_{i=1}\brb{\brb{S_i- \mu}\brb{S_{i-1} - \mu}  -  \brb{S_{i-1} - \mu}^2e^{-\theta\delta}} \\% & & \qquad - \frac{2\theta\delta  e^{-\theta\delta}}{\sigma^2\brb{1-e^{-2\theta\delta}}} \sum^n_{i=1}\brb{S_i- \mu -  \brb{S_{i-1} - \mu}e^{-\theta\delta}}\brb{S_{i-1} - \mu} \\
\ra \rho^2 & = & \frac 1n  \sum^n_{i=1}\brb{S_i- S_{i-1}e^{-\theta \delta} - \mu\brb{1-e^{-\theta\delta}}}^2 \\
& = & \frac 1n  \brb{S_{yy} - 2 S_{xy}e^{-\theta \delta} + S_{xx}e^{-2\theta\delta} - 2\mu\brb{1-e^{-\theta \delta}}\brb{S_y - S_xe^{-\theta \delta}} + n\mu^2 \brb{1-e^{-\theta \delta}}^2} . %\sum^n_{i=1}\brb{S_i- \mu -  \brb{S_{i-1} - \mu}e^{-\theta\delta}}^2 %-\frac 1{\delta}\log\brb{\frac{\sum^n_{i=1}\brb{S_i- \mu}\brb{S_{i-1} - \mu}}{\sum^n_{i=1}\brb{S_{i-1} - \mu}^2}} %\frac 1{n\brb{1-e^{-\theta\delta}}} \sum^n_{i=1}\brb{S_i- S_{i-1}e^{-\theta \delta} }^2
\eeast

Thus, we plug expression of $\theta$ into $\mu$ and get
\beast
n\mu = \frac {S_y - S_x \brb{\frac{S_{xy} - \mu S_x - \mu S_y + n\mu^2}{S_{xx} - 2\mu S_x + n\mu^2}} }{1-\brb{\frac{S_{xy} - \mu S_x - \mu S_y + n\mu^2}{S_{xx} - 2\mu S_x + n\mu^2}}} & = &  \frac{S_y \brb{S_{xx} - 2\mu S_x + n\mu^2} -\brb{ S_{xy} - \mu S_x - \mu S_y + n\mu^2}S_x}{S_{xx} - 2\mu S_x + n\mu^2 - \brb{S_{xy} - \mu S_x - \mu S_y + n\mu^2} } \\
& = & \frac{S_{xx}S_y - S_{xy}S_x + \mu\brb{S_x^2 - S_xS_y} + n\mu^2\brb{ S_{y} - S_x}}{S_{xx} - S_{xy} + \mu \brb{ S_y - S_x } }
\eeast

Balancing the equation we have
\beast
n\mu\brb{S_{xx}- S_{xy}} + n\mu^2 \brb{ S_y - S_x } & = & S_{xx}S_y - S_{xy}S_x + \mu\brb{S_x^2 - S_xS_y} + n\mu^2\brb{ S_{y} - S_x} \\
n\mu\brb{S_{xx}- S_{xy}} - \mu\brb{S_x^2 - S_xS_y} & = & S_{xx}S_y - S_{xy}S_x
\eeast
which implies
\be
\wh{\mu} = \frac{S_{xx}S_y - S_{xy}S_x}{n\brb{S_{xx}- S_{xy}} - \brb{S_x^2 - S_xS_y}}.
\ee

Thus, we can plug this into the expression of $\theta$
\beast
\wh{\theta} & = & -\frac 1{\delta}\log\brb{\frac{S_{xy} - \mu S_x - \mu S_y + n\mu^2}{S_{xx} - 2\mu S_x + n\mu^2}}
\eeast
and
\beast
\wh{\sigma}^2 & =& \frac{2\theta}{1-e^{-2\theta\delta }}\rho^2 \\
& = & \frac{2\theta}{n\brb{1-e^{-2\theta\delta }}}  \brb{S_{yy} - 2 S_{xy}e^{-\theta \delta} + S_{xx}e^{-2\theta\delta} - 2\mu\brb{1-e^{-\theta \delta}}\brb{S_y - S_xe^{-\theta \delta}} + n\mu^2 \brb{1-e^{-\theta \delta}}^2}
\eeast

%\subsection{MLE for OU process}

Use Fisher information\footnote{theorem needed. see MLE in wiki} to have
\be
\sqrt{n}\brb{\bepm \wh{\theta}_n \\ \wh{\mu}_n \\ \wh{\sigma}_n\eepm -\bepm \theta \\ \mu \\ \sigma \eepm } \stackrel{d}{\to} \sN\brb{\bepm 0\\ 0\\ 0 \eepm,
\bepm \frac{e^{2t\theta}-1}{t^2}  & 0 & \frac {\sigma\brb{e^{2t\theta} -1 - 2t\theta}}{2t^2\theta} \\ 0 & \frac{\sigma^2\brb{e^{t\theta}+1}}{2\brb{e^{t\theta}-1}\theta} & 0 \\ \frac {\sigma\brb{e^{2t\theta} -1 - 2t\theta}}{2t^2\theta} & 0 &  \frac{\sigma^2 \brb{\brb{e^{2t\theta}-1}^2 + 2t^2\theta^2 \brb{e^{2t\theta}+1} -4t\theta\brb{e^{2t\theta}-1} }}{4t^2\theta^2\brb{e^{2t\theta}-1}} \eepm}
\ee

\section{Cox-Ingersoll-Ross Process}

\begin{definition}[Cox-Ingersoll-Ross (CIR) process\index{Cox-Ingersoll-Ross process}\index{CIR process}]\label{def:cox_ingersoll_ross_process_sde}
SDE of Cox-Ingersoll-Ross (CIR) process is defined by
\be
dX_t = \theta(\mu - X_t) dt + \sigma\sqrt{X_t} dB_t.
\ee
where $\mu$, $\theta$, $\sigma$ are constants with $\theta,\sigma >0$.
\end{definition}

It is known (see \cite{Cox_Ingersoll_Ross_1985,Jeanblanc_Yor_Chesney_2009}) that given initial point $x_0$ the closed form transition density function of CIR process at time $t$ is
\be
f(x) = \frac{e^{\theta t}}{2c} \brb{\frac{x e^{\theta t}}{x_0}}^{\nu/2} \exp\brb{-\frac{x_0 + xe^{\theta t}}{2c}} I_\nu\brb{\frac{\sqrt{x_0xe^{\theta t}}}c}
\ee
where
\be
c = \frac{\sigma^2}{4\theta}\brb{e^{\theta t}-1}, \qquad \nu = \frac{2\mu \theta}{\sigma^2}-1
\ee
and $I_\nu$ is Bessel function of the first kind\footnote{proof needed.}.

\section{Bessel Processes}

\begin{definition}[Bessel process\index{Bessel process}]
Bessel process $\bes(\nu)$ is a diffusion with infinitesmal generator
\be
\frac 12 \brb{\frac{\nu-1}{r}\fp{}{r} + \fpp{}{r}},\qquad r>0 . %,\qquad \nu>0,\theta \in (0,\pi).
\ee
\end{definition}

\section{Legendre Process}


\begin{definition}[Legendre process\index{Legendre process}]
Legendre process $\leg(\nu)$ is a diffusion with infinitesmal generator
\be
\frac 12 \brb{\sin\theta}^{1-\nu}\fp{}{\theta}\brb{\brb{\sin\theta}^{\nu-1}\fp{}{\theta}} = \frac {\nu-1}{2\tan\theta} \fp{}{\theta} + \frac 12 \fpp{}{\theta}  ,\qquad \nu\geq 0,\theta \in (0,\pi).
\ee
where parameter $\nu$ is called index of Legendre process. The corresponding SDE of Legendre process is
\be
d\Theta_t = \frac {\nu-1}{2\tan\Theta_t}dt +  dB_t,\qquad \Theta_t \in (0,\pi).
\ee

Then we let $X_t = \cos \Theta_t$. By It\^o lemma\footnote{theorem needed.}
\beast
dX_t & = & -\sin \Theta_t d\Theta_t - \frac 12 \cos \Theta_t\brb{d\Theta_t}^2 =  -\sin \Theta_t\brb{\frac {\nu-1}{2\tan\Theta_t}dt + dB_t} - \frac 12 \cos \Theta_t dt \\
& = & - \frac {\nu}2 \cos \Theta_t dt - \sin \Theta_t  dB_t = - \frac{\nu X_t}2 dt + \sqrt{1-X_t^2}dB_t,\qquad X_t \in (-1,1).
\eeast
\end{definition}

\begin{example}
For $\nu =2$, we use the infinitesimal generator with extra parameter $\sigma$ (see David, R. Brillinger (1997), A particle migrating randomly on a sphere, Journal of Theoretical Probability, 10, 2, 429-443.)
\be
\frac {\sigma}{2\sin\theta}\fp{}{\theta}\brb{\sin\theta\fp{}{\theta}} = \frac {\sigma^2}{2\tan\theta}\fp{}{\theta} + \frac 12 \sigma^2 \fpp{}{\theta} .
\ee

Then the corresponding SDE\footnote{theorem needed.} is
\be
d\Theta_t = \frac {\sigma^2}{2\tan\Theta_t}dt + \sigma dB_t,\qquad \Theta_t \in (0,\pi).
\ee

Then we let $X_t = \cos \Theta_t$. By It\^o lemma\footnote{theorem needed.}
\beast
dX_t & = & -\sin \Theta_t d\Theta_t - \frac 12 \cos \Theta_t\brb{d\Theta_t}^2 =  -\sin \Theta_t\brb{\frac {\sigma^2}{2\tan\Theta_t}dt + \sigma dB_t} - \frac 12 \cos \Theta_t\sigma^2 dt \\
& = & - \sigma^2 \cos \Theta_t dt - \sigma\sin \Theta_t  dB_t = - \sigma^2 X_t dt + \sigma\sqrt{1-X_t^2}dB_t,\qquad X_t \in (-1,1).
\eeast
\end{example}


\begin{theorem}
The transition density of $\leg(\nu)$ is given $\theta_0$ at time 0,
\be
f(\theta_t) = \sin\theta_t \brb{\frac{\sin\theta_t}{\sin\theta_0}}^{\nu}\sum^\infty_{k=0}\brb{\nu + k + \frac 12} \frac{\Gamma(2\nu + k+1)}{\Gamma(k+1)}\exp\brb{-\frac{k(2\nu+k+1)t}{2}}P_{k+\nu}^{-\nu}(\cos\theta_t)P_{k+\nu}^{-\nu}(\cos\theta_0)  \nonumber
\ee
where $\Gamma(\cdot)$ denotes Euler's gamma function and $P_n^m(\cdot)$ is the associated Legendre function of the first kind\footnote{definition needed.}.

\end{theorem}


\begin{proof}[\bf Proof]
\footnote{proof needed.}
\end{proof}


\section{Jacobi Process}

\subsection{SDE}

\begin{definition}[Jacobi process\index{Jacobi process}]\label{def:jacobi_process}
The stochastic process $(X_t)_{t\geq 0}$ is called Jacobi process if its SDE is given by
\be
dX_t = \theta \brb{\mu - X_t} dt + \sigma\sqrt{X_t(1-X_t)}dB_t
\ee

More generally, if bounds are $a,b\in \R$ with $a<b$. Then its SDE is
\be
dX_t = \theta \brb{\mu - X_t} dt + \sigma\sqrt{(X_t-a)(b-X_t)}dB_t
\ee
\end{definition}

\subsection{Parameters calibration of Jacobi process}

For Jacobi process
\be
X_{t+\delta} = X_t + \theta\brb{\mu - X_t}\delta + \sigma\sqrt{X_t(1-X_t)} Z
\ee
where $Z\sim \sN(0,1)$ is a standard normal random variable. Let $S_i = X_{t+\delta}$ and $S_{i-1} = X_t$. Thus,
\beast
S_i & = & S_{i-1} + \theta\brb{\mu - S_{i-1}}\delta + \sigma\sqrt{S_{i-1}(1-S_{i-1})} Z \\
& = & \theta\mu \delta + (1-\theta \delta) S_{i-1} + \sigma\sqrt{S_{i-1}(1-S_{i-1})} Z
\eeast

Then
\be
\frac{S_i}{\sqrt{S_{i-1}(1-S_{i-1})}} = \theta\mu \delta \frac{1}{\sqrt{S_{i-1}(1-S_{i-1})}} + (1-\theta \delta)\frac{S_{i-1}}{\sqrt{S_{i-1}(1-S_{i-1})}} + \sigma Z
\ee

Let $Y = (Y_1,\dots,Y_n)^T$ and $X = \brb{X_1^T,\dots,X_n^T}^T$ where
\be
Y_i = \frac{S_i}{\sqrt{S_{i-1}(1-S_{i-1})}},\qquad X_i = \brb{X_1^i,X^i_2} = \brb{{\frac{1}{\sqrt{S_{i-1}(1-S_{i-1})}},\frac{S_{i-1}}{\sqrt{S_{i-1}(1-S_{i-1})}} }}
\ee

%Let
%\be
%S_Y =
%\ee

Then
\beast
\bepm \wh{\theta}\wh{\mu}\delta \\ 1-\wh{\theta} \delta \eepm = \wh{\beta } & =& (X^TX)^{-1} X^TY =  \bepm X_1^TX_1 & X_1^T X_2 \\ X_2^TX_1 & X_2^T X_2 \eepm^{-1} \bepm X_1^T Y \\ X_2^TY \eepm \\
& = & \frac 1{X_1^TX_1X_2^TX_2 - X_1^TX_2X_2^TX_1} \bepm X_2^TX_2 & -X_1^T X_2 \\ -X_2^TX_1 & X_1^T X_1 \eepm \bepm X_1^T Y \\ X_2^TY \eepm
\eeast

Therefore,
\be
1-\wh{\theta} \delta = \frac {X_1^T X_1X_2^TY -X_2^TX_1X_1^T Y }{X_1^TX_1X_2^TX_2 - X_1^TX_2X_2^TX_1} \ \ra\ \wh{\theta} = \frac 1{\delta}\brb{1 - \frac {X_1^T X_1X_2^TY -X_2^TX_1X_1^T Y }{X_1^TX_1X_2^TX_2 - X_1^TX_2X_2^TX_1}}
\ee

\be
\wh{\mu} = \left.\frac {X_2^T X_2X_1^TY -X_1^TX_2X_2^T Y }{X_1^TX_1X_2^TX_2 - X_1^TX_2X_2^TX_1}  \right/\brb{1 - \frac {X_1^T X_1X_2^TY -X_2^TX_1X_1^T Y }{X_1^TX_1X_2^TX_2 - X_1^TX_2X_2^TX_1}}.
\ee

\be
\wh{\sigma}^2 = \frac 1{n-1} \dabs{Y - \wh{\theta}\wh{\mu}\delta X_1 - (1-\wh{\theta} \delta)X_2}^2 .
\ee

\section{Planar Brownian motion}

The polar Brownian motion transition density is $p_{s,t}\brb{(\rho,\phi),(r,\theta)}$. The transition density of 2-dimensional Brownian motion is $p_{s,t}(x,y)$ where $x,y\in \R^n$. Kolmogorov backward equation of 2-dimensinal Brownian motion is
\be
\fp{}{s}p + \frac 12 \Delta p = \frac 12 \brb{\fpp{}{x_1} + \fpp{}{x_2}}p = 0.
\ee

By letting $x_1 = \rho\cos\phi$ and $x_2 = \rho\sin\phi$, we have
\be
\rho = \brb{x_1^2 + x_2^2}^{1/2},\quad \phi = \arctan \brb{\frac{x_2}{x_1}}.
\ee

Then
\beast
\fp{p}{x_1} & = & \fp{p}{\rho}\fp{\rho}{x_1} + \fp{p}{\phi}\fp{\phi}{x_1} = \frac{x_1}{\sqrt{x_1^2 + x_2^2}}\fp{p}{\rho} + \frac{-x_2}{x_1^2+x_2^2}\fp{p}{\phi} = \cos\phi \fp{p}{\rho} -\frac{\sin\phi}{\rho} \fp{p}{\phi},  \\
\fp{p}{x_2} & = & \fp{p}{\rho}\fp{\rho}{x_2} + \fp{p}{\phi}\fp{\phi}{x_2} = \frac{x_2}{\sqrt{x_1^2 + x_2^2}}\fp{p}{\rho} + \frac{x_1}{x_1^2+x_2^2}\fp{p}{\phi} = \sin\phi \fp{p}{\rho} + \frac{\cos\phi}{\rho} \fp{p}{\phi}.
\eeast

This implies that
\beast
\fpp{p}{x_1} & = & \fp{}{\rho}\brb{\fp{p}{x_1}}\fp{\rho}{x_1} + \fp{}{\phi}\brb{\fp{p}{x_1}}\fp{\phi}{x_1}\\
& = & \cos\phi \fp{}{\rho}\brb{\cos\phi \fp{p}{\rho} -\frac{\sin\phi}{\rho} \fp{p}{\phi} } - \frac{\sin\phi}{\rho} \fp{}{\phi}\brb{\cos\phi \fp{p}{\rho} -\frac{\sin\phi}{\rho} \fp{p}{\phi}  } \\
& = & \brb{\cos\phi}^2 \fpp{p}{\rho} - \frac{\sin\phi\cos\phi}{\rho}  \frac{\partial^2 p}{\partial\rho\partial \phi} + \frac{\sin\phi\cos\phi}{\rho^2}  \fp{p}{\phi}  -\frac{\sin\phi\cos\phi}{\rho} \frac{\partial^2 p}{\partial \phi\partial\rho} + \frac{ \brb{\sin\phi}^2}{\rho} \fp{p}{\rho} + \frac{\sin\phi\cos\phi}{\rho^2} \fp{p}{\phi} + \frac{\brb{\sin\phi}^2}{\rho^2} \fpp{p}{\phi} ,
\eeast

\beast
\fpp{p}{x_1} & = & \fp{}{\rho}\brb{\fp{p}{x_2}}\fp{\rho}{x_2} + \fp{}{\phi}\brb{\fp{p}{x_2}}\fp{\phi}{x_2}\\
& = & \sin\phi \fp{}{\rho}\brb{\sin\phi \fp{p}{\rho} + \frac{\cos\phi}{\rho} \fp{p}{\phi} }  + \frac{\cos\phi}{\rho} \fp{}{\phi}\brb{\sin\phi \fp{p}{\rho} + \frac{\cos\phi}{\rho} \fp{p}{\phi} } \\
& = & \brb{\sin\phi}^2 \fpp{p}{\rho} + \frac{\sin\phi\cos\phi}{\rho}  \frac{\partial^2 p}{\partial\rho\partial \phi} - \frac{\sin\phi\cos\phi}{\rho^2}  \fp{p}{\phi}  + \frac{\sin\phi\cos\phi}{\rho} \frac{\partial^2 p}{\partial \phi\partial\rho} + \frac{ \brb{\cos\phi}^2}{\rho} \fp{p}{\rho} - \frac{\sin\phi\cos\phi}{\rho^2} \fp{p}{\phi} + \frac{\brb{\cos\phi}^2}{\rho^2} \fpp{p}{\phi}.
\eeast

Then we have
\beast
0 = \fp{p}{s} + \frac 12 \brb{\fpp{}{x_1} + \fpp{}{x_2}}p =  \fp{p}{s} + \frac 12 \brb{\fpp{}{\rho} + \frac{1}{\rho} \fp{}{\rho} + \frac{1}{\rho^2} \fpp{}{\phi}}p.
\eeast

Since the forward and backward equation of 2-dimensional Brownian motion have the same term, we can simply replace the $\rho$ and $\phi$ with $r$ and $\theta$
\be
\fp{p}{t} = \frac 12 \brb{\fpp{}{r} + \frac{1}{r} \fp{}{r} + \frac{1}{r^2} \fpp{}{\theta}}p.
\ee

Note that this forward equation cannot be derived by integral by part since the corresponding value will not vanish at the extreme point (i.e., $r=0$).

The transition density is\footnote{see A note on planar Brownian motion, Stella Brassesco, The Annals of Probability, 1992, Vol 20, No 3., 1498-1503.}
\be
p_{s,t}\brb{(\rho,\phi),(r,\theta)} = \frac 1{\pi t}\exp\brb{-\frac{r^2 + \rho^2}{2t}} \int^\infty_0 \cos\brb{\nu(\theta -\phi)}I_{\nu}\brb{\frac{\rho r}{t}} d\nu.
\ee

\section{Spherical Brownian Motion}


\begin{definition}[spherical Brownian motion]
The spherical Brownian motion $B\brb{\sS^d}$ is the diffusion on the spherical surface $\bra{\sS^d:\dabs{x}=1}\subseteq \R^{d+1}$ with infinitesimal generator $\frac 12 \Delta^d$ where the spherical Laplace operator $\Delta^d$ is
\be
\Delta^d = \brb{\sin\phi}^{1-d}\fp{}{\phi}\brb{(\sin\phi)^{d-1}\fp{}{\phi}} + \brb{\sin\phi}^{-2}\Delta^{d-1},\qquad \Delta^1 = \frac{\partial^2}{\partial\theta^2}.
\ee

Note that $\phi$ is the colatitude, the complementary angle of latitude.

In particular, $B(\sS^1)$ is the projection modulo $2\pi$ of the standard 1-dimensional Brownian motion $B(\R^1)$ onto the unit circle $\sS^1$.
\end{definition}


\begin{example}[$B(\sS^2)$]
The infinitesimal generator of $B(\sS^2)$ is
\beast
\frac 12\Delta^2  & = & \frac 12\brb{\brb{\sin\phi}^{-1}\fp{}{\phi}\brb{ \sin\phi \fp{}{\phi}} + \brb{\sin\phi}^{-2} \fpp{}{\theta} }\\
& = & \frac 12 \brb{ \fpp{}{\phi} + \brb{\tan \phi}^{-1} \fp{}{\phi} + \brb{\sin\phi}^{-2} \fpp{}{\theta}}
\eeast

The backward Kolmogorov equation is
\be
\fp{p}{s} + \frac {1}{2\tan\phi} \fp{p}{\phi} + \frac 12 \fpp{p}{\phi} + \frac 1{2\sin^2\phi} \fpp{p}{\theta} = 0
\ee
and forward Kolmogorov equation is
\beast
\fp{p}{t} & = & - \frac 12\fp{}{\phi}\brb{\frac {1}{\tan\phi}p} + \frac 12 \fpp{}{\phi} p+ \frac 12 \fpp{}{\theta}\brb{\frac 1{\sin^2\phi}p} \\
& = &  \frac p{2\sin^2\phi}  - \frac {1}{2\tan\phi} \fp{p}{\phi} + \frac 12 \fpp{p}{\phi}+ \frac 1{2\sin^2\phi} \fpp{p}{\theta}
\eeast%The forward equation is
%\beast
%\fp{}{t}p_{s,t}\brb{(\theta_s,\phi_s),(\theta_t,\phi_t)} & = & \frac 12 \Delta p_{s,t}\brb{(\theta_s,\phi_s),(\theta_t,\phi_t)} \\
%& = & \frac 12  \brb{\fpp{}{\phi_t} + \brb{\tan \phi_t}^{-1} \fp{}{\phi_t} + \brb{\sin\phi_t}^{-2} \fpp{}{\theta_t}}p_{s,t}\brb{(\theta_s,\phi_s),(\theta_t,\phi_t)}
%\eeastfor $\kappa >0$.

Furthermore\footnote{check needed.},
\beast
p_{s,t}\brb{(\theta_s,\phi_s),(\theta_t,\phi_t)} & = & \sum^\infty_{k=0} \exp\brb{-\frac{k(k+1) t}{2}} \sum_{m=-k}^k Y_k^m(\theta_s,\phi_s) \ol{Y_k^m(\theta_t,\phi_t)} \\
& = & \sum^\infty_{k=0} \exp\brb{-\frac{k(k+1) t}{2}} \sum_{m=-k}^k Y_{k,m}(\theta_s,\phi_s) Y_{k,m}(\theta_t,\phi_t) \\
& = & \frac 1{4\pi}\sum^\infty_{k=0} (2k+1)\exp\brb{-\frac{k(k+1) t}{2}} P_k(\cos \vp)
\eeast%\footnote{definition needed. $Y_{n,k}(\theta,\phi)=e^{ik\phi}P_{n}^k(\cos\theta)$ where $P_n^k$ is the associated Legendre function and $P_n^0 = P_n$.}\footnote{$P_n(r\cdot r') = \frac{4\pi}{2n+1} \sum^n_{k=-n} Y_{n,k}(\theta,\phi)Y_{n,k}^*(\theta',\phi')$.}.
where $P_n(\cdot)$ is the $n$th Legendre polynomial, $Y^k_n(\cdot)$ is the spherical harmonics (see Definition \ref{def:spherical_harmonics}), $Y_{n,k}(\cdot)$ is Laplacian (real) spherical harmonics (see Definition \ref{def:spherical_harmonics_laplacian}) and $\phi$ the angle between unit vector $r_s$ and $r_t$ with corresponding coordinates $(\theta_s,\phi_s)$ and $(\theta_t,\phi_t)$. That is,
\be
\cos\vp = \cos\brb{\theta_t - \theta_s}\sin\phi_s\sin\phi_t + \cos\phi_s\cos\phi_t.
\ee
\end{example}

\begin{theorem}[$B(\sS^d) = \leg(d)\times B(\sS^{d-1})$]
\ben
\item [(i)] $B(\sS^2)$ is the skew product of Legendre process $\leg(2)$ with infinitesimal generator
\be
\frac 12 \brb{\sin\phi}^{-1}\fp{}{\phi}\brb{\sin\phi\fp{}{\phi}} ,\qquad \phi \in (0,\pi)
\ee
and an independent spherical Brownian motion $B\brb{\sS^1}$ with the clock
\be
I = \int^t_0 \sin^{-2}\brb{\phi(s)} ds.
\ee

\item [(ii)] $B(\sS^d)$ is the skew product of Legendre process $\leg(d)$ with infinitesimal generator
\be
\frac 12 \brb{\sin\phi}^{1-d}\fp{}{\phi}\brb{\brb{\sin\phi}^{d-1}\fp{}{\phi}} ,\qquad \phi \in (0,\pi)
\ee
and an independent spherical Brownian motion $B(\sS^{d-1})$ with the clock
\be
I = \int^t_0 \sin^{-2}\brb{\phi(s)} ds.
\ee
\een
\end{theorem}

\begin{remark}
See for \footnote{McKean book P270} skew product calculation. (F. Spitzer, Some theorem concerning 2-dimensional Brownian motion, TAMS 87, 187-197 (1958).)
\end{remark}

\begin{proof}[\bf Proof]
\footnote{proof needed.}
\end{proof}



\begin{theorem}[$B(\R^d) = \bes(d)\times B(\sS^{d-1})$]
$B(\R^d)$ is the skew product of Bessel process $\bes(d)$ with infinitesimal generator
\be
\frac 12 \brb{\frac{d-1}{r}\fp{}{r} + \fpp{}{r}},\qquad r>0
\ee
and an independent spherical Brownian motion $B(\sS^{d-1})$ with the clock
\be
I = \int^t_0 \sin^{-2}\brb{(s)} ds.
\ee
\end{theorem}

\begin{proof}[\bf Proof]
\footnote{proof needed.}
\end{proof}


