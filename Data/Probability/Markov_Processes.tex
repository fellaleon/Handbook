\chapter{Markov Processes}

\section{Markov Processes}

\subsection{Markov transition kernel and transition density}

\begin{definition}[kernel\index{kernel}]\label{def:kernel_markov_process}
If $\sB(\sS)$ is the Borel $\sigma$-algebra on a metric space $\sS$, a kernel $Q(x,A)$ on $\sS$ is a map from $\sS\times \sB(\sS)\to \R$ satisfying the following:
\ben
\item [(i)] For each $x\in \sS$, $A\mapsto Q(x,A)$ is a measure on $(\sS,\sB(\sS))$.
\item [(ii)] For each $A\in \sB(\sS)$, the function $x\mapsto Q(x,A)$ is Borel ($\sB(\sS)$) measurable.
\een
\end{definition}

\begin{definition}[probability kernel\index{probability kernel}]
Suppose $(P_t)_{t\geq 0}$ is a collection of kernels mapping from $\sS\times\sB(\sS)\to [0,1]$. Then it is called a collection of probability kernels.
\end{definition}

Then we can define Markov transition probabilities by the kernels.

\begin{definition}[Markov transition kernel\index{Markov transition kernel}]\label{def:markov_transition_kernel}
Let $\sS$ be a separable metric space. Then a collection of probability kernels $(P_{s,t})_{s,t\geq 0}$ are Markov transition kernels\footnote{The first two conditions are actually from definition of probability kernel.}  (or Markov transition probabilities) if %for a Markov process $\bb{(X_t)_{t\geq 0},\pro_x}$ on $(\sS,\sB(\sS))$ if
\ben
\item [(i)] For each $x\in \sS$, $A\mapsto P_{s,t}(x,A)$ is a probability measure on $(\sS,\sB(\sS))$.
\item [(ii)] For each $A\in \sB(\sS)$, the function $x\mapsto P_{s,t}(x,A)$ is Borel ($\sB(\sS)$) measurable.
%$P_{s,t}(x,\sS) = 1$ for each $t>s\geq 0$ and each $x\in \sS$.
\item [(iii)] For each $x\in \sS$, each Borel subset $A$ of $\sS$ ($A\in \sB(\sS)$), and each $\forall s<t<u$,
\be
P_{s,u}(x,A) = \int_{\sS} P_{t,u}(y,A)P_{s,t}(x,dy).\qquad (*)
\ee
which is known as the Chapman-Kolmogorov equation\index{Chapman-Kolmogorov equation}. 
%We can interpret it as $\forall s<t$ and any $A\in \sB(\sS)$,
%\be
%P_{s,t}(x,A) \stackrel{\pro_x\text{-a.s.}}{=} \pro_x\bb{X_{t}\in A|X_s}
%\ee
They can be rephrased in terms of equality of measures: for each $x\in \sS$, %\footnote{details needed.}
\be
P_{s,u}(x,dz) = \int_{y\in \sS}P_{t,u}(y,dz)P_{s,t}(x,dy).
\ee
%\item [(iii)] For each $x\in \sS$, each Borel subset $A$ of $\sS$, and each $t\geq 0$,
%\be
%P_t(x,A) := P_{0,t}(x,A) = \pro_x\bb{X_t\in A}.
%\ee
%We can also say that $\bb{(X_t)_{t\geq 0},\pro_x}$ is a Markov process on $(\sS,\sB(\sS))$ with transition kernels $(P_{s,t})_{s,t\geq 0}$. 
\een
\end{definition}

%\begin{remark}\label{rem:markov_transition_kernel}
%\end{remark}

\begin{definition}[Markov transition density\index{Markov transition density}]%of Markov process $\bb{(X_t)_{t\geq 0},\pro_x}$ on $(\sS,\sB(\sS))$ %of $\bb{(X_t)_{t\geq 0},\pro_x}$
Let $P_t(x,A)$ be a Markov transition kernel where $A$ is Borel ($\sB(\sS)$)-measurable subset of $\sS$. The Markov transition density $p_{s,t}(x,y)$ is defined by
\be
P_{s,t}(x,A) = \int_A p_{s,t}(x,y)dy
\ee
for $x,y\in \sS$ and $s<t$. 
%\be
%p_{s,t}(x,y) := P_{s,t}(x,dy).%= dP_{s,t}(x,y) 
%\ee
%In other words,
\end{definition}

\subsection{Markov processes}

To define a Markov process, we start with a measurable space $(\Omega,\sF)$ and suppose we have a filtration $(\sF_t)_{t\geq 0}$ (not necessarily satisfying the usual conditions).

\begin{definition}[Markov process\index{Markov process}]\label{def:markov_process}
Let $\sS$ be a separable metric space (see Definition \ref{def:separable_space}) and $\sB(\sS)$ be its Borel $\sigma$-algebra. Then a Markov process $\bb{(X_t)_{t\geq 0},\bb{\pro_x}_{x\in \sS}}$ with Markov transition kernels $(P_{s,t})_{s,t\geq 0}$ is a stochastic process $(X_t(\omega) = X(t,\omega))_{t\geq 0}$ on $(\sS,\sB(\sS))$ with respect to $(\Omega,\sF,(\sF_t)_{t\geq 0})$ such that
\be
X :[0,\infty) \times \Omega \to \sS,\qquad X_0 = x
\ee
and a family of probability measures $\bb{\pro_x}_{x\in\sS}$ on $(\Omega,\sF)$ satisfying the following conditions.
\ben
\item [(i)] For each $t$, $X_t$ is $\sF_t$-measurable (i.e., $X$ is adapted with respect to $(\sF_t)_{t\geq 0}$).
\item [(ii)] For each $0\leq s<t$ and each Borel subset $A$ of $\sS$ ($A\in \sB(\sS)$), the map $x\mapsto P_{s,t}(x,A)$ is Borel ($\sB(\sS)$) measurable. That is, $\forall B\in \sB([0,1])$, \
\be
f^{-1}(B) = \bra{x\in \sS: f(x):= P_{s,t}(x,A) \in B} \in \sB(\sS).
\ee

Furthermore, we define $\pro_x$ by
\be
\pro_x(X_t \in A) := P_{0,t}(x,A) 
\ee

Then we define the conditional probability by 
\be
\pro_x(X_t \in A|X_s) := P_{s,t}(X_s, A)\qquad \pro_x\text{-a.s.}.
\ee

\item [(iii)] For each $t>s\geq 0$, $x\in \sS$ and any bounded Borel measurable function $f:\sS\to \R$, we have $\E_x\bb{\left.f\bb{X_{t}}\right|\sF_s}$ is $\sigma(X_s)$-measurable where $\sigma(X_s)$ is generated by $X_s^{-1}(B)$ for $B\in \sB(\sS)$ and
\beast
\E_x\bb{\left.f\bb{X_{t}}\right|\sF_s} & \stackrel{\pro_x\text{-a.s.}}{=} & \E_x\bb{\left.f\bb{X_{t}}\right| X_s} \qquad (*).
\eeast



%or we can express it by 

% original definition
%\item [(iii)] For each $s,t\geq 0$, each Borel subset $A$ of $\sS$, and each $x\in \sS$, we have
%\be
%\pro_x\bb{X_{s+t}\in A|\sF_s} := \E_x\bb{\left.\ind_{\bra{X_{s+t}\in A}}\right|\sF_s}\text{ is $\sigma(X_s)$-measurable}
%\ee
%where $\sigma(X_s)$ is generated by $X_s^{-1}(B)$ for $B\in \sB(\sS)$ and
%\beast
%\pro_x\bb{X_{s+t}\in A|\sF_s} & \stackrel{\pro_x\text{-a.s..}}{=} & \pro_x\bb{X_{s+t}\in A|X_s}   \qquad (*) \\
%& \stackrel{\pro_x\text{-a.s..}}{=}  & \pro_{X_s}(X_t\in A).% = P_t(X_s,A),
%\eeast
%or we can express it by 
%\be
%\E_x\bb{\left.\ind_{\bra{X_{s+t}\in A}}\right|\sF_s} = \E_x\bb{\left.\ind_{\bra{X_{s+t}\in A}}\right|\sigma(X_s)} = \E_{X_s}\bb{\ind_{\bra{X_t\in A}}},\quad \pro_x\text{-a.s..}
%\ee

$(*)$ means that we can predict Markov process for the future based solely on its present state just as well as one could knowing the process's full history.
\een
\end{definition}

\begin{remark}
\ben
\item [(i)] Usually, we take $\sS$ to be $\R$. 
\item [(ii)] Note that we can take $\sS$ to be $\R^2$. That is, $((X_t,Y_t)_{t\geq 0},\bb{\pro_{x,y}}_{(x,y)\in \R^2})$ can be made up by multiple stochastic processes.
\item [(iii)] For discrete-time Markov chains where $\sS$ is a discrete set with the discrete $\sigma$-algebra and $n \in \mathbb{N}$, (iii) can be reformulated as follows:
\be
\pro(X_n=x_n|X_{n-1}=x_{n-1},X_{n-2}=x_{n-2}, \dots, X_0=x_0)=\pro(X_n=x_n|X_{n-1}=x_{n-1}).
\ee
\item [(iv)] Note that Markov transition kernel in Markov processes plays the role of the transition matrix $P$ in Markov Chains. 
%\item [(v)] It is easy to see that
%\be
%\pro_x(X_t\in A) = \E_x\bb{\pro_x(X_t)}
%\ee
%which is Chapman-Kolmogorov equation in Definition \ref{def:markov_transition_kernel}.(iii).
\een
\end{remark}

(iii) in Definition \ref{def:markov_process} can be rephrased as saying that for each $x$, the measures $P_{s,t}(x,dy)$ and $\pro_{x}(X_t\in dy|X_s)$ are the same. Then for proper function $f$ we can have the following lemma.

\begin{lemma}\label{lem:markov_transition_kernel_on_function}
Let $\bb{(X_t)_{t\geq 0},\bb{\pro_x}_{x\in \sS}}$ be a Markov process on $(\sS,\sB(\sS))$ with Markov transition kernels $\bb{P_{s,t}}_{s,t\geq 0}$ with respect to $(\Omega,\sF,(\sF_t)_{t\geq 0})$. Let $f:\sS\to \R$ be a Borel measurable and non-negative $\pro_x$-a.s. (or bounded $\pro_x$-a.s.) function %with respect to measure $P_{s,t}(x,\cdot)$ 
and define 
\be
P_{s,t} f(x) := \int f(y)P_{s,t}(x,dy).
\ee
for any $x\in \sS$. In paricular, $P_{t,t} f(x) := f(x)$ as $P_{t,t}(x,dy)$ is actually a Dirac mass at $x$\footnote{details needed.}. Then $P_{s,t}f(X_s)$ is non-negative $\pro_x$-a.s. (or bounded $\pro_x$-a.s., repectively) and Borel measurable and for any $x\in \sS$,
\be
P_{s,t} f(X_s) \stackrel{\pro_x\text{-a.s.}}{=} \E_x\bb{f(X_t)|X_s}.
\ee
\end{lemma}

\begin{proof}[\bf Proof]
By Proposition \ref{pro:conditional_probability_density_function}, %Theorem \ref{thm:image_measure_probability}.% and Remark \ref{rem:markov_transition_kernel}.
\be
P_{s,t} f(X_s) := \int f(y)P_{s,t}(X_s,dy) =\int f(y)\pro_x(X_t \in dy|X_s) %= \int f(y) \mu_{X_t}(dy) \\& = & \int f(y) d\mu_{X_t}(y) = \int f(X_t(\omega)) d\pro_x(\omega) = 
\stackrel{\pro_x\text{-a.s.}}{=} \E_x\bb{f(X_t)|X_s}
\ee
%where $\mu_{X_t} = \pro_x \circ X_t^{-1}$. 

Then it is non-negative $\pro_x$-a.s. by the similar argument in Lemma \ref{lem:non_negative_bounded_conditional_expectation}. Also, for any $A\in \sigma(X_s)$,
\be
\E_x\bb{P_{s,t}f(X_s)\ind_A} = \E_x\bb{\E_x\bb{f(X_t)|X_s}\ind_A} = \E_x\bb{f(X_t)\ind_A} < \infty
\ee
by Proposition \ref{pro:conditional_expectation_basic_property}.(i). Let $A=\Omega$, we have
$P_{s,t}f(X_s)$ is bounded $\pro_x$-a.s..

For the measurability, we first take $f(x) = \ind_A$ where $A\in \sB(\sS)$, then
\be
P_{s,t} f(X_s) = \int_{y\in \sS} f(y) P_{s,t}(X_s,dy) = \int_{y\in A}P_{s,t}(X_s,dy) = P_{s,t}(X_s,A) %= \pro_x(X_t\in A|X_s)
\ee 
which is Borel measurable by Definition \ref{def:markov_process}.(ii).

Then by linearity this holds for simple funcitons, and then using monotone convergence it holds for non-negative functions. Using linearity again, we have measurability conclusion holds for all bounded functions.%\footnote{see proof of Theorem \ref{thm:monotone_convergence_pointwise} for reference.} 
\end{proof}

\begin{proposition}%Let $(X_t,\pro_x)$ be a Markov process. 
In Definition \ref{def:markov_process}, the condition ($*$) is equivalent to the following:
\ben
\item [(i)]  For any Borel subset $A\in \sB(\sS)$, we have
\be
\E_x\bb{\left.\ind_A\bb{X_{t}}\right|\sF_s} \stackrel{\pro_x\text{-a.s.}}{=}  \E_x\bb{\left.\ind_A\bb{X_{t}}\right| X_s}
\ee
which is
\be
\pro_x\bb{\left.X_{t}\in A\right|\sF_s} \stackrel{\pro_x\text{-a.s.}}{=} \pro_x\bb{\left. X_{t}\in A\right| X_s}.
\ee
\item [(ii)] For any bounded $\sigma\bb{X_t:t\geq s}$-measurable function $Y$,
\be
\E_x\bb{Y|\sF_s} \stackrel{\pro_x\text{-a.s.}}{=} \E_x \bb{Y|X_s}.\qquad (*)
\ee

Equivalently, for any $A\in \sigma\bb{X_t:t\geq s}$, we have
\be
\E_x\bb{\ind_A|\sF_s} \stackrel{\pro_x\text{-a.s.}}{=} \E_x\bb{\ind_A|X_s} \qquad (\dag)
\ee
which can also expressed by
\be
\pro_x\bb{A|\sF_s} \stackrel{\pro_x\text{-a.s.}}{=} \pro_x\bb{A|X_s}.
\ee

\item [(iii)] For any continuous function $f:\sS\to \R$ with compact support, that is, $\forall f\in C_c(\sS)$,
\be
\E_x\bb{\left.f\bb{X_{t}}\right|\sF_s} \stackrel{\pro_x\text{-a.s.}}{=} \E_x\bb{\left.f\bb{X_{t}}\right| X_s} .\qquad (\dag\dag)
\ee
\een
\end{proposition}

\begin{proof}[\bf Proof]
\ben
\item [(i)] Obviously, $f = \ind_A$ implies (i). Then we can have ($*$) in Definition \ref{def:markov_process} by applying the fact (i) and the results of simple funcitons, bounded non-negative functions and bounded functions (see Theorem \ref{thm:monotone_convergence_conditional_expectation}, Theorem \ref{thm:dominated_convergence_conditional_expectation}).

\item [(ii)] Similar with (i), we can have $(*)$ and $(\dag)$ are equivalent by Theorem \ref{thm:monotone_convergence_conditional_expectation}, Theorem \ref{thm:dominated_convergence_conditional_expectation}. 

It is also obvious that $(*)$ implies $(*)$ in Definition \ref{def:markov_process} as $f(X_t)$ is bounded random variable (measurable function). So we need to prove that $(*)$ in Definition \ref{def:markov_process} implies $(\dag)$.

Let $B\in \sigma\bb{X_t:t\geq s}$ such that 
\be
B = B_1\cap B_2\cap \dots \cap B_n
\ee
where $B_i \in \sigma(X_{t_i})$ with $s\leq t_1<t_2<\dots t_n$. Then
\beast
\E_x\bb{\ind_B|\sF_s} & \stackrel{\pro_x\text{-a.s.}}{=} & \E_x\bb{\E_x\bb{\dots \E_x\bb{\ind_B|\sF_{t_{n-1}}} \dots |\sF_{t_1}}|\sF_s} \\
& \stackrel{\pro_x\text{-a.s.}}{=} & \E_x\bb{\E_x\bb{\dots \E_x\bb{\ind_{B_n}|\sF_{t_{n-1}}} \dots \ind_{B_2}|\sF_{t_1}}\ind_{B_1}|\sF_s}
\eeast

We know that by ($*$) in Definition \ref{def:markov_process} we can define $f_n(X_{t_{n-1}}) := \E_x(\ind_{B_n}|\sF_{t_{n-1}}) \stackrel{\pro_x\text{-a.s.}}{=}  \E_x(\ind_{B_n}|\sF_{t_{n-1}})$ is $\pro_x$-a.s. bounded measurable function of $X_{t_{n-1}}$ by Lemma \ref{lem:markov_transition_kernel_on_function}. Therefore,
\be
f_2(X_{t_1}):= \E_x\bb{\dots \E_x\bb{\ind_{B_n}|\sF_{t_{n-1}}} \dots \ind_{B_2}|\sF_{t_1}}
\ee
is a $pro_x$-a.s. bounded measurable function of $X_{t_1}$. Then by ($*$) in Definition \ref{def:markov_process} again, we can have 
\beast
\E_x\bb{\E_x\bb{\dots \E_x\bb{\ind_{B_n}|\sF_{t_{n-1}}} \dots \ind_{B_2}|\sF_{t_1}}\ind_{B_1}|\sF_s}  & \stackrel{\pro_x\text{-a.s.}}{=} & \E_x\bb{\E_x\bb{\dots \E_x\bb{\ind_{B_n}|\sF_{t_{n-1}}} \dots \ind_{B_2}|\sF_{t_1}}\ind_{B_1}|X_s} \\
&  \stackrel{\pro_x\text{-a.s.}}{=} & \E_x\bb{\ind_B|X_s}. 
\eeast

Since $B$ is in a $\pi$-system generating $\sigma\bb{X_t:t\geq s}$, we can have that for any $A\in \sigma\bb{X_t:t\geq s}$, 
\be
\E_x\bb{\ind_A|\sF_s}  \stackrel{\pro_x\text{-a.s.}}{=} \E_x\bb{\ind_A|X_s}.
\ee

\item [(iii)] $C_c(\sS)$ is continous on a compact set, which will imply that $\forall f\in C_c(\sS)$, $f(x)$ is compact as well and thus bounded (as $f$ is real-valued). In other words, ($*$) in Definition \ref{def:markov_process} implies (iii).%\footnote{proof needed.}

Also, since $C_c(\sS)$ is dense in $\sL^p(\sS,\sB(\sS),\pro_x)$\footnote{theorem needed.}, $(\dag\dag)$ holds for all bounded measurable functions. This is ($*$) in Definition \ref{def:markov_process}.
\een
\end{proof}

%In particular, if $f(x) = \ind_A(x)$ with Borel subset $A$ of $\sS$, we have
%\be
%\pro_x\bb{X_{s+t}\in A|\sF_s} \stackrel{\pro_x\text{-a.s.}}{=} \pro_x\bb{X_{s+t}\in A|X_s}  \stackrel{\pro_x\text{-a.s.}}{=}  \pro_{X_s}(X_t\in A).% = P_t(X_s,A),
%\ee

%homogeneous
%\\
%& \stackrel{\pro_x\text{-a.s.}}{=} & \E_{X_s}\bb{f\bb{X_t}}.




%\begin{remark}
%\end{remark}



\subsection{Transformation of Markov process}

\begin{theorem}[transformation of Markov process]\label{thm:transformation_of_markov_process}
Let $(\sS,\sB(\sS))$ and $(\sS',\sB(\sS'))$ be measurable spaces where $\sS$ and $\sS'$ are seperable metric spaces, and suppose that the measurable function $\phi:(\sS,\sB(\sS))\to (\sS',\sB(\sS'))$ is (onto) surjective, i.e., for any $y\in \sS'$, there is $x\in \sS$ such that $\phi(x) = y$.
%Suppose that for any bounded and Borel measurable function $f$ on $(\sS,\sB(\sS))$ and $s,t\geq 0$, $\E_x\bb{f(X_{t+s})|\sF_s}$ is $\sigma(X_s)$-measurable and 
%\be
%\E_x\bb{f(X_{t+s})|\sF_s} = \E_x\bb{f(X_{t+s})|\sigma(X_s)} = \E_{X_s}\bb{f(X_t)} \qquad \pro_x\text{-a.s.}.\qquad (\dag)
%\ee
% where $\sF_t = \sF^{X}_t = \sigma(X_s:s\leq t)$ is the natural filatration. 

Let $\bb{(X_t)_{t\geq 0},\pro_x}$ be a Markov process on $(\sS,\sB(\sS))$ with Markov transition kernels $(P_{s,t})_{s,t\geq 0}$ with respect to $(\Omega,\sF,(\sF_t)_{t\geq 0})$\footnote{Note that the natural filtration $\sigma(X_s:s\leq t)=\sF^{X}_t \subseteq \sF_t$ for any $t\geq 0$.} and $Y_t = \phi(X_t)$. Let $(Q_{s,t})_{s,t\geq 0}$ be a collection of probability kernels on $\sS'$ such that for all bounded $\Q_{\phi(x)}$-a.s. and Borel measurable function $f:(\sS',\sB(\sS'))\to \R$, for any $x\in \sS$ and any $t>s\geq 0$,
\be
P_{s,t}(f\circ \phi) (x) = Q_{s,t} f (\phi(x))\qquad (*)
\ee
where for any $ A'\in \sB(\sS')$,
\beast
\Q_{y}(Y_t\in A') & = & \Q_{\phi(x)}(\phi(X_t)\in A')  := Q_{0,t}({\phi(x)},A'), \\
\Q_{y}(Y_t\in A'|Y_s) & = & \Q_{\phi(x)}\bb{\phi(X_t)\in A'|\phi(X_s)} \stackrel{\Q_{\phi(x)}\text{-a.s.}}{:=} Q_{s,t}({\phi(x)},A') .
\eeast

Then $\bb{(Y_t)_{t\geq 0}, \Q_y}$ ($\bb{(\phi(X_t))_{t\geq 0}, \Q_{\phi(x)}}$) is a Markov process on $(\sS',\sB(\sS'))$ with Markov transition kernels $(Q_{s,t})_{s,t\geq 0}$ with respect to $(\Omega,\sF,(\sF_t^{\phi(X)})_{t\geq 0})$.
\end{theorem}

\begin{remark}
Note that the filtration is $\sF_t^{\phi(X)}$ but not $\sF_t$ for $X$. This is reasonable as $\sigma(\phi(X_t))$ may not contain enough information as some of the information can be lost during the transformation.
\end{remark}

\begin{proof}[\bf Proof]
First, we prove that $(Q_{s,t})_{s,t\geq 0}$ is Markov transition kernerls.

(i) and (ii) in Definition \ref{def:markov_transition_kernel} is satisfied since $\bb{Q_{s,t}}_{s,t\geq 0}$ is known as a collection of probability kernels on $\sS'$.

%\item [(ii)] Since $Q_{s,t})_{s,t\geq 0}$

For (iii) in Definition \ref{def:markov_transition_kernel}, we can see that $f\circ \phi$ is bounded and Borel ($\sB(\sS)$) measurable by Proposition \ref{pro:composition_measurable} and thus $P_t(f\circ \phi)$ is well-defined. 

Then by ($*$) for $\forall A'\in \sB(\sS')$, and $A = \phi^{-1}(A')\in \sB(\sS)$, for any $0\leq s< t$,
\be
Q_{s,t}\bb{\phi(x),A'} = Q_{s,t} \ind_{A'} (\phi(x)) = P_{s,t}\bb{\ind_{A'}\circ \phi}(x) = P_{s,t}(\ind_A)(x).
\ee

%Then by Lemma \ref{lem:markov_transition_kernel_on_function} and semigroup property (Proposition \ref{pro:semigroup_homogeneous_markov_process_transition}), we have for $t,s\geq 0$,
%\be
%(Q_{t+s}f)\circ \phi = P_{t+s}(f\circ \phi) = P_tP_s(f\circ \phi) = P_t ((Q_s f) \circ \phi) = %(Q_tQ_sf)\circ \phi.
%\ee

%Note that $Q_{s,t} f$ is well-defined as $Q_{s,t}$ is probability kernel and $f$ is bounded and thus $Q_{s,t}f$ is bounded . Since $\phi$ is onto, we have\footnote{details needed.}
%\be
%Q_{t+s}f = Q_tQ_sf.
%\ee

%This is,
%\be
%\int_{z\in \sS'} f(z)Q_{t+s}(x,dz) = \int_{z\in \sS'} Q_sf(z)Q_{t}(x,dz) = \int_{z\in \sS'} \int_{y\in \sS'}f(y) Q_s(z,dy) Q_{t}(x,dz).
%\ee

%Then by letting $f(x) = \ind_{x\in B}$ where $B\in \sB(\sS')$, we can have 

Thus, for any $0\leq s<t<u$  and each Borel subset $A'\in \sB(\sS')$
\be
Q_{s,u}(\phi(x),A') = P_{s,t}(\ind_A)(x) = \int_{\sS} P_{t,u}(y,A)d P_{s,t}(x,y) = \int_{\phi(\sS)} Q_{s,t}(\phi(y),A') d Q_{s,t}(\phi(x),\phi(y)).  
%\int_{z\in \sS'} f(z)Q_{t+s}(x,dz)  = \int_{z\in \sS'} \int_{y\in \sS'}f(y) Q_s(z,dy) Q_t(x,dz) = \int_{z\in \sS'} Q_s(z,B) Q_t(x,dz)
\ee

%This is actually (ii) in Definition \ref{def:markov_transition_kernel}.

%\item [(iii)] Then we can define 
%\be
%\Q_{\phi(x)}(\phi(X_t)\in A') := Q_t({\phi(x)},A'),\qquad \forall A'\in \sB(\sS').
%\ee
%\een

Now we prove that $\phi(X_t)$ is a Markov process.
\ben
\item [(i)] Since $\phi$ is Borel measurable we can have that $f\circ \phi$ is Borel measurable by Proposition \ref{pro:composition_measurable}. Thus, $\phi(X_{t})$ is $\sF_t^X$(and $\sF_t$)-measurable.

\item [(ii)] We have $Q_{s,t}(y,A')$  is Borel measurable (wrt $\sB(\sS')$) from the assumption that $Q_{s,t}(y,A')$ is a probability kernel.

Then we define $\Q_y$ by %$\Q_{\phi(x)}(\phi(X_t)\in A'|\phi(X_s)) = Q_{s,t}(\phi(x),A')$
\beast
\Q_{y}(Y_t\in A') & = & \Q_{\phi(x)}(\phi(X_t)\in A')  := Q_{0,t}({\phi(x)},A'), \\
\Q_{y}(Y_t\in A'|Y_s) & = & \Q_{\phi(x)}\bb{\phi(X_t)\in A'|\phi(X_s)} \stackrel{\Q_{\phi(x)}\text{-a.s.}}{:=} Q_{s,t}({\phi(x)},A') .
\eeast

%\be
%(Q_t\ind_B)\circ \phi = P_t(\ind_B\circ \phi) = P_t\ind_{\bra{\phi(x)\in B}}
%\ee by Lemma \ref{lem:markov_transition_kernel_on_function}.




%Let  $B\in \sB(\sS')$, $C\in \sF_s$ and $f(y) = \ind_C\ind_{y\in B}$\footnote{}. Then by $(*)$
%\beast
%\E^\Q_{\phi(x)}\bb{\ind_{\phi(X_{t+s})\in B}\ind_C} & = & Q_{t+s} (\ind_C\circ \ind_B)\circ \phi (x) = P_{t+s}\bb{(\ind_C\circ \ind_B)\circ\phi} (x) \\
%& = & P_t P_s\bb{(\ind_C\circ \ind_B)\circ\phi} (x)
%\eeast\\ 
%& = & \E^{\pro}_{x}\bsb{\E^{\pro}_{x}\bb{\left.\prod^n_{k=2}f_k(\phi(X_{t_k}))\right|\sF_{t_1} }f_1(\phi(X_{t_1})) } \\
%& = & \E^{\pro}_{x}\bsb{f_1(\phi(X_{s_1}))\E^{\pro}_{X_{t_1}}\bb{\prod^n_{k=2}%f_k(\phi(X_{t_k-t_1}))} } = \E^{\pro}_{x}\bsb{f_1(\phi(X_{s_1}))\E^{\pro}_{X_{t_1}}\bb{\E^{\pro}_{X_{t_1}}\bb{\left.\prod^n_{k=2}f_k(\phi(X_{t_k-t_1}))} \right|\sF_{t_2}}}

%Then by  $(*)$, we have that 
%\be
%P_{s_1}\bb{\bb{f_1\circ \phi} P_{s_2} \bb{\bb{f_2\circ \phi } \dots P_{s_n}\bb{f\circ \phi}\dots}} (x) = \bb{Q_{s_1}f_1\bb{Q_{s_2}f_2\bb{\dots Q_{s_n}f_{n}\dots}}}\circ \phi(x).\qquad (\dag)
%\ee

%Now we can pick any $B\in \sigma(\phi\circ X_s)$, we know that we can find $A'\in \sB(\sS')$ such that $B = (\phi\circ X_s)^{-1}(A')$ as $\sigma(X_s)$ is smallest $\sigma$-algebra with respect to $\phi\circ X_s$ (see Proposition \ref{pro:inverse_image_preserves_set_operation_measure}). %Definition \ref{def:sigma_algebra_generated_by_measurable_function} and
%Accordingly, we can find $A\in \sB(\sS)$ such that $A = \phi^{-1}(A')$ as $\phi$ is Borel measurable.


%For any $B\in \sigma(X_s)$, 

%Then we can have %find a subset $A\in \sB(\sS)$ such that $X_s^{-1}(A) = B$  

%Thus, we have (as $f\bb{\phi(X_{t+s})}\ind_{\bra{\phi(X_s)\in A'}}$ is bounded and Borel measurable) by $(\dag)$,
%\beast
%\E^{\Q}_{\phi(x)}\bb{f\bb{\phi(X_{t+s})}\ind_B} & = & \E^{\Q}_{\phi(x)}\bb{f\bb{\phi(X_{t+s})}\ind_{\bra{\phi(X_s)\in A'}}} = \bb{Q \bb{f\bb{\phi(X_{t+s})}\ind_{\bra{\phi(X_s)\in A'}}}} \circ \phi(x) \\%
%& = & (Q_{t+s}\wt{f})\circ \phi(x) = P_{t+s}\bb{\wt{f}\circ \phi}(x) = P_s\bb{(\ind_{A'} \circ \phi)P_t(f \circ \phi)}(x) \\
%& = & \bb{Q_s \ind_{A'}(Q_t f)}\circ \phi(x) = \E^{\Q}_{\phi(x)}\bb{\ind_{\bra{\phi(X_s)\in A'}}\E_{\phi(X_s)}^{\Q}\bb{f\bb{\phi(X_t)}}} \\
%& = &  \E^{\Q}_{\phi(x)}\bb{\E_{\phi(X_s)}^{\Q}\bb{f\bb{\phi(X_t)}}\ind_B}.
%\eeast

%For any $C'\in \sB(\sS')$, we have $\phi^{-1}(C') = C\in \sB(\sS)$ since $\phi$ is Borel measurable. Now we can pick any $B\in \sigma(\phi\circ X_s)$, we know that we can find $A'\in \sB(\sS')$ such that $B = (\phi\circ X_s)^{-1}(A')$ as $\sigma(X_s)$ is smallest $\sigma$-algebra with respect to $\phi\circ X_s$ (see Proposition \ref{pro:inverse_image_preserves_set_operation_measure}). %Definition \ref{def:sigma_algebra_generated_by_measurable_function} and
%Accordingly, we can find $A\in \sB(\sS)$ such that $A = \phi^{-1}(A')$ as $\phi$ is Borel measurable.

%\beast
%\E^{\Q}_{\phi(x)}\bb{\ind_{\bra{\phi(X_{t+s})\in C'}}\ind_B} & = & \E^{\Q}_{\phi(x)}\bb{\ind_{\bra{\phi(X_{t+s})\in C'}}\ind_{\bra{\phi(X_s)\in A'}}} = \bb{Q \bb{\ind_{\bra{\phi(X_{t+s})\in C'}}\ind_{\bra{\phi(X_s)\in A'}}}} \circ \phi(x) \\%
%& = & (Q_{t+s}f)\circ \phi(x) = P_{t+s}(f\circ \phi)(x) = P_s\bb{(\ind_{A'} \circ \phi)P_t(\ind_{C'}\circ \phi)}(x) \\
%& = & \bb{Q_s \ind_{A'}(Q_t\ind_{C'})}\circ \phi(x) = \E^{\Q}_{\phi(x)}\bb{\ind_{\bra{\phi(X_s)\in A'}}\E_{\phi(X_s)}^{\Q}\bb{\ind_{\bra{\phi(X_t)\in C'}}}} \\
%& = &  \E^{\Q}_{\phi(x)}\bb{\ind_B\E_{\phi(X_s)}^{\Q}\bb{\ind_{\bra{\phi(X_t)\in C'}}}}
%\eeast

%\be
%\E_x^{\pro}\bb{f\circ \phi(X_{t+s})\ind_{\bra{X_{s}\in A}}} = \E_x^{\pro}\bb{f\circ \phi(X_{t+s})\ind_{\bra{\phi(X_{s})\in A'}}} 
%\ee
%where $A'\in \sB(\sS)$ such that $(\phi\circ X_s)^{-1}(A') = B$ (as $\phi$ is Borel measurable). Therefore, $B\in \sigma(\phi(X_s))$.  

%Thus, we can have that 
%\be
%\E_x^{\pro}\bb{\ind_{\bra{X_{t+s\in C}}}\ind_{\bra{X_s\in A}}} = \E_x^{\pro}\bb{\E_{X_s}^{\pro}\bb{\ind_{\bra{X_{t\in C}}}}\ind_{\bra{X_s\in A}}}.
%\ee

%Now let $C'\in \sB(\sS')$ such that $\phi^{-1}(C')=C$. Then by $(**)$ we have % and therefore we can have $(\phi\circ X_s)^{-1}(C')=B$ .
%\beast
%\E_x^{\pro}\bb{\ind_{\bra{X_{t+s\in C}}}\ind_{\bra{X_s\in A}}} & = & \E_x^{\pro}\bb{\ind_{\bra{\phi(X_{t+s})\in C'}}\ind_{\bra{\phi(X_s)\in A'}}} = P_s\bb{(\ind_{A'} \circ \phi)P_t(\ind_{C'}\circ \phi)}(x) \\
%& = & \bb{Q_s \ind_{A'}(Q_t\ind_{C'})}\circ \phi(x) = \E^{\Q}_{\phi(x)}\bb{\ind_{\bra{\phi(X_s)\in A'}}\E_{\phi(X_s)}^{\Q}\bb{\ind_{\bra{\phi(X_t)\in C'}}}} \\
%& = &  \E^{\Q}_{\phi(x)}\bb{\ind_B\E_{\phi(X_s)}^{\Q}\bb{\ind_{\bra{\phi(X_t)\in C'}}}}
%\eeast

%Thus, we have 
%\beast
%\E^{\Q}_{\phi(x)}\bb{f\bb{\phi(X_{t+s})}|\phi(X_s)} \stackrel{\text{$\Q_{\phi(x)}$-a.s.}}{=}  \E_{\phi(X_s)}^{\Q}\bb{f\bb{\phi(X_t)}} .
%\eeast


%Thus, we have 
%\beast
%\Q_{\phi(x)}\bb{\phi(X_{t+s})\in C'|\phi(X_s)} & = &  \E^{\Q}_{\phi(x)}\bb{\left.\ind_{\bra{\phi(X_{t+s})\in C'}}\right|\sigma(\phi(X_s)) } \\
%& \stackrel{\text{$\Q_{\phi(x)}$-a.s.}}{=} & \E_{\phi(X_s)}^{\Q}\bb{\ind_{\bra{\phi(X_t)\in C'}}} = \Q_{\phi(X_s)}\bb{\phi(X_t)\in C'}.
%\eeast
%$s_k = t_k - t_{k-1}$ 

\item [(iii)]  Now we take $0=t_0 \leq t_1 \leq \dots \leq t_n= s\leq t $ and $f_k$ be bounded $\sB(\sS')$-measurable function. Then by definition of Markov process and tower property (Proposition \ref{pro:conditional_expectation_tower_independence}.(i)) and Lemma \ref{lem:markov_transition_kernel_on_function}, %define $f_{m}'(\phi(X_{t_{m-1}})) := P_{t_{m-1},t_m}f_m\bb{\phi(X_{t_{m-1}})}$,
\beast% (Q_{t+s}\prodf) \circ (\phi(x)) = \E^\Q_{\phi(x)}\bb{\prod_n_{k=1}f_k(\phi(X_{t_k}))} = = P_{t+s}(f\circ \phi(x)) = P_tP_s(f\circ \phi(x)) = P_t\bb{(Q_sf)\circ \phi(x)} =  (Q_tQ_sf)\circ \phi(x)
\E^{\pro}_{x}\bsb{\prod^n_{k=1}f_k(\phi(X_{t_k}))} & = & \E^{\pro}_{x}\bsb{\E^{\pro}_{x}\bb{\left.\prod^n_{k=1}f_k(\phi(X_{t_k}))\right| X_{t_1} }} = \dots \\
& = &  \E^{\pro}_{x}\bsb{f_1(\phi(X_{t_1})) \E^{\pro}_{x}\bb{\left.\dots f_{n-1}(\phi(X_{t_{n-1}})) \E^{\pro}_{x}\bb{ \left. f_{n}(\phi(X_{t_n}))\right| X_{t_{n-1} }} \dots \right| X_{t_1} }} \\
& = &  \E^{\pro}_{x}\bsb{f_1(\phi(X_{t_1})) \E^{\pro}_{x}\bb{\left.\dots f_{n-2}(\phi(X_{t_{n-2}})) \E^{\pro}_{x}\bb{ \left. f_{n-1}(\phi(X_{t_{n-1}})) P_{t_{n-1},t_n}f_n(\phi(X_{t_{n-1}}))\right|X_{t_{n-2} } } \dots \right|X_{t_1} }} \\
& = &  \E^{\pro}_{x}\bsb{f_1(\phi(X_{t_1})) \E^{\pro}_{x}\bb{\left.\dots f_{n-2}(\phi(X_{t_{n-2}})) P_{t_{n-2},t_{n-1}}\bb{f_{n-1}(\phi(X_{t_{n-2}})) P_{t_{n-1},t_n} f_n (\phi(X_{t_{n-2}}))  } \dots \right|X_{t_1} } }\\
& = & P_{0,t_1}\bb{\bb{f_1\circ \phi} P_{t_1,t_2} \bb{\bb{f_2\circ \phi } \dots P_{t_{n-1},t_n}\bb{f_n\circ \phi}\dots}} (x).
\eeast%& = & \E^{\pro}_{x}\bsb{f_1(\phi(X_{s_1})) \E^{\pro}_{X_{s_1}}\bb{\dots f_{n-2}(\phi(X_{t_{n-2}})) \E^{\pro}_{X_{t_{n-2}}}\bb{ f_{n-1}(\phi(X_{s_{n-1}})) \E^{\pro}_{X_{s_{n-1}}}\bb{f_n(\phi(X_{s_n}))} }\dots }} \\

Then by  $(*)$, we have that 
\beast
P_{0,t_1}\bb{\bb{f_1\circ \phi} P_{t_1,t_2} \bb{\bb{f_2\circ \phi } \dots P_{t_{n-1},t_n}\bb{f\circ \phi}\dots}} (x) = \bb{Q_{0,t_1}f_1\bb{Q_{t_2,t_2}f_2\bb{\dots Q_{t_{n-1},t_n}f_{n}\dots}}}\circ \phi(x).\qquad (**)
\eeast

Since $(X_t,\pro_x)$ is a Markov process, we have for any $t>s\geq 0$ and any bounded and Borel measurable $f$,
%\be
%\E_x^{\pro}\bb{\ind_{\bra{X_{t+s\in C}}}|\sigma(X_s)} = \E_{X_s} \bb{\ind_{\bra{X_t\in C}}} \qquad \pro_x\text{-a.s.}. 
%\ee
\be
\E^{\pro}_x\bb{f(X_t)|\sF_s} =  \E^{\pro}_x\bb{f(X_t)|X_s}\qquad \pro_x\text{-a.s.}
\ee

Thus, for any $C\in \sF_s$,
\be
\E^{\pro}_x\bb{f(X_t)\ind_C} = \E^{\pro}_x\bb{\E^{\pro}_x\bb{f(X_t)|X_s} \ind_C}.
\ee

Then for any measurable function $\phi:(\sS,\sB(\sS)) \to (\sS',\sB(\sS'))$, $f\circ \phi$ is bounded and Borel measurable. Then we have
\be
\E^{\pro}_x\bb{f\circ \phi (X_{t})\ind_C} = \E^{\pro}_x\bb{\E^{\pro}_{x}\bb{f\circ \phi(X_t)|X_s} \ind_C} .\qquad (\dag)%= \E^{\pro}_x\bb{\E^{\Q}_{\phi(X_s)}\bb{f(\phi(X_t))} \ind_C}.
\ee

%Note that $\E^{\Q}_{\phi(X_s)}\bb{f(\phi(X_t))} $ is bounded $\Q_{\phi(x)}$-a.s. and Borel function as well by Lemma \ref{lem:markov_transition_kernel_on_function}. 

We know from Proposition \ref{pro:process_law_is_uniquely_determined_by_its_finite_dimensional_marginal_distribution} that 
\be
\sD_s := \bra{\omega:\bigcap_{u\in J,u\leq s}\bra{\phi(X_u(\omega))\in A_u}:J\text{ is finite, }A_u\in \sB(\sS')}
\ee
is a $\pi$-system generating $ \sF_s^{\phi(X)}$. Then $\forall D\in \sD_s \subseteq \sF_s^{\phi(X)}$ and we can have that the function (for particular $J$)
\be
\ind_D = \ind_{\bra{\omega:\bigcap_{u\in \bra{t_1,\dots,t_n},u\leq s}\bra{\phi(X_u)\in A_u}:A_u\in \sB(\sS')}} = \prod_{u\in \bra{t_1,\dots,t_n},u\leq s} \ind_{\bra{\phi(X_u)\in A_u}} = \prod_{i=1}^n \ind_{D_i}
\ee
is bounded and Borel measurable as $\ind_{\bra{X_u\in \phi^{-1}(A_u)}}$ is Borel measurable on $(\sS,\sB(\sS))$. Thus, we have by $(*)$, $(**)$ %and define $D_n' = \bb{\omega: \phi(X_{t_n}(\omega))\in D_n}$
\beast
\E^{\Q}_{\phi(x)}\bb{f( \phi (X_{t}))\ind_D} & = & \E^{\Q}_{\phi(x)}\bb{\E^{\Q}_{\phi(x)} \bb{\left. f( \phi (X_{t}))\prod^n_{n=2}\ind_{D_i}\right|\phi(X_{t_1})}\ind_{D_1}} \\
& = & \E^{\Q}_{\phi(x)}\bb{\E^{\Q}_{\phi(x)} \bb{\left. \dots \E^{\Q}_{\phi(x)} \bb{\left.  f( \phi (X_{t}))\ind_{D_n}\right|\phi(X_{t_{n-1}})} \dots \ind_{D_2}\right|\phi(X_{t_1})}\ind_{D_1}} \\
& = & \E^{\Q}_{\phi(x)}\bb{\E^{\Q}_{\phi(x)} \bb{\left. \dots \E^{\Q}_{\phi(x)} \bb{\left.  f( \phi (X_{t}))\ind_{A_{t_n}}(\phi(X_{t_n}))\right|\phi(X_{t_{n-1}})} \dots \ind_{A_{t_2}}(\phi(X_{t_2}))\right|\phi(X_{t_1})}\ind_{A_{t_1}}(X_{t_1})} \\
& = &  \bb{Q_{0,t_1}\ind_{A_{t_1}}\bb{Q_{t_1,t_2}\ind_{A_{t_2}}\bb{\dots Q_{t_{n-1},t_n}f\ind_{A_{t_n}}\dots}}}\circ \phi(x) \\
& = & P_{0,t_1}\bb{\bb{\ind_{A_{t_1}}\circ \phi} P_{t_1,t_2} \bb{\bb{\ind_{A_{t_2}}\circ \phi } \dots P_{t_{n-1},t_n}\bb{(f\ind_{A_{t_n}})\circ \phi}\dots}} (x) \\
& = & P_{0,t_1}\bb{\bb{\ind_{\phi^{-1}(A_{t_1})}} P_{t_1,t_2} \bb{\bb{\ind_{\phi^{-1}(A_{t_2})} } \dots P_{t_{n-1},t_n}\bb{((f\circ \phi)\ind_{\phi^{-1}(A_{t_n})})}\dots}} (x) \\
& = & \E^{\pro}_{x}\bb{\E^{\pro}_{x} \bb{\left. \dots \E^{\pro}_{x} \bb{\left.  f\circ \phi (X_{t})\ind_{\phi^{-1}(A_{t_n})}(X_t)\right| X_{t_{n-1}}} \dots \ind_{\phi^{-1}(A_{t_2})}(X_{t_2})\right|X_{t_1}}\ind_{\phi^{-1}(A_{t_1})} (X_{t_1})} \\
& = & \E^{\pro}_x\bb{f\circ \phi (X_{t})\ind_D}.\qquad (\dag\dag)%\stackrel{(**)}{=}  
\eeast

%\E^{\pro}_x\bb{\E^{\pro}_{x}\bb{f\circ \phi(X_t)|X_s} \ind_D} = P\\& = & \E^{\pro}_x\bb{\E^{\Q}_{\phi(X_s)}\bb{f(\phi(X_t))} \ind_D} = \E^{\Q}_{\phi(x)}\bb{\E^{\Q}_{\phi(X_s)}\bb{f(\phi(X_t))} \ind_D}. 

Then by $(\dag)$ we have that 
\beast
\E^{\pro}_x\bb{f\circ \phi (X_t)\ind_D} & = & \E^{\pro}_x\bb{\E^{\pro}_x\bb{f\circ \phi(X_t))|X_s} \ind_D} = \E^{\pro}_x\bb{P_{s,t}(f\circ \phi)(X_s) \ind_D} \\
& = & \E^{\pro}_x\bb{Q_{s,t}f(\phi(X_s)) \ind_D} = \E^{\pro}_x\bb{((Q_{s,t}f) \circ \phi )(X_s) \ind_D}
\eeast

Since $Q_{s,t}f$ is bounded $\Q_{\phi(x)}$-a.s. and Borel measurable function (by Definition of $\Q$ and Lemma \ref{lem:markov_transition_kernel_on_function}), we can use $(\dag\dag)$ and get
\be
\E^{\pro}_x\bb{((Q_{s,t}f) \circ \phi )(X_s) \ind_D} = \E^{\Q}_{\phi(x)}\bb{Q_{s,t}f( \phi (X_{s}))\ind_D} =  \E^{\Q}_{\phi(x)}\bb{\E^{\Q}_{\phi(x)}\bb{f( \phi (X_{t}))|\phi(X_s)}\ind_D} 
\ee



%\be
%D : = {\omega:\bigcap_{u\in J,u\leq s}\bra{X_u\in A_u}:J\text{ is finite, }A_u\in \sB(\sS)} 
%\ee

 %Then by Proposition \ref{pro:inverse_image_preserves_set_operation}
%Since $\bra{X_u\in A_u}\in \sB(\sS)$, we can have that 
%\be
%\bigcap_{u\in J,u\leq s}\bra{X_u\in A_u} \in \sB(\sS).% \text{ is Borel measurable}
%\ee
%\be
%\bra{\omega:\bigcap_{u\in J,u\leq s}\bra{X_u(\omega)\in A_u}:J\text{ is finite, }A_u\in \sB(\sS)}
%\ee
%is a $\pi$-system generating $ \sF_s$. 

%Since $\ind_D$ is bounded and Borel measurable, 


Since $\sD_s$ is a $\pi$-system generating $\sF_s^{\phi(X)}$, we can apply Theorem \ref{thm:uniqueness_of_extension_measure} and have for any $C\in \sF_s^{\phi(X)}$,


\be
\E^{\Q}_{\phi(x)}\bb{f( \phi (X_{t}))\ind_C} = \E^{\Q}_{\phi(x)}\bb{\E^{\Q}_{\phi(x)}\bb{f(\phi(X_t))|\phi(X_s)} \ind_C}.
\ee
which implies that
\be
\E^{\Q}_{\phi(x)}\bb{\left.f( \phi (X_{t}))\right|\sF_s^{\phi(X)}} = \E^{\Q}_{\phi(x)}\bb{f(\phi(X_t))|\phi(X_s)}  \qquad \Q_{\phi(x)}\text{-a.s.}.
\ee



%Then letting $f=\ind_{\phi(X_t)\in A'}$ for $A' \in \sB(\sS')$, we have 
%\beast
%\Q_{\phi(x)}\bb{\phi (X_{t+s}))\in A'|\sF_s} & = & \E^{\Q}_{\phi(x)}\bb{\ind_{\bra{\phi (X_{t+s}))}}|\sF_s} \\
%& \stackrel{\Q_{\phi(x)}\text{-a.s.}}{=} & \E^{\Q}_{\phi(X_s)}\bb{\ind_{\bra{\phi(X_t)\in A'}}} = \Q_{\phi(X_s)}\bb{\phi(X_t)\in A'}.
%\eeast
%Then
%\be
%\bigcap_{u\in J,u\leq s}\bra{X_u\in A_u} \otimes \sT \text{ is a $\pi$-system generating $\sB(\sS)\otimes \sT$}.
%\ee

%Since $\E^{\pro}_x\bb{f(X_{t+s})|\sF_s}$ is $\sigma(X_s)$-measurable for bounded and Borel measurable $f:\sS\to \sS$, we have by $(\dag)$ for any $A'\in \sigma(\phi\circ X_s)$ and $\phi^{-1}(A') = A\in \sigma(X_s)$, 

%\beast
% \E^{\Q}_{\phi(x)} \bb{\E^{\Q}_{\phi(x)}\bb{f(\phi(X_{t+s}))|\sF_s}\ind_{\bra{\phi(X_s)\in A'}}} & = &  \E^{\Q}_{\phi(x)} \bb{f(\phi(X_{t+s}))\ind_{\bra{\phi(X_s)\in A'}}} = \E^{\pro}_x\bb{f\circ \phi (X_{t+s}) \ind_{\bra{X_s\in A}}} \\
% & = & \E^{\pro}_x\bb{\E^{\pro}_x\bb{f\circ \phi(X_{t+s})|\sF_s} \ind_{\bra{X_s\in A}}} = \E^{\pro}_x\bb{f\circ\phi (X_{t+s}) \ind_{\bra{X_s\in A}}} \\
% & = & \E^{\Q}_{\phi(x)}\bb{f(\phi(X_{t+s}))\ind_{\bra{\phi(X_s)\in A'}}}
%\eeast
%as we use $(*)$ repeatly. 

%\be
%\E^{\Q}_{\phi(x)} \bb{\E^{\Q}_{\phi(x)}\bb{f(\phi(X_{t+s}))|\sF_s}\ind_{\bra{\phi(X_s)\in A'}}} = \E^{\Q}_{\phi(x)} \bb{f(\phi(X_{t+s}))\ind_{\bra{\phi(X_s)\in A'}}} .
%\ee

%Finally, we want to show that $\E^{\Q}_{\phi(x)}\bb{f(\phi(X_{t+s}))|\sF_s}$ is $\sigma(\phi(X_s))$-measurable.\footnote{still need to show.}

This finishes the proof.
\een
\end{proof}

\section{Homogeneous Markov process}

\subsection{Homogeneous Markov process}

\begin{definition}[stationary Markov transition kernel, homogeneous Markov process]\label{def:homogeneous_markov_process_stationary_markov_transition_kernel}
Let $\bb{(X_t)_{t\geq 0},\pro_x}$ be a Markov process with Markov transition kernels $(P_{s,t})_{s,t\geq 0}$ with respect to $(\Omega,\sF,(\sF_t)_{t\geq 0})$. Now let $P_{s,t} = P_{t-s} := P_{0,t-s}$ for $t>s$, we can have for any $s,t\geq 0$ and any $A\in \sB(\sS)$,
\be
\pro_x\bb{X_{t+s}\in A|X_s} \stackrel{\pro_x\text{-a.s.}}{=} P_{s,s+t}(X_s,A) = P_{0,t}(X_s,A) = P_t(X_s,A) = \pro_{X_s}\bb{X_t\in A}.
\ee

Then for each $x\in \sS$, each Borel subset $A$ of $\sS$ ($A\in \sB(\sS)$), and each $s,t\geq 0$,
\be
P_{t+s}(x,A) = \int_{y\in \sS}P_t(y,A)P_s(x,dy),
\ee
then $(P_t)_{t\geq 0}$ is called stationary Markov transition kernel\index{stationary Markov transition kernel}. They can be rephrased in terms of equality of measures: for each $x\in \sS$, %\footnote{details needed.}
\be
P_{s+t}(x,dz) = \int_{y\in \sS}P_t(y,dz)P_s(x,dy).
\ee

Therefore, the corresponding transition density $p_t(x,y)$ satisfies
\be
P_t(x,A) = \int_A p_t(x,y)dy.% := dP_t(x,y) = P_t(x,dy).
\ee

Then we say $\bb{(X_t)_{t\geq 0},\pro_x}$ be a (time) homogeneous Markov process\index{homogeneous Markov process} with stationary Markov transition kernels $(P_t)_{t\geq 0}$ with 
\be
\E_x\bb{\left.f\bb{X_{t}}\right|\sF_s} \stackrel{\pro_x\text{-a.s.}}{=} \E_x\bb{\left.f\bb{X_{t}}\right| X_s} \stackrel{\pro_x\text{-a.s.}}{=} \E_{X_s}\bb{f(X_t)}
\ee
for any $x\in \sS$ and any bounded Borel measurable function $f$.
\end{definition}

\begin{lemma}\label{lem:stationary_markov_transition_kernel_on_function}
Let $\bb{(X_t)_{t\geq 0},\pro_x}$ be a (time) homogeneous Markov process and $(P_t)_{t\geq 0}$ are its stationary Markov transition kernels. Let $f$ be a Borel measurable and either non-negative (or bounded) function with respect to measure $P_t(x,\cdot)$ and suppose $(P_t)_{t\geq 0}$ are Markov transition kernels. We define 
\be
P_t f(x) := \int f(y)P_t(x,dy).
\ee

In paricular, $P_0 f(x) := f(x)$ as $P_0(x,dy)$ is actually a Dirac mass at $x$.\footnote{details needed.} Then $P_tf$ is non-negative (or bounded, repectively) and Borel measurable and
\be
P_t f(x) = \E_x\bb{f(X_t)},\qquad x\in \sS.
\ee
\end{lemma}

\begin{proof}[\bf Proof]
Non-negative (or bounded) conclusion is direct result from Theorem \ref{thm:image_measure_probability}.% and Remark \ref{rem:markov_transition_kernel}.
\beast
P_t f(x) := \int f(y)P_t(x,dy) & = & \int f(y)\pro_x(X_t \in dy) = \int f(y) \mu_{X_t}(dy) \\
& = & \int f(y) d\mu_{X_t}(y) = \int f(X_t(\omega)) d\pro_x(\omega) = \E_x\bsb{f(X_t)}
\eeast
where $\mu_{X_t} = \pro_x \circ X_t^{-1}$. 

For the measurability, we first take $f(x) = \ind_A$ where $A\in \sB(\sS)$, then
\be
P_t f(x) = \int_{y\in \sS} f(y) P_t(x,dy) = \int_{y\in A}P_t(x,dy) = P_t(x,A) = \pro_x(X_t\in A)
\ee 
which is Borel measurable by Definition \ref{def:markov_process}.(ii).

Then by linearity this holds for simple funcitons, and then using monotone convergence it holds for non-negative functions. Using linearity again, we have measurability conclusion holds for all bounded functions.%\footnote{see proof of Theorem \ref{thm:monotone_convergence_pointwise} for reference.} 
\end{proof}


\begin{proposition}[homogeneous Markov transition kernel is semigroup]\label{pro:semigroup_homogeneous_markov_process_transition}
Let $\bb{(X_t)_{t\geq 0},\pro_x}$ be a homogeneous Markov process and $\bb{P_t}_{t\geq 0}$ are its stationary Markov transition kernels. Then for any bounded Borel measurable funciton $f$, we have
\be
P_{t+s}f(x) = \int P_tf(y)P_s(x,dy).
\ee

The right-hand side is the same as $P_s(P_tf)(x)$, so we have
\be
P_{s+t}f(x) = P_s(P_tf)(x) = P_s P_t f(x),
\ee
i.e., the function $P_{s+t}f$ and $P_sP_tf$ are the same. This is known as the semigroup property\footnote{Note that Lemma \ref{lem:markov_transition_kernel_on_function} proves $P_tf$ is Borel measurable so we take $P_s(P_tf)$}. We can also write
\be
P_{s+t} = P_sP_t = P_tP_s.\qquad (*)
\ee

Then we can also say that $\bb{(X_t)_{t\geq 0},\pro_x}$ is a homogeneous Markov process on $(\sS,\sB(\sS))$ with transition semigroup\index{semigroup!transition} $(P_t)_{t\geq 0}$.
\end{proposition}

\begin{remark}
Operator satisfying $(*)$ is called a semigroup, and much studied in functional analysis\footnote{link needed.}.
\end{remark}

\begin{proof}[\bf Proof]
Multiplying equation $(*)$ in Definition \ref{def:markov_transition_kernel} by a bounded Borel measurable function $f(z)$ and integrating gives\footnote{see Lemma \ref{lem:markov_transition_kernel_on_function} to have $P_t f(x) := \int f(y)P_t(x,dy)$}
\beast
P_{s+t}f(x) & = & \int_{z\in \sS}f(z)P_{s+t}(x,dz) = \int_{z\in\sS} f(z) \int_{y\in \sS}P_t(y,dz)P_s(x,dy) \\
& = & \int_{y\in \sS} P_s(x,dy) \int_{z\in\sS} f(z) P_t(y,dz) = \int_{y\in \sS} P_tf(y) P_s(x,dy) = P_s(P_tf)(x)
\eeast
by Fubini theorem (Theorem \ref{thm:fubini}) as all the integrands are bounded. Note that $(P_t f)$ is bounded and thus we can treat it as a bounded Borel measurable function. Then we can switch $s$ and $t$ and get the result $(*)$.
\end{proof}

%\footnote{add $(X_t,\pro_x)$ in $P_t f(x) := \int f(y)P_t(x,dy)$.}
%\be
%\E_x\bb{f(X_{t+s})|\sF_s} = \E_{X_s}\bb{f(X_t)}\quad \pro_x\text{-a.s.}
%\ee


\subsection{Transformation of homogeneous Markov process}

\begin{theorem}[transformation of homogeneous Markov process]\label{thm:transformation_of_homogeneous_markov_process}
Let $(\sS,\sB(\sS))$ and $(\sS',\sB(\sS'))$ be measurable spaces where $\sS$ and $\sS'$ are seperable metric spaces, and suppose that the measurable function $\phi:(\sS,\sB(\sS))\to (\sS',\sB(\sS'))$ is (onto) surjective, i.e., for any $y\in \sS'$, there is $x\in \sS$ such that $\phi(x) = y$.

Let $\bb{(X_t)_{t\geq 0},\pro_x}$ be a homogeneous Markov process on $(\sS,\sB(\sS))$ with transition semigroup $(P_t)_{t\geq 0}$ with respect to $(\Omega,\sF,(\sF_t)_{t\geq 0})$. Let $(Q_t)_{t\geq 0}$ be a collection of probability kernels on $\sS'$ such that for all bounded and Borel measurable function $f:(\sS',\sB(\sS'))\to (\sS,\sB(\sS))$,
\be
P_t(f\circ \phi) = (Q_tf)\circ \phi.\qquad (*)
\ee

Then $\bb{(Y_t)_{t\geq 0}, \Q_y}$ $\bb{\bb{(\phi(X_t))_{t\geq 0}, \Q_y}}$ is a homogeneous Markov process on $(\sS',\sB(\sS'))$ with transition semigroup $(Q_t)_{t\geq 0}$ with respect to $(\Omega,\sF,(\sF_t^{\phi(X)})_{t\geq 0})$ where 
\be
\Q_{y}(Y_t\in A') = \Q_{\phi(x)}(\phi(X_t)\in A') := Q_t({\phi(x)},A'),\qquad \forall A'\in \sB(\sS').
\ee
\end{theorem}

\begin{remark}
Note that the filtration is $\sF_t^{\phi(X)}$ but not $\sF_t$ for $X$. This is reasonable as $\sigma(\phi(X_t))$ may not contain enough information as some of the information can be lost during the transformation.
\end{remark}

\begin{proof}[\bf Proof]
First, we prove that $(Q_t)_{t\geq 0}$ is Markov transition kernerls.

\ben
\item [(i)] By definition of $(Q_t)_{t\geq 0}$, it is obvious the condition (i) in Definition \ref{def:markov_transition_kernel} is satisfied.

\item [(ii)] We can see that $f\circ \phi$ is bounded and Borel ($\sB(\sS)$) measurable by Proposition \ref{pro:composition_measurable} and thus $P_t(f\circ \phi)$ is well-defined. Then by Lemma \ref{lem:markov_transition_kernel_on_function} and semigroup property (Proposition \ref{pro:semigroup_homogeneous_markov_process_transition}), we have for $t,s\geq 0$,
\be
(Q_{t+s}f)\circ \phi = P_{t+s}(f\circ \phi) = P_tP_s(f\circ \phi) = P_t ((Q_s f) \circ \phi) = (Q_tQ_sf)\circ \phi.
\ee

Note that $Q_t f$ is well-defined as $Q_t$ is probability kernel and $f$ is bounded and thus $Q_tf$ is bounded. Since $\phi$ is onto, we have\footnote{details needed.}
\be
Q_{t+s}f = Q_tQ_sf.
\ee

This is,
\be
\int_{z\in \sS'} f(z)Q_{t+s}(x,dz) = \int_{z\in \sS'} Q_sf(z)Q_{t}(x,dz) = \int_{z\in \sS'} \int_{y\in \sS'}f(y) Q_s(z,dy) Q_{t}(x,dz).
\ee

Then by letting $f(x) = \ind_{x\in B}$ where $B\in \sB(\sS')$, we can have 
\be
Q_{t+s}(x,B) = \int_{z\in \sS'} f(z)Q_{t+s}(x,dz)  = \int_{z\in \sS'} \int_{y\in \sS'}f(y) Q_s(z,dy) Q_t(x,dz) = \int_{z\in \sS'} Q_s(z,B) Q_t(x,dz)
\ee

This is actually (ii) in Definition \ref{def:markov_transition_kernel}.

\item [(iii)] Then we can define 
\be
\Q_{\phi(x)}(\phi(X_t)\in B) := Q_t({\phi(x)},B),\qquad \forall B\in \sB(\sS').
\ee
\een

Now we prove that $\phi(X_t)$ is a homogeneous Markov process.
\ben
\item [(i)] Since $\phi$ is Borel measurable we can have that $f\circ \phi$ is Borel measurable by Proposition \ref{pro:composition_measurable}. Thus, $\phi(X_{t})$ is $\sF_t^{X}$ (and $\sF_t$)-measurable.

\item [(ii)] We have $\Q_{\phi(x)}(\phi(X_t)\in B) = Q_t(\phi(x),B)$ is Borel measurable (wrt $\sB(\sS')$) from the assumption that $Q_t(y,B)$ is a probability kernel.
%\be
%(Q_t\ind_B)\circ \phi = P_t(\ind_B\circ \phi) = P_t\ind_{\bra{\phi(x)\in B}}
%\ee by Lemma \ref{lem:markov_transition_kernel_on_function}.

\item [(iii)] Now we take $0=t_0 \leq t_1 \leq \dots \leq t_n$, $s_k = t_k - t_{k-1}$ and $f_k \in \sB(\sS')$. Then by definition of Markov process and tower property (Proposition \ref{pro:conditional_expectation_tower_independence}.(i)),
\beast% (Q_{t+s}\prodf) \circ (\phi(x)) = \E^\Q_{\phi(x)}\bb{\prod_n_{k=1}f_k(\phi(X_{t_k}))} = = P_{t+s}(f\circ \phi(x)) = P_tP_s(f\circ \phi(x)) = P_t\bb{(Q_sf)\circ \phi(x)} =  (Q_tQ_sf)\circ \phi(x)
\E^{\pro}_{x}\bsb{\prod^n_{k=1}f_k(\phi(X_{t_k}))} & = & \E^{\pro}_{x}\bsb{\E^{\pro}_{x}\bb{\left.\prod^n_{k=1}f_k(\phi(X_{t_k}))\right|\sF_{t_1} }} = \dots \\
& = &  \E^{\pro}_{x}\bsb{f_1(\phi(X_{t_1})) \E^{\pro}_{x}\bb{\left.\dots f_{n-1}(\phi(X_{t_{n-1}})) \E^{\pro}_{x}\bb{ \left. f_{n}(\phi(X_{t_n}))\right|\sF_{t_{n-1} }} \dots \right|\sF_{t_1} }} \\
& = &  \E^{\pro}_{x}\bsb{f_1(\phi(X_{t_1})) \E^{\pro}_{x}\bb{\left.\dots f_{n-2}(\phi(X_{t_{n-2}})) \E^{\pro}_{x}\bb{ \left. f_{n-1}(\phi(X_{t_{n-1}})) \E^{\pro}_{X_{t_{n-1}}}\bb{f_n(\phi(X_{s_n}))}\right|\sF_{t_{n-2} } } \dots \right|\sF_{t_1} }} \\
& = &  \E^{\pro}_{x}\bsb{f_1(\phi(X_{s_1})) \E^{\pro}_{x}\bb{\left.\dots f_{n-2}(\phi(X_{t_{n-2}})) \E^{\pro}_{X_{t_{n-2}}}\bb{ f_{n-1}(\phi(X_{s_{n-1}})) \E^{\pro}_{X_{s_{n-1}}}\bb{f_n(\phi(X_{s_n}))} } \dots \right|\sF_{t_1} }} \\
& = & \E^{\pro}_{x}\bsb{f_1(\phi(X_{s_1})) \E^{\pro}_{X_{s_1}}\bb{\dots f_{n-2}(\phi(X_{t_{n-2}})) \E^{\pro}_{X_{t_{n-2}}}\bb{ f_{n-1}(\phi(X_{s_{n-1}})) \E^{\pro}_{X_{s_{n-1}}}\bb{f_n(\phi(X_{s_n}))} }\dots }} \\
& = & P_{s_1}\bb{\bb{f_1\circ \phi} P_{s_2} \bb{\bb{f_2\circ \phi } \dots P_{s_n}\bb{f_n\circ \phi}\dots}} (x)
\eeast

Then by  $(*)$, we have that 
\be
P_{s_1}\bb{\bb{f_1\circ \phi} P_{s_2} \bb{\bb{f_2\circ \phi } \dots P_{s_n}\bb{f\circ \phi}\dots}} (x) = \bb{Q_{s_1}f_1\bb{Q_{s_2}f_2\bb{\dots Q_{s_n}f_{n}\dots}}}\circ \phi(x).\qquad (\dag)
\ee

Now we can pick any $B\in \sigma(\phi\circ X_s)$, we know that we can find $A'\in \sB(\sS')$ such that $B = (\phi\circ X_s)^{-1}(A')$ as $\sigma(X_s)$ is smallest $\sigma$-algebra with respect to $\phi\circ X_s$ (see Proposition \ref{pro:inverse_image_preserves_set_operation_measure}). %Definition \ref{def:sigma_algebra_generated_by_measurable_function} and
Accordingly, we can find $A\in \sB(\sS)$ such that $A = \phi^{-1}(A')$ as $\phi$ is Borel measurable.


%For any $B\in \sigma(X_s)$, 

%Then we can have %find a subset $A\in \sB(\sS)$ such that $X_s^{-1}(A) = B$  

Thus, we have (as $f\bb{\phi(X_{t+s})}\ind_{\bra{\phi(X_s)\in A'}}$ is bounded and Borel measurable) by $(\dag)$,
\beast
\E^{\Q}_{\phi(x)}\bb{f\bb{\phi(X_{t+s})}\ind_B} & = & \E^{\Q}_{\phi(x)}\bb{f\bb{\phi(X_{t+s})}\ind_{\bra{\phi(X_s)\in A'}}} = \bb{Q \bb{f\bb{\phi(X_{t+s})}\ind_{\bra{\phi(X_s)\in A'}}}} \circ \phi(x) \\%
& = & (Q_{t+s}\wt{f})\circ \phi(x) = P_{t+s}\bb{\wt{f}\circ \phi}(x) = P_s\bb{(\ind_{A'} \circ \phi)P_t(f \circ \phi)}(x) \\
& = & \bb{Q_s \ind_{A'}(Q_t f)}\circ \phi(x) = \E^{\Q}_{\phi(x)}\bb{\ind_{\bra{\phi(X_s)\in A'}}\E_{\phi(X_s)}^{\Q}\bb{f\bb{\phi(X_t)}}} \\
& = &  \E^{\Q}_{\phi(x)}\bb{\E_{\phi(X_s)}^{\Q}\bb{f\bb{\phi(X_t)}}\ind_B}.
\eeast

Thus, we have 
\beast
\E^{\Q}_{\phi(x)}\bb{f\bb{\phi(X_{t+s})}|\phi(X_s)} \stackrel{\text{$\Q_{\phi(x)}$-a.s.}}{=}  \E_{\phi(X_s)}^{\Q}\bb{f\bb{\phi(X_t)}} .
\eeast

Since $(X_t,\pro_x)$ is a Markov process, we have for any bounded and Borel measurable $f$,
\be
\E^{\pro}_x\bb{f(X_{t+s})|\sF_s} =  \E^{\pro}_x\bb{f(X_{t+s})|X_s} = \E^{\pro}_{X_s}\bb{f(X_t)} \qquad \pro_x\text{-a.s.}
\ee

Thus, for any $C\in \sF_s$,
\be
\E^{\pro}_x\bb{f(X_{t+s})\ind_C} = \E^{\pro}_x\bb{\E^{\pro}_x\bb{f(X_{t+s})|X_s} \ind_C} = \E^{\pro}_x\bb{\E^{\pro}_{X_s}\bb{f(X_t)} \ind_C}.
\ee

Then for any $f:(\sS',\sB(\sS'))\to (\sS,\sB(\sS))f$ and $\phi:(\sS,\sB(\sS)) \to (\sS',\sB(\sS'))$, $f\circ \phi$ is bounded and Borel measurable. Then we have
\be
\E^{\pro}_x\bb{f\circ \phi (X_{t+s})\ind_C} = \E^{\pro}_x\bb{\E^{\pro}_{X_s}\bb{f\circ \phi(X_t)} \ind_C} = \E^{\pro}_x\bb{\E^{\Q}_{\phi(X_s)}\bb{f(\phi(X_t))} \ind_C}.
\ee

Note that $\E^{\Q}_{\phi(X_s)}\bb{f(\phi(X_t))} $ is bounded and Borel function as well by Lemma \ref{lem:markov_transition_kernel_on_function}. We know from Proposition \ref{pro:process_law_is_uniquely_determined_by_its_finite_dimensional_marginal_distribution} that 
\be
\sD_s := \bra{\omega:\bigcap_{u\in J,u\leq s}\bra{\phi(X_u(\omega))\in A_u}:J\text{ is finite, }A_u\in \sB(\sS')}
\ee
is a $\pi$-system generating $ \sF_s^{\phi(X)}$. Then $\forall D\in \sD_s\subseteq \sF_s^{\phi(X)}$ and we can have that % that the function (for particular $J$)
%\be
%\ind_D = \ind_{\bra{\omega:\bigcap_{u\in J,u\leq s}\bra{\phi(X_u)\in A_u}:J\text{ is finite, }A_u\in \sB(\sS')}} = \prod_{u\in J,u\leq s} \ind_{\bra{\phi(X_u)\in A_u}} = \prod^n_{i=1}\ind_{D_i}
%\ee
%is bounded and Borel measurable as $\ind_{\bra{X_u\in \phi^{-1}(A_u)}}$ is Borel measurable on $(\sS,\sB(\sS))$. Then we have that 
\be
\E^{\pro}_x\bb{f\circ \phi (X_{t+s})\ind_D} = \E^{\pro}_x\bb{\E^{\Q}_{\phi(X_s)}\bb{f(\phi(X_t))} \ind_D}.
\ee

%\be
%D : = {\omega:\bigcap_{u\in J,u\leq s}\bra{X_u\in A_u}:J\text{ is finite, }A_u\in \sB(\sS)} 
%\ee

 %Then by Proposition \ref{pro:inverse_image_preserves_set_operation}
%Since $\bra{X_u\in A_u}\in \sB(\sS)$, we can have that 
%\be
%\bigcap_{u\in J,u\leq s}\bra{X_u\in A_u} \in \sB(\sS).% \text{ is Borel measurable}
%\ee
%\be
%\bra{\omega:\bigcap_{u\in J,u\leq s}\bra{X_u(\omega)\in A_u}:J\text{ is finite, }A_u\in \sB(\sS)}
%\ee
%is a $\pi$-system generating $ \sF_s$. 

Since $\ind_D$ is bounded and Borel measurable, we have by $(*)$
\beast
\E^{\Q}_{\phi(x)}\bb{f( \phi (X_{t+s}))\ind_D} & = & \E^{\pro}_x\bb{f\circ \phi (X_{t+s})\ind_D} \\
& = & \E^{\pro}_x\bb{\E^{\Q}_{\phi(X_s)}\bb{f(\phi(X_t))} \ind_D} = \E^{\Q}_{\phi(x)}\bb{\E^{\Q}_{\phi(X_s)}\bb{f(\phi(X_t))} \ind_D}. 
\eeast

Since $\sD_s$ is a $\pi$-system generating $\sF_s$, we can apply Theorem \ref{thm:uniqueness_of_extension_measure} and have for any $C\in \sF_s$,
\be
\E^{\Q}_{\phi(x)}\bb{f( \phi (X_{t+s}))\ind_C} = \E^{\Q}_{\phi(x)}\bb{\E^{\Q}_{\phi(X_s)}\bb{f(\phi(X_t))} \ind_C}.
\ee
which implies that
\be
\E^{\Q}_{\phi(x)}\bb{\left.f( \phi (X_{t+s}))\right|\sF_s} = \E^{\Q}_{\phi(X_s)}\bb{f(\phi(X_t))}  \qquad \Q_{\phi(x)}\text{-a.s.}.
\ee
\een
\end{proof}

\section{Strong Markov process}

\section{Brownian Motions}

\subsection{Brownian motion is strong Markov process}

\begin{definition}[transition density and transition kernel of $d$-dimensional Euclidean Brownian motion]\label{def:transition density_d_dimensional_brownian_motion}
The transition density of $d$-dimensional Euclidean Brownian motion $(B_t)_{t\geq 0}$ on $\R^d$ is defined by
\be
p_t(x,y) := \bb{2\pi t}^{-d/2}\exp\bb{-\frac{\dabs{x-y}^2}{2t}}.
\ee
where $\dabs{\cdot}$ is Euclidean norm. This is due to the Gaussian distribution of $B_t$.

Then $P_t(x,y)$ is defined to be the transition kernel of $d$-dimensional Brownian motion. That can also be expressed by for $x\in \R^d$
\be
P_t(x,A) = \pro_x(X_t\in A) = \pro\bb{B_t+x\in A} = \int_A p_t(x,y) dt = (2\pi t)^{-d/2}\int_A \exp\bb{-\frac{\dabs{x-y}^2}{2t}} dy.
\ee

%Since Brownian motion is a Markov process, w

To check this is actually is Markov transition kernel, we need the following the steps:
\ben
\item [(i)] It is obvious that 
\be
P_t(x,\sS) = (2\pi t)^{-d/2}\int_\sS \exp\bb{-\frac{\dabs{x-y}^2}{2t}} dy.
\ee
as the integrand is the density of $d$-dimensional Gaussian random variable.

\item [(ii)] We also have
\beast
\int_{y\in \sS} P_t(y,A)P_s(x,dy) & = & \int_{y\in \sS}\bb{ (2\pi t)^{-d/2}\int_{z\in A}  \exp\bb{-\frac{\dabs{y-z}^2}{2t}} dz } P_s(x,dy) \\
& = & (2\pi t)^{-d/2} (2\pi s)^{-d/2} \int_{y\in \sS} \int_{z\in A}  \exp\bb{-\frac{s\dabs{y-z}^2 + t\dabs{x-y}^2}{2ts}}    dz dy\\
& = & (2\pi t)^{-d/2} (2\pi s)^{-d/2} \int_{z\in A} \exp\bb{-\frac{\dabs{x-z}^2}{2(t+s)}}    \int_{y\in \sS} \exp\bb{-\frac{(s+t)\dabs{y-\frac{tx}{t+s} - \frac{sz}{t+s}}^2}{2ts}}  dy  dz\\
& = & (2\pi (t+s)^{-d/2} \int_{z\in A} \exp\bb{-\frac{\dabs{x-z}^2}{2(t+s)}}  dz = P_{t+s}(x,A).
\eeast

\item [(iii)] Then we can define probability measure
\be
\pro_x(B_t \in A) := P_t(x,A).
\ee
\een
\end{definition}

\begin{theorem}[$d$-dimensional Brownian motion on $\R^d$ is Markov process]\label{thm:d_dimensional_brownian_motion_is_markov_process}
Let $(B_t)_{t\geq 0}$ be $d$-dimensional Brownian motion on $\R^d$. Then $\bb{B_t,\pro_x}$ is a Markov process where $\pro_x$ is defined by
\be
\pro_x\bb{B_t\in A}  = \pro\bb{x+\wt{B}_t\in A} = \int_A p_t(x,y)dy,\qquad \forall A\in \sB(\R^d)
\ee
where $\wt{B}_t$ is a standard Brownian motion and
\be
p_t(x,y) = (2\pi t)^{-d/2} \exp\bb{-\frac{\dabs{x-y}^2}{2t}}.
\ee
\end{theorem}

\begin{proof}[\bf Proof]
We check the definition of Markov processes with following steps.
\ben
\item [(i)] Since Brownian motion is adapted (by definition), $B_t$ is $\sF_t$-measurable.

\item [(ii)] %We know that for $A\in \sB(\sS)$,
%\be
%\pro_x(X_t\in A) = P_t(x,A) = \int_{y\in S}\ind_{y\in A} P_t(x,dy) = P_t\ind_{A}(x)
%\ee
%is Borel measruable as $\ind_{x\in A}$ is bounded and Borel measurable by Lemma \ref{lem:markov_transition_kernel_on_function}.

Since the transition kernel $P_t(x,A)$ is a continuous function\footnote{This is because the continuity of transition density with respect to $x$ and limit can be taken into the integral.} of $x$, we have for any subset $B\in \sB(\R^d)$ which is open, $P_t^{-1}(x,A)$ is also open and thus in $\sB(\sS)$. This is due to the continuity of the function on metrix space (see Theorem \ref{thm:metric continuous open}).  Therefore, $\pro_x(X_t\in A) = P_t(x,A)$ is $\sB(\R^d)$-measurable.

\item [(iii)] Now we want to show for any bounded and Borel measurable function $f$,%\footnote{Instead, we can apply the statement that for any bounded and Borel measurable function $f$, $\E_x\bb{f(B_{t+s})|\sF_s} = \E_{B_s}\bb{f(B_t)}$ $\pro_x$-a.s.. by taking $f(x) = \ind_{\bra{x\in A}}$.}
%\be
%\pro_x\bb{B_{t+s}\in A|\sF_s} = \pro_{B_s}(B_t\in A)\qquad \pro_x\text{-a.s..}
%\ee
%We will show for $\pro_x$-Brownian motion $(B_t)_{t\geq 0}$
%\be
%\pro_x(B_{t+s}\in A|\sF_s) = \pro_x(B_{t+s}\in A|B_s)\qquad \pro_x\text{-a.s..}
%\ee
%we can show that 
\be
\E_x\bb{f(B_{t+s})|\sF_s} = \E_x\bb{f(B_{t+s})|B_s}\qquad \pro_x\text{-a.s..}\qquad (*)
\ee

%Then we will have the result by letting $f(x) = \ind_{\bra{x\in A}}$.

First, we prove this for $f(x) = e^{iux}$ for $u\in \R^d$. Since $B_s$ is $\sF_s$-measurable and $B_{t+s}-B_s$ is independent of $B_s$, we have 
\beast
\E_x\bb{f(B_{t+s})|\sF_s} & = &  \E_x\bb{e^{iuB_{t+s}}|\sF_s} = \E_x\bb{e^{iu(B_{t+s}-B_s)}e^{iuB_s}|\sF_s} \stackrel{\pro_x\text{-a.s.}}{=} e^{iuB_s} \E_x\bb{e^{iu(B_{t+s}-B_s)}} \\
& \stackrel{\pro_x\text{-a.s.}}{=} & e^{iuB_s} \E_x\bb{e^{iu(B_{t+s}-B_s)}|B_s} \stackrel{\pro_x\text{-a.s.}}{=} \E_x\bb{f(B_{t+s})|B_s} . %e^{iuB_s} \E_0\bb{e^{iuB_t}} = e^{iuB_s} e^{-u^2 s/2}.
\eeast
by Proposition \ref{pro:conditional_expectation_tower_independence}.(iv) and (v) and Proposition \ref{pro:mgf_gaussian}. Now suppose that $f\in C^\infty$ with compact support and let $\wh{f}$ be the Fourier transform of $f$, then we can replace $u$ by $-u$ and get
\be
\E_x\bb{f(B_{t+s})|\sF_s} = \E_x\bb{\left.(2\pi)^{-d}\int e^{-iuB_{t+s}}\wh{f}(u)du \right|\sF_s}
\ee

Thus, for any $B\in \sF_s$, we have that 
\beast
\E_x\bb{(2\pi)^{-d}\int e^{-iuB_{t+s}}\wh{f}(u)du \ind_B} & = & (2\pi)^{-d}\int \E_x\bb{e^{-iuB_{t+s}}\ind_B}\wh{f}(u)du \\
& = & (2\pi)^{-d}\int \E_x\bb{\ind_B \E_x\bb{e^{-iuB_{t+s}}|\sF_s}}\wh{f}(u)du \\
& = & (2\pi)^{-d}\int \E_x\bb{\ind_B \E_x\bb{e^{-iuB_{t+s}}|B_s}}\wh{f}(u)du \qquad (\text{for function }f(x) = e^{-iux})\\ 
& = & \E_x\bb{(2\pi)^{-d}\int \E_x\bb{e^{-iuB_{t+s}}|B_s} \wh{f}(u)du \ind_B }.%\\& = & (2\pi)^{-1}\int \E_x\bb{e^{-iuB_s}\ind_B e^{-u^2t/2}}\wh{f}(u)du .
\eeast

For any $C\in \sigma(B_s)$, we have
\beast
\E_x\bb{\int e^{-iuB_{t+s}}\wh{f}(u)du  \ind_C} & = & \int \E_x\bb{ e^{-iuB_{t+s}} \ind_C }\wh{f}(u)du = \int \E_x\bb{ \E_x\bb{ e^{-iuB_{t+s}} \ind_C | B_s}}\wh{f}(u)du \\
& = & \E_x\bb{\int \E_x\bb{e^{-iuB_{t+s}}| B_s} \wh{f}(u)du \ind_C }
\eeast

Thus, by definition of conditional expectation,
\be
\E_x\bb{\int e^{-iuB_{t+s}}\wh{f}(u)du |B_s} = \int \E_x\bb{e^{-iuB_{t+s}}|B_s} \wh{f}(u)du \qquad \pro_x\text{-a.s..}
\ee

Then plug this back to the previous equation,
\beast
\E_x\bb{(2\pi)^{-d}\int e^{-iuB_{t+s}}\wh{f}(u)du \ind_B} & = & \E_x\bb{\E_x\bb{\left.(2\pi)^{-d}\int e^{-iuB_{t+s}}\wh{f}(u)du\right| B_s} \ind_B}
\eeast

Therefore,
\be
\E_x\bb{\left.(2\pi)^{-d}\int e^{-iuB_{t+s}}\wh{f}(u)du \right|\sF_s} = \E_x\bb{\left.(2\pi)^{-d}\int e^{-iuB_{t+s}}\wh{f}(u)du\right| B_s}\qquad \pro_x\text{-a.s..}
\ee

Hence,
\be
\E_x\bb{f(B_{t+s})|\sF_s} = \E_x\bb{f(B_{t+s})| B_s} \qquad \pro_x\text{-a.s..}
\ee

Note that we used Fubini theorem (Theorem \ref{thm:fubini}) several times to interchange expectation and integration; this is justified because $f\in C^\infty$ with compact support implies $\wh{f}$ is in the Schwartz class\footnote{theorem needed. see Section B.2 in \cite{Bass_2011}}. This proves that ($*$) for $f\in C^\infty$ with compact support. Then a limit argument\footnote{original from Richard F. Bass, Stochastic Processes, theorem needed.} gives that ($*$) holds for all bounded and Borel measurable $f$.% by monotone class theorem (Theorem \ref{thm:monotone_class}). %It can be done by checking the $f$ in $\pi$-system of Borel $\sigma$-algebra.



%Since $C^\infty$ functions with compact support are dense in $\sL^1(E,\sE,\mu)$\footnote{theorem needed.}, for any non-negative bounded $f$ we can find a sequence $f_n$ of $C^\infty$ with compact support such that $f_n \ua f$. Then monotone convergence theorem (Theorem \ref{thm:monotone_convergence_probability}) we have that ($*$) for $f$. Also, we can have that for any Borel measurable subset $A\in \sE$ and any $B\in \sF_s$, 
%\beast
%\E_x\bb{f(X_{t+s})\ind_B} & = & \E_x\bb{\ind_A\bb{X_{t+s}}\ind_B} = \E\bb{\ind_A\bb{x+\wt{B}_{t+s}}\ind_B} \\
%& = & \E\bb{\E\bb{\left.\ind_A\bb{x + \wt{B}_{t+s}}\right|\sF_s}\ind_B} = \E\bb{\E\bb{\left.\ind_A\bb{X_s + \wt{B}_{t}}\right|\sF_s}\ind_B} \\
%& = & \E\bb{\E\bb{\ind_A\bb{X_s + \wt{B}_{t}}}\ind_B} = \E_x\bb{\E_{X_s}\bb{\ind_A\bb{X_{t}}}\ind_B}
%\eeast
%which implies that
%\be
%\E_x\bb{\left.\ind_A\bb{X_{t+s}}\right| \sF_s} \stackrel{\pro_x\text{-a.s.}}{=}\E_{X_s}%\bb{\ind_A\bb{X_{t}}}.
%\ee


Finally, we want to prove 
\be
\E_x\bb{f\bb{B_{t+s}}| B_s} \stackrel{\pro_x\text{-a.s.}}{=} \E_{B_s}\bb{f(B_t)}= P_t f(B_s) = \int_A f(y)P_t(B_s,dy) = \int_A f(y)p_t(B_s,y)dy,
\ee
we can use the similar argument of Proposition \ref{pro:conditional_probability_density_function} to get
\be
\E_x\bb{f\bb{B_{t+s}}|B_s} \stackrel{\pro_x\text{-a.s.}}{=} \int_{\sS} f(z) f_{B_{t+s}|B_s}(z|B_s) dz =\int_{\sS}f(z)f_{B_{t+s}|B_s}(z|B_s)  dz %\frac{f_{B_{t+s},B_s}(z,B_s)}{f_{B_s}(B_s)}
\ee
where $f_{B_{t+s},B_s}(x,y)$ is joint density function of $B_{t+s}$ and $B_s$ and $f_{B_s}(y)$ is the density function of $B_s$. Since $B_{t+s}-B_s$ is independent of $B_s-x$ (by definition of Brownian motion), we have that the covariance matrix of $B_{t+s} - B_s$ and $B_s$ is
\be
\Sigma = \bepm
tI_{d} & 0 \\ 0 & s I_{d} 
\eepm 
\quad \ra \quad \abs{\Sigma} = (ts)^d,\quad 
\Sigma^{-1} = \bepm
t^{-1}I_{d} & 0 \\ 0 & s^{-1} I_{d}
\eepm.
\ee

Therefore, by Definition \ref{def:non_degenerate_multivariate_gaussian} the joint density of $B_{t+s}$ and $B_s$ is 
\beast
f_{B_s,B_{t+s}}(y,z) & = & (2\pi)^{-d/2}\abs{\Sigma}^{-1/2} \exp\bb{-\frac{1}{2}((z-y)^T, (y-x)^T){\Sigma}^{-1}\bepm z-y\\ y-x\eepm } \\
& = & (2\pi ts)^{-d/2} \exp\bb{-\frac{\dabs{z-y}^2}{2t} - \frac{\dabs{y-x}^2}{2s}}.
\eeast

Also, we have 
\be
f_{B_s}(y) = (2\pi s)^{-d/2} \exp\bb{- \frac{\dabs{y-x}^2}{2s}} .
\ee

Therefore,
\beast
f_{B_{t+s}|B_s}(z|y) = (2\pi t)^{-d/2} \exp\bb{-\frac{\dabs{z-y}^2}{2t}}\ \ra\ f_{B_{t+s}|B_s}(z|B_s) = (2\pi t)^{-d/2} \exp\bb{-\frac{\dabs{z-B_s}^2}{2t}} = p_t(B_s,z)
\eeast
as required.%Thus,
%\be
%\pro_x(B_{t+s}\in A|B_s) \stackrel{\pro_x\text{-a.s.}}{=} \int_{A}f_{B_{t+s}|B_s}(z|B_s) dz = \int_{A}p_t(z,B_s) dz = P_t(B_s,A)
%\ee
%by definition of transition density.
\een
\end{proof}

\subsection{Transformation of Brownian motion}

\begin{corollary}[reflected Brownian motion is Markov process]\label{cor:reflected_bm_is_markov_process}
Let $(B_t)_{t\geq 0}$ be 1-dimensional Brownian motion on $\R$ with transition density
\be
p_t(x,y) = (2\pi t)^{-1/2} \exp\bb{-\frac{\abs{x-y}^2}{2t}},\qquad x,y\in \R^{++}.
\ee

Then $\bb{\abs{B_t},\Q_x}$ is a Markov process for $A\in \sB(\R^{++}) = \sB([0,\infty))$,
\be
\Q_x(\abs{B_t}\in A) = \int_{y\in A} Q_t(x,dy) = \int_{y\in A} q_t(x,y)dy
\ee
where for $x,y\in \R^{++}$,
\be
q_t(x,y) = p_t(x,y) + p_t(x,-y) = \sqrt{\frac{2}{\pi t}}\exp\bb{-\frac{x^2 +y^2}{2t}}\cosh\bb{\frac{xy}{t}}.
\ee
\end{corollary}



\begin{proof}[\bf Proof]
We can apply Theorem \ref{thm:transformation_of_markov_process} by letting $\phi(x) = \abs{x}$. The only thing left is to check $(*)$ in Theorem \ref{thm:transformation_of_markov_process}\footnote{Note that $(\dag)$ has been proved in Theorem \ref{thm:d_dimensional_brownian_motion_is_markov_process}}. Note that the filtration $\sF_s$ might be bigger than the natural filtration $\sF_s^B$ and thus the conclude in Theorem \ref{thm:transformation_of_markov_process} still holds.

For any bounded and Borel function $f:\R^{++}\to \R$, we have that
\beast
P_t(f\circ \phi)(x) & = & \int f\circ \phi(y) P_t(x,dy) = \int_{\R} f(\abs{y}) (2\pi t)^{-1/2} \exp\bb{-\frac{\abs{x-y}^2}{2t}} dy\\
& = & \int_{\R^{++}} f(y) (2\pi t)^{-1/2} \exp\bb{-\frac{\abs{\abs{x}-y}^2}{2t}} dy + \int_{\R^{++}} f(y)  (2\pi t)^{-1/2} \exp\bb{-\frac{\abs{\abs{x}+y}^2}{2t}} dy \\
 & = & \int_{\R^{++}} f(y) (2\pi t)^{-1/2} \bb{\exp\bb{-\frac{\abs{\abs{x}-y}^2}{2t}}  + \exp\bb{-\frac{\abs{\abs{x}+y}^2}{2t}} } dy \\
 & = & \int_{\R^{++}} f(y) q_t(\abs{x},y) dy = \int_{\R{++}} f(y)Q_t(\abs{x},dy) = (Q_t f)(\abs{x}) = (Q_t f)\circ \phi(x)
\eeast
as required.
\end{proof}

\subsection{Brownian motion is a strong Markov process}

$(S_t,S_t-B_t)$ is continuous strong Markov process in $\bb{\R^{++}}^2$


\begin{proposition}\label{pro:markov_transition_density_semigroup}
Let $(P_{s,t})_{s,t\geq 0}$ be Markov transition kernel on $(\sS,\sB(\sS))$ with Markov transition density $p_{s,t}(x,y)$ such that
\be
P_{s,t}(x,A) = \int_A p_{s,t}(x,y)dy
\ee
for any $s<t$ and $x,y\in \sS$. Then for any $s<t<u$ and $x,y,z\in \sS$,
\be
p_{s,u}(x,z) = \int_{\sS} p_{s,t}(x,y)p_{t,u}(y,z)dy.
\ee
\end{proposition}

\begin{proof}[\bf Proof]
By Definition \ref{def:markov_transition_density}, \ref{def:markov_transition_kernel}.(iii),
\beast
\int_A p_{s,u}(x,z) dz & = & P_{s,u}(x,A) = \int_{\sS} P_{s,t}(x,dy)P_{t,u}(y,A) = \int_{\sS} P_{s,t}(x,dy)\int_A p_{t,u}(y,z)dz \\
& = & \int_{\sS} p_{s,t}(x,y)dy\int_A p_{t,u}(y,z)dz = \int_A \int_{\sS} p_{s,t}(x,y) p_{t,u}(y,z)dy dz = \int_A \bb{\int_{\sS} p_{s,t}(x,y) p_{t,u}(y,z)dy} dz
\eeast
as required.
\end{proof}


\section{SDE and PDE}

\subsection{Infinitesimal generator}

\begin{definition}[infinitesimal generator\index{infinitesimal generator}]
For $(X_t)_{t\geq 0}$ on $(\sS,\sB(\sS))$ be a Markov process with Markov transition kernel $P_{s,t}(x,A)$, the infinitesimal generator of $X$ is defined by for $X_s = x$
\be
\sA f(x) := \lim_{t\da s} \frac 1{t-s} \E_{x}\bb{f(X_t)-f(x)} =\lim_{t\da s} \frac 1{t-s} \bb{P_{s,t}f(x)-P_{s,s}f(x)},\qquad  x\in \sS%\stackrel{\pro_x\text{-a.s.}}{=}
\ee
for suitable function $f$.

For time-homogeneous Markov process $X$ with semigroup $P_t(x,A)$, the infinitesimal generator of $X$ is
\be
\sA f(x) := \lim_{t\da 0} \frac 1t \E_x\bb{f(X_t)-f(x)} = \lim_{t\da 0} \frac 1t \bb{P_{t}f(x)-f(x)},\qquad  x\in \sS
\ee
\end{definition}

\begin{example}
For the SDE
\be
dX_t = \mu(t,X_t)dt + \sigma(t,X_t)dB_t,
\ee
we can pick $f\in C^2(\R)$ such that by It\^o lemma
\beast
df(X_t) & = & \fp{}{x}f(X_t) dX_t + \frac 12 \fpp{}{x}f(X_t)(dX_t)^2 \\
& = & \bb{\mu(t,X_t)\fp{}{x}f(X_t) +  \frac 12\sigma^2(t,X_t) \fpp{}{x}f(X_t)} dt + \sigma(t,X_t) \fp{}{x}f(X_t)dB_t
\eeast

Therefore,
\be
f(X_t) - f(x) = \int^t_0 \bb{\mu(t,X_t)\fp{}{x}f(X_t) +  \frac 12\sigma^2(t,X_t) \fpp{}{x}f(X_t)} dt + \int^t_0\sigma(t,X_t) \fp{}{x}f(X_t)dB_t
\ee
and
\beast
\E_x\bb{f(X_t) - f(x)} & = & \E \bb{\int^t_0 \bb{\mu(s,X_s)\fp{}{x}f(X_s) +  \frac 12\sigma^2(s,X_s) \fpp{}{x}f(X_s)} ds}%& = &  \int^t_0 \E \bb{\mu(s,X_s)\fp{}{x}f(X_s) +  \frac 12\sigma^2(s,X_s) \fpp{}{x}f(X_s)} ds
\eeast

Thus,
\beast
\sA f(x) & = & \E \bb{\left.\mu(t,X_t)\fp{}{x}f(X_t) +  \frac 12\sigma^2(t,X_t) \fpp{}{x}f(X_t)\right|_{t=s,X_t=x}}\\
& = & \E \bb{\mu(s,x)\fp{}{x}f(x) +  \frac 12\sigma^2(s,x) \fpp{}{x}f(x)} = \mu(s,x)\fp{}{x}f(x) +  \frac 12\sigma^2(s,x) \fpp{}{x}f(x).
\eeast
\end{example}

\begin{example}
From the above example, we have the following results.
\ben
\item [(i)] The infinitesimal generator of Brownian motion
\be
dX_t = \sigma dB_t \qquad \text{is }\quad \sA = \frac 12 \sigma^2\fpp{}{x}.
\ee

\item [(ii)] The infinitesimal generator of OU process 
\be
dX_t = \theta(\mu - X_t)dt + \sigma dB_t \qquad \text{is }\quad \sA = \theta(\mu-x)\fp{}{x} + \frac 12 \sigma^2\fpp{}{x}.
\ee

\item [(iii)] The infinitesimal generator of CIR process
\be
dX_t = \theta(\mu - X_t)dt + \sigma \sqrt{X_t}dB_t \qquad \text{is }\quad \sA = \theta(\mu-x)\fp{}{x} + \frac 12 \sigma^2x \fpp{}{x}.
\ee
\een
\end{example}

\begin{definition}[adjoint infinitesimal generator\index{adjoint infinitesimal generator}]
Let $\sA$ be an infinitestimal generator for Markov process $X$ on $(\sS,\sB(\sS))$. Then $\sA^*$, its adjoint infinitesimal generator with respect to the quadratic inner product is defined by
\be
\int_{\sS} f(x)\sA g(x) dx = \int_{\sS} g(x)\sA^* f(x)dx
\ee
for all suitable functions $f$ and $g$.
\end{definition}

\subsection{Kolmogorov Backward and Forward Equations}

\begin{theorem}[Kolmogorov backward equation\index{Kolmogorov backward equation!one dimensional}, Kolmogorov forward equation\index{Kolmogorov forward equation!one dimensional}]
Let a one-dimensional diffusion process $(X_t)_{t\geq 0}$ have SDE\footnote{condition of $\mu$ and $\sigma$ needed.}
\be
dX_t = \mu(t,X_t) dt + \sigma(t,X_t)dB_t.
\ee

For some function $g(x)$ and define for fixed $t$,
\be
u(s,x) := \E\bb{g(X_t)|X_s = x}
\ee
such that $u:s\mapsto u(s,x) \in C(\R^{++})$ and $u:x \mapsto u(s,x)\in C^2(\R)$. 

Then Kolmogorov backward equation is given by
\be
\fp{u}{s} + \mu(s,x)\fp{u(s,x)}{x} + \frac 12 \sigma^2(s,x) \fpp{u(s,x)}{x} = \fp{u}{s} + \sA u  = 0
\ee
with terminal condition $u(t,x) = g(x)$ where $\sA$ is the infinitesimal generator of $X$,
\be
\sA = \mu(s,x)\fp{}{x} + \frac 12 \sigma^2(s,x) \fpp{}{x}.
\ee

In particular, if $P_{s,t}(x,A)$ is continuously differentiable in $s$ and twice continuously differentiable in $x$, then
\be
\fp{}{s}P_{s,t}(x,A) + \mu(s,x)\fp{}{x}P_{s,t}(x,A) + \frac 12 \sigma^2(s,x) \fpp{}{x}P_{s,t}(x,A) = 0.
\ee

Furthermore, if the transition density of $X$, $p_{s,t}(x,y)$ exists and is continuously differentiable in $s$ and twice continuously differentiable in $x$, then
\be
\fp{}{s}p_{s,t}(x,y) + \mu(s,x)\fp{}{x}p_{s,t}(x,y) + \frac 12 \sigma^2(s,x) \fpp{}{x}p_{s,t}(x,y) = 0.
\ee

%In addition, if $X$ has a transition density $p_{s,t}(x,y)$, then

In addition, if $p_{s,t}(x,y)$ exists and is continuously differentiable in $t$ and twice continuously differentiable in $y$, Kolmogorov forward equation (also called Fokker-Planck equation) is then given by 
\be
\fp{}{t}p_{s,t}(x,y) = - \fp{}{y} \bb{\mu(t,y)p_{s,t}(x,y)} + \frac 12 \fpp{}{y}\bb{\sigma^2(t,y)p_{s,t}(x,y)} = \sA^* p_{s,t}(x,y)
\ee
where $\sA^*$ is the adjoint operator of $\sA$, defined as
\be
\sA^* v(t,y)= - \fp{}{y} \bb{\mu(t,y)v(t,y)} + \frac 12 \fpp{}{y}\bb{\sigma^2(t,y)v(t,y)}.
\ee
\end{theorem}

\begin{proof}[\bf Proof]
{\bf Backward equation.} By It\^o lemma
\be
du(s,X_s) = \bb{\fp{}{s}u(s,X_s)  + \mu(s,X_s)\fp{}{x} u(s,X_s) + \frac 12 \sigma^2(s,X_s) \fpp{u(s,X_s)}{x}}ds + \sigma(s,X_s) \fp{}{x} u(s,X_s) dB_s
\ee

Then
\beast
& & u(t,X_t) - u(s,X_s) \\
& = & \int^t_s \bb{\fp{}{r}u(r,X_r)  + \mu(r,X_r)\fp{}{x} u(r,X_r) + \frac 12 \sigma^2(r,X_r) \fpp{u(r,X_r)}{x}}dr + \int^t_s\sigma(r,X_r) \fp{}{x} u(r,X_r) dB_r
\eeast

Taking the expectation, by tower property we have
\be
\E\bb{u(t,X_t) - u(s,X_s)} = \E\bb{\E(g(X_t)|X_t=x) - \E(g(X_t)|X_s=x) } = \E(g(X_t)) - \E(g(X_t)) = 0.% \E(u(s,X_s)|X_s=x) = u(s,X_s) - u(X_s)=0 \text{a.s.}
\ee

Therefore, for any $s,t,x$,
\be
0 = \E\bb{\int^t_s \bb{\fp{}{r}u(r,X_r)  + \mu(r,X_r)\fp{}{x} u(r,X_r) + \frac 12 \sigma^2(r,X_r) \fpp{u(r,X_r)}{x}}dr }%\right|X_s = x}
\ee
which implies that $u(s,x)$ solves the PDE
\be
\fp{}{s}u(s,x)  + \mu(s,x)\fp{}{x} u(s,x) + \frac 12 \sigma^2(s,x) \fpp{u(s,x)}{x} = 0.
\ee

In particular, letting $g(x) = \ind_A$, we have
\be
u(s,x) = \E\bb{\ind_A(X_t)|X_s = x} = \pro\bb{X_t\in A|X_s = x} = P_{s,t}(x,A).
\ee

\be
\fp{}{s}P_{s,t}(x,A) + \mu(s,x)\fp{}{x}P_{s,t}(x,A) + \frac 12 \sigma^2(s,x) \fpp{}{x}P_{s,t}(x,A) = 0.
\ee

Since $A$ could be any set, we can let $A$ shrink to any point $y$ and therefore we can have $g(x) = \delta(y-x)$ and thus
\be
u(s,x) = \E\bb{\delta(y-X_t)|X_s = x} = \int \delta(y-z)p_{s,t}(x,z)dz = p_{s,t}(x,y)% \pro\bb{X_t= A|X_s = x} = P_{s,t}(x,A).
\ee
which implies that
\be
\fp{}{s}p_{s,t}(x,y) + \mu(s,x)\fp{}{x}p_{s,t}(x,y) + \frac 12 \sigma^2(s,x) \fpp{}{x}p_{s,t}(x,y) = 0.
\ee




%Approach 1. The backward equation is direct result from Feynman-Kac formula\footnote{theorem needed.}.
%
%Approach 2. see Rogers' book

{\bf Forward equation.} By Proposition \ref{pro:markov_transition_density_semigroup}, we have for sufficient small $\ve>0$,
\beast
p_{s,t}(x,z) & = & \int^\infty_{-\infty} p_{s,t-\ve}(x,y)p_{t-\ve,t}(y,z)dy.
\eeast

Then by Taylor expansion and backward equation
\beast
p_{s,t}(x,z) & = & \int^\infty_{-\infty} \bb{p_{s,t}(x,y) - \ve \fp{}{t} p_{s,t}(x,y)} \bb{p_{t,t}(y,z) - \ve \left.\fp{}{s} p_{s,t}(y,z)\right|_{s=t}} dy + O(\ve^2)\\
& = & \int^\infty_{-\infty} \bb{p_{s,t}(x,y) - \ve \fp{}{t} p_{s,t}(x,y)} \bb{\delta(z-y) + \ve \left.\bb{\mu(s,y)\fp{p_{s,t}(y,z)}{y} + \frac 12 \sigma^2(s,y) \fpp{p_{s,t}(y,z)}{y}}\right|_{s=t}} dy + O(\ve^2) \\
& = & p_{s,t}(x,z) - \ve \fp{}{t} p_{s,t}(x,z) + \ve \int^\infty_{-\infty}p_{s,t}(x,y)  \bb{\left.\mu(s,y)\fp{p_{s,t}(y,z)}{y} + \frac 12 \sigma^2(s,y) \fpp{p_{s,t}(y,z)}{y}\right|_{s=t}} dy + O(\ve^2)
\eeast

Then we discard term $O(\ve^2)$ and get
\beast
\fp{}{t} p_{s,t}(x,z) & = & \int^\infty_{-\infty} p_{s,t}(x,y)  \bb{\left.\mu(t,y)\fp{p_{s,t}(y,z)}{y} + \frac 12 \sigma^2(t,y) \fpp{p_{s,t}(y,z)}{y}\right|_{s=t}} dy\\
& = & \int^\infty_{-\infty}  \mu(t,y) p_{s,t}(x,y)  d \bb{\left.p_{s,t}(y,z)\right|_{s=t}} + \frac 12  \sigma^2(t,y)p_{s,t}(x,y) d\bb{\left.\fp{p_{s,t}(y,z)}{y}\right|_{s=t}}\\
& = & -\int^\infty_{-\infty}   \bb{\left.p_{s,t}(y,z)\right|_{s=t}}\fp{}{y}\bb{\mu(t,y) p_{s,t}(x,y)}dy - \frac 12   \bb{\left.\fp{p_{s,t}(y,z)}{y}\right|_{s=t}}\fp{}{y}\bb{\sigma^2(t,y)p_{s,t}(x,y)}dy\\
& = & -\int^\infty_{-\infty}   \delta(z-y)\fp{}{y}\bb{\mu(t,y) p_{s,t}(x,y)}dy + \frac 12  \fpp{}{y}\bb{\sigma^2(t,y)p_{s,t}(x,y)} \bb{\left.p_{s,t}(y,z)\right|_{s=t}}dy\\
& = & - \fp{}{z}\bb{\mu(t,z) p_{s,t}(x,z)} + \frac 12  \fpp{}{z}\bb{\sigma^2(t,z)p_{s,t}(x,z)}
\eeast
as we assume that $p_{s,t}(x,y)$ and its derivative vanish at $y=\pm\infty$.
\end{proof}


\begin{example}
For the transition density of Brownian motion,
\be
p_{s,t}(x,y) = \frac 1{\sqrt{2\pi\sigma^2 (t-s)}} \exp\bb{-\frac{(y-x)^2}{2\sigma^2(t-s)}}
\ee
where $\mu(t,x) = 0$ and $\sigma(t,x) = \sigma$. Then we have
\be
\fp{}{t}p_{s,t}(x,y) = -\frac 12 \frac 1{\sqrt{2\pi\sigma^2 (t-s)^3}} \exp\bb{-\frac{(y-x)^2}{2\sigma^2(t-s)}}  + \frac{(y-x)^2}{2\sigma^2(t-s)^2} \frac 1{\sqrt{2\pi\sigma^2 (t-s)}} \exp\bb{-\frac{(y-x)^2}{2\sigma^2(t-s)}}
\ee

\be
\fp{}{y}p_{s,t}(x,y) = -\frac{y-x}{\sigma^2(t-s)} \frac 1{\sqrt{2\pi\sigma^2 (t-s)}} \exp\bb{-\frac{(y-x)^2}{2\sigma^2(t-s)}}
\ee


\be
\fpp{}{y}p_{s,t}(x,y) = \bb{-\frac{1}{\sigma^2(t-s)} \frac 1{\sqrt{2\pi\sigma^2 (t-s)}} + \frac{(y-x)^2}{\sigma^4(t-s)^2} \frac 1{\sqrt{2\pi\sigma^2 (t-s)}}}\exp\bb{-\frac{(y-x)^2}{2\sigma^2(t-s)}}.
\ee

Obviously,
\be
\fp{}{t}p_{s,t}(x,y)= \frac 12 \sigma^2 \fpp{}{y}p_{s,t}(x,y)
\ee
which is Kolmogorov forward equation for Brownian motion.
\end{example}




\begin{theorem}[multiple dimensional Kolmogorov equations]
Let $n$-dimensional diffusion process $(X_t)_{t\geq 0}$ on $\sS \subseteq \R^n$ have SDE
\be
dX_t = \mu(t,X_t) dt + \sigma(t,X_t) dB_t
\ee
where $\mu = \bb{\mu_1,\dots,\mu_n}^T$, $\sigma$ is $n\times m$ matrix and $B$ is $m$-dimensional Brownian motion.

For some function $g(x)$ and define for fixed $t$ and any $s<t$
\be
u(s,x) := \E\bb{g(X_t)|X_s = x}
\ee
such that $u:s\mapsto u(s,u)\in C(\R^{++},\R)$ and $u:x\mapsto u(s,x)\in C^2(\R^n,\R)$. Then
\be
\fp{}{s}u(s,x) + \sum^n_{i=1}\mu_i(s,x) \fp{}{x_i} u(s,x) + \frac 12 \sum^n_{i=1}\sum^n_{j=1} A_{ij}(s,x) \frac{\partial^2}{\partial x_i\partial x_j} u(s,x) = 0
\ee
where
\be
A = \sigma\sigma^T,\qquad A_{ij}(t,x) := \sum^n_{k=1} \sigma_{ik}(t,x) \sigma_{jk}(t,x).
\ee


Then if its transition density exists and is continuously differentiable in $s,t$ with $s<t$ and twice differentiable in $x,y\in\sS$, its Kolmogorov backward equation is
\be
\fp{}{s}p_{s,t}(x,y) + \sum^n_{i=1}\mu_i(s,x) \fp{}{x_i}\bb{p_{s,t}(x,y)} + \frac 12 \sum^n_{i=1}\sum^n_{j=1} A_{ij}(s,x) \frac{\partial^2}{\partial x_i\partial x_j}\bb{p_{s,t}(x,y)} = 0
\ee
%where
%\be
%A = \sigma\sigma^T,\qquad A_{ij}(t,x) := \sum^n_{k=1} \sigma_{ik}(t,x) \sigma_{jk}(t,x).
%\ee

Also, Kolmogorov forward equation is
\be
\fp{}{t} p_{s,t}(x,y)= - \sum^n_{i=1}\fp{}{y_i}\bb{\mu_i(t,y)p_{s,t}(x,y)} + \frac 12 \sum^n_{i=1}\sum^n_{j=1} \frac{\partial^2}{\partial y_i\partial y_j}\bb{A_{ij}(t,y)p_{s,t}(x,y)}
\ee
\end{theorem}

\begin{proof}[\bf Proof]
By It\^o lemma,
\beast
u(t,X_t) & = & u(s,X_s) + \int^t_s \fp{}{r} u(r,X_r)dr + \sum^n_{i=1} \int^t_s \fp{}{x_i} u(r,X_r)dX^i_r + \frac 12 \sum^n_{i=1} \sum^n_{j=1} \int^t_s \frac{\partial^2}{\partial x_i\partial x_j} u(r,X_r)d[X^i,X^j]_r \\
& = & u(s,X_s) + \int^t_s \fp{}{r} u(r,X_r)dr + \sum^n_{i=1} \int^t_s \fp{}{x_i} u(r,X_r)\bb{\mu_i dr + \bb{\sigma^T}_i^T dB_r} \\
& & \qquad \qquad\qquad \qquad \qquad \qquad\qquad \qquad + \frac 12 \sum^n_{i=1} \sum^n_{j=1} \int^t_s \frac{\partial^2}{\partial x_i\partial x_j} u(r,X_r)d\bsb{Y^i,Y^j}
\eeast
where $dY_t = \sigma dB_t$. Then the final term can be expressed by\footnote{theorem needed.}
\beast
\frac 12 \sum^n_{i=1} \sum^n_{j=1} \int^t_s \frac{\partial^2}{\partial x_i\partial x_j} u(r,X_r) d\bsb{\bb{\sigma^T}_i^TB_r, B_r^T\bb{\sigma^T}_j} & = & \frac 12 \sum^n_{i=1} \sum^n_{j=1} \int^t_s \frac{\partial^2}{\partial x_i\partial x_j} u(r,X_r)\bb{\sigma^T}_i^T\bb{\sigma^T}_j d\bsb{B_r, B_r^T} \\
& = & \frac 12 \sum^n_{i=1} \sum^n_{j=1} \int^t_s \frac{\partial^2}{\partial x_i\partial x_j} u(r,X_r) \bb{\sigma^T}_i^T \bb{\sigma^T}_j I dr\\
& = & \frac 12 \sum^n_{i=1} \sum^n_{j=1} \int^t_s \frac{\partial^2}{\partial x_i\partial x_j} u(r,X_r) \bb{\sigma^T}_i^T \bb{\sigma^T}_j dr\\
\eeast

Since $\bb{\sigma^T}_i^T$ is the $i$th row of $\sigma$, we have that
\be
\bb{\sigma^T}_i^T \bb{\sigma^T}_j = \sum^n_{k=1} \sigma_{ik}\sigma_{jk}
\ee
which implies that
\beast
u(t,X_t) & =& u(s,X_s) + \int^t_s \fp{}{r} u(r,X_r)dr + \sum^n_{i=1} \int^t_s \fp{}{x_i} u(r,X_r)\bb{\mu_i dr + \bb{\sigma^T}_i^T dB_r} \\
 & & \qquad \qquad\qquad \qquad \qquad \qquad\qquad \qquad + + \frac 12 \sum^n_{i=1} \sum^n_{j=1} \int^t_s A_{ij}(r,X_r) \frac{\partial^2}{\partial x_i\partial x_j} u(r,X_r) dr.
\eeast

Then we take the expectation and have that for any $s,t,x$ given $X_s = x$
\be
0 = \E\bb{\int^t_s \bb{\fp{}{r} u(r,X_r) + \sum^n_{i=1} \mu_i (r,X_r)\fp{}{x_i} u(r,X_r)  + \frac 12 \sum^n_{i=1} \sum^n_{j=1}   A_{ij}(r,X_r) \frac{\partial^2}{\partial x_i\partial x_j} u(r,X_r) }dr}
\ee
which implies that $u(s,x)$ solves the PDE
\be
\fp{}{s} u(s,x) + \sum^n_{i=1} \mu_i(s,x) \fp{}{x_i} u(s,x)  + \frac 12 \sum^n_{i=1} \sum^n_{j=1}   A_{ij}(s,x) \frac{\partial^2}{\partial x_i\partial x_j} u(s,x) = 0.
\ee

Then we have the rest results with the same argument with one-dimensional case.
\end{proof}



\begin{example}
For 2 dimensional Brownian motion, $\mu = 0$ and $\sigma = \bepm 1 & 0 \\ 0 & 1\eepm$. Thus, $A = \bepm 1 & 0 \\ 0 & 1 \eepm$ and its Kolmogorov backward equation is
\be
\fp{}{s}p_{s,t}(x,y) + \frac 12\bb{ \fpp{}{x_1} + \fpp{}{x_2}}p_{s,t}(x,y) =0
\ee

The Kolmogorov forward equation is
\be
\fp{}{t}p_{s,t}(x,y) = \frac 12\bb{ \fpp{}{y_1} + \fpp{}{y_2}}p_{s,t}(x,y).
\ee
\end{example}



\section{Feller Processes}



\section{Problems}