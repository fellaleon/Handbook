

\chapter{Metric and Topological Space}



\section{Topological Space}






\qcutline

Given a normed space X, a subspace E of X is

1) Banach(=complete) if every Cauchy sequence in E converge to a point of E.

2) Closed if every sequence in E that converge in X, converge to a point of E.

If you want to make sure you understand the distinction and relation between the two, prove these two elementary observations: "If X is Banach and E is closed, then E is Banach" and this: "If E is Banach, then is it closed."

1) Since X is Banach, a given Cauchy sequence in E (which must then also be in X) converges to a point in X and since E is closed, every sequence from E that converges in X has a limit in E - and so has our Cauchy sequence. Summary: any given Cauchy sequence in E has limit in E which is the definition of completness.

2) Since E is Banach, every Cauchy seq. from E has limit in E. Also every convergent (with limit in X, generally) sequence in E must be Cauchy sequence -> these two together imply that every convergent sequence from E must have limit in E which is what I want to prove.




\qcutline

\section{Higher dimensions}

\begin{definition}\label{def:norm_metop}
If $\dabs{\cdot}$ is the Euclidean norm on $\R^n$ then
\ben
\item [(i)] $\dabs{x} \geq 0$ for all $x\in \R^n$,
\item [(ii)] If $\dabs{x} = 0$, then $x=0$,
\item [(iii)] If $\lm \in \R$ and $x\in \R^n$ then $\dabs{\lm x} = \abs{\lm}\dabs{x}$,
\item [(iv)] (The triangle inequality) If $x,y\in \R^n$, then $\dabs{x+y} \leq \dabs{x} + \dabs{y}$.
\een
\end{definition}

\begin{definition}\label{def:uniform_norm_metop}
The uniform norm assigns to real- or complex-valued bounded functions $f$ defined on a set $S$ the nonnegative number
\be
\dabs{f}_\infty=\dabs{f}_{\infty,S}=\sup\left\{\abs{f(x)}:x\in S\right\}.
\ee
This norm is also called the supremum norm, the Chebyshev norm, or the infinity norm.
\end{definition}

\begin{definition}\label{def:lp_space_metop}
\ben
\item [(i)] $\ell^1$, the space of sequences whose series is absolutely convergent,
\item [(ii)] $\ell^2$, the space of square-summable sequences, which is a Hilbert space,
\item [(iii)] $\ell^\infty$, the space of bounded sequences.
\item [(iv)] $x\in\ell_0$, then $\exists k\in\N$ such that
\be
\sum_{i>k} \abs{x_i} < 1.
\ee
\een
\end{definition}

\begin{lemma}[Young inequality]\label{lem:young_inequality_met_top}
Since the function $\log$ is concave, for all $a,b>0$ we have
\be
\log ab = \tfrac 1p \log a^p + \tfrac 1q \log b^q \leq \log \bb{\tfrac 1p a^p + \tfrac 1q b^q} \ \ra \ ab \leq \tfrac 1p a^p + \tfrac 1q b^q.
\ee
\end{lemma}

\begin{lemma}[H\"older's inequality]
Given $x,y\in \ell_0$ we have
\be
\frac{\abs{\sum^\infty_{k=1} x_ky_k}}{\dabs{x}_p\dabs{y}_q} \leq \frac{\sum^\infty_{k=1} \abs{x_k}\cdot\abs{y_k}}{\dabs{x}_p\dabs{y}_q} \leq \sum^\infty_{k=1} \frac{\abs{x_k}}{\dabs{x}_p} \cdot \frac{\abs{y_k}}{\dabs{y}_q} \leq \frac 1p \sum^\infty_{k=1} \frac{\abs{x_k}^p}{\dabs{x}_p^p} + \frac 1q\frac{\abs{y_k}^q}{\dabs{y}_q^q} = \frac 1p + \frac 1q = 1.
\ee
Thus,
\be
\abs{\sum^\infty_{k=1}x_ky_k}\leq \dabs{x}_p\dabs{y}_q = \bb{\sum^\infty_{k=1}\abs{x_k}^p}^{\frac 1p}\bb{\sum^\infty_{k=1}\abs{y_k}^q}^{\frac 1q}
\ee
\end{lemma}



\section{Metrics}

\begin{definition}[metric\index{metric}]\label{def:metric_space_met_top}
We say that $(X,d)$ is a metric space if $X$ is a set and $d:X\times X\to \R$ is a function with the following properties:
\ben
\item [(i)] $d(x,y)\geq 0$ for all $x,y\in X$.
\item [(ii)] $d(x,y) = 0$ if and only if $x=y$.
\item [(iii)] $d(x,y) = d(y,x)$ for all $x,y\in X$.
\item [(iv)] (The triangle inequality) $d(x,z) \leq d(x,y) + d(y,z)$.
\een
\end{definition}

\begin{definition}\label{def:metrics_lipschitz_equivalence_metop}
Let $A$ be a set upon which there are two metrics imposed: $d$ and $d'$. $\exists h,k \in \R: h>0,k>0$ such that
\be
\forall x,y\in A: \quad hd'(x,y)\leq d(x,y)\leq kd'(x,y).
\ee
Then $d$ and $d'$ are described as Lipschitz equivalent.
\end{definition}

\begin{definition}\label{def:metrics_space_lipschitz_equivalence_metop}
Let $M=(A,d)$ and $M'=(A',d')$ be metric spaces. Let $f:M\to M'$ be a mapping such that $\exists h,k\in\R:h>0,k>0$ such that:
\be
\forall x,y\in A:hd'(f(x),f(y))\leq d(x,y)\leq kd'(f(x),f(y)).
\ee
Then $f$ is a Lipschitz equivalence, and $M$ and $M$ are described as Lipschitz equivalent.
\end{definition}

\begin{remark}
all norms on a finite-dimensional space are Lipschitz equivalent.
\end{remark}




\begin{remark}
If $E$ is a closed and bounded set in $\R^n$, then any continuous function defined on $E$ has bounded image. Conversely, if every continuous real-valued function on $E\subseteq \R^n$ is bounded, then $E$ is closed and bounded.
\end{remark}


\begin{theorem}[Contraction mapping theorem]\label{thm:contraction_mapping_metop}
Let $(X, d)$ be a non-empty complete metric space. Let $T : X \to X$ be a contraction mapping on $X$, i.e.: there is a nonnegative real number $q < 1$ such that
\be
d(T(x),T(y)) \le q\cdot d(x,y)
\ee
for all $x, y \in X$. Then the map $T$ admits one and only one fixed point $x^* \in X$ (this means $T(x^*) = x^*$). Furthermore, this fixed point can be found as follows: start with an arbitrary element $x_0 \in X$ and define an iterative sequence by $x_n = T(x_{n-1})$ for $n = 1, 2, 3,\dots$. This sequence converges, and its limit is $x^*$. The following inequality describes the speed of convergence:
\be
d(x^*, x_n) \leq \frac{q^n}{1-q} d(x_1,x_0).
\ee
Equivalently,
\be
d(x^*, x_{n+1}) \leq \frac{q}{1-q} d(x_{n+1},x_n)
\ee
and
\be
d(x^*, x_{n+1}) \leq q d(x^*,x_n).
\ee
Any such value of $q$ is called a Lipschitz constant for $T$, and the smallest one is sometimes called "the best Lipschitz constant" of $T$.

Note that the requirement $d(T(x), T(y)) < d(x, y)$ for all unequal $x$ and $y$ is in general not enough to ensure the existence of a fixed point, as is shown by the map $T : [1,\infty) \to [1,\infty)$ with $T(x) = x + 1/x$, which lacks a fixed point. However, if the metric space $X$ is compact, then this weaker assumption does imply the existence and uniqueness of a fixed point, that can be easily found as the limit of any sequence of iterations of $T$, as in the fixed point theorem for contractions, or also variationally, as a minimizer of $d(x, T(x))$: indeed, a minimizer exists by compactness, and has to be a fixed point of $T$.

When using the theorem in practice, the most difficult part is typically to define $X$ properly so that $T$ actually maps elements from $X$ to $X$, i.e. that $T(x)$ is always an element of $X$.
\end{theorem}



\section{Open and closed sets}

\begin{definition}\label{def:closed_set_metop}
A set $E \subseteq \R^m$ is closed if whenever $x_n \in E$ for each $n$ and $x_n \to x$ as $n\to \infty$ then $x\in E$.
\end{definition}

\begin{theorem}[Bolzano-Weierstrass]\label{thm:closed_set_subsequence_metop}
If $E$ is closed bounded set in $\R^m$, then every sequence in $E$ has a subsequence converging to a point of $E$.
\end{theorem}

\begin{definition}\label{def:open_set_met_top}
A set $E \subseteq \R^m$ is open if whenever $x_n \in E$ there exists a $\ve(x)>0$ such that $\dabs{x-y}<\ve(x)$ we have $y \in E$.
\end{definition}

\begin{theorem} Any finite dimensional subspace of a normed vector space is closed.
\end{theorem}
\begin{proof}[{\bf Proof}] Let $(V,\dabs{\cdot})$ be such a normed vector space, and $S\subset V$ a finite dimensional vector subspace.

Let $x\in V$, and let $(s_n)_n$ be a sequence in $S$ which converges to $x$. We want to prove that $x\in S$. Because $S$ has finite dimension, we have a basis $\{x_1,\dots x_k\}$ of $S$. Also, $x \in \text{span}(x_1,\dots, x_k)$. But, as proved in the case when $V$ is finite dimensional, we have that $S$ is closed in $\text{span}(x_1,\dots,x_k)$ (taken with the norm induced by $(V,\dabs{\cdot})$ with $s_n \to x$, and then $x\in S$.
\end{proof}

\begin{note}
The definition of a normed vector space requires the ground field to be the real or complex numbers. Indeed, consider the following counterexample if that condition doesn't hold:

$V=\R$ is a $\Q$-vector space, and $S=\Q$ is a vector subspace of $V$. It is easy to see that $\dim(S)=1$ (while $\dim(V)$ is infinite), but $S$ is not closed on $V$.
\end{note}


\section{Continuity}

\begin{definition}\label{def:continous_rn}
Let $E\subseteq \R^m$ we say that a function $f:E\to \R^p$ is continuous on $E$ if given any point $x \in E$ and any $\ve>0$ we can find a $\delta(\ve,x)>0$ such that whenever $y\in E$ and $\dabs{x-y} < \delta(\ve,x)$ we have
\be
\dabs{f(x)-f(y)} < \ve.
\ee
\end{definition}

\begin{definition}\label{def:uniformly_continous}
Let $E\subseteq \R^m$ we say that a function $f:E\to \R^p$ is uniformly continuous on $E$ if given any $\ve>0$ we can find a $\delta(\ve)>0$ such that whenever $x,y\in E$ and $\dabs{x-y} < \delta(\ve)$ we have
\be
\dabs{f(x)-f(y)} < \ve.
\ee
\end{definition}

\begin{theorem} \label{thm:_uniformly_continuous}
Let $E$ be a closed bounded subset of $\R^m$. If $f: E\to \R^p$ is continuous on $E$, then $f$ is uniformly continuous on $E$.
\end{theorem}

\begin{definition}\label{def:continously_differentiable_met_top}
The class $C^1$ consists of all differentiable functions whose derivative is continuous; such functions are called continuously differentiable.
\end{definition}

\section{Convergence}

\begin{definition}\label{def:uniform_convergence_met_top}
If $E$ is non-empty set and $f_n:E\to \F$ and $f:E\to\F$ are functions we say that $f_n$ converges uniformly to $f$ as $n\to \infty$, if given any $\ve>0$ we can find an $N(\ve)>0$ such that
\be
\dabs{f_n(x)-f(x)} < \ve
\ee
for all $x\in E$ and all $n\geq N(\ve)$.
\end{definition}

\begin{theorem}[Uniform convergence]\label{thm:function_uniform_convergence_continuous}
If $(E,d)$ is a non-empty metric space and $f_n: E\to \F$ form a sequence of continuous functions tending uniformly to $f$, then $f$ is continuous.
\end{theorem}


\begin{proof}[{\bf Proof}]
add things here.
\end{proof}




\section{Isomorphism}

\begin{definition}
an isomorphism is a bijective map $f$ such that both $f$ and its inverse $f^{-1}$ are homomorphisms, i.e., structure-preserving mappings. In the more general setting of category theory, an isomorphism is a morphism $f: X \to Y$ in a category for which there exists an "inverse" $f^{-1}: Y \to X$, with the property that both $f^{-1}f = \text{id}_X$ and $f f^{-1} = \text{id}_Y$.
\end{definition}

\begin{theorem}[Inverse function theorem]\label{thm:inverse_function_single_value}
For functions of a single variable, the theorem states that, if $f$ is a continuously differentiable function and $f$ has a nonzero derivative at $a$, then $f$ is invertible in a neighborhood of $a$, the inverse is continuously differentiable, and
\be
\bigl(f^{-1}\bigr)'(b) = \frac{1}{f'(a)}
\ee
where $b = f(a)$, that is, $(x,y)$ for the original function is $(a, b)$.
\end{theorem}

\begin{theorem}[Inverse function theorem]\label{thm:inverse_function_manifold}
The inverse function theorem can be generalized to differentiable maps between differentiable manifolds. In this context the theorem states that for a differentiable map $F : M \to N$, if the derivative of $F$,
\be
(dF)_p : T_pM \to T_{F(p)}N
\ee
is a linear isomorphism at a point $p$ in $M$ then there exists an open neighborhood $U$ of $p$ such that
\be
F|_U : U \to F(U)
\ee
is a diffeomorphism. Note that this implies that $M$ and $N$ must have the same dimension. If the derivative of $F$ is an isomorphism at all points $p$ in $M$ then the map $F$ is a local diffeomorphism.
\end{theorem}



%%%%%%%%%%%%%%%%%%%%%%%%%%%%%%%












\section{Preface} Within the last fifty years the material in this course has been taught at Cambridge in the fourth (postgraduate), third, second and first years or left to students to pick up for themselves. Under present arrangements students may take the course either at the end of their first year
(before they have met metric spaces in analysis) or at the end of their second year (after they have met metric spaces).

Because of this, the first third of the course presents a rapid overview of metric spaces (either as revision or a first glimpse) to set the scene for the main topic of topological spaces. This arrangement is recognised in the examination structure where the 12~lecture course is treated as though it were an 8~lecture course.

The first part of these notes states and discusses the main results of the course. Usually, each statement is followed by directions to a proof in the final part of these notes. Whilst I do not expect the reader to find all the proofs by herself, I do ask that she \emph{tries} to give a proof herself before looking one up. Some of the more difficult theorems have been provided with hints as well as proofs.

In my opinion, the two sections on compactness are the deepest part of the course and the reader who has mastered the proofs of the results therein is well on the way to mastering the whole course.

May I repeat that, as I said in the small print, I welcome corrections and comments.


\section{What is a metric?} If I wish to travel from Cambridge to Edinburgh, then I may be interested in one or more of the following numbers.

(1) The distance, in kilometres, from Cambridge to Edinburgh 'as the crow flies'.

(2) The distance, in kilometres, from Cambridge to Edinburgh by road.

(3) The time, in minutes, of the shortest journey from Cambridge to Edinburgh by rail.

(4) The cost, in pounds, of the cheapest journey from Cambridge to Edinburgh by rail.

Each of these numbers is of interest to someone and none of them is easily obtained from another. However, they do have certain properties in common which we try to isolate in the following definition.

\begin{definition}\label{D;metric}
Let $X$ be a set\footnote{We thus allow $X=\emptyset$.
This is purely a question of taste. If we did not
allow this possibility, then, every time we
defined a metric
space $(X,d)$, we would need to prove that $X$ was non-empty.
If we do allow this possibility, and we prefer
to reason about non-empty spaces, then we can begin our proof
with the words 'If $X$ is empty, then the result is
vacuously true, so we may assume that $X$ is non-empty.'
(Of course, the result may be false for $X=\emptyset$,
in which case the statement of the theorem must include
the condition $X\neq\emptyset$.)} and
$d:X^{2}\rightarrow\R$ a function with the
following properties:-

(i) $d(x,y)\geq 0$ for all $x,\,y\in X$.

(ii) $d(x,y)=0$ if and only if $x=y$.

(iii) $d(x,y)=d(y,x)$ for all $x,\,y\in X$.

(iv) $d(x,y)+d(y,z)\geq d(x,z)$ for all $x,\,y,\,z\in X$.
(This is called the 'triangle inequality' after the
result in Euclidean geometry that the sum of the lengths
of two sides of a triangle is at least as great as the
length of the third side.)

Then we say that $d$ is a metric on $X$ and that $(X,d)$
is a metric space.
\end{definition}
You should imagine the author muttering under his breath

'(i) Distances are always positive.

(ii) Two points are zero distance apart if and only
if they are the same point.

(iii) The distance from $A$ to $B$ is the same
as the distance from $B$ to $A$.

(iv) The distance from $A$ to $B$ via $C$
is at least as great as the distance from $A$ to $B$
directly.'

\begin{theorem}
If $d:X^{2}\rightarrow\R$ is a function with the
following properties:-

(ii) $d(x,y)=0$ if and only if $x=y$,

(iii) $d(x,y)=d(y,x)$ for all $x,\,y\in X$,

(iv) $d(x,y)+d(y,z)\geq d(x,z)$ for all $x,\,y,\,z\in X$,

\noindent show that $d$ is a metric on $X$.
\end{theorem}
\begin{proof}[\bf Proof] Setting $z=x$ in condition~(iv) and using
(iii) and (ii), we have
\[2d(x,y)=d(x,y)+d(y,x)\geq d(x,x)=0 \ \ \ra \ d(x,y)\geq 0 .\]
\end{proof}

\begin{theorem}\label{P;not metric}
Let $X=\{a,\,b,\,c\}$ with $a$, $b$ and $c$
distinct. Write down functions $d_{j}:X^{2}\rightarrow\R$
satisfying condition~(i) of Definition~\ref{D;metric}
such that

(a) $d_{1}$ satisfies conditions~(ii) and~(iii) but not~(iv).

(b) $d_{2}$ satisfies conditions~(iii) and~(iv) but it is not
true that $x=y$ implies $d(x,y)=0$.

(c) $d_{3}$ satisfies conditions~(iii) and~(iv)
and $x=y$ implies $d_{3}(x,y)=0$.
but it is not
true that  $d_{3}(x,y)=0$ implies $x=y$.

(d) $d_{4}$ satisfies conditions~(ii) and~(iv) but not~(iii).

You should verify your statements.
\end{theorem}
\begin{proof}[\bf Proof] Here are some possible choices.

(a) Take $d_{1}(x,x)=0$ for all $x\in X$,
$d_{1}(a,b)=d_{1}(b,a)=d_{1}(a,c)=d_{1}(c,a)=1$
and $d_{1}(b,c)=d_{1}(c,b)=3$. Conditions~(ii) and~(iii)
hold by inspection but
\[d_{1}(b,a)+d_{1}(a,c)=2<3=d_{1}(b,c).\]

(b) Take $d_{2}(x,x)=1$
and $d_{2}(x,y)=2$ if $x\neq y$. Condition~(ii) fails
and condition~(iii) holds by inspection. We observe that
\[d_{2}(x,y)+d_{2}(y,z)\geq 1+1=2\geq d_{2}(x,z)\]
so the triangle law holds.

(c) Take
$d_{2}(x,y)=0$ for all $x,\,y\in X$.

(d) Take $d_{4}(x,x)=0$ for all $x\in X$, $d_{4}(a,b)=d_{4}(b,a)=d_{4}(a,c)=d_{4}(c,a)=1$ and $d_{1}(b,c)=d_{1}(c,b)=\tfrac{5}{4}$. Conditions~(ii) holds,
and condition~(iii) fails by inspection and
\begin{alignat*}{2}
d(x,y)+d(y,z)&=d(x,y)=d(x,z)\geq d(x,z)&&\qquad\text{if $y=z$},\\
d(x,y)+d(y,z)&=d(y,z)=d(x,z)\geq d(x,z)&&\qquad\text{if $x=y$},\\
d(x,y)+d(y,z)&\geq
1+1=2\geq\tfrac{5}{4}\geq d(x,z)&&\qquad\text{otherwise,}
\end{alignat*}
so the triangle law holds.
\end{proof}

We give another axiom grubbing exercise as Exercise~\ref{E;axiom grubbing}.
\begin{problem} Let $X$ be a set and
$\rho:X^{2}\rightarrow\R$ a function with the
following properties.

(i) $\rho(x,y)\geq 0$ for all $x,\,y\in X$.

(ii) $\rho(x,y)=0$ if and only if $x=y$.

(iv) $\rho(x,y)+\rho(y,z)\geq \rho(x,z)$ for all $x,\,y,\,z\in X$.

\noindent Show that, if we set $d(x,y)=\rho(x,y)+\rho(y,x)$,
then $(X,d)$ is a metric space.
\end{problem}

%%%%%%%%%%%%%%%%


Here are some examples of metric spaces. You have met (or you will meet) the concept of a normed vector space both in algebra and analysis courses.
\begin{definition}\label{D;norm}
Let $V$ be a vector space over $\F$ (with $\F=\R$ or $\F={\mathbb C}$) and $N:V\rightarrow\R$ a map such that, writing $N({\mathbf u})=\|{\mathbf u}\|$, the following results hold.

(i) $\|{\mathbf u}\|\geq 0$ for all ${\mathbf u}\in V$.

(ii) If  $\|{\mathbf u}\|=0$, then ${\mathbf u}={\mathbf 0}$.


(iii) If $\lambda\in{\mathbb F}$ and ${\mathbf u}\in V$,
then $\|\lambda{\mathbf u}\|=|\lambda| \|{\mathbf u}\|$.

(iv) [Triangle law.]
If ${\mathbf u},\,{\mathbf v}\in V$, then
$\|{\mathbf u}\|+\|{\mathbf v}\|\geq \|{\mathbf u}+{\mathbf v}\|$.

\noindent Then we call $\|\ \|$ a norm and say that
$(V,\|\ \|)$ is a normed vector space.


\end{definition}




%%%%%%%%%%%%%%%%%%%%%%%%%%%%%%


\begin{problem} By putting $\lambda=0$ in Definition~\ref{D;norm}~(iii),
show that $\|{\mathbf  0}\|=0$.
\end{problem}
Any normed vector space can be made into a metric space in a natural way.

\begin{theorem}\label{P;norm to metric}
If $(V,\dabs{\cdot})$ is a normed vector space,
then the condition
\[d({\mathbf u},{\mathbf v})=\|{\mathbf u}-{\mathbf v}\|\]
defines a metric $d$ on $V$.
\end{theorem}
\begin{proof}[\bf Proof] We observe that
\[d({\mathbf u},{\mathbf v})=\|{\mathbf u}-{\mathbf v}\|\geq 0\]
and
\[d({\mathbf u},{\mathbf u})=\|{\mathbf  0}\|=
\|0{\mathbf  0}\|=|0|\|{\mathbf  0}\|
=0\|{\mathbf  0}\|=0.\]
Further, if $d({\mathbf u},{\mathbf v})=0$,
then $\|{\mathbf u}-{\mathbf v}\|=0$
so ${\mathbf u}-{\mathbf v}={\mathbf  0}$ and ${\mathbf u}={\mathbf v}$.
We also observe that
\[d({\mathbf u},{\mathbf v})=\|{\mathbf u}-{\mathbf v}\|=
\|(-1)({\mathbf v}-{\mathbf u})\|
=|-1|\|{\mathbf v}-{\mathbf u}\|=d({\mathbf v},{\mathbf u})\]
and
\be
d({\mathbf u},{\mathbf v})+d({\mathbf v},{\mathbf w}) =  \|{\mathbf u}-{\mathbf v}\|+\|{\mathbf v}-{\mathbf w}\| \geq \|({\mathbf u}-{\mathbf v})+({\mathbf v}-{\mathbf w})\| =\|{\mathbf u}-{\mathbf w}\|= d({\mathbf u},{\mathbf w}).
\ee
\end{proof}

Many (but not all) metrics that we meet in analysis arise in this way.

However, not all metrics can be derived from norms in this way. Here is a metric that turns out to be more important and less peculiar than it looks at first sight.

\begin{definition} If $X$ is a set and we define
$d:X^{2}\rightarrow{\mathbb R}$ by
\begin{equation*}
d(x,y)=
\begin{cases}
0&\text{if $x=y$},\\
1&\text{if $x\neq y$},
\end{cases}
\end{equation*}
then $d$ is called the discrete metric on $X$.
\end{definition}





%/////////////////////////////////

\begin{theorem}
The the discrete metric on $X$ is indeed a metric.
\end{theorem}
\begin{proof}[\bf Proof] The only non-evident condition is the triangle law.
But
\begin{alignat*}{2}
d(x,y)+d(y,z)&=d(x,y)=d(x,z)\geq d(x,z)&&\qquad\text{if $y=z$},\\
d(x,y)+d(y,z)&=d(y,z)=d(x,z)\geq d(x,z)&&\qquad\text{if $x=y$},\\
d(x,y)+d(y,z)&\geq 1+1=2\geq 1\geq d(x,z)&&\qquad\text{otherwise,}
\end{alignat*}
so we are done.
\end{proof}

The next result, although easy,
is very definitely not part of the course,
so I leave it as an exercise for the interested reader.
\begin{problem} (i) If $V$ is a vector space over ${\mathbb R}$
and $d$ is a metric derived from a norm in the manner described above,
then, if ${\mathbf u}\in V$
we have $d({\mathbf  0},2{\mathbf u})=2d({\mathbf  0},{\mathbf u})$.

(ii) If $V$ is non-trivial (i.e. not zero-dimensional) vector space
over ${\mathbb R}$ and $d$ is the discrete metric on $V$, then
$d$ cannot be derived from a norm on $V$.
\end{problem}

In algebra you have learnt (or you will learn) about inner product
spaces. You have learnt (or you will learn) that every inner product
gives rise to a norm in a natural way. Most norms in analysis
do not arise in this way\footnote{This is not part of the
course, but see Exercise~\ref{E;parallelogram} if you are interested.}
but the few that do are very important.
\begin{definition} If ${\mathbf x}\in{\mathbb R}^{n}$, we write
\[\|{\mathbf x}\|_{2}=\left(\sum_{j=1}^{n}x_{j}^{2}\right)^{1/2},\]
where the positive square root is taken. We call $\|\ \|_{2}$
the Euclidean norm on ${\mathbb R}^{n}$.
\end{definition}
The reader should prove the next lemma before proceeding further.
(Pay particular attention to the triangle inequality.
In my opinion, the easiest proof uses inner products but
this is only an opinion and you may ignore it.).
\begin{lemma} The Euclidean norm on ${\mathbb R}^{n}$ is indeed a norm.
\end{lemma}
The metric derived from the Euclidean norm is called the Euclidean
metric. You should test any putative theorems on metric spaces
on both ${\mathbb R}^{n}$ with the Euclidean metric and
${\mathbb R}^{n}$ with the discrete metric.
\begin{problem}{\bf [The counting metric.]} If $E$ is a finite
set and ${\mathcal E}$ is the collection of subsets of $E$,
we write $\card C$ for the number of elements in $C$ and
\[d(A,B)=\card A\triangle B.\]
Show that $d$ is a metric on ${\mathcal E}$. The reader may be
inclined to dismiss this metric as uninteresting but it plays an
important role (as the Hamming metric) in the Part~II course
\emph{Codes and Cryptography}.
\end{problem}
Here are two metrics which are included simply to show that
metrics do not always look as simple as the ones above.
I shall use them as examples once or twice but
they do not form part of standard mathematical knowledge
and you do not have to learn their definition.
\begin{definition}\label{D;British Rail} (i) If we define
$d:{\mathbb R}^{2}\times{\mathbb R}^{2}\rightarrow{\mathbb R}$
by
\[
d({\mathbf u},{\mathbf v})=
\begin{cases}
\|{\mathbf u}\|_{2}+\|{\mathbf v}\|_{2},&
\text{if ${\mathbf u}\neq{\mathbf v}$},\\
0&\text{if ${\mathbf u}={\mathbf v}$,}
\end{cases}
\]
then $d$ is called the British Rail express metric. [To get from
$A$ to $B$ travel via London.]

(ii) If we define
$d:{\mathbb R}^{2}\times{\mathbb R}^{2}\rightarrow{\mathbb R}$
by
\[
d({\mathbf u},{\mathbf v})=
\begin{cases}
\|{\mathbf u}-{\mathbf v}\|_{2}&\text{if ${\mathbf u}$ and ${\mathbf v}$
are linearly dependent,}\\
\|{\mathbf u}\|_{2}+\|{\mathbf v}\|_{2}&\text{otherwise,}
\end{cases}
\]
then $d$ is called the British Rail stopping metric. [To get from
$A$ to $B$ travel via London unless $A$ and $B$ are on the same London
route.]
\end{definition}
(Recall that ${\mathbf u}$ and ${\mathbf v}$
are linearly dependent if ${\mathbf u}=\lambda{\mathbf v}$
for some real $\lambda$ and/or ${\mathbf v}={\mathbf  0}$.)

\begin{theorem}\label{P;British Rail is metric}
Show that the British Rail express metric
and the British Rail stopping metric are indeed metrics.
\end{theorem}
\begin{proof}[\bf Proof]
We show that the British Rail stopping metric is indeed a metric.
The case of the British Rail express metric is left to the reader.

Let $d$ be the British rail stopping metric on ${\mathbb R}^{2}$.
It is easy to see that $d({\mathbf u},{\mathbf v})\geq 0$
and that $d({\mathbf u},{\mathbf v})=d({\mathbf v},{\mathbf u})$.
Since ${\mathbf u}$ and ${\mathbf u}$ are linearly dependent,
\[d({\mathbf u},{\mathbf u})=\|{\mathbf u}-{\mathbf u}\|_{2}
=\|{\mathbf  0}\|_{2}=0.\]

If $d({\mathbf u},{\mathbf v})=0$, then we know that at least one
of the following statements is true

(1) $\|{\mathbf u}-{\mathbf v}\|_{2}=0$ and so
${\mathbf u}-{\mathbf v}={\mathbf  0}$,

(2) $\|{\mathbf u}\|_{2}+\|{\mathbf v}\|_{2}=0$ and so
$\|{\mathbf u}\|_{2}=\|{\mathbf v}\|_{2}=0$ whence
${\mathbf u}-{\mathbf v}={\mathbf  0}$.

\noindent In either case ${\mathbf u}={\mathbf v}$ as required.

It only remains to prove the triangle inequality.
Observe that, if ${\mathbf v}$ and ${\mathbf w}$ are not linearly
dependent,
\[d({\mathbf u},{\mathbf v})+d({\mathbf v},{\mathbf w})
\geq \|{\mathbf u}-{\mathbf v}\|_{2}
+\|{\mathbf v}\|_{2}+\|{\mathbf w}\|_{2}
\geq \|{\mathbf u}\|_{2}+\|{\mathbf w}\|_{2}\geq d({\mathbf u},{\mathbf w}).\]
By similar reasoning
\[d({\mathbf u},{\mathbf v})+d({\mathbf v},{\mathbf w})
\geq d({\mathbf u},{\mathbf w})\]
if ${\mathbf u}$ and ${\mathbf v}$ are not linearly
dependent. Finally, if ${\mathbf u}$ and ${\mathbf v}$ are linearly
dependent and ${\mathbf v}$ and ${\mathbf w}$ are linearly
dependent, then ${\mathbf u}$ and ${\mathbf w}$ are linearly
dependent so
\[d({\mathbf u},{\mathbf v})+d({\mathbf v},{\mathbf w})
= \|{\mathbf u}-{\mathbf v}\|_{2}
+\|{\mathbf v}-{\mathbf w}\|_{2}
\geq \|{\mathbf u}-{\mathbf w}\|_{2}=d({\mathbf u},{\mathbf w}).\]
Thus the triangle law holds.
\end{proof}

In the long Exercise~\ref{E;hyperbolic} we look at a metric
which plays an important role in complex analysis and
in geometry.







\section{Continuity and open sets for metric spaces}

Some definitions and results transfer essentially unchanged
from classical analysis on ${\mathbb R}$ to metric spaces.
Recall the classical definition of continuity.
\begin{definition}{\bf [Old definition.]}
A function $f:{\mathbb R}\rightarrow{\mathbb R}$
is called continuous if, given $t\in{\mathbb R}$ and $\epsilon>0$,
we can find a $\delta(t,\epsilon)>0$ such that
\[|f(t)-f(s)|<\epsilon\ \text{whenever $|t-s|<\delta(t,\epsilon)$}.\]
\end{definition}
It is not hard to extend this definition to our new, wider context.



It may help you grasp this definition
if you read '$\rho(f(t),f(s))$' as 'the distance from $f(t)$ to $f(s)$
in $Y$'
and '$d(t,s)$' as 'the distance from $t$ to $s$ in $X$'.

\begin{theorem}{\bf [The composition law.]} \label{T;composition metric}
If $(X,d)$ and $(Y,\rho)$
and $(Z,\sigma)$ are metric spaces and $g:X\rightarrow Y$, $f:Y\rightarrow Z$
are continuous, then so is the composition $fg$.
\end{theorem}
\begin{proof}[\bf Proof] Let $\epsilon>0$ be given and let $x\in X$. Since
$f$ is continuous, we can find a $\delta_{1}>0$
(depending on $\epsilon$ and $fg(x)=f(g(x))$ with
\[\sigma(f(g(x)),f(y))<\epsilon\ \text{whenever $\rho(g(x),y)<\delta_{1}$}.\]
Since $g$ is continuous, we can find a $\delta_{2}>0$ such that
\[\rho(g(x),g(t))<\delta_{1}\ \text{whenever $d(x,t)<\delta_{2}$}.\]
We now have
\[\sigma(f(g(x)),f(g(t)))<\epsilon\ \text{whenever $d(x,t)<\delta_{2}$}\]
as required.
\end{proof}




\begin{theorem}\label{E;exercise composition}
Let ${\mathbb R}$ and ${\mathbb R}^{2}$ have their usual (Euclidean) metric.

(i) Suppose that $f:{\mathbb R}\rightarrow{\mathbb R}$ and
$g:{\mathbb R}\rightarrow{\mathbb R}$ are continuous. Show that
the map $(f,g):{\mathbb R}^{2}\rightarrow{\mathbb R}^{2}$
is continuous.

(ii) Show that the map $M:{\mathbb R}^{2}\rightarrow{\mathbb R}$
given by $M(x,y)=xy$ is continuous.

(iii) Use the composition law to show that the map
$m:{\mathbb R}^{2}\rightarrow{\mathbb R}$
given by $m(x,y)=f(x)g(y)$ is continuous.
\end{theorem}
\begin{proof}[\bf Proof] (i) Let $(x,y)\in{\mathbb R}^{2}$.
Given $\epsilon>0$, we can find $\delta_{1}>0$ such that
\[|f(x)-f(s)|<\epsilon/2\ \text{whenever $|x-s|<\delta_{1}$}\]
and
$\delta_{2}>0$ such that
\[|g(y)-g(t)|<\epsilon/2\ \text{whenever $|y-t|<\delta_{2}$}.\]
If we set $\delta=\min(\delta_{1},\delta_{2})$, then
$\|(x,y)-(s,t)\|_{2}<\delta$ implies
\[
|x-s|<\delta\leq\delta_{1}
\ \text{and}\ |y-t|<\delta\leq\delta_{2}
\]
so that
\[|f(x)-f(s)|<\epsilon/2\ \text{and}\ |g(y)-g(t)|<\epsilon/2\]
whence
\begin{align*}
\|(f(x),g(y))-(f(s),g(t))\|_{2}
&\leq \|(f(x),0)-(f(s),0)\|_{2}+\|(0,g(y))-(0,g(t))\|_{2}\\
&=|f(x)-f(s)|+|g(y)-g(t)|<\epsilon
\end{align*}
as required.
\end{proof}


Exercise~\ref{E;exercise composition} may look perverse
at first sight but, in fact, we usually show functions
to be continuous by considering them as compositions
of simpler functions rather than using the definition directly.
Think about
\[x\mapsto \log\left(2+\sin\frac{1}{1+x^{2}}\right).\]
If you are interested, we continue the chain of thought
in Exercise~\ref{E;extend composition}. If you are not interested
or are mildly confused by all this,
just ignore this paragraph.

Just as there are 'well behaved' and 'badly behaved' functions
between spaces
so there are 'well behaved' and 'badly behaved' subsets
of spaces. In classical analysis and analysis on metric spaces
the notion of continuous function is sufficiently wide
to give us a large collection of interesting functions
and sufficiently narrow to ensure reasonable
behaviour\footnote{Sentences like this are not mathematical
statements but many mathematicians find them useful.}.
In introductory analysis we work on ${\mathbb R}$ with
the Euclidean metric and only consider subsets in the
form of intervals. Once we move to ${\mathbb R}^{2}$ with
the Euclidean metric, it becomes clear that there is
no appropriate analogue to intervals. (We want some rectangles
to be well behaved but we also want to talk about discs and
triangles and blobs.)

Cantor identified two particular classes of 'well behaved'
sets. We start with open sets.




Suppose we work in ${\mathbb R}^{2}$ with the Euclidean
metric. If $E$ is an open set then any point ${\mathbf e}$ in $E$
is the centre of a disc of strictly positive radius
all of whose points lie in $E$. If we are sufficiently
short sighted, every point that we can see from ${\mathbf e}$
lies in $E$. This property turns out to be a key to
many proofs in classical analysis (remember that in the
proof of Rolle's theorem it was vital that the maximum
did not lie at an end point) and complex analysis
(where we examine functions analytic \emph{on an open set}).

Here are a couple of simple examples of an open set
and a simple example
of a set which is not open.



%We call $B({\mathbf x},r)$ the open ball
%with centre ${\mathbf x}$ and radius $r$.
%The following result is very important for the course
%but is also very easy to check.






\begin{theorem}\label{T;open not continuous}
Let $X={\mathbb R}$ and $d$ be the discrete metric. Let $Y={\mathbb R}$ and $\rho$ be the usual (Euclidean) metric.

(i) If we define $f:X\rightarrow Y$ by $f(x)=x$, then $f$ is continuous but there exist open sets $U$ in $X$ such that $f(U)$ is not open.

(ii) If we define $g:Y\rightarrow X$ by $g(y)=y$, then $g$ is not continuous
but $g(V)$ is open in $X$ whenever $V$ is open in $Y$.
\end{theorem}

\begin{proof}[\bf Proof] Since every set is open in $X$, we have
$f^{-1}(V)=g(V)$ open for every $V$ in $Y$ and so, in particular,
for every open set. Thus $f$ is continuous.

We observe that $U=\{0\}$ is open in $X$
and $g^{-1}(U)=f(U)=U=\{0\}$
is not open in $Y$. Thus $g$ is not continuous.
\end{proof}

The message of this example is reinforced by the more complicated Exercise~\ref{E;nasty discontinuous}.

Observe that Theorem~\ref{T;metric continuous open} gives a very neat proof of the composition law.

\begin{theorem}[Theorem~\ref{T;composition metric}]\label{New proof composition}
If $(X,d)$ and $(Y,\rho)$
and $(Z,\sigma)$ are metric spaces and $g:X\rightarrow Y$, $f:Y\rightarrow Z$
are continuous,
then so is the composition $fg$.
\end{theorem}
\begin{proof}[\bf New proof] If $U$ is open in $Z$, then, by continuity, $f^{-1}(U)$ is open in $Y$ and so, by continuity, $(fg)^{-1}(U)=g^{-1}\big(f^{-1}(U)\big)$ is open in $X$. Thus $fg$ is continuous.
\end{proof}

This confirms our feeling that the ideas of this chapter
are on the right track.

We finish with an exercise which may be omitted
at first reading but which should be done at some time
as examples of what open sets can look like.




\begin{theorem}\label{T;British rail balls}
Consider ${\mathbb R}^{2}$.
For each of the British rail express and
British rail stopping metrics:-

(i) Describe the open balls. (Consider both large
and small radii.)

(ii) Describe the open sets as well as you can.
(There is a nice description for the British rail express
metric.) Give reasons for your answers.
\end{theorem}
\begin{proof}[\bf Solution] We start with the British rail express metric.
Write
\[B_{E}(\delta)=\{{\mathbf x}\,:\,\|{\mathbf x}\|_{2}<\delta\}\]
for the Euclidean ball centre ${\mathbf  0}$ $[\delta>0]$.
If $0<r<\|{\mathbf x}\|_{2}$, then
\[B({\mathbf x},r)=\{{\mathbf x}\}.\]
If $\|{\mathbf x}\|_{2}>r>0$, then
\[B({\mathbf x},r)=\{{\mathbf x}\}\cup B_{E}(r-\|{\mathbf x}\|).\]
Since open balls are open and the union of open sets is open,
we deduce that every set not containing ${\mathbf  0}$ and every
set containing $B_{E}(\delta)$ for some $\delta>0$ is open.

On the other hand, if $U$ is open and ${\mathbf 0}\in U$ then
$U$ must contain $B_{E}(\delta)$ for some $\delta>0$.
It follows that the collection of sets described
in the last sentence of the previous
paragraph constitute the open sets for the British rail express metric.

We turn now to the stopping metric. We observe that
\[B({\mathbf  0},r)=B_{E}(r)\]
for $r>0$. If ${\mathbf x}\neq{\mathbf  0}$  and $0<r<\|{\mathbf x}\|_{2}$,
then
\[B({\mathbf x},r)=\left\{\lambda\frac{\mathbf x}{{\|{\mathbf x}\|_{2}}}
\,:\,\lambda\in(\|{\mathbf x}\|_{2}-r,\|{\mathbf x}\|_{2}+r)\right\}.\]
If ${\mathbf x}\neq{\mathbf  0}$ and $\|{\mathbf x}\|_{2}>r>0$, then
\[B({\mathbf x},r)=\left\{\lambda\frac{\mathbf x}{{\|{\mathbf x}\|_{2}}}
\,:\,\lambda\in(0,\|{\mathbf x}\|_{2}+r)\right\}
\cup B_{E}(r-\|{\mathbf x}\|).\]
A similar argument to the previous paragraph shows that the open sets are
precisely the unions of sets of the form
\[l({\mathbf e},(a,b))=\{\lambda{\mathbf e}\,:\,\lambda\in(a,b)\}\]
where ${\mathbf e}$ is a unit vector and $0\leq a<b$ and
unions sets of the form $l({\mathbf e},(a,b))$ together with
some  $B_{E}(\delta)$ with $\delta>0$.
\end{proof}






\section{Closed sets for metric spaces}






The following exercise is easy but instructive.
\begin{problem} (i) If $(X,d)$ is any metric space,
then $X$ and $\emptyset$ are both open and closed.

(ii) If we consider ${\mathbb R}$ with the usual metric
and take $b>a$, then $[a,b]$ is closed but not open,
$(a,b)$ is open but not closed and $[a,b)$ is neither
open nor closed.

(iii) If $(X,d)$ is
a metric space with discrete metric $d$, then
all subsets of $X$ are both open and closed.
\end{problem}
It is easy to see why closed sets will be useful in
those parts of analysis which involve taking limits.
The reader will recall theorems in elementary analysis
(for example the boundedness of continuous functions)
which were true for closed intervals but not for
other types of intervals.

Life is made much easier by the very close link between
the notions of closed and open sets given by our next theorem.









\section{Topological spaces}

We now investigate general objects which have the structure described by Theorem~\ref{thm:properties_metric_open} .
\begin{definition}\label{D;topology}
Let $X$ be a set and $\tau$ a collection of subsets of $X$
with the following properties.

(i) The empty set $\emptyset\in \tau$ and the space $X\in\tau$.

(ii) If $U_{\alpha}\in\tau$ for all $\alpha\in A$, then
$\bigcup_{\alpha\in A} U_{\alpha}\in\tau$.

(iii) If $U_{j}\in\tau$ for all $1\leq j\leq n$, then
$\bigcap_{j=1}^{n} U_{j}\in\tau$.

Then we say that $\tau$ is a topology on $X$ and
that $(X,\tau)$ is a topological space.
\end{definition}

\begin{theorem}\label{T;metric topology}
If $(X,d)$ is a metric space, then the collection of open sets forms a topology.
\end{theorem}

\begin{proof}[\bf Proof] This is Theorem~\ref{thm:properties_metric_open} .
\end{proof}
If $(X,d)$ is a metric space we call the collection of
open sets the topology induced by the metric.

If $(X,\tau)$ is a topological space we extend the notion of
open set by calling the members of $\tau$ open sets.
The discussion above ensures what computer scientists call
'downward compatibility'.

\begin{problem} If $(X,d)$ is a metric space with the
discrete metric, show that the induced topology consists
of all the subsets of $X$.
\end{problem}
We call the topology consisting of all subsets of $X$
the discrete topology on $X$.

\begin{problem}\label{E;indiscrete}
If $X$ is a set and $\tau=\{\emptyset,X\}$,
then $\tau$ is a topology.
\end{problem}
We call $\{\emptyset,X\}$ the indiscrete topology on $X$.

\begin{problem} (i) If $F$ is a  finite set
and $(F,d)$ is a metric space show that the induced
topology is the discrete topology.

(ii) If $F$ is a finite set with more than one point,
show that the indiscrete topology is not induced by any
metric.
\end{problem}

You should test any putative theorems on topological spaces
on the discrete topology and the indiscrete topology,
${\mathbb R}^{n}$ with the topology derived from
the Euclidean metric and
$[0,1]$ with the topology derived from
the Euclidean metric.

The following exercise is tedious but instructive (the tediousness is the instruction).


\begin{theorem}\label{T;silly count}
Write ${\mathcal P}(Y)$ for the collection
of subsets of $Y$. If $X$ has three elements, how many
elements does ${\mathcal P}\big({\mathcal P}(X)\big)$
have?

How many topologies are there on $X$?
\end{theorem}
\begin{proof}[\bf Solution]
If $Y$ has $n$ elements ${\mathcal P}(Y)$ has $2^{n}$
elements so ${\mathcal P}\big({\mathcal P}(X)\big)$
has $2^{2^{3}}=2^{8}=256$ elements.

Let $X=\{x,y,z\}$. We set out the types of possible
topologies below.
\begin{center}
\begin{tabular}{c|c}
type&number of this type\\
$\{\emptyset,X\}$&$1$\\
$\{\emptyset,\{x\},X\}$&$3$\\
$\{\emptyset,\{x\},\{y\},\{x,y\},X\}$&$3$\\
${\mathcal P}(X)$&$1$\\
$\{\emptyset,\{x,y\},X\}$&$3$\\
$\{\emptyset,\{x\},\{x,y\},X\}$&$6$\\
$\{\emptyset,\{z\},\{x,y\},X\}$&$3$\\
$\{\emptyset,\{x\},\{z\},\{x,y\},\{x,z\},X\}$&$6$\\
$\{\emptyset,\{x\},\{x,y\},\{x,z\},X\}$&$3$
\end{tabular}
\end{center}

There are that $29$ distinct topologies on $X$.

The moral of this question is that although there are far
fewer topologies than simple collections of subsets
and even fewer different types (non-homeomorphic
topologies in later terminology) there are still
quite a lot even for spaces of three points.
\end{proof}




The proof of Theorem~\ref{T;composition metric}
given on page~\pageref{New proof composition}
carries over unchanged to give the following generalisation.
\begin{theorem}
If $(X,\tau)$ and $(Y,\sigma)$
and $(Z,\mu)$ are topological spaces and
$g:X\rightarrow Y$, $f:Y\rightarrow Z$
are continuous, then so is the composition $fg$.
\end{theorem}

Downward compatibility suggests the definition of a closed
set for a topological space based on
Theorem~\ref{T;closed complements open}.
\begin{definition}\label{D;closed topologically}
Let $(X,\tau)$ be a topological space. A set $F$ in $X$
is said to be closed if its complement is open.
\end{definition}
Theorem~\ref{T;closed complements open} tells us that
if $(X,d)$ is a metric space the notion
of a closed set is the same
whether we consider the metric or the topology derived
from it.

Just as in the metric case, we can
deduce properties of closed sets from properties of open sets
by complementation.In particular, the same proofs as we
gave in the metric case give the following extensions of
Theorems~\ref{T;properties metric closed}
and~\ref{T;metric continuous closed}
\begin{theorem}
If $(X,\tau)$ is a topological space,
then the following statements are true.

(i) The empty set $\emptyset$ and the space $X$ are closed.

(ii) If $F_{\alpha}$ is closed for all $\alpha\in A$, then
$\bigcap_{\alpha\in A} F_{\alpha}$ is closed. (In other words,
the intersection of closed sets is closed.)

(iii) If $F_{j}$ is closed for all $1\leq j\leq n$, then
$\bigcup_{j=1}^{n} F_{j}$ is closed.
\end{theorem}
\begin{theorem}
Let $(X,\tau)$ and $(Y,\sigma)$
be topological spaces.
A function $f:X\rightarrow Y$ is continuous if and only
if $f^{-1}(F)$ is closed in $X$ whenever $F$ is closed in $Y$.
\end{theorem}















\section{More on topological structures}

Two groups are the same for the purposes of group theory if they are (group) isomorphic. Two vector spaces are the same for the purposes of linear algebra if they are (vector space) isomorphic. When
are two topological spaces $(X,\tau)$ and $(Y,\sigma)$ the same for the purposes of topology? In other words, when does there exist a bijection between $X$ and $Y$ in which open sets correspond to
open sets, and the grammar of topology (things like union and inclusion) is preserved? A little reflection shows that the next definition provides the answer we want. (Exercise~\ref{E;inverse} is
vaguely relevant.)
\begin{definition}\label{D;homeomorphisms}
We say that two topological spaces
$(X,\tau)$ and $(Y,\sigma)$ are homeomorphic if there
exists a bijection
$\theta:X\rightarrow Y$ such that $\theta$ and $\theta^{-1}$
are continuous. We call $\theta$ a homeomorphism.
\end{definition}
The following exercise acts as useful revision of concepts
learnt last year.
\begin{problem} Show that homeomorphism is an equivalence
relation on topological spaces.
\end{problem}
Homeomorphism only implies
equivalence\label{R;topological properties}
\emph{for the purposes of topology}.
To emphasise this, we introduce a couple of related ideas which are
fundamental to analysis on metric spaces but which will
only be referred to here in this course.
\begin{definition}\label{D;completeness}
(i) If $(X,d)$ is a metric space,
we say that a sequence $x_{n}$ in $X$ is Cauchy if, given
$\epsilon>0$, we can find an $N_{0}(\epsilon)$ with
\[d(x_{n},x_{m})<\epsilon\ \text{whenever $n,\,m\geq N_{0}(\epsilon)$}.\]

(ii) We say that a metric space $(X,d)$ is complete if every Cauchy sequence converges.
\end{definition}

\begin{theorem}\label{T;Cauchy not topological}
Let $X={\mathbb R}$ and let $d$ be the usual metric on ${\mathbb R}$.
Let $Y=(0,1)$ (the open interval with end points $0$ and $1$)
and let $\rho$ be the usual metric on $(0,1)$. Then
$(X,d)$ and $(Y,\rho)$ are homeomorphic as topological spaces
but $(X,d)$ is complete and $(Y,\rho)$ is not.
\end{theorem}
\begin{proof}[\bf Proof] We know from first year analysis that
$f(x)=\tan(\pi(y-1/2))$ is a bijective function $f:Y\rightarrow X$
which is continuous with continuous inverse. (Recall that a strictly increasing
continuous function has continuous inverse.)  Thus
$(X,d)$ and $(Y,\rho)$ are homeomorphic. We know that
$(X,d)$ is complete by the general principle of analysis.

However $1/n$ is a Cauchy sequence in $Y$ with no limit in $Y$. (If $y\in (0,1)$, then there exists an $N$ with $y>N^{-1}$. If $m\geq 2N$, then $|1/m-y|\geq 1/2N$ so $1/n\nrightarrow y$.)
\end{proof}

We say that 'completeness is not a topological property'.

In group theory we usually prove that two groups are isomorphic
by constructing an explicit isomorphism and that two groups
are not isomorphic by finding a group property
exhibited by one but not by the other. Similarly in
topology we usually prove that two topological spaces
are homeomorphic
by constructing an explicit homeomorphism and that two
topological spaces
are not homeomorphic by finding a topological property
exhibited by one but not by the other. Later in this course
we will meet some topological properties like being
Hausdorff and compactness and you will be able to
tackle Exercise~\ref{E;homeomorphic, non-homeomorphic}.

We also want to be able to construct new topological
spaces from old. To do this we we make use of a simple
but useful lemma.

\begin{lemma}\label{L;coarsest topology}
Let $X$ be a space and let ${\mathcal H}$ be a
collection of subsets of $X$. Then there exists a unique topology
$\tau_{{\mathcal H}}$ such that

(i) $\tau_{{\mathcal H}}\supseteq{\mathcal H}$, and

(ii) if $\tau$ is a topology with $\tau\supseteq {\mathcal H}$,
then $\tau\supseteq \tau_{{\mathcal H}}$.
\end{lemma}
\begin{proof}[\bf Proof] The proof follows a standard pattern which is
worth learning.

\noindent\emph{Uniqueness} Suppose that $\sigma$ and $\sigma'$ are topologies
such that

(i) $\sigma\supseteq{\mathcal H}$,

(ii) if $\tau$ is a topology with $\tau\supseteq {\mathcal H}$,
then $\tau\supseteq \sigma$,

(i)$'$ $\sigma'\supseteq{\mathcal H}$,

(ii)$'$ if $\tau$ is a topology with $\tau\supseteq {\mathcal H}$,
then $\tau\supseteq \sigma'$.

\noindent By (i) and (ii)$'$, we have $\sigma\supseteq \sigma'$
and by (i)$'$ and (ii), we have $\sigma'\supseteq \sigma$.
Thus $\sigma=\sigma'$.

\noindent\emph{Existence} Let $T$ be the set of topologies
$\tau$ with $\tau\supseteq {\mathcal H}$. Since the discrete
topology contains ${\mathcal H}$, $T$ is non-empty.
Set
\[\tau_{\mathcal H}=\bigcap_{\tau\in T}\tau.\]
By construction $\tau_{\mathcal H}\supseteq{\mathcal H}$ and
$\tau\supseteq \tau_{\mathcal H}$ whenever $\tau\in T$.
Thus we need only show that $\tau_{\mathcal H}$ is a topology
and this we now do.

(a) ${\emptyset},\,X\in\tau$ for all $\tau\in T$ so ${\emptyset},\,X\in\tau_{\mathcal H}$.

(b) If $U_{\alpha}\in \tau_{\mathcal H}$ then $U_{\alpha}\in\tau$ for all $\alpha\in A$ and so  $\bigcup_{\alpha\in A}U_{\alpha}\in\tau$ for all $\tau\in T$ whence $\bigcup_{\alpha\in A}U_{\alpha}\in \tau_{\mathcal H}$.

(c) If $U_{j}\in \tau_{\mathcal H}$ then $U_{j}\in\tau$ for all $1\leq j\leq n$ and so $\bigcap_{j=1}^{n}U_{j}\in\tau$ for all $\tau\in T$ whence
$\bigcap_{j=1}^{n} U_{j}\in \tau_{\mathcal H}$.

Thus $\tau_{\mathcal H}$ is a topology, as required.
\end{proof}


We call $\tau_{\mathcal H}$ the smallest (or coarsest)  topology
containing ${\mathcal H}$.

\begin{lemma} Suppose that $A$ is non-empty and the
spaces $(X_{\alpha},\tau_{\alpha})$
are topological spaces and we have maps
$f_{\alpha}:X\rightarrow X_{\alpha}$ $[\alpha\in A]$.
Then there is a smallest topology $\tau$ on $X$ for which
the maps $f_{\alpha}$ are continuous.
\end{lemma}

\begin{proof} A topology $\sigma$ on $X$ makes all the $f_{\alpha}$
continuous if and only if it contains
\[{\mathcal H}=\{f_{\alpha}^{-1}(U)\,:\,
U\in\tau_{\alpha},\ \alpha\in A\}.\]
Now apply Lemma~\ref{L;coarsest topology}.
\end{proof}

Recall that, if $Y\subseteq X$, then the inclusion map $j:Y\rightarrow X$
is defined by $j(y)=y$ for all $y\in Y$.

\begin{definition}\label{D;subspace topology}
If $(X,\tau)$ is a topological space
and $Y\subseteq X$ then the subspace topology $\tau_{Y}$ on $Y$
induced by $\tau$ is the smallest topology on $Y$ for which the
inclusion map is continuous.
\end{definition}


\begin{lemma}\label{L;subspace topology}
If $(X,\tau)$ is a topological space
and $Y\subseteq X$ then the subspace topology $\tau_{Y}$ on $Y$
is the collection of sets $Y\cap U$ with $U\in\tau$.
\end{lemma}
\begin{proof}[\bf Proof] Let $j:Y\rightarrow X$ be the inclusion
map given by $j(y)=y$ for all $y\in Y$.
Write
\[\sigma=\{Y\cap U\,:\,U\in\tau\}.\]
Since $Y\cap U=j^{-1}(U)$ we know that $\tau_{Y}$
is the smallest topology containing $\sigma$
and that the result will follow if we show that $\sigma$ is a topology
on $Y$. The following observations show this and complete
the proof.

(a) ${\emptyset}=Y\cap{\emptyset}$ and $Y=Y\cap X$.

(b) $\bigcup_{\alpha\in A}(Y\cap U_{\alpha})
=Y\cap\bigcup_{\alpha\in A}U_{\alpha}$.

(c) $\bigcap_{j=1}^{n}(Y\cap U_{j})=Y\cap\bigcap_{j=1}^{n}U_{j}$.
\end{proof}



\begin{problem}
(i) If $(X,\tau)$ is a topological space
and $Y\subseteq X$ is open,
show that the subspace topology $\tau_{Y}$ on $Y$
is the collection of sets $U\in\tau$ with $U\subseteq Y$.

(ii) Consider ${\mathbb R}$ with the usual topology $\tau$
(that is, the topology derived from the Euclidean metric).
If $Y=[0,1]$, show that $[0,1/2)\in\tau_{Y}$ but $[0,1/2)\notin\tau$.
\end{problem}
\begin{problem}
Let $(X,d)$ be a metric space, $Y$ a subset
of $X$ and $d_{Y}$ the metric $d$ restricted to $Y$
(formally, $d_{Y}:Y^{2}\rightarrow{\mathbb R}$ is given by
$d_{Y}(x,y)=d(x,y)$ for $x,\,y\in Y$). Then if we
give $X$ the topology induced by $d$, the subspace
topology on $Y$ is identical with the topology induced by $d_{Y}$.

\noindent[This is an exercise in stating the obvious.]
\end{problem}
Next recall that if $X$ and $Y$ are sets
the projection maps $\pi_{X}:X\times Y\rightarrow X$ and
$\pi_{Y}:X\times Y\rightarrow Y$
are given by
\begin{align*}
\pi_{X}(x,y)&=x,\\
\pi_{Y}(x,y)&=y.
\end{align*}

\begin{definition}\label{D;product topology}
If $(X,\tau)$ and $(Y,\sigma)$
are topological spaces, then the product topology
$\mu$ on $X\times Y$ is
the smallest topology on $X\times Y$ for which the
projection maps $\pi_{X}$ and $\pi_{Y}$ are continuous.
\end{definition}



\begin{lemma}\label{L;product topology}
Let $(X,\tau)$ and $(Y,\sigma)$
be topological spaces and $\lambda$ the product topology
on $X\times Y$.
Then $O\in\lambda$ if and only if, given $(x,y)\in O$,
we can find $U\in \tau$ and $V\in \sigma$ such that
\[(x,y)\in U\times V\subseteq O.\]
\end{lemma}
\begin{proof}[\bf Proof] Let $\mu$ be the collection of subsets
$E$ such that, given $(x,y)\in E$,
we can find $U\in \tau$ and $V\in \sigma$ with
\[(x,y)\in U\times V\subseteq E.\]

If $U\in \tau$ then, since $\pi_{X}$ is continuous
$U\times Y=\pi_{X}^{-1}(U)\in\lambda$. Similarly,
if $V\in\sigma$ then $X\times V\in\lambda$. Thus
\[U\times V=U\times Y\cap X\times V\in\lambda.\]
If $E\in\mu$ then, given $(x,y)\in E$, we can find
$U_{(x,y)}\in \tau$ and $V_{(x,y)}\in \sigma$ such that
\[(x,y)\in U_{(x,y)}\times V_{(x,y)}\subseteq E,.\]
We observe that
\[E\subseteq \bigcup_{(x,y)\in E}U_{(x,y)}\times V_{(x,y)}\subseteq E\]
so $E=\bigcup_{(x,y)\in E}U_{(x,y)}\times V_{(x,y)}$ and, since the
union of open sets is open, $E\in\lambda$. Thus $\mu\subseteq\lambda$.

It is easy to check that $\mu$ is a topology as follows.

(a) $\emptyset\in\mu$ vacuously. If $(x,y)\in X\times Y$,
then $X\in\tau$, $Y\in\sigma$ and
$(x,y)\in X\times Y\subseteq X\times Y$. Thus $X\times Y\in\mu$.

(b) Suppose $E_{\alpha}\in\mu$ for all $\alpha\in A$.
If $(x,y)\in\bigcup_{\alpha\in A}E_{\alpha}$,
then $(x,y)\in E_{\beta}$ for some $\beta\in A$.
We can find $U\in \tau$ and $V\in \sigma$ such that
\[(x,y)\in U\times V\subseteq E_{\beta}\]
and so
\[(x,y)\in U\times V\subseteq\bigcup_{\alpha\in A}E_{\alpha}.\]
Thus $\bigcup_{\alpha\in A}E_{\alpha}\in\mu$.

(c) Suppose $E_{j}\in\mu$ for all $1\leq j\leq n$.
If $(x,y)\in\bigcap_{j=1}^{n}E_{j}$,
then $(x,y)\in E_{j}$ for all $1\leq j\leq n$.
We can find $U_{j}\in \tau$ and $V_{j}\in \sigma$ such that
\[(x,y)\in U_{j}\times V_{j}\subseteq E_{j}\]
and so
\[(x,y)\in \bigcap_{j=1}^{n} U_{j}\times \bigcap_{j=1}^{n} V_{j}
\subseteq\bigcap_{j=1}^{n}E_{j}.\]
Since $\bigcap_{j=1}^{n} U_{j}\in\tau$
and $\bigcap_{j=1}^{n} V_{j}\in\sigma$,
we have shown that $\bigcap_{j=1}^{n}E_{j}\in\mu$.

Finally, we observe that, if $U\in\tau$, then
\[\pi_{X}^{-1}(U)=U\times Y\]
and $(x,y)\in U\times Y\subseteq \pi_{X}^{-1}(U)$
with $U\in\tau$, $Y\in\sigma$,
so $\pi_{X}^{-1}(U)\in\mu$. Thus $\pi_{X}:X\times Y\rightarrow X$
is continuous if we give $X\times Y$ the topology $\mu$. A similar
result holds for $\pi_{Y}$ so, by the minimality of $\lambda$,
$\mu=\lambda$.
\end{proof}

The next remark is useful for proving results like those in Exercise~\ref{E;metric product}.



\begin{theorem}\label{L;same topology}
Let $\tau_{1}$ and $\tau_{2}$ be two topologies on the same space $X$.

(i) We have $\tau_{1}\subseteq\tau_{2}$ if and only if, given $x\in U\in\tau_{1}$, we can find $V\in\tau_{2}$ such that $x\in V\subseteq U$.

(ii)We have $\tau_{1}=\tau_{2}$ if and only if, given $x\in U\in\tau_{1}$, we can find $V\in\tau_{2}$ such that $x\in V\subseteq U$ and, given $x\in U\in\tau_{2}$, we can find $V\in\tau_{1}$ such that $x\in V\subseteq U$.
\end{theorem}

\begin{proof}[\bf Proof] (i) If $\tau_{1}\subseteq\tau_{2}$
and $x\in U\in\tau_{1}$, then setting $V=U$ we
automatically
have $V\in\tau_{2}$ and $x\in V\subseteq U$.

Conversely, suppose that,
given $x\in U\in\tau_{1}$, we can find
$V\in\tau_{2}$ such that $x\in V\subseteq U$.
Then, if $U\in\tau_{1}$ is fixed, we can find
$V_{x}\in\tau_{2}$ such that $x\in V_{x}\subseteq U$
for each $x\in U$.

Now
\[
U\subseteq \bigcup_{x\in U}V_{x}\subseteq U
\]
so $U=\bigcup_{x\in U}V_{x}$ and, since the union of open sets is open, $U\in\tau_{2}$. Thus $\tau_{1}\subseteq \tau_{2}$.

(ii) Observe that $\tau_{1}=\tau_{2}$ if and only if $\tau_{1}\subseteq\tau_{2}$ and $\tau_{2}\subseteq\tau_{1}$.
\end{proof}



\begin{problem}\label{E;metric product}
Let $(X_{1},d_{1})$ and $(X_{2},d_{2})$ be metric
spaces. Let $\tau$ be the product topology on $X_{1}\times X_{2}$
where $X_{j}$ is given the topology induced by $d_{j}$ $[j=1,2]$.

Define $\rho_{k}:(X_{1}\times X_{2})^{2}\rightarrow {\mathbb R}$
by
\begin{align*}
\rho_{1}((x,y),(u,v))&=d_{1}(x,u),\\
\rho_{2}((x,y),(u,v))&=d_{1}(x,u)+d_{2}(y,v),\\
\rho_{3}((x,y),(u,v))&=\max(d_{1}(x,u),d_{2}(y,v)),\\
\rho_{4}((x,y),(u,v))&=(d_{1}(x,u)^{2}+d_{2}(y,v)^{2})^{1/2}.
\end{align*}

Establish that $\rho_{1}$ is not a metric
and that $\rho_{2}$, $\rho_{3}$ and $\rho_{4}$ are.
Show that each of the $\rho_{j}$ with $2\leq j\leq 4$ induce the
product topology $\tau$ on $X_{1}\times X_{2}$.
\end{problem}
It is easy to extend our definitions and results to
any finite product of topological spaces. In fact,
it is not difficult to extend our definition
to the product of an infinite collection
of topological spaces but I feel
that it is important for the reader to concentrate on
first thoroughly
understanding the finite product case and I have relegated the
infinite case to an exercise (Exercise~\ref{E;Infinite product}).

We conclude this chapter by looking briefly at the
quotient topology. This will not play a major part in
our course and the reader should not worry too much about it.

If $\sim$ is an equivalence relation on a
set $X$
then we know from previous courses that it gives
rise to equivalence classes
\[[x]=\{y\in X\,:\,y\sim x\}.\]
There is a natural map $q$ from $X$ to the space $X/\negthinspace\sim$
of equivalence classes given by $q(x)=[x]$. When
we defined the subspace and product topologies
we used natural maps from the new spaces to the
old spaces. Here we have a natural map from
the old space to the new, so our definition
has to take a different form.

Since intersection and union behave well under
inverse mappings it is
easy to check the following statement.
\begin{lemma}\label{L;quotient lemma}
Let $(X,\tau)$ be a topological space
and $\sim$ an equivalence relation on $X$.
Write $q$ for the map from $X$ to the
quotient space $X/\negthinspace\sim$
given by $q(x)=[x]$. Then
\[\sigma=\{U\subseteq X/\negthinspace\sim\,:\,q^{-1}(U)\in\tau\}\]
is a topology.
\end{lemma}
\begin{definition} Under the assumptions and with the notation
of Lemma~\ref{L;quotient lemma} we call $\sigma$ the quotient
topology on $X/\negthinspace\sim$.
\end{definition}
The following is just a restatement of the definition.
\begin{lemma} Under the assumptions and with the notation
of Lemma~\ref{L;quotient lemma}, the quotient topology
consists of the sets $U$ such that
\[\bigcup_{[x]\in U}[x]\in\tau.\]
\end{lemma}
Later we shall give an example (Exercise circle as quotient)
of a nice quotient topology.
Exercise~\ref{E;nasty line}, which requires
ideas from later in the course, is an example of really
nasty  quotient topology.

In general, the quotient topology can be extremely
unpleasant (basically because equivalence relations
form a very wide class) and although nice equivalence
relations sometimes give useful quotient topologies
you should always think before using one.
Exercises~\ref{E;quotient remarks} and~\ref{E;quotient non Hausdorff}
give some further information.

















\section{Hausdorff spaces}\label{S;Hausdorff spaces}
When we work in a metric space we make
repeated use of the fact that if $d(x,y)=0$ then $x=y$.
The metric is 'powerful enough to separate points'.
The indiscrete topology, on the other hand, clearly
can not separate points.

When Hausdorff first crystallised the modern idea
of a topological space he included an extra condition
to ensure 'separation of points'. It was later
discovered that topologies without this extra condition
could be useful so it is now considered separately.

\begin{definition} A topological space $(X,\tau)$ is called
Hausdorff if, whenever $x,\,y\in X$ and $x\neq y$, we can
find $U,\,V\in\tau$ such that $x\in U$, $y\in V$ and
$U\cap V=\emptyset$.
\end{definition}
In the English educational system it is traditional
to draw $U$ and $V$ as little huts containing $x$ and $y$
and to say that $x$ and $y$ are 'housed off from each other'.

The next exercise requires a one line answer but you should
write that line down.
\begin{problem} Show that,
if $(X,d)$ is a metric space, then the
derived topology is Hausdorff.
\end{problem}

Although we defer the discussion of neighbourhoods
in general to towards the end of the course, it is natural
to introduce the following locution here.
\begin{definition}\label{D;open neighbourhood}
If $(X,\tau)$ is a topological space and $x\in U\in\tau$,
we call $U$ an open neighbourhood of $x$.
\end{definition}



\begin{theorem}\label{T;open via neighbourhood}
If $(X,\tau)$ is a topological space, then
a subset $A$ of $X$ is open if and only if every
point of $A$ has an open neighbourhood $U\subseteq A$.
\end{theorem}
\begin{proof}[\bf Proof]
If $A$ is open, then $A$ is an open neighbourhood of every $x\in A$.

Conversely, suppose that every $x\in A$ has an open neighbourhood
$U_{x}$ lying entirely within $A$. Then
\[A\subseteq \bigcup_{x\in A} U_{x}\subseteq A\]
so $A= \bigcup_{x\in A} U_{x}$. Thus $A$ is the union of open sets
and so open.
\end{proof}


\begin{theorem}\label{L;Hausdorff point}
If $(X,\tau)$ is a Hausdorff space, then the one
point sets $\{x\}$ are closed.
\end{theorem}
\begin{proof}[\bf Proof] We must show that $A=X\setminus\{x\}$ is open.
But, if $y\in A$ then $y\neq x$ so by, the Hausdorff condition,
we can find $U,\,V\in\tau$ such that $x\in U$, $y\in V$
and $U\cap V=\emptyset$. We see that $y\in V\subseteq A$,
so every point of $A$ has an open neighbourhood
lying entirely within $A$. Thus $A$ is open.
\end{proof}


The following exercise shows that the converse
to Lemma~\ref{L;Hausdorff point} is false
and that, if we are to acquire any intuition about
topological spaces, we will need to study a wide range
of examples.


\begin{theorem}\label{T;finite complement}
Let $X$ be infinite (we could take $X={\mathbb Z}$ or $X={\mathbb R}$). We say that a subset $E$ of $X$ lies in $\tau$ if either $E=\emptyset$ or $X\setminus E$ is finite. Show that $\tau$ is a topology and that every one point set $\{x\}$ is closed but that $(X,\tau)$ is not Hausdorff.

What happens if $X$ is finite?
\end{theorem}

\begin{proof}[\bf Proof] (a) We are told that $\emptyset\in \tau$.
Since $X\setminus X=\emptyset$, $X\in\tau$.

(b) If $U_{\alpha}\in\tau$ for all $\alpha\in A$ then,
either $U_{\alpha}=\emptyset$ for all $\alpha\in A$
so $\bigcup_{\alpha\in A}U_{\alpha}=\emptyset\in\tau$
or we can find a $\beta\in A$ such that
$X\setminus U_{\beta}$ is finite.
In the second case we observe that
\[X\setminus\bigcup_{\alpha\in A}U_{\alpha}
\subseteq X\setminus U_{\beta},\]so
$X\setminus\bigcup_{\alpha\in A}U_{\alpha}$
is finite and
$\bigcup_{\alpha\in A}U_{\alpha}\in\tau$

(c) If $U_{j}\in\tau$ for all $1\leq j\leq n$ then,
either $U_{k}=\emptyset$ for some $1\leq k\leq n$
so $\bigcap_{j=1}^{n}U_{j}=\emptyset\in\tau$
or $X\setminus U_{j}$ is finite for all $1\leq j\leq n$.
In the second case then, since
\[X\setminus \bigcap_{j=1}^{n}U_{j}
=\bigcup_{j=1}^{n}(X\setminus U_{j}),\]
it follows that
$X\setminus \bigcap_{j=1}^{n}U_{j}$ is finite
and so $\bigcap_{j=1}^{n}U_{j}\in\tau$.

Thus $\tau$ is a topology.

Since $\{x\}$ is finite $X\setminus\{x\}$ is open
and so $\{x\}$ is closed.

Suppose that $x \neq y$ and $x\in U\in\tau$,
$y\in V\in \tau$. Then $U,\,V\neq\emptyset$
so $X\setminus U$ and $X\setminus V$ is finite.
It follows that
\[X\setminus U\cap V=(X\setminus U)\cup(X\setminus V)\]
is finite, and so, since $X$ is infinite, $U\cap V\neq\emptyset$.
Thus $\tau$ is not Hausdorff.

If $X$ is finite then $\tau$ is the discrete metric
which is Hausdorff.
\end{proof}


It is easy to give examples of topologies which are not
derived from metrics. It is somewhat harder to give examples
of Hausdorff topologies which are not derived from metrics.
An example is given in Exercise~\ref{E;Hausdorff not metric}.

The next two lemmas are very useful.

\begin{lemma}\label{L;inherit Hausdoff subspace}
If $(X,\tau)$ is a Hausdorff topological space
and $Y\subseteq X$, then $Y$ with the subspace topology
is also Hausdorff.
\end{lemma}
\begin{proof}[\bf Proof] Write $\tau_{Y}$ for the subspace topology.
If $x,\,y\in Y$ and $x\neq y$, then $x,\,y\in X$ and $x\neq y$ so we can find $U,\,V\in\tau$ with $x\in U$, $y\in V$ and $U\cap V=\emptyset$. Set $\tilde{U}=U\cap Y$ and $\tilde{V}=V\cap Y$. Then $\tilde{U},\,\tilde{V}\in\tau_{Y}$ $x\in \tilde{U}$, $y\in \tilde{V}$ and $\tilde{U}\cap\tilde{V}=\emptyset$.
\end{proof}



\begin{lemma}\label{L;inherit Hausdoff product}
If $(X,\tau)$ and $(Y,\sigma)$ are Hausdorff topological spaces,
then $X\times Y$ with the product topology
is also Hausdorff.
\end{lemma}
\begin{proof}[\bf Proof] Suppose $(x_{1},y_{1}),\,(x_{2},y_{2})$
and $(x_{1},y_{1})\neq(x_{2},y_{2})$. Then we know that at least
one of the statements $x_{1}\neq x_{2}$ and $y_{1}\neq y_{2}$
is true\footnote{But not necessarily both. This is the traditional
silly mistake.}. Without loss of generality we may suppose
$x_{1}\neq x_{2}$. Since $(X,\tau)$ is Hausdorff we can find
$U_{1},\,U_{2}$ disjoint open neighbourhoods of $x_{1}$ and $x_{2}$.
We observe that  $U_{1}\times Y$ and $U_{2}\times Y$
are disjoint open neighbourhoods of $(x_{1},y_{1})$
and $(x_{2},y_{2})$ so we are done.
\end{proof}


Exercise~\ref{E;quotient non Hausdorff} shows that, even when
the original topology is Hausdorff, the resulting quotient
topology need not be.
















\section{Compactness} Halmos says somewhere that if
an idea is used once it is a trick, if used twice it
is a method, if used three times a theorem but if used
four times it becomes an axiom.

Several important theorems in analysis
hold for closed bounded intervals. Heine
used a particular idea to prove one of
these. Borel isolated the idea as a theorem
(the Heine-Borel theorem), essentially
Theorem~\ref{T;Heine-Borel} below.
Many treatments of analysis (for example,
Hardy's \emph{Pure Mathematics}) use the
Heine-Borel theorem as a basic tool.
The notion of compactness represents the last stage in
in the Halmos progression.

\begin{definition}\label{D;compact}
A topological space $(X,\tau)$
is called compact if, whenever we have a collection
$U_{\alpha}$ of open sets $[\alpha\in A]$
with $\bigcup_{\alpha\in A}U_{\alpha}=X$, we can find a
finite subcollection $U_{\alpha(1)}$,
$U_{\alpha(2)}$, \dots, $U_{\alpha(n)}$
with $\alpha(j)\in A$ $[1\leq j\leq n]$
such that $\bigcup_{j=1}^{n}U_{\alpha(j)}=X$.
\end{definition}
\begin{definition} If $(X,\tau)$ is a topological space,
then a subset $Y$ is called compact if the subspace topology
on $Y$ is compact.
\end{definition}
The reader should have no difficulty in combining
these two definitions to come up with the
following restatement,
\begin{lemma}\label{L;compact equivalence} If $(X,\tau)$
is a topological space,
then a subset $Y$ is compact if, whenever we have a collection
$U_{\alpha}$ of open sets $[\alpha\in A]$
with $\bigcup_{\alpha}U_{\alpha}\supseteq Y$, we can find a
finite subcollection $U_{\alpha(1)}$,
$U_{\alpha(2)}$, \dots, $U_{\alpha(n)}$
with $\alpha(j)\in A$ $[1\leq j\leq n]$
such that $\bigcup_{j=1}^{n}U_{\alpha(j)}\supseteq Y$.
\end{lemma}
In other words, 'a set is compact if any cover by
open sets has a finite subcover'.

The reader is warned that compactness is a subtle
property which requires time and energy
to master\footnote{My generation only reached compactness
after a long exposure to the classical Heine--Borel
theorem.}.
(At the simplest level, a substantial minority
of examinees fail to get the definition correct.)
Up to this point most of the proofs in this course
have been simple deductions from
definitions. Several of our theorems
on compactness go much deeper and have quite intricate proofs.

Here are some simple examples of compactness and non-compactness.

\begin{theorem}\label{T;simple compact examples}
(i) Show that, if $X$ is finite, every
topology on $X$ is compact.

(ii) Show that the discrete topology on a set $X$ is
compact if and only if $X$ is finite.

(iii) Show that the indiscrete topology is always compact.

(iv) Show that the topology described in
Exercise finite complement is compact.

(v) Let $X$ be uncountable (we could take
$X={\mathbb R}$).
We say that a subset $A$ of $X$ lies in $\tau$ if
either $A=\emptyset$ or $X\setminus A$ is countable.
Show that $\tau$ is a topology
but that $(X,\tau)$
is not compact.
\end{theorem}

\begin{proof}[\bf Solution]
(iv) If $X=\emptyset$ there is nothing to prove.
If not, let $U_{\alpha}$ $[\alpha\in A]$ be an open cover.
Since $X\neq\emptyset$ we can choose a $\beta\in A$
such that $U_{\beta}\neq\emptyset$ and so
$U_{\beta}=X\setminus F$ where $F$ is a finite set.
For each $x\in F$ we know that $x\in X=\bigcup_{\alpha\in A}U_{\alpha}$
so there exists an $\alpha(x)\in A$ with $x\in U_{\alpha(x)}$.
We have
\[U_{\beta}\cup\bigcup_{x\in F}U_{\alpha(x)}=X,\]
giving us the desired open cover.

(v) I leave it the reader to show that $\tau$ is a topology.
Let $x_{1}$, $x_{2}$, \dots, be distinct points of $X$.
Let
\[U=X\setminus\{x_{j}\,:\,1\leq j\}\]
and $U_{k}=U\cup\{x_{k}\}$. Then $U_{k}\in\tau$ $[k\geq 1]$
and $\bigcup_{k\geq 1}U_{k}=X$.
Now suppose $k(1)$, $k(2)$,\dots, $k(N)$ given.
If $m=\max_{1\leq r\leq N}k(r)$, then
\[x_{m+1}\notin\bigcup_{r=1}^{N}U_{k(r)}\]
so there is no finite subcover.
\end{proof}

We now come to our first major theorem.


\begin{theorem}\label{T;Heine-Borel}
{\bf Heine-Borel Theorem\index{Heine-Borel Theorem}}
Let ${\mathbb R}$ be given its usual (Euclidean) topology. Then the closed bounded interval $[a,b]$ is compact.
\end{theorem}

\begin{proof}[\bf Proof] Suppose that ${\mathcal C}$
is an open cover of $[a,b]$ (i.e. the elements of ${\mathcal C}$
are open sets and $\bigcup_{U\in{\mathcal C}}U\supseteq[a,b]$).
If  ${\mathcal C}_{1}$
is a finite subcover of $[a,c]$
and ${\mathcal C}_{2}$
is a finite subcover of $[c,b]$
then ${\mathcal C}_{1}\cup{\mathcal C}_{2}$
is a finite subcover of $[a,b]$.

Suppose now that $[a,b]$ has no finite subcover using ${\mathcal C}$.
Set $a_{0}=a$, $b_{0}=b$, and $c_{0}=(a_{0}+b_{0})/2$.
By the first paragraph at least one of $[a_{0},c_{0}]$
and $[c_{0},b_{0}]$ has no finite subcover using ${\mathcal C}$.
If $[a_{0},c_{0}]$ has no finite subcover, set $a_{1}=a_{0}$,
$b_{1}=c_{0}$. Otherwise, set $a_{1}=c_{0}$,
$b_{1}=b_{0}$. In either case, we know that

(i) $a=a_{0}\leq a_{1}\leq b_{1}\leq b_{0}=b$,

(ii) if ${\mathcal F}$ is a finite subset of ${\mathcal C}$,
then $\bigcup_{U\in{\mathcal F}}U\not\supseteq [a_{1},b_{1}]$,

(iii) $b_{1}-a_{1}=(b-a)/2$.

Proceeding inductively, we obtain

(i)$_{n}$ $a\leq a_{n-1}\leq a_{n}\leq b_{n}\leq b_{n-1}\leq b$.

(ii)$_{n}$ If ${\mathcal F}$ is a finite subset of ${\mathcal C}$,
then $\bigcup_{U\in{\mathcal F}}U\not\supseteq [a_{n},b_{n}]$.

(iii)$_{n}$ $b_{n}-a_{n}=2^{-n}(b-a)$.

The $a_{n}$ form an increasing sequence bounded above by $b$, so,
by the fundamental axiom of analysis, $a_{n}\rightarrow A$
for some $A \leq b$. Similarly $b_{n}\rightarrow B$ for some
$B\geq a$. Since $b_{n}-a_{n}\rightarrow 0$, $A=B=x$, say,
for some $x\in [a,b]$. Since $x\in [a,b]$ and
$\bigcup_{U\in{\mathcal C}}U\supseteq[a,b]$
we can find a $V\in {\mathcal C}$
with $x\in V$. Since $V$ is open in the Euclidean metric,
we can find a $\delta>0$
such that $(x-\delta,x+\delta)\subseteq V$. Since $a_{n},\,b_{n}\rightarrow x$
we can find an $N$ such that $|x-a_{N}|,\,|x-b_{N}|<\delta$ and so
\[[a_{N},b_{N}]\subseteq (x-\delta,x+\delta)\subseteq V\]
contradicting (ii)$_{N}$. (Just take ${\mathcal F}=\{V\}$.)

The theorem follows by reductio ad absurdum.
\end{proof}


Lemma~\ref{L;compact equivalence} gives the following equivalent statement.

\begin{theorem}\label{T;closed interval compact}
Let $[a,b]$ be given its usual topology
(that is to say the topology derived
from the usual Euclidean metric).
Then the derived topology is compact.
\end{theorem}

We now have a couple of very useful results.


\begin{theorem}\label{T;closed subset compact}
A closed subset of a compact set is compact.
[More precisely, if $E$ is compact and $F$ closed
in a given topology, then, if $F\subseteq E$, it follows that
$F$ is compact.]
\end{theorem}
\begin{proof}[\bf Proof] Suppose $(X,\tau)$ is a topological space,
$E$ is a compact set in $X$ and $F$ is a closed subset of $E$.
If $U_{\alpha}\in\tau$ $[\alpha\in A]$ and
$\bigcup_{\alpha\in A}U_{\alpha}\supseteq F$, then
$X\setminus F\in\tau$ and
\[(X\setminus F)\cup\bigcup_{\alpha\in A}U_{\alpha}=X\supseteq E.\]
By compactness, we can find $\alpha(j)\in A$
$[1\leq j\leq n]$ such that
\[(X\setminus F)\cup\bigcup_{j=1}^{n}U_{\alpha(j)}\supseteq E.\]
Since $(X\setminus F)\cap F=\emptyset$ and $E\supseteq F$,
it follows that
\[\bigcup_{j=1}^{n}U_{\alpha(j)}\supseteq F\]
and we are done.
\end{proof}


\begin{theorem}\label{T;compact closed}
If $(X,\tau)$ is Hausdorff, then every compact set is closed.
\end{theorem}

\begin{proof}[\bf Proof] Let $K$ be a compact set. If $x\notin K$,
then, given, any $k\in K$ we know that $k\neq x$ and so,
since $X$ is Hausdorff, we can find open sets $U_{k}$
and $V_{k}$ such that
\[x\in V_{k},\ k\in U_{k}\ \text{and}\ V_{k}\cap U_{k}=\emptyset.\]
Since $\bigcup_{k\in K}U_{k}\supseteq\bigcup_{k\in K}\{k\}=K$,
we have an open cover of $K$. By compactness, we can find
$k(1),\,k(2),\,\dots,\,k(n)\in K$ such that
\[\bigcup_{j=1}^{n}U_{k(j)}\supseteq K.\]
We observe that the finite intersection $V=\bigcap_{j=1}^{n}V_{k(j)}$
is an open neighbourhood of $x$ and that
\[V\cap K\subseteq V\cap\bigcup_{j=1}^{n}U_{k(j)}=\emptyset,\]
so $V\cap K$ and we have shown that every $x\in X\setminus K$ has an
open neighbourhood lying entirely within $X\setminus K$.
Thus $X\setminus K$ is open and $K$ is closed.
\end{proof}


\begin{theorem}\label{T;compact not closed}
Give an example of a topological space
and a compact set which is not closed.
\end{theorem}
\begin{proof}[\bf Proof] If $(X,\tau)$ has the indiscrete topology,
then, if $Y\subseteq X$, $Y\neq X,\,\emptyset$, we have $Y$
compact but not closed. We can take $X=\{a,b\}$ with $a\neq b$
and $Y=\{a\}$.
\end{proof}


Combining the Heine--Borel theorem with
Theorems~\ref{T;closed subset compact}
and~\ref{T;compact closed} and a little thought,
we get a complete characterisation of the compact subsets
of ${\mathbb R}$ (with the standard topology).

\begin{theorem}\label{T;closed bounded}
Consider $({\mathbb R},\tau)$ with the standard
(Euclidean) topology. A set $E$ is compact if and
only if it is closed and bounded
(that is to say, there exists a
$M$ such that $|x|\leq M$ for all $x\in E$).
\end{theorem}
\begin{proof}[\bf Proof] If $E$ is bounded, then $E\subseteq [-M,M]$
for some $M$. By the theorem of Heine--Borel, $[-M,M]$
is compact so, if $E$ is closed, $E$ is compact.

Since $({\mathbb R},\tau)$ is Hausdorff any compact set must be closed.
Finally suppose that $E$ is compact. We have
\[E\subseteq \bigcup_{j=1}^{\infty}(-j,j).\]
By compactness, we can find $j(r)$ such that $E \subseteq\bigcup_{r=1}^{N}(-j(r),j(r))$. Taking $M=\max_{1\leq r\leq n}j(r)$ we have $E\subseteq(-M,M)$ so
$E$ is bounded.
\end{proof}


In Example~\ref{T;open not continuous} we saw that the
continuous image of an open set need not be open.
It also easy to see that the
continuous image of a closed set need not be closed.
\begin{problem} Let ${\mathbb R}$ have the usual metric.
Give an example of a continuous injective function
$f:{\mathbb R}\rightarrow{\mathbb R}$ such that
$f({\mathbb R})$ is not closed.
\end{problem}
\begin{proof}[\bf Hint] Look at the solution of Example~\ref{T;Cauchy not topological} if you need a hint.
\end{proof}

However, the continuous image of a compact set
is always compact.




\begin{theorem}\label{T;compact image}
Let $(X,\tau)$ and $(Y,\sigma)$
be topological spaces and $f:X\rightarrow Y$ a continuous
function. If $K$ is a compact subset of $X$,
then $f(K)$ is a compact subset of $Y$.
\end{theorem}
\begin{proof}[\bf Proof] Suppose that $U_{\alpha}\in\sigma$
for all $\alpha\in A$
and $\bigcup_{\alpha\in A}U_{\alpha}\supseteq f(K)$.
Then
\[\bigcup_{\alpha\in A}f^{-1}(U_{\alpha})
=f^{-1}\left(\bigcup_{\alpha\in A}U_{\alpha}\right)\supseteq K\]
and, since $f$ is continuous $f^{-1}(U_{\alpha})\in\tau$
for all $\alpha\in A$. By compactness, we can find
$\alpha(j)\in A$ $[1\leq j\leq n]$ such that
\[\bigcup_{j=1}^{n}f^{-1}(U_{\alpha(j)})\supseteq K\]
and so
\[\bigcup_{j=1}^{n}U_{\alpha(j)}=
f\left(\bigcup_{j=1}^{n}f^{-1}(U_{\alpha(j)})\right)
\supseteq f(K)\]
and we are done.
\end{proof}


This result has many delightful consequences.
Recall, for example, that the quotient topology
$X/\negthinspace\sim$ is defined in such a way that the
quotient map $q:X\rightarrow X/\negthinspace\sim$ is continuous.
Since $q(X)=X/\negthinspace\sim$,
Theorem~\ref{T;compact image} gives us
a positive property of the quotient topology.


\begin{theorem}\label{T;quotient compact}
Let $(X,\tau)$ be a compact topological space
and $\sim$ an equivalence relation on $X$.
Then the quotient topology on $X/\negthinspace\sim$ is compact.
\end{theorem}

The next result follows at once from our
characterisation of compact sets for the real
line with the usual topology.
\begin{theorem}
Let ${\mathbb R}$ have the usual metric.
If $K$ is a closed and bounded subset of ${\mathbb R}$
and $f:K\rightarrow{\mathbb R}$ is continuous,
then $f(K)$ is closed and bounded.
\end{theorem}
This gives a striking extension of one of the
crowning glories of a first course in analysis.


\begin{theorem}\label{T;attains bounds}
Let ${\mathbb R}$ have the usual metric.
If $K$ is a
closed and bounded subset of ${\mathbb R}$
and $f:K\rightarrow{\mathbb R}$ is continuous,
then $f$ is bounded and attains its bounds.
\end{theorem}
\begin{proof}[\bf Proof] If $K$ is empty there is nothing to prove,
so we assume $K\neq\emptyset$.

Since $K$ is compact and $f$ is continuous
$f(K)$ is compact. Thus $f(K)$ is a non-empty closed
bounded set. Since $f(K)$ is non-empty and bounded,
it has a supremum $\alpha$, say. Since $f(K)$ is closed,
it contains its supremum.
[Observe that we can find $k_{n}\in K$ such that
\[\alpha-1/n\leq f(k_{n})\leq \alpha\]
and so $f(k_{n})\rightarrow\alpha$. Since $f(K)$
is closed, $\alpha\in f(K)$.]
\end{proof}



\begin{theorem}\label{T;converse attains bounds}
Let ${\mathbb R}$ have the usual metric.
If $K$ is a subset of ${\mathbb R}$ with the property
that, whenever $f:K\rightarrow{\mathbb R}$ is continuous,
then $f$ is bounded and attains its bounds, it follows
that $K$ is closed and bounded.
\end{theorem}
\begin{proof}[\bf Proof] If $K=\emptyset$ there is nothing to prove,
so we assume $K\neq\emptyset$.

Let $f:K\rightarrow {\mathbb R}$ be defined by
$f(k)=|k|$. Since $f$ is bounded, $K$ must be.

If $x\notin K$, then the function $f:K\rightarrow {\mathbb R}$
given by $f(k)=|k-x|^{-1}$ is continuous and so bounded.
Thus we can find an $M>0$ such that $|f(k)|<M$ for all $k\in K$.
It follows that $|x-k|> M^{-1}$ for all $k\in K$
and the open ball $B(x,M^{-1})$ lies entirely in the complement
of $K$. Thus $K$ is closed.
\end{proof}



Theorem~\ref{T;attains bounds}
has the following straightforward generalisation
whose proof is left to the reader.
\begin{theorem}\label{T;compact attains bounds}
If $K$ is a compact space
and $f:K\rightarrow{\mathbb R}$ is continuous
then $f$ is bounded and attains its bounds.
\end{theorem}

We also have the following useful result.


\begin{theorem}\label{T;bijection compact}
Let $(X,\tau)$ be a compact and $(Y,\sigma)$
a Hausdorff topological space. If $f:X\rightarrow Y$
is a continuous bijection, then it is a homeomorphism.
\end{theorem}

\begin{proof}[\bf Proof] Since $f$ is a bijection, $g=f^{-1}$ is a well
defined function. If $K$ is closed in $X$, then (since a closed
subset of a compact space is compact) $K$ is compact
so $f(K)$ is compact. But a compact subset of a Hausdorff
space is closed so $g^{-1}(K)=f(K)$ is closed. Thus
$g$ is continuous and we are done. (If $U$ is open in $X$
then $X\setminus U$ is closed so $Y\setminus g^{-1}(U)=g^{-1}(X\setminus U)$
is closed and $g^{-1}(U)$ is open.)
\end{proof}

Theorem~\ref{T;bijection compact} is illuminated by the following
almost trivial remark.
\begin{lemma} Let $\tau_{1}$ and $\tau_{2}$ be topologies
on the same space $X$. The identity map
\[\iota:(X,\tau_{1})\rightarrow(X,\tau_{2})\]
from $X$ with topology $\tau_{1}$ to
$X$ with topology $\tau_{2}$ given by $\iota(x)=x$
is continuous if and only if $\tau_{1}\supseteq\tau_{2}$.
\end{lemma}

\begin{theorem}\label{T;compare topologies}
Let $\tau_{1}$ and $\tau_{2}$ be topologies
on the same space $X$.

(i) If $\tau_{1}\supseteq \tau_{2}$ and $\tau_{1}$ is compact,
then so is $\tau_{2}$.

(ii) If $\tau_{1}\supseteq \tau_{2}$ and $\tau_{2}$ is Hausdorff,
then so is $\tau_{1}$.

(iii) If $\tau_{1}\supseteq \tau_{2}$, $\tau_{1}$ is compact
and $\tau_{2}$ is Hausdorff, then $\tau_{1}=\tau_{2}$.
\end{theorem}
\begin{proof}[\bf Proof] (i) The map $\iota:(X,{\tau}_{1})\rightarrow(X,{\tau}_{2})$
is continuous and so takes compact sets to compact sets. In particular,
since $X$ is compact, in $\tau_{1}$, $X=\iota X$ is compact in $\tau_{2}$.

(ii) If $x\neq y$ we can find $x\in U\in\tau_{2}$ and $y\in V\in\tau_{2}$
with $U\cap V=\emptyset$. Automatically
$x\in U\in\tau_{1}$ and $y\in V\in\tau_{1}$ so we are done.

(iii) The map $\iota:(X,{\tau}_{1})\rightarrow(X,{\tau}_{2})$
is a continuous bijection and so a homeomorphism.
\end{proof}


The reader may care to
recall that 'Little Bear's porridge was neither too hot nor
too cold but just right'.

With the hint given by the previous theorem it should
be fairly easy to do do the next exercise.


\begin{theorem}\label{T;too hot}
(i) Give an example of a compact space $(X,\tau)$ and
a topological space $(Y,\sigma)$
together with a continuous bijection $f:X\rightarrow Y$
which is not a homeomorphism.

(ii) Give an example of a topological space $(X,\tau)$ and
a Hausdorff space $(Y,\sigma)$
together with a continuous bijection $f:X\rightarrow Y$
which is not a homeomorphism.
\end{theorem}
\begin{proof}[\bf Solution] Let $\tau_{1}$ be the
indiscrete topology on $[0,1]$,
$\tau_{2}$ be the usual (Euclidean) topology on $[0,1]$
and $\tau_{3}$ be the discrete topology on $[0,1]$.
Then $({\mathbb R},\tau_{1})$ is compact (but not Hausdorff),
$({\mathbb R},\tau_{2})$ is compact and Hausdorff,
and $({\mathbb R},\tau_{3})$ is Hausdorff (but not compact).
The identity maps $\iota:(X,\tau_{1})\rightarrow(X,\tau_{2})$
and $\iota:(X,\tau_{2})\rightarrow(X,\tau_{3})$ are continuous
bijections but not homeomorphisms.
\end{proof}


We shall give a (not terribly convincing) example
of the use of Theorem~\ref{T;bijection compact}
in our proof of Exercise circle as quotient.

The reader may have gained the impression that
compact Hausdorff spaces form an ideal backdrop
for continuous functions to the reals.
Later work shows that the impression is
absolutely correct
but it must be remembered that many important
spaces (including the the real line with the usual topology)
are not compact.




















\section{Products of compact spaces}

The course contains one further major theorem on compactness.

\begin{theorem}\label{T;product compact}
The product of two compact spaces is compact. (More formally, if $(X,\tau)$ and $(Y,\sigma)$ are compact topological spaces and $\lambda$ is the product topology then $(X\times Y,\lambda)$ is compact.)
\end{theorem}
\begin{proof}[\bf Proof] Let $O_{\alpha}\in\lambda$ $[\alpha\in A]$ and
\[\bigcup_{\alpha\in A}O_{\alpha}=X\times Y.\]
Then, given $(x,y)\in X\times Y$, we can find
$U_{x,y}\in\tau$, $V_{x,y}\in\sigma$ and $\alpha(x,y)\in A$ such
that
\[(x,y)\in U_{x,y}\times V_{x,y}\subseteq O_{\alpha(x,y)}.\]
In particular, we have
\[\bigcup_{y\in Y}\{x\}\times V_{x,y}=\{(x,y)\,:\,y\in Y\}\]
for each $x\in X$ and so
\[\bigcup_{y\in Y}V_{x,y}=Y.\]
By compactness, we can find a positive
integer $n(x)$ and $y(x,j)\in Y$ $[1\leq j\leq n(x)]$
such that
\[\bigcup_{j=1}^{n(x)}V_{x,y(x,j)}=Y.\]
Now $U_{x}=\bigcap_{j=1}^{n(x)}U_{x,y(x,j)}$ is the finite intersection
of open sets in $X$ and so open. Further $x\in U_{x}$ and so
\[\bigcup_{x\in X}U_{x}=X.\]
By compactness, we can find $x_{1},\,x_{2},\,\dots,\,x_{m}$ such that
\[\bigcup_{r=1}^{m}U_{x_{r}}=X.\]

It follows that
\be
\bigcup_{r=1}^{m}\bigcup_{j=1}^{n(x_{r})}O_{x_{r},y(x_{r},j)} \supseteq \bigcup_{r=1}^{m} \bigcup_{j=1}^{n(x_{r})}U_{x_{r},y(x_{r},j)}\times V_{x_{r},y(x_{r},j)} \supseteq \bigcup_{r=1}^{m}\bigcup_{j=1}^{n(x_{r})} U_{x_{r}}\times V_{x_{r},y(x_{r},j)} \supseteq \bigcup_{r=1}^{m}U_{x_{r}}\times Y \supseteq X\times Y
\ee
and we are done.
\end{proof}


Tychonov showed that
the general product of compact spaces
is compact (see the note to Exercise~\ref{E;Infinite product})
so Theorem~\ref{T;product compact} is often
referred to as Tychonov's theorem.

The same proof, or the remark that the subspace topology
of a product topology is the product topology of the subspace
topologies (see Exercise~\ref{E;Product subspace}),
gives the closely related result.
\begin{theorem}\label{T;product subspace compact}
Let $(X,\tau)$ and $(Y,\sigma)$ be topological spaces
and let $\lambda$ be the product topology.
If $K$ is a compact subset of $X$ and
$L$ is a compact subset of $Y$, then $K\times L$
is a compact in $\lambda$.
\end{theorem}
We know (see Exercise~\ref{E;metric product}) that the topology
on ${\mathbb R}^{2}$ derived from the Euclidean metric
is the same as the product topology when we
give ${\mathbb R}$ the topology derived from the Euclidean metric.
Theorem~\ref{T;closed interval compact} thus has the following
corollary\footnote{Stated more poetically by Conway.
\begin{verse}
If $E$'s closed and bounded, says Heine--Borel,\\
And also Euclidean, then we can tell\\
That, if it we smother\\
With a large open cover,\\
There's a finite refinement as well.
\end{verse}
}.
\begin{theorem}
Let $[a,b]\times[c,d]$ be given its usual (Euclidean)
topology. Then the derived topology is compact.
\end{theorem}
The arguments of the previous section carry over to give
results like the following.
\begin{theorem}
Consider ${\mathbb R}^{2}$ with the standard
(Euclidean) topology. A set $E$ is compact if and
only if it is closed and bounded
(that is to say, there exists a
$M$ such that $\|{\mathbf x}\|\leq M$ for all ${\mathbf x}\in E$).
\end{theorem}
\begin{theorem}\label{T;compact Euclidean}
Let ${\mathbb R}^{2}$ have the usual metric.
If $K$ is a
closed and bounded subset of ${\mathbb R}^{2}$
and $f:K\rightarrow{\mathbb R}$ is continuous,
then $f$ is bounded and attains its bounds.
\end{theorem}
\begin{problem}
Let ${\mathbb R}^{2}$ have the usual metric.
If $K$ is a subset of ${\mathbb R}^{2}$ with the property
that, whenever $f:K\rightarrow{\mathbb R}$ is continuous,
then $f$ is bounded and attains its bounds, it follows
that $K$ is closed and bounded.
\end{problem}
The generalisation to ${\mathbb R}^{n}$ is left to the reader.

The next exercise brings together many of the themes of this course.
The reader should observe that we \emph{know} what we want
the circle to look like. This exercise \emph{checks} that
defining the circle via quotient maps gives us what we want.



\begin{problem}\label{P;circle as quotient}
Consider the complex plane with its usual metric.
Let
\[{\partial}D=\{z\in{\mathbb C}\,:\,|z|=1\}\]
and give ${\partial}D$ the subspace topology $\tau$.
Give ${\mathbb R}$ its usual topology and define
an equivalence relation $\sim$ by $x\sim y$ if $x-y\in{\mathbb Z}$.
We write ${\mathbb R}/\negthinspace\sim={\mathbb T}$ and give ${\mathbb T}$
the quotient topology. The object of this exercise
is to show that ${\partial}D$ and ${\mathbb T}$ are homeomorphic.

(i) Verify that $\sim$ is indeed an equivalence relation.

(ii) Show that, if we define
$f:{\mathbb R}\rightarrow {\partial}D$
by $f(x)=\exp(2\pi ix)$, then $f(U)$ is
open whenever $U$ is open.

(iii) If $q:{\mathbb R}\rightarrow {\mathbb T}$ is the quotient map
$q(x)=[x]$ show that $q(x)=q(y)$ if and only if $f(x)=f(y)$.
Deduce that $q\big(f^{-1}(\{\exp(2\pi ix)\})\big)=[x]$
and that the equation $F(\exp(2\pi ix))=[x]$ gives a well
defined bijection $F:{\partial}D\rightarrow{\mathbb T}$.

(iv) Show that $F^{-1}(V)=f\big(q^{-1}(V)\big)$ and deduce
that $F$ is continuous.

(v) Show that ${\mathbb T}$ is Hausdorff and explain why
${\partial}D$ is compact. Deduce that $F$ is a homeomorphism.
\end{problem}
\begin{proof}[\bf Solution]
(i) Observe that $x-x=0\in{\mathbb Z}$ so $x\sim x$.

Observe that $x\sim y$ implies $x-y\in{\mathbb Z}$
so $y-x=-(x-y)\in{\mathbb Z}$ and $y\sim x$.

Observe that, if  $x\sim y$ and $y\sim z$, then $x-y,\,y-z\in{\mathbb Z}$
so
\[x-z=(x-y)+(y-z)=x-z\in{\mathbb Z}\]
and $x\sim z$.

(ii) If $x\in U$ an open set, then we can find a $1>\delta>0$
such that $|x-y|<\delta$ implies $y\in U$.

By simple geometry, any $z\in{\mathbb C}$ with $|z|=1$ and
$|\exp(2\pi ix)-z|<\delta/100$
can be written as $z=\exp(2\pi iy)$ with $|y-x|<\delta$.
Thus
\[\partial D\cap\{z\in{\mathbb C}\,:\,|z-\exp(2\pi ix)|<\delta/100\}
\subseteq f(U).\]
We have shown that $f(U)$ is open.

(iii) We have
\begin{align*}
q(x)=q(y)&\Leftrightarrow y\in[x] \Leftrightarrow
x-y\in{\mathbb Z} \Leftrightarrow
\exp(2\pi i(x-y))=1\\
&\Leftrightarrow \exp(2\pi ix)=\exp(2\pi iy)
\Leftrightarrow f(x)=f(y).
\end{align*}

It follows that the equation $F(\exp(2\pi ix))=[x]$ gives a well
defined bijection $F:{\partial}D\rightarrow{\mathbb T}$.

(iv) Observe that
\[F^{-1}([x])=\{\exp(2\pi it)\,:\,\exp(2\pi it)=\exp(2\pi ix)\}
=f\big(q^{-1}([x])\big)\]
and so $F^{-1}(V)=f\big(q^{-1}(V)\big)$. If $V$ is open,
then, since $q$ is continuous, $q^{-1}(V)$ is open so, by (ii),
$F^{-1}(V)$ is open. Thus $F$ is continuous.

(v) If $[x]\neq [y]$, then we know that $x-y\notin{\mathbb Z}$
and the set
\[\{|t|\,:\,t-(x-y)\in{\mathbb Z},\,|t|<1\}\]
is finite and non-empty. Thus there exists a $\delta>0$
such that
\[\{|t|\,:\,t-(x-y)\in{\mathbb Z},\,|t|<\delta\}=\{\emptyset\}.\]
Let
\[U_{x}=\bigcup_{j=-\infty}^{\infty}(j+x-\delta/4,j+x+\delta/4)
\ \text{and}
\ U_{y}=\bigcup_{j=-\infty}^{\infty}(j+y-\delta/4,j+y+\delta/4).\]
Observe that $U_{x}$ and $U_{y}$ are open in ${\mathbb R}$
and $q^{-1}\big(q(U_{x}))=U_{x}$, $q^{-1}\big(q(U_{y}))=U_{y}$,
and so $q(U_{x})$ and $q(U_{y})$ are open in the quotient topology.
Since $[x]\in q(U_{x})$, $[y]\in q(U_{y})$
and $q(U_{x})\cap q(U_{y})=\emptyset$,
we have shown that the quotient topology is Hausdorff.

Since ${\partial}D$ is closed and bounded in ${\mathbb C}$
and we can identify ${\mathbb C}$ with ${\mathbb R}^{2}$ as a metric
space, ${\partial}D$ is compact.

Since a continuous bijection from a compact to a Hausdorff space
is a homeomorphism, $F$ is a homeomorphism.

[Remark. It is just as simple to show that the natural
map from ${\mathbb T}$ (which we know to be compact, why?)
to ${\partial}D$ (which we know to be Hausdorff, why?)
is a bijective continuous map. Or we could show continuity
in both directions and not use the result on continuous bijections.]
\end{proof}
























\section{Connectedness} This section deals with a problem
which the reader will meet (or has met) in her first
complex variable course. Here is a similar problem that
occurs on the real line. Suppose that $U$ is an open
subset of ${\mathbb R}$ (in the usual topology)
and $f:U\rightarrow {\mathbb R}$ is a differentiable function
with $f'(u)=0$ for all $u\in U$. We would like to
conclude that $f$ is constant, but the example
$U=(-2,-1)\cup (1,2)$, $f(u)=1$ if $u>0$,
$f(u)=-1$ if $u<0$ shows that the general result
is false. What extra condition should we put on $U$
to make the result true?

After some experimentation, mathematicians have come up
with the following idea.
\begin{definition}\label{D;disconnected}
A topological space $(Y,\sigma)$ is
said to be disconnected if we can find non-empty
open sets $U$ and $V$ such that $U\cup V=Y$
and $U\cap V=\emptyset$. A space which is not
disconnected is called connected.
\end{definition}
\begin{definition} If $E$ is a subset of a topological
space $(X,\tau)$ then $E$ is called connected (respectively
disconnected) if the subspace topology on $E$ is
connected (respectively
disconnected).

\end{definition}
The definition of a subspace topology gives the following
alternative characterisation which the reader may prefer.
\begin{lemma} If $E$ is a subset of a topological
space $(X,\tau)$, then $E$ is disconnected if
and only if we can find
open sets $U$ and $V$ such that $U\cup V\supseteq E$,
$U\cap V\cap E=\emptyset$, $U\cap E\neq\emptyset$
and $V\cap E\neq\emptyset$
\end{lemma}

Here is another alternative characterisation which
shows that we are on the right track.



\begin{theorem}\label{T;connected via integer}
If $E$ is a subset of a topological
space $(X,\tau)$, then $E$ is disconnected if and
only if we can find a non-constant continuous
function $f:E\rightarrow{\mathbb R}$ which only
takes the value $0$ or $1$.
\end{theorem}
\begin{proof}[\bf Proof] Since we are dealing with a subspace topology,
we can take $E=X$.

If $f:X\rightarrow{\mathbb R}$ is a continuous
non-constant function which only
takes the value $0$ or $1$, then
$U=f^{-1}(\{0\})=f^{-1}((-1/2,1/2))$ is open and non-empty
and similarly $V=f^{-1}(\{1\})$ is. Since $V\cup U=X$
and $V\cap U=\emptyset$, it follows that $X$ is disconnected.

Conversely, if $X$ is disconnected we can find non-empty
open sets $U$ and $V$ such that  $V\cup U=X$
and $V\cap U=\emptyset$. If we set $f(u)=0$ when $u\in U$
and $f(v)=1$ when $v\in V$, then
$f:X\rightarrow{\mathbb R}$ is a well defined
non-constant function which only
takes the value $0$ or $1$. If $A\subset{\mathbb R}$,
the $f^{-1}(A)$ must be one of the four sets
$\emptyset$, $U$, $V$ or $X$ all of which are open.
Thus $f$ is continuous.
\end{proof}



The following deep result is now easy to prove.



\begin{theorem}\label{T;reals connected}
If we give ${\mathbb R}$ the usual topology,
then ${\mathbb R}$ and the  intervals $[a,b]$
and $(a,b)$ are connected.
\end{theorem}
\begin{proof}[\bf Proof] We prove the result for $(a,b)$. The other
results are proved similarly.

Suppose $f:(a,b)\rightarrow{\mathbb R}$ is continuous
and there exist $c,\,d\in (a,b)$ with $f(c)=0$
and $f(d)=1$. Without loss of generality
we may suppose that $c<d$ and so $a<c<d<b$.
By the intermediate value theorem, we can find $\gamma\in(c,d)$
with $f(\gamma)=1/2$. Since $\gamma\in(a,b)$, $f$ takes
at least three values and $(a,b)$ must be connected.
\end{proof}



The reader will find it instructive to use Theorem~\ref{T;connected via integer} to prove parts~(i) and~(iii) of the next exercise.



\begin{theorem}\label{T;quotient connected}
(i) If $(X,\tau)$ and $(Y,\sigma)$ are topological spaces,
$E$ is a connected subset of $X$ and $g:E\rightarrow Y$
is continuous, then $g(E)$ is connected.
(More briefly the continuous image of a connected
set is connected.)

(ii) If $(X,\tau)$ is a connected topological space
and $\sim$ is an equivalence relation on $X$, then $X/\negthinspace\sim$
with the quotient topology is connected.

(iii) If $(X,\tau)$ and $(Y,\sigma)$ are
connected topological spaces, then $X\times Y$
with the product topology is connected.

(iv) If $(X,\tau)$ is a
connected topological space and $E$ is a subset of $X$,
it does not follow that $E$ with the subspace topology is connected.
\end{theorem}
\begin{proof}[\bf Proof] (i) If $g(E)$ is not connected we can find a non-constant
continuous
$f:g(E)\rightarrow{\mathbb R}$ taking only the values $0$ and $1$.
Setting $F=f\circ g$ (the composition of $f$ and $g$),
we know that $F:E\rightarrow{\mathbb R}$ is non-constant,
continuous and only takes the values $0$ and $1$.
Thus $E$ is not connected.

(ii) $X/\negthinspace\sim$ is the continuous image of $X$ under the
quotient map which we know to be continuous.

(iii) Suppose $X\times Y$ with the product topology
is not connected. Then we can find
a non-constant continuous function
$f:X\times Y\rightarrow{\mathbb R}$ taking only the values $0$ and $1$.
Take $(x,y),\,(u,v)\in X\times Y$ with $f(x,y)\neq f(u,v)$.
Then, if $f(x,v)=f(x,y)$, it follows that $f(x,v)\neq f(u,v)$.
Without loss of generality, suppose that $f(x,v)\neq f(x,y)$.
Then we know that the function $\theta:Y\rightarrow X\times Y$
given by $\theta(z)=(x,z)$ is continuous.
(If $\Omega$ is open in $X\times Y$ and $z\in\theta^{-1}(\Omega)$,
then $(x,z)\in\Omega$ so we can find $U$ open in $X$
and $V$ open in $Y$ such that
$(x,z)\in U\times V\subseteq \Omega$. Thus
$z\in V\subseteq \theta^{-1}(\Omega)$
and we have shown $\theta^{-1}(\Omega)$ open.)
If we set $F=f\circ\theta$, then
$F:Y\rightarrow{\mathbb R}$ is non-constant,
continuous and only takes the values $0$ and $1$.
Thus $Y$ is not connected.

(iv) ${\mathbb R}$ is connected with the usual topology
but $E=(-2,-1)\cup(1,2)$ is not.
\end{proof}

The proof of the next example is particularly
important because it gives a standard technique
for using connectedness in practice.



\begin{theorem}\label{T;locally constant}
Suppose that $E$ is a connected subset of a topological
space $(X,\tau)$. Suppose that $f:E\rightarrow{\mathbb R}$
is `locally constant' in the sense that, given any
$e\in E$, we can find an open neighbourhood $U$ of $e$
such that $f$ is constant on $U\cap E$. Then $f$
is constant.
\end{theorem}
\begin{proof}[\bf Proof] Since we are dealing with the subspace
topology on $E$, there is no loss in generality
in taking $E=X$. If  $X=\emptyset$ the result is vacuous
so we may take $X\neq\emptyset$.

Choose an $x_{0}\in X$ and set $c=f(x_{0})$.
Now consider the sets
\[U=\{x\in X\,:\,f(x)=c\}\ \text{and}
\ V=\{x\in X\,:\,f(x)\neq c\}.\]
We claim that $U$ and $V$ are open.
For suppose $v\in V$.
Then we can find an open neighbourhood $N$
of $v$ such that $f$ is constant on $N$. Thus $f(x)=f(v)\neq c$
for all $x\in N$, so $N\subseteq V$. It follows that $V$
is open. A similar, slightly simpler, argument shows that $U$ is open.

Since $U\cap V=\emptyset$, $U\cup V=X$ and $U\neq \emptyset$ the connectedness of $X$ tells us that $V=\emptyset$ and $U=X$. The result follows.
\end{proof}



\begin{problem}\label{E;locally constant converse}
Suppose that $E$ is subset of a topological
space $(X,\tau)$ such that any locally constant
$f:E\rightarrow{\mathbb R}$ is constant.
Show that $E$ is connected.
\end{problem}
\begin{proof} If you need a hint, look at the proof of Theorem~\ref{T;connected via integer}.
\end{proof}



Example E;locally constant and
Exercise~\ref{E;locally constant converse} together
completely settle the question posed in the first paragraph
of this section.

The following lemma outlines a very natural development.

\begin{lemma}\label{L;connected components}
We work in a topological space $(X,\tau)$.

(i) Let $x_{0}\in X$. If $x_{0}\in E_{\alpha}$
and $E_{\alpha}$ is connected for all $\alpha\in A$,
then $\bigcup_{\alpha\in A}E_{\alpha}$ is connected.

(ii) Write $x\sim y$ if there exists a connected set $E$
with $x,\,y\in E$. Then $\sim$ is an equivalence relation.

(iii) The equivalence classes $[x]$ are connected.

(iv) If $F$ is connected and $F\supseteq [x]$, then $F=[x]$.
\end{lemma}
\begin{proof}[\bf Proof] (i) Let $U$ and $V$ be open sets such that
\[U\cup V\supseteq \bigcup_{\alpha\in A}E_{\alpha}
\ \text{and}
\ U\cap V\cap \bigcup_{\alpha\in A}E_{\alpha}=\emptyset.\]
Without loss of generality, let $x_{0}\in U$.
Then
\[U\cup V\supseteq E_{\alpha}
\ \text{and}
\ U\cap V\cap E_{\alpha}=\emptyset\]
for each $\alpha\in A$. But
$x_{0}\in U\cap E_{\alpha}$ so
$U\cap E_{\alpha}\neq\emptyset$ and so, by the
connectedness of $E_{\alpha}$, we have
\[U\supseteq E_{\alpha}\]
for all $\alpha\in A$. Thus
$U\supseteq \bigcup_{\alpha\in A}E_{\alpha}$.
We have shown that $\bigcup_{\alpha\in A}E_{\alpha}$ is connected.

(ii) Observe that if $U$ and $V$ are sets (open or not) such that
\[U\cup V\supseteq \{x\},
\ \text{and}
\ U\cap V\cap \{x\}=\emptyset.\]
then either $x\notin U$ and $U\cap\{x\}=\emptyset$
or $x\in U$ so $U\supseteq\{x\}$. Thus the one point set
$\{x\}$ is connected and $x\sim x$.

The symmetry of the definition tells us that,
if $x\sim y$, then $y\sim x$.

If $x\sim y$ and $y\sim z$, then $x,\,y\in E$ and $y,\,z\in F$
for some connected sets $E$ and $F$. By part~(i),
$E\cup F$ is connected (observe that $y\in E,\,F$)
so, since $x,\,z\in E\cup F$, $x\sim z$.

We have shown that $\sim$ is an equivalence relation.

(iii) If $y\in [x]$, then there exists a connected set $E_{y}$
with $x,\,y\in E_{y}$. By definition $[x]\supseteq E_{y}$ so
\[[x]=\bigcup_{y\in[x]}\{y\}\subseteq \bigcup_{y\in[x]}E_{y}\subseteq [x]\]
whence
\[[x]=\bigcup_{y\in[x]}E_{y}\]
and, by part~(i), $[x]$ is connected.

(iv) If $F$ is connected and $[x]\subseteq F$ then $x\in F$
and, by definition of $\sim$, $[x]\supseteq F$.
It follows that $F=[x]$.
\end{proof}


The sets $[x]$ are known as the \emph{connected components} of $(X,\tau)$.

Connectedness is related to  another older concept.

\begin{definition}\label{D;path-connected}
Let $(X,\tau)$ be a topological space.
We say that $x,\,y\in X$ are path-connected if
(when $[0,1]$ is given its standard Euclidean topology)
there exists a continuous
function $\gamma:[0,1]\rightarrow X$ with
$\gamma(0)=x$ and $\gamma(1)=y$.
\end{definition}

Of course, $\gamma$ is referred to as a path from $x$ to $y$.


\begin{lemma}\label{L;path-connected equivalence}
If $(X,\tau)$ is a topological space and
we write $x\sim y$ if $x$ is path-connected
to $y$, then $\sim$ is an equivalence relation.
\end{lemma}
\begin{proof}[\bf Proof]
If $x\in X$, then the map $\gamma:[0,1]\rightarrow X$ defined by
$\gamma(t)=x$ for all $t$ is continuous.
(Observe that, if $F$ is a closed set in X, then
$\gamma^{-1}(F)$ takes the value $\emptyset$
or $[0,1]$ both of which are closed.) Thus $x\sim x$.

If $x\sim y$, then we can find a continuous map
$\gamma:[0,1]\rightarrow X$ with $\gamma(0)=x$
and $\gamma(1)=y$. The map $T:[0,1]\rightarrow [0,1]$
given by $T(t)=1-t$ is continuous so the composition
$\tilde{\gamma}=\gamma\circ T$ is. Observe
that $\tilde{\gamma}(0)=y$ and $\tilde{\gamma}(1)=x$
so $y\sim x$.

If $x\sim y$ and $y\sim z$, then we can find  continuous maps
$\gamma_{j}:[0,1]\rightarrow X$ with $\gamma_{1}(0)=x$
$\gamma_{1}(1)=y$, $\gamma_{2}(0)=y$ and $\gamma_{2}(1)=z$.
Define $\gamma:[0,1]\rightarrow X$ by
\[
\gamma(t)=
\begin{cases}
\gamma_{1}(2t)&\text{if $t\in[0,1/2]$}\\
\gamma_{2}(2t-1)&\text{if $t\in(1/2,1]$}.
\end{cases}
\]
If $U$ is open in $X$ then
\[\gamma^{-1}(U)=\{t/2\,:\,t\in\gamma_{1}^{-1}(U)\}
\cup\{(1+t)/2\,:\,t\in\gamma_{2}^{-1}(U)\}\]
is open.

(If more detail is required we argue as follows.
Suppose $s\in\gamma^{-1}(U)$. If $s\in(0,1/2)$,
then $2s\in \gamma_{1}^{-1}(U)$ so, since $\gamma_{1}^{-1}(U)$
is open we can find a $\delta>0$ with $s>\delta$
such that $(2s-\delta,2s+\delta)\subseteq \gamma_{1}^{-1}(U)$.
Thus $(s-\delta/2,s+\delta/2)\subseteq \gamma^{-1}(U)$.
If $s=0$ then $0\in\gamma_{1}^{-1}(U)$ so, since $\gamma_{1}^{-1}(U)$
is open we can find a $\delta>0$ with $1>\delta$ such that
$[0,\delta)\subseteq \gamma_{1}^{-1}(U)$. Thus
$[s,\delta/2)=[0,\delta/2)\subseteq \gamma^{-1}(U)$.
The cases $s\in(1/2,1]$ are dealt with similarly.
This leaves the case $s=1/2$. Arguing as before, we can find
$\delta_{1},\delta_{2}>0$ with $1>\delta_{1},\delta_{2}$
such that
\[(1-\delta_{1},1]\subseteq \gamma_{1}^{-1}(U)
\ \text{and}
\ [0,\delta_{2})\subseteq \gamma_{2}^{-1}(U).\]
Setting $\delta=\min(\delta_{1},\delta_{2})$ we have
\[(s-\delta/2,s+\delta/2)=(1/2-\delta/2,1/2+\delta/2)
\gamma^{-1}(U).\]
We see that the case $s=1/2$ is really the only one which requires
care.)

Thus $\gamma$ is continuous
and, since $\gamma(0)=x$, $\gamma(1)=z$, $x\sim z$.
\end{proof}



We say that a topological space is path-connected if every two
points in the space are path-connected.

The following theorem is often useful.


\begin{theorem}\label{T;path-connected to connected}
If a topological space is path-connected, then it is connected.
\end{theorem}
\begin{proof}[\bf Proof]
Suppose that $(X,\tau)$ is path-connected and
that $U$ and $V$ are
open sets with $U\cap V=\emptyset$ and $U\cup V=X$.
If $U\neq \emptyset$, choose $x\in U$.
If $y\in X$, we
can find $f:[0,1]\rightarrow X$ continuous with
$f(0)=x$ and $f(1)=y$. Now the continuous image of a connected set
is connected and $[0,1]$ is connected, so $f([0,1])$ is connected.
Since
\[U\cap V\cap f([0,1])=\emptyset,
\ U\cup V\supseteq f([0,1])
\ \text{and}\ U\cap f([0,1])\neq\emptyset,\]
we know that $U\supseteq f([0,1])$ so $y\in U$. Thus $U=X$.
We have shown that $X$ is connected.
\end{proof}

The converse is false (see Example connected not path-connected
below) but there is one very important case where connectedness
implies path-connectedness.



\begin{theorem}\label{T;connected to path}
If we give ${\mathbb R}^{n}$ the usual topology
then any open set $\Omega$ which is connected is path-connected.
\end{theorem}
\begin{proof}[\bf Proof]
If $\Omega=\emptyset$ there is nothing to prove,
so we assume $\Omega$ non-empty.

Pick ${\mathbf x}\in\Omega$ and let $U$ be the set of all points
in $\Omega$ which are path-connected to $x$ and let $V$ be the set
of all points in $\Omega$ which are not.
We shall prove
that $U$ and $V$ are open.

Suppose first that ${\mathbf u}\in U$. Since $\Omega$
is open, we can find an open ball $B({\mathbf u},\delta)$
centre ${\mathbf u}$, radius $\delta>0$ lying entirely within
$\Omega$. If ${\mathbf y}\in B({\mathbf u},\delta)$,
then ${\mathbf u}$ is path-connected to ${\mathbf y}$ in
$B({\mathbf u},\delta)$ and so in $U$.
(Consider $\gamma:[0,1]\rightarrow\Omega$ given by
$\gamma(t)=t{\mathbf u}+(1-t){\mathbf y}$.)
Since ${\mathbf x}$ is path-connected to ${\mathbf u}$
and
${\mathbf u}$ is path-connected to ${\mathbf y}$,
it follows that
${\mathbf x}$ is path-connected to ${\mathbf y}$
in $\Omega$ so ${\mathbf y}\in U$.

Now suppose that ${\mathbf v}\in V$. Since $\Omega$
is open, we can find an open ball $B({\mathbf v},\delta)$
centre ${\mathbf v}$, radius $\delta>0$ lying entirely within
$\Omega$. If ${\mathbf y}\in B({\mathbf v},\delta)$,
then ${\mathbf v}$ is path-connected to ${\mathbf y}$ in
$B({\mathbf v},\delta)$ and so in $V$. It follows that,
if ${\mathbf y}$ is path-connected to ${\mathbf x}$,
then so is  ${\mathbf v}$. But ${\mathbf v}\in V$,
so ${\mathbf y}$ is not path-connected to ${\mathbf x}$.
Thus ${\mathbf y}\in V$.

Since $U\cup V=\Omega$ and $U\cap V=\emptyset$, the connectedness
of $\Omega$ shows that $U=\Omega$ and $\Omega$ is path-connected.
\end{proof}



The following example shows that, even in ${\mathbb R}^{2}$,
we can not remove the condition $\Omega$ open.



\begin{theorem}\label{T;connected not path-connected}
We work in ${\mathbb R}^{2}$ with the usual topology. Let
\[E_{1}=\{(0,y)\,:\,|y|\leq 1\}
\ \text{and}
\ E_{2}=\{(x,\sin 1/x)\,:\,0<x\leq 1\}\]
and set $E=E_{1}\cup E_{2}$.

(i) Sketch $E$.

(ii) Explain why $E_{1}$ and $E_{2}$ are path-connected
and show that $E$ is closed and connected.

(iii) Suppose, if possible, that ${\mathbf x}:[0,1]\rightarrow E$
is continuous and ${\mathbf x}(0)=(1,0)$, ${\mathbf x}(1)=(0,0)$.
Explain why we can find $0<t_{1}<t_{2}<t_{3}<\dots$ such that
$x(t_{j})=\big((j+\tfrac{1}{2})\pi)^{-1}$. By considering the behaviour
of $t_{j}$ and $y(t_{j})$, obtain a contradiction.

(iv) Deduce that $E$ is not path-connected.
\end{theorem}
\begin{proof}[\bf Proof] Part~(i) is left to the reader.

(ii) If $y_{1},\,y_{2}\in[-1,1]$, the function
${\mathbf f}:[0,1]\rightarrow E_{1}$ given by
\[{\mathbf f}(t)=\big(0,(1-t)y_{1}+ty_{2}\big)\]
is continuous and ${\mathbf f}(0)=(x_{1},0)$ and
${\mathbf f}(1)=(x_{2},0)$,
so $E_{1}$ is path-connected.

If $(x_{1},y_{1}),\,(x_{2},y_{2})\in E_{2}$,
then $y_{j}=\sin 1/x_{j}$ and setting
\[{\mathbf g}(t)=
\biggr((1-t)x_{1}+tx_{2},\sin\big(1/((1-t)x_{1}+tx_{2})\big)\biggr)
\]
we see that ${\mathbf g}$ is continuous and
${\mathbf g}(0)=(x_{1},y_{1})$ and
${\mathbf g}(1)=(x_{2},y_{2})$,
so $E_{2}$ is path-connected.

We next show that $E$ is closed.
Suppose that $(x_{r},y_{r})\in E$
and $(x_{r},y_{r})\rightarrow (x,y)$. If $x=0$, then
we note that, since $|y_{r}|\leq 1$ for all $r$ and
$y_{r}\rightarrow y$, we have $|y|\leq 1$ and
$(x,y)\in E_{1}\subseteq E$. If $x\neq 0$, then
$1\geq x>0$ (since $x_{r}\geq 0$ for all $r$). We can find an $N$
such that $|x-x_{r}|<x/2$ and so $x_{r}>x/2$ for all $r\geq N$.
Thus, by continuity,
\[(x_{r},y_{r})=(x_{r},\sin 1/x_{r})\rightarrow (x,\sin 1/x)
\in E_{2}\subseteq E.\]
Thus $E$ is closed.

Now suppose, if possible, that $E$ is disconnected.
Then we can find $U$ and $V$ open such that
\[U\cap E\neq\emptyset,\ V\cap E\neq\emptyset,\ U\cup V\supseteq E
\ \text{and $U\cap V\cap E=\emptyset$}.\]
Then
\[U\cup V\supseteq E_{j}
\ \text{and $U\cap V\cap E_{j}=\emptyset$}.\]
and so, since $E_{j}$ is path-connected, so connected,
we have $U\cap E_{j}=\emptyset$ or $V\cap E_{j}=\emptyset$
$[j=1,2]$. Without loss of generality, assume $V\cap E_{1}=\emptyset$
so $U\supseteq E_{1}$. Since $(0,0)\in E_{1}$, we have
$(0,0)\in U$. Since $U$ is open, we can find a $\delta>0$
such that $(x,y)\in U$ whenever $\|(x,y)\|_{2}<\delta$.
If $n$ is large,
\[((n\pi)^{-1},0)\in U\cap E_{2}=U\cap V\cap E,\]
contradicting our initial assumptions.
By reductio ad absurdum, $E$ is connected.

(iii) Write ${\mathbf x}(t)=(x(t),y(t))$. Since ${\mathbf x}$
is continuous so is $x$. Since $x(0)=1$ and $x(1)=0$,
the intermediate value theorem tells us that we can find
$t_{1}$ with $0<t_{1}<1$ and $x(t_{1})=(\tfrac{3}{2}\pi)^{-1}$.
Applying the  intermediate value theorem again, we can
find $t_{2}$ with $0<t_{2}<t_{1}$ and
$x(t_{2})=(\tfrac{5}{2}\pi)^{-1}$. We continue inductively.

Since the $t_{j}$ form a decreasing sequence bounded below
by $0$, we have $t_{j}\rightarrow T$ for some $T\in[0,1]$.
Since $y$ is continuous
\[(-1)^{j}=\sin\big(1/x(t_{j})\big)=y(t_{j})\rightarrow y(T)\]
which is absurd.

(iv) Part~(iii) tells us that there is no path joining
$(0,0)$ and $(1,0)$ in $E$, so $E$ is not path-connected.
\end{proof}

Paths play an important role in complex analysis and algebraic topology.













\section{Compactness in metric spaces} When we work in ${\mathbb R}$
(or, indeed, in ${\mathbb R}^{n}$) with the usual metric,
we often use the theorem of Bolzano--Weierstrass that every
sequence in a bounded closed set has a subsequence with a limit
in that set. It is also easy to see that closed bounded
sets are the only subsets of ${\mathbb R}^{n}$ which have the
property that every sequence in the set
has a subsequence with a limit in that set. This suggests
a series of possible theorems some of which turn out to be false.



\begin{theorem}\label{T;bounded not convergent}
Give an example of metric space $(X,d)$ which is bounded (in the sense that there exists an $M$
with $d(x,y)\leq M$ for all $x,\,y\in X$) but
for which there exist sequences with no convergent subsequence.
\end{theorem}
\begin{proof}[\bf Solution] Consider the discrete metric on
${\mathbb Z}$. If $x_{n}=n$ and $x\in {\mathbb Z}$,
then $d(x,x_{n})=1$ for all $n$
with at most one exception. Thus the sequence $x_{n}$
can have no convergent subsequence.
\end{proof}


Fortunately we do have a very neat and useful true theorem.
\begin{definition}\label{D;sequential compactness}
A metric space $(X,d)$ is said to be
sequentially compact if every sequence in $X$ has a
convergent subsequence.
\end{definition}

\begin{theorem}\label{T;sequence same}
A metric space is sequentially compact if and only if it
is compact.
\end{theorem}

We prove the if and only if parts separately.
The proof of the if part is quite simple when you see how.



\begin{theorem}\label{T;compact implies sequence}
If the metric space $(X,d)$ is compact, it is sequentially compact.
\end{theorem}
\begin{proof}[\bf Proof] Let $x_{n}$ be a sequence in $X$. If it has no convergent subsequence, then, for each $x\in X$ we can find a $\delta(x)>0$ and an $N(x)$ such that $x_{n}\notin B(x,\delta(x))$ for all $n\geq N(x)$. Since
\[X=\bigcup_{x\in X}\{x\} \subseteq\bigcup_{x\in X}B(x,\delta(x))\subseteq X,\]
the $B(x,\delta(x))$ form an open cover and, by compactness,
have a finite subcover. In other words, we can find
an $M$ and $y_{j}\in X$ $[1\leq j\leq M]$ such that
\[X=\bigcup_{j=1}^{M}B\big(y_{j},\delta(y_{j})\big).\]

Now set $N=\max_{1\leq j\leq M}N(y_{j})$. Since
$N\geq N(y_{j})$, we have
$x_{N}\notin B\big(y_{j},\delta(y_{j})\big)$
for all $1\leq j\leq M$. Thus
$x_{N}\notin\bigcup_{j=1}^{M}B\big(y_{j},\delta(y_{j})\big)
=X$ which is absurd.

The result follows by reductio ad absurdum.
\end{proof}


The only if part is more difficult to prove (but also, in my opinion, less important). We start by proving a result of independent interest.


\begin{lemma}\label{L;Lebesgue}
Suppose that $(X,d)$ is a sequentially compact
metric space and that the collection $U_{\alpha}$
with $\alpha\in A$ is an open cover of $X$.
Then there exists a $\delta>0$ such that, given any
$x\in X$, there exists an $\alpha(x)\in A$
such that the open ball $B(x,\delta)\subseteq U_{\alpha(x)}$.
\end{lemma}
\begin{proof}[\bf Proof] Suppose the first sentence is true and the
second sentence false. Then, for each $n\geq 1$
we can find an $x_{n}$ such that the open ball
$B(x_{n},1/n)\not\subseteq U_{\alpha}$
for all $\alpha\in A$. By sequential compactness,
we can find $y\in X$ and $n(j)\rightarrow\infty$
such that $x_{n(j)}\rightarrow y$.

Since $y\in X$, we must have $y\in U_{\beta}$
for some $\beta\in A$. Since $U_{\beta}$ is open,
we can find an $\epsilon$ such that
$B(y,\epsilon)\subseteq U_{\beta}$.
Now choose $J$ sufficiently large that $n(J)>2\epsilon^{-1}$
and $d(x_{n(J)},y)<\epsilon/2$. We now have,
using the triangle inequality, that
\[B(x_{n(J)},1/n(J))\subseteq B(x_{n(J)},\epsilon/2)
\subseteq B(y,\epsilon)\subseteq U_{\beta},\]
contradicting the definition of $x_{n(J)}$.

The result follows by reductio ad absurdum.
\end{proof}

We now prove the required result.

\begin{theorem}\label{T;sequence implies compact}
If the metric space $(X,d)$ is sequentially compact, it is compact.
\end{theorem}
\begin{proof}[\bf Proof] Let $(U_{\alpha})_{\alpha\in A}$ be an open cover
and let $\delta$ be defined as in Lemma~\ref{L;Lebesgue}.
The $B(x,\delta)$ form a cover of $X$. If they have no finite
subcover, then given $x_{1}$, $x_{2}$, \dots $x_{n}$
we can find an $x_{n+1}\notin\bigcup_{j=1}^{n}B(x_{j},\delta)$.
Consider the sequence $x_{j}$ thus obtained.
We have $d(x_{n+1},x_{k})>\delta$ whenever $n\geq k\geq 1$
and so $d(x_{r},x_{s})>\delta$ for all $r\neq s$.
It follows that, if $x\in X$, $d(x_{n},x)>\delta/2$
for all $n$ with at most one exception. Thus
the sequence of $x_{n}$ has no convergent subsequence.

It thus follows, by reductio ad absurdum, that the
$B(x,\delta)$ have a finite subcover. In other words,
we can find
an $M$ and $y_{j}\in X$ $[1\leq j\leq M]$ such that
\[X=\bigcup_{j=1}^{M}B(y_{j},\delta).\]
We thus have
\[X=\bigcup_{j=1}^{M}B(y_{j},\delta)
\subseteq \bigcup_{j=1}^{M}U_{\alpha(y_{j})}\subseteq X\]
so $X=\bigcup_{j=1}^{M}U_{\alpha(y_{j})}$ and we have
found a finite subcover.

Thus $X$ is compact.
\end{proof}

This gives an alternative but less instructive
proof of the theorem of Heine--Borel.

\begin{proof}[\bf Alternative proof of Theorem (Heine-Borel)]%
\label{Alternative Heine--Borel}
By the Bolzano--Weierstrass theorem, $[a,b]$ is
sequentially compact. Since we are in a metric space,
it follows that $[a,b]$ is compact.
\end{proof}


If you prove a theorem on metric spaces using
sequential compactness it is good practice to
try and prove it directly by compactness.
(See, for example,
Exercise~\ref{E;uniform continuity}.)

The reader will hardly need to be warned that this chapter
dealt only with metric spaces. Naive generalisations
to general topological spaces are likely to be meaningless
or false.


\section{The language of neighbourhoods}
One of the lines of thought involved in the birth
of analytic topology was initiated by Riemann.
We know that many complicated mathematical structures
can be considered as a space which locally looks like a simpler
space. Thus the surface of the globe we live on is sphere
but we consider it locally as a plane (ie like ${\mathbb R}^{2}$).
The space we live in looks locally like ${\mathbb R}^{3}$
but its global structure could be very different.
For example, Riemann says 'Space would necessarily be
finite if \dots [we] ascribed to it a constant curvature,
as long as that curvature had a positive value, however small.'
[Riemann's discussion
\emph{On the Hypotheses which lie at the Foundations of Geometry}
is translated and discussed in the second volume of Spivak's
\emph{Differential Geometry}.]

Unfortunately the mathematical language of his time
was not broad enough to allow the expression of
Riemann's insights. If we are given a particular
surface such as sphere, it is easy, starting with
the complete structure, to see what 'locally'
and 'resembles' might mean, but, in general, we seem to be stuck
in a vicious circle with 'locally' only meaningful
when the global structure is known and the global structure
only knowable when the meaning of 'locally' is known.

The key to the problem was found by Hilbert who,
in the course of his investigations into the axiomatic
foundations of geometry produced an axiomatisation of
the notion of neighbourhood in the Euclidean plane
${\mathbb R}^{2}$. By developing Hilbert's ideas,
Weyl obtained what is essentially the modern definition
of a Riemann surface (this object, which looks locally like
${\mathbb C}$, was another brilliant creation of Riemann).

However, although the notion of an abstract space with
an abstract notion of closeness was very much in the air,
there were a large number of possible candidates
for such an abstraction. It was the achievement
of Hausdorff to see in Hilbert's work the general notion
of a neighbourhood.

Although Hausdorff defined topologies in terms of neighbourhoods,
it appears to be technically easier to define
topologies in terms of open sets as we have done in this course.
However, topologists still use the notion of neighbourhoods.

We have already defined an open neighbourhood of $x$ to be
an open set containing $x$. We now give the more general
definition.
\begin{definition}\label{D;neighbourhood}
Let $(X,\tau)$ be a topological space.
If $x\in X$ we say that $N$ is a neighbourhood of $x$
if we can find $U\in\tau$ with $x\in U\subseteq N$.
\end{definition}

The reader may check her understanding by proving
the following easy lemmas.


\begin{lemma}\label{L;open topology via neighbourhood}
Let $(X,\tau)$ be a topological space.
Then $U\in\tau$ if and only if, given $x\in U$,
we can find a neighbourhood $N$ of $x$
with $N\subseteq U$.
\end{lemma}
\begin{proof}[\bf Proof]
If $U\in\tau$ then $U$ is a neighbourhood of $x$ for all $x\in U$.

Conversely, if given any $x\in U$, we can find a neighbourhood
$N_{x}$ of $x$ with $N_{x}\subseteq U$, then we can find
an open neighbourhood $U_{x}$ of $x$ with $U_{x}\subseteq N_{x}$.
Since
\[U\subseteq\bigcup_{x\in U}\{x\}
\subseteq\bigcup_{x\in U}U_{x}
\subseteq\bigcup_{x\in U}N_{x}
\subseteq\bigcup_{x\in U}U=U,\]
we have $U=\bigcup_{x\in U}U_{x}\in\tau$.

[Return to page~\pageref{L;open topology via neighbourhood}.]
\end{proof}



\begin{lemma}\label{L;continuous via neighbourhood}
Let $(X,\tau)$ and $(Y,\sigma)$ be topological spaces.
Then $f:X\rightarrow Y$ is continuous
if and only if, given $x\in X$ and $M$
a neighbourhood of $f(x)$ in $Y$,
we can find a neighbourhood $N$ of $x$
with $f(N)\subseteq M$.
\end{lemma}
\begin{proof}[\bf Proof]
\noindent\emph{If} If $f:X\rightarrow Y$ is continuous,
$x\in X$ and $M$ is
a neighbourhood of $f(x)$, then we can find
a $V\in\sigma$ with $f(x)\in V\subseteq M$.
Since $f$ is continuous $f^{-1}(V)\in\tau$.
Thus, since $x\in f^{-1}(V)$, we have that $f^{-1}(V)$
is an open neighbourhood and so a neighbourhood of $x$.
Setting $N=f^{-1}(V)$, we have $f(N)=V\subseteq M$
as required.

\vspace{1\baselineskip}

\noindent\emph{Only if} Suppose that,
given $x\in X$ and $M$
a neighbourhood of $f(x)$ in $Y$,
we can find a neighbourhood $N$ of $x$
with $f(N)\subseteq M$.
Let $V$ be open in $Y$. If $x\in X$
and $f(x)\in V$, then $V$ is a neighbourhood
of $f(x)$ so there exists a neighbourhood $N_{x}$
of $x$ with $f(N_{x})\subseteq V$. We now choose
$U_{x}$ an open neighbourhood of $x$ with $U_{x}\subseteq N_{x}$.
We have
\[f(U_{x})\subseteq V\]
and so $U_{x}\subseteq f^{-1}(V)$ for all $x\in f^{-1}(V)$.
Thus
\[f^{-1}(V)=\bigcup_{x\in f^{-1}(V)}\{x\}
\subseteq \bigcup_{x\in f^{-1}(V)}U_{x}
\subseteq \bigcup_{x\in f^{-1}(V)}f^{-1}(V)
=f^{-1}(V).\]
It follows that $f^{-1}(V)=\bigcup_{x\in f^{-1}(V)}U_{x}\in\tau$.
We have shown that $f$ is continuous.
\end{proof}


\begin{problem}\label{E;non-closed neighbourhood}
(i) If $(X,d)$ is a metric space, show that $N$
is a neighbourhood of $x$ if and only we can find an
$\epsilon>0$ such that the open ball
$B(x,\epsilon)\subseteq N$.

(ii) Consider ${\mathbb R}$ with the usual topology.
Give an example of a neighbourhood which is
not an open neighbourhood. Give an example  of
an unbounded neighbourhood. Give an example  of
a neighbourhood which is not connected.
\end{problem}

We end the course with a warning. Just as it is
possible to define continuous functions in terms
of neighbourhoods so it is possible to define convergence
in terms of neighbourhoods.
This works well in metric spaces.
\begin{lemma}\label{L;convergence, neighbourhood, metric}
If $(X,d)$ is a metric space, then $x_{n}\rightarrow x$,
if and only if
given $N$ a neighbourhood of $x$, we can find an
$n_{0}$ (depending on $N$) such that $x_{n}\in N$
for all $n\geq n_{0}$.
\end{lemma}

However, things are not as simple in general topological spaces.
\begin{definition}{\bf [WARNING. Do not use this definition
without reading the commentary that follows.]}%
\label{R;treacherous}
Let $(X,\tau)$ be a topological space. If $x_{n}\in X$
and $x\in X$ then we say $x_{n}\rightarrow x$
if, given $N$ a neighbourhood of $x$ we can find
$n_{0}$ (depending on $N$) such that $x_{n}\in N$
for all $n\geq n_{0}$.
\end{definition}

Any hopes that limits of sequences
will behave as well in general topological spaces
are dashed by the following example.
\begin{example} Let $X=\{a,\,b\}$ with $a\neq b$.
If we give $X$ the indiscrete topology, then,
if we set $x_{n}=a$ for all $n$, we have
$x_{n}\rightarrow a$ and $x_{n}\rightarrow b$.
\end{example}
Thus limits need not be unique.

Of course, it is possible to persist in
spite of this initial shock, but the reader
will find that she can not prove the links
between limits of sequences and topology
that we would wish to be true. This failure
is not the reader's fault. Deeper investigations
into set theory reveal that sequences are
inadequate tools for the study of topologies
which have neighbourhood systems which are 'large
in the set theoretic sense'. (Exercise~\ref{E;large set}
represents an attempt to show what this means.)
It turns out that the deeper study of set theory
reveals not only the true nature of the problem
but the solution via \emph{nets} (a kind of generalised sequence)
or \emph{filters} (preferred by the majority of mathematicians).





\section{Books}

If the reader looks at a very old book on \emph{general} (or \emph{analytic}) topology, she may find both the language and the contents rather different from what she is used to. In 1955 Kelley wrote a book \emph{General Topology}~\cite{Kelley_1955} which stabilised the content and notation which might be expected in advanced course on the subject. Texts like ~\cite{Mendeleson_1990} (now in a very cheap Dover reprint\footnote{October, 2004.}) and~\cite{Mansfield_1963} which extracted a natural elementary course quickly appeared and later texts followed the established pattern. Both~\cite{Mendeleson_1990}
and~\cite{Mansfield_1963} are short and sweet.

With luck, they should be in your college library. The book of Sutherland~\cite{Sutherland_1975}  has the possible advantage of being written for a British audience and the certain advantage of being in print\footnote{October, 2004.}.

Many books on Functional Analysis, Advanced Analysis, Algebraic Topology and Differential Geometry cover the material in this course and then go on to develop it in the directions demanded by their particular subject.

%%%%%%%%%%%%%%%

%%%%%%%%%%%%%%%%%%%5


\section{Exercises}
\begin{problem}\label{E;axiom grubbing}
Let $X$ be a set and
$d:X^{2}\rightarrow{\mathbb R}$ a function with the
following properties.
\ben
(i) $d(x,x)=0$ for all $x\in X$.

(ii) $d(x,y)=0$ implies $x=y$.

(iii) $d(y,x)+d(y,z)\geq d(x,z)$ for all $x,\,y,\,z\in X$.
\een

\noindent Show that $d$ is a metric on $X$.
\end{problem}
\begin{problem}\label{E;parallelogram} (i) If $V$ is an inner product
space and $\|\ \|$ is the standard norm derived from the inner product,
prove the parallelogram law
\[\|{\mathbf a}+{\mathbf b}\|^{2}+\|{\mathbf a}-{\mathbf b}\|^{2}
=2(\|{\mathbf a}\|^{2}+\|{\mathbf b}\|^{2}).\]

(ii) Give an example of a normed vector space where
their norm can not be derived from an inner product
in a standard way.
\end{problem}
\begin{problem}\label{E;extend composition}
Let ${\mathbb R}^{N}$ have its usual (Euclidean) metric.

(i) Suppose that $f_{j}:{\mathbb R}^{n_{j}}\rightarrow{\mathbb R}^{m_{j}}$
is continuous for $1\leq j\leq k$. Show that the map
$f:{\mathbb R}^{n_{1}+n_{2}+\dots+n_{k}}\rightarrow
{\mathbb R}^{m_{1}+m_{2}+\dots+m_{k}}$
given by
\[f({\mathbf x}_{1},{\mathbf x}_{2},\dots,{\mathbf x}_{k})
=(f_{1}({\mathbf x}_{1}),f_{2}({\mathbf x}_{2}),\dots,f_{k}({\mathbf x}_{k}))\]
is continuous.

(ii) Show that the map $U:{\mathbb R}^{n}\rightarrow{\mathbb R}^{kn}$
given by
\[U({\mathbf x})=({\mathbf x},{\mathbf x},\dots,{\mathbf x})\]
is continuous.

(iii)  Suppose that $g_{j}:{\mathbb R}^{n}\rightarrow{\mathbb R}^{m_{j}}$
is continuous for $1\leq j\leq k$.
Use the composition law to show that the map
$g:{\mathbb R}^{n}\rightarrow
{\mathbb R}^{m_{1}+m_{2}+\dots+m_{k}}$
given by
\[g({\mathbf x})
=(g_{1}({\mathbf x}),g_{2}({\mathbf x}),\dots,g_{k}({\mathbf x}))\]
is continuous.

(iv) Show that the map $A:{\mathbb R}^{2}\rightarrow{\mathbb R}$
given by $A(x,y)=x+y$ is continuous.

(v) Use the composition law repeatedly to show that the map
$f:{\mathbb R}^{2}\rightarrow{\mathbb R}$
given by
\[f(x,y)=\sin\left(\frac{xy}{x^{2}+y^{2}+1}\right)\]
is continuous.

\noindent[If you have difficulty with (v) try smaller subproblems.
For example, can you show that $(x,y)\mapsto x^{2}+y^{2}$ is continuous?]
\end{problem}
\begin{problem}\label{E;nasty discontinuous}
Consider ${\mathbb R}$ with the ordinary
Euclidean metric.

(i) We know that $\sin:{\mathbb R}\rightarrow{\mathbb R}$
is continuous. Show that if $U={\mathbb R}$, then $U$ is open but $\sin U$
is not.

(ii) We define a function $f:{\mathbb R}\rightarrow{\mathbb R}$ as follows.
If $x\in{\mathbb R}$ set $\langle x\rangle=x-[x]$ and write
\[\langle x\rangle=.x_{1}x_{2}x_{3}\dots\]
as a decimal, choosing the terminating form in case of ambiguity.
If $x_{2n+1}=0$ for all sufficiently large $n$, let $N$ be the
least integer such that $x_{2n+1}=0$ for all $n\geq N$, and set
\[f(x)=(-1)^{N}\sum_{j=0}^{\infty}x_{2N+2j}10^{N-j}.\]
We set $f(x)=0$ otherwise.

Show that if $U$ is a non-empty open set,
$f(U)={\mathbb R}$ and so $f(U)$ is open but that $f$ is not
continuous.
\end{problem}
\begin{problem} Let $(X,d)$ be a metric space
and let $r>0$.
Show that
\[\overline{B(x,r)}=\{y\,:\,d(x,y)\leq r\}\]
is a closed set:-

(a) By using the definition of a closed set in terms of limits.

(b) By showing that the complement of $\overline{B(x,r)}$
is open.

We call $\overline{B(x,r)}$ the closed ball centre $x$ and radius
$r$.
\end{problem}
\begin{problem} Prove Theorems~\ref{T;properties metric closed}
and~\ref{T;metric continuous closed}
directly from
the definition of a closed set in terms of limits
without using open sets.
\end{problem}
\begin{problem}\label{E;new metrics}
(i) Let $(X,d)$ be a metric space. Show that
\[\rho(x,y)=\frac{d(x,y)}{1+d(x,y)}\]
defines a new metric on $X$.

(ii) Show that, in~(i), $d$ and $\rho$ have the
the same open sets.

(iii) Suppose that $d_{1}$, $d_{2}$, \dots are metrics
on $X$. Show that
\[\theta(x,y)=\sum_{n=1}^{\infty}
\frac{2^{-n}d_{n}(x,y)}{1+d_{n}(x,y)})\]
defines a metric $\theta$ on $X$.
\end{problem}

\begin{problem}\label{E;inverse}
(This is just intended
to remind of you of some elementary results on maps.)
Let $X$ and $Y$ be non-empty sets and $f:X\rightarrow Y$
be a function. Suppose that $A,\,A'\subseteq X$, $B,\,B'\subseteq Y$,
$A_{\gamma}\subseteq X$, $B_{\gamma}\subseteq Y$
for all $\gamma\in\Gamma$. Which of the following statements
are always true and which may be false?
Give a counter example or a brief explanation in each case.


(i) $f(\bigcup_{\gamma\in\Gamma}A_{\gamma})
= \bigcup_{\gamma\in\Gamma}f(A_{\gamma})$.

(ii) $f(\bigcap_{\gamma\in\Gamma}A_{\gamma})
= \bigcap_{\gamma\in\Gamma}f(A_{\gamma})$.

(iii) $f(A\setminus A')=f(A)\setminus f(A')$.

(iv) $f^{-1}(\bigcup_{\gamma\in\Gamma}B_{\gamma})
= \bigcup_{\gamma\in\Gamma}f^{-1}(B_{\gamma})$.

(v) $f^{-1}(\bigcap_{\gamma\in\Gamma}B_{\gamma})
= \bigcap_{\gamma\in\Gamma}f^{-1}(B_{\gamma})$.

(vi) $f^{-1}(B\setminus B')=f^{-1}(B)\setminus f^{-1}(B')$.

How would your answers change if $f$ was bijective?
\end{problem}
\begin{problem}\label{E;interior}
Let $(X,\tau)$ be a topological space.

(i) Show that, if $E$ is subset of $X$, there exists a unique
open set $V$ such that

\ \ (a) $V\subseteq E$,

\ \ (b) if $U$ is a open set with $U\subseteq E$,
then $U\subseteq V$.

We call $V$ the interior of $E$ and write $\Int E=V$.

(ii) Show by means of an example that the following statement
may be false.

If $E$ is subset of $X$, there exists a unique
open set $V$ such that

\ \ (a) $V\supseteq E$,

\ \ (b) if $U$ is a open set with $U\supseteq E$,
then $U\supseteq V$.

(iii) Show that, if $E$ is subset of $X$, there exists a unique
closed set $F$ such that

\ \ (a) $F\supseteq E$,

\ \ (b) if $G$ is a closed set with $G\supseteq E$,
then $G\supset F$.

We call $F$ the closure of $E$ and write $\Cl E=F$.

(iv) Show that $X\setminus \Int E=\Cl(X\setminus E)$.

(v) If $\tau$ is derived from a metric $d$,
show that the closure of $E$ consists precisely
of those points $x\in X$ such that there exists a sequence
of points $e_{n}$ in $E$
with $d(e_{n},x)\rightarrow 0$ as $n\rightarrow\infty$.
\end{problem}
\begin{problem}\label{E;Infinite product}
(i) Suppose that $A$ is non-empty and that $(X_{\alpha},\tau_{\alpha})$
is a topological space. Explain what is meant by saying that
$\tau$ is the smallest topology on
$\prod_{\alpha\in A}X_{\alpha}$ for which each of the projection maps
$\pi_{\beta}:\prod_{\alpha\in A}X_{\alpha}\rightarrow X_{\beta}$
is continuous and explain why we know that it exists.
We call $\tau$ the product topology.

(ii) Show that $U\in\tau$ if and only if,
given $x\in U$,
we can find $U_{\alpha}\in\tau_{\alpha}$ $[\alpha\in A]$
such that
\[x\in\prod_{\alpha\in A}U_{\alpha}\]
and $U_{\alpha}=X_{\alpha}$ for all but finitely many of the $\alpha$.

(iii) By considering $A=[0,1]$ and taking each $(X_{\alpha},\tau_{\alpha})$
to be a copy of ${\mathbb R}$ show that the following
condition defines a topology $\sigma$
on the space ${\mathbb R}^{[0,1]}$ of functions
$f:[0,1]\rightarrow{\mathbb R}$.
A set $U\in\sigma$ if and only if, given any $f_{0}\in U$,
there exists an $\epsilon>0$
and $x_{1},\,x_{2},\,\dots,\,x_{n}\in[0,1]$ such that
\[\{f\in {\mathbb R}^{[0,1]}\,:\,
|f(x_{j})-f_{0}(x_{j})|<\epsilon
\ \text{for all $1\leq j\leq n$}\}\subseteq U.\]


\noindent[The reader who can not see the point of this topology
is in good, but mistaken, company. The great topologist
Alexandrov recalled that when Tychonov (then aged only~20)
produced this definition 'His chosen \dots definition seemed
not only unexpected but perfectly paradoxical. [I remember]
with what mistrust [I] met Tychonov's proposed definition.
How was it possible that a topology induced by means of
such enormous neighbourhoods, which are only distinguished
from the whole space by a finite number of the coordinates,
could catch any of the essential characteristics of a
topological product?' However, Tychonov's choice was justified
by its consequences, in particular, the generalisation (by Tychonov)
of Theorem~\ref{T;product compact} to show that the
(Tychonov) product of compact spaces is compact.
This theorem called \emph{Tychonov's theorem} is
one of the most important in modern analysis.

In common with many of the most brilliant members
of the Soviet school, Tychonov went on to work in
a large number of branches of pure and applied
mathematics. His best known work includes a remarkable
paper on solutions of the heat
equation\footnote{A substantial part of Volume~22, Number~2
of \emph{Russian Mathematical Surveys} 1967 is devoted
to Tychonov and his work. The quotation from Alexandrov
is taken from there.}.]
\end{problem}
\begin{problem}\label{E;quotient remarks}
(i) Let $X=\{a,\,b\}$ with $a\neq b$. Show that there
does not exist a largest topology contained in
$\sigma=\{\emptyset,\{a\},\{b\},X\}$. (More formally,
show that there does not exist a topology $\tau$
on $X$ such that $\tau\subseteq \sigma$ and
such that, if $\mu$ is any topology with $\mu\subseteq\sigma$,
then $\mu\subseteq \tau$.) Compare and contrast
Lemma~\ref{L;coarsest topology}.

(ii) Show (with the notation of Lemma~\ref{L;quotient lemma})
that the quotient topology on $X/\negthinspace\sim$ is the largest
topology (in the sense of~(i)) such that
$q:X\rightarrow X/\negthinspace\sim$ is continuous.
\end{problem}
\begin{problem}\label{E;quotient non Hausdorff}
Consider ${\mathbb R}$ with the usual (Euclidean)
topology. Let $x\sim y$ if and only if $x-y\in{\mathbb Q}$.
Show that $\sim$ is an equivalence relation. Show that
${\mathbb R}/\negthinspace\sim$ is uncountable but that the quotient
topology on ${\mathbb R}/\negthinspace\sim$ is the indiscrete topology.
\end{problem}
\begin{problem}\label{E;Hausdorff not metric}
(i) If $(X,\sigma)$ is a topology derived from a metric
show that, given $x\in X$, we can find open sets
$U_{j}$ $[1\leq j]$
such that $\{x\}=\bigcap_{j=1}^{\infty}U_{j}$.

(ii) Show, by verifying the conditions for a topological
space directly (so you may not quote Exercise~\ref{E;Infinite product},
that the following condition defines
a topology $\tau$
on the space ${\mathbb R}^{[0,1]}$ of functions $f:[0,1]\rightarrow{\mathbb R}$.
A set $U\in\tau$ if and only if, given any $f_{0}\in U$,
there exists an $\epsilon>0$
and $x_{1},\,x_{2},\,\dots,\,x_{n}\in[0,1]$ such that
\[\{f\in {\mathbb R}^{[0,1]}\,:\,
|f(x_{j})-f_{0}(x_{j})|<\epsilon\ \text{for $1\leq j\leq n$}\}\subseteq U.\]

(iii) Show that the topology $\tau$ is Hausdorff
but can not be derived from a metric.
\end{problem}
\begin{problem}\label{E;Product subspace}
Let $(X,\tau)$ and $(Y,\sigma)$ be topological spaces
with subsets $E$ and $F$. Let the subspace topology
on $E$ be $\tau_{E}$ and  the subspace topology
on $F$ be $\sigma_{F}$. Let the product topology
on $X\times Y$ derived from $\tau$ and $\sigma$
be $\lambda$ and let the product topology
on $E\times F$ derived from $\tau_{E}$ and $\sigma_{F}$
be $\mu$. Show that $\mu$ is the subspace topology on $E\times F$
derived from $\lambda$.
\end{problem}
\begin{problem} (i) Let ${\mathcal H}_{i}$ be a collection of
subsets of $X_{i}$ and let $\tau_{i}$ be the smallest topology
on $X_{i}$ containing ${\mathcal H}_{i}$ $[i=1,2]$. If
$f:X_{1}\rightarrow X_{2}$ has the property that
$f^{-1}(H)\in {\mathcal H}_{1}$ whenever $H\in{\mathcal H}_{2}$,
show that $f$ is continuous
(with respect to the topologies $\tau_{1}$ and $\tau_{2}$).

(ii) Suppose that $(X,\tau)$ and $(Y,\sigma)$ are topological
space and we give $X\times Y$ the product topology.
If $(Z,\lambda)$ is a topological space, show that
$f:Z\rightarrow X\times Y$ is continuous if and only
if $\pi_{X}\circ f:Z\rightarrow X$ and
$\pi_{Y}\circ f:Z\rightarrow Y$ are continuous.

(iii) Let ${\mathbb R}$ have the usual topology (induced
by the Euclidean metric) and let ${\mathbb R}^{2}$
have the product topology (which we know to be
the usual topology induced by the Euclidean metric).
Define
\[
f(x,y)=
\begin{cases}
\frac{xy}{x^{2}+y^{2}}&\text{if $(x,y)\neq (0,0)$,}\\
0&\text{if $(x,y)=(0,0)$.}
\end{cases}
\]
Show that, if we define $h_{x}(y)=g_{y}(x)=f(x,y)$
for all $(x,y)\in{\mathbb R}^{2}$, then the function
$h_{x}:{\mathbb R}\rightarrow{\mathbb R}$ is continuous
for each $x\in{\mathbb R}$ and the function
$g_{y}:{\mathbb R}\rightarrow{\mathbb R}$ is continuous
for each $y\in{\mathbb R}$. Show, however, that $f$ is not
continuous.

\end{problem}
\begin{problem} In complex variable theory we encounter
'uniform convergence on compacta'. This question
illustrates the basic idea basic in the case of
$C(\Omega)$ the space of continuous
functions $f:\Omega\rightarrow{\mathbb C}$
where
\[\Omega=\{z\in{\mathbb C}\,:\,|z|<1\}.\]

(i) Show, by means of an example, that an $f\in C(\Omega)$
need not be bounded on $C(\Omega)$.

(ii) Explain why
\[d_{n}(f,g)=\sup_{|z|\leq 1-1/n}|f(z)-g(z)|\]
exists and is finite for each $n\geq 1$ and all
$f,\,g\in C(\Omega)$. Show that $d_{n}$ satisfies the
triangle law and symmetry but give an example of
a pair of functions
$f,\,g\in C(\Omega)$ with $f\neq g$ yet $d_{n}(f.g)=0$.

(iii) Show that
\[d(f,g)=\sum_{n=1}^{\infty}\frac{2^{-n}d_{n}(f,g)}{1+d_{n}(f,g)}\]
exists and is finite for all
$f,\,g\in C(\Omega)$.

(iv) Show that $d$ is a metric on $C(\Omega)$.

[If you require a hint, do Exercise~\ref{E;new metrics}~(i).]
\end{problem}
\begin{problem} {\bf [This requires Exercise~\ref{E;interior}.]}

(i) Show that the closure of a connected set is connected.

(ii) Deduce that connected components are closed.

(iii) Show that if there are only finitely many
components they must all be open.

(iv) Find the connected components of
\[\{0\}\cup\bigcup\{1/n\,:\,n\geq 1,\ n\in{\mathbb Z}\}\]
with the usual metric.

Which are open and which are not? Give reasons.
\end{problem}
\begin{problem}\label{E;quotient path-connected}
(i) If $(X,\tau)$ and $(Y,\sigma)$ are topological spaces,
$E$ is a path-connected subset of $X$ and $g:E\rightarrow Y$
is continuous, show that $g(E)$ is path-connected.
(More briefly the continuous image of a path-connected
set is path-connected.)

(ii) If $(X,\tau)$ is a path-connected topological space
and $\sim$ is an equivalence relation on $X$,
show that $X/\negthinspace\sim$
with the quotient topology is path-connected.

(iii) If $(X,\tau)$ and $(Y,\sigma)$ are
path-connected topological spaces, show that $X\times Y$
with the product topology is path-connected.

(iv) If $(X,\tau)$ is a
path-connected topological space and $E$ is a subset of $X$,
show that it does not follow that $E$ with the subspace
topology is path-connected.
\end{problem}
\begin{problem}\label{E;uniform continuity}
Suppose that $(X,d)$ is a compact metric space,
$(Y,\rho)$ is a metric space and $f:X\rightarrow Y$
is continuous. Explain why, given $\epsilon>0$,
we can find, for each $x\in X$, a $\delta_{x}>0$
such that, if $d(x,y)<2\delta_{x}$, it follows that
$\rho(f(x),f(y))<\epsilon/2$. By considering
the open cover $B(x,\delta_{x})$ and using compactness,
show that there exists a $\delta>0$ such that
$d(x,y)<\delta$ implies $\rho(f(x),f(y))<\epsilon$.
(In other words, a continuous function from a
compact metric space to a metric space is uniformly continuous.)
\end{problem}
\begin{problem}\label{E;homeomorphic, non-homeomorphic}
Which of the following spaces are
homeomorphic and which are not? Give reasons.

(i) ${\mathbb R}$ with the usual topology.

(ii) ${\mathbb R}$ with the discrete topology.

(iii) ${\mathbb Z}$ with the discrete topology.

(iv) $[0,1]$ with the usual topology.

(v) $(0,1)$ with the usual topology.

[This is rather feeble question but in this short
course we have not found enough topological properties
to distinguish between some clearly distinguishable
topological spaces. We return to this matter
in Exercise~\ref{E;dimension}.]
\end{problem}
\begin{problem}\label{E;dimension}
Suppose that $f:[0,1]\rightarrow{\mathbb R}$
and $g:[0,1]\rightarrow{\mathbb R}$
are continuous maps with $f(0)=-1$, $f(1)=2$,
$g(0)=0$ and $g(1)=1$. Show that
\[f([0,1])\cap g([0,1])\neq\emptyset\]
(In other words, the two paths must cross.)

Show that ${\mathbb R}$ and ${\mathbb R}^{2}$
with the usual topologies are not homeomorphic.
Are $[0,1]$ and the circle
\[\{z\in{\mathbb C}\,:\,|z|=1\}\]
homeomorphic and why?

[But are ${\mathbb R}^{2}$ and ${\mathbb R}^{3}$
homeomorphic? Questions like this form the beginning of modern
algebraic topology.]
\end{problem}
\begin{problem} Which of the following statements
are true and which false. Give a proof or counter-example.

(i) If a topological space $(X,\tau)$ is connected
then the only sets which
are both open and closed are $X$ and $\emptyset$.

(ii) If every set in a topological space $(X,\tau)$
is open or closed (or both) then $\tau$ is the discrete
topology.

(iii) Every open cover of ${\mathbb R}$ with the usual topology
has a countable subcover.

(iv) Suppose that $\tau$ and $\sigma$ are topologies
on a space $X$ with $\sigma\supseteq\tau$. Then,
if $(X,tau)$ is connected, so is $(X,\sigma)$.

(v) Suppose that $\tau$ and $\sigma$ are topologies
on a space $X$ with $\sigma\supseteq\tau$. Then,
if $(X,\sigma)$ is connected, so is $(X,\tau)$.
\end{problem}
\begin{problem}{\bf [Bases of neighbourhoods.]}%
\label{E;bases of neighbourhoods}%
(i) Let $(X,\tau)$ be a
topological space.
Write ${\mathcal N}_{x}$ for the set of neighbourhoods
of $x\in X$. Prove the following results.

\ \ (1) ${\mathcal N}_{x}\neq\emptyset$.

\ \ (2) If $N\in {\mathcal N}_{x}$, then $x\in N$.

\ \ (3) If $N,\ M\in {\mathcal N}_{x}$,
then $N\cap M\in{\mathcal N}_{x}$.

\ \ (4) If $N\in{\mathcal N}_{x}$ and $M\supseteq N$
then $M\in{\mathcal N}_{x}$.

\ \ (5) If $N\in{\mathcal N}_{x}$ then there exists
an $U\in{\mathcal N}_{x}$ such that
$U\subseteq N$ and $U\in{\mathcal N}_{y}$ for all
$y\in U$.

(ii) Suppose that $X$ is a set such that each $x\in X$
is associated with a collection ${\mathcal N}_{x}$
of subsets of $X$. If conditions (1) to (4) of
part~(ii) hold, show that the family $\tau$ of sets
$U$ such that, if $x\in U$, then we can
find an $N\in{\mathcal N}_{x}$ with $N\subseteq U$
is a topology on $X$. If, in addition, condition~(5)
holds show that ${\mathcal N}_{x}$ is the collection
of $\tau$-neighbourhoods of $x$ for each $x\in X$.
\end{problem}
\begin{problem}\label{E;nasty line}
Consider ${\mathbb R}^{2}$
with the usual Euclidean topology.
Let
\[E=\{(x,-1)\,:\,x\in{\mathbb R}\}\cup
\{(x,1)\,:\,x\in{\mathbb R}\}\]
and give $E$ the subspace topology.

Define a relation $\sim$ on $E$ by taking
\begin{align*}
(x,y)\sim(x,y)&\qquad\text{for all $(x,y)\in E$}\\
(x,y)\sim(x,-y)&\qquad\text{for all $(x,y)\in E$ with $x\neq 0$}.
\end{align*}
Show that that $\sim$ is an equivalence relation on $E$.

Now give $E/\negthinspace\sim$ the equivalence relation. Show
that if $[(x,y)]\in E/\negthinspace\sim$ we can find an open neighbourhood
$U$ of $[(x,y)]$ which is homeomorphic to ${\mathbb R}$.
Show, however, that $E/\negthinspace\sim$ is not Hausdorff.

[This nasty example shows that 'looks nice locally'
is not sufficient to give 'looks nice globally'.
It is good start to a course in differential geometry
to ask what extra conditions are required to make
sure that a space that 'looks locally like a line'
'looks globally like a line or a circle'.]
\end{problem}
\begin{problem}\label{E;large set}
Consider the collection $X_{*}$ of all
functions
$f:[0,1]\rightarrow{\mathbb R}$ with $f(x)>0$ for $x>0$,
$f(0)=0$ and $f(x)\rightarrow 0$ as $x\rightarrow 0$.
We take $X=X_{*}\cup\{f_{0}\}$ where $f_{0}$ is
the zero function
defined by $f_{0}(x)=0$ for all $x\in [0,1]$.
If $g\in X_{*}$, write
\[U_{g}=\{f\in X\,:\, f(x)/g(x)\rightarrow 0
\ \text{as $x\rightarrow 0$}\}.\]
Show that, given  $g_{1},\,g_{2}\in X_{*}$ we can find
a $g_{3}\in X_{*}$ such that
\[U_{g_{3}}\subseteq U_{g_{1}}\cap U_{g_{2}}.\]
Conclude that, if $\tau$ consists of $\emptyset$ together
with all those sets $V$ such that $V\supseteq U_{g}$ for
some $g\in X_{*}$, then $\tau$ is a topology on $X$.
Show that
\[\bigcap_{g\in X_{*}}U_{g}=\{f_{0}\}\]

Now suppose $g_{j}\in X^{*}$.
If we set $g(0)=0$ and
\[g(t)=n^{-1}\min_{1\leq j\leq n}g_{j}(t)
\ \text{for $t\in\big((n+1)^{-1},n^{-1}\big]$},\]
show that $g\in X^{*}$ and $g_{j}\notin U_{g}$.
Conclude that, although every open neighbourhood
of $f_{0}$ contains infinitely many points
and the intersection of the open neighbourhoods of
$f_{0}$ is the one point set $\{f_{0}\}$, there is no sequence
$g_{j}$ with $g_{j}\neq f_{0}$ such that
$g_{j}\rightarrow f_{0}$.

[If you just accept this result without thought it
is not worth doing the question. You should compare
and contrast the metric case. I would say that $f_{0}$
is 'surrounded by too many neighbourhood-shells to be approached
by a sequence' but the language of the course is inadequate
to make this thought precise.

I am told that the ancient Greek geometers used a similar
counter example for a related purpose.]
\end{problem}
\begin{problem}\label{E;hyperbolic}
The object of this exercise is to produce an interesting metric
(the Poincar\'{e} metric) associated with complex analysis.

(i) Let $D=\{z\in{\mathbb C}\,:\,|z|<1\}$ Show that, if $|a|<1$
the M\"{o}bius map $T_{a}$ given by
\[S_{a}(z)=\frac{a-z}{1-a^{*}z}\]
maps $D$ to $D$ and interchanges the points $0$ and $1$.

(ii) Show that the only M\"{o}bius maps which take $D$ to $D$
and fix $0$ are rotations.

(iii) Show that the collection ${\mathcal G}$ of
M\"{o}bius maps which take $D$ to $D$ is a subgroup of ${\mathcal M}$.

(iv) Show that the elements of ${\mathcal G}$
are compositions of rotations and maps of the form $S_{a}$.

(iv) Suppose that we seek a metric $d$ on $D$ which is invariant
under M\"{o}bius maps $S$ which take $D$ to $D$
(i.e. such that $d(Sz_{1},Sz_{2})=d(z_{1},z_{2})$
for all $z_{1},\,z_{2}\in {\mathbb C}$
and $S\in {\mathcal G}$). Show that, if we write
$f(x)=d(x,0)$ for $x$ real and $0\leq x<1$, then
\[d(z_{1},z_{2})=f\left(\frac{z_{1}-z}{1-a^{*}z}\right)\]

(v) Let us guess that $d(0,x)+d(x,y)=d(0,y)$ whenever $x$ and $y$
are real and $0\leq x\leq y$ and that $f:[0,1)\rightarrow{\mathbb R}$
is everywhere differentiable (in particular, right differentiable at $0$).
Show that
\[f'(x)=\frac{f'(0)}{1-x^{2}}.\]
Deduce that
\[d(z_{1},z_{2})=A\log\left(
\frac{|1-z_{1}z_{2}^{*}|+|z_{1}-z_{2}|}{|1-z_{1}z_{2}^{*}|-|z_{1}-z_{2}|}
\right)\]
for all $z_{1},\,z_{2}\in D$ and some real constant $A>0$.

(vi) We still have to show that
\[
\rho(z_{1},z_{2})=\log\left(
\frac{|1-z_{1}z_{2}^{*}|+|z_{1}-z_{2}|}{|1-z_{1}z_{2}^{*}|-|z_{1}-z_{2}|}
\right)\]
does, in fact, give a well defined metric on $D$ which is invariant under
M\"{o}bius maps which take $D$ to $D$.
You should find is straightforward to prove all of these
statements with the possible exception of the triangle inequality.

(vii) Here is one way to prove the triangle inequality.
(I make no claim that it is the best.) First show, by
taking a large sheet of paper and doing the algebra, that,
if $R$, $r$ and $\theta$ are real and $1>R,\,r\geq 0$,
then
\[(R+r)^{2}|1-Rre^{i\theta}|^{2}-
(1+Rr)^{2}|R-re^{i\theta}|^{2}\geq 0.\]
Deduce that, if $z_{1},\,z_{2}\in D$, then
\[\frac{|z_{1}|+|z_{2}|}{1+|z_{1}||z_{2}|}
\geq \left|\frac{z_{1}-z_{2}}{1-z_{1}z_{2}^{*}}\right|.\]
Hence show that
\[\log\left(\frac{1+|z_{1}|}{1-|z_{1}|}\right)
+\log\left(\frac{1+|z_{2}|}{1-|z_{2}|}\right)
\geq\log\left(\frac{1+
\left|\frac{z_{1}-z_{2}}{1-z_{1}z_{2}^{*}}\right|}
{1-\left|\frac{z_{1}-z_{2}}{1-z_{1}z_{2}^{*}}\right|}\right)\]
and deduce the triangle inequality.

[This proof is not very illuminating and the usual approach to
the Poincar\'{e} metric is through geodesics.]

(viii) Show that given $a\in D$ and $r>0$
we can find $b\in D$ and $R>0$ such that
\[\{z\,:\,\rho(z,a)=r\}=\{z\,:\,|z-b|=R\}.\]
Sketch $\{z\,:\,\rho(z,a)=r\}$ for a sequence of $r\rightarrow\infty$.

(xi) Let $H=\{z\in{\mathbb C}\,:\,\Re z>0$. Show that
writing
\[d(z_{1},z_{2})=
\log\frac{|z_{1}-z_{2}^{*}|+|z_{1}-z_{2}|}{|z_{1}-z_{2}^{*}|-|z_{1}-z_{2}|}\]
gives a well defined metric $d$
on $H$ which is invariant under
M\"{o}bius maps which take $H$ to $H$.
\end{problem}


\section{Problems}

\begin{problem}
\ben
\item [(a)] Give a bounded open subset of $(\R, \sT_{\text{eucl}})$ which is a not a finite union of open intervals.
\item [(b)] Is the trigonometric function $\sin : (\R, \sT_{\text{Zariski}}) \to (\R, \sT_{\text{Zariski}})$ continuous?
\item [(c)] Give a topology on $\R$ which is neither trivial nor discrete such that every open set is closed (and vice-versa).
\item [(d)] Let $\sT$ be the topology on $\R$ for which open sets are $\phi$, $\R$ and open intervals of the form $(-\infty, a)$. Show this is a topology, and describe the closure of the singleton set $\{a\}$. What are the continuous functions from $(\R,\sT)$ to $(\R, \sT_{\text{eucl}})$?
\item [(e)] Give $\R$ the following topology: a subset $H$ is closed in $\R$ if and only if $H$ is closed and bounded in the usual (Euclidean metric) topology. Show that this is a topology, that points are closed sets, but that this topology is not Hausdorff.
\item [(f)] Define a subset of the integers $\Z$ to be open either if it is empty or if for some $k \in \Z$ the set $S$ contains all integers $\geq k$. Show this defines a topology. Is it metrisable ?
\een
\end{problem}

\begin{solution}[\bf Solution.]
\ben
\item [(a)] An example of a bounded open subset of the real line which is not a finite union of open intervals. The empty set will do. Construction:
\be
U_1:=\bb{\frac 13, \frac 23},\quad U_2:=\bb{\frac 1{3^2}, \frac 2{3^2}},\quad \dots, U_n:=\bb{\frac 1{3^n}, \frac {2^{n-1}}{3^n}}.
\ee

That is, at each step, divide the reference set into three portions. Take the middle portion to define the $U_n$ in the sequence at that stage and take the first third to define the reference set for the next stage.

Define the set $U:= \bigcup^\infty_{n=1}U_N$ then since the union of open sets is open the set $U$ is an open set. By construction $U$ is a subset of the first reference set $[0,1]$ and therefore is bounded. By construction the intervals $U_n$ are all disjoint and therefore $U$ is not a finite union of intervals.

\item [(b)] Claim. The $\sin$ function in the Zariski topology on $\R$ is not continuous.

A set is define as open if it is either empty or the complement is a finite set. Since the topology is not Hausdorff we must use the topological definition of continuity. We use the closed set definition of continuity.

Consider the closed set $\bra{0}$. The inverse image is a countably infinite number of points. This is not a closed set. So the inverse image of a closed set is not closed and therefore the function is not continuous.

Lemma. The Zariski Topology defines a topology on $\R$.

A set is defined as open if it is either empty or the complement is a finite set. The complement of the whole space is the empty set so the whole space is open. We use the dual definition of a topology in terms of closed sets. An arbitrary intersection of finite sets is finite. A finite union of finite sets is finite. Therefore the Zarishi Topology is a topology.

Lemma. The Zariski Topology on $\R$ is not Hausdorff

Let $x$ and $y$ be two distinct points in $\R$ and suppose for contradiction that the space is Hausdorff. Define $U_x$ and $U_y$ disjoint open sets containing $x$ and $y$ respectively. Then $U_x\subseteq R-U_y$ and therefore $U_x$ is a finite set but then $R-U_x$ cannot be finite which means $U_x$ is not open. Contradiction.

Claim. The Zariski Topology on $\R$ is not metrizable.

Let $x$ and $y$ be distinct points and suppose $d(x,y)=\ve$, then
\be
U_x:= \bra{z:d(x,z) < \frac {\ve}4},\quad U_y :=\bra{z:d(y,z) < \frac {\ve}4}
\ee
then both sets are open and separate $x$ and $y$. Let $w\in U_x\cap U_y$ then $d(x,y) \leq d(x,w) + d(w,y) \leq \frac{\ve}2$ therefore no such $w$ exists. So the sets are disjoint and open and separate the two points. Contradiction because the space is not Hausdorff.

\item [(c)] Define $(X,\tau)$ by $X=(0,1)$ and $\tau = \bra{X,\Phi,R-Q,Q}$. The complement of the rationals is the irrationals and so all the open sets are closed and vice versa.

\item [(d)] Claim. This is a topology

The open sets are the sets in the reals of the form $(-\infty,a)$.

Clearly the empty set and the real line satisfy this definition of open set. Finite intersection of intervals of this form are also of this form. Infinite unions of this form are also of this form. Therefore the sets define a topology.

Claim. The closure of the set $\bra{a}$.

The open sets are the sets in the reals of the form $(-\infty,a)$.

The closure is the smallest closed set which contains $\bra{a}$. All the closed sets have the form $[x,\infty)$. So the closure is $[a,\infty)$.

Claim. The continuous functions from $\bra{R,\bra{(-\infty,a)}}$ to $\bra{R,\bra{\tau_{\text{Euclid}}}}$.

We require that the inverse image of every closed interval $[x,y]$ must be closed and so of the form $[a,\infty)$.

Points are closed in the Euclidean metric. So the inverse image of every point must be of the form $[a,\infty)$. Therefore there are no continuous functions.

\item [(e)] The empty is closed and bounded. The whole space $\R$ is closed.

Finite unions of closed and bounded subsets of $\R$ are closed and bounded in the Euclidean topology and therefore are closed in the new topology. Easily proved by induction on the number of sets in the union. Arbitrary intersections of closed and bounded sets on $\R$ are closed and bounded.

Clearly bounded so only need to check closure. Suppose for contradiction that the intersection is not closed. Then there is a Cauchy sequence in the intersection which does not tend to a limit point in the intersection in the Euclidean topology on $\R$. Since $\R$ is complete this sequence has a limit point, say $p$, which does not lie in the intersection of the closed and bounded sets. Consider a small ball around this point. Since $p$ is a limit point there must be an infinite subsequence of the original sequence tending to $p$ and inside one of the closed and bounded sets of the intersection. Contradiction. So the intersection is closed.

Claim. Points are closed.
\be
\bra{a} = \bigcap_{n>1}\bsb{a-\frac 1{2^n}, a+ \frac 1{2^n}}.
\ee

Claim. Topology is not Hausdorff.

All the open sets are unions of sets of the form $(-\infty,a)\cup (b,\infty)$. Therefore any pair of open sets has a non-empty intersection. Therefore the space is not Hausdorff.

\item [(f)] Claim. This define a topology.

Lemma 1. The empty set and the whole space are open by definition of the open sets.

Obvious.

Lemma 2. Arbitrary unions of open sets are open.

Consider any open set in the arbitrary union. For this set there is an integer such that it and all higher integer are in this open set. Therefore these integers are also in the union and the union is open.

Lemma 3. Finite intersection of open sets are open.

Each of the open sets in the interseciton is either the whole of the integers or contains a least integer such that this integer and all higher integers are in the set. Call the set of these least integers $S$. (Ignoring copies of the integers.) Since there are a finite number of sets in the intersection is either the whole set of integers or there is a highest element in $S$. In the latter case all the integers above this highest element of $S$ are in all the sets in the intersection and therefore the intersection is open.

Claim. $X$ is not Hausdorff.

Suppose two points $m$ and $n$ are separated by two open sets. Then by the definition of open there is, in each case, some sufficiently high integer, say $M$ and $N$ respectively, such that all higher integers are in each of the sets. Therefore all integers greater than $M+N+1$ are in both sets and the sets are not disjoint.

Claim. $X$ is not metrisable.

Every metric space is Hausdorff so this space cannot be metrisable.
\een
\end{solution}


\begin{problem}
Which of the following are open in $(\R^2, \sT_{\text{eucl}})$? [Convincing pictures are sufficient.]
\ben
\item [(i)] $\{y > x^2\}$
\item [(ii)] $\{y > x^2, y \leq 1\}$
\item [(iii)] $\{y > x^2, y \leq -1\}$.
\een
\end{problem}

\begin{solution}[\bf Solution.]
\ben
\item [(i)] Open.
\item [(ii)] Not open.
\item [(iii)] Open because empty.
\een
\end{solution}


\begin{problem}
Prove or give counterexamples to:
\ben
\item [(i)] A continuous function $f : X \to Y$ is an \emph{open map} i.e. if $U \subset X$ is an open subset then $f(U)$ is an open subset of $Y$.
\item [(ii)] If $f : X \to Y$ is continuous and bijective (that is, one-to-one and onto) then $f$ is a homeomorphism.
\item [(iii)] If $f : X \to Y$ is continuous, open and bijective then $f$ is a homeomorphism.
\een
\end{problem}

\begin{solution}[\bf Solution.]
\ben
\item [(i)] Claim. image of an open set under a continuous mapping need not be open.

Note: If the image of every open set is open then the mapping is called an open mapping. All complex analytic maps are open mappings.

Example of a continuous mapping which is not an open mapping: $f,\R\to \R:x\mapsto \cos x$.

The set $(-2\pi,2\pi)$ is open but the image is $[-1,1]$ closed. ALT: $f,\R\to \R:x\mapsto x^2$ then $f((-1,1)) = [0,1)$.

\item [(ii)] Claim. A continuous bijection between topological spaces need not be a homeomorphism.

For a homeomorphism we requre the two sets to be indistinguishable at the topological level i.e., we require the open sets to be 'the same'. We require the inverse image of every open set to be open under the homeomorphism and also under its inverse. i.e., we require that the homeomorphism be continuous, bijective and an open mapping. Equivalently we require the homeomorphism to be a continuous bijection and have a continuous inverse.

Example: Of a continuous bijection between topologically inequivalent spaces $f$
\be
[0,2\pi) \to S^1:x\mapsto e^{ix}.
\ee

The inverse map is not continuous at 1, for example $f$
\be
[0,2\pi)\mapsto \bra{e^{ix}: \arg(e^{ix}\in [0,\pi))}.
\ee

Claim. A continuous bijection between topological spaces need not be a homeomorphism.

Let $X$ be the discrete topology on the space with $n$ elements and let $Y$ be the same space of $n$ elements with the indiscrete topology. Then the identity map $i$,
\be
X\to Y : x\mapsto x
\ee
is a continuous bijection but the inverse map is not continuous so the map is not a homeomorphism.

\item [(iii)] Since the map is bijective the inverse map is well defined and onto. Since the map is open then inverse image map is continuous. Therefore the open sets in each space are mapped to open sets in the other space.

Topologically the two spaces are indistinguishable and therefore the map is a homeomorphism.
\een
\end{solution}


\begin{problem}
Let $X$ and $Y$ be topological spaces.
\ben
\item [(a)] If $X = A \cup B$ is a union of (not necessarily disjoint) closed subsets, prove that a function $f$ on $X$ is continuous if and only if $f|_A$ and $f|_B$ are continuous functions on $A$, $B$ respectively, where $A$, $B$ have the subspace topology induced from $X$. (The notation refers to the restriction of the function to the appropriate domain.)
\item [(b)] Show $f : X \to Y$ is continuous if and only if for all $A\subset X$, $f\bb{\cl{A}} \subseteq \cl{f(A)}$. Deduce that if $f$ is surjective, the continuous image of a dense set is dense.
\item [(c)] If now $X$ and $Y$ are metric spaces, show $f : X \to Y$ is continuous if it preserves limits of sequences, i.e. if for every sequence $(x_n) \subset X$ converging to $a \in M_1$, the sequence $(f(x_n)) \subset Y$ converges to $f(a)$.
\een
\end{problem}

\begin{solution}[\bf Solution.]
\ben
\item [(a)] Lemma 1. The restrictions of a continuous map are themselves continuous.

The open sets in the subspace topology are given by the intersections of the subspace with open sets in the original space. Now consider the complement in the subspace topology of an open set of the form $Y\cap A = Y-(Y\cap A) = Y\cap (X-A)$.

Therefore we can also describe the subspace topology in terms of the dual concept of closed sets since $Y$ is both open and closed. Therefore the closed sets in the subspace topology are all the original closed sets of $Y$ and all intersection of closed sets in $X$ with $Y$. Since $Y$ is closed in $X$ these sets are also closed in $X$. Therefore using the closed set definition of continuity it follows that the two restrictions are also continuous.

Lemma 2. If the restrictions of a continuous map are continuous then the map is continuous.

The closed sets in the subspaces are also closed in the topology on $X$. We need only consider closed subsets made from finite unions of closed sets from the two different subspaces. Suppose the map fails to be continuous at some point $p$. Then the map fails to be continuous on some small closed set which contains $p$ and is contained in one of the subspaces. This is a contradiction. So the map is continuous.

\item [(b)] Claim. For a function $f:X\to Y$, the following three statements are equivalent
\ben
\item [(i)] $f$ is continuous, i.e., the inverse image of an open set is open.
\item [(ii)] The preimages of closed subsets in $Y$ are closed in $X$.
\item [(iii)] For all $A\subseteq X$, $f\bb{\cl{A}} \subseteq \cl{f(A)}$.
\een

Lemma 1. (i) $\ra$ (ii).

Let $V$ be a closed set in $Y$. The complement $U$ in $Y$ is open in $Y$ and its inverse image $f^{-1}(U)$ under $f$ is open in $X$. Now consider a point $x$ in the complement of $f^{-1}(U)$, its image is not in $U$ and therefore must be in $V$. So the inverse image of points in $V$ are in the complement of $f^{-1}(U)$ which is a closed set.

Therefore the inverse image of a closed set is closed.

Lemma 2. (ii) $\ra$ (iii).

Let $p$ be a point in the image. (Note that we do not assume points are closed.) Let $f(B) = p$. By continuity the inverse image of the closure of $p$ is a closed set $C$. So $f(C) = \cl{p}$.

Clearly $B\subseteq C$ and $\cl{B}\subseteq \cl{C} = C$. So $f\bb{\cl{B}} \subseteq f(C) = \cl{p} = \cl{f(B)}$. Hence for any set we have $f\bb{\cl{A}} \subseteq \cl{f(A)}$.

We should now show that (iii) $\ra$ (i) but it is easier to go via (ii).

Lemma 3. (iii) $\ra$ (ii).

Let $B$ be a closed set and let the inverse image be $C = f^{-1}(B)$. We need to show that $C$ is closed. Use
\be
f\bb{\cl{C}} \subseteq \cl{f(C)} \ \ra \ f\bb{\cl{f^{-1}(B)}} \subseteq \cl{f(f^{-1}(B))} = \cl{B} \ \ra \ f\bb{\cl{f^{-1}(B)}} \subseteq \cl{B}.
\ee

Hence
\be
\cl{f^{-1}(B)}\subseteq f^{-1}\bb{\cl(B)} \ \ra \ \cl{f^{-1}(B)} \subseteq f^{-1}(B) \ \ra \ \cl{C} \subseteq C \ \ra \ C \text{ is closed.}
\ee

Lemma 4. (ii) $\ra$ (i).

Let $B$ be a open set and let the inverse image be $C = f^{-1}(B)$. We need to show that $C$ is open. The complement of $B$ is a closed set and its inverse image is closed. Taking complements we have, $X-f^{-1}(X-B)$ is open. Any point, $p$ which is not in $B$ has an inverse image in $f^{-1}(X-B)$. So a function between metric spaces is continuous iff it preserves limits of sequences.

Claim. The function is continuous, $f:X\to Y$ iff for every sequence $x_n \to x$ then $f(x_n) \to f(x)$.

Assume for every sequence $x_n \to x$ then $f(x_n)\to f(x)$. Suppose for contradiction that there is some sequence such that $x_n \to x$ but $f(x_n) \nrightarrow f(x)$ then there is an $\ve>0$ such that $d\bb{f(x_n),f(x)} > \ve$, but by definition of continuous for all $\ve >0$, including this one, there is a $\delta >0$ such that $d(x_n,x)<\delta$ implies $d\bb{f(x_n),f(x)} < \ve$. Contradiction.

Assume $f:X\to Y$ is continuous. Then by definition for any $\ve>0$ there is a $\delta >0$ such that $d(x_n,x)<\delta$ implies $d\bb{f(x_n),f(x)} <\ve$ as required.

\item [(c)] Assume for every sequence $x_n \to x$ then $f(x_n) \to f(x)$.

Suppose for contradiction that there is some sequence such that $x_n \to x$ but $f(x_n) \nrightarrow f(x)$ then there is an $\ve>0$ such that $d\bb{f(x_n),f(x)} >\ve$, but by definition of continuous for all $\ve>0$, including this one, there is a $\delta>0$ such that $d(x_n,x) <\delta$ implies $d\bb{f(x_n),f(x)} <\ve$. Contradiction.

Assume $f:X\to Y$ is continuous. Then by definition for any $\ve>0$ there is a $\delta >0$ such that $d(x_n,x)<\delta$ implies $d\bb{f(x_n),f(x)} < \ve$ as required.

\een
\end{solution}


\begin{problem}
Let $f, g : X \to Y$ be continuous functions where $X$ is any topological space and $Y$ is a Hausdorff topological space. Prove that $W = \{x \in X | f(x) = g(x)\}$ is a closed subspace of $X$. Deduce that the fixed point set of a continuous function on a Hausdorff space is closed.
\end{problem}

\begin{solution}[\bf Solution.]
Assume for contradiction that there is a sequence of points in $W$ whose limit point, say $y$, is not in $W$. Since $f(y) \neq g(y)$ and the image space $Y$ is Hausdorff we can find disjoint open sets, $U$ and $V$, such that $f(y)\in U$ and $g(y)\in V$. Since $f$ is continuous every open neighbourhood of $f(y)\in U$ contains images of the sequence of points in $W$ which converge to $y$. But points in $W$ are mapped to the same image by both functions. Therefore they cannot be elements of two disjoint open sets. Therefore $W$ is closed.

Claim: Trivially by taking one of the maps to be the identity it follows that the fixed point set of a continuous function on a Hausdorff space is closed.
\end{solution}


\begin{problem}
Let $X = Z_{>0}$ be the set of strictly positive integers. Define a topology on $X$ by saying that the basic open sets are the arithmetic progressions
\be
U_{a,b} = \{na + b \in X | n \in Z\}
\ee
for pairs $(a, b)$ with $\hcf(a, b) = 1$. (Hence, a general open set is a union of these.)
\ben
\item [(a)] Show this does define a topology. Is it Hausdorff?
\item [(b)] If $p$ is prime, show that $U_{p,0}$ is a closed subset of $X$. Deduce there are infinitely many prime numbers.
\een
\end{problem}

\begin{solution}[\bf Solution.]
\ben
\item [(a)] Let $a=1$ and $b=0$ so the whole space is open. $U_{10,3}\cap U_{10,7} = \emptyset$ so the empty set to be open. We claim the arbitrary unions of open sets is open. We claim the finite intersections $\bigcap_{a,b}\bra{na+b}$ of open sets are open.

Note that as above the $a$ values of the elements of the intersection must be distinct otherwise the intersection will be empty or the sets will be idential.
\be
\bra{na+b}\cap \bra{na+B} = \emptyset,\quad b\neq B,\ b\neq B (\bmod a).
\ee

We claim the finte intersection $\bigcap_{a,b}\bra{na+b}$ is open by the Chinese Remainder Theorem for such distinct '$a$ values'.

Incongruent sets will be open because their intersection is empty. For example $U_{3,1}\cap U_{6,5} =\emptyset$. If we assume, therefore, that the sets are not incogruent then we only have to check that for our intersection set $\bigcap_{a,b}\bra{na+b} = (nA+B)$ we have $(A,B) = 1$.

To see this consider the pair $\bra{na+b}$ and $\bra{nc+d}$ with $(a,b)=1$ and $(c,d) =1$. Now
\be
\bra{na+b} \cap \bra{nc+d} = \bra{x\in X:x\equiv b (\bmod a),x\equiv d(\bmod c)}
\ee
so $xa+b = yc + d$ with integer $x$ and $y$. Hence $xa-yc = -d + b$ but we have constructed this so that it is solvable. We can find $x_0,y_0$ such that $x_0 a - y_0 c = -d +b$ is solved for these. So for any $n\in \bra{na+b}\cap \bra{nc +d}$
\be
n = b+ a\bra{x_0 + \frac{\lcm(a,c)k}{\hcf(a,c)}},\quad k\in \Z
\ee
We have
\be
n \equiv b+ ax_0 \bmod(a,c) = \equiv d + cx_0 \bmod(a,c).
\ee

We need to show $(b+ax_0,(a,c)) = 1$. Now $a|(a,c)$ but $a\nmid (b+ax_0)$ and $(a,b) =1$. Similarly for all divisors of $a$. $a\mid (a,c)$ and similarly for all the divisors of $a$. So $a\mid (a,c)$ and $c\nmid (d+cx_0)$.

Claim. The topology is not Hausdorff.

Given $P$ and $3P$ with $P$ prime. $\bra{n(3P)}\cap \bra{n(P-1)+1}$.

\item [(b)] The set, $\bra{np}$ is closed.

Consider $p$ prime. We have $(p,k) =1$, $\forall k \in X-k = np$ with $n$ in $X$. Then $\bra{np+1},\dots,\bra{np + (p-1)}$ are open so the union is also open.
\be
\bigcup_{k=1}^{p-1} \bra{np + k} \text{ is open } \ \ra \ X- \bigcup_{k=1}^{p-1}\bra{np+k}\text{ is closed.}
\ee

Suppose for contradiction that there are finitely many primes. Then they can be listed in order of size. Now consider the product $P = p_1\dots p_n + 1$ with $(P,p_k)$ for each $k$ in the list. Note that $\bigcup_{k=1}^{p-1}\bra{nP + p_k}$ is open.

Similarly all the other $k$s, numbers less than $P$ must also be composed of $\bra{pn + k}$ open for all $k$ less than $P$. Therefore we have a set, say, $\bra{pn + k}$ open for all $k$. So $\bra{pn + 0}$ is closed. Hence $P$ is prime. So the number of primes is infinite.

\een
\end{solution}


\begin{problem}
\ben
\item [(a)] Show the quotient space $([0, 1] \cup [2, 3])/1 \sim 2$ is homeomorphic to a closed interval.
\item [(b)] Define an equivalence relation $\sim$ on the interval $[0,1] \subset \R$ by $x \sim y \Longleftrightarrow x - y \in \Q$. Describe the quotient space $I/ \sim$.
\een
\end{problem}

\begin{solution}[\bf Solution.]
\ben
\item [(a)] $f:[0,1]\cup [2,3] \to [0,1]$
\be
x\in [0,1]:x\mapsto \frac x2,\qquad x\in [2,3]:x\mapsto \frac {x-1}2
\ee

The map is a continuous bijection with a continuous inverse.

\item [(b)] The rationals are in the kernel of the quotient map. The equivalence classes are indexed by a subset of the irrationals, since irrationals which differ by a rational are in the same class. By construction the projection map is continuous so we are looking for the largest number of open sets in the quotient such that their inverse images are open in the original space. The inverse image of a single point in the quotient is that point and then all the points which differ from it by a rational. This set is dense in $[0,1]$. Any non-empty open set in the quotient must contain points so the inverse image must contain such a dense set and be such that all the points are interior points. The samllest open set which contains an everywhere dense set in the space is the whole space. So the only open sets are the whole quotient space and the empty set as required.
\een
\end{solution}


\begin{problem}
Let $S_1$ be the quotient space given by identifying the north and south poles on the 2-sphere. Let $S_2$ be the quotient space given by collapsing one circle $S^1\times \{pt\}$ inside the 2-dimensional torus $S^1 \times S^1$ to a point. Draw pictures of $S^1$ and $S^2$ and sketch an argument to show that they are homeomorphic.
\end{problem}

\begin{solution}[\bf Solution.]

\centertexdraw{
\move (0 0) \lcir r:0.4 % circle
%\htext(1.55 0){$S_i$}
\htext(-0.1 -0.6){sphere}

\htext(0.6 0){$\Longrightarrow$}

\move (1.5 0)
\larc r:0.4 sd:20 ed:340
\larc r:0.15 sd:60 ed:300
\move(1.575 0.13) \clvec (1.6 0.1)(1.8 0)(1.87 0.13)
\move(1.575 -0.13) \clvec (1.6 -0.1)(1.8 0)(1.87 -0.13)

\htext(2.2 0){$\Longrightarrow$}

\move (3.15 0) \lcir r:0.25 % circle
\move (3 0) \lcir r:0.4 % circle
\htext(2.8 -0.6){The space $S^1$}


\move (0 -1.2) \lcir r:0.4 % circle
\move (0 -1.2) \lcir r:0.2 % circle
%\htext(1.55 0){$S_i$}
\htext(-0.1 -1.8){torus}

\htext(0.7 -1.2){$\Longrightarrow$}

\move (1.65 -1.2) \lellip rx:0.25 ry:0.15
\move (1.5 -1.2) \lcir r:0.4 % circle
\htext(1.3 -1.8){The space $S^2$}

}

Both identifications give a torus with one circle identified to a point.
\end{solution}


\begin{problem}
\ben
\item [(a)] Prove that the product of two metric spaces admits a metric inducing the product topology.
\item [(b)] Show that if the product of two metric spaces is complete, then so are the factors.
\item [(c)] Let $M$ denote the space of bounded sequences of real numbers with the \emph{sup} metric. Show
\ben
\item [(i)] the subspace of convergent sequences is complete and
\item [(ii)] the subspace of sequences with only finitely many non-zero values is not complete.
\een
\item [(d)] Let $M$ be a complete metric space and $f : M \to M$ be continuous. Suppose for some $r$ the iterate $f^r = f \circ \dots \circ f$ ($r$ times) is a contraction. Prove $f$ has a unique fixed point.
\een
\end{problem}

\begin{solution}[\bf Solution.]
\ben
\item [(a)] The product topology is defined by a basis consisting of products of open sets in the two metric spaces. So an open set in the product topology may be written as a union of products of open sets in the two spaces. Let the two metric spaces be $\bra{X,d(,)}$ and $\bra{Y,\eta(,)}$. We form $\bra{X\times Y, d(,)+\eta(,)}$.

Trivially, since in the metric terms open balls are defined by the less than relation, using the metric definition, all the products in the basis are open with respect to the new metric as well.

Note: The choice of metric on the product topology is not unique and it is possible to have district metrics with the same underlying topology. We could have use $\bra{X\times Y, \sqrt{d(,)^2 + \eta(,)^2}}$. (Think in terms of the analogy with the product topology on the Euclidean plane arising as the product of two Euclidean real lines.)

\item [(b)] $\bra{X\times Y, d(,)+\eta(,)}$. Let $\bra{(x_i,y_i)}_{i>0}$ be a Cauchy sequence in the product space then $(x_i,y_i) \to (x,y)$ as $i\to \infty$. From the definition of the metric above it follows that both $\bra{x_i}_{i>0}$ and $\bra{y_i}_{i>0}$ are Cauchy in $X$ and $Y$. Trivially $x\in X$ and $y\in Y$ and $d(x_i,x)\to 0$ as $i \to \infty$ and $\eta(y_i,y)\to 0$ as $i\to \infty$.

So the two corresponding sequences in the factors are convergent. Now consider a Cauchy sequence in one of the factors, suppose $d(x_i,x_{i+j}) \to 0$ as $i \to \infty$. We need to prove that the sequence is convergent in $\bra{X,d(,)}$. Consider the sequence $\bra{(x_i,y)}_{i>0}$ where $y$ is in a constant sequence. This is Cauchy in the product space because the contribution from the second metric is zero as the sequence is constant on that component. It follows that this sequence is convergent and therefore by the first part is follows that the original Cauchy sequence in $X$ is convergent to a point in $X$.

\item [(c)] Let $M$ be the space of bounded real sequences with the sup metric. Let $a$ and $b$ be points in $M$, then $d_M(a,b) = \sup_{i>0}\abs{a_i-b_i}$ where the sup is over elements in the sequence.

Claim. The subspace of convergent sequences is complete.

Let $\bra{a(j)}_{j>0}$ be Cauchy sequence in $M$.
\be
\ba{cccccc}
a(1)_1 & a(2)_1 & a(3)_1 & \dots & a(j)_1 & \dots\\
a(1)_2 & a(2)_2 & a(3)_2 & \dots & a(j)_2 & \dots\\
a(1)_3 & a(2)_3 & a(3)_3 & \dots & a(j)_3 & \dots\\
\vdots & \vdots & \vdots & \vdots & \vdots & \vdots \\
a(1)_k & a(2)_k & a(3)_k & \dots & a(j)_k & \dots\\
\vdots & \vdots & \vdots & \vdots & \vdots & \vdots \\
\ea
\ee

Since the sequence is Cauchy we have $\abs{a(j)_i - a(k)_i}\leq \sup_{i>0}\abs{a(j)_i - a(k)_i}$ so the horizontal sequences are also Cauchy in $\R$. Since $\R$ is complete the horizontal sequences are convergent in $\R$. We claim that the limit sequence defined in this way is the limit in $M$ of the Cauchy sequence in $M$.

\be
\ba{ccccccc}
a(1)_1 & a(2)_1 & a(3)_1 & \dots & a(j)_1 & \dots & a(\infty)_1 \\
a(1)_2 & a(2)_2 & a(3)_2 & \dots & a(j)_2 & \dots & a(\infty)_2\\
a(1)_3 & a(2)_3 & a(3)_3 & \dots & a(j)_3 & \dots & a(\infty)_3\\
\vdots & \vdots & \vdots & \vdots & \vdots & \vdots \\
a(1)_k & a(2)_k & a(3)_k & \dots & a(j)_k & \dots & a(\infty)_k\\
\vdots & \vdots & \vdots & \vdots & \vdots & \vdots \\
\ea
\ee

Clearly, by construction, the limit sequence is a bounded sequence and so is an element in $M$.
\be
\sup_{i>0}\abs{a(\infty)_i - a(k)_i} \leq \abs{a(\infty)_i - a(k)_i}
\ee
for some $j$ so since the horizontal sequence is Cauchy and convergent we have $\sup_{i>0}\abs{a(\infty)_i - a(k)_i} \to 0$ as $i\to \infty$.

Claim. The subspace of sequences with only a finite number of non-zero terms is not complete.

Let $\bra{a(j)}_{j>0}$ be a Cauchy sequence in $M$. The first $j$ elements of the $j$th sequence take the value $1/j$, the remaining elements take the value zero. Each element of the sequence in $M$ has a finite number of non-zero elements. The horizontal limits are clearly $1/j$ in the $j$th entry and the sequence is Cauchy in $M$ since for $j<k$ we have $\sup_{i>0}\abs{a(j)_i - a(k)_i} = \abs{\frac 1k - 0} \to 0$ as $i\to \infty$.
\be
\ba{ccccccc}
1 & 1 & 1 & \dots & 1 & \dots & 1 \\
0 & \frac 12 & \frac 12 & \dots & \frac 12 & \dots & \frac 12 \\
0 & 0 & \frac 13 & \dots & \frac 13 & \dots & \frac 13\\
\vdots & \vdots & \vdots & \vdots & \vdots & \vdots \\
0 & 0 & 0 & \dots & \frac 1j & \dots & \frac 1j\\
\vdots & \vdots & \vdots & \vdots & \vdots & \vdots \\
\ea
\ee

The limit sequence would have to be the harmonic sequence but that has an infinite number of non-zero terms.

\item [(d)] Since $f^r$ is a contraction on a complete metric space it has a unique fixed point. (See lecture notes for proof.) Therefore there is a unique $x$ such that $f^r(x) = x$.
\be
f^{r+1}(x) = f^r(f(x)) = f(x).
\ee

So $f(x)$ is a fixed point of $f^r$ but by uniqueness $f(x)=x$ as required.

\een
\end{solution}


\begin{problem}
Let $G$ be a topological group, so $G$ is a topological space and there are given a distinguished point $e \in G$, continuous functions $m : G \times G \to G$ and $i : G \to G$ (multiplication and inverse) which satisfy the (usual, algebraic) group axioms. Typical examples are matrix groups like $SL_2(\R)$, $SO(3)$, the unit circle in $\C$, etc. Prove the following:
\ben
\item [(a)] $G$ is homogeneous: given any $x, y \in G$ there is a homeomorphism $\phi : G \to G$ such that $\phi(x) = y$.
\item [(b)] If $\{e\} \subset G$ is a closed subset, then the diagonal $\triangle G = \{(g, g) | g \in G\} \subset G \times G$ is a closed subgroup of $G \times G$.
\item [(c)] If $\{e\}$ is closed in $G$ then the centre $Z(G) = \{g \in G | gh = hg \forall h \in G\}$ is a closed normal subgroup of $G$.
\item [(d)] Let $H$ be an algebraic subgroup of $G$. Give the set of cosets $(G:H)$ the quotient topology from the natural projection map $\pi : G \to (G : H)$. Prove that $\pi$ is an open map (images of open sets are open).
\item [(e)] Prove $(G : H)$ is Hausdorff if and only if $H$ is closed in $G$.
\een
\end{problem}

\begin{solution}[\bf Solution.]
\ben
\item [(a)] Consider the multiplication map in the group. We are given $x$ and $y$ then want $x*z = y$ define $\vp(x) = x*x^{-1}*y = y$.

\item [(b)] We can define the maps on the element $\vp(e,e):= (\vp(e),\vp(e))$. Then from the first part we can reach any element in the group by a suitable choice of map. Hence the diagonal is subgroup. Since $e$ is closed and since the maps are homeomorphisms the image of a closed set is closed so all the images are closed sets. Hence the diagonal is itself closed in $G\times G$. To see that the diagonal is closed consider any 'Cauchy' sequence in the diagonal. It has a subsequence with an infinite number of points in one of the image sets of the identity. By the closure of this set the sequence has a limit in the set. Therefore all the limit points of sequences are in the set and the diagonal is closed.

\item [(c)] The centre is normal subgroup by the results of IA Groups. Since the identity is in the subgroup and it is a closed set we know that all its images under homeomorphisms are closed. By similar argument to that above we know the centre is closed.

\item [(d)] The projection map is defined from a homeomorphism and therefore is an open mapping.

\item [(e)] If $H$ is closed then the cosets are closed sets. Argue similarly to next question.
\een
\end{solution}


\begin{problem}
Let $X$ be a normal topological space: given disjoint closed sets $A,B$ in $X$ we may separate them by disjoint open sets $U\supseteq A$, $V\supseteq B$. Assume moreover that the singleton points of $X$ are closed sets. Let $W\subseteq X$ be closed. Prove that the quotient space $X/W$ (quotienting by the equivalence relation which collapses $W$ to a point) is Hausdorff.
\end{problem}

\begin{solution}[\bf Solution.]
The map from the space $X$ to the quotient is the identity on points not in $W$ and projection onto a single point say $w$ on point in $W$.
\beast
& & X\to X/W\\
x\notin W & & x\mapsto x\\
x\in W & & x\mapsto w
\eeast

Consider two distinct points $x$ and $y$ in the image. Let $\tilde{x}= pr^{-1}(x)$ and $\tilde{y}= pr^{-1}(y)$. Without loss of generality we may assume that $\tilde{x}\mapsto x$ so the inverse image is a single point. $\tilde{y}$ is either a single point or the whole of $W$. We can consider both cases by forming $Z = \bra{\tilde{y}}\cup W$. Note that points in $X$ are closed so this union is also closed. Since the space is normal we can find open disjoint sets say $U$ and $V$ in $X$ such that $\bra{\tilde{x}}\subseteq U$ and $Z = \bra{\tilde{y}}\cup W \subseteq V$. Since the projection map is the identity on everything except $W$, the images are disjoint in the quotient. The topology on the quotient is given by the largest number of open sets such that the projection is continuous. Therefore the two projected open sets have images which are open. (We are NOT assuming the projection is an open mapping). Hence we have a disjoint union of open sets in the quotient which separates the two points. Therefore the space is Hausdorff.
\end{solution}


\begin{problem}
\ben
\item [(a)] Show that a space $X$ may be homeomorphic to a subspace of a space $Y$ whilst $Y$ is homeomorphic to a subspace of $X$, but where $X$, $Y$ are not themselves homeomorphic.
\item [(b)] Give an example of a pair of spaces $X$, $Y$ which are not homeomorphic but for which $X \times I$ and $Y \times I$ are homeomorphic, where $I$ denotes the unit interval with its usual topology. [Hint: draw a flat handbag, i.e. two arcs attached to a disk.]
\een
\end{problem}

\begin{solution}[\bf Solution.]
\ben
\item [(a)] $X=[0,1]$ and $Y = \R$.

A subspace of $X$ is $(0,1)$ this is homeomorphic to $\R$ the real line.

A subspace of $Y$ is $[0,1]$ this is homeomorphic to $X = [0,1]$.

\item [(b)] But there is no bijection between a one point and a two point set so $\bra{p}$ and $\bra{p,q}$ are not homeomorphic.
\een
\end{solution}

\begin{problem}
Which of the following subspaces of $\R^2$ are (a) connected (b) path-connected? $B_{(x,y)}(t)$ denotes the open $t$-disc about $(x, y) \in \R^2$ and $\ol{X} = cl(X)$ denotes closure.
\ben
\item [(i)] $B_{(1,0)}(1) \cup B_{(-1,0)}(1)$,
\item [(ii)] $\ol{B_{(1,0)}(1)} \cup \ol{B_{(-1,0)}(1)}$
\item [(iii)] $B_{(1,0)}(1) \cup \ol{B_{(-1,0)}(1)}$
\item [(iv)] $\{(x, y) | x = 0 \text{ or }y/x \in \Q\}$.
\een
\end{problem}

\begin{solution}[\bf Solution.]
\ben
\item [(i)] Mutually tangent open discs $B_1(0,1)\cup B_1(0,-1)$. It is not connected and therefore not path connected.
\item [(ii)] Mutually tangent closed discs $\ol{B_1(0,1)}\cup \ol{B_1(0,-1)}$. It is path connected and therefore connected.
\item [(iii)] Mutually tangent discs $\ol{B_1(0,1)}\cup B_1(0,-1)$. It is path connected and therefore connected.
\item [(iv)] $\bra{(x,y)\in R^2:x=0 \text{ or } y/x \in \Q}$. It is path connected and therefore connected.
\een
\end{solution}


\begin{problem}
\ben
\item [(a)] Let $\phi : [0, 1] \to [0, 1]$ be continuous. Prove that $\phi$ has a fixed point.
\item [(b)] Prove that an odd degree real polynomial has a real root.
\item [(c)] Let $\mathbb{S}^1 \subset \R^2$ denote the unit circle in the Euclidean plane (with the subspace topology) and let $f : \mathbb{S}^1 \to \R$ be continuous. Prove there is some $x \in \mathbb{S}^1$ such that $f(x) = f(-x)$.
\item [(d)] Suppose $f : [0, 1] \to \R$ is continuous and has $f(0) = f(1)$. For each integer $n \geq 2$ show there is some $x$A s.t. $f(x) = f(x + \tfrac 1n)$.
\een
\end{problem}

\begin{solution}[\bf Solution.]
\ben
\item [(a)] Suppose not, for contradiction.
\be
A:= \bra{x:\phi(x)<x},\quad B:=\bra{x:\phi(x)>x},\quad [0,1] = A\cup B.
\ee

By construction all the points in these sets are interior points so we have an open disjoint union of non empty sets which cover the original space. Therefore there is a continuous map from these two sets onto $\bra{0,1}$. This contradicts the fact that $[0,1]$ is connected. Therefore there is fixed point.

\item [(b)] An add degree polynomial is a continuous function from the real line onto the real line. By the result of $(a)$ we can set $x$ to be zero and then by the connectedness of the real line it follows that the polynomial has a real root.
\item [(c)] Suppose not, for contradiction.
\be
A := \bra{x:f(x) < f(-x)},\quad B:=\bra{x:f(x) > f(-x)},\quad S^1 = A\cup B.
\ee

By construction all the points in these sets are interior points so we have an open disjoint union of non-empty sets which cover the original space. Therefore there is a continuous map from these two sets onto $\bra{0,1}$. This contradicts the fact that $S^1$ is connected. Therefore there is a fixed point.

\item [(d)] Suppose not, for contradiction.
\be
A:= \bra{x:f(x) < f\bb{x+\frac 1n}},\quad B:= \bra{x:f(x) > f\bb{x+\frac 1n}},\quad S^1 = A\cup B.
\ee

By construction all the points in these sets are interior points so we have an open disjoint union of non-empty sets which cover the original space. Therefore there is a continuous map from these two sets onto $\bra{0,1}$. This contradicts the fact that $S^1$ is connected. Therefore there is a fixed point.
\een
\end{solution}


\begin{problem}
Prove there is no continuous function $f : [0, 1] \to \R$ such that $x \in \Q \Longleftrightarrow f(x) \notin \Q$ (where $\Q$ denotes the rational numbers).
\end{problem}

\begin{solution}[\bf Solution.]

\end{solution}


\begin{problem}
\ben
\item [(a)] Suppose $A \subset \R^n$ is not compact. Show there is a continuous function on $A$ which is not bounded.
\item [(b)] I am in an infinite forest and can't see out. A troop of renegade beavers gnaw down all but finitely many trees. Can I see out now ?
\een
\end{problem}

\begin{solution}[\bf Solution.]
\ben
\item [(i)] In $\R^n$ the compact sets are exactly the closed and bounded sets.

Case 1: Let $A$ be bounded but not closed.

Pick a point, $p$, which is a limit point of $A$ but is not in $A$ and define the function,
\be
f:x\mapsto \frac 1{d(x,p)}.
\ee

Case 2: Let $A$ be closed but not bounded.

Pick a point, $p$, which is a point of $A$ but not in $A$ and define the function,
\be
f: x \mapsto d(x,p).
\ee

\item [(b)] If the trees have finite diameters then they could still block the view.

\een
\end{solution}


\begin{problem}
\ben
\item [(i)] Give an example of a sequence of closed connected subsets $C_n ⊂ \R^2$ s.t. $C_n \supset C_{n+1}$ but $\cap^\infty_{n=1} C_n$ not connected.
\item [(ii)] If $C_n \subset X$ is compact and connected in a Hausdorff space, and $C_n \supset C_{n+1}$ for each $n$, show $\cap^\infty_{n=1} C_n$ is connected.
\een
\end{problem}

\begin{solution}[\bf Solution.]
\ben
\item [(i)] $C_1\supset C_2 \supset \dots$ as required and $\bigcap_n C_n$ is not connected.
\be
C_n := \R^2 - \bsb{\bra{(x,y):x,y >0} \cup \bra{(x,y):x>-n,-1< y<1}}.
\ee

The union of two open sets is open so the complement is closed set. Each $C_n$ is connected and their intersection, in the limit, is disconnected.

\item [(ii)] Suppose for contradiction that the intersection is not connected then we have $C_1\supset C_2 \supset \dots$ all connected and compact but $\bigcap_n C_n = U\cup V$ with $U\cap V = \emptyset$ with $U$ and $V$ non-empty and open. Choose a fixed $C_k$ and note that $U \subseteq C_k$ and $V \subseteq C_k$.

Pick a finite cover of $C_k$, say $\bra{W_i}$, now note that $\bra{U\cap W_i}$ and $\bra{V\cap W_i}$ are open and each set is finite hence $\bigcup_i \bra{U\cap W_i}$ and $\bigcup_i \bra{V\cap W_i}$ are open disjoint and non-empty subsets of $C_k$. Since their union is $C_k$ the set $C_k$ is not connected. Contradiction. Therefore $\bigcap_n C_n$ is connected.
\een
\end{solution}


\begin{problem}
Let $X$ be a topological space. The \emph{one-point compactification} $X^+$ of $X$ is set-wise the union of $X$ and an additional point $\infty$ (thought of as "at infinity") with the topology: $U \subset X^+$ is open if either
\ben
\item [(i)] $U \subset X$ is open in $X$ or
\item [(ii)] $U = V \cup \{\infty\}$ where $V \subset X$ and $X\bs V$ is both compact and closed in $X$.
\een
Prove that $X^+$ is a topological space and prove that it is compact (N.B. regardless of whether $X$ is compact or not!).
\end{problem}

\begin{solution}[\bf Solution.]
\ben
\item [(i)] Claim. This is a topological space.

The whole space and the empty set are open. Let $U = \bb{\bigcup_i U_i} \cup \bb{\bigcup_i (V_i\cup \bra{\infty})}$ be an arbitrary union of open sets. $U - \bra{\infty} \subseteq X$ so it has the form $U_i$ for some $U_i$ open in $X$. If $\bra{\infty} \in U$ then $U$ has the form $U_i \cup \bra{\infty}$ otherwise it is just $U_i$. Only remains to check that $U_i \cup \bra{\infty}$ restricted to $X$ always has a compact and closed complment in $X$.

Lemma 1. A closed subset of a compact set is compact.

Let $C$ be a closed subset of a compact space $K$. Let $\bra{U_i}$ be a cover of $C$. Then since $C$ is closed it complement in $K$ is open, therefore $\bra{U_i} \cup \bra{K-C}$ is a cover for $K$. Since $K$ is compact, this has a finite subcover. Therefore $\bra{U_i}$ has a finite subcover, so $C$ is compact.

Lemma 2. If $V$ is a set with a compact complement in $X$ and $U$ is any open set in $X$ then $V\cup U$ has a compact complement in $X$.

$X -(U\cup V)$ is a closed set since $U\cup V$ is open in $X$. $X-(U\cup V)$ is a subset of $X-V$. So $X-(U\cup V)$ is a closed subset of a compact set and by Lemma 1 is itself compact.

Finite intersections are then clearly of the required form. So this defines a topology on the new space.

\item [(ii)] Let $\bra{U_i}$ be a open cover for $X^+$. Note that one of the $U_i$ must contain `infinity' call this set $V$. Then $V$ is an open set in $X$ with compact complement in $X$. Note that this means that the complement of $V$ is also compact in $X^+$. Now $\bra{U_i \cup V}$ are a cover for $X^+$. Pick any $U_j \cup V$ and note that by Lemma 2 this has a compact complement in $X^+$. Therefore $\bra{U_i \cup V}_{i\neq j}$ is a cover of the complement $X^+ - (U_j \cup V)$, and therefore has a finite subcover. Call this subcover $\bra{\tilde{U}_i \cup \tilde{V}}_{i\neq j}$. Then $\bra{\tilde{U}_i \cup \tilde{V}}_{i\neq j} \cup \bra{V}$ is a finite subcover for $X^+$, so $X^+$ is compact.
\een
\end{solution}


\begin{problem}
A family of sets has the \emph{finite intersection property} if and only if every finite subfamily has non-empty intersection. Prove that a space $X$ is compact if and only if whenever $\{V_a\}_{a\in A}$ is a family of closed subsets of $X$ with the finite intersection property, the whole family has non-empty intersection.
\end{problem}

\begin{solution}[\bf Solution.]
Claim. If every family with FIP has non-empty intersection then $X$ is compact.

Suppose every family $\bra{V_i}$ of closed sets with FIP has non-empty intersection. Suppose for contradiction $X$ is not compact. Then there exists a cover $\bra{U_i}$ with no finite subcover. Define the family of closed sets $\bra{V_j = X - \bigcup_{i\neq j}U_i}$.

We claim that the covering set has FIP. Suppose not then $\bigcap_{j} V_j = \emptyset$ for some finite subset. In that case
\be
X = X - \bigcap_j V_j = \bigcup_j (X-V_j).
\ee

So the cover has a finite has a finite subcover. Contradiction. Therefore the set has finite intersection property.

It follows that the whole set has non-empty intersection. In that case the elements in the intersection are in none of the complements. So $\bra{U_i}$ is not a cover. Contradiction. Therefore no such cover exists and the space is compact.

Claim. If $X$ is compact then every family with FIP has non-empty intersection.

Let $\bra{V_i}$ be a family of closed sets with FIP. Let $X$ be a compact space.

Suppose for contradiction that $\bigcap_i V_i = \emptyset$. Then the complements $U_i = X-V_i$ are an open cover of $X$. Then since $X$ is compact we have a finite subcover of $X$, say $\bra{\tilde{U}_i}$ and the corresponding subset of $V_i$ given by $\tilde{U}_i = X - \tilde{V}_i$. $\bigcap_i V_i = \bigcap_i \tilde{V}_i \neq \emptyset$ by the FIP. Contradiction with $\bigcap_i V_i = \emptyset$. Therefore every family with FIP has non-empty intersection.
\end{solution}


\begin{problem}
Let $(X, d)$ be a compact metric space. Prove that a subspace $Z \subset X$ is compact only if every sequence in $Z$ has a subsequence which converges in the metric to a point of $Z$. [Note: the question requires "only if" and not "if".]

Let $X$ be the space of continuous functions from $[0, 1]$ to the reals $\R$ with the metric $d(f, g) = \sup\{|f(x) - g(x)| : x \in [0, 1]\}$. Prove that the unit ball $\{u \in X | d(0, u) \leq 1\}$ is not compact, where 0 denotes the obvious zero-function. [Thus the "Heine-Borel" theorem is not valid in arbitrary metric spaces.]
\end{problem}

\begin{solution}[\bf Solution.]
Let $\bra{x_i}$ be a Cauchy sequence in $Z$ and let $\bra{y_i}$ be a convergent subsequence to a limit $y$. Then
\be
d(x_i,y) \leq d(x_i,y_j) + d(y_j ,y) < \frac{\ve}2 + \frac{\ve}2
\ee
so the whole sequence is also convergent to the same limit.

We now claim that every sequence in $Z$ has a Cauchy subsequence. Let $\bra{x_i}$ be any sequence in $Z$. Take a cover of $Z$ by balls of radius $\ve$. Complete this cover to a cover of the whole space $X$. Since $X$ is compact this cover has a finite subcover. This subcover contains a finite cover of $Z$. Therefore the sequence has an infinite number of points in at least one of the covering sets. Pick a point of the sequence in this set, say $B(w_1,\ve)$ and call it $y_1$. Delete all the points in $\bra{x_i}$ up to and including $y_1$. Consider the remainder of the sequence. Take a cover of $Z$ by sets of radius $\frac{\ve}2$. As before find a finite subcover of $Z$. Again at least one of the sets has an infinite number of points in the sequence. Consider the intersection of one such set, say $B\bb{w_2,\frac {\ve}2}$ with $B(w_1,\ve)$ to form a new set $U\bb{\frac {\ve}2}$. Note that $U\bb{\frac {\ve}2} \subseteq B(w_1,\ve)$ and this smaller set contains an infinite number of points of the sequnce. Pick $y_2$ one of the sequence points in this set. Continue inductively. This defines a nested sequence of open sets each of which contains a point in the sequence. By construction the diameters of the sets are tending to zero so the subsequence is Cauchy.

It follows that every sequence in $Z$ has a convergent subsequence in $Z$ and therefore $Z$ is closed. A closed set of a compact space is compact so $Z$ is compact.
\end{solution}


\begin{problem}
Let $M$ be a compact metric space and suppose that for every $n \in \Z_{\geq 0}$, $V_n \subset M$ is a closed subset and $V_{n+1} \subset V_n$. Prove that
\be
\diam\bb{\bigcap^\infty_{n=1} V_n} = \inf\left\{\diam(V_n) | n \in \Z_{\geq 0}\right\}.
\ee
[Hint: suppose the LHS is smaller by some amount $\ve$.]
\end{problem}

\begin{solution}[\bf Solution.]
Since the sets are nested we have $\bigcap^N_{n=1}V_n \subseteq V_k$, $k\leq N$. Hence
\be
\diam \bigcap^\infty_{n=1}V_n \leq \diam (V_k),\quad \forall k \ \ra \ \diam \bigcap^\infty_{n=1}V_n \leq \inf_n \bra{\diam(V_n)}.
\ee

Now suppose $\diam \bigcap^\infty_{n=1}V_n + \ve = \inf_n \bra{\diam(V_n)}$, $\ve>0$. Then for each $V_k$ we have $\inf_n \bra{\diam(V_n)} \leq \diam(V_k)$ so we can find pairs of points in each set such that
\be
d(x_k,y_k) > \diam \bigcap^\infty_{n=1}V_n + \ve \ \ra \ \inf\bra{d(x_k,y_k)} \geq \diam \bigcap^\infty_{n=1}V_n + \ve.
\ee

We take the infimum on the LHS above over all pairs and all sets. Note that $\inf\bra{d(x_k,y_k)} > \diam \bigcap^\infty_{n=1}V_n$ although the individual points form sequences which tend to limit points in the intersection so
\be
\inf\bra{d(x_k,y_k)} \to \diam \bigcap^\infty_{n=1}V_n.
\ee

Therefore we have $\diam \bigcap^\infty_{n=1} V_n = \inf_n \bra{\diam(V_n)}$.
\end{solution}


\begin{problem}
Fix a prime $p$ and let $a \in \Q$ be non-zero. One can uniquely write $a = p^n \frac xy$ with $x$ and $y$ coprime, $n \in \Z$ and $xy$ not divisible by $p$. Define
\be
v_p(a) = n,\quad v_p(0) = \infty, \quad \text{ and }\ |a|_p = p^{-v_p(a)},\quad  |0|_p = 0.
\ee
\ben
\item [(a)] Prove $v(a - b) \geq \min\{v(a), v(b)\}$ for any $a, b \in \Q$.
\item [(b)] Defining $d_p(a, b) = |a - b|_p$, prove that $d_p$ is a metric on $\Q$, this is called the \emph{$p$-adic} metric, of much importance in number theory.
\item [(c)] Show that if $p$ and $q$ are distinct primes, $d_p$ and $d_q$ are inequivalent metrics.
\item [(d)] Suppose $p \neq 2$. Choose $a \in \Z$ which is not a square in $\Q$ and which is not divisible by $p$. Suppose $x^2 \equiv a (\bmod p)$ has a solution. Show there is $x_1$ s.t. $x_1 \equiv x_0 (\bmod p)$ and $x^2_1 \equiv a \bmod p^2$, and iteratively that there is $x_n$ s.t. $x_n \equiv x_{n-1} \bmod p$ and $x^2_n \equiv a \bmod p^{n+1}$. Show that $(x_n)$ is a Cauchy sequence in $(\Q, d_p)$ with no convergent subsequence. Deduce $(\Q, d_p)$ is not complete.
\een

[Any incomplete metric space admits a canonical \emph{completion}. The completion of $(\Q, d_{\text{eucl}})$ is $\R$, a general element of which can be written as $a = \sum^\infty_{k=m} a_k 10^{-k}$ with $0 \leq a_k \leq 9$. The completion $\Q_p$ of $(\Q, d_p)$ comprises the expressions $\sum^\infty_{i=m} a_ip^i$ with $0 \leq a_i \leq p - 1$.]
\end{problem}

\begin{solution}[\bf Solution.]
\ben
\item [(a)] We write $a= p^r x$, $b = p^{r+s}y$ wlog and in the notation of the question.
\be
v(a) = p^r,\quad v(b) = p^{r+s} \ \ra \ b-a = p^r(p^s y -x) \ \ra \ v(b-a) = p^r \geq \min \bra{v(a),v(b)}
\ee
as required.

\item [(b)] $x-y=p^rk: k\nmid p$ for some $r$ and $p$.
\be
d(x,y) := \left\{\ba{ll}
\frac 1{p^r} \quad \quad & x \neq y\\
0 & x=y
\ea\right.
\ee

Positivity: $d(x,y) >0$ for $x\neq y$ as required.

Symmetry: $d(x,y) = \frac 1{p^r} = d(y,x)$.

Triangle: Wlog let
\be
x-y = p^r k: k\nmid p,\quad y-z = p^{r+s}m:m\nmid p \ \ra \ x-y + y-z = p^r k + p^{r+s} m = p^r (k+ p^s m)
\ee
and so
\be
d(x,z) = \left\{\ba{ll}
\frac 1{p^r} \quad\quad & s>0\\
\frac 1{p^{r+1}} & (k+m)\mid p,\ s= 0
\ea\right. \ \ra \ d(x,z) \leq \frac 1{p^r}.
\ee

Thus, $d(y,z) = \frac 1{p^{r+s}}$
\be
d(x,y) + d(y,z) = \frac 1{p^r} + \frac 1{p^{r+s}} = \frac 1{p^r} \bb{1 + \frac 1{p^s}} \geq d(x,z).
\ee
\item [(c)]
\item [(d)]
\een
\end{solution}


\begin{problem}
\ben
\item [(a)] Draw an example of a smooth connected surface in $\R^3$ with infinitely many "ends" (i.e. for which complements of arbitrarily large compact sets have infinitely many connected components). Hence, or otherwise, draw three pairwise non-homeomorphic connected infinite genus smooth surfaces in $\R^3$. (The \emph{genus} is the number of holes: muffins have genus 0, bagels have genus 1, pretzels have genus 3.)
\item [(b)] Sketch an informal argument to explain why the Euclidean spaces $\R^2$ and $\R^3$ are pairwise not homeomorphic.
\item [(c)] Let $X$ be a topological space and $x_0 \in X$ a distinguished point. Show that the set of connected components of the based loop space $\Omega X = \{\gamma : [0, 1] \to X | \gamma \text{ is continuous}, \gamma(0) = \gamma(1) = x_0\}$ forms a group.
\item [(d)] Give examples in which this group is non-trivial. Can it be non-trivial and finite? Can it be non-abelian?
\een
\end{problem}

\begin{solution}[\bf Solution.]
\end{solution}
