
\chapter{Groups, Rings and Modules}

\section{Groups}

\begin{problem}
Let $G$ be a group of even order with a cyclic Sylow 2-subgroup. By considering the regular action of $G$, show that $G$ has a normal subgroup of index 2. %[If $x$ is a generator of a Sylow 2-subgroup, show that $x$ is an odd permutation by working out its cycle structure.]
\end{problem}

\begin{solution}[\bf Solution.]
The action is regular (or simply transitive) if it is both transitive and free; this is equivalent to saying that for any two $x, y \in X$ there exists precisely one $g \in G$ such that $gx = y$. In this case, $X$ is known as a principal homogeneous space for $G$ or as a $G$-torsor.

{\bf [The following proof is from Zexiang Chen, need checking]}

Let $X$ be a generator of the Sylow 2-subgroup $H$. $H\leq G$. Let $G$ act on $H$, by $G\times G\to G$, $(g,h) \mapsto gh$ which induces a homomorphism $\phi:G \to \sym G, \phi(g) = \phi_g$.

$\abs{G} = 2^a b$, where $b$ is odd. $o(x) = 2^a$, $\phi(x) = \phi_x$, which has order $2^a$, because $o(\phi(x)) = k\mid 2^a$, but kernel of $\phi$ is trivial so $k = 2^a$. So $\phi_x$ must be a product of $m$ lots of $2^a$ cycles and so $\ve(\phi_x) = (2^a-1)m = \text{odd}^m = \text{odd}$. So $\phi_x$ is an odd permutation.

Then since $\phi(G) \leq \sym(G)$, and we have an odd permutation, so $\phi(G)$ contains hthe same number of odd and even permutations. (by the map $\ve:\phi(G)\to \bra{\pm 1}$), pick thoe even permutations, which is a normal subgroups of $\phi(G)$, and claim that the preimage of that is a normal subgroup of $G$ of index 2.

\end{solution}

{\large \textcolor{red}{from Part IA Groups}}


%%%%%%%%%%%%%%%%%%%%%%%%%%%%%%%%%%%%%%%%%%%%%%%%%%%%%%%%%%%%%%%%%%%%%%%%%%%%%%%%%%%%%%%%%%%%%%%%%%%%%%%%%%%%%%%%%%%%%%%%%%%%%%%%%%%%%%%%%%%%%%%%%%%%%%%%%%%%%%%%%%%%%%%%%%%%%%%%%%%%%%%%%%%%%%%%%%%%%%%%%%%%%%%%%%%%%%%%%%%%%%%%%%%%%%%%%%%%%%%%%%%%%%%%%%%%%%%%%%%%%%


\begin{problem}
Consider the M\"obius maps $f(z) = e^{2\pi i/n} z$ and $g(z) = 1/z$. Show that the subgroup $G$ of the M\"obius group $\sM$ generated by $f$ and $g$ is a dihedral group. ["Generated" here means that every element in $G$ is the product of elements of the form $f$, $g$, $f^{-1}$ and $g^{-1}$.]
\end{problem}

\begin{solution}[\bf Solution.]
Certainly $g^2 = e$. If $n=1$ then $f=e$, so $G = \bsa{f,g} = \bsa{e,g}\cong C_2$; if $n=2$, then $f(z) =-z$, so $f^2 = e$ and $fg= gf$, and thus $G = \bsa{f,g} = \bra{e,f,g,fg} \cong D_{4}$; so assume $n\geq 3$. Both $f$ and $g$ preserve the unit circle, and on it the set of $n$th of unity, which lie at the vertices of a regular $n$-gon; we may look at the effect on these vertices, since no non-identity element of $\sM$ fixes all of them - doing so we see that $f$ rotates them anticlockwise by $\frac{2\pi}n$ and $g$ reflects them in the real axis. Since the corresponding symmetries of the regular $n$-gon generate $D_{2n}$, we have $G\cong D_{2n}$.
\end{solution}

%%%%%%%%%%%%%%%%%%%%%%%%%%%%%%%%%%%%%%%%%%%%%%%%%%%%%%%%%%%%%%%%%%%%%%%%%%%%%%%%%%

\begin{problem}
Express the M\"obius tranformation $f(z) = (2z + 3)/(z - 4)$ as the composition of maps of the form $z \mapsto az$, $z \mapsto z +b$ and $z \mapsto  1/z$. Hence show that $f$ maps the circle $\abs{z -2i} = 2$ onto the circle $\abs{8z + (6 + 11i)} = 11$.
\end{problem}

\begin{solution}[\bf Solution.]
As $\frac{2z+3}{z-4} = 2 + \frac {11}{z-4}$, $f$ is the composition of $z \mapsto z-4$, $z \mapsto 1/z$, $z \mapsto 11z$ and $z \mapsto 2+z$.

The circle $\abs{z-2i} =2$ has centre $2i$ and radius 2; applying $z-4$ moves to centre $2i-4$ and preserves the radius. Inversing sends the circle with centre $c$ and radius $r$ to the circle with centre $\frac{\ol{c}}{c\ol{c}-r^2}$ and radius $\frac{r}{\abs{c\ol{c}-r^2}}$ 
\beast
\abs{z-c} = r & \lra & (z-c)(\ol{z}-\ol{c}) = r^2 \ \lra\ z\ol{z} - c\ol{z} - \ol{c}z + c\ol{c} - r^2 = 0 \\
& \lra & \frac 1{c\ol{c} - r^2} - \frac{c}{z(c\ol{c} - r^2)} - \frac {\ol{c}}{\ol{z}(c\ol{c} - r^2 )} + \frac 1{z\ol{z}} = 0\\
& \lra & \bb{\frac 1z - \frac{\ol{c}}{c\ol{c} - r^2 }}\bb{\frac 1{\ol{z}} - \frac{c}{c\ol{c} - r^2 }} = \frac{(c\ol{c})-(c\ol{c} - r^2)}{(c\ol{c} - r^2 )^2} = \frac{r^2}{(c\ol{c} - r^2 )^2}.
\eeast

So the new circle has centre $\frac{-4-2i}{20-4} = -\frac{2+i}8$ and radius $\frac{2}{20-4} = \frac 18$. Applying $z \mapsto 11z$ and then $z\mapsto z+2$ moves the centre to $11\bb{-\frac{2+i}8} +2 = -\frac {6+11i}8$ and changes the radius to $\frac {11}8$, so the final circle is $\abs{8z + (6 + 11i)} = 11$.
\end{solution}

%%%%%%%%%%%%%%%%%%%%%%%%%%%%%%%%%%%%%%%%%%%%%%%%%%%%%%%%%%%%%%%%%%%%%%%%%%%%%%%%%%

\begin{problem}
Let $G$ be the subgroup of M\"obius transformations which map the set $\{0, 1,\infty\}$ onto itself. Find all the elements in $G$. To which standard group is $G$ isomorphic? Find the group of M\"obius transformations which map the set $\{0, 2,\infty\}$ onto itself.
\end{problem}

\begin{solution}[\bf Solution.]
For each ordering of $0,1,\infty$, there exists a unique element of $\sM$ taking 0 to the first, 1 to the second and $\infty$ to the third: $z\mapsto z$ sends 0 to 0, 1 to 1, $\infty$ to $\infty$; $z\mapsto 1-z$ sends 0 to 1, 1 to 0, $\infty$ to $\infty$; $z \mapsto 1/z$ sends 0 to $\infty$, 1 to 1, $\infty$ to 0; $z \mapsto \frac{z-1}z$ sends 0 to $\infty$, 1 to 0, $\infty$ to 1; $z\mapsto \frac 1{1-z}$ sends 0 to 1, 1 to $\infty$, $\infty$ to 0;  $z \mapsto z/{z-1}$ sends 0 to 0, 1 to $\infty$, $\infty$ to 1. 

As it consists of permutation of 3 points, $G\cong S_3$. 

Similarly the M\"obius transformations sending $\bra{0,2,\infty}$ to itself are $z\mapsto z$, $z\mapsto 2-z$, $z\mapsto 4/z$, $z\mapsto \frac{2z-4}z$, $z\mapsto \frac 4{2-z}$, $z\mapsto \frac{2z}{z-2}$.
\end{solution}

%%%%%%%%%%%%%%%%%%%%%%%%%%%%%%%%%%%%%%%%%%%%%%%%%%%%%%%%%%%%%%%%%%%%%%%%%%%%%%%%%%%%%%%%%%%%%%%%%%%%%%%%%%%%%%%%%%%%%%%%%%%%%%%%%%%%%%%%%%%%%%%%%%%%%%%%%%%%%%%%%%%%

\begin{problem}
Let $G$ be a group of even order. Show that $G$ contains an element of order two.
\end{problem}

\begin{solution}[\bf Solution.]
Pair each non-self-inverse element of $G$ with its inverse (note that $(x^{-1})^{-1}=x$ so this makes sense). This shows that there are evenly many non-self-inverse elements. As $G$ is even, there are also evenly many self-inverse elements. One such is $e$, so there must be another, say $g$, and then $g = g^{-1} \ \ra \ g^2 =e$ so that $g$ has order two.
\end{solution}

%%%%%%%%%%%%%%%%%%%%%%%%%%%%%%%%%%%%%%%%%%%%%%%%%%%%%%%%%%%%%%%%%%%%%%%%%%%%%%%%%%

\begin{problem}
A \emph{fifteen puzzle} consists of fifteen small square tiles, numbered 1 to 15, which are mounted in a $4 \times 4$
frame in such a way that each tile can slide vertically or horizontally into an adjacent square (if it is not already occupied by another tile), but the tiles cannot be lifted out of the tray. On the packet in which the puzzle was sold, it is asserted that it is impossible to manoeuvre the tiles from the first to the second of the configurations shown below. The packet is too small to contain a proof. Is the assertion true? Prove it.
\be
\ba{ccccccccc}
1 & 2 & 3 & 4 & \quad\quad & 15 & 14 & 13 & 12\\
5 & 6 & 7 & 8 & & 11 & 10 & 9 & 8\\
9 & 10 & 11 & 12 & & 7 & 6 & 5 & 4\\
13 & 14 & 15 & - & & 3 & 2 & 1 & -
\ea
\ee
\end{problem}

\begin{solution}[\bf Solution.]

\end{solution}

%%%%%%%%%%%%%%%%%%%%%%%%%%%%%%%%%%%%%%%%%%%%%%%%%%%%%%%%%%%%%%%%%%%%%%%%%%%%%%%%%%





\begin{problem}
Describe the finite subgroups of the group of isometries of the plane. [If we think of the plane as $\C$ you may assume that all isometries have the form $z \mapsto  az + b$ or $z \mapsto  a\bar{z} + b$, where $a$ and $b$ are complex numbers and in both cases $\abs{a} = 1$. Hint: is there a point in the plane fixed by all the elements in the group? ]
\end{problem}

\begin{solution}[\bf Solution.]
First consider isometries $z\mapsto az + b$, if $a=1$ and $b\neq 0$ this is a translation, hence of infinite order, so cannot lie in a finite subgroup $G$. 

If $a\neq 1$ it is a rotation with centre $\frac b{1-a}$ (as this is the unique fixed point); if $G$ contains rotations $f$,$f'$ with different centres, say $f(z) = az + b$, $f'(z) = a'z + b$, then $f'f(z) = a'a z + a'b + b'$ while $ff'(z) = a'a z + ab' + b$, and $\frac b{1-a} \neq \frac {b'}{1-a'}$ implies that $a'b + b' \neq ab' + b$, so $(f'f)^{-1}ff'$ is a translation. Contradiction to finite subgroup again. 

Thus $G$ can only contain rotations about one centre; as in question \ref{que:multiplication_finite}, as $G$ is finite, the rotations form a cyclic group, say $\bsa{f}$. Now if $G$ contains a non-rotation $g$, where $g(z) = a\ol{z} + b$, then if $g'\in G$ is any other non-rotation then $g^{-1}g'$ is a rotation, say $f^i$ for some $i$. (Otherwise, it's a translation.)

So there are equally many rotations and non-rotations, and $G = \bsa{f,g} = \bra{e,f,f^2,\dots,f^{n-1}, g, gf,gf^2 ,\dots, gf^{n-1}}$. As 
\be
g^2(z) = a(\ol{a}z+ \ol{b}) + b = z + a \ol{b} + b,
\ee
we must have $a\ol{b} + b = 0$, and $g^2 = e$. (Otherwise $g^2$ is a translation). Similarly, $(gf)^2 = e$, i.e. $gf gf = e$, so $gfg^{-1} = f^{-1}$. So $G$ is dihedral.
\end{solution}

%%%%%%%%%%%%%%%%%%%%%%%%%%%%%%%%%%%%%%%%%%%%%%%%%%%%%%%%%%%%%%%%%%%%%%%%%%%%%%%%%%%%%%%%%%%%%%%%%%%%%%%%%%%%%%%%%%%%%%%%%%%%%%%%%%%%%%%%%%%%%%%%%%%%%%%%%%%%%%%%%%%%

\begin{problem}
Show that the set $\{1, 3, 5, 7\}$ with multiplication modulo 8 is a group. Is this group isomorphic to $C_4$ or $C_2 \times C_2$? Justify your answer.
\end{problem}

\begin{solution}[\bf Solution.]
We have
\begin{center}
\begin{tabular}{c|cccc}
$X_8$ & 1 & 3 & 5 & 7\\
\hline
1 & 1 & 3 & 5 & 7\\
3 & 3 & 1 & 7 & 5\\
5 & 5 & 7 & 1 & 3\\
7 & 7 & 5 & 3 & 1
\end{tabular}
\end{center}

So $X_8$ is a binary operation on $\bra{1,3,5,7}$ (i.e., we have closure); multiplication is associative so $X_8$ is as well; 1 is the identity; and each element is its own inverse. Thus we have a group. 

Since no element has order 4, it is not isomorphic to $C_4$ - but it is isomorphic to $C_2 \times C_2$ as each element is its own inverse.
\end{solution}

%%%%%%%%%%%%%%%%%%%%%%%%%%%%%%%%%%%%%%%%%%%%%%%%%%%%%%%%%%%%%%%%%%%%%%%%%%%%%%%%%%

%\begin{problem}
%Let $H$ be a subgroup of the group $G$. Find a (natural) bijection between the set of all left cosets and the set of all right cosets of $H$ in $G$.
%\end{problem}

%\begin{solution}[\bf Solution.]
%\end{solution}

%%%%%%%%%%%%%%%%%%%%%%%%%%%%%%%%%%%%%%%%%%%%%%%%%%%%%%%%%%%%%%%%%%%%%%%%%%%%%%%%%%

\begin{problem}
Let $H$ be a subgroup of the (finite) group $G$, let $K$ be a subgroup of $H$. Show that the index $|G : K|$
equals the product $|G : H||H : K|$.
\end{problem}

\begin{solution}[\bf Solution.]

\end{solution}

%%%%%%%%%%%%%%%%%%%%%%%%%%%%%%%%%%%%%%%%%%%%%%%%%%%%%%%%%%%%%%%%%%%%%%%%%%%%%%%%%%

\begin{problem}
Let $G$ be a subgroup of the symmetric group $S_n$. Show that if $G$ contains any odd permutations then precisely half of the elements of $G$ are odd.
\end{problem}

\begin{solution}[\bf Solution.]

\end{solution}

%%%%%%%%%%%%%%%%%%%%%%%%%%%%%%%%%%%%%%%%%%%%%%%%%%%%%%%%%%%%%%%%%%%%%%%%%%%%%%%%%%

\begin{problem}
Show that any subgroup of a cyclic group is cyclic. Find all the subgroups of the cyclic group $C_n$.
\end{problem}

\begin{solution}[\bf Solution.]
Let $G = \bsa{g}$ be a cyclic group, and $H$ be a subgroup of $G$. If $H = \bra{e}$, then $H=\bsa{e}$ is cyclic so we may assume $H \neq \bsa{e}$; take $h_1\in H\bs \bra{e}$, then $h_1 = g^n$ for some $n \in \Z$. Since $h_1^{-1} = g^{-n} \in H$, $\exists m\in \N$ with $g^m \in H$. 

Let $n\in \N$ be minimal with $g^n \in H$, and set $h = g^n$. Certainly $\bsa{h}\subseteq H$. Conversely, given $g^r\in H$ write $c = qn + r$ with $q,r\in \Z$ and $0\leq r < n-1$, then $g^r = g^{c -qn} = g^c \bb{g^n}^{-q} = g^c h^{-q} \in H$, so by the minimality of $n$ we must have $r=0$, whence $c = qn$ and $g^c = g^{qn} = h^q \in \bsa{h}$. Thus, $H\subseteq \bsa{h}$ which gives $\bsa{h} = H$, and thus $H$ is cyclic.
\end{solution}

%%%%%%%%%%%%%%%%%%%%%%%%%%%%%%%%%%%%%%%%%%%%%%%%%%%%%%%%%%%%%%%%%%%%%%%%%%%%%%%%%%

\begin{problem}
Show that the symmetric group $S_4$ has a subgroup of order $d$ for each divisor $d$ of 24, and find two
non-isomorphic subgroups of order 4.

Show that the alternating group $A_4$ has a subgroup of each order up to 4, but there is no subgroup of order 6.
\end{problem}

\begin{solution}[\bf Solution.]

\end{solution}

%%%%%%%%%%%%%%%%%%%%%%%%%%%%%%%%%%%%%%%%%%%%%%%%%%%%%%%%%%%%%%%%%%%%%%%%%%%%%%%%%%

\begin{problem}
List all the subgroups of the dihedral group $D_8$, and indicate which pairs of subgroups are isomorphic. Repeat for the quaternion group $Q_8$.
\end{problem}

\begin{solution}[\bf Solution.]

\end{solution}

%%%%%%%%%%%%%%%%%%%%%%%%%%%%%%%%%%%%%%%%%%%%%%%%%%%%%%%%%%%%%%%%%%%%%%%%%%%%%%%%%%%%%%%%%%%%%%%%%%%%%%%%%%%%%%%%%%%%%%%%%%%%%%%%%%%%%%%%%%%%%%%%%%%%%%%%%%%%%%%%%%%

\begin{problem}
Show that the dihedral group $D_{12}$ is isomorphic to the direct product $D_6 \times C_2$.
\end{problem}

\begin{solution}[\bf Solution.]

\end{solution}

%%%%%%%%%%%%%%%%%%%%%%%%%%%%%%%%%%%%%%%%%%%%%%%%%%%%%%%%%%%%%%%%%%%%%%%%%%%%%%%%%%

%
%\begin{problem}
%A finite group $G$ is generated by a set $T$ of elements of $G$ if each element of $G$ can be written as a finite product (possibly with repetitions) of powers of elements of $T$. Show that the symmetric group $S_n$ is generated by each of the following sets of permutations:
%\ben
%\item [(i)] the set $\{(j, k) : 1 \leq j < k \leq n\}$ of all transpositions in $S_n$;
%\item [(ii)] the set $\{(j, j + 1) : 1 \leq j < n\}$;
%\item [(iii)] the set $\{(1, k) : 1 < k \leq n\}$;
%\item [(iv)] the set $\{(1, 2), (12 \dots n)\}$ consisting of a transposition and an $n$-cycle.
%\een
%\end{problem}

%\begin{solution}[\bf Solution.]
%\end{solution}

%%%%%%%%%%%%%%%%%%%%%%%%%%%%%%%%%%%%%%%%%%%%%%%%%%%%%%%%%%%%%%%%%%%%%%%%%%%%%%%%%%

\begin{problem}
Consider a pack of $2n$ cards, numbered from 0 to $2n-1$. An outer perfect shuffle is a shuffle of the cards,
in which one first splits the pack in two halves of equal sizes and then interleaves the cards of the two halves in such a way that the top and bottom card remain in the top and bottom position. Show that the order of the outer shuffle is the multiplicative order of 2 modulo $2n - 1$. 

Deduce that after at most $2n - 2$ repetitions of the outer shuffle we get the cards in the pack into the original position.

What is the actual order of the outer shuffle of the usual pack of 52 cards?

(There is also an inner perfect shuffle which differs from the outer shuffle in that the interleaving of the cards of the two halves is done so that neither the top nor the bottom card remains in the same position. What is the order of this shuffle of the usual pack of 52 cards?)
\end{problem}


\begin{solution}[\bf Solution.]

\end{solution}

%%%%%%%%%%%%%%%%%%%%%%%%%%%%%%%%%%%%%%%%%%%%%%%%%%%%%%%%%%%%%%%%%%%%%%%%%%%%%%%%%%

\begin{problem}
How many subgroups of order four does the quaternion group $Q_8$ have?
\end{problem}

\begin{solution}[\bf Solution.]
In $Q_8$ the only element of order 2 is -1, so there is no subgroup isomorphic to $C_2\times C_2$; there are 6 element of order 4, forming 3 pairs $\bra{g,g^{-1}}$ as $\bsa{g} = \bsa{g^{-1}}$ there are 3 subgroups isomorphic to $C_4$. Thus, $Q_8$ has 3 subgroups of order 4: $\bra{1,i,-1,-i}$, $\bra{1,j,-1,-j}$ and $\bra{1,k,-1,-k}$.
\end{solution}

%%%%%%%%%%%%%%%%%%%%%%%%%%%%%%%%%%%%%%%%%%%%%%%%%%%%%%%%%%%%%%%%%%%%%%%%%%%%%%%%%%


%%%%%%%%%%%%%%%%%%%%%%%%%%%%%%%%%%%%%%%%%%%%%%%%%%%%%%%%%%%%%%%%%%%%%%%%%%%%%%%%%%

\begin{problem}
What is the order of the M\"obius map $f(z) = iz$? If $h$ is any M\"obius map, find the order of $hfh^{-1}$ and its fixed points. Use this to construct a M\"obius map of order four that fixes 1 and -1.
\end{problem}

\begin{solution}[\bf Solution.]
$f^4 z = i^4 z = z$ while $f^2 z = i^2 z = -z$, so $f$ has order 4. Given $h$, 
\be
(hfh^{-1})^4 = hfh^{-1} hfh^{-1} hfh^{-1} hfh^{-1} = hffffh^{-1} = hf^4h^{-1} = hh^{-1} = 1
\ee
while 
\be
(hfh^{-1})^2 = hfh^{-1} hfh^{-1} = hf^2h^{-1}  \neq 1.
\ee
since $f^2 \neq 1$. So $hfh^{-1}$ has order 4. Since the fixed points of $f$ are 0 and $\infty$, we have 
\be
hfh^{-1} h(0) = hf(0) = h(0),\quad hfh^{-1} h(\infty) = hf(\infty) = h(\infty).
\ee

So $h(0)$ and $h(\infty)$ are fixed points of $hfh^{-1} $; as no non-identity M\"obius map fixes 3 distinct points, these are the only fixed points of $hfh^{-1} $. 

If we take $h(z) = \frac{z-1}{z+1}$ then $h(0) = -1$, $h(\infty) = 1$ thus $hfh^{-1} $ is of order 4 and fixes 1 and -1. Since $h^{-1}(z) = \frac{z+1}{-z+1}$, we have 
\be
hfh^{-1} (z) = hf\bb{\frac{z+1}{-z+1}} = h\bb{\frac{iz + i}{-z+1}} = \bb{\frac{iz+i}{-z+1}-1}\left/\bb{\frac{iz+i}{-z+1} +1}\right. = \frac{(i+1)z + (i-1)}{(i-1)z + (i+1)}.
\ee
\end{solution}

%%%%%%%%%%%%%%%%%%%%%%%%%%%%%%%%%%%%%%%%%%%%%%%%%%%%%%%%%%%%%%%%%%%%%%%%%%%%%%%%%%

\begin{problem}
Let $X = \bra{1, 2, 3, 4, 5, 6}$, and let $G$ be the cyclic group generated by the permutation $\sigma(1) = 2$, $\sigma(2) = 1, \sigma(3) = 4, \sigma(4) = 5, \sigma(5) = 6, \sigma(6) = 3$. Since $G$ is a subgroup of $S_6$, it acts on $X$. Find all orbits and stabilizers for the action of $G$ on $X$ and check that your answers are consistent with the Orbit-stabilizer theorem.
\end{problem}

\begin{solution}[\bf Solution.]
Since $\sigma$ sends 1,2,3,4,5,6 to 2,1,4,5,6,3 in order, $\sigma^2$ sends 1,2,3,4,5,6 to 1,2,5,6,3,4 in order, $\sigma^3$ sends 1,2,3,4,5,6 to 2,1,6,3,4,5 in order, and $\sigma^4$ sends 1,2,3,4,5,6 to 1,2,3,4,5,6 in order. So $\sigma^4 = \iota$ and $G = \bra{1,\sigma,\sigma^2,\sigma^3}$. Write $O_i$ for the orbit of $i$, so that $O_i = \bra{i,\sigma(i),\sigma^2(i),\sigma^3(i)}$ by the above.

$O_1 = O_2 = \bra{1,2}$ while $O_3 = O_4 = O_5 = O_6 = \bra{3,4,5,6}$. Also by the above, $\stab(1) = \stab(2) = \bra{\iota,\sigma^2}$ while $\stab(3) = \stab(4) = \stab(5) = \stab(6) = \bra{\iota}$. Thus in all cases $\abs{O_i}\cdot \abs{\stab(i)} = 4 = \abs{G}$, as given by the orbit-stabilizer theorem.
\end{solution}

%%%%%%%%%%%%%%%%%%%%%%%%%%%%%%%%%%%%%%%%%%%%%%%%%%%%%%%%%%%%%%%%%%%%%%%%%%%%%%%%%%

\begin{problem}
Show that $\rho(t, (x, y)) = (e^tx, e^{-t}y)$, where $t \in \R$ and $(x, y) \in \R^2$, defines an action of $(R, +)$ on $\R^2$. Describe the orbits of the action and find all possible stabilizers.
\end{problem}

\begin{solution}[\bf Solution.]
Certainly $\rho(0,(x,y)) = (x,y)$; also 
\be
\rho(t,\rho(t',(x,y))) = \rho(t,(e^{t'}x, e^{-t'}y)) = \bb{e^te^{t'}x,e^{-t}e^{-t'}y} = \rho\bb{t+t',(x,y)}.
\ee

Thus this gives an action of $(\R,+)$ on $\R^2$. The orbits of the action are as follows: the origin; the positive $x$-axis; the negative $x$-axis; the positive $y$-axis; the negative $y$-axis; and curves $xy$=constant restricted to a quadrant of the plane. The first orbit has stabilizer $\R$; all other stabilizers are $\bra{0}$, since if $(x,y)\neq (0,0)$ then $\bb{e^tx,e^{-t}y} = (x,y) \ \ra \ t = (0)$.
\end{solution}

%%%%%%%%%%%%%%%%%%%%%%%%%%%%%%%%%%%%%%%%%%%%%%%%%%%%%%%%%%%%%%%%%%%%%%%%%%%%%%%%%%


%%%%%%%%%%%%%%%%%%%%%%%%%%%%%%%%%%%%%%%%%%%%%%%%%%%%%%%%%%%%%%%%%%%%%%%%%%%%%%%%%%

\begin{problem}
Let $G$ be a finite group and let $X$ be the set of all it subgroups. Given $g \in G$ and $H \in X$ show that $(g,H) \mapsto gHg^{-1}$ defines an action of $G$ on $X$. Show that the orbit of $H$ has at most $\abs{G}=\abs{H}$ elements. If $H \neq G$, show that there is an element in $G$ which does not belong to any conjugate of $H$.
\end{problem}

\begin{solution}[\bf Solution.]
Certainly, $gHg^{-1}$ is still a subgroup of $G$ (it contains $geg^{-1} = e$, and given $ghg^{-1}$, $ghg^{-1}\in gHg^{-1}$) we have 
\be
ghg^{-1} \cdot gh'g^{-1} = g(hh')g^{-1} \in gHg^{-1},\quad \bb{ghg^{-1}}^{-1} = gh^{-1}g^{-1} \in gHg^{-1}),
\ee
and $(e,H)\mapsto eHe^{-1} = H$; given $g,g'\in G$ and $H\in X$ we have 
\be
\rho(gg',H) = gg'H(gg')^{-1} = gg' H g'^{-1}g^{-1} = \rho(g,g'Hg'^{-1}) = \rho(g,\rho(g',H)).
\ee

Thus, this gives an action of $G$ on $X$.

If $h\in H$ we have $\rho(h,H) = hHh^{-1} = H$, so $\stab(H)$ contains $H$. Thus
\be
\abs{\text{orbit of }H} = \frac{\abs{G}}{\abs{\stab(H)}} = \leq \frac{\abs{G}}{\abs{H}}.
\ee

If $H\neq G$, Let $\frac{\abs{G}}{\abs{H}} = k >1$ then as each conjugate of $H$ contains $\abs{H}$ elements, one of which is $e$, the number of elements of $G$ contained in conjugates of $H$ is at most 
\be
1 + k(\abs{H}-1) = 1+ k\abs{H} -k = 1+ \abs{G} - k < \abs{G},
\ee
so there is an element of $G$ which does not lie in any conjugate of $H$.
\end{solution}

%%%%%%%%%%%%%%%%%%%%%%%%%%%%%%%%%%%%%%%%%%%%%%%%%%%%%%%%%%%%%%%%%%%%%%%%%%%%%%%%%%

%%%%%%%%%%%%%%%%%%%%%%%%%%%%%%%%%%%%%%%%%%%%%%%%%%%%%%%%%%%%%%%%%%%%%%%%%%%%%%%%%%

\begin{problem}
Let $G$ be the group of all symmetries of the cube and let $\ell$ be the line between two diagonally opposite vertices. Let $H = \bra{g \in G : g\ell = \ell }$. Show that $H$ is a subgroup of $G$ isomorphic to $S_3 \times C_2$.
\end{problem}

\begin{solution}[\bf Solution.]
There are 4 axes joining diagonally opposite vertices; since $G$ has a single orbit on vertices it also has a single orbit on such axes, and hence $H = \stab(1)$ has order $\abs{G}/4 = 48/4 = 12$.

Let the vertices at each end of $\ell$ be $v$ and $v'$, then $H$ contains $\stab(v)$, whose order is $\abs{G}/8 = 48/8 = 6$ since $\stab(v)$ acts on the 3 vertices $w,x,y$ nearest $v$, with a single orbit, and there is a reflection fixing $v$ and $w$ but interchanging $x$ and $y$, we have $\stab(v) \cong S_3$.

There is a symmetry $\sigma$ interchanging each vertex with that diagonally opposite it, and $\sigma$ commutes with every element of $G$; then $\sigma \in H$, and so $H=\bsa{\stab(v),\sigma} = \stab(v)\times \bsa{\sigma} \cong S_3 \times C_2$.
\end{solution}

%%%%%%%%%%%%%%%%%%%%%%%%%%%%%%%%%%%%%%%%%%%%%%%%%%%%%%%%%%%%%%%%%%%%%%%%%%%%%%%%%%



%%%%%%%%%%%%%%%%%%%%%%%%%%%%%%%%%%%%%%%%%%%%%%%%%%%%%%%%%%%%%%%%%%%%%%%%%%%%%%%%%%

\begin{problem}
Let $S^1$ denote the unit circle in $\C$ and let $S^3 = \bra{(w_1,w_2) \in \C^2 : \abs{w_1}^2 + \abs{w_2}^2 = 1}$. Show that given $(t_1, t_2) \in S^1 \times S^1$ and $(w_1,w_2) \in S^3$, 
\be
((t_1, t_2),(w_1,w_2)) \mapsto (t_1w_1, t_2w_2)
\ee
defines an action of $S^1 \times S^1$ on $S^3$. Describe the orbits and find all stabilizers.
\end{problem}

\begin{solution}[\bf Solution.]
Clearly, $\forall t_1,t_2\in S^1\times S^1$, $(w_1,w_2)\in S^2$ we have $(t_1w_1,t_2w_2)\in \C^2$, and 
\be
\abs{t_1w_1}^2 + \abs{t_2w_2}^2 = \abs{t_1}^2\abs{w_1}^2 + \abs{t_2}^2 \abs{w_1}^2 = \abs{w_1}^2 + \abs{w_2}^2 = 1
\ee
since $\abs{t_1} = \abs{t_2} =1$ - so $(t_1w_1,t_2w_2)\in S^3$. Certainly $\rho((1,1),(w_1,w_2)) = (w_1,w_2)$,
\be
\rho((t_1,t_2),\rho((s_1,s_2),(w_1,w_2))) = \rho((t_1,t_2),\rho(s_1w_1 ,s_2w_2)) = (t_1s_1w_1,t_2s_2w_2) = \rho((t_1s_1,t_2s_2),(w_1,w_2)).
\ee

So this gives an action of $S^1 \times S^1$ on $S^3$. 

For each $\alpha$ with $0\leq \alpha \leq 1$, let $0_\alpha = \bra{(w_1,w_2)\in S^3:\abs{w_1}^2 = \alpha} = \bra{(w_1,w_2)\in \C^2:\abs{w_1}^2 = \alpha, \abs{w_2}^2 = 1-\alpha}$ - then the $O_\alpha$ are the orbits of the action, because each set is preserved by the action and if $(w_1,w_2),(w_1',w_2')\in O_\alpha$ then $\abs{w_1} = \abs{w_1'}$ and $\abs{w_2} = \abs{w_2'}$, so with $t_1 = w_1'/w_1$, $t_2 = w_2'/w_2$ we have $(t_1,t_2)\in S^1 \times S^1$ and $\rho((t_1,t_2),(w_1,w_2)) = (w_1',w_2')$. If $0<\alpha <1$ the stabilizer of a point $(w_1,w_2)$ in $O_\alpha$ is trivial, since $w_1,w_2\neq 0$; if $\alpha = 0$ the stabilizer is $S^1\times \bra{1}$, while if $\alpha = 1$ it is $\bra{1}\times S^1$.
\end{solution}

%%%%%%%%%%%%%%%%%%%%%%%%%%%%%%%%%%%%%%%%%%%%%%%%%%%%%%%%%%%%%%%%%%%%%%%%%%%%%%%%%%

%\begin{problem}
%Let $p$ be a prime. Show that every group of order $p^2$ is abelian. [Hint: consider the action of the group on itself by conjugation.]
%\end{problem}

%\begin{solution}[\bf Solution.]
%Let $G$ be a group of order $p^2$. In the action of $G$ on itself by conjugation, one orbit is $\bra{e}$ as the possible stabilizer order are $1,p,p^2$ (by Lagrange's theorem), the possible orbit sizes are $p^2,p,1$. So all orbits have size 1 or $p$, and as the sume of the orbit sizes is $\abs{G}= p^2$, there must be at least $p$ orbits of size 1 (by considering the sum module $p$).

%Now $\bra{z}$ is an orbit of size 1 $\lra \ \forall g\in G, gzg^{-1} =z \ \lra\ \forall g\in G,gz =zg$. Write 
%\be
%Z(G) = \bra{z\in G:\forall g\in G,gz = zg},
%\ee
%so that $Z(G)\geq p$. If $\abs{Z(G)} < p^2$, take $g\in G\bs Z(G)$, then $\stab(g)$ contains $Z(G)$ since
%\be
%z\in Z(G)\ \ra \ zg = gz \ \ra \ zgz^{-1} = g
%\ee
%and also $g$ itself since $ggg^{-1} = g$, so $\stab(g) >p$ (as $g\notin Z(G)$ and $Z(G)\geq p$). As $\stab{g}$ is a subgroup of $G$ we must have $\stab(g) = G$, whence the orbit containing $g$ is of size 1, contrary to $g\in Z(G)$. Thus $\abs{Z(G)}=p^2$ and so $Z(G) =G$. That is, for all $g,h\in G$, we have $gh = hg$, i.e., $G$ is abelian.
%\end{solution}

%%%%%%%%%%%%%%%%%%%%%%%%%%%%%%%%%%%%%%%%%%%%%%%%%%%%%%%%%%%%%%%%%%%%%%%%%%%%%%%%%%
%%%%%%%%%%%%%%%%%%%%%%%%%%%%%%%%%%%%%%%%%%%%


\begin{problem}
Let $G$ be a group. If $H$ is a normal subgroup of $G$ and $K$ is a normal subgroup of $H$, is $K$ a normal subgroup of $G$?
\end{problem} 

\begin{solution}[\bf Solution.]
No. Normality is not transitive.
\end{solution}

\begin{problem}
Let $K$ be a normal subgroup of index $m$ in the group $G$. Show that $g^m \in K$ for any element $g\in G$.
\end{problem} 

\begin{solution}[\bf Solution.]
\end{solution}

\begin{problem}
Let $H$ be a subgroup of the cyclic group $C_n$. What is the quotient $C_n/H$? 

Let $D_{2n}$ be the group of symmetries of a regular $n$-gon. Show that any subgroup $K$ of rotations is normal in $D_{2n}$, and identify the quotient $D_{2n}/K$.
\end{problem} 

\begin{solution}[\bf Solution.]
Let $C_n = \bsa{g}$ and $H = \bsa{g^m}$ where $m\mid n$, then $C_n/H = \bra{H,gH,g^2H,\dots, g^{m-1}H}= \bsa{gH} \cong C_m$.

\end{solution}

\begin{problem}
Show that $D_{2n}$ has two conjugacy classes of reflections if $n$ is even, but only one if $n$ is odd.
\end{problem} 

\begin{solution}[\bf Solution.]
If $n$ is odd, any axis of symmetry joins a vectex to the midpoint of the opposite edge; if $r_v$ is the reflection in the axis through the vectex $v$, then given any other vertex $w$, there exists a rotation $g$ with $gv = w$, and then $gr_v g^{-1}$ is a reflection with 
\be
\bb{gr_v g^{-1}}w = gr_v g^{-1} gv = gr_v v = gv = w,
\ee
i.e., $gr_v g^{-1} = r_w$, so there is one conjugacy class of reflections. 

If however $n$ is even there are two types of axis of symmetry, joining opposite vertices and joint the midpoints of opposite edges; just as in the ease where $n$ is odd, any two reflections about axes of the same type are conjugate, but the first type of reflection fixes two vertices, and so any conjugate will fix two vertices, while the second type fixes no vertices; so there are two conjugacy classes of reflections.

\end{solution}

\begin{problem}
Let $D_8$ be the dihedral group of order 8. Find the conjugacy classes of $D_8$ and their sizes. Show that the centre $Z$ of the group has order 2, and identify the quotient group $D_8/Z$ of order 4. Repeat with the quaternion group $Q_8$.
\end{problem} 

\begin{solution}[\bf Solution.]
\end{solution}

\begin{problem}
Let $Q$ be a plane quadrilateral. Show that its group $G(Q)$ of symmetries has order at most 8. For which $n$ in the set $\{1, 2,\dots, 8\}$ is there a quadrilateral $Q$ with $G(Q)$ of order $n$?

\end{problem} 

\begin{solution}[\bf Solution.]

\end{solution}

\begin{problem}
What is the group of all rotational symmetries of a Toblerone chocolate bar, a solid triangular prism with an equilateral triangle as a cross-section, with ends orthogonal to the longitudinal axis of the prism? And the group of all symmetries?

\end{problem} 

\begin{solution}[\bf Solution.]

\end{solution}

\begin{problem}
Show that the subgroup $H$ of the group $G$ is normal in $G$ if and only if $H$ is the union of some conjugacy classes of $G$.

Show that the symmetric group $S_4$ has a normal subgroup (usually denoted $V_4$) of order 4.

To which group of order 6 is the quotient group $S_4/V_4$ isomorphic?

Find an action of $S_4$ giving rise to this isomorphism.

\end{problem} 

\begin{solution}[\bf Solution.]

\end{solution}



\begin{problem}
Let $G$ be a finite group and let $X$ be the set of all subgroups of $G$. Show that $G$ acts on $X$ by $g : H \mapsto g H g^{-1}$ for $g \in G$ and $H \in X$, where $g H g^{-1} = \{ g h g^{-1} : h \in H\}$. Show that the orbit containing $H$ in this action of $G$ has size at most $|G|/|H|$. If $H$ is a proper subgroup of $G$, show that there exists an element of $G$ which is contained in no conjugate $g H g^{-1}$ of $H$ in $G$.

\end{problem} 

\begin{solution}[\bf Solution.]

\end{solution}

\begin{problem}
Let $G$ a finite group of prime power order $p^a$, with $a > 0$. By considering the conjugation action of $G$, show that the centre $Z$ of $G$ is non-trivial.

Show that any group of order $p^2$ is abelian, and that there are up to isomorphism just two groups of that order for each prime $p$.

\end{problem} 

\begin{solution}[\bf Solution.]

\end{solution}

\begin{problem}
Find the conjugacy classes of elements in the alternating group $A_5$, and determine their sizes. 

Show that $A_5$ has no non-trivial normal subgroups (so $A_5$ is a \emph{simple} group).

Show that if $H$ is a proper subgroup of index $n$ in $A_5$ then $n > 4$. [Consider the left coset action of $A_5$ on the set of left cosets of $H$ in $A_5$.]

\end{problem} 

\begin{solution}[\bf Solution.]

\end{solution}

\begin{problem}
Let $G$ be a finite group of order $p^a m$, where $p^a$ is the highest power of the prime $p$ dividing $|G|$. Let $X$ be the set of all subsets of $G$ of size $p^a$. Show that $G$ acts on $X$ by $g : A \mapsto g A$ for $g \in G$ and $A\in X$.

Show that $X$ has size prime to $p$, and deduce that there is a $G$-orbit $X'$ of $G$ on $X$ of size prime to $p$. By considering the stabiliser of an element in $X'$, show that $G$ has a subgroup of order $p^a$, a \emph{Sylow} subgroup of $G$.

\end{problem} 

\begin{solution}[\bf Solution.]

\end{solution}

\begin{problem}
If $H$ is a subgroup of a finite group $G$ and $G$ has twice as many elements as $H$, show that $H$ is normal in $G$.

\end{problem} 

\begin{solution}[\bf Solution.]
If $\abs{G} = 2\abs{H}$, there are 2 left cosets of $H$ in $G$, and 2 right cosets - in each case, one coset is $H$ so the other must be $G\bs H$. Hence the left and right cosets of $H$ are the same; so $H$ is normal in $G$.
\end{solution}

\begin{problem}
Show that every subgroup of rotations in the dihedral group $D_{2n}$ is normal.
\end{problem} 

\begin{solution}[\bf Solution.]
If $r$ is a rotation in $D_{2n}$, its conjugacy class is $\bra{r,r^{-1}}$, since $grg^{-1} =r^{-1}$ is a rotation and $g$ is a reflection. Thus, if $H$ is a subgroup of rotations in $D_{2n}$, then $\forall g\in D_{2n}$, $h\in H$ we have $ghg^{-1}\in \bra{h,h^{-1}} \subseteq H$, so $H$ is normal in $G$.
\end{solution}


\begin{problem}
Show that a subgroup $H$ of a group $G$ is normal if and only if it is a union of conjugacy classes.

\end{problem} 

\begin{solution}[\bf Solution.]
$H$ is normal in $G\ \lra \ \forall g\in G,h\in H$ we have $ghg^{-1} \in H \ \lra \ H$ is closed under conjugation by elements of $G\ \lra \ H$ is a union of conjugacy classes.
\end{solution}


\begin{problem}
We know that in an abelian group every subgroup is normal. Now, let $G$ be a group in which every subgroup is normal, is it true that $G$ must be abelian?

\end{problem} 

\begin{solution}[\bf Solution.]
Let $G$ be the quaternian group $Q_8 = \bra{\pm 1,\pm i,\pm j,\pm k}$ - this is non-abelian, but its subgroups are all normal (to show this it suffices to consider subgroups of order 2 or 4; subgroups for order 4 are normal, and as -1 is the only element of order 2, the only subgroup of order 2 is $\bra{\pm 1}$, which is normal as it is the union of the conjugacy classes $\bra{1}$ and $\bra{-1}$).
\end{solution}

\begin{problem}
Show that $\Q/\Z$ is an infinite group in which every element has finite order.

\end{problem} 

\begin{solution}[\bf Solution.]
$\Q/\Z$ contains distinct cosets $\frac 1n + \Z$ for all $n\in \N$, so is infinite; given a coset $q+\Z$, write $q = \frac ab$ with $a\in \Z,b\in \N$, then $b(q+\Z) = bq + \Z = a +\Z:\Z$, so $q+\Z$ has order at most $b$, i.e., every element of $\Q/\Z$ has finite order.
\end{solution}

\begin{problem}
Let $G$ be the set of all $3 \times 3$ matrices of the form
\be
\bepm
1 & x & y\\
0 & 1 & z\\
0 & 0 & 1
\eepm,
\ee
with $x, y, z \in \R$. Show that $G$ is a subgroup of the group of invertible real matrices under multiplication. Let $H$ be the subset of $G$ given by those matrices with $x = z = 0$. Show that $H$ is a normal subgroup of $G$ and find $G/H$. [Use the isomorphism theorem.]

\end{problem}

\begin{solution}[\bf Solution.]
Since $\bepm 1 & 0 & 0\\ 0 & 1 & 0 \\ 0 & 0 & 1\eepm \in G$, $\bepm 1 & x & y\\ 0 & 1 & z \\ 0 & 0 & 1\eepm^{-1} = \bepm 1 & -x & xz -y\\ 0 & 1 & -z \\ 0 & 0 & 1\eepm \in G$ and $\bepm 1 & x & y\\ 0 & 1 & z \\ 0 & 0 & 1\eepm \bepm 1 & x' & y'\\ 0 & 1 & z' \\ 0 & 0 & 1\eepm = \bepm 1 & x+x' & y+y' + xz'\\ 0 & 1 & z+z' \\ 0 & 0 & 1\eepm\in G$, we see that $G$ is a subgroup of the group of invertible real matrices under multiplication. 

Define $\theta:G \to \R^2$ by $\theta \bepm 1 & x & y\\ 0 & 1 & z \\ 0 & 0 & 1\eepm = (x,z)$ - then from the above
\be
\theta \bb{\bepm 1 & x & y\\ 0 & 1 & z \\ 0 & 0 & 1\eepm \bepm 1 & x' & y'\\ 0 & 1 & z' \\ 0 & 0 & 1\eepm} = (x+x', z+z') = (x,z)+(x' +z') = \theta \bepm 1 & x & y\\ 0 & 1 & z \\ 0 & 0 & 1\eepm+\theta \bepm 1 & x' & y'\\ 0 & 1 & z' \\ 0 & 0 & 1\eepm, 
\ee
so $\theta$ is a homomorphism. Given $(a,b)\in \R^2$ we have $\theta \bepm 1 & a & 0\\ 0 & 1 & b \\ 0 & 0 & 1\eepm = (a,b)$, so $\theta$ is surjective, i.e., $\im \theta = \R^2$; since $\ker \theta = \bra{g\in G:\theta(g)=(0,0)} = \bra{\bepm 1 & x & y\\ 0 & 1 & z \\ 0 & 0 & 1\eepm,x=z=0} = H$, we see that $H$ is normal subgroup of $G$ and $G/H = G/\ker \theta \cong \im \theta = \R^2$.
\end{solution}



\begin{problem}
Consider the additive group $\C$ and the subgroup $\Gamma$ consisting of all Gaussian integers $m+in$, where $m, n \in \Z$. By considering the map
\be
x + iy \mapsto (e^{2\pi ix}, e^{2\pi iy}),
\ee
show that the quotient group $C/\Gamma$ is isomorphic to the torus $S^1 \times S^1$.

\end{problem} 

\begin{solution}[\bf Solution.]

Define $\theta:\C \to S^1 \times S^1$ by $\theta(x+iy) = \bb{e^{2\pi ix}, e^{2\pi iy}}$ - then 
\be
\theta((x+iy)+(x'+iy')) = \theta((x+x') + i(y + y')) =  \bb{e^{2\pi i(x+x')}, e^{2\pi i(y+y')}} = \bb{e^{2\pi ix}, e^{2\pi iy}}\cdot \bb{e^{2\pi ix'}, e^{2\pi iy'}} = \theta (x+iy) \theta (x'+iy'),
\ee
so $\theta$ is a homomorphism (and is visibly surjective). Since 
\be
\ker\theta = \bra{z\in \C:\theta(z) = (1,1)} = \bra{x+iy: e^{2\pi ix} = e^{2\pi iy} = 1} = \bra{m+in:m,n\in \Z} = \Gamma,
\ee
we have $\C/\Gamma = \C/\ker\theta \cong \im \theta = S^1 \times S^1$.

\end{solution}

\begin{problem}
Let $H$ be a subgroup of a group $G$. Show that $H$ is a normal subgroup of $G$ if and only if there is some group $K$, and some homomorphism $\theta : G \to K$, whose kernel is $H$.
\end{problem} 

\begin{solution}[\bf Solution.]
If $H$ is the kernel of a homomorphism it is certainly a normal subgroup of $G$; conversely if $H$ is normal in $G$ we have a quotient group $G/H$ and a quotient map $\theta:G\to G/H$ defined by $\theta (g) = g+H$, and then $\theta$ is a homomorphism whose kernel is $\bra{g\in G:g+H=H} =  H$.
\end{solution}

\begin{problem}
Let $GL(2,\R)$ be the group of all $2 \times 2$ invertible matrices and let $SL(2,\R)$ be the subset of $GL(2,\R)$ consisting of matrices of determinant 1. Show that $SL(2,\R)$ is a normal subgroup of $GL(2,\R)$. Show that the quotient group $GL(2,\R)/SL(2,\R)$ is isomorphic to the multiplicative group of non-zero real numbers.
\end{problem} 

\begin{solution}[\bf Solution.]
Write $\R^*$ for the multiplicative group of non-zero real numbers. Define $\theta: GL(2,\R)\to \R^*$ by $\theta(A) = \det A$ - as $\det AB = \det A \det B$, $\theta$ is a homomorphism (and is surjective since $\det \bepm x & 0 \\ 0 & 1\eepm = x$, $\forall x\in \R^*$). We have 
\be
\ker\theta = \bra{A\in GL(2,\R):\det A =1} = SL(2,\R);
\ee
so $SL(2,\R)$ is a normal subgroup of $GL(2,\R)$, and $GL(2,\R)/SL(2,\R) = GL(2,\R)/\ker\theta \cong \im \theta = \R^*$.
\end{solution}

\begin{problem}
Let $G$ be a finite group and $H \neq G$ a subgroup. Let $k$ be the cardinality of the set of left cosets of $H$ ($k$ is sometimes called the index of $H$) and suppose that $|G|$ does not divide $k!$. Show that $H$ contains a non-trivial normal subgroup of $G$. [Let $G$ act on the set of left cosets and reinterpret the action as a homomorphism from $G$ to the group of permutations of the set of left cosets.] Show that a group of order 28 has a normal subgroup of order 7. [Use Cauchy's theorem.]
\end{problem} 

\begin{solution}[\bf Solution.]
Let $X$ be the set of left coset of $H$ in $G$, i.e., $X = \bra{kH:k\in G}$ - then $G$ acts on $X$ by left multiplication, i.e., $g\cdot kH = gk H$. This gives a homomorphism $\theta : G\to \sym(X)$, where $\theta(g)$ is the bijection which sends each coset $kH$ to $gkH$. Clearly $\ker \theta \leq H$, since $g\in \ker\theta \ \ra \ \forall kH \in X$, $g\cdot kH = kH \ \ra \ g\cdot H = H \ \ra \ g\in H$; we have $G/\ker\theta \cong \im \theta \leq \sym(X)$, so by Lagrange's theorem $\abs{G/\ker\theta}$ must divide $\abs{\sym(x)} = k!$. Since $\abs{G}$ does not divide $k!$, we cannot have $\abs{\ker\theta} = 1$ - so $\ker\theta$ is a non-trivial normal subgroup of $G$ lying in $H$.

Let $G$ be a group of order 28. By Cauchy's theorem $G$ contains an element $g$ of order 7, so $H = \bsa{g}$ is a subgroup of order 7. Since there are $\frac{28}7 = 4$ left cosets of $H$ in $G$, and $28\nmid 24 = 4!$, $H$ contains a non-trivial normal subgroup of $G$ - but the only subgroups of $H$ are itself and $\bra{e}$, so the non-trivial normal subgroup must be $H$. Thus $G$ has a normal subgroup of order 7.
\end{solution}

\begin{problem}
Show that if a group $G$ of order 28 has a normal subgroup of order 4, then G is abelian. [Use previous question. You might wish to note that if $H$ is a subgroup of order 4 and $K$ is a subgroup of order 7, then $H \cap K = {e}$.]
\end{problem} 

\begin{solution}[\bf Solution.]
Let $G$ be a group of order 28; by previous question it has a normal subgroup $H$ of order 7. Assume $G$ also has a normal subgroup $K$ of order 4. By Lagrange's theorem $\abs{H\cap K}$ divides both $\abs{H} = 7$ and $\abs{K} = 4$, so we must have $H\cap K = \bra{e}$. Thus the 28 elements $hk$ for $h\in H$, $k\in K$ must all be distinct (as $hk = h'k' \ \ra \ h'h^{-1} = k'k^{-1} \in H\cap K \ \ ra \ h'h^{-1} = e = k'k^{-1} \ \ra \ h = h',k = k'$), so they form the whole of $G$. Given $h\in H$, $k\in K$, let $x = hkh^{-1}k^{-1}$ - then $x = hkh^{-1}\cdot k^{-1}\in K$ as $hkh^{-1}, k^{-1}\in K$, and also $x = h\cdot kh^{-1}k^{-1}\in H$ as $h,kh^{-1}k^{-1}\in H$ (since both $K$ and $H$ are normal subgroups); so $x\in H\cap K = \bra{e}$, i.e., $hkh^{-1}k^{-1} = e$ or $hk = kh$. Now as both $H$ and $K$ are abelian (as $H$ is cyclic and $K$ is either cyclic or $C_2 \times C_2$), $\forall hk,h'k'\in G$ we have $hk\cdot h'k' = hh' kk' = h'h k'k = h'k'\cdot hk$, so $G$ is abelian.
\end{solution}

\begin{problem}
Let $G$ be a subgroup of the group of isometries of the plane. Show that the set $T$ of translations in $G$ is a normal subgroup of $G$ ($T$ is called the translation subgroup). [If we think of the plane as $\C$ you may assume that all isometries have the form $z \mapsto az + b$ or $z \mapsto a\ol{z} + b$, where $a$ and $b$ are complex numbers and in both cases $|a| = 1$.]
\end{problem} 

\begin{solution}[\bf Solution.]
Take $t\in T$, where $t(z) = z+c$. If $g(z) = az+b$ then $g^{-1}(z) = a^{-1}z-a^{-1}b$, so 
\be
gtg^{-1}(z) = gt\bb{a^{-1}z-a^{-1}b} = g\bb{a^{-1}z-a^{-1}b + c} = z + ac,
\ee
while if $g(z) =a\ol{z} +b$ then $g^{-1}(z) = \ol{a}^{-1}\ol{z}-\ol{a}^{-1}\ol{b}$, so 
\be
gtg^{-1}(z) = gt \bb{\ol{a}^{-1}\ol{z}-\ol{a}^{-1}\ol{b}} = g \bb{\ol{a}^{-1}\ol{z}-\ol{a}^{-1}\ol{b} + c} = z + a\ol{c}
\ee
so if $g\in G$ then $gtg^{-1}$ is a translation in $G$, i.e., $gtg^{-1}\in T$. Thus $T$ is a normal subgroup of $G$.
\end{solution}






\begin{problem}
A frieze group is a group $F$ of isometries of $\C$ that leaves the real line invariant (that is, if $z \in\C$ has zero imaginary part and $g \in F$, then $g(z)$ also has zero imaginary part) and whose translation subgroup $T$ is infinite cyclic. If $F$ is a frieze group, classify $F\bs T$.
\end{problem} 

\begin{solution}[\bf Solution.]
We may assume $T$ is generated by $z\mapsto z+1$. We begin by analysing the possible types of elements of $F\bs T$. An isometry $z\mapsto az +b$ or $z \mapsto a\ol{z} +b$ which leaves $\R$ invariant must have both $a$ and $b$ real, and as $\abs{a} = 1$ this forces $a=\pm 1$. Thus there are three possible types of elements of $F\bs T:(I)z\mapsto -z +b$, $(II) z\mapsto \ol{z}+b$, $(III)z\mapsto -\ol{z}+b$.

Note that if $F$ contains two such elements $f$,$f'$ of the same type, then $f^{-1}f'$ will be a translation, so must lie in $T$ - so $F$ will contain a single coset $fT$ of elments of the type concerned, and it suffices to consider isometries of each type with $0\leq b <1$. Note that in type II, the square of $z\mapsto \ol{z}+b$ is $z\mapsto \ol{(\ol{z}+b)} +b = z + 2b$ which is a translation, so we must have $2b\in \Z$ and hence $b=0$ or $1/2$. Also, any element of type I or type III may be conjugated to $z\mapsto z$ or $z\mapsto -\ol{z}$ respectively.

Now a product of two elements of different types is an element of the third type - thus we may have elements of none, one or all three of the types I, II and III. Write 
\be
f_1(z) = -z,\quad f_2(z) = \ol{z},\quad f_3(z) = \ol{z}+ \frac 1z, \quad f_4(z) = - \ol{z},\quad f_5(z) = - \ol{z}+ \frac 1z.
\ee

Then it follows that $F/T$ is conjugate to one of the following, where we write $\ol{g}$ to denote the coset $gT$, and $e(z) =z$, (i) $\bra{e}$; (ii) $\bra{\ol{e},\ol{f_1}}$; (iii) $\bra{\ol{e},\ol{f_2}}$; (iv) $\bra{\ol{e}, \ol{f_3}}$; (v) $\bra{\ol{e},\ol{f_4}}$; (vi) $\bra{\ol{e},\ol{f_1},\ol{f_2},\ol{f_4}}$; (vii) $\bra{\ol{e},\ol{f_1},\ol{f_3},\ol{f_5}}$. Thus $F/T \cong \bra{e},C_2\text{ or } C_2 \times C_2$. 
\end{solution}

\begin{problem} 
Consider the M\"obius maps $f(z) = e^{2\pi i/n}z$ and $g(z) = 1/z$. Show that the subgroup $G$ of the M\"obius group $M$ generated by $f$ and $g$ is a dihedral group of order $2n$.

\end{problem} 

\begin{solution}[\bf Solution.]

\end{solution}

\begin{problem}Let $g(z) = (z + 1)/(z - 1)$. By considering the points $g(0)$, $g(\infty)$, $g(1)$ and $g(i)$, find the image of the real axis $\R$ and of the imaginary axis $\I$ under $g$. What is $g(\Sigma)$, where $\Sigma$ is the first quadrant in $\C$?

\end{problem} 

\begin{solution}[\bf Solution.]

\end{solution}

\begin{problem}What is the order of the M\"obius map $f(z) = iz$? If $h$ is any M\"obius map, find the order of $h f h^{-1}$ and its fixed points. Use this to construct a M\"obius map of order four that fixes 1 and -1.

\end{problem} 

\begin{solution}[\bf Solution.]

\end{solution}


\begin{problem}Show that the set $SL_2(\Z)$ of all $2 \times 2$ matrices of determinant 1 with integer entries is a group under multiplication.

\end{problem} 

\begin{solution}[\bf Solution.]

\end{solution}

\begin{problem}Let $G$ be the group of M\"obius transformations which map the set $\{0, 1,\infty\}$ onto itself. Find all the elements in $G$. To which standard group is $G$ isomorphic? Justify your answer.

Find the group of M\"obius transformations which map the set $\{0, 2,\infty\}$ onto itself. [Try to do as little calculation as possible.]

\end{problem} 

\begin{solution}[\bf Solution.]


\end{solution}

\begin{problem}Let $G$ be as in the previous question. Show that, given $\sigma\in S_4$, there exists $f_\sigma \in G$ for which, whenever $z_1$, $z_2$, $z_3$ and $z_4$ are four distinct points in $\C_\infty$, we have $f_\sigma([z_1, z_2, z_3, z_4]) = [z_{\sigma(1)}, z_{\sigma(2)}, z_{\sigma(3)}, z_{\sigma(4)}]$.

Show that the map $\sigma \mapsto f_{\sigma^{-1}}$ from $S_4$ to $G$ gives a homomorphism from $S_4$ onto $S_3$. Find its kernel.

\end{problem} 

\begin{solution}[\bf Solution.]

\end{solution}

\begin{problem}The \emph{centre} of a group $G$ consists of all those elements of $G$ that commute with all the elements of $G$. Show that the centre $Z$ of the general linear group $GL_2(\C)$ consists of all scalar matrices. Identify the centre of the special linear group $SL_2(\C)$.

\end{problem} 

\begin{solution}[\bf Solution.]

\end{solution}

\begin{problem}Let $G$ be the set of all $3\times 3$ real matrices of determinant 1 of the form 
\be
\bepm
a & 0 & 0\\
b & x & y\\
c & z & w
\eepm.
\ee
Verify that $G$ is a group. Find a homomorphism from $G$ onto the group $GL_2(\R)$ of all non-singular $2\times 2$ real matrices, and find its kernel.

\end{problem} 

\begin{solution}[\bf Solution.]

\end{solution}

\begin{problem}When do two elements of $SO_3$ commute?

\end{problem} 

\begin{solution}[\bf Solution.]

\end{solution}

\begin{problem}Let $K$ be a normal subgroup of order 2 in the group $G$. Show that $K$ lies in the centre of $G$, that is $kg = gk$ for all $k \in K$ and $g \in G$.

Describe a surjective homomorphism of the orthogonal group $O(3)$ onto $C_2$ and another onto the special orthogonal group $SO(3)$.

\end{problem} 

\begin{solution}[\bf Solution.]

\end{solution}

\begin{problem}If $A$ is a complex $n \times n$ matrix with entries $a_{ij}$, let $A^*$ be the complex $n \times n$ matrix $\bar{A}^t$ with entries $\bar{a}_{ji}$.

The matrix $A$ is called unitary if $AA^* = I$. Show that the set $U(n)$ of unitary matrices forms a group under matrix multiplication. Show that
\be
SU(n) = \{A \in U(n) : \det A = 1\}
\ee
is a normal subgroup of $U(n)$ and that $U(n)/SU(n)$ is isomorphic to $S^1$, the group of the unit circle in $\C$ under multiplication. Show that $SU(2)$ contains the quaternion group $Q_8$ as a subgroup.

\end{problem} 

\begin{solution}[\bf Solution.]

\end{solution}

\begin{problem}Let $G$ be the special linear group $SL_2(5)$ of $2\times 2$ matrices of determinant 1 over the field $\F_5$ of integers modulo 5, so that the arithmetic in $G$ is modulo 5. Show that $G$ is a group of order 120. Prove that $-I$ is the only element of $G$ of order 2.

Find a subgroup of $G$ of order 8 isomorphic to $Q_8$, and an element of order 3 normalising it in $G$. Deduce that $G$ has a subgroup of index 5, and obtain a homomorphism from $G$ to $S_5$. Deduce that $SL_2(5)/\{\pm I\}$ is isomorphic to the alternating group $A_5$. Show that $SL_2(5)$ has no subgroup isomorphic to $A_5$.

\end{problem} 

\begin{solution}[\bf Solution.]

\end{solution}



\begin{problem}What is the largest possible order of an element in $S_5$? And in $S_9$? Show that every element in $S_{10}$ of order 14 is odd.

\end{problem} 

\begin{solution}[\bf Solution.]Largest order of an element in $S_5$ is 6, with cycle type 32 (e.g. (123)(45)); in $S_9$ it is 20, with cyple type 54 (e.g. (12345)(6789)). In $S_{10}$ on element of order 14 must have cycles of lengths divisible by 7 and 2, so its cycle type is 721, giving sign $(-1)^{6+1} = -1$, i.e., it is odd. 

\end{solution}

\begin{problem} Show that any subgroup of $S_n$ which is not contained in $A_n$ contains an equal number of odd and even permutations.

\end{problem} 

\begin{solution}[\bf Solution.]Let $H\leq S_n$ with $H\nleq A_n$, and take $\pi \in H\bs A_n$. Then $x\mapsto \pi x$ is a map from $H\cap A_n$ to $H\bs A_n$ (since if $x$ is even, $\pi x$ is odd); it is injective as $\pi x = \pi y \ \ra \ x = y$, and it is surjective because if $z\in H\bs A_n$ then $\pi^{-1}z\in H\cap A_n$ so it is bijective, so $\abs{H\cap A_n} = \abs{H\bs A_n}$, i.e., $H$ contains an equal number of odd and even permutations.

\end{solution}

\begin{problem} Let $N$ be a normal subgroup of the orthogonal group $O(2)$. Show that if $N$ contains a reflection in some line through the origin, then $N = O(2)$.

\end{problem} 

\begin{solution}[\bf Solution.]Take $r\in N$ a reflection about an axis $l$. If $r'$ is another reflection in $O(2)$, about an axis $l'$, let $p\in O(2)$ be a rotation which sends $l$ to $l'$; then $prp^{-1}$ is reflection with $prp^{-1}(l') = pr(l) = p(l) = l'$, i.e., $prp^{-1} = r'$ - so as $N\lhd O(2)$ we have $r'\in N$. Now let $p \in O(2)$ be a rotation; then $pr\in O(2)$ is a reflection, so $pr\in N$, so $p\in N$. Thus $N = O(2)$.

\end{solution}

\begin{problem} Show that $S_n$ is generated by the two elements (12) and $(123 \dots n)$.

\end{problem} 

\begin{solution}[\bf Solution.]Let $H = \bsa{(12),(123\dots n)}$. We have $(123\dots n)(12)(123\dots n)^{-1} = (23)$, then $(123\dots n)(23)(123\dots n)^{-1} = (34)$ - in general $(123\dots n)^k(12)(123\dots n)^{-k} = (k+1\ k+2)$, so $H$ contains $(12),(23),\dots (n-1\ n)$. Now if $i<j$ we may write $(i\ j) = (i\ i+1)(i+1\ i+2)\dots (j-2\ j-1)(j-1\ j)(j-2\ j-1) \dots (i+1\ i+2)(i \ i+1)$; so $H$ contains all transpositions. Since any element of $S_n$ is a product of transpositions, we have $H=S_n$, i.e., $S_n$ is generated by $(12)$ and $(123\dots n)$.

\end{solution}

\begin{problem}
Let $H$ be a normal subgroup of a group G and let $K$ be a normal subgroup of $H$. Is it true that $K$ must be a normal subgroup of $G$?
\end{problem} 

\begin{solution}[\bf Solution.]Let $G=S_4$, $H=\bsa{(12)(34),(13)(24)}$ and $K = \bsa{(12)(34)}$ - then $H\lhd G$ since it is the union of the conjugacy classes $\bra{1}$ and $\bra{(12)(34),(14)(23)}$, and $K \lhd H$ as $H$ is abelian, but $K\ntriangleleft G$ as $(13)(12)(34)(13) = (14)(23)\notin K$.

\end{solution}

\begin{problem} Find the elements in $S_n$ that commute with (12).

\end{problem} 

\begin{solution}[\bf Solution.]Take $\pi\in S_n$ commuting with $(12)$ - then $K=\bsa{(12)(34)}$ - then $\pi(1) = \pi(12)(2) = (12)\pi (2)$, so (12) sends $\pi(1)$ to $\pi(2)$ so $\bra{\pi(1),\pi(2)} = \bra{1,2}$, so $\pi$ either fixes or interchanges 1 and 2. Conversely such a permutation clearly commutes with (12).

\end{solution}

\begin{problem} Let $z_1$, $z_2$, $z_3$ and $z_4$ be four distinct points in $\C_\infty$ and let $\lm = [z_1, z_2, z_3, z_4]$ be the cross ratio of the four points. Let $G$ be the group of M\"obius maps which map the set $\bra{0,1,\infty}$ onto itself. Show that given $\sigma \in S_4$, there exists $f_\sigma \in G$ such that $f_\sigma (\lm) = [z_{\sigma(1)}, z_{\sigma(2)}, z_{\sigma(3)}, z_{\sigma(4)}]$.

\end{problem} 

\begin{solution}[\bf Solution.]We know that $\bsb{z_1,z_2,z_3,z_4} = \bsb{w_1,w_2,w_3,w_4} \ \lra \ \exists $ a M\"obius map $g$ with $g(z_j) = w_j$ for $j= 1,2,3,4$. Thus gives $\sigma \in S_4$, the map $f_8:\C_\infty \to \C_\infty$ defined by $f_\sigma([z_1,z_2,z_3,z_4]) = [z_{\sigma(1)},z_{\sigma(2)},z_{\sigma(3)},z_{\sigma(4)}]$ is well-defined, because if $[z_1,z_2,z_3,z_4] = [w_1,w_2,w_3,w_4]$ $\exists $ a M\"obius map $g$ with $g(z_j) = w_j$ for all $j$, and then 
\be
\bsb{z_{\sigma(1)},z_{\sigma(2)},z_{\sigma(3)},z_{\sigma(4)}} = \bsb{g(z_{\sigma(1)}),g(z_{\sigma(2)}),g(z_{\sigma(3)}),g(z_{\sigma(4)})} = \bsb{w_{\sigma(1)},w_{\sigma(2)},w_{\sigma(3)},w_{\sigma(4)}};
\ee
we must shwo that $f_\sigma \in G$. Given $\sigma,\pi\in S_4$, for $1\leq j\leq 4$ set $w_j = z_{\sigma(j)}$,
\beast
f_\pi\bb{f_\sigma(\lm)} & = & f_\pi\bb{\bsb{z_{\sigma(1)},z_{\sigma(2)},z_{\sigma(3)},z_{\sigma(4)}}} = f_\pi \bb{\bsb{w_1,w_2,w_3,w_4}} \\
& = & \bsb{w_{\pi(1)},w_{\pi(2)},w_{\pi(3)},w_{\pi(4)}} = \bsb{z_{\sigma\pi(1)},z_{\sigma\pi(2)},z_{\sigma\pi(3)},z_{\sigma\pi(4)}} = f_{\sigma \pi}(\lm),
\eeast
so $f_\pi f_\sigma = f_{\sigma \pi}$, $f_\pi f_\sigma = f_{\pi^{-1}}f_{\sigma^{-1}} = f_{\sigma^{-1} \pi^{-1}} = f_{(\pi \sigma)^{-1}}$, so the map sending $\sigma$ to $f_{\sigma^{-1}}$ is a group homomorphism from $S_4$ into the group of invertible maps $\C_\infty \to \C_\infty$. Now let $g$ be the M\"obius map sending $z_1,z_2,z_3$ to 0,1,$\infty$ respectively - then 
\be
\lm = [z_1,z_2,z_3,z_4] = [g(z_1),g(z_2),g(z_3),g(z_4)] = [0,1,\infty,g(z_4)] = g(z_4).
\ee

Thus 
\be
f_{(12)}(\lm) = [z_2,z_1,z_3,z_4] = [g(z_2),g(z_1),g(z_3),g(z_4)] = [1,0,\infty,\lm] = \frac{(\lm-1)(0-\infty)}{(\lm -\infty)(0-1)} = 1-\lm,
\ee
while 
\be
f_{(1234)}(\lm) = [z_2,z_3,z_4,z_1] = [g(z_2),g(z_3),g(z_4),g(z_1)] = [1,\infty,\lm, 0] = \frac{(0-1)(\infty-\lm)}{(0-\lm)(\infty-1)} = \frac 1{\lm}.
\ee

So $f_{(12)},f_{(1234)}\in G$, so as (12) and (1234) generate $S_4$ by previous question and $\sigma \mapsto f_{\sigma^{-1}}$ is a homomorphism, we must have $f_{\sigma}\in G, \forall \sigma \in S_4$.
\end{solution}

\begin{problem} Show that the map $S_4 \ni \sigma \mapsto f_{\sigma^{-1}} \in G \cong S_3$ given by the previous question is a surjective homomorphism. Find its kernel.

\end{problem} 

\begin{solution}[\bf Solution.]In previous question we showed that $\sigma \mapsto f_{\sigma^{-1}}$ is a homomorphism $S_4\to G$ - as $f_{(12)}$ and $f_{(1234)}$ generate $G$, the homomorphism is surjective. Thus if $K$ is the kernel, then $\abs{K} = \frac{\abs{S_4}}{\abs{G}} = \frac{24}6 = 4$ - we have 
\be
f_{(12)(34)}(\lm) = \bsb{z_2,z_1,z_4,z_3} = \frac{(z_3-z_2)(z_1 -z_4)}{(z_3-z_4)(z_1-z_2)} = \lm,\quad f_{(13)(24)}(\lm) = \bsb{z_3,z_4,z_1,z_2} = \frac{(z_2-z_3)(z_4-z_1)}{(z_2-z_1)(z_4-z_3)} = \lm,
\ee
so $(12)(34),(13)(24)\in K$, thus as $\abs{\bsa{(12)(34),(13)(24)}} = 4$ we must have $K = \bsa{(12)(34),(13)(24)}$.

\end{solution}

\begin{problem} Let $X$ be the set of all $2 \times 2$ real matrices with trace zero. Given $A \in SL(2,\R)$ and $B \in X$, show that $(A,B) \mapsto ABA^{-1}$ defines an action of $SL(2,\R)$ on $X$. Find the orbit and stabilizer of
\be
B = \bepm 0 & 1\\ 0 & 0\eepm.
\ee

Show that the set of matrices in $X$ with zero determinant is a union of 3 orbits.

\end{problem} 

\begin{solution}[\bf Solution.]If $\tr B = 0$ then $\tr(ABA^{-1}) = \tr(A^{-1}AB) = \tr(B) = 0$, so $ABA^{-1}\in X$ - clearly $IBI^{-1} = B$, and given $A,A'\in SL(2,\R)$ we have 
\be
\rho(A,\rho(A',B)) = \rho(A,A'BA^{-1}) = AA' BA'^{-1}A^{-1} = (AA')B(AA')^{-1} = \rho(AA',B),
\ee
so $(A,B)\mapsto ABA^{-1}$ defines on action of $SL(2,\R)$ on $X$.

If $A = \bepm a & b\\ c& d \eepm \in SL(2,\R)$, then 
\be
A\bepm 0 & 1 \\ 0 & 0 \eepm A^{-1} = \bepm a & b\\ c& d \eepm \bepm 0 & 1\\ 0 & 0 \eepm \bepm d & -b\\ -c & a \eepm  = \bepm 0 & 0\\ 0 & c \eepm \bepm d & -b\\ -c & a \eepm = \bepm -ac & a^2\\ -c^2 & ac \eepm;
\ee
so the orbit of $B = \bepm 0 & 1\\ 0 & 0 \eepm$ is
\be
\bra{\bepm -ac & a^2\\ -c^2 & ac \eepm:a,c\in \R,(0,c)\neq (0,0)},
\ee
and $A\in \stab(B)\ \lra\ c=0,a=\pm 1$ so the stabilizer of $B$ is $\bra{\pm \bepm 1 & b\\ 0 & 1\eepm:b\in \R}$. Likewise if $C= \bepm 0 & 0 \\ 1& 0 \eepm$ then 
\be
ACA^{-1} = \bepm a & b \\ c & d \eepm \bepm 0 & 0 \\ 1& 0 \eepm\bepm d & -b \\ -c & a \eepm = \bepm b & 0 \\ d & 0 \eepm\bepm d & -b \\ -c& a \eepm = \bepm bd & -b^2 \\ d^2 & -bd \eepm
\ee
so the orbit of $C$ is $\bra{\bepm bd & -b^2 \\ d^2 & -bd \eepm: b,d\in \R, (b,d)\neq (0,0)}$. Now if $M= \bepm u & v \\ w & -u \eepm \in X$ with $\det M = 0$, then $-u^2 - vw = 0$, so $vw = -u^2 \leq 0$. So either $v\geq 0\geq w$ or $v\geq 0 \geq w$ and $(v,w)\neq (0,0)$, let $a= \sqrt{v}$, $c=\pm \sqrt{-w}$ then $(ac)^2= -vw = u^2$, so we may choose the sign of $c$ to ensure $ac=-u$, and then $M = \bepm ac & a^2 \\ -c^2 & ac \eepm$ so $M$ lie in the orbit of $B$. Simiarly if $w\geq 0\geq v$ and $(v,w)\neq (0,0)$, let $b=\sqrt{-v}$, $d = \pm \sqrt{w}$ - then $(bd)^2 = -vw = u^2$, so we may choose the sign of $d$ to ensure $bd = u$, and then $M = \bepm bd & -b^2 \\ d^2 & -bd \eepm$ so $M$ lies in the orbit of $C$. Finally if $v=w=0$ then $u^2 = 0$ so $u=0$, and $M = \bepm 0 & 0 \\ 0 & 0 \eepm$ (which of course is an orbit on its own). Thus the set of matrices in $X$ with zero determinant is the union of the orbit of $B,C$ and $\bepm 0 & 0 \\ 0 & 0 \eepm$.
\end{solution}

\begin{problem} When do two elements in $SO(3)$ commute?

\end{problem} 

\begin{solution}[\bf Solution.]Take $A,B\in SO(3)$ having axes $l,m$ - then $BAB^{-1}$ has axis $B(l)$, so it $A$ and $B$ commute then $B(l) =l$. If $B$ presents the orientation of $l$, it fixes $l$ pointwise, so we must have $m=l$, i.e., $A$ and $B$ have the same axis. If instead $B$ reverses $l$, then $B$ must be a rotation by $\pi$, and $m$ must be orthogonal to $l$; arguing similarly with $ABA^{-1}$ we see that $A$ and $B$ are both rotations by $\pi$ about orthogonal axes.
\end{solution}

\begin{problem} 
If $A$ is a complex $n \times n$ matrix with entries $a_{ij}$, let $A^*$ be the complex $n\times n$ matrix with entries $\ol{a_{ji}}$. The matrix $A$ is called unitary if $AA^* = I$. Show that the set $U(n)$ of unitary matrices forms a group under matrix multiplication. Show that
\be
SU(n) = \bra{A \in U(n) : \det A = 1}
\ee
is a normal subgroup of $U(n)$ and that $U(n)/SU(n)$ is isomorphic to $S^1$. Show that $SU(2)$ contains the quaternion group $Q_8$ as a subgroup.

\end{problem} 

\begin{solution}[\bf Solution.]
Certainly if $A\in U(n)$ then $A$ is invertible, so $U(n) \subseteq GL(n)$; we have $II^* = II = I$ so $I \in U(n)$; given $A,B\in U(n)$ we have $AA^* = BB^* = I$, so $(AB)(AB)^* = ABB^*A^* = AA^* = I$, so $AB\in U(n)$; given $A\in U(n)$ we have $AA^* = I$, so $A^{-1} = A^*$, so
\be
A^{-1}(A^{-1})^* = A^{-1}A = I \ \ra \ A^{-1} = U(n).
\ee

Thus $U(n)$ is a group. Define $\theta : U(n) \to \C^\times$ by $\theta(A) =\det(A)$ - then $\theta$ is a homomorphism with kernel $SU(n)$; if $A\in U(n)$ then we have $1=\det AA^* =\det A \det A^* = (\det A)\ol{(\det A)}$, so $\det A\in S^1$, i.e., $\im\theta \leq S^1$; given $z\in S^1$, let $A =\bepm z_1 & & \\ & \ddots & \\ & & 1 \eepm$, then $A\in U(n)$ and $\det(A) = 2$, so $\im \theta = S^1$. Hence $SU(n)\lhd U(n)$ and $U(n)/SU(n) = U(n)/\ker\theta \cong \im \theta = S^1$.

In $SU(2)$ write $1 = \bepm 1 & \\ & 1\eepm$, $i = \bepm i & \\ & -i\eepm$, $j = \bepm & 1 \\ -1 & \eepm$, $k = \bepm & i \\ i & \eepm$ - then $i^2 = j^2 = k^2 = ijk = -1$, so $\bra{\pm 1,\pm i,\pm j, \pm k} = Q_8$.
\end{solution}

\begin{problem} 
Show that any subgroup of $A_5$ has order at most 12. [Use Question in Example Sheet 3.]
\end{problem} 

\begin{solution}[\bf Solution.]
Take $H\leq A_5$ and suppose $\abs{H}>12$ - then there are at most 4 left cosets of $H$ in $A_5$, so by previous of sheet 3 $H$ contains a non-trivial normal subgroup of $A_5$; as $A_5$ is simple this normal subgroup must be $A_5$ itself, so we must have $H=A_5$. Thus any prper subgroup of $A_5$ has order at most 12.

\end{solution}

\begin{problem} Let $G$ be a finite non-trivial subgroup of $SO(3)$. Let $X$ be the set of points on the unit sphere in $\R^3$ which are fixed by some non-trivial rotation in $G$. Show that $G$ acts on $X$ and that the number of orbits is either 2 or 3. What is $G$ if there are only two orbits? [With more work one can show that if there are three orbits, then $G$ must be dihedral or the group of rotational symmetries of a Platonic solid.]

\end{problem} 

\begin{solution}[\bf Solution.]
Take $x\in X$, so $\exists g\in G$ non-trivial with $gx = x$; gives $h\in G$, we have $hgh^{-1}$ non-trivial and $(hgh^{-1})(hx) = hgx = hx$, so that $hx$ is fixed by $hgh^{-1}$ and therefore lies in $X$. Hence we have a map $\rho:G\times X \to X$ given by $\rho(h,x) = hx$; certainly $\rho(I,x) =x$, $\forall x\in X$, and $\rho(g,\rho(h,x)) = \rho(g,hx) = g(hx) = \rho(gh,x)$, so $\rho$ is an action. 

Set $\Gamma = \bra{(g,x)\in G\times X: gx = x}$, and let the orbits of $G$ on $X$ be $O_1,\dots, O_k$. By the orbit-stabilizer theorem, if $x\in O_i$ then $\abs{\stab(x)} = \frac{\abs{G}}{\abs{O_i}}$. Thus 
\be
\abs{\Gamma} = \sum^k_{i=1}  \sum_{x\in O_i}\abs{\stab(x)} = \sum^k_{i=1} \sum_{x\in O_i} \frac{\abs{G}}{\abs{O_i}} = \sum^k_{i=1} \abs{O_i} \cdot \frac{\abs{G}}{\abs{O_i}} = \sum^k_{i=1}\abs{G} = k\abs{G} .
\ee

On the other hand, if $g\in G$ is non-trivial then $g$ fixes just 2 points in $X$ (while $I$ fixes all points in $X$), so $\abs{\Gamma} = \abs{X} + 2(\abs{G}-1)$, so $\abs{X} + 2\abs{G}-2 = k\abs{G}$. So $\abs{X} -2 = (k-2)\abs{G}$ - as $G$ is non-trivial $\exists$ at least 2 points in $X$, so $(k-2)\abs{G}\geq 0$ and hence $k\geq 2$; on the other hand $\abs{X} leq 2(\abs{G}-1)$ since each non-identity element of $G$ contributes just 2 points to $X$, so $(k-2)\abs{G} \leq 2(\abs{G}-1)-2 = 2\abs{G} -4$, so $k =2$ or 3, i.e., the number of orbits is 2 or 3. If $k=2$ then $\abs{X}-2 = 0$ so $\abs{X}=2$, and thus all elements of $G$ have the same axis. So $G$ is cyclic.
\end{solution}

%%%%%%%%%%%%%%%%%%%%%%%%%%%%%%%%%%%%%%%%%%%%%%%%%%%%%%%%%%%%%%%%%%%%%%%%%%%%%%%%%%%%%%%%%%%%%%%%%%%%%%%%%%%%%%%%%%%%%%%%%%%%%%%%%%%%%%

