\chapter{Functional Analysis (draft)}

\section{Some notes on prerequisites} Many years ago
it was more or less clear what could and what could
not be assumed in an introductory functional analysis
course. Since then, however, many of the concepts
have drifted into courses at lower levels. 

I shall therefore assume that you know what is a normed space,
and what is a
a linear map and that you can do the following exercise.
\begin{exercise} Let $(X,\|\ \|_{X})$ and $(Y,\|\ \|_{Y})$
be normed spaces.

(i) If $T:X\rightarrow Y$ is linear, then $T$ is continuous
if and only if there exists a constant $K$ such that
\[\|Tx\|_{Y}\leq K\|x\|_{X}\]
for all $x\in X$.

(ii) If $T:X\rightarrow Y$ is linear and $x_{0}\in X$,
then $T$ is continuous at $x_{0}$
if and only if there exists a constant $K$ such that
\[\|Tx\|_{Y}\leq K\|x\|_{X}\]
for all $x\in X$. 

(iii) If we write ${\mathcal L}(X,Y)$ for the space of continuous
linear maps from $X$ to $Y$ and write
\[\|T\|=\sup\{\|Tx\|_{Y}\,:\, \|x\|_{X}=1,\ x\in X\}\]
then $({\mathcal L}(X,Y),\|\ \|)$  is a normed space.
\end{exercise}

I also assume familiarity with the concept of a metric
space and a complete metric space. You should be able
to do at least parts~(i) and~(ii) of the following exercise
(part~(iii) is a little harder).
\begin{exercise}  Let $(X,\|\ \|_{X})$ and $(Y,\|\ \|_{Y})$
be normed spaces.

(i) If $(Y,\|\ \|_{Y})$ is complete then 
$({\mathcal L}(X,Y),\|\ \|)$ is.

(ii) Consider the set $s$ of sequences $x=(x_{1},x_{2},\dots)$
in which only finitely many of the $x_{j}$ are non-zero.
Explain briefly how $s$ may be considered as a vector space.
If we write
\[\|x\|=\sup_{j}|x_{j}|\]
show that $(s,\|\ \|)$ is a normed vector space which
is not complete.

(iii) If $(X,\|\ \|_{X})$ is complete does it follow that 
$({\mathcal L}(X,Y),\|\ \|)$ is? Give a proof or a counter-example.
\end{exercise}

The reader will notice that I have not distinguished
between vector spaces over ${\mathbb R}$ and those over ${\mathbb C}$.
I shall try to make the distinction when it matters but,
if the two cases are treated in the same way, I shall
often proceed as above.

Although I shall stick with metric spaces as much
as possible, there will be points where we shall need
the notions of a topological space, a compact
topological space and a Hausdorff topological space.
I would be happy, if requested, to give a supplementary
lecture introducing these notions. (Even where I use them,
no great depth of understanding is required.)

I shall also use, without proof, the famous
Stone-Weierstrass theorem.
\begin{theorem} (A) Let $X$ be a compact space and
$C(X)$ the space of real valued continuous functions
on $X$. Suppose $A$ is a subalgebra of $C(X)$ (that is a
subspace which is algebraically closed under multiplication)
and

(i) $1\in A$, 

(ii) Given any two distinct points $x$ and $y$
in $X$ there is an $f\in A$ with $f(x)\neq f(y)$.

Then $A$ is uniformly dense in $C(X)$.

(B) Let $X$ be a compact space and
$C(X)$ the space of complex valued continuous functions
on $X$. Suppose $A$ is a subalgebra of $C(X)$
and

(i) $1\in A$, 

(ii) Given any two distinct points $x$ and $y$
in $X$ there is an $f\in A$ with $f(x)\neq f(y)$.

(iii) If $f\in A$ then $f^{*}\in A$.

Then $A$ is uniformly dense in $C(X)$.
\end{theorem}

The proof will not be examinable, but if you have not
met it, you may wish to request a supplementary lecture
on the topic.

Functional analysis goes hand in hand with measure theory.
Towards the end of the course I will need to refer
Borel measures on the line. However, I will not
use any theorems from measure theory proper
and I will make my treatment independent of previous knowledge.
Elsewhere I may make a few remarks involving measure
theory. These are for interest only and will not be 
examinable\footnote{In this course, as in other Part~III
courses you should assume that everything in the
lectures and nothing outside them is examinable
unless you are explicitly told to the contrary. If you
are in any doubt, ask the lecturer.}. I intend the course
to be fully accessible without measure theory.
\section{Baire category} If $(X,d)$ is a metric space
we say that a set $E$ in $X$ has dense 
complement\footnote{If the lecturer uses the words
`nowhere dense' correct him for using an old fashioned
and confusing terminology} if,
given $x\in E$ and $\delta>0$, we can find a $y\notin E$
such that $d(x,y)<\delta$.
\begin{exercise} Consider the space $M_{n}$ of $n\times n$
complex matrices with an appropriate norm. Show that
the set of matrices which do not have $n$ distinct
eigenvalues is a closed set with dense complement.
\end{exercise}

\begin{theorem}[Baire's theorem]
If $(X,d)$ is a complete metric
space and $E_{1}$, $E_{2}$, \dots are
closed sets with dense complement
then $X\neq \bigcup_{j=1}^{\infty}E_{j}$.
\end{theorem}
\begin{exercise} (If you are happy with general topology.)
Show that a result along the same lines holds true
for compact Hausdorff spaces.
\end{exercise}
We call the countable union of closed
sets with dense complement
a set of first category. The following observations
are trivial but useful.
\begin{lemma} (i) The countable union of first
category sets is itself of first category.

(ii) If $(X,d)$ is a complete metric
space, then Baire's theorem asserts that $X$
is not of first category.
\end{lemma}
\begin{exercise}
If $(X,d)$ is a complete metric space
and $X$ is countable show that there
is an $x\in X$ and a $\delta>0$ such that
the ball $B(x,\delta)$ with centre $x$ and
radius $\delta$ consists of one point.
\end{exercise}

The following exercise is a standard application of
Baire's theorem.
\begin{exercise} Consider the space $C([0,1])$
of continuous functions under the uniform norm
$\|\ \|$. Let
\begin{align*}
E_{m}=\{f\in C([0,1])\,:&\, \text{there exists an $x\in [0,1]$
with}\\
&\text{$|f(x+h)-f(x)|\leq m|h|$ for all $x+h\in[0,1]$}\}.
\end{align*}

(i) Show that $E_{m}$ is closed in $(C([0,1],\|\ \|_{\infty})$.

(ii) If $f\in C([0,1])$ and $\epsilon>0$ explain why we can
find an infinitely differentiable function $g$ such that
$\|f-g\|_{\infty}<\epsilon/2$. By considering the function $h$
given by
\[h(x)=g(x)+\tfrac{\epsilon}{2}\sin Nx\]
with $N$ large show that $E_{m}$ has dense complement.

(iii) Using Baire's theorem show that there exist continuous
nowhere differentiable functions.
\end{exercise}
\begin{exercise} (This is quite long and not very
central.)

(i) Consider the space ${\mathcal F}$ of non-empty
closed sets in $[0,1]$. Show that if we write
\[d_{0}(x,E)=\inf_{e\in E}|x-e|\]
when $x\in[0,1]$ and $E\in{\mathcal F}$
and write
\[d(E,F)=\sup_{f\in F}d_{0}(f,E)+\sup_{e\in E}d_{0}(e,F)\]
then $d$ is a metric on ${\mathcal F}$.

(ii) Suppose $E_{n}$ is a Cauchy sequence in 
$({\mathcal F},d)$. By considering
\[E=\{x\,:\,\text{there exist $e_{n}\in E_{n}$ such that
$e_{n}\rightarrow x$}\},\]
or otherwise, show that $E_{n}$ converges.
Thus $({\mathcal F},d)$ is complete.

(iii) Show that the set
\[\mathcal{A}_{n}=\{E\in{\mathcal F}\,:\,\text{there exists
an $x\in E$ with $(x-1/n,x+1/n)\cap E=\{x\}$}\}\]
is closed with dense complement in $({\mathcal F},d)$.
Deduce that the set of elements of ${\mathcal F}$ with
isolated points is of first category. (A set $E$ has
an isolated point $e$ if we can find a $\delta>0$
such that $(e-\delta,e+\delta)\cap E=\{e\}$.)

(iv) Let $I=[r/n,(r+1)/n]$ with $0\leq r\leq n-1$ and
$r$ and $n$ integers. Show that the set
\[\mathcal{B}_{r,n}=
\{E\in{\mathcal F}\,:\,E\supseteq I\}\]
is closed with dense complement in $({\mathcal F},d)$.
Deduce that the set of elements of ${\mathcal F}$
containing an open interval is of first category.

(v) Deduce the existence of non-empty closed sets
which have no isolated points and contain no intervals.
\end{exercise} 
\section{Non-existence of functions of several variables}
This course is very much a penny plain rather than tuppence
coloured\footnote{And thus suitable for those
`who want from books plain cooking made still
plainer by plain cooks'.}. One exception is the
theorem proved in this section.
\begin{theorem}\label{Hilbert; Theorem} Let $\lambda$ be irrational
We can find increasing continuous functions
$\phi_{j}:[0,1]\rightarrow{\mathbb R}$ $[1\leq j\leq 5]$
with the following property. Given any continuous
function
$f:[0,1]^{2}\rightarrow{\mathbb R}$ we can find a
function $g:{\mathbb R}\rightarrow{\mathbb R}$ such that
\[f(x,y)=\sum_{j=1}^{5}g(\phi_{j}(x)+\lambda\phi_{j}(y)).\]
\end{theorem}
The main point of Theorem~\ref{Hilbert; Theorem}
may be expressed as follows.
\begin{theorem}\label{Gilbert; Theorem}
Any continuous function of two variables
can be written in terms of 
continuous functions of one variable
and addition.
\end{theorem}
That is, there are no true functions of two variables!
(We shall explain why this statement is slightly
less shocking than it seems at the end of this section.)

For the moment we merely observe that the result is due
in successively more exact forms to Kolmogorov, Arnol'd
and a succession of mathematicians ending with Kahane 
whose proof we use here. It is, of course, much easier to
prove a specific result like Theorem~\ref{Hilbert; Theorem}
than one like Theorem~\ref{Gilbert; Theorem}.

Our first step is to observe that Theorem~\ref{Hilbert; Theorem}
follows from the apparently simpler result that follows.
\begin{lemma}\label{Hilbert A; Lemma} Let $\lambda$ be irrational.
We can find increasing continuous functions
$\phi_{j}:[0,1]\rightarrow{\mathbb R}$ $[1\leq j\leq 5]$
with the following property. Given any continuous
function
$F:[0,1]^{2}\rightarrow{\mathbb R}$ we can find a
function $G:{\mathbb R}\rightarrow{\mathbb R}$ such that
$\|G\|_{\infty}\leq\|F\|_{\infty}$ and 
\[\sup_{(x,y)\in[0,1]^{2}}
\left|F(x,y)-\sum_{j=1}^{5}G(\phi_{j}(x)+\lambda\phi_{j}(y))
\right|\leq \frac{999}{1000}\|F\|_{\infty}.\]
\end{lemma}

Next we make the following observation.
\begin{lemma}\label{Hilbert B; Lemma} We can find
a sequence of functions $f_{n}:[0,1]^{2}\rightarrow{\mathbb R}$
which are uniformly dense in $C([0,1])^{2}$.
\end{lemma}
This enables us to obtain Lemma~\ref{Hilbert A; Lemma}
from a much more specific result.
\begin{lemma}\label{Hilbert C; Lemma} Let $\lambda$ be irrational
and let the $f_{n}$ be as in Lemma~\ref{Hilbert B; Lemma}.
We can find increasing continuous functions
$\phi_{j}:[0,1]\rightarrow{\mathbb R}$ $[1\leq j\leq 5]$
with the following property. We can find 
functions $g_{n}:{\mathbb R}\rightarrow{\mathbb R}$ such that
$\|g_{n}\|_{\infty}\leq\|f_{n}\|_{\infty}$ and 
\[\sup_{(x,y)\in[0,1]^{2}}
\left|f_{n}(x,y)-\sum_{j=1}^{5}g_{n}(\phi_{j}(x)+\lambda\phi_{j}(y))
\right|\leq \frac{998}{1000}\|f_{n}\|_{\infty}.\]
\end{lemma}

Now that we have reduced the matter to 
satisfying a countable set of conditions,
we can use a Baire category argument.
We need to use the correct metric space.
\begin{lemma}\label{Lemma Kahane space}
The space $Y$ of continuous functions
${\boldsymbol \phi}:[0,1]\rightarrow{\mathbb R}^{5}$
with norm
\[\|{\boldsymbol \phi}\|_{\infty}=
\sup_{t\in[0,1]}\|{\boldsymbol \phi}(t)\|\]
is complete. The subset $X$ of $Y$ consisting of
those ${\boldsymbol \phi}$ such that each $\phi_{j}$
is increasing is a closed
subset of $Y$. Thus if $d$ is the metric on $X$
obtained by restricting the metric on $Y$ derived from
$\|\ \|_{\infty}$  we have $(X,d)$ complete.
\end{lemma}
\begin{exercise} Prove Lemma~\ref{Lemma Kahane space}
\end{exercise}

\begin{lemma}\label{Hilbert D; Lemma}  Let
$f:[0,1]^{2}\rightarrow{\mathbb R}$ be continuous
and let $\lambda$ be irrational. Consider the
set $E$ of ${\boldsymbol \phi}\in X$ such that
there exists a 
$g:{\mathbb R}\rightarrow{\mathbb R}$ such that
$\|g\|_{\infty}\leq\|f\|_{\infty}$
\[\sup_{(x,y)\in[0,1]^{2}}
\left|f(x,y)-\sum_{j=1}^{5}g(\phi_{j}(x)+\lambda\phi_{j}(y))
\right|<\frac{998}{1000}\|f\|_{\infty}.\]
Then $X\setminus E$ is a closed set with dense complement
in $(X,d)$.
\end{lemma}
(Notice that it is important to take `$<$' rather than
`$\leq$' in the displayed formula of Lemma~\ref{Hilbert D; Lemma}.)
Lemma~\ref{Hilbert D; Lemma} is the heart of the proof
and once it is proved we can easily retrace our steps
and obtain Theorem~\ref{Hilbert; Theorem}.

By using appropriate notions of information
Vitushkin was able to show that we can not replace
continuous by continuously differentiable
in Theorem~\ref{Gilbert; Theorem}. Thus
Theorem~\ref{Hilbert; Theorem} is an `exotic'
rather than a `central' result.
\section{The principle of uniform boundedness}
We start with a result which is sometimes useful
by itself but which, for us, is merely a stepping 
stone to Theorem~\ref{Theorem, Banach Steinhaus}.
\begin{lemma}[Principle of uniform boundedness]
Suppose that $(X,d)$ is a complete metric space
and we have a collection ${\mathcal F}$ of continuous
functions $f:X\rightarrow{\mathbb R}$ which
are pointwise bounded, that is, given any $x\in X$
we can find a $K(x)>0$ such that
\[|f(x)|\leq K(x)\ \text{for all $f\in{\mathcal F}$}.\]
Then we can find a ball $B(x_{0},\delta)$ and a $K$
such that
\[|f(x)|\leq K\ \text{for all $f\in{\mathcal F}$
and all $x\in B(x_{0},\delta)$ }.\]
\end{lemma}
\begin{exercise} (i) Suppose that $(X,d)$ is a complete metric space
and we have a sequence of continuous
functions $f_{n}:X\rightarrow{\mathbb R}$ and
a function $f:X\rightarrow{\mathbb R}$
such that $f_{n}$ converges pointwise
that is
\[f_{n}(x)\rightarrow f(x)\ \text{for all $f\in{\mathcal F}$}.\]
Then we can find a ball $B(x_{0},\delta)$ and a $K$
such that
\[|f_{n}(x)|\leq K\ \text{for all $n$
and all $x\in B(x_{0},\delta)$ }.\]

(ii) (This is elementary but acts as a hint for (iii).)
Suppose $y\in [0,1]$. Show that we can find
a sequence of continuous functions $f_{n}:[0,1]\rightarrow{\mathbb R}$
such that $1\geq f_{n}(x)\geq 0$ for all $x$ and $n$,
$f_{n}$ converges pointwise to $0$ everywhere,
$f_{n}$ converges uniformly on $[0,1]\setminus (y-\delta,y+\delta)$
and fails to converge uniformly on 
$[0,1]\cap (y-\delta,y+\delta)$  for all $\delta>0$.

(iii) State with reasons whether the following statement
is true or false. Under the conditions of (i) we can obtain the 
stronger conclusion that
we can find a ball $B(x_{0},\delta)$ such that
\[f_{n}(x)\rightarrow f(x)\ \text{uniformly on}
\ B(x_{0},\delta).\]
\end{exercise}
\begin{exercise}
Suppose that $(X,d)$ is a complete metric space
and $Y$ is a subset of $X$ which is of first category
in $X$.
Suppose further that we have a collection ${\mathcal F}$ of continuous
functions $f:X\rightarrow{\mathbb R}$ which
are pointwise bounded on $X\setminus Y$,
that is, given any $x\notin Y$,
we can find a $K(x)>0$ such that
\[|f(x)|\leq K(x)\ \text{for all $f\in{\mathcal F}$}.\]
Show that
we can find a ball $B(x_{0},\delta)$ and a $K$
such that
\[|f(x)|\leq K\ \text{for all $f\in{\mathcal F}$
and all $x\in B(x_{0},\delta)$ }.\]
\end{exercise}

We now use the principle of uniform boundedness to prove
the Banach-Steinhaus theorem\footnote{You should be warned that
a lot of people, including the present writer, tend to
confuse the names of these two theorems. My research supervisor
took the simpler course of referring to all the theorems
of functional analysis as `Banach's theorem'.}.
\begin{theorem}{\bf (Banach-Steinhaus theorem)}%
\label{Theorem, Banach Steinhaus}
Let $(U,\|\ \|_{U})$ and $(V,\|\ \|_{V})$ be normed
spaces and suppose $\|\ \|_{U}$ is complete.
If we have a collection ${\mathcal F}$ of continuous
linear maps from $U$ to $V$ which are pointwise bounded
then we can find a $K$ such that $\|T\|\leq K$
for all $T\in{\mathcal F}$.
\end{theorem}

Here is a typical use of the Banach-Steinhaus theorem.
\begin{theorem} There exists a continuous
$2\pi$ periodic function $f:{\mathbb R}\rightarrow{\mathbb R}$
whose Fourier series fails to converge at a given point.
\end{theorem}
The next exercise contains results that most
of you will have already met.
\begin{exercise} (i) Show that the set $l^{\infty}$ of
bounded sequences over ${\mathbb F}$ (with
${\mathbb F}={\mathbb R}$ or ${\mathbb F}={\mathbb C}$) 
\[{\mathbf a}=(a_{1},\ a_{2},\ \dots)\]
can be made into a vector space in a natural manner.
Show that $\|{\mathbf a}\|_{\infty}=\sup_{j\geq 1}|a_{j}|$
defines a complete norm on $l^{\infty}$.

(ii) Show that $s$, the set of convergent sequences
and $s_{0}$ the set of sequences convergent to $0$
are both closed subspaces of $(l^{\infty},\|\ \|_{\infty})$.

(iii) Show that the set $l^{1}$ of
sequences
\[{\mathbf a}=(a_{1},\ a_{2},\ \dots)
\ \text{such that $\sum_{j=1}^{\infty}|a_{j}|$ converges}\]
can be made into vector space in a natural manner.
Show that $\|{\mathbf a}\|_{1}=\sum_{j=1}^{\infty}|a_{j}|$
defines a complete norm on $l^{1}$.

(iv) Show that, if ${\mathbf a}\in l^{1}$, then
\[T_{\mathbf a}({\mathbf b})=\sum_{j=1}^{\infty}a_{j}b_{j}\]
defines a continuous linear map from $l^{\infty}$ to ${\mathbb F}$
and that $\|T_{\mathbf a}\|=\|{\mathbf a}\|_{1}$.
\end{exercise}
Here is another use of the Banach-Steinhaus theorem.
\begin{lemma}\label{Lemma, summation}
Let $a_{ij}\in{\mathbb R}$ $[i,j\geq 1]$.
We say that the $a_{ij}$ constitute a summation
method if whenever $c_{j}\rightarrow c$ we have
$\sum_{j=1}^{\infty}a_{ij}c_{j}$ convergent for each $i$
and
\[\sum_{j=1}^{\infty}a_{ij}c_{j}\rightarrow c\]
as $i\rightarrow\infty$.

The following conditions are necessary and sufficient
for the $a_{ij}$ to constitute a summation method:-

(i) There exists a $K$ such that
\[\sum_{j=1}^{\infty}|a_{ij}|\leq K\ \text{for all $i$}.\]

(ii) ${\displaystyle
\sum_{j=1}^{\infty}a_{ij}\rightarrow 1\ \text{as $i\rightarrow\infty$}}$.

(iii) $a_{ij}\rightarrow 0$ as $i\rightarrow\infty$ for each $j$.
\end{lemma}
\begin{exercise} Ces\`{a}ro's summation method takes
a sequence $c_{0},\ c_{1},\ c_{2},\dots $ and replaces it
with a new sequence whose $n$th term
\[b_{n}=\frac{c_{1}+c_{2}+\dots+c_{n}}{n}\]
is the average of the first $n$ terms of the old sequence.

(i) By rewriting the statement above along the lines
of Lemma~\ref{Lemma, summation} show that if the old
sequence converges to $c$ so does the new one.

(ii) Examine what happens when $c_{j}=(-1)^{j}$.
Examine what happens if $c_{j}=(-1)^{k}$ when
$2^{k}\leq j<2^{k+1}$.

(iii) Show that, in the notation of Lemma~\ref{Lemma, summation},
taking $a_{n,2n}=1$, $a_{n,m}=0$, otherwise, gives a summation
method. Show that
taking $a_{n,2n+1}=1$, $a_{n,m}=0$, otherwise, 
also gives a summation
method. Show that the two methods disagree
when presented with the sequence $1$, $-1$, $1$, $-1$, \dots. 
\end{exercise}
Another important consequence of the Baire
category theorem is the open mapping theorem.
(Recall that a complete normed space is called
a Banach space.)
\begin{theorem}[Open mapping theorem]
Let $E$ and $F$ be Banach spaces and
$T:E\rightarrow F$ be a continuous linear
surjection. Then $T$ is an open map
(that is to say, if $U$ is open in $E$
we have $TU$ open in $F$.)
\end{theorem}
This has an immediate corollary.
\begin{theorem}[Inverse mapping theorem]
Let $E$ and $F$ be Banach spaces and let
$T:E\rightarrow F$ be a continuous linear
bijection. Then $T^{-1}$ is continuous.
\end{theorem}
The next exercise is simple, and if you can not
do it this reveals a gap in your knowledge
(which can be remedied by asking the lecturer)
rather than in intelligence.
\begin{exercise} Let $(X,d)$ and $(Y,\rho)$
be metric spaces with associated topologies
$\tau$ and $\sigma$. Then the product topology
induced on $X\times Y$ by $\tau$ and $\sigma$
is the same as the topology given by the metric
\[\triangle((x_{1},y_{1}),(x_{2},y_{2}))
=d(x_{1},x_{2})+\rho(y_{1},y_{2}).\]
\end{exercise} 
The inverse mapping theorem has the
following useful consequence.
\begin{theorem}[Closed graph theorem]
Let $E$ and $F$ be Banach spaces and let
$T:E\rightarrow F$ be linear.
Then $T$ is continuous if and only
the graph
\[\{(x,Tx)\,:\,x\in E\}\]
is closed in $E\times F$ with the product topology.
\end{theorem}
\section{Zorn's lemma and Tychonov's theorem}
Let $A$ be a non-empty set and, for each $\alpha\in A$,
let $X_{\alpha}$
be a non-empty set. Is $\prod_{\alpha\in A}X_{\alpha}$
non-empty (or, equivalently, does there exist a function
$f:A\rightarrow\bigcup_{\alpha\in A}X_{\alpha}$ with
$f(\alpha)\in X_{\alpha})$? It is known that the standard
axioms of set theory do not suffice to answer this question
in general. (In particular cases they do suffice. If
$X_{\alpha}=A$ for all $\alpha\in A$ then $f(\alpha)=\alpha$
will do.) Specifically, if there exists any
model for standard
set theory, then there exist models for set theory
obeying the standard axioms in which the answer to
our question is always yes (such systems are said to
obey the axiom of choice) and there exist models in
which the answer is sometimes no.

Most mathematicians are happy to add the axiom of choice
to the standard axioms and this is what we shall do.
Note that if we prove something using 
the standard axioms and the axiom of choice
then we will be unable to find a counter-example
using only the standard axioms. Note also that,
when dealing with specific systems we may be able
to prove the result for that system without using
the axiom of choice.

The axiom of choice is not very easy to use in the form
that we have stated it and it is usually more convenient
to use an equivalent formulation called Zorn's lemma.
\begin{definition} Suppose $X$ is a non-empty set.
We say that
$\succeq$ is a partial order on $X$, that is to say,
that $\succeq$ is a relation on $X$ with

(i) $x\succeq y$, $y\succeq z$ implies $x\succeq z$,

(ii) $x\succeq y$ and $y\succeq x$ implies $x=y$,

(iii) $x\succeq x$

\noindent for all $x$, $y$, $z$.

We say that a subset $C$ of $X$ is a chain
if, for every $x,\ y\in C$ at least one of
the statements $x\succeq y$, $y\succeq x$ is true.

If $Y$ is a non-empty subset of $X$ we say that $z\in X$
is an upper bound for $Y$ if $z\succeq y$ for all $y\in Y$.

We say that $m$ is a maximal element for $(X,\succeq)$
if $x\succeq m$ implies $x=m$.
\end{definition}
You must be able to do the following exercise.
\begin{exercise} (i) Give an example of a partially
ordered set which is not a chain.

(ii) Give an example of a partially ordered set
and a chain $C$ such that (a) the chain has an upper bound
lying in $C$, (b) the chain has an upper bound but no
upper bound within $C$, (c) the chain has no upper bound.

(iii) If a chain $C$ has an upper bound lying in $C$,
show that it is unique. Give an example to show that,
even in this case $C$ may have infinitely many
upper bounds (not lying in $C$).

(iv) Give examples of partially ordered sets
which have (a) no maximal elements, (b) exactly one
maximal element, (b) infinitely many maximal elements.

(v) how should a minimal element be defined?
Give examples of partially ordered sets
which have (a) no maximal or minimal elements, (b) exactly one
maximal element and no minimal element, 
(c) infinitely many maximal elements and infinitely many minimal
elements.
\end{exercise}
\begin{axiom}[Zorn's lemma] Let $(X,\succeq)$ be
a partially ordered set. If every chain in $X$ has
an upper bound then $X$ contains a maximal element.
\end{axiom}

Zorn's lemma is associated with a proof routine
which we illustrate in Lemmas~\ref{Zorn to choice}
and~\ref{Hamel}
\begin{lemma}\label{Zorn to choice} Zorn's lemma implies the axiom of choice.
\end{lemma}
The converse result is less important to us
but we prove it for completeness.
\begin{lemma}\label{choice to Zorn} The axiom of choice
implies Zorn's lemma.
\end{lemma}
\begin{proof} (Since the proof we use is non-standard,
I give it in detail.)
Let $X$ be a non-empty set with a partial order $\succeq$
having no maximal elements. We show that the assumption
that every chain has a upper bound leads to a contradiction.

We write $x\succ y$ if $x\succeq y$ and $x\neq y$.
If $C$ is a chain we write 
\[C_{x}=\{c\in C\,:\,x\succ c\}.\]

Observe that, if $C$ is a chain in $X$, we can
find an $x\in X$ such that $x\succ c$ for all $c\in C$.
(By assumption, $C$ has an upper bound, $x'$, say.
Since $X$ has no maximal elements, we can find an 
$x\in X$ such that $x\succ x'$.) We shall take
$\emptyset$ to be a well ordered chain. 

We shall look at well ordered chains, that is to say,
chains for which every non-empty subset has a minimum.
(Formally, if $S\subseteq C$ is non-empty we can
find an $s_{0}\in C$ such that $s\succeq s_{0}$
for all $s\in S$. We write $\min C=s_{0}$.)
By the previous paragraph 
\[A_{C}=\{x\,:\,x\succ c\ \text{for all $c\in C$}\}\neq\emptyset.\]
Thus, if we write ${\mathcal W}$ for the set of all well
ordered chains, the axiom of choice tells us that
there is a function $\kappa:{\mathcal W}\rightarrow X$
such that $\kappa(C)\succ c$ for all $c\in C$.

We now consider `special chains' defined to be
well ordered chains $C$ such that
\[\kappa(C_{x})=x\ \text{for all $x\in C$}.\]
(Note that `well ordering' is an important general
idea, but `special chains' are an ad hoc notion
for this particular proof. Note also that if $C$
is a special chain and $x\in C$ then $C_{x}$ is a special
chain.)

The key point is that, if $K$ and $L$ are special
chains, then either $K=L$ or $K=L_{x}$ for some $x\in L$
or $L=K_{x}$ for some $x\in K$.
\newline\emph{Subproof} If $K=L$, we are done. If not, at
least one of $K\setminus L$ and $L\setminus K$ is non-empty.
Suppose, without loss of generality, that $K\setminus L\neq \emptyset$.
Since $K$ is well ordered, $x=\min K\setminus L$ exists.
We observe that $K_{x}\subseteq L$. If $K_{x}=L$, we are done.

We show that the remaining possibility $K_{x}\neq L$ leads
to contradiction. In this case, $L\setminus K_{x}\neq \emptyset$
so $y=\min L\setminus K_{x}$ exists. 

If $L_{y}=K_{x}$
then
\[y=\kappa(L_{y})=\kappa(K_{x})=x\]
so $x=y\in L\cap K$ contradicting the statement that $x\in K\setminus L$.

If $L_{y}\neq K_{x}$ let $z$
be the least member of $K_{x}\setminus L_{y}$.   
Observe that, since $K_{x}\subseteq L$
and so
\[w\in L_{y},\ z'\in K_{x},\ w\succ z'\Rightarrow
z'\in L,\ y\succ w\succ z'\Rightarrow z'\in L_{y}\]
whence
\[z'\notin L _{y},\ z'\in K_{x},\ w\succ z'\Rightarrow
w\notin L_{y}.\]
Thus $K_{z}=L_{y}$ and
\[y=\kappa(L_{y})=\kappa(K_{z})=z\]
so $y=z\in K\cap L$ contradicting the definition of $y$.
\newline\emph{End subproof}

We now take $S$ to be the union of all special chains.
Using the key observation, it is routine to see that:

(i) $S$ is a chain. (If $a,\,b\in S$, then $a\in L$ and $b\in K$
for some special chains. By our key observation, either $L\supseteq K$
of $K\supseteq L$. Without loss of generality, $K\supseteq L$ so $a,\,b \in K$
and $a\succeq b$ or $b\succeq a$.)

(ii) If $a\in S$, then $S_{a}$ is a special chain. (We must have
$a\in K$ for some special chain $K$. Since $K\subseteq S$, we have 
$K_{a}\subseteq S_{a}$. On the other hand, if $b\in S_{a}$
then $b\in L$ for some special chain $L$  and each of the three
possible relationships given in our key observation imply $b\in K_{a}$.
Thus $S_{a}\subseteq K_{a}$, so $S_{a}=K_{a}$ and $S_{a}$ is a special chain.)

(iii) $S$ is well ordered. (If $E$ is a non empty subset of $S$,
pick an $x\in E$. If $S_{x}\cap E=\emptyset$, then $x$ is a minimum for $E$.
If not, then $S_{x}\cap E$ is a non-empty subset of the special, so
well ordered, chain  $S_{x}$, so $\min S_{x}\cap E$ 
exists and is a minimum for $E$.)

(iv) $S$ is a special chain. (If $x\in S$, we can find a special chain $K$
such that $x\in K$. Let $y=\kappa(K)$. Then $L=K\cup\{y\}$ is a special
chain. As in (ii), $S_{y}=L_{y}$, so $S_{x}=L_{x}$ and
$\kappa(S_{x})=\kappa(L_{x})=x$.)

We can now swiftly obtain a contradiction. Since $S$ is well ordered
$\kappa(S)$ exists and does not lie in $S$. But $S$ is special,
so $S\cup \kappa(S)$ is, so $S\cup\kappa(S)\subseteq S$,
so $\kappa(S)$ lies in $S$. The required result follows
by reductio ad absurdum\footnote{To the best of my knowledge, 
this particular proof is due to
Jonathon Letwin (\emph{American Mathematical Monthly},
Volume 98, {\bf 1991}, pp. 353--4). If you know about
transfinite induction, there are more direct proofs.}.
\end{proof}
\begin{lemma}[Hamel basis theorem]\label{Hamel} 
(i) Every vector space
has a basis.

(ii) If $U$ is an infinite dimensional normed space
over ${\mathbb F}$ then we can find a
linear map $T:U\rightarrow{\mathbb F}$.

(iii) If $U$ is an infinite dimensional normed space
over ${\mathbb F}$ then we can find a
discontinuous linear map $T:U\rightarrow{\mathbb F}$.
\end{lemma}
[Note that we do not claim that $T$ in (ii) is continuous.]
\begin{exercise} (i) Show that, if 
$f:{\mathbb R}\rightarrow{\mathbb R}$ is continuous
and satisfies the equation
\[f(x+y)=f(x)+f(y)\]
for all $x,\ y\in{\mathbb R}$, then there exists a $c$
such that $f(x)=cx$ for all $x\in{\mathbb R}$.

(ii) Show that there exists a discontinuous function
$f:{\mathbb R}\rightarrow{\mathbb R}$
satisfying the equation
\[f(x+y)=f(x)+f(y)\]
for all $x,\ y\in{\mathbb R}$.

\noindent[Hint. Consider ${\mathbb R}$ as a vector space
over ${\mathbb Q}$.]
\end{exercise}

The rest of this section is devoted to a proof of Tychonov's
theorem. We recall a definition.
\begin{added}[Definition]\label{D;weak} 
Let $(X_{\alpha},\tau_{\alpha})$
be a topological space for each $\alpha\in A$.
The weak (or Tychonov) topology $\tau$ on 
$\prod_{\alpha\in A}X_{\alpha}$ is the collection of sets
$U$ such that if ${\mathbf u}\in U$ we can find
$\alpha_{1},\,\alpha_{2},\,\dots,\,\alpha_{n}\in A$
and $O_{\alpha_{j}}\in\tau_{\alpha_{j}}$ such that
${\mathbf x}\in U$ whenever $x_{\alpha_{j}}\in O_{\alpha_{j}}$
for $1\leq j\leq n$.  
\end{added} 
\begin{added}[Exercise] We retain the notation of
Definition Added~\ref{D;weak} and write 
$\pi_{\alpha}{\mathbf x}=x_{\alpha}$.

(i) Show that $\tau$ is indeed a topology
and that, with this topology, the maps 
$\pi_{\alpha}:X\rightarrow X_{\alpha}$
are continuous. 

(ii) Show that $\tau$ is the weakest topology
for which the $\pi_{\alpha}$ are continuous.
[Thus, if $\sigma$ is a topology for which
the $\pi_{\alpha}$ are continuous, we have 
$\sigma\supseteq\tau$.]

(iii) Show that if all the $\tau_{\alpha}$ are Hausdorff
so is $\tau$. 

(iv) Suppose that $A=[0,1]$ and $X_{t}={\mathbb R}$
and $\tau_{t}$ is the usual Euclidean topology
on $X_{t}$. Explain how an $f\in\prod_{t\in[0,1]}X_{t}$
can be identified in a natural manner with a function
$f:[0,1]\rightarrow{\mathbb R}$. With this identification
show that the sequence of functions $f_{n}\rightarrow f$
pointwise if and only if, given any $U\in\tau$ with $f\in U$
we can find an $N$ such that $f_{n}\in U$ for all $n\geq N$.
\end{added}
\begin{theorem}[Tychonov] The product of compact
spaces is itself compact.
\end{theorem}
We follow the presentation in~\cite{Bollobas}.
(The method of proof is due to Bourbaki.)

The following result should be familiar to almost
all of my readers.
\begin{lemma}[Finite intersection property] 
(i) A topological
space is compact if and only if whenever a non-empty collection
of closed sets ${\mathcal F}$ has the property
that $\bigcap_{j=1}^{n}F_{j}\neq\emptyset$,
for any $F_{1}$, $F_{2}$, \dots, $F_{n}\in {\mathcal F}$
it follows that $\bigcap_{F\in{\mathcal F}}F\neq\emptyset$.

(ii) A topological
space is compact if and only if whenever a non-empty collection
of sets ${\mathcal A}$ has the property
that $\bigcap_{j=1}^{n}A_{j}\neq\emptyset$
for any $A_{1}$, $A_{2}$, \dots, $A_{n}\in {\mathcal A}$
it follows that $\bigcap_{A\in{\mathcal A}}\bar{A}\neq\emptyset$.
\end{lemma}
\begin{definition} A system ${\mathcal F}$
of subsets of a given set $S$ is said to be of finite character
if 

(i) whenever every finite subset of a set $A\subseteq S$
belongs to ${\mathcal F}$ it follows that $A\in{\mathcal F}$
and

(ii) whenever $A\in{\mathcal F}$ every finite subset
of $A$ belongs to ${\mathcal F}$.
\end{definition}
\begin{lemma} [Tukey's lemma] If a system ${\mathcal F}$
of subsets of a given set $S$ has finite character
and $F\in{\mathcal F}$ then ${\mathcal F}$ has a maximal 
(with respect to inclusion) element containing $F$.
\end{lemma}
We now prove Tychonov's theorem.

\begin{added}[Exercise] If $(X,\tau)$ is a Hausdorff
space and ${\mathcal G}$ is a maximal collection 
of sets with the finite intersection property
explain why $\bigcap_{G\in{\mathcal G}}\bar{G}$
consists of one point.

If you are interested, examine how the second 
(but not the first) appeal to the axiom of choice
may be avoided in our proof of Tychonov's theorem
if all our spaces are Hausdorff.
\end{added} 

The reason why Tychonov's theorem demands the axiom 
of choice is made clear by the final result
of this section.
\begin{lemma} Tychonov's theorem implies the axiom 
of choice.
\end{lemma}
\section{The Hahn-Banach theorem} A good example
of the use of Zorn's lemma occurs when we ask
if given a Banach space $(U,\|\ \|)$ (over ${\mathbb C}$,
say) there exist
any non-trivial continuous linear maps $T:U\rightarrow{\mathbb C}$.
For any space that we can think of, the answer is
obviously yes, but to show that the result is
always yes we need Zorn's lemma\footnote{In fact the statement
is marginally weaker than Zorn's lemma but you need to
be logician either to know or care about this.}.
Our proof uses the theorem of Hahn-Banach.

One form of this theorem is the following.
\begin{theorem}{\bf (Hahn--Banach)}\label{T;Hahn--Banach} Let $U$
be a real vector space. Suppose
$p:U\rightarrow{\mathbb R}$ is such that
\[p(u+v)\leq p(u)+p(v)\ \text{and}\ p(au)=ap(u)\]
for all $u,\ v\in U$ and all real and positive $a$.

If $E$ is a subspace of $U$ and there exists
a linear map $T:E\rightarrow{\mathbb R}$
with $Tx\leq p(x)$ for all $x\in E$
then there exists 
a linear map $\tilde{T}:U\rightarrow{\mathbb R}$
with $Tx\leq p(x)$ for all $x\in U$
and $\tilde{T}(x)=Tx$ for all $x\in E$.
\end{theorem}
\noindent[Note that we do not assume that
the vector space $U$ is normed but we do assume
that the vector space is real.]
\begin{added}[Exercise] We say that a function
$f:{\mathbb R}\rightarrow{\mathbb R}$ is convex if
\[f\big(\lambda x+(1-\lambda)y\big)\leq \lambda f(x)+(1-\lambda)f(y)\]
for all $x,\,y\in{\mathbb R}$. Using the ideas of the
proof of the Hahn--Banach theorem but not the result itself
show that a convex function $f$ is continuous
and that given any $a\in{\mathbb R}$ we can find
a $c$ such that
\[f(x)\geq f(a)+c(x-a)\]
for all $x\in{\mathbb R}$. Give an example to show that
$f$ need not be differentiable.
\end{added}
\begin{added}[Exercise] Let $X$ be a real vector space and
$p,\,q:X\rightarrow {\mathbb R}$ be functions such that
$p(\lambda x)=\lambda p(x)$, $q(\lambda x)=\lambda q(x)$
for all $\lambda\in{\mathbb R}$ with $\lambda\geq 0$
and all $x\in X$, whilst
\[p(x+y)\leq p(x)+p(y),\ q(x)+q(y)\leq q(x+y)\]
for all $x,\,y\in X$.

(i) Suppose that $Y$ is a subspace of $X$ and 
$S:Y\rightarrow{\mathbb R}$ a linear function such that
\[S(y)\leq p(x+y)-q(x)\]
for all $x\in X$, $y\in Y$. Show that
\[S(y')-p(x'+y'-z)+q(x')\leq -S(y)+p(x+y+z)-q(x)\]
for all $x,\,x',\,z\in X$ and $y,\,y' \in Y$.

(ii) Suppose that $Y_{0}$ is a subspace of $X$ and
$T_{0}:Y\rightarrow{\mathbb R}$ a linear function such that
\[T_{0}(y)\leq p(x+y)-q(x)\]
for all $x\in X$, $y\in Y_{0}$. Show that there exists a linear
function $T_{0}:X\rightarrow{\mathbb R}$ such that
\[T(y)\leq p(x+y)-q(x)\]
for all $x,\,y\in X$ and $Tu=Tu_{0}$  for all $u\in Y_{0}$.
Show that 
\[q(x)\leq T(x)\leq p(x)\]
for all $x\in X$.

(iii) Suppose $p(x)\geq q(x)$ for all $x\in X$. 
Show that there exists 
a linear function (possibly the zero function) 
$U:X\rightarrow{\mathbb R}$
such that
\[q(x)\leq U(x)\leq p(x)\]
for all $x\in X$.

(iv) Let $X={\mathbb R}^{2}$, ${\mathbf n}$ be a unit vector
and $p({\mathbf x})=|{\mathbf n}.{\mathbf x}|$
(the absolute value of the usual inner product)
and $q({\mathbf x})=-|{\mathbf n}.{\mathbf x}|$.
Show that $p$ and $q$ obey the conditions of the introductory
paragraph and part~(iii). What can you say about $U$? 
\end{added}
We have the following important corollary
to Theorem~\ref{T;Hahn--Banach}.
\begin{theorem} Let $(U,\|\ \|)$
be a real normed vector space.
If $E$ is a subspace of $U$ and there exists
a continuous linear map $T:E\rightarrow{\mathbb R}$,
then there exists 
a continuous linear map $\tilde{T}:U\rightarrow{\mathbb R}$
with $\|\tilde{T}\|=\|T\|$.
\end{theorem}
The next result is famous as `the result that
Banach did not prove'.
\begin{theorem} Let $(U,\|\ \|)$
be a complex normed vector space.
If $E$ is a subspace of $U$ and there exists
a continuous linear map $T:E\rightarrow{\mathbb C}$
then there exists 
a continuous linear map $\tilde{T}:U\rightarrow{\mathbb C}$
with $\|\tilde{T}\|=\|T\|$.
\end{theorem}
We can now answer the question posed in the first 
sentence of this section.
\begin{lemma} If $(U,\|\ \|)$ is
normed space over the field ${\mathbb F}$
of real or complex numbers
and $a\in U$ with $a\neq 0$, then
we can find a continuous linear map
$T:U\rightarrow{\mathbb F}$ with $Ta\neq 0$.
\end{lemma} 
The importance of this lemma becomes more obvious if
we state it in reverse. If $Ta=0$ for all continuous linear
maps then $a=0$.


Here are a couple of results proved by Banach using his theorem.
\begin{theorem}[Generalised limits]\label{Theorem, Banach limits}
Consider the vector space $l^{\infty}$ of bounded real
sequences. There exists a linear map $L:l^{\infty}\rightarrow {\mathbb R}$
such that

(i) If $x_{n}\geq 0$ for all $n$ then $L{\mathbf x}\geq 0$.

(ii) $L((x_{1},x_{2},x_{3},\dots))=L((x_{0},x_{1},x_{2},\dots))$.
                                                                
(iii) $L((1,1,1,\dots))=1$.
\end{theorem}
The theorem is illustrated by the following lemma.
\begin{lemma} Let $L$ be as in Theorem~\ref{Theorem, Banach limits}.
Then
\[\limsup_{n\rightarrow\infty}x_{n}\geq L({\mathbf x})
\geq \liminf_{n\rightarrow\infty}x_{n}.\]
In particular, if $x_{n}\rightarrow x$ then $L({\mathbf x})=x$.
\end{lemma}
\begin{exercise} (i) Show that, even though the sequence
$x_{n}=(-1)^{n}$ has no limit, $L({\mathbf x})$ is uniquely
defined.

(ii) Find, with reasons, a sequence ${\mathbf x}\in l^{\infty}$ 
for which $L({\mathbf x})$ is not uniquely
defined.
\end{exercise}

Banach used the same idea to prove the following
odd result.
\begin{lemma}\label{Banach integral} 
Let ${\mathbb T}={\mathbb R}/{\mathbb Z}$
be the unit circle and let $B({\mathbb T})$ be the 
vector space
of real valued bounded functions.
Then we can find a linear map 
$I:B({\mathbb T})\rightarrow{\mathbb R}$ obeying the following conditions.

(i) $I(1)=1$.

(ii) $If\geq 0$ if $f$ is positive.

(iii) If $f\in B({\mathbb T})$, $a\in{\mathbb T}$ and
we write $f_{a}(x)=f(x-a)$ then $If_{a}=If$.
\end{lemma}
\begin{exercise} Show that if $I$ is as in
Lemma~\ref{Banach integral} and $f$ is Riemann
integrable then 
\[If=\int_{\mathbb T}f(t)\,dt.\]
\end{exercise}

However, Lemma~\ref{Banach integral} is put in context
by the following.
\begin{lemma}\label{Two generators}
Let $G$ be the group freely generated by
two generators and $B(G)$ be the 
vector space
of real valued bounded functions on $G$.
If $f\in B(G)$ let us write $f_{c}(x)=f(xc^{-1})$
for all $x,\ c\in G$.  

There exists a function $f\in B(G)$ 
and $c_{1},\ c_{2},\ c_{3}$ such that
$f(x)\geq 0$ for all $x\in G$ and
\[f(x)+f_{c_{1}}(x)-f_{c_{2}}(x)-f_{c_{3}}(x)\leq -1\]
for all $x\in G$. 
\end{lemma}
\begin{exercise} If $G$ is as in Lemma~\ref{Two generators}
then there is no linear map $I:B(G)\rightarrow{\mathbb R}$
obeying the following conditions.

(i) $I(1)=1$.

(ii) $If\geq 0$ if $f$ is positive.

(iii)  $If_{c}=If$ for all $c\in G$.
\end{exercise}
It can be shown that there is a finitely additive,
congruence respecting integral for ${\mathbb R}$ and
${\mathbb R}^{2}$ but not ${\mathbb R}^{n}$ for $n\geq 3$.
\section{Banach algebras} Many of the objects studied
in analysis turn out to be Banach algebras.
\begin{definition} An algebra $(B,+,.,\times)$
is a vector space $(B,+,.,{\mathbb C})$ equipped
with a multiplication $\times$ such that

(i) $x\times( y\times z)=(x\times y)\times z$,

(ii) $(x+y)\times z=x\times z+y\times z$
and $z\times(x+y)=z\times x+z\times y$,

(iii) $(\lambda x)\times y=x\times(\lambda y)=\lambda(x\times y)$
for all $x,\ y,\ z\in B$.

\noindent[We shall write $x\times y=xy$.]
\end{definition}
Note that there is no assumption that multiplication
is commutative. In principle, we could talk about
real Banach algebras (in which ${\mathbb C}$ is replaced by
${\mathbb R}$) but, though some elementary results carry
over, our treatment will only cover complex Banach algebras.
\begin{definition} A Banach algebra 
$(B,+,.,\times,\|\ \|)$ is an algebra
$(B,+,.,\times,{\mathbb C})$ such that
$(B,+,.,{\mathbb C},\|\ \|)$ is a 
Banach space and such that the map $(x,y)\mapsto xy$
is continuous.
\end{definition}

Note that the two definitions above are not to
be memorised; so far as this course is concerned the
following definition is all that is required.
\begin{definition} A Banach algebra $(B,\|\ \|)$
is a Banach space equipped with a continuous multiplication
which makes it an algebra.
\end{definition}
As usual there is a little amount of playing about
with the definition.
\begin{lemma} The following statements about
$(B,\|\ \|)$ a Banach space equipped with a multiplication
are equivalent.

(i) Multiplication is left and right continuous
(that is, the map $x\mapsto xy$ is continuous
for all $y$ and the map $y\mapsto xy$ is continuous
for all $x$).

(ii) There exists a $K$ such that $\|xy\|\leq K\|x\|\|y\|$
for all $x$ and $y$.

(iii) $(B,\|\ \|)$ is a Banach algebra.
\end{lemma}
\begin{lemma}\label{Lemma, renorm}
If $(B,\|\ \|)$ is a Banach algebra
we can find a norm $\|\ \|_{B}$ on $B$ which is
equivalent to $\|\ \|$ (that is, there exists a $C>0$
such that $C^{-1}\|x\|\leq \|x\|_{B}\leq C\|x\|$)
such that
\[\|xy\|_{B}\leq \|x\|_{B}\|y\|_{B}\]
for all $x,\ y\in B$.
\end{lemma}
\emph{Unless specifically indicated otherwise} you
may assume both in the rest of the notes and in the
literature generally that the norm on a Banach algebra
has been chosen to satisfy
\[\|xy\|\leq \|x\|\|y\|\]
for all $x$ and $y$.

\begin{definition} We say that a Banach algebra $B$ has a unit
$e$ if $xe=ex=x$ for all $x\in B$.
\end{definition}
The following remarks form part of the course but
are left as an exercise.
\begin{exercise} (i) If a Banach algebra has a unit 
that unit is unique.

(ii) If $(B,\|\ \|)$ is a Banach algebra with unit $e$
we can find a norm $\|\ \|_{B}$ on $B$ which is
equivalent to $\|\ \|$ such that
\[\|xy\|_{B}\leq \|x\|_{B}\|y\|_{B}\]
for all $x,\ y\in B$ and
\[\|e\|_{B}=1.\]
\end{exercise}
\emph{Unless specifically indicated otherwise} you
may assume both in the rest of the notes and in the
literature generally that the norm on a Banach algebra
with unit $e$
has been chosen to satisfy
$\|e\|= 1$ and $\|xy\|_{B}\leq \|x\|_{B}\|y\|_{B}$
for all $x$ and $y$.
\begin{example} (i) A Banach space $(B,\|\ \|)$ becomes
a commutative Banach algebra if we define $xy=0$ for
all $x,\ y\in B$. If $B$ is non-trivial the resulting algebra
has no unit.

(ii) Consider the Banach space $l^{1}$ of sequences
${\mathbf a}=(a_{0},a_{1},\dots)$. If we define
${\mathbf a}*{\mathbf b}={\mathbf c}$ with
\[c_{r}=\sum_{k+j=r,\,k\geq 0,j\geq 0}a_{j}b_{k}\]
then $*$ is a well defined multiplication
and $l^{1}$ is a Banach algebra with this multiplication.
As a Banach algebra, $l^{1}$ is commutative with a unit.

(iii) Consider the Banach space $l^{1}$ of sequences
${\mathbf a}=(a_{1},a_{2},\dots)$. If we define
${\mathbf a}*{\mathbf b}={\mathbf c}$ with
\[c_{r}=\sum_{k+j=r,\,k\geq 1,j\geq 1}a_{j}b_{k}\]
then $*$ is a well defined multiplication
and $l^{1}$ is a Banach algebra with this multiplication.
As a Banach algebra, $l^{1}$ is commutative but has no unit.

(iv) Consider the Banach space $l^{1}$ of 
two sided sequences
${\mathbf a}=(\dots,a_{-2},a_{-1},a_{0},a_{1},\dots)$. If we define
${\mathbf a}*{\mathbf b}={\mathbf c}$ with
\[c_{r}=\sum_{k+j=r}a_{j}b_{k}\]
then $*$ is a well defined multiplication
and $l^{1}$ is a Banach algebra with this multiplication.
As a Banach algebra, $l^{1}$ is commutative and has a unit.

\end{example}
\begin{exercise}\label{L one} 
If you know measure theory you ought
to work through this exercise. We work in $L^{1}$
the space of Lebesgue integrable functions 
$f:{\mathbb R}\rightarrow{\mathbb C}$.

(i) Use Fubini's theorem to show that, if $f,\ g\in L^{1}$,
then
\[f*g(x)=\int_{-\infty}^{\infty}f(x-t)g(t)\,dx\]
is well defined almost everywhere and that 
$f*g\in L^{1}$ with
\[\|f*g\|_{1}\leq \|f\|_{1}\|g\|_{1}\]

(ii) Use Fubini's theorem to show that, if $f,\ g\in L^{1}$
then
\[\widehat{f*g}(\lambda)=\hat{f}(\lambda)\hat{g}(\lambda)\]
for all $\lambda\in{\mathbb R}$.

(iii) If $e_{a}(t)=e^{-iat}$ compute $\hat{e_{a}f}$
for $f\in L^{1}$. Show that if $e\in L^{1}$ is a unit then
$\hat{e}=1$.

(iv) Show that if $f\in L^{1}$ then 
$\sup_{\lambda\in{\mathbb R}}|\hat{f}(\lambda)|\leq \|f\|_{1}$.
Show that if $f$ is once continuously differentiable with
$f,\ f'\in L^{1}$ and $f(t),\ f'(t)\rightarrow 0$ 
as $|t|\rightarrow \infty$ then $\hat{f}(\lambda)\rightarrow 0$
as $|\lambda|\rightarrow\infty$. Use a density argument
to show that $\hat{g}(\lambda)\rightarrow 0$
as $|\lambda|\rightarrow\infty$ whenever $g\in L^{1}$
(this is the Lebesgue-Riemann lemma).

(v) Use~(iii) and~(iv) to show that $(L^{1},*)$ 
has no unit.
\end{exercise}
\begin{lemma}\label{Lemma, add unit} 
If $B$ is a Banach algebra without unit
we can find $\tilde{B}$ a Banach algebra with a unit $e$
such that

(i) $B$ is a sub Banach algebra of $\tilde{B}$,

(ii) $B$ is closed in $\tilde{B}$,

(iii) $\tilde{B}=\operatorname{span}(B,e)$ in the
algebraic sense.
\end{lemma}
\begin{exercise} (i) Suppose we apply the construction 
of Lemma~\ref{Lemma, add unit} to a Banach algebra $B$
with unit $u$. Is $u$ a unit of the extended algebra
$\tilde{B}$? Does $\tilde{B}$ have a unit? 

(ii) (Needs measure theory.) Can you find a natural
identification for the unit of $\widetilde{L^{1}}$
where $L^{1}$ is the Banach algebra of Exercise~\ref{L one}.
\end{exercise}
Thus any Banach algebra $B$ without a unit
can be studied by `adjoining
a unit and then removing it'. This is our excuse
for only studying Banach algebras with a unit. 

The following result is easy but fundamental.
\begin{lemma}\label{Lemma, nice inverse}
Let $B$ be a Banach algebra with
unit $e$. 

(i) If $\|e-a\|<1$ then $a$ is invertible (that is
has a multiplicative inverse).

(ii) If $E$ is the set of invertible elements
in $B$ then $E$ is open.
\end{lemma}

Lemma~\ref{Lemma, nice inverse}~(i) can be improved
in a useful way.
\begin{theorem}\label{Theorem, spectral radius}
(i) If $B$ is a Banach algebra and $b\in B$ then,
writing $\rho(b)=\inf_{n}\|b^{n}\|^{1/n}$
we have
\[\|b^{n}\|^{1/n}\rightarrow \rho(b)\]
as $n\rightarrow\infty$.

(ii) If $B$ is a Banach algebra with unit $e$
and $\rho(e-a)<1$ then $a$ is invertible.
\end{theorem}
We call $\rho(a)$
the spectral radius of $a$.

\begin{exercise} Consider the space $M_{n}$
of $n\times n$ matrices over ${\mathbb C}$
with the operator norm.

(i) Show that $M_{n}$ is a Banach algebra with
unit. For which values of $n$ is it commutative?

(ii) Give an example of an $A\in M_{2}$ with
$A\neq 0$ but $\rho(A)=0$.

(iii) If $A$ is diagonalisable show that 
\[\rho(A)=\max\{|\lambda|\,:\,\lambda\ \text{an eigenvalue of $A$}\}.\]

(iv) (Harder and not essential.) Show that the formula
of (iii) holds in general.
\end{exercise}
\section{Maximal ideals} We now embark on a line of reasoning
which will eventually lead to a characterisation of
a large class of commutative Banach algebras.

Initially we continue to deal with Banach algebras which
are not necessarily commutative. The generality is
more apparent than real as the next exercise reveals.
\begin{exercise} Let $B$ be a Banach algebra with unit
$e$. Let $A$ be the closed Banach algebra generated by 
$e$ and some $a\in B$. (Formally, $A$ is the smallest closed sub
Banach algebra containing $e$ and $a$.) Then $A$ is
commutative.
\end{exercise}
\begin{definition}\label{Definition, resolvent}
Let $B$ be a Banach space with unit $e$.
If $x\in B$ the resolvent $R(x)$ of $x$ is defined by
\[R(x)=\{\lambda\in{\mathbb C}\,:\,x-\lambda e
\ \text{is invertible}\}.\]
\end{definition}
\begin{lemma}\label{Lemma, not all resolvent}
We use the notation of 
Definition~\ref{Definition, resolvent}.

(i) ${\mathbb C}\setminus R(x)$ is bounded.

(ii) $R(x)$ is open.

(iii) If $\mu\in R(x)$ we can find a $\delta>0$
and $a_{0}$, $a_{1}$, \dots $\in B$ such that
$\sum_{j=0}^{\infty}a_{j}z^{j}$ converges for all
$|z|<\delta$ and
\[(x-\lambda e)^{-1}=\sum_{j=0}^{\infty}a_{j}(\lambda-\mu)^{j}\]
for $\lambda\in{\mathbb C}$ and $|\lambda-\mu|<\delta$.

(iv) $R(x)\neq {\mathbb C}$.
\end{lemma}
Lemma~\ref{Lemma, not all resolvent} gives us our first
substantial result on the nature of commutative
Banach algebras.
\begin{theorem}[Gelfand-Mazur] Any Banach algebra which
is also a field is isomorphic as a Banach algebra to
${\mathbb C}$.
\end{theorem}
\section{Analytic functions}
In order to extract more information on the resolvent
we take a detour through a little (easy)
integration theory and complex variable theory.
\begin{theorem} Let $U$ be a Banach space, $[a,b]$
a closed bounded interval in ${\mathbb R}$.
Then we can define an integral $\int_{a}^{b}F(t)\,dt$
for every $F:[a,b]\rightarrow U$ a continuous function
having the following properties (here
$F,\ G:[a,b]\rightarrow U$ are continuous and 
$\lambda,\ \mu\in{\mathbb C}$).

(i) ${\displaystyle \int_{a}^{b}\lambda F(t)+\mu G(t)\,dt 
=\lambda \int_{a}^{b} F(t)\,dt
+\mu \int_{a}^{b} G(t)\,dt}$.

(ii) If $a<c<b$ then
\[ \int_{a}^{b} F(t)\,dt
=\int_{a}^{c} F(t)\,dt+\int_{c}^{b} F(t)\,dt.\]

(iii) ${\displaystyle \left\|\int_{a}^{b} F(t)\,dt\right\|
\leq \int_{a}^{b} \|F(t)\|\,dt.}$

(iv) If $T:U\rightarrow{\mathbb C}$ is a continuous
linear functional
\[ \int_{a}^{b}T (F(t))dt 
=T\int_{a}^{b} F(t)\,dt.\]
\end{theorem}

Using the integral just defined we can define
contour integrals as we did in the complex variable course.
\begin{definition} If $\gamma:[a,b]\rightarrow{\mathbb C}$
is continuously differentiable with $\gamma(a)=\gamma(b)$
and 
$F:[a,b]\rightarrow U$ a continuous function
we define
\[\int_{\gamma}F(z)\,dz=\int_{a}^{b}F(\gamma(t))\gamma'(t)\,dt.\]
\end{definition}
(We shall talk about the `closed contour' $\gamma$.)         

We can now introduce the notion of an
analytic Banach algebra valued function.
\begin{definition} Let $B$ be a Banach algebra
and $\Omega$ a simply connected\footnote{Informally `with no holes'.}
open set in ${\mathbb C}$.
A function $f:\Omega\rightarrow B$ is said to be analytic
on $\Omega$ if there exists an $f':\Omega\rightarrow B$
such that, for all $z\in \Omega$
\[\left\|\frac{f(z+h)-f(z)}{h}-f'(z)\right\|\rightarrow 0\]
as $h\rightarrow 0$ through values of $h$
such that $z+h\in\Omega$.
\end{definition}
\begin{theorem} Let $B$ be a Banach algebra,
$\Omega$ an open simply connected set in ${\mathbb C}$,
and $\gamma$ a closed contour in $\Omega$.
Then, if $f$ is analytic on $\Omega$,
\[\int_{\gamma}f(z)\,dz=0.\]
\end{theorem}

We can follow a first undergraduate complex variable
course to prove the following results.
\begin{lemma} Let $B$ be a Banach algebra
with a unit $e$,
$\Omega$ an open set in ${\mathbb C}$
containing a disc $D(z_{0},R)$,
$f:\Omega\rightarrow B$ an analytic function
and $\gamma$ a contour describing a circle
centre $z_{0}$ radius $0<r<R$. If
$|z_{0}-z|<r$ then
\[f(z)=\frac{1}{2\pi i}\int_{\gamma}\frac{f(w)}{w-z}\,dw.\]
\end{lemma}
\begin{lemma} Let $B$ be a Banach algebra
with a unit $e$,
$\Omega$ an open set in ${\mathbb C}$
containing a disc $D(z_{0},R)$
and $f$ an analytic function on $\Omega$.
There exist unique $a_{0}$, $a_{1}$, $a_{2}$, \dots$\in B$
such that $\sum_{j=0}^{\infty}a_{r}(z-z_{0})^{r}$ converges and
\[f(z)=\sum_{j=0}^{\infty}a_{r}(z-z_{0})^{r}\]
for all $|z-z_{0}|<R$.
\end{lemma}
\begin{theorem} If $B$ is a Banach algebra with unit
\[\sup\{|\lambda|\,:\,\lambda\notin R(x)\}=\rho(x).\]
\end{theorem}
\begin{added}[Exercise] It may help the reader's understanding to
prove the corresponding classical result.

We work in ${\mathbb C}$. Suppose that $a_{j}\in{\mathbb C}$
and that $\limsup_{n\rightarrow\infty}|a_{n}|^{1/n}<\infty$.
Write
\[R=\frac{1}{\limsup_{n\rightarrow\infty}|a_{n}|^{1/n}}\]
and 
\[D(0,r)=\{z\in\in{\mathbb C}\,:\,|z|<r\}.\]


(i) Show that there exists an analytic function
$f:D(0,r)\rightarrow{\mathbb C}$ such that $f^{(j)}(0)=a_{j}$
for all $j$.

(ii) Show that, if $\delta>0$ there does not exist
an analytic function
$g:D(0,r+\delta)\rightarrow{\mathbb C}$ such that $g^{(j)}(0)=a_{j}$
for all $j$.
\end{added} 
\section{Maximal ideals}
One way of exploiting the Gelfand-Mazur theorem is
to introduce the notion of maximal ideals. (From
now on all our Banach algebras will be commutative.)
\begin{lemma} Every proper ideal in a commutative algebra
with unit is contained in a maximal ideal.
\end{lemma}

(Recall that an ideal $I$ in a commutative algebra
$B$ is a vector subspace of $B$ such that
if $a\in B$ and $b\in I$ then $ab\in I$.
An ideal $J$ is maximal if $J\neq B$ but
whenever an ideal $K$ satisfies $J\subseteq K\subseteq B$
either $K=J$ or $K=B$.)

\begin{added}[Lemma] If $I$ is a closed ideal in a Banach algebra
then $B/I$ is a Banach algebra under the norm
\[\|x+I\|_{B/I}=\inf\{\|y\|\,:\,y-x\in I\}\]
If $I\neq B$, then
\[\|e+I\|_{B/I}=1.\]
\end{added}
\begin{added}[Exercise] The following remark is useful
in proving results like the above.

Show that a normed space $(V,\|\ \|)$ is complete
if and only if $\sum_{j=1}^{\infty}x_{j}$ converges whenever
$\sum_{j=1}^{\infty}\|x_{j}\|$ does.
\end{added}
\begin{lemma} Every maximal ideal $M$ in a commutative
Banach algebra with unit is closed.
\end{lemma}
\begin{lemma} If $M$ is a maximal ideal in a commutative
Banach algebra with unit then the quotient $B/M$
is isomorphic to ${\mathbb C}$ as a Banach algebra.
\end{lemma} 


The notion of a maximal ideal is closely linked
to that of a multiplicative linear functional.
\begin{definition} A multiplicative linear functional
on a Banach algebra is  a non-trivial
(i.e. not the zero map)
linear map $\chi:B\rightarrow{\mathbb C}$
such that $\chi(xy)=\chi(x)\chi(y)$ for all $x,\ y\in B$.
\end{definition}
\begin{lemma} If $B$ is a commutative Banach algebra
with identity and $\chi$ is a multiplicative linear
functional then the following results hold.

(i) $\ker\chi$ is a maximal ideal.

(ii) The map $x+\ker\chi\mapsto\chi(x)$ is an algebraic 
isomorphism of $B/\ker\chi$ with ${\mathbb C}$.

(iii) $\chi$ is continuous and $\|\chi\|=1$.
\end{lemma}
\begin{theorem}\label{Theorem, functionals and ideals}
If $B$ is a commutative Banach algebra
with identity then the mapping $\chi\mapsto\ker\chi$
is a bijection between the set of multiplicative
linear functionals on $B$ and its maximal ideals.
\end{theorem}
We now have the following useful corollary.
\begin{lemma} If $B$ is a commutative Banach algebra
with identity then an element $x\in B$ is invertible if
and only $\chi(x)\neq 0$ for all multiplicative
linear functionals $\chi$.
\end{lemma}

The Banach algebra proof Theorem~\ref{Weiner via Banach}
was the first result to convince classical analysts
of the utility of these ideas. The lemma that precedes
it places the result in context.
\begin{lemma} If $f\in C({\mathbb T})$ has 
an absolutely convergent
Fourier series (that is to say,
$\sum_{-\infty}^{\infty}|\hat{f}(n)|<\infty$)
then
\[f(t)=\sum_{-\infty}^{\infty}\hat{f}(n)\exp(int).\]
\end{lemma}
\begin{theorem}[Wiener's theorem]\label{Weiner via Banach} 
Suppose
$f\in C({\mathbb T})$ has an absolutely convergent
Fourier series.
Then, if $f(t)\neq 0$ for all $t\in{\mathbb T}$,
$1/f$ also has an absolutely convergent
Fourier series.
\end{theorem}
\begin{exercise}\label{Exercise, not many ideals}
Let $B$ be any Banach space.
Make it into a Banach algebra by defining $xy=0$
for all $x,\ y\in B$. Now add an identity in the
usual manner. Identify all the multiplicative 
linear functionals.
\end{exercise}
\section{The Gelfand representation}
Throughout
this section $B$ will be a commutative Banach
algebra with a unit $e$ and ${\mathcal M}$ will
be the space of maximal ideals.
If $x\in B$ and $M\in{\mathcal M}$
we know by
Theorem~\ref{Theorem, functionals and ideals}
that there is a unique multiplicative linear
functional $\chi_{M}$ with kernel $M$
so we may write $M(x)=\chi_{M}(x)$.
We give the space
${\mathcal M}$ the weak star topology, that
is to say, the smallest topology containing
sets of the form
\[\{M\in {\mathcal M}\,:\ |M(x)-M_{0}(x)|<\epsilon\}\]
with $M_{0}\in {\mathcal M}$ and $x\in B$.
\begin{lemma} Under the weak topology ${\mathcal M}$ is
a compact Hausdorff space.
\end{lemma}

If $x\in B$ and $M\in{\mathcal M}$ we now write
$\hat{x}(M)=M(x)$.
\begin{lemma} Let $B$ be a commutative Banach algebra with unit.
The mapping $x\mapsto \hat{x}$
is an algebraic homomorphism of $B$ into $C({\mathcal M})$.
As linear map from $(B,\|\ \|)$ to $C({\mathcal M},\|\ \|_{\infty})$
it is continuous with operator norm exactly $1$.
\end{lemma}

We know that the homomorphism $x\mapsto \hat{x}$
need not be injective
\begin{exercise} Justify this statement by considering
the Banach algebra of Exercise~\ref{Exercise, not many ideals}.
\end{exercise}
The following simple observation is the key
to the question of when we have isomorphism.

\begin{lemma} Suppose $x$ is an element of a 
commutative Banach algebra with unit.
Then the complement of the resolvent $R(x)$ is the range of $\hat{x}$.
\end{lemma}
That is to say,
\[\{\hat{x}(M)\,:\,M\in{\mathcal M}\}
=\{\lambda\in{\mathbb C}\,:\,
(x-\lambda e)\ \text{is not invertible}\}.\]

There are two immediate corollaries.
\begin{lemma} If $x$ is an element of a 
commutative Banach algebra with unit,
then $\|\hat{x}\|_{\infty}=\rho(x)$.
\end{lemma}
\begin{lemma} If $x$ is an element of a 
commutative Banach algebra with unit,
then $\rho(x)=0$ if and only if $x$ is contained
in every maximal ideal.
\end{lemma}

We make the following definitions.
\begin{definition} If $B$ is a commutative
Banach algebra with unit we define the radical of $B$
to be the set of all elements contained in every
maximal ideal.
\end{definition}
Thus $x\in\operatorname{radical}(B)$ if and only
if $\rho(x)=0$.
\begin{definition} We say that a commutative
Banach algebra with unit is semi-simple if
and only if its radical consists of $0$ alone.
\end{definition}
\begin{theorem} Let $B$ be a commutative Banach algebra with unit.
The mapping $x\mapsto \hat{x}$
is injective if and only if $B$ is semi-simple.
\end{theorem}
\begin{exercise} Consider the Banach algebra $X$ of continuous
linear maps $T:l^{\infty}\rightarrow l^{\infty}$.
Let $S$ be the map given by
\[S(a_{1},a_{2},\dots)=(0,c_{1}a_{1},c_{2}a_{2},\dots),\]
with the sequence $c_{j}$ bounded.
Explain why the closed Banach subalgebra generated by
$I$ and $S$ is a commutative Banach algebra. Show that
with an appropriate choice of $c_{j}$ we can have $S^{n}\neq 0$
for all $n$ but $\rho(S)=0$.
\end{exercise}
\begin{theorem} Let $B$ be a commutative Banach algebra with unit.
If there exists a $K>0$ such that $\|x\|^{2}\leq K\|x^{2}\|$
for all $x\in B$, then $\rho$ is a norm equivalent to
the original norm on $B$.
\end{theorem}

\section{Finding the Gelfand representation}
Suppose we are given a commutative Banach algebra $B$ and
we wish to find its Gelfand representation. It is not
enough to find its maximal ideals (or, equivalently
its multiplicative linear functionals). We must also
find the correct topology on the space of maximal ideals.
The following simple remarks resolve the problem in
all the cases that we shall consider.
\begin{exercise} Write out the proof that if $(X,\tau)$
and $(Y,\sigma)$ are topological spaces
with $(X,\tau)$ compact and $(Y,\sigma)$ Hausdorff
then, if $f:(X,\tau)\rightarrow (Y,\sigma)$ is
a continuous bijection, $f$ is a homeomorphism.
\end{exercise}
\begin{lemma} Suppose $\tau$ is a compact
topology on the space
${\mathcal M}$ of maximal ideals of commutative Banach
space $B$ with identity. If the maps 
$\hat{x}:({\mathcal M},\tau)\rightarrow{\mathbb C}$
are continuous for each $x\in B$
then $\tau$ is the weak star topology on ${\mathcal M}$.
\end{lemma}

Our first identification was adumbrated in our proof of
Wiener's theorem.
\begin{example} Consider the space $A({\mathbb T})$
of continuous functions $f:{\mathbb T}\rightarrow{\mathbb C}$
with absolutely convergent
Fourier series (that is to say,
$\sum_{-\infty}^{\infty}|\tilde{f}(n)|<\infty$
where $\tilde{f}(n)$ is the $n$th Fourier coefficient). If we set
\[\|f\|_{A}=\sum_{-\infty}^{\infty}|\tilde{f}(n)|,\]
then $(A({\mathbb T}),\|\ \|_{A})$ is a commutative
Banach algebra with unit $1$ under pointwise multiplication.
$(A({\mathbb T}),\|\ \|_{A})$ has maximal ideal space
(identified with) ${\mathbb T}$ under its usual topology.
We have $\hat{f}(t)=f(t)$.
\end{example}
\begin{example} The `transform' nature of the Gelfand transform
is clearer if we seek the maximal ideal space and transform
associated with the Banach algebra $l^{1}({\mathbb Z})$
with standard norm and addition and multiplication
given by convolution (that is ${\mathbf a}*{\mathbf b}={\mathbf c}$
where $c_{m}=\sum_{r=-\infty}^{\infty}a_{m-r}b_{r}$).
\end{example}

Here is a variation on the theme.
\begin{lemma} Let $D=\{z\in{\mathbb C}\,:\,|z|<1\}$
and $\bar{D}=\{z\in{\mathbb C}\,:\,|z|\leq1\}$.
Consider $A(D)$ the set of continuous functions
$f:\bar{D}\rightarrow{\mathbb C}$ such that $f$
is analytic in $D$. If $f_{1},\ f_{2},\ \dots,\ f_{n}\in A(D)$
are such that $\sum_{j=1}^{n}|f_{j}(z)|>0$ for all $z\in{\mathbb C}$
(that is to say that the $f_{j}$ do not vanish simultaneously)
show that we can find $g_{1},\ g_{2},\ \dots,\ g_{n}\in A(D)$
such that $\sum_{j=1}^{n}f_{j}(z)g_{j}(z)=1$ for
all $z\in\bar{D}$.
\end{lemma}
We deviate slightly from our main theme to
introduce some interesting algebras.
\begin{added}[Lemma] If $K$ is a compact set in ${\mathbb C}$,
then the following are closed subalgebras of $C(K)$.

(i) $A(K)$ the collection of $f\in C(K)$ such that
$F$ is analytic on $\Int K$.

(ii) $R(K)$ the collection of $f\in C(K)$ which are
uniform limits of rational functions with poles outside $K$.

(iii) $P(K)$ the collection of $f\in C(K)$ which are
uniform limits of polynomials.

We have $C(K)\supseteq A(K)\supseteq R(K)\supseteq P(K)$.
\end{added}
\begin{added}[Exercise] (i) If $K$ is the closed unit disc, then
$P(K)=R(K)=A(K)\neq C(K)$.

(ii) If $K=\{z\,:\,|z|=1\}$, then
$P(K)\neq R(K)=A(K)=C(K)$.
\end{added}
\begin{added}[The Swiss Cheese] There exists a $K$ with
$R(K)\neq A(K)$.
\end{added}
\begin{added}[Exercise] There exists a $K$ with
$P(K)\neq R(K)\neq A(K)\neq C(K)$.
\end{added}


The next example is a key one in understanding the kind
of problem we face.
\begin{example} Consider the sub Banach algebra $A_{+}({\mathbb T})$
of $A({\mathbb T})$ consisting of elements $f$ of $A({\mathbb T})$
with $\tilde{f}(n)=0$ for $n<0$. Show that $A_{+}({\mathbb T})$
has maximal ideal space
(identified with) ${\mathbb D}$ the closed unit disc.
We have $\hat{f}(z)=\sum_{n=0}^{\infty}\tilde{f}(n)z^{n}$.
\end{example}
\begin{exercise} Consider the sub Banach algebra $A_{-}({\mathbb T})$
of $A({\mathbb T})$ consisting of elements $f$ of $A({\mathbb T})$
with $\tilde{f}(n)=0$ for $n>0$. Find the maximal ideal
space and associated Gelfand transform.
\end{exercise}
\begin{exercise} Consider the space $B({\mathbb T})$
of continuous functions $f:{\mathbb T}\rightarrow{\mathbb C}$
with
$\sum_{-\infty}^{\infty}|n\tilde{f}(n)|<\infty$.
Show that if we set
\[\|f\|_{B}=\sum_{-\infty}^{\infty}(|n|+1)|\tilde{f}(n)|\]
then $(B({\mathbb T}),\|\ \|_{B})$ is a commutative
Banach algebra with unit $1$ under pointwise multiplication.
Find the maximal ideal
space and associated Gelfand transform.
\end{exercise}
\begin{exercise} Consider the sub Banach algebra $B_{+}({\mathbb T})$
of $B({\mathbb T})$ consisting of elements $f$ of $B({\mathbb T})$
with $\tilde{f}(n)=0$ for $n<0$. Find the maximal ideal
space and associated Gelfand transform.
\end{exercise}

Our next example is fundamental.
\begin{example} Let $(X,\tau)$ be a compact Hausdorff
space. The space $C(X)$
of continuous functions $f:X\rightarrow{\mathbb C}$
with the uniform norm is a commutative
Banach algebra with unit $1$ under pointwise operations.
$C(X)$ has maximal ideal space
(identified with) $X$ under its usual topology.
We have $\hat{f}(t)=f(t)$.
\end{example}

One way of expressing many of our results is in
terms of function algebras.
\begin{definition}\label{Definition, function algebra}
Let $(X,\tau)$ be a compact Hausdorff
space. If we consider $C(X)$ as a Banach algebra
in the usual way then any subalgebra $A$ with
a norm which makes it a Banach algebra is
called a function algebra.
\end{definition}
\begin{lemma} With the notation of 
Definition~\ref{Definition, function algebra},
if $A$ is a function algebra containing $1$
with norm $\|\ \|$,
then $\|f\|\geq\|f\|_{\infty}$ for all $f\in A$.
\end{lemma}
\begin{lemma} We use the notation of 
Definition~\ref{Definition, function algebra}.

(i) If
$A$ separates points (that is, given $x,\, y\in X$
with $x\neq y$, we can find an $f\in A$ such that
$f(x)\neq f(y)$),
if $f\in A$ implies $f^{*}\in A$ and
if $f(t)\neq 0$ for all $t\in X$ and $f\in A$ implies $1/f\in A$,
then
$A$ has maximal ideal space
(identified with) $X$ under its usual topology.
We have $\hat{f}(t)=f(t)$.

(ii) If $A$ satisfies (i) and, in addition,
there exists a $K$ such that
$\|f\|^{2}\leq K\|f^{2}\|$, then
$A=C(X)$ and there exists a $\kappa$ such that
\[\kappa \|f\|_{\infty}\geq\|f\|\geq\|f\|_{\infty}\]
for all $f\in A$ 
(so the norms $\|\ \|$ and $\|\ \|_{\infty}$
are Lipschitz equivalent)
\end{lemma}
\begin{exercise} Show that the space $B$
of continuous functions $f:[0,1]\cup[2,3]\rightarrow{\mathbb C}$
such that $f(2+t)=f(t)$ for $t\in[0,1]$ equipped with the
uniform norm
is a function algebra.
Find the maximal ideal
space and associated Gelfand transform.
\end{exercise}

\begin{exercise} Show that the space $C^{1}([0,1])$
of once
continuously differentiable functions equipped
with norm 
\[\|f\|=\|f\|_{\infty}+\|f'\|_{\infty}\]
is a function algebra. 
Find the maximal ideal
space and associated Gelfand transform.
\end{exercise}
Our final set of results on Gelfand transforms is introduced by the 
following fairly simple exercise.
\begin{added}[Exercise] (i) If $B$ is a (not necessarily commutative) 
Banach algebra with unit
then $\exp a$ can be defined in a natural manner using power series.
Show that if $a$ and $b$ commute then
\[\exp(a+b)=\exp(a)\exp(b).\]

(ii) Show that this result may be false if $a$ and $b$ do not commute.

[Hint: Suppose that $a$ and $d$ do not commute. Consider
the expression
$\exp(\eta a+\eta b)-\exp(\eta a)\exp(\eta b)$ when $\eta$ is very
small.]
\end{added}
\begin{added}[Definition] If $B$ is a commutative Banach algebra
with a unit we say that $B$ has an involution if there
exists a map $B\rightarrow B$ given by $f\mapsto f^{*}$
with the following properties.

(i) $f^{**}=f$.

(ii) $(\lambda f)^{*}=\bar{\lambda}f^{*}$ where $\lambda\in{\mathbb C}$
and $\bar{\lambda}$ is the complex conjugate of $\lambda$.

(iii) $(f+g)^{*}=f^{*}+g^{*}$.

(iv) $(fg)^{*}=f^{*}g^{*}$.

(v) $\|ff^{*}\|=\|f\|^{2}$
\end{added}
\begin{added}[Theorem] If $B$ is a commutative Banach algebra
with a unit and $B$ has an involution   $f\mapsto f^{*}$
then the Gelfand transform $B\rightarrow C({\mathcal M})$
is a norm preserving isomorphism with $(f^{*})\hat{\ }=(f\hat{\ })\bar{\ }$.
\end{added}
\begin{added}[Corollary] Suppose that $V$ is a finite dimensional
complex inner product space and
$\alpha:V\rightarrow V$
is a linear map which is normal 
(that is to say $\alpha\alpha^{*}=\alpha^{*}\alpha$
where $\alpha^{*}$ denotes the adjoint map). Then we can
find a set $\pi_{1}$, $\pi_{2}$, \dots, $\pi_{m}$
of commuting orthogonal projections and $\lambda_{1}$, $\lambda_{2}$,
\dots, $\lambda_{m}\in{\mathbb C}$ such that
\[\alpha=\sum_{j=1}^{m}\lambda_{j}\pi_{j}.\]
\end{added}
The result and proof extend without problems to
normal compact operators on Hilbert space.
The general
spectral theorem for normal operators in Hilbert space
can also be obtained in the same way but requires some discussion
of integration theory.
\begin{added}[Exercise] Consider the Banach algebra of $n\times n$
complex matrices. 
Let $A$ be an $n\times n$ diagonal matrix.
Write down a typical matrix in the closed subalgebra generated
by $A$, $A^{*}$ and $I$. What are the multiplicative linear functionals.
\end{added}
\section{Three more uses of Hahn-Banach} The following exercise
provides background for our first discussion but is not examinable.
For the moment $C([a,b])$ will be the set of real valued
continuous functions.
\begin{exercise} We say that a function $G:[a,b]\rightarrow{\mathbb R}$
is of bounded variation if there exists a $K$ such that
whenever we have a dissection 
\[{\mathcal D}=\{x_{0},\ x_{1},\ x_{2},\ \dots,\ x_{n}\}\]
$a=x_{0}<x_{1}<x_{2}<\dots<x_{n}=b$ we have
\[\sum_{j=1}^{n}|G(x_{j})-G(x_{j-1})|\leq K.\]
We write
\[\|G\|_{BV}=\sup_{\mathcal D}\sum_{j=1}^{n}|G(x_{j})-G(x_{j-1})|\]
where the supremum is taken over all possible dissections.

Suppose $f:[a,b]\rightarrow{\mathbb R}$ is continuous.
Let us write
\[S({\mathcal D},f,G)=\sum_{j=1}^{n}f(x_{j})\big(G(x_{j})-G(x_{j-1})\big).\]
If ${\mathcal D}=\{x_{0},\ x_{1},\ x_{2},\ \dots,\ x_{n}\}$
and ${\mathcal D'}=\{x_{0}',\ x_{1}',\ x_{2}',\ \dots,\ x_{n'}'\}$
are such that $|f(t)-f(s)|<\epsilon$ for all $t,\ s\in[x_{j-1},x_{j}]$
$[1\leq j\leq n]$ and for all $t,\ s\in[x_{j-1}',x_{j}']$
$[1\leq j\leq n']$ show by considering 
${\mathcal D}\cup{\mathcal D'}$, or otherwise that
\[|S({\mathcal D},f,G)-S({\mathcal D}',f,G)|\leq 2K\epsilon.\]

Hence, or otherwise, show that there exists a unique $I(f,G)$
such that, given any $\epsilon>0$ we can find a $\delta>0$
such that, given any 
\[{\mathcal D}=\{x_{0},\ x_{1},\ x_{2},\ \dots,\ x_{n}\}\]
with $|x_{j-1}-x_{j}|<\delta$ $[1\leq j\leq n]$ we have
\[|S({\mathcal D},f,G)-I(f,G)|<\epsilon.\] 

We write 
\[I(f,G)=\int_{a}^{b}f(t)\,dG(t).\]

(i) Let $[a,b]=[0,1]$. Find elementary expressions
for $\int_{a}^{b}f(t)\,dG(t)$ 
in the three cases when $G(t)=t$, when $G(t)=-t$
and when $G(t)=0$ for $t<1/2$, $G(t)=1$ for $t\geq 1/2$.

(ii) Show that the map $T:(C([a,b]),\|\ \|_{\infty})\rightarrow{\mathbb R}$
given by
\[Tf=\int_{a}^{b}f(t)\,dG(t)\]
is linear and continuous with $\|T\|\leq\|G\|_{BV}$.
\end{exercise}
(In order to obtain the more satisfactory result $\|T\|=\|G\|_{BV}$
we must put an extra condition on $G$ such as left continuity.)
\begin{added}[Exercise] (i) Suppose $G:{\mathbb R}\rightarrow{\mathbb R}$
is of bounded variation in every interval $[a,b]$. 
Show that if we write
\[G_{+}(t)=G(a)+\sup\left\{\sum_{j=1}^{m}G(b_{j})-G(a_{j})
\,:\,a\leq a_{1}\leq b_{1}\leq a_{2}\leq b_{2}\leq\dots\leq
a_{m}\leq b_{m}\leq t,\ m\geq 1\right\}\]
and
\[G_{-}(t)=\sup\{\sum_{j=1}^{m}G(a_{j})-G(b_{j})
\,:\,a\leq a_{1}\leq b_{1}\leq a_{2}\leq b_{2}\leq\dots\leq
a_{m}\leq b_{m}\leq t,\ m\geq 1\}\]
then $G_{+},\,G_{-}:[a,b]\rightarrow{\mathbb R}$
are increasing functions with
\[G(t)=G_{+}(t)-G_{-}(t)\]
and $\|G\|_{BV}=(G_{+}(b)-G_{+}(a))+(G_{-}(b)-G_{-}(a))$.

(ii) By first considering increasing functions, or otherwise,
show that if $G$ is a function of bounded variation in every interval
then the left and right limits
\[G(t+)=\lim_{h\rightarrow 0,h>0}G(t+h), 
G(t-)=\lim_{h\rightarrow 0,h>0}G(t-h)\]
exist everywhere. 
\end{added}
\begin{theorem} If $T:C([a,b])\rightarrow{\mathbb R}$
is a continuous linear function then we can find
a left continuous function $G:{mathbb R}\rightarrow{\mathbb R}$
of bounded variation in every interval such that
\[Tf=\int_{a}^{b}f(t)\,dG(t)\]
for all $f\in C([a,b])$.
\end{theorem}

If you know a little measure theory you can restate
the theorem in more modern language.
\begin{theorem} {\bf (The Riesz representation theorem.)}
The dual of $C([a,b])$ is the space of Borel
measures on $[a,b]$.
\end{theorem}
The method used can easily be extended to all compact
spaces.

Our second result is more abstract. We require
Aloaoglu's theorem.
\begin{theorem} The unit ball of the dual of a normed
space $X$ is compact in the weak star topology.
\end{theorem}
Our proof of the Riesz representation theorem
used the Hahn-Banach theorem as a convenience.
Our proof of the next result uses it as basic
ingredient.
\begin{theorem} Every Banach space is isometrically
isomorphic to some subspace of $C(K)$ for some
compact space $K$.
\end{theorem}
(In my opinion this result looks more interesting
than it is.)

Our third result requires us to recast 
the Hahn Banach theorem in a geometric form.
\begin{lemma}\label{L;geometrical Banach} 
If $V$ is a real normed space
and $E$ is a convex subset of $V$ containing
$B({\boldsymbol 0},\epsilon)$ for some $\epsilon>0$,
then, given any ${\mathbf x}\notin E$ we can find
a continuous linear map $T:V\rightarrow{\mathbb R}$
such that $T{\mathbf x}=1\geq T{\mathbf e}$ for
all ${\mathbf e}\in E$.
\end{lemma}
\begin{theorem} 
If $V$ is a real normed space
and $F$ is a closed convex subset of $V$,
then, given any ${\mathbf x}\notin F$ we can find
a continuous linear map $T:V\rightarrow{\mathbb R}$
and a real $\alpha$
such that $T{\mathbf x}>\alpha>T{\mathbf k}$ for
all ${\mathbf k}\in F$.
\end{theorem}
\begin{definition} Let $V$ be a real or complex
vector space. If $K$ is a non-empty subset of $V$
we say that $E\subseteq K$ is an \emph{extreme set}
of $K$ if, whenever $u,\ v\in K$, $1>\lambda>0$
and $\lambda u+(1-\lambda)v\in E$, it follows that
$u,\ v\in E$. If $\{e\}$ is an extreme set we
call $e$ an extreme point.
\end{definition}
\begin{exercise} Define an extreme point directly.
\end{exercise}
\begin{exercise} We work in ${\mathbb R}^{2}$.
Find the extreme points, if any, of the following sets
and prove your statements.

(i) $E_{1}=\{{\mathbf x}\,:\,\|{\mathbf x}\|<1\}$.
 
(ii) $E_{2}=\{{\mathbf x}\,:\,\|{\mathbf x}\|\leq 1\}$.

(iii) $E_{3}=\{(x,0)\,:\,x\in{\mathbb R}\}$.

(iv) $E_{4}=\{(x,y)\,:\,|x|,|y|\leq 1\}$.
\end{exercise}
\begin{theorem}{\bf (Krein-Milman).} A non-empty
compact convex subset $K$ of a normed vector space
has at least one extreme point.
\end{theorem}
\begin{theorem} A non-empty
compact convex subset $K$ of a normed vector space
is the closed convex hull of its extreme points
(that is, is the smallest closed convex set
containing its extreme points).
\end{theorem}
Our hypotheses in our version of the Krein-Milman theorem
are so strong as to make the conclusion practically useless.
However the hypotheses can be much weakened as is indicated
by the following version.
\begin{theorem}{\bf (Krein-Milman).} Let $E$
be the dual space of a normed vector space.
A non-empty
convex subset $K$ which is compact in the weak star topology
has at least one extreme point.
\end{theorem}
\begin{theorem} Let $E$
be the dual space of a normed vector space.
A non-empty
convex subset $K$ which is compact in the weak star topology
is the weak star closed convex hull of its extreme points.
\end{theorem}
The results follow at once from a new version of 
Lemma~\ref{L;geometrical Banach}
\begin{added}[Lemma]
If $V$ is a real normed space
and $E$ is a convex subset of the dual space $V'$ containing
an open neighbourhood of ${\boldsymbol 0}$,
then, given any ${\mathbf x}\notin E$ we can find
a continuous linear map $T:V\rightarrow{\mathbb R}$
such that $T{\mathbf x}=1\geq T{\mathbf e}$ for
all ${\mathbf e}\in E$.
\end{added}

\begin{lemma} The extreme points of the closed
unit ball of the dual of $C([0,1])$ are the delta masses
$\delta_{a}$ and $-\delta_{a}$ with $a\in[0,1]$.
\end{lemma}
 


\section{The Rivlin-Shapiro formula} In this section
we give an elegant use of extreme points due to
Rivlin and Shapiro.
\begin{lemma}{\bf Carath\'{e}odory}
We work in ${\mathbb R}^{n}$. Suppose
that ${\mathbf x}\in{\mathbb R}^{n}$ and we are given
a finite set of points ${\mathbf e}_{1}$,
${\mathbf e}_{2}$, \dots, ${\mathbf e}_{N}$ and positive
real numbers $\lambda_{1}$, $\lambda_{2}$, \dots, $\lambda_{N}$
such that
\[\sum_{j=1}^{N}\lambda_{j}=1,
\ \sum_{j=1}^{N}\lambda_{j}{\mathbf e}_{j}={\mathbf x}.\]
Then {\bf after renumbering} the ${\mathbf e}_{j}$
we can find positive
real numbers $\lambda_{1}'$, $\lambda_{2}'$, \dots, $\lambda_{m}'$
with $m\leq n+1$ such that
\[\sum_{j=1}^{m}\lambda_{j}'=1,
\ \sum_{j=1}^{m}\lambda_{j}'{\mathbf e}_{j}={\mathbf x}.\]
\end{lemma}
\begin{added}[Exercise] Show by means of an example
that we can not necessarilly 
take $m=n$ in Carath\'{e}odory's lemma.
\end{added} 
\begin{lemma}\label{Shapiro 1} 
Consider ${\mathcal P}_{n}$, the subspace
of $C([-1,1])$ consisting of real polynomials of degree
$n$ or less. If $S:{\mathcal P}_{n}\rightarrow{\mathbb R}$
is linear then we can find an $N\leq n+2$
and 
distinct points $x_{1}$, $x_{2}$, \dots, $x_{N}\in[-1,1]$
and non-zero
real numbers $\lambda_{1}$, $\lambda_{2}$, \dots, $\lambda_{N}$
such that
\[\sum_{j=1}^{N}|\lambda_{j}|=1,
\ \|S\|\sum_{j=1}^{N}\lambda_{j}P(x_{j})=SP\]
for all $P\in{\mathcal P}_{n}$. 
\end{lemma}
\begin{lemma}\label{Shapiro 2} 
We continue with the hypotheses
and notation of Lemma~\ref{Shapiro 1}
There exists a $P_{*}\in{\mathcal P}_{n}$ such that
\[P_{*}(x_{j})=\|P_{*}\|_{\infty}\sgn\lambda_{j}\]
for all $j$ with $0\leq j\leq N$.
Further, if $P\in{\mathcal P}_{n}$ satisfies
\[P(x_{j})=\|P\|_{\infty}\sgn\lambda_{j}\]
then $\|P\|_{\infty}\|S\|=SP$.
\end{lemma}

The following
results are of considerable interest
in view of Lemma~\ref{Shapiro 2}.
\begin{lemma} We have $\cos n\theta=T_{n}(\cos\theta)$
where $T_{n}$ is a real polynomial of degree $n$.
Further

(i) $|T_{n}(x)|\leq 1$ for all $x\in [-1,1]$.

(ii) There exist $n+1$ 
distinct points $x_{1}$, $x_{2}$, \dots, $x_{n+1}\in[-1,1]$
such that $|T_{n}(x_{j})|=1$ for all $1\leq j\leq n+1$.
\end{lemma}
\begin{lemma}\label{L;Tchebychev extremme}
Suppose that $P$ is a real polynomial of degree
$n$ or less such that

(i) $|P(x)|\leq 1$ for all $x\in [-1,1]$ and

(ii) there exist $n+1$ 
distinct points $x_{1}$, $x_{2}$, \dots, $x_{n+1}\in[-1,1]$
such that $|P(x_{j})|=1$ for all $1\leq j\leq n+1$.

\noindent Then $P=\pm T_{n}$.
\end{lemma}
Note that Lemma~\ref{L;Tchebychev extremme} tells
us that there is no  real polynomial of degree
$n$ or less which takes its extreme absolute value
on $[-1,1]$ at $n+2$ points. (Thus we can replace the condition
$N\leq n+2$ in Lemma~\ref{Shapiro 1} by $N\leq n+1$.)
\begin{theorem}If
$P$ is a real polynomial of degree at most $n$
and $t\notin[-1,1]$, then
\[|P(t)|\leq \sup _{|x|\leq 1}|P(x)||T_{n}(t)|.\]
\end{theorem}
\begin{exercise} If
$P$ is a real polynomial of degree at most $n$
and $t\notin[-1,1]$, then
then
\[|P^{(r)}(t)|\leq |T^{(r)}(t)|\sup_{|x|\leq 1}|P(x)|.\]
\end{exercise}
\begin{exercise} (This exercise is part of the course.)
(i) Show that if $n\geq 1$ the coefficient of $t^{n}$ in
$T_{n}(t)$ is $2^{n-1}$.

(ii) Show that if $n\geq 1$ and $P$ is a real polynomial
of degree $n$ or less with $|P(t)|\leq 1$ then 
the coefficient of $t^{n}$ in
$P(t)$ has absolute value at most $2^{n-1}$.

(iii) Find, with proof, a polynomial $P$ of degree at
most $n-1$ which minimises
\[\sup_{t\in[-1,1]}|t^{n}-P(t)|.\]
Show that $P$ is unique. (Tchebychev introduced his polynomials
$T_{n}$ in this context.)
\end{exercise}

   
\begin{thebibliography}{9}
\bibitem{Bollobas} B. Bollob\'as \emph{Linear Analysis
: an Introductory Course} (CUP 1991)
\bibitem{Gofman}
C. Gofman and G. Pedrick \emph{A First
Course in Functional Analysis} (Prentice Hall 1965,
available as a Chelsea reprint from the AMS)
\bibitem{Pryce} J. D. Pryce \emph{Basic Methods of Linear Functional
Analysis} (Hutchinson 1973)
\bibitem{Rudin 1}
W. Rudin \emph{Real and Complex Analysis} (McGraw Hill, 2nd Edition, 1974)
\bibitem{Rudin 2}
W. Rudin \emph{Functional Analysis} (McGraw Hill 1973)
\end{thebibliography}
