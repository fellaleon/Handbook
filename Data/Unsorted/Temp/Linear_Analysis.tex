\chapter{Linear Analysis}

\section{Some inequalities} Inequalities lie at the heart of analysis.
In this section we prove some inequalities which lie at
the heart of this course.

We start with some observations from Part~1A.
\begin{exercise} Suppose $f:(0,\infty)\rightarrow{\mathbb R}$
is twice differentiable with $f''(x)<0$ for all $x\in{\mathbb R}$.
Then if $0<t<s$ and $0<\lambda<1$, it follows that
\[\lambda f(t)+(1-\lambda) f(s)<f\big(\lambda t+(1-\lambda)s\big).\]
\end{exercise}
In other words, if $f''(x)<0$, then $f$ is strictly concave.
Applying the result with $f=\log$, we obtain the following result.
\begin{lemma} Suppose $p,\ q$ are real and positive with
\[\frac{1}{p}+\frac{1}{q}=1.\]
Then, if $a,\,b>0$, we have
\[a^{1/p}b^{1/q}\leq \frac{a}{p}+\frac{b}{q}\]
with equality if and only if $a=b$.

Our result remains true if $a,\,b\geq 0$.
\end{lemma}
We can now obtain our first version of the
H{\"o}lder inequality\footnote{Mathematical 
nomenclature should not be confused
with historic truth. H\"{o}lder's paper makes it clear
that he is discussing an inequality previously proved by
L.~G.~Rogers (1862--1933). Rogers was Professor of Mathematics at Yorkshire
College (now the University of Leeds) and a fine mathematician
`with little ambition or desire for recognition'. In 1913,
Ramanujan conjectured some remarkable identities which
no-one could prove. In 1917, whilst looking through old journals,
Ramanujan came across a 1894 paper of Rogers in which
a more general form of the identities were proved.
Rogers' election to the Royal Society followed.}.
(Here and elsewhere
we will write ${\mathbb F}$ to mean either
${\mathbb R}$ or ${\mathbb C}$. This reflects the
fact that many theorems of Linear Analysis
apply both to real and complex vector spaces. However, just as in
other branches of analysis and algebra, there are important
theorems which apply only to complex or only to real spaces.)
\begin{theorem}\label{T;Holder 1} Suppose $p,\ q$ are real and positive with
\[\frac{1}{p}+\frac{1}{q}=1.\]
If $a_{j},\,b_{j}\in{\mathbb F}$, then
\[\sum_{j=1}^{n}|a_{j}b_{j}|\leq
\left(\sum_{j=1}^{n}|a_{j}|^{p}\right)^{1/p}
\left(\sum_{j=1}^{n}|b_{j}|^{q}\right)^{1/q}\]
with equality if and only if we can find $A,\,B\in{\mathbb R}$,
not both zero,
with $A|b_{j}|^{q}=B|a_{j}|^{p}$ for all $1\leq j\leq n$.
\end{theorem}
I will probably leave out the examination of the case of equality
in Theorems~\ref{T;Holder 1} and~\ref{T;Minkowski} but once
you have mastered the main proof it a very instructive exercise
to look closely the cases of equality.

If we set $p=2$ in H{\"o}lder's inequality
we recover the Cauchy-Schwarz inequality.

Our H{\"o}lder inequality is complemented by
a `reverse H{\"o}lder inequality'.
\begin{theorem} (i) Suppose $p,\ q$ are real and positive with
\[\frac{1}{p}+\frac{1}{q}=1.\]
If $a_{j}\in{\mathbb F}$ and
\[\sum_{j=1}^{n}|a_{j}b_{j}|\leq A
\left(\sum_{j=1}^{n}|b_{j}|^{q}\right)^{1/q}\]
for all choices of $b_{j}\in{\mathbb F}$  
then
\[\left(\sum_{j=1}^{n}|a_{j}|^{p}\right)^{1/p}\leq A.\]

(ii) (Stronger version) Suppose $p,\ q$ are real and positive with
\[\frac{1}{p}+\frac{1}{q}=1.\]
If $a_{j}\in{\mathbb F}$ and
\[\left|\sum_{j=1}^{n}a_{j}b_{j}\right|\leq A
\left(\sum_{j=1}^{n}|b_{j}|^{q}\right)^{1/q}\]
for all choices of $b_{j}\in{\mathbb F}$
then
\[\left(\sum_{j=1}^{n}|a_{j}|^{p}\right)^{1/p}\leq A.\]
\end{theorem}

Putting H{\"o}lder's inequality and the reverse
H{\"o}lder inequality together, we
get Minkowski's inequality.
\begin{theorem}\label{T;Minkowski} 
Suppose $p>1$. Then 
\[\left(\sum_{j=1}^{n}|a_{j}|^{p}\right)^{1/p} 
+\left(\sum_{j=1}^{n}|b_{j}|^{p}\right)^{1/p}
\geq \left(\sum_{j=1}^{n}|a_{j}+b_{j}|^{p}\right)^{1/p}\]
with equality only if we can find $\alpha,\,\beta\in{\mathbb F}$,
not both zero,
such that $\beta a_{j}=\alpha b_{j}$ for all $1\leq j\leq n$.
\end{theorem}
(Exercise~\ref{C5.2} contains an apparently simpler proof,
but the explicit use of the reverse
H{\"o}lder inequality in the proof above may help fix it 
in the reader's mind.)

For us, as for Minkowski, this has a clear geometrical interpretation.
\begin{theorem} Suppose $p>1$. Then
\[\|{\mathbf a}\|_{p}=\left(\sum_{j=1}^{n}|a_{j}|^{p}\right)^{1/p}\]
defines a norm on ${\mathbb F}^{n}$.
\end{theorem}
We can extend most of our results (but not all) to
the cases $p=1$ and $p=\infty$.
\begin{exercise}
(i) Show that
\[\|{\mathbf a}\|_{1}=\sum_{j=1}^{n}|a_{j}|\]
defines a norm on ${\mathbb F}^{n}$.

(ii) Show that
\[\|{\mathbf a}\|_{\infty}=\max_{1\leq j\leq n}|a_{j}|\] 
defines a norm on ${\mathbb F}^{n}$.

(iii) Show that, if ${\mathbf a},\,{\mathbf b}\in{\mathbb F}^{n}$,
then
\[\sum_{j=1}^{n}|a_{j}||b_{j}|\leq \|{\mathbf a}\|_{\infty}
\|{\mathbf b}\|_{1}.\]

(iv) State and prove the \emph{two} `reverse H{\"o}lder'
inequalities corresponding to~(iii).

(v) If $\|{\mathbf a}+{\mathbf b}\|_{1}=
\|{\mathbf a}\|_{1}+\|{\mathbf b}\|_{1}$,
does it follow that we can find $\alpha,\,\beta\in{\mathbb F}$,
not both zero,
such that $\beta a_{j}=\alpha b_{j}$ for all $1\leq j\leq n$?
Give reasons.

(vi) If $\|{\mathbf a}+{\mathbf b}\|_{\infty}= 
\|{\mathbf a}\|_{\infty}+\|{\mathbf b}\|_{\infty}$,
does it follow that we can find $\alpha,\,\beta\in{\mathbb F}$,
not both zero,
such that $\beta a_{j}=\alpha b_{j}$ for all $1\leq j\leq n$?
Give reasons.
\end{exercise}
\section{In finite dimension, norms are equivalent}
We have produced a variety of norms on ${\mathbb F}^{n}$,
but the following theorem (which the reader may have seen in
Part 1B) shows that they are all equivalent in a rather
strong sense (called Lipschitz equivalence).
\begin{theorem}\label{T;norms same} 
Suppose $E$ is a finite dimensional vector space
over ${\mathbb F}$.
If $\|.\|_{1}$ and $\|.\|_{2}$ are two norms on $E$,
then we can find a constant $K>0$ such that
\[K\|{\mathbf x}\|_{1}\geq \|{\mathbf x}\|_{2}
\geq K^{-1}\|{\mathbf x}\|_{1}\]
for all ${\mathbf x}\in E$.
\end{theorem}
Note that the proof depends ultimately on the fundamental axiom
of analysis.

However, the unit balls of the different norms $\|.\|_{p}$
look more and more different as the dimension of the space increases.
\begin{exercise}\label{E;increase norm} (i) If $\infty>s>r\geq 1$ 
show, by applying H\"{o}lder's inequality with
$a_{j}=x_{j}^{s}$, $b_{j}=1$ and $p=s/r$, that,
if $x_{j}\geq 0$, we have
\[\left(\sum_{j=1}^{n}x_{j}^{r}\right)^{1/r}\leq
\left(\sum_{j=1}^{n}x_{j}^{s}\right)^{1/s}n^{r^{-1}-s^{-1}}\]
with equality if and only if $x_{1}=x_{2}=\dots=x_{n}$.

Thus, if we work in ${\mathbb F}^{n}$,
\[\|{\mathbf x}\|_{r}\leq n^{r^{-1}-s^{-1}}\|{\mathbf x}\|_{s}\]
and we cannot improve the inequality.

(ii) If $\infty>s>r\geq 1$, show, by setting $t=x_{j}/\|{\mathbf x}\|_{s}$
in the inequality $t^{r}\geq t^{s}$ $[0\leq t\leq 1]$,
or otherwise, that, if $x_{j}\geq 0$,
\[\left(\sum_{j=1}^{n}x_{j}^{s}\right)^{1/s}\leq
\left(\sum_{j=1}^{n}x_{j}^{r}\right)^{1/r}\]
and identify the cases of equality.

Conclude that, if we work in ${\mathbb F}^{n}$,
\[\|{\mathbf x}\|_{s}\leq \|{\mathbf x}\|_{r}\]
and we cannot improve the inequality.

(iii) State and prove (the proofs are easy)
the corresponding results when we allow $s=\infty$.
\end{exercise}

Once we reach the simplest infinite dimensional space
Theorem~\ref{T;norms same} fails.
\begin{definition} We write $c_{00}$ for the space
of sequences
\[{\mathbf a}=(a_{1},a_{2},\dots)\]
with $a_{n}\in{\mathbb F}$ and only finitely many $a_{j}$
non-zero. [Warning: The notation $c_{00}$ is not universal.]
\end{definition}
\begin{exercise}\label{E;many norms} 
(i) Check that  $c_{00}$ is a vector space.

(ii) Check that 
\[\|{\mathbf a}\|_{p}=\left(\sum_{j=1}^{\infty}|a_{j}|^{p}\right)^{1/p}\]
defines a norm on $c_{00}$ for $\infty>p\geq 1$.

(iii) Define an appropriate norm $\|.\|_{\infty}$ and
show that is a norm.

(iv) Show that, given any   $\infty\geq s>r\geq 1$
and any $K>0$, we can find a non-zero ${\mathbf a}\in c_{00}$
such that
\[\frac{\|{\mathbf a}\|_{r}}{\|{\mathbf a}\|_{s}}>K.\]
\end{exercise} 
\section{Banach spaces} It is generally thought that
the algebraic properties of large objects are rather dull
and this is certainly the case for normed spaces.
The remedy is to introduce extra analytic structure,
in this case completeness.
\begin{definition} A Banach space is a complete normed space.
\end{definition}
We observe that the norms we have placed on $c_{00}$
are not complete. 
\begin{exercise} Verify this statement.
\end{exercise}
Later, in Lemma~\ref{L;no countable}, 
we shall show that $c_{00}$ cannot be given a complete norm.
Thus $c_{00}$ is good for nothing except the production of counterexamples.
(But it is very good for this purpose.)
 
However, the work we have already done enables us to produce
some nice infinite dimensional Banach spaces.
\begin{theorem} Let $1\leq p<\infty$. If $l^{p}$ is
the collection of sequences 
\[{\mathbf a}=(a_{1},a_{2},\dots)\]
with $a_{n}\in{\mathbb F}$ and
\[\sum_{j=1}^{\infty}|a_{j}|^{p}\ \text{convergent},\]
then $l^{p}$ is a vector space with the usual coordinatewise
definition of addition and multiplication by scalars.
If we set
\[\|{\mathbf a}\|_{p}
=\left(\sum_{j=1}^{\infty}|a_{j}|^{p}\right)^{1/p},\]
then $\|.\|_{p}$ is a complete norm on $l^{p}$.
\end{theorem}
\begin{lemma} If $l^{\infty}$ is
the collection of bounded sequences 
\[{\mathbf a}=(a_{1},a_{2},\dots)\]
with $a_{n}\in{\mathbb F}$,
then $l^{\infty}$ is a vector space with the usual coordinatewise
definition of addition and multiplication by scalars.
If we set
\[\|{\mathbf a}\|_{\infty}=\sup_{1\leq j}|a_{j}|\]
then $\|.\|_{\infty}$ is a complete norm on $l^{\infty}$.
\end{lemma}
At some point, it became fashionable to write $l_{p}$
instead of $l^{p}$. The reader must be prepared for both notations.
\begin{exercise} (i) Show that, if $\infty\geq s>r\geq 1$,
then $l^{r}\subseteq l^{s}$, $l^{r}\neq l^{s}$.

(ii) If $\infty\geq s>r\geq 1$ show that, if ${\mathbf x}\in l^{r}$,
then ${\mathbf x}\in l^{s}$ and $\|{\mathbf x}\|_{s}\leq\|{\mathbf x}\|_{r}$.
Give an explicit example of a sequence ${\mathbf x}(k)\in l^{r}$
with $\|{\mathbf x}(k)\|_{s}=1$ but 
$\|{\mathbf x}(k)\|_{r}\rightarrow\infty$
as $k\rightarrow\infty$.
\end{exercise}

Although we seek to study infinite dimensional
Banach spaces by using geometrical
intuition, they differ in important respects from finite
dimensional spaces. Here is one example.
\begin{theorem}\label{T;not compact} The closed unit ball
\[\bar{B}=\{{\mathbf x}\in E\,:\,\|{\mathbf x}\|\leq 1\}\]
of an infinite dimensional Banach space $(E,\|.\|)$
is not compact.
\end{theorem}
Our proof depends on two results which are of independent interest.
\begin{lemma} Any finite dimensional subspace
of a Banach space is closed.
\end{lemma}
\begin{theorem}\label{T;Riesz}{\bf [Lemma of F. Riesz]} 
If $F$ is a closed subspace of a  Banach space $(E,\|.\|)$
and $F\neq E$, then, given any $\epsilon>0$, we can find
an ${\mathbf e}\in E$ with $\|{\mathbf e}\|=1$ such that
\[\|{\mathbf e}- {\mathbf f}\|>1-\epsilon\]
for all ${\mathbf f}\in F$.
\end{theorem}
The technique that we use to prove Theorem~\ref{T;Riesz}
is very useful in a wide variety of circumstances.
\section{Continuous linear functions} When we studied linear maps on
finite dimensional normed spaces we made repeated use of
the fact that they were continuous.
\begin{exercise} Suppose $(U,\|.\|_{U})$ and
and $(V,\|.\|_{V})$ are finite dimensional normed spaces.
Show that every linear map $\alpha:U\rightarrow V$
is continuous.
\end{exercise}
We cannot make this assumption when we study linear
maps on infinite dimensional spaces.
\begin{exercise}\label{E;not continuous linear} 
Consider the space $c_{00}$ of sequences
of complex numbers only finitely many of which are non-zero,
equipped with the norm
\[\|{\mathbf a}\|_{1}=\sum_{j=1}^{\infty}|a_{j}|.\]
Give an example of a linear map
$\alpha:c_{00}\rightarrow{\mathbb F}$ which is not continuous.
\end{exercise}
The treatment of continuous linear maps on
infinite
dimensional vector spaces is much aided by some simple observations.
\begin{lemma}  Suppose $(U,\|.\|_{U})$ and
and $(V,\|.\|_{V})$ are normed spaces
and $\alpha:U\rightarrow V$ is linear. 
The following statements are equivalent.

(i) $\alpha$ is continuous everywhere.

(ii) $\alpha$ is continuous at ${\boldsymbol 0}$.

(iii) There exists a $C$ such that
\[\|\alpha{\mathbf u}\|_{V}\leq C\|{\mathbf u}\|_{U}\]
for all ${\mathbf u}\in U$.
\end{lemma}
For this reason continuous linear maps are sometimes called bounded linear 
maps. 

We can define the \emph{operator norm}
and derive its elementary properties in exactly
the same way as in the finite dimension case.
\begin{definition} Suppose $(U,\|.\|_{U})$ and
$(V,\|.\|_{V})$ are normed spaces
and $\alpha:U\rightarrow V$ is continuous and linear.
We define the operator norm $\|\alpha\|$ of $\alpha$ as
follows.
\[\|\alpha\|=\sup_{\|{\mathbf u}\|_{U}\leq 1}\|\alpha{\mathbf u}\|_{V}.\]
\end{definition}
\begin{exercise} (i) Suppose $(U,\|.\|_{U})$ and
$(V,\|.\|_{V})$ are normed spaces. The collection
${\mathcal L}(U,V)$ 
of continuous linear maps $\alpha:U\rightarrow V$
forms a vector space under 
pointwise addition and multiplication by scalars.
The operator norm is indeed a norm on ${\mathcal L}(U,V)$.

(ii) Suppose $(U,\|.\|_{U})$ is
a normed space. Then the identity map $\iota\in{\mathcal L}(U,U)$
and $\|\iota\|=1$.

(iii) Suppose $(U,\|.\|_{U})$,
$(V,\|.\|_{V})$ and $(W,\|.\|_{W})$ are normed spaces.
If $\alpha\in {\mathcal L}(U,V)$ and $\beta\in{\mathcal L}(V,W)$,
then $\beta\alpha\in {\mathcal L}(U,W)$ and
\[\|\beta\alpha\|\leq \|\beta\|\|\alpha\|.\]
\end{exercise}
\begin{theorem}\label{T;dual complete} If $(U,\|.\|_{U})$ and
and $(V,\|.\|_{V})$ are Banach spaces, it follows that
$\big({\mathcal L}(U,V),\|.\|\big)$ is a Banach space.
\end{theorem}
(For an improvement see Exercise~\ref{E;more dual complete}.)
A particularly important example of 
the space $({\mathcal L}(U,V),\|.\|)$
occurs when $V={\mathbb F}$.
\begin{definition} If $(U,\|.\|_{U})$ is a normed space
over ${\mathbb F}$ we say that a continuous
linear map $T:U\rightarrow{\mathbb F}$
is a \emph{bounded linear functional}.
The space $U'={\mathcal L}(U,{\mathbb F},\|.\|)$
is called the \emph{dual} space of $U$.
\end{definition}
\begin{theorem}\label{T;dual p} If $\infty>p\geq 1$ 
and $p^{-1}+q^{-1}=1$, then $({l^{p}})'=l^{q}$
(More exactly, there is a natural identification
of $({l^{p}})'$ and $l^{q}$. When we have introduced more definitions,
you may prefer the statement that there is a natural isometric
isomorphism between  $({l^{p}})'$ and $l^{q}$.)
\end{theorem}
The following important exercise illustrates both our method of
proving Theorem~\ref{T;dual p} and its limitations.
\begin{exercise}\label{E;dual $c$} 
(i) Show that $c$, the set of all sequences 
${\mathbf a}$ such that $a_{j}$ tends to a limit 
as $j\rightarrow\infty$, is a closed subspace of
$l^{\infty}$ with its usual norm. Show that $c_{0}$, 
the set of all sequences 
${\mathbf a}$ such that $a_{j}\rightarrow 0$ 
as $j\rightarrow\infty$, is a closed subspace of
$l^{\infty}$ with its usual norm.

Because of these results we may consider
$c$ and $c_{0}$ as Banach spaces under the supremum norm
(that is to say the $l^{\infty}$ norm).

(ii) Show that $c_{0}'$ can be identified in a natural manner
with $l^{1}$.

(iii) Identify the elements of $c'$ in a reasonably
natural manner.
\end{exercise}
It is an important fact that the method we use
to prove Theorem~\ref{T;dual p} fails when we
try to find the dual of $l^{\infty}$.
In some sense $l^{\infty}$ is too large
for us to deal with. Lemma~\ref{L;large space}
make it clear what is going on.

\begin{definition}\label{def:dense} 
We say that a subset $E$ of a metric space $(X,d)$ is dense\index{dense} in $X$, if given any $x\in X$, we can find $x_{n}\in E$  with $x_{n}\rightarrow x$.
\end{definition}

\begin{definition}\label{def:seperable}
A metric space $(X,d)$ is separable\index{separable} if we can find a countable dense subset of $X$.
\end{definition}

\begin{lemma}\label{L;large space}
The space $l^{p}$ is separable if $\infty>p\geq 1$
but $l^{\infty}$ is not separable.
\end{lemma}  
\section{Second duals} Students find it hard to understand
the treatment of second duals in 1B Algebra but the
subject actually becomes easier to understand when treated
in a more general setting.
\begin{lemma}\label{L;double dual 1} Let $U$ be a normed vector space.
Then the map $J:U\rightarrow U''$ given by
\[(Ju)(T)=Tu\]
for $u\in U$ and $T\in U'$ is a well
defined continuous linear map.
\end{lemma}
In 1B Algebra, when $U$ is a finite dimensional space
it is easy to show that $J$ is injective.
In the infinite dimensional case we need a further idea.
\begin{definition}\label{D;sufficiently rich} 
[Warning. This is non-standard.]
We say that a dual $U'$ of a normed space $U$ is
\emph{sufficiently rich} if, whenever $u\in U$, 
\[\sup_{T\in U',\|T\|\leq 1}\|Tu\|=\|u\|.\]
\end{definition}
It is easy to check that all the spaces we have met
so far are sufficiently rich.
\begin{exercise} (i) Show that the duals of the $l^{p}$
spaces are sufficiently rich $[1\leq p\leq \infty]$.
(Note that, although we cannot fully identify the
dual of $l^{\infty}$ we can, nonetheless, show that
it is sufficiently rich.)

(ii) Show that if $U$ is a Banach space with sufficiently
rich dual then ${\mathcal L}(U,U)$ has sufficiently rich dual.
(We need this later in the proof of Lemma~\ref{L;non-empty spectrum}
but the reader should not loose any sleep over this.)
\end{exercise}
The reader will find it easy to check
(and should carry out the check) that any
specific space has a sufficiently rich dual.

If we use the axiom of choice, then it can be shown
that all duals are sufficiently rich. Any text that the
reader consults will use the  axiom of choice
and will therefore omit the condition that
the dual is sufficiently rich.  
\begin{lemma} Let $U$ be a normed vector space
with a sufficiently rich dual.
Then the map $J:U\rightarrow U''$ introduced
in Lemma~\ref{L;double dual 1} is isometric
(that is to say $\|Ju\|_{U''}=\|u\|_{U}$)
and so injective.
\end{lemma}
(Here and everywhere else, we use the operator norm on
$U'$ and $U''$.)

If $U'$ is sufficiently rich,
it is reasonable to use $J$ to give natural identification
of $U$ with $JU$ and write $U\subseteq U''$.
\begin{lemma} If $U$ is a Banach space with sufficiently rich dual
then (with the natural identification) $U$ is a closed
subspace of $U''$.
\end{lemma}

In 1B Algebra, when $U$ is a finite dimensional space,
a dimensional argument shows that $J$ is surjective
(and so, with the natural identification, $U=U''$).
If $1<p<\infty$ then the work we have already done shows 
that $(l^{p})''=l^{p}$.

However, the following important example shows that
$U$ may be a proper subspace of its second dual $U''$.
\begin{exercise} (Part~(i) and most of part~(iii) 
were done in  Exercise~\ref{E;dual $c$}.)
Let $c_{0}$ be the the subset of $l^{\infty}$
consisting of those sequences ${\mathbf a}$ such that
that $a_{j}\rightarrow 0$ as $j\rightarrow\infty$.

(i) Show that $c_{0}$ is a closed subspace of $l^{\infty}$
and so $(c_{0},\|.\|_{\infty})$ is a Banach space.

(ii) Show that $c_{0}$ is separable.

(iii) Show that $c_{0}'$ can be identified in a natural manner
with $l^{1}$. Show also that the dual of $c_{0}$ is sufficiently rich.

(iv) Deduce that $c_{0}''$ can identified in a natural manner with 
$l^{\infty}$ and so the mapping $J:c_{0}\rightarrow c_{0}''$ introduced
in Lemma~\ref{L;double dual 1} cannot be surjective.
\end{exercise}

There are two ways of looking at Banach spaces. One is to study
each space in its own right using the the language and insights
of linear analysis as tools. Each of the spaces $l^{1}$,
$l^{2}$ and $l^{\infty}$ could be, and has been, the
object of a lifetime's study. This is the point of view
of the present course or, at least, the present 
lecturer\footnote{Who has spent most of his mathematical life studying
a particular representation of $l^{1}$.}.

The second way of looking at Banach spaces is to study
them as general structures. In this case we do not study
individual spaces but isomorphism classes or
isometric isomorphism classes of Banach spaces.
\begin{definition} (i) The normed spaces $(U,\|.\|_{U})$
and $(V,\|.\|_{V})$ are \emph{isomorphic} if there exists a
vector space isomorphism $T:U\rightarrow V$ such that
both $T$ and $T^{-1}$ are continuous.

(ii)  The normed spaces $(U,\|.\|_{U})$
and $(V,\|.\|_{V})$ are 
\emph{isometrically isomorphic} if there exists a
vector space isomorphism $T:U\rightarrow V$ such that
\[\|Tu\|_{V}=\|u\|_{U}\]
for all $u\in U$.
\end{definition}
From this point of view, all we know so far
about the $l^{p}$ spaces is that 
$l^{\infty}$ is not isomorphic to $l^{p}$ for $1\leq p<\infty$.
(If the reader is interested, Exercise~\ref{E;not $l^{1}$}
shows that $l^{1}$ is not isomorphic to $l^{p}$ for $p\neq 1$
and Exercise~\ref{E;not Hilbert}
shows that $l^{2}$ is not isomorphic to $l^{p}$ for $p\neq 2$.
It has been shown that, in fact, spaces $l^{p}$ with distinct
$p$ are not isomorphic.) We give a very simple example of Banach space
isomorphism in Exercise~\ref{E;see isomorphism}.

For the avoidance of doubt, the reader is instructed that,
both in the notes and in examination questions, statements
about $l^{p}$ and similar spaces refer to their concrete realisations
and not to their isomorphism classes. If isomorphism
is to be considered, this will be stated explicitly.
\section{Baire category} The Baire category is a profound triviality
which condenses the folk wisdom of a generation of ingenious
mathematicians into a single statement.

\begin{theorem}\label{T;Baire 1} If $(X,d)$ is a complete metric space
and $U_{j}$ is an open dense subset of $X$ for each $j\geq 1$,
then $\bigcap_{j=1}^{\infty}U_{j}\neq\emptyset$.
\end{theorem}
(The same ideas are used to prove a similar result for compact
topological spaces in Exercise~\ref{E;Baire compact}.)

In some sense,the property of belonging to $U_{j}$ is
stable (since $U_{j}$ is open, small perturbations 
leave us within $U_{j}$) and the property of not belonging
to $U_{j}$ unstable (since $U_{j}$ is dense, we can move
to $U_{j}$ by arbitrarily small perturbations). 
\begin{exercise}\label{E;dense matrices}
Consider the space of $n\times n$ complex
matrices with the operator norm. Recall that given
any matrix $A$ we can find a non-singular matrix $B$
such that $BAB^{-1}$ is upper triangular. By using this result,
or otherwise, 
show that the set of of matrices with $n$ distinct eigenvalues
is open and dense.
\end{exercise}
For historical reasons Baire's theorem is associated
with some rather unhelpful nomenclature.
\begin{definition} Consider a metric space $(X,d)$.
If $E_{j}$ is closed with dense complement and
$E\subseteq\bigcup_{j=1}^{\infty}E_{j}$, then
$E$ is said to be of \emph{first category}.
\end{definition}
Baire's theorem can be restated as follows.
\begin{theorem}{\bf[Baire's category theorem]}
If $(X,d)$ is a complete metric space, then
$X$ is not of first category.
\end{theorem}
The following remark is very useful.
\begin{lemma} Consider a metric space $(X,d)$.
The countable union of sets of the first category
is of first category.
\end{lemma}
The next remark that may already have occurred to the reader.
\begin{exercise} If $(X,d)$ is a complete metric space
and $E$ is a subset of first category, then $X\setminus E$
is dense in $E$.
\end{exercise}
Recall that a point $x$ in a metric space $(X,d)$ is called
\emph{isolated} if we can find a $\delta>0$ such that
$d(x,y)<\delta$ implies $y=x$.
\begin{lemma} A complete metric space without isolated
points is uncountable.
\end{lemma}
Observe that this gives us a new proof that 
${\mathbb R}$ is uncountable (and so transcendental numbers exist)
which does not depend on establishing decimal representation.
\begin{lemma}\label{L;no countable} (i) If $E$ is an infinite dimensional 
Banach space over ${\mathbb F}$, 
then $E$ cannot have a countable spanning set.
In other words, we cannot find a sequence $e_{1}$, $e_{2}$, \dots
in $E$ such that every $u\in E$ can be  written
\[u=\sum_{j=1}^{N}\lambda_{j}e_{j}\]
for some $\lambda_{j}\in{\mathbb F}$ and some $N\geq 1$.

(ii) The space $c_{00}$ cannot be given a complete norm.
\end{lemma}

\begin{exercise} Consider the Banach space $(l^{p},\|.\|_{p})$.
We know that \emph{considered as a set}, $l^{r}$ is a subset
of $l^{p}$ whenever $p\geq r\geq 1$. With this convention
$\bigcup_{p>r}l^{r}$ is of first category in $(l^{p},\|.\|_{p})$.
\end{exercise}
Banach and Steinhauss used the Baire category theorem to isolate
another piece of folk wisdom.
\begin{theorem}{\bf[Principle of uniform boundedness]}
Suppose $(U,\|.\|_{U})$
and $(V,\|.\|)$ are Banach spaces. If we have a family
${\mathcal T}$ of continuous linear maps $T:U\rightarrow V$
such that $\sup_{T\in{\mathcal T}}\|Tu\|_{V}<\infty$
for each $u\in U$, then  $\sup_{T\in{\mathcal T}}\|T\|<\infty$.
\end{theorem}
Here is a typical use of the principle. We work
on the circle ${\mathbb T}={\mathbb R}/{2\pi\mathbb Z}$
(but if the reader prefers she may work on $[-\pi,\pi]$).
To see why this result may be interestiong recall
applied lecturers writing down the following
`aspirational prose'.

\begin{small}\begin{sf}
We have $g_{n}\rightarrow \delta$, that is to say
the  continuous function $g_{n}$ tends to the delta function,
and so
\[\int g_{n}(t)f(t)\,dt\rightarrow\int g_{n}(t)\delta(t)\,dt=f(0).\]
\end{sf}\end{small}

\begin{exercise}\label{E;kernel 1} Suppose that 
$g_{n}:{\mathbb T}\rightarrow{\mathbb R}$
is continuous and
\[\frac{1}{2\pi}\int_{\mathbb T}g_{n}(t)f(t)\,dt 
\rightarrow f(0)\]
as
$n\rightarrow\infty$ for all continuous functions
$f:{\mathbb T}\rightarrow{\mathbb R}$. Then
the following must be true. 

(i) There exists a constant $K$
such that
\[\frac{1}{2\pi}\int_{\mathbb T}|g_{n}(t)|\,dt\leq K\]
for all $n\geq 1$.

(ii) If $\delta>0$ and $f$ is a continuous function
with $f(t)=0$ for $|t|<\delta$, then
\[\frac{1}{2\pi}\int_{\mathbb T}f(t)g_{n}(t)\,dt\rightarrow 0\] 
as $n\rightarrow \infty$.

(iii) We have
\[\frac{1}{2\pi}\int_{\mathbb T}g_{n}(t)\,dt\rightarrow 1\]
as $n\rightarrow\infty$.

\noindent[In Exercise~\ref{E;kernel 2} 
we establish that these necessary conditions 
are also sufficient. In Exercise~\ref{E;kernel 4}
we use Exercise~\ref{E;kernel 1} to establish that the
Fourier series of a continuous function
need not converge pointwise to that function.]
\end{exercise}
We use the Baire category theorem to prove the following series
of rather more subtle results.
\begin{theorem}{\bf [Open mapping theorem]}\label{T;open mapping}
Suppose $(U,\|.\|_{U})$
and $(V,\|.\|)$ are Banach spaces. If $T\in{\mathcal L}(U,V)$
is surjective, then $T$ maps open sets in $U$ to open sets in $V$.
\end{theorem}
\begin{exercise} It is easy to see that linearity is essential
for results like these. Give an example of a continuous
surjective map $f:{\mathbb R}\rightarrow{\mathbb R}$
which is not open.
\end{exercise}

We give an example of the use of the open mapping theorem in
Exercise~\ref{E;equivalent norms} 
The reader may recall a very useful `open mapping theorem'
in complex variable theory. The following is an immediate
consequence of Theorem~\ref{T;open mapping}.
\begin{theorem}{\bf [Inverse mapping theorem]}\label{T:inverse}%
\footnote{Called the \emph{inversion theorem} in the syllabus.}
Suppose that $(U,\|.\|_{U})$
and $(V,\|.\|_{V})$ are Banach spaces. If $T\in{\mathcal L}(U,V)$
is bijective, then $T^{-1}$ is continuous (so $T$ is an isomorphism).
\end{theorem}
(For a variation on this theme, see Exercise~\ref{E;dense surjective}.)

Here is a simple example of the use of Theorem~\ref{T:inverse}.
\begin{exercise}\label{E;see isomorphism} 
The space $c$ of sequences with limits and the space
$c_{0}$ of sequences with limit zero (both equipped with the
supremum norm) are Banach space isomorphic.
\end {exercise}

We introduce the last of this group of theorems with an exercise.
\begin{exercise} (i) Let $(X,d)$ be a metric space.
If $f:X\rightarrow X$ is continuous, then the graph
\[\left\{\big(x,f(x)\big)\,:\,x\in X\right\}\]
is closed with respect to the product metric.

(ii) If $g:{\mathbb R}\rightarrow{\mathbb R}$ is given
by $g(x)=x^{-2}$ for $x\neq 0$ and $g(0)=0$, then
the graph
\[\left\{\big(x,g(x)\big)\,:\,x\in X\right\}\]
is closed in the usual metric but $g$ is not continuous.
\end{exercise}  
\begin{theorem}{\bf [Closed graph theorem]} Suppose $(U,\|.\|_{U})$
is  Banach space and $T:U\rightarrow U$ is a linear function.
If the graph
\[ \left\{\big(u,T(u)\big)\,:\,u\in U\right\}\]
is closed with respect to the product norm, then
$T$ is continuous.
\end{theorem}
To see how such a theorem can be used, we recall some definitions
and results from from 1B algebra. The reader can check that
they apply without change in the infinite dimensional case.
\begin{exercise} If $U$ is a vector space, we say that
a linear map $P:U\rightarrow U$ is a \emph{projection} if
$P^{2}=P$. Show that, for such a $P$,
\[(I-P)^{-1}(0)=P(U)\ \text{and}\ P^{-1}(0)=(I-P)(U).\]
Show further that every $u\in U$ can be written uniquely
in the form $u=v+w$ with $v\in P(U)$ and $w\in P^{-1}(0)$.
\end{exercise}
\begin{theorem}  Suppose $(U,\|.\|_{U})$
is  Banach space and $P:U\rightarrow U$ is a projection.
Then $P$ is continuous if and only if the kernel $P^{-1}(0)$
and image $P(U)$ are closed.
\end{theorem}
\section{Continuous functions} We recall the discussion of continuous
functions in the Topological and Metric Spaces course.
\begin{exercise}\label{E;continuous} Let $(X,\tau)$ 
and $(Y,\sigma)$ be topological
spaces. The following two statements about a function
$f:X\rightarrow Y$ are equivalent.

(i) If $U\in\sigma$, then $f^{-1}(U)\in\tau$.

(ii) Given $x\in X$ and $V\in\sigma$ with $f(x)\in V$,
we can find a $W\in\tau$ with $x\in W$ and $F(W)\subseteq V$.
\end{exercise}  
Any $f$ satisfying the conditions of Exercise~\ref{E;continuous}
is called continuous. We shall be interested 
in continuous functions $F:X\rightarrow{\mathbb F}$
where $(X,\tau)$ is a topological space\footnote{As usual,
we shall sometimes merely refer to $X$
with the topology $\tau$ being understood.} 
and ${\mathbb F}$
has its usual topology. 

Even if the reader has not
seen the next three exercises before, she should have no difficulty
with them.
\begin{exercise} (i) Let $(X,\tau)$ be a topological space
and let $f_{n}:X\rightarrow{\mathbb F}$ be continuous. Suppose
that $f:X\rightarrow{\mathbb F}$ is such that we can find 
$\epsilon_{n}\rightarrow 0$ with
\[|f_{n}(x)-f(x)|<\epsilon_{n}\ \text{for all $x\in X$}.\]
Show that $f$ is continuous. (In other words, the uniform limit
of continuous functions is continuous.)

(ii) Let $C_{0}(X)$ be the space of bounded continuous
functions $f:X\rightarrow{\mathbb F}$.
Show that
\[\|f\|_{\infty}=\sup_{x\in X}|f(x)|\]
defines a complete norm on $C_{0}(X)$.
\end{exercise}
Note that the completeness of $l^{\infty}$ is a special case
where $X={\mathbb N}$ and $\tau$ is the discrete topology.
\begin{exercise} If $(X,\tau)$ is compact, show that every continuous
function $f:X\rightarrow{\mathbb R}$ is bounded.
\end{exercise}
\begin{exercise} Show that, if $E$ is subset of ${\mathbb F}^{n}$
with the usual topology, then every continuous function
$f:E\rightarrow{\mathbb F}^{n}$ is bounded if and only if
$E$ is compact.
\end{exercise}
These results strongly suggest that we should study
the space $C(X)=C_{\mathbb F}(X)$ of continuous functions
$f:X\rightarrow{\mathbb F}$ with the uniform norm
$\|f\|_{\infty}=\sup_{x\in X}|f(x)|$ in the case when $X$ is compact.

However, if we simply demand that $X$ is compact, the 
\emph{space} $C(X)$
may not have much to do with the \emph{set} X.
\begin{exercise} If $X$ has the indiscrete topology
$\tau=\{X,\,\emptyset\}$, then $C(X)$ consists of the constant functions.
\end{exercise}
The following simple observation puts us on a profitable path.
\begin{exercise} If $C(X)$ is such that, given $x\neq y$,
we can find an $f\in C(X)$ with $f(x)\neq f(y)$
(informally, if $C(X)$ \emph{separates} the points of $X$),
then $X$ is Hausdorff.
\end{exercise}
In this section we prove the remarkable fact that the converse
also holds for compact spaces. 
Thus it is natural to study $C(X)$ when $X$
is compact and Hausdorff\footnote{This result so impressed
a retired French general that he proposed using the
word `compact' to mean `compact and Hausdorff'. The innovation
was not popular but the reader should be aware of this possible
source of confusion.}.

We need to recall a couple of elementary topological results.
\begin{exercise}\label{E;Watson} 
(i) In a compact space, every closed set is compact.

(ii) In a Hausdorff space, singleton sets $\{a\}$ are closed.
\end{exercise}
We now start our theorem sequence.
\begin{theorem}\label{T;normal} If $(X,\tau)$ is compact and Hausdorff,
then, given $A$ and $B$ non-empty disjoint closed sets, 
we can find disjoint open sets $U$ and $V$ such that
$A\subseteq U$ and $B\subseteq V$.
\end{theorem}
(A space satisfying the conclusions of Theorem~\ref{T;normal}
is called normal. See Exercise~\ref{E;normal one} for more on this topic.)
\begin{theorem} {\bf[Urysohn's lemma]} If $(X,\tau)$ is
compact and Hausdorff,
then, given $A$ and $B$ non-empty disjoint closed sets, 
we can find an $f\in C_{\mathbb R}(X)$ such that
$0\leq f(x)\leq 1$ for all $x\in X$ and 
\begin{align*}
f(a)&=1\ \text{when $a\in A$}\\
f(b)&=0\ \text{when $b\in B$}.
\end{align*}
\end{theorem}
Exercise~\ref{E;Watson} now tells us that $C(X)$ separates
points whenever $X$ is compact and Hausdorff.
It is, perhaps, worth remarking that Urysohn's lemma
has a much simpler proof if $\tau$ is derived from a metric.


The following simple remark comes in useful in our
proof of Urysohn's lemma.
\begin{exercise} Let $(X,\tau)$ be a topological space
and let ${\mathbb R}$ have its usual topology.
A function $f:X\rightarrow {\mathbb R}$ is continuous
if and only if $f^{-1}\big((-\infty,a)\big)$ is open
and  $f^{-1}\big((-\infty,a]\big)$ is closed for all $a\in{\mathbb R}$.
\end {exercise}


In fact we can prove an apparently stronger result than Urysohn's
lemma.
\begin{theorem} {\bf[Tietze's extension theorem]}\label{T;Tietze}
If $Y$ is closed subset of a compact Hausdorff space
$(X,\tau)$, then, given any $f\in C_{\mathbb R}(Y)$
(where $Y$ has the subspace topology), we can find
an $F\in C_{\mathbb R}(X)$ such that $F(y)=f(y)$ for all $y\in Y$.
\end{theorem}

To see that Tietze's extension theorem is non-trivial
consider the following example.
\begin{exercise} Consider the closed interval
$X=[-4,4]$ with the usual topology
and the open interval $Y=(0,1)$. Show that if $f:Y\rightarrow{\mathbb R}$ 
is defined by $f(y)=\sin(1/y)$ then $f$ is continuous
but  there does not exist an
$F\in C(X)$ such that $F(y)=f(y)$ for all $y\in Y$.
\end{exercise}

We strengthen Theorem~\ref{T;Tietze} in two steps.
\begin{corollary}  If $Y$ is closed subset of a compact Hausdorff space
$(X,\tau)$, then, given any $f\in C_{\mathbb F}(Y)$,
we can find
an $F\in C_{\mathbb F}(X)$ such that $F(y)=f(y)$ for all $y\in Y$.
\end{corollary}
\begin{corollary}  If $Y$ is closed subset of a compact Hausdorff space
$(X,\tau)$, then, given any $f\in C_{\mathbb F}(Y)$,
we can find
an $F\in C_{\mathbb F}(X)$ such that $F(y)=f(y)$ for all $y\in Y$
and $\|F\|_{\infty}=\|f\|_{\infty}$.
\end{corollary}
\section{The Stone--Weierstrass theorem} Unless the reader
has lead a very sheltered life she will have done the following
important exercise many times before.
(If not, she should do it at once.)
\begin{exercise} {\bf [Cauchy's example]}
Let $E(x)=\exp(-1/x^{2})$ for $x\neq 0$
and $E(0)=0$.

(i) Show that $E$ is infinitely differentiable
on ${\mathbb R}\setminus\{0\}$ with
\[E^{(n)}(x)=P_{n}(1/x)E(x)\]
for some polynomial $P_{n}$.

(ii) Show that $E$ is infinitely differentiable
everywhere with $E^{(n)}(0)=0$ for all $n$.

(iii) Use the fact that a power series is infinitely differentiable
term by term to show that we cannot find $a_{j}\in{\mathbb R}$
with $E(x)=\sum_{j=-\infty}^{\infty}a_{j}x^{j}$.
\end{exercise}
(Exercise~\ref{E;smooth non-analytic}, which
uses the Baire category theorem from a later section, 
provides an even stronger result.)
Weierstrass must, therefore, have been delighted to
prove the following result.
\begin{theorem} The set of real polynomials is
uniformly dense in $C_{\mathbb R}([a,b])$.
\end{theorem}
In other words, given any continuous real function
$f:[a,b]\rightarrow{\mathbb R}$ and any $\epsilon>0$,
we can find a real polynomial with 
\[|P(t)-f(t)|<\epsilon\]
for all $t\in [a,b]$.

When Stone was asked to contribute an article to the
first issue of the \emph{American Mathematical Monthly},
he produced the following far reaching extension of Weierstrass's
theorem.
\begin{theorem}{\bf [The Stone--Weierstrass theorem]}
Consider a compact Hausdorff space $X$. Suppose
that $A$ is a subspace of $C_{\mathbb R}(X)$ 
with the following properties.

(i) If $f,\,g\in A$ then $f\times g\in A$.

(ii) $1\in A$.

(iii) If $x,\,y\in X$ then we can find an $f\in A$ 
such that $f(x)\neq f(y)$.

Then $A$ is dense in $(C_{\mathbb R},\|.\|_{\infty})$.
\end{theorem}
(If $A$ is a subspace of $C(X)$ satisfying~(i), we
sometimes say that $A$ is a \emph{subalgebra} of $C(X)$.)

Our proof of the Stone--Weierstrass theorem makes use
of the following fact.
\begin{lemma}\label{L;Taylor for Stone} 
We can find $a_{j}\in{\mathbb R}$ such that
\[(1-x)^{1/2}=\sum_{j=0}^{\infty}a_{j}x^{j}\]
for all real $x$ with $|x|<1$.
\end{lemma}

Our version of the Stone--Weierstrass theorem 
deals with \emph{real valued} functions.
The following example shows that it will not
apply in the complex case without modification.
\begin{example} We work in the complex plane ${\mathbb C}$.
Let
\[\bar{D}=\{z\in{\mathbb C}\,:\,|z|\leq 1\}
\ \text{and}
\ D=\{z\in{\mathbb C}\,:\,|z|<1\}.\]
We write $A(\bar{D})$ for the set of $f\in C(\bar{D})$
such that $f$ is analytic on $D$. Then
$A(\bar{D})$ is a subspace of $C_{\mathbb C}(\bar{D})$ 
with the following properties.

(i) If $f,\,g\in A(\bar{D})$ then $f\times g\in A$.

(ii) $1\in A(\bar{D})$.

(iii) If $z,\,w\in \bar{D}$ then we can find an $f\in A(\bar{D})$ 
such that $f(z)\neq f(w)$.

However, $A(\bar{D})$ is not uniformly dense in $C(\bar{D})$.
\end{example}

Instead we produce the following variation.
\begin{theorem}{\bf [The complex Stone--Weierstrass theorem]}
Consider a compact Hausdorff space $X$. Suppose
that $A$ is a subspace of $C_{\mathbb C}(X)$
with the following properties.

(i) If $f,\,g\in A$, then $f\times g\in A$.

(ii) $1\in A$.

(iii) If $x,\,y\in X$, then we can find an $f\in A$
such that $f(x)\neq f(y)$.

(iv) If $f\in A$, then its complex conjugate $f^{*}\in A$.

Then $A$ is dense in $(C_{\mathbb C}(X),\|.\|_{\infty})$.
\end{theorem}
The following exercise gives a typical application
and clears up matters left vague in the 1B methods course.
\begin{exercise}\label{E; Fourier dense}
We work on the circle
${\mathbb T}={\mathbb R}/2\pi{\mathbb Z}$.
If $f:{\mathbb T}\rightarrow{\mathbb C}$ is continuous
we write
\[\hat{f}(n)=\frac{1}{2\pi}\int_{\mathbb T}f(t)\exp(-int)\,dt.\]

(i) The collection of trigonometric polynomials
$\sum_{j=-n}^{n}a_{j}\exp(ijt)$ is uniformly dense in
$C_{\mathbb C}({\mathbb T})$.

(ii) (Uniqueness of Fourier series.)
If $f,\,g\in C_{\mathbb C}({\mathbb T})$ and
$\hat{f}(n)=\hat{g}(n)$ for all $n\in{\mathbb Z}$,
then $f=g$.

(iii) If $f\in C_{\mathbb C}({\mathbb T})$
and $\sum_{n=-\infty}^{\infty}|\hat{f}(n)|$ converges,
then
\[f(t)=\sum_{n=-\infty}^{\infty}\hat{f}(n)\exp(int)\]
for all $t\in{\mathbb T}$.
\end{exercise}
Exercise~\ref{E;Fubini} gives another example
of Stone--Weierstrass in action.
\section{Ascoli--Arzel{\`a}}
It is frequently possible to show that a problem can be solved
`apart from an error which can be made as small as we like'.
Under these circumstances an appeal to compactness,
if available, will often show that the problem has an exact solution.

The Ascoli--Arzel{\`a} theorem enables us to characterise
the compact subsets of $C(X)$ when $X$ is a 
compact metric space.
\begin{definition} Let $(X,\tau)$ be a topological space
and $(Y,\rho)$ a metric space. We say that
a collection ${\mathcal F}$ of functions $f:X\rightarrow Y$
is \emph{equicontinuous} at $x$ if given $\epsilon>0$
we can find a $U\in\tau$ with $x\in U$ such that
\[y\in U
\ \text{implies $\rho(f(x),f(y))<\epsilon$
for all $f\in{\mathcal F}$}.\]

If  ${\mathcal F}$
is equicontinuous at all points of $X$ we say that
${\mathcal F}$ is equicontinuous.
\end{definition}
\begin{exercise} If $(X,d)$ and $(Y,\rho)$ are metric spaces,
write out the definition of equicontinuity in $\epsilon$,
$\delta$ form.
\end{exercise}
\begin{theorem}{\bf [Ascoli--Arzel{\`a}]}\label{T;Ascoli}
Let $(X,\tau)$ be a compact Hausdorff space.
Then a subset ${\mathcal F}$ of $C(X)$ is compact
under the the uniform norm if and only if
both the following conditions hold.

(i) ${\mathcal F}$ is closed and bounded in the uniform norm.

(ii) ${\mathcal F}$ is equicontinuous.
\end{theorem}
We shall prove the Ascoli--Arzel{\`a} theorem by a
direct attack. A cleaner proof depending on results
from the Metric and Topological course is given in
Exercise~\ref{E;alternate Ascoli} but the basic ideas
of the two proofs are the same.  

A typical example of the use of these ideas appears in the proof
of the following nice result.
\begin{theorem}\label{T;differential equation} 
If $\eta>0$ and
$f:[x_{0}-\eta,x_{0}+\eta]\times[y_{0}-\eta,y_{0}+\eta]
\rightarrow{\mathbb R}$
is continuous, then  we can find
a $\delta$ with $\eta\geq\delta>0$
and a differentiable
function
\[\phi:(x_{0}-\delta,x_{0}+\delta)\rightarrow{\mathbb R}\]
such that $\phi(x_{0})=y_{0}$ and
\[\phi'(t)=f(t,\phi(t))\]
for all $t\in(x_{0}-\delta,x_{0}+\delta)$.
\end{theorem}
In Part~1B we used the contraction mapping theorem
(another idea from `abstract analysis') to prove the following theorem.
\begin{theorem}\label{T;Picard} If $\eta>0$, $K>0$ and
$f:[x_{0}-\eta,x_{0}+\eta]\times[y_{0}-\eta,y_{0}+\eta]
\rightarrow{\mathbb R}$ satisfies
the Lipschitz condition.
\[|f(x,y)-f(x',y)|\leq K|x-x'|\]
for all $x,\,x'\in  [x_{0}-\eta,x_{0}+\eta]$
and all $y\in [y_{0}-\eta,y_{0}+\eta]$,
then we can find
a $\delta$ with $\eta\geq\delta>0$
and a unique differentiable
function
\[\phi:(x_{0}-\delta,x_{0}+\delta)\rightarrow{\mathbb R}\]
such that $\phi(x_{0})=y_{0}$ and
\[\phi'(t)=f(t,\phi(t))\]
for all $t\in(x_{0}-\delta,x_{0}+\delta)$.
\end{theorem}
\begin{exercise} By using the mean value theorem, establish that
if 
\[f:[x_{0}-\eta,x_{0}+\eta]\times[y_{0}-\eta,y_{0}+\eta]
\rightarrow{\mathbb R}\]
has continuous first partial
derivative $\partial f(x,y)/\partial x$, then it
satisfies a  Lipschitz condition.
\end{exercise}
Our new theorem establishes existence
under much more general conditions 
that those of Theorem~\ref{T;Picard},
but the solution need not be unique.
\begin{exercise} The differential equation $x'(t)=3x(t)^{2/3}$
has more than one solution with $x_{0}=0$.
\end{exercise}
It is helpful, when considering the form of our proof for
Theorem~\ref{T;differential equation}, to observe that
if there are different solutions of the equations
then a series of `approximate solutions' may switch between
approximating one solution and another.

Even in the Lipschitz case we cannot hope to prove more than the existence
of \emph{local} solutions since no \emph{global} solution may exist.
\begin{exercise} Find all the solutions of $x'(t)=(1+x(t)^{2})$.
Observe that there is no solution which is valid over
an interval of length greater than $\pi$.
\end{exercise}   
\section{Inner product spaces} Since the reader's first arrival 
in Cambridge she has been bombarded with inner product spaces.
In this section we recall some of the results she already knows.
She should check where appropriate that the results hold
in infinite dimensional spaces.
\begin{definition} Let $V$ be a vector space over ${\mathbb C}$.
Suppose that exists a map $p:V^{2}\rightarrow{\mathbb C}$
such that, writing $\langle u,v\rangle=p(u,v)$
we have

(i) $\langle\lambda_{1} u_{1}+\lambda_{2} u_{2},v\rangle
= \lambda_{1}\langle u_{1},v\rangle
+\lambda_{2}\langle u_{2},v\rangle$
for all $\lambda_{1},\,\lambda_{2}\in{\mathbb C}$,
$u_{1},\,u_{2},\,v\in V$.

(ii) $\langle u,v\rangle=\langle v,u\rangle^{*}$
for all $u,\,v\in V$.

(iii) $\langle u,u\rangle\geq 0$ for all $u\in V$.

(iv) $\langle u,u\rangle=0$ implies $u=0$.

Then we say that $(V,p)$ is an inner product space. 
We call $p$ an inner product.
\end{definition}
A similar definition applies with ${\mathbb C}$ replaced by ${\mathbb R}$
except that the complex conjugation in
condition~(ii) is superfluous.
\begin{exercise}\label{E;inner product norm} 
(i) (Cauchy--Schwarz)
If $V$ is an inner product space
then
\[|\langle u,v\rangle|^{2}\leq \langle u,u\rangle\langle v,v\rangle\]
with equality if and only if $u$ and $v$ are linearly dependent.

(ii) If $V$ is an inner product space then
\[\|u\|_{2}^{2}=\langle u,u\rangle,\ \|u\|_{2}\geq 0\]
defines a norm on $V$.

(iii) (Parallelogram law)
With the notation of~(ii)
\[\|u+v\|_{2}^{2}+\|u-v\|_{2}^{2}=2\big(\|u\|_{2}^{2}+\|v\|_{2}^{2}).\]
\end{exercise}

We derived the norm from the inner product but the process can be reversed
and we can
recover the inner product from the norm.
\begin{exercise} {\bf [The polarisation identity]}\label{E;polarisation}
With the notation and assumptions
of Exercise~\ref{E;inner product norm},
\[4\langle u,v\rangle=\|u+v\|_{2}^{2}-\|u-v\|_{2}^{2}
+i(\|u+iv\|_{2}^{2}-\|u-iv\|_{2}^{2})\]
for all $u\,\,v\in V$.
\end{exercise}
(For an interesting sidelight see Exercise~\ref{E;parallelepiped}.)
\begin{definition} Let $V$ be an inner product space.

(i) If $u,\,v\in V$ and $\langle u,v\rangle=0$
we say that $u$ and $v$ are orthogonal
and write $u\perp v$.

(ii) A collection $E$ of vectors is said to be orthonormal
if, whenever $e,\,f\in E$
\[\langle e,f\rangle
=\begin{cases}
0&\text{if $e\neq f$,}\\
1&\text{if $e=f$.}
\end{cases}\]
\end{definition}
We have the following extensions of Pythagoras's theorem.
\begin{exercise} Consider an inner product space $V$.
Suppose $e_{1},\,e_{2},\,\dots,\,e_{n}$ are orthonormal
vectors in $V$ and $f\in V$. Then
\[\|f-\sum_{j=1}^{n}\lambda_{j}e_{j}\|_{2}^{2}
\geq \|f\|_{2}^{2}-\sum_{j=1}^{n}|\langle f,e_{j}\rangle|^{2}\]
with equality if and only if $\lambda_{j}=\langle f,e_{j}\rangle$.
\end{exercise}
\begin{theorem} {\bf [Bessel's inequality]}
Consider an inner product space $V$.
Suppose $e_{1},\,e_{2},\,\dots$ is an orthonormal
sequence of vectors in $V$ and $f\in V$. Then

\[
\sum_{j=1}^{\infty}|\langle f,e_{j}\rangle|^{2}
\leq \|f\|_{2}^{2}\]
with equality if and only if
\[\left\|f-\sum_{j=1}^{N}\langle f,e_{j}\rangle e_{j}\right\|_{2}
\rightarrow 0\]
as $n\rightarrow\infty$.
\end{theorem}
We illustrate these familiar general results with
a familiar special case. Note that Exercise~\ref{E;inner continuous}~(iii)
resolves a problem left open by the 1B mathematical methods course.
\begin{exercise}\label{E;inner continuous} 
We work on ${\mathbb T}={\mathbb R}/{2\pi\mathbb Z}$.

(i) Show that if $f\in C_{\mathbb R}({\mathbb T})$, $f(t)\geq 0$
for all $t$ and 
\[\frac{1}{2\pi}\int_{\mathbb T}f(t)\,dt=0,\]
then $f(t)=0$ for all $t$.

(ii) Show that the formula
\[\langle f,g\rangle=\frac{1}{2\pi}\int_{\mathbb T}f(t)g(t)^{*}\,dt=0\]
defines an inner product on $C_{\mathbb C}({\mathbb T})$.
From now on we consider $C_{\mathbb C}({\mathbb T})$
with this inner product.

(iii) Show that, if we write $e_{j}(t)=\exp ijt$, then the $e_{j}$
are orthonormal. By using the fact that the trigonometric polynomials
are dense in $(C_{\mathbb C}({\mathbb T}),\|.\|_{\infty})$
(Exercise~\ref{E; Fourier dense}), show that
the trigonometric polynomials
are dense in $(C_{\mathbb C}({\mathbb T}),\|.\|_{2})$.
Hence show that
\[\left\|f-\sum_{j=-M}^{N}\langle f,e_{j}\rangle e_{j}\right\|_{2}
\rightarrow 0\]
as $M,N\rightarrow\infty$.

(iv) (Parseval's formula) Use~(iii) to
show that, if we write $\hat{f}(j)=\langle f,e_{j}\rangle$,
then
\[\sum_{j=-\infty}^{\infty}|\hat{f}(n)|^{2}=
\frac{1}{2\pi}\int_{\mathbb T}|f(t)|^{2}\,dt\]
for all $f\in C_{\mathbb C}({\mathbb T})$. Show also that
\[\sum_{j=-\infty}^{\infty}\hat{f}(n)\hat{g}(n)^{*}=
\frac{1}{2\pi}\int_{\mathbb T}f(t)g(t)^{*}\,dt.\]
\end{exercise}
However, we note the following important fact.
\begin{exercise}\label{E;continuous square not complete}
$(C({\mathbb T}),\|.\|_{2})$ is not complete.
\end{exercise}
This result needs careful proof. We need to show, not
that Cauchy sequence does not converge to the \emph{obvious} answer,
but that it does not converge to \emph{any} continuous function.
The next exercise illustrates this remark.
\begin{exercise} Write
\[\Delta_{n}(t)=
\begin{cases}
1-2^{n}|t|&\text{for $|t|\leq 2^{-n}$,}\\
0&\text{otherwise.}
\end{cases}\]
Show that,if we define $f_{n}\in C({\mathbb T})$ by
\[f_{n}=\sum_{j=1}^{n}n\Delta_{n}(t-2\pi r/n),\]
then $\|f\|_{2}\rightarrow 0$.
\end{exercise}
\section{Hilbert space}  The work of this section depends 
on the following key result.
\begin{theorem} Let $V$ be an infinite inner product space. The
following statements are equivalent.

(i) $V$ is separable.

(ii) There exists an orthonormal sequence $e_{j}$
such that
\[\left\|f-\sum_{j=1}^{n}\langle f,e_{j}\rangle e_{j}\right\|_{2}
\rightarrow 0\]
as $n\rightarrow\infty$ for all $f\in V$.
\end{theorem}
Our proof calls on an old friend, the Gramm--Schmidt
orthogonalisation process.
\begin{exercise} Suppose $V$ is an inner product space.
If $e_{1},\,e_{2},\,\dots,\,e_{n}$ are orthonormal
and $f\in V$ then either

(i) $f=\sum_{j}^{n}\langle f,e_{j}\rangle e_{j}$
and $f\in\Span\{e_{1},\,e_{2},\,\dots,\,e_{n}\}$, or

(ii) $f\neq\sum_{j}^{n}\langle f,e_{j}\rangle e_{j}$
in which case $f\notin \Span\{e_{1},\,e_{2},\,\dots,\,e_{n}\}$.
In this case, setting
\[u=f-\sum_{j}^{n}\langle f,e_{j}\rangle e_{j}\]
and $e_{n+1}=\|u\|_{2}^{-1}u$, we have
$e_{1},\,e_{2},\,\dots,\,e_{n+1}$ orthonormal and
\[\Span\{e_{1},\,e_{2},\,\dots,\,e_{n+1}\}
=\Span\{e_{1},\,e_{2},\,\dots,\,f\}.\]
\end{exercise}
From now on, if
\[\left\|f-\sum_{j=1}^{n} f_{j}\right\|_{2}
\rightarrow 0,\]
we feel free to write
\[f=\sum_{j=1}^{\infty} f_{j}.\] 
\begin{exercise} (Uniqueness)
Let $V$ be an infinite inner product space. If
we have an orthonormal sequence $e_{j}$, then,
if $\lambda_{j}\in{\mathbb F}$,
\[\sum_{j=1}^{\infty}\lambda_{j}e_{j}=0\]
implies $\lambda_{j}=0$ for all $j$.

\noindent[Note this is result of analysis and not
of algebra since it involves limits.]
\end{exercise}
\begin{definition} If $U$ is an inner product space,
we say that an orthonormal sequence $e_{j}$ in $U$ is a 
basis\footnote{NB This is not an \emph{algebraic} basis.}
(or more exactly an orthonormal basis)
for $U$ if 
\[x= \sum_{j=1}^{\infty}\langle x,e_{j}\rangle e_{j}\]
for all $x\in U$. 
\end{definition}

We immediately obtain the following remarkable result.
\begin{theorem} {\bf [Riesz--Fisher]} All separable
complete infinite dimensional inner product spaces are inner
product isomorphic. 
More precisely, if $U$ and $V$
are separable complete infinite dimensional
inner product spaces with inner products $p_{U}$ and $p_{V}$,
then there exists a linear map $T:U\rightarrow V$
such that
\[p_{V}(Tx,Ty)=p_{U}(x,y)\]
for all $x,\,y\in U$. We note that $T$ is automatically an
isometric Banach space isomorphism.
\end{theorem}
Since all separable
complete infinite dimensional inner product spaces
are isomorphic, we simply talk about the 
Hilbert\footnote{Hilbert developed the theory 
$H$ in a non-abstract way for particular purposes. 
There is a, no doubt
apocryphal, story of his asking `What is this Hilbert space
which the young people are talking about?'.} space $H$.
Sometimes people talk about complete inner product
spaces which are not separable and are then careful
to talk about `non-separable Hilbert spaces' but
the study of such large 
spaces has not yet been very profitable. 
(If you want to see such a space, 
consult Exercise~\ref{E;non-separable Hilbert}.)
 
Our arguments also give the following results
more or less for free.
\begin{exercise} Consider $l^{2}$. If ${\mathbf a},\,{\mathbf b}\in l^{2}$
then $\sum_{j=1}^{\infty}a_{j}b_{j}^{*}$ is 
absolutely convergent. Further
\[\langle {\mathbf a},\,{\mathbf b}\rangle=\sum_{j=1}^{\infty}a_{j}b_{j}^{*}\]
defines an inner product which induces the norm $\|.\|_{2}$.
With this inner product, $l^{2}$ is (inner product isomorphic to)
Hilbert space.
\end{exercise}
\begin{exercise}\label{E;completion} Let $U$ be a
separable infinite dimensional
inner product space.
Then there exists an inner product preserving 
linear map
$J:U\rightarrow H$ of $U$ into the Hilbert space $H$
such that $J(U)$ is dense in $H$.
\end{exercise}
If the reader knows about such things, she will be able to
restate Exercise~\ref{E;completion} as the observation that
the completion of a separable infinite dimensional
inner product space is (inner product isomorphic to)
Hilbert space. 
\begin{lemma} If $U$ is an inner product space,
with basis $e_{j}$,
then $U$ is complete if and only if
\[\sum_{j=1}^{\infty}x_{j}e_{j}\]
converges whenever $\sum_{j=1}^{\infty}|x_{j}|^{2}$ converges.
\end{lemma}
\section{The dual of Hilbert space} We already know that
Hilbert space is isometrically isomorphic to $l^{2}$
and we know that $l^{2}$ has dual space isometrically isomorphic  
to itself. Thus the dual space of Hilbert space is isometrically
isomorphic to itself. 

However, Hilbert space is the 
infinite dimensional space in which our geometrical
intuition has freest play and it is instructive to follow a geometric
path to a closely related result. Not only 
does this avoid the inelegant use
of specific bases, but it provides additional insight into
the structure of Hilbert space\footnote{Since we do not
use bases, our results will also apply to non-seperable
Hilbert spaces but the reader may ignore this.}..
\begin{theorem}\label{T;closest Hilbert} 
Let $U$ be a complete inner product space.
If $F$ is closed subspace and $a\in U$, then we can find a unique
$f_{0}\in F$ such that
\[\|a-f_{0}\|_{2}\leq\|a-f\|_{2}\]
for all $f\in F$.
\end{theorem}
(See also Exercises~\ref{E;closest convex}
and~\ref{E;not closed closest}.)
\begin{lemma}\label{L;closest orthogonal} With the hypotheses and notation of 
Theorem~\ref{T;closest Hilbert}, $f_{0}\in F$ is the unique
element of $F$ such that $a-f_{0}$ is orthogonal to every element
of $F$.
\end{lemma}
We immediately deduce the following pleasing result. 
\begin{theorem}{\bf [Riesz representation]}\label{T;Riesz representation}
If
$U$ is a complete inner product space and
$T\in U'$ (that is to say, $T:U\rightarrow{\mathbb F}$
is a continuous linear map), then there is a unique
$w\in U$ with
\[Tu=\langle u,w\rangle\]
for all $u\in U$.
\end{theorem}
\begin{exercise} If
$U$ is a complete inner product space and we define
by
\[J(v)u=\langle u,v\rangle\]
for all $u,\,v\in U$, then $J(v)\in U'$ for all $v\in U$
and $J$ has the following properties.

(i) $J(\lambda_{1}v_{1}+\lambda_{2}v_{2})=
\lambda_{1}^{*}J(v_{1})+\lambda_{2}^{*}J(v_{2})$
for all $\lambda_{1},\,\lambda_{2}\in{\mathbb C}$
and all $v_{1},\,v_{2}\in V$. (We say that $J$
is \emph{anti-linear}.)

(ii) $\|J(v)\|=\|v\|_{2}$ for all $v\in V$.

(iii) $J$ is surjective.
\end{exercise}
Thus (using the polarisation identity of 
Exercise~\ref{E;polarisation}) $J:U\rightarrow U'$
is an inner product \emph{anti-isomorphism} 
and $U'$ is naturally anti-isomorphic to $U$.
(If the reader is interested, but only if she is interested,
she may glance at Exercise~\ref{E;anti}.)

Theorem~\ref{T;closest Hilbert} 
and Lemma~\ref{L;closest orthogonal}
also give us information on orthogonal complements
which will be used later.
\begin{lemma} If $F$ is a closed subspace of a Hilbert space $H$,
then 
\[F^{\perp}=
\{g\in H\,:\,\langle g,f\rangle=0\ \text{for all $f\in F$}\}\]
is a closed subspace of $H$. Every $u\in H$ can be written in one
and only one way as
\[u=f+g\]
with $f\in F$ and $g\in F^{\perp}$.
\end{lemma} 
\section{The spectrum} When we studied linear maps
$\alpha:{\mathbb C}^{n}\rightarrow{\mathbb C}^{n}$,
we were particularly interested in those $\lambda\in{\mathbb C}$
such that $\alpha-\lambda\iota$ was not invertible. This interest
carries over to infinite dimensional spaces\footnote{Even in the finite
dimensional case, the study of such things in real vector spaces
turned out to be less interesting, so we shall stick to
complex Banach spaces.}. The elementary theory is no harder
for general Banach spaces than for Hilbert 
spaces\footnote{However
is only true for the elementary theory.} so we shall
work in the general context.

\begin{definition} If $U$ is a Banach space over ${\mathbb C}$
and $T:U\rightarrow U$ is a continuous linear map, we define
the spectrum $\sigma(T)$ of $T$ by
\[\sigma(T)=\{\lambda\in{\mathbb C}\,:\,T-\lambda I
\ \text{not invertible}\}.\]
\end{definition}
The inverse mapping theorem shows that, if $\lambda\notin \sigma(T)$,
then $(T-\lambda I)^{-1}$ is a continuous linear map.
The structure of the spectrum can be exceedingly intricate
but some useful general results can be obtained by applying
the following simple `master theorem'.
\begin{theorem} If $U$ is a Banach space over ${\mathbb C}$
and $T:U\rightarrow U$ is a continuous linear map
with $\|T\|<1$, then $\sum_{j=0}^{\infty}T^{j}$ converges
in the uniform norm and $I-T$ is invertible with
\[(I-T)^{-1}=\sum_{j=0}^{\infty}T^{j}.\]
\end{theorem}
 
\begin{lemma}  If $U$ is a Banach space over ${\mathbb C}$
and $T:U\rightarrow U$ is a continuous linear map, then
$\sigma(T)$ is bounded.
\end{lemma}
\begin{lemma}  If $U$ is a Banach space over ${\mathbb C}$
and $T:U\rightarrow U$ is a continuous linear map,
then $\sigma(T)$ is closed.
\end{lemma}
\begin{definition} If $U$ is a Banach space over ${\mathbb C}$
and $T:U\rightarrow U$ is a continuous linear map, we say
that $\lambda$ is an \emph{eigenvalue} of $T$ if
\[\ker(T-\lambda I)\neq \{0\}.\]
If $u$ is a non-zero element of $\ker(T-\lambda I)$,
we call $u$ an \emph{eigenvector} with 
associated eigenvalue $\lambda$.
\end{definition}
\begin{exercise} With the notation just introduced,
every eigenvalue of $T$ lies in $\sigma(T)$.
\end{exercise}
\begin{example} (i) If $K$ is a non-empty closed bounded set in
${\mathbb C}$, then we can find a continuous linear map
$T:l^{2}\rightarrow l^{2}$ with $\sigma(T)=K$.

(ii) We can find a continuous linear map
$T:l^{2}\rightarrow l^{2}$ such that $\sigma(T)=\{0\}$
but $0$ is not an eigenvalue.
\end{example}
Recall that every Banach space we have studied has
a sufficiently rich dual in the sense of 
Definition~\ref{D;sufficiently rich}.
\begin{lemma}\label{L;non-empty spectrum} 
If $U$ is a Banach space over ${\mathbb C}$
with sufficiently rich dual
and $T:U\rightarrow U$ is a continuous linear map,
then $\sigma(T)$ is non-empty.
\end{lemma}
If we were prepared to develop complex analysis for ${\mathcal L}(U,U)$
valued functions from scratch, we could replace
Lemma~\ref{L;non-empty spectrum}
by a stronger and simpler result.
\begin{lemma} 
If $U$ is a Banach space over ${\mathbb C}$
and $T:U\rightarrow U$ is a continuous linear map,
then $\sigma(T)$ is non-empty.
\end{lemma}
\section{Self-adjoint compact operators on Hilbert space}
In the previous section we developed the elementary theory
of the spectrum for general Banach spaces. From now on we
are only interested in Hilbert space.

The reader will recall the very pretty theory of diagonalisation
for self-adjoint (that is to say, Hermitian)
maps $\alpha:V\rightarrow V$ on finite dimensional
inner product spaces. We conclude this course by developing a parallel
theory for Hilbert space. We need two definitions of which
only the second is really new.
\begin{definition} Let $H$ be a Hilbert space. A continuous linear
map $T:H\rightarrow H$ is called \emph{self-adjoint}
(or \emph{Hermitian}) if
\[\langle Tx,y\rangle=\langle x,Ty\rangle\]
for all $x,\,y\in H$.
\end{definition}
\begin{exercise} The eigenvalues of a self-adjoint 
continuous linear map are real.
\end{exercise}
\begin{definition}\label{D;compact} 
Let $H$ be a Hilbert space. A continuous linear
map $T:H\rightarrow H$ is called \emph{compact} if 
$\Cl(T(B))$,
the closure of the image under $T$ of the unit ball
$B=\{x\,:\,\|x\|\leq 1\}$, is compact.
\end{definition}
\begin{exercise}  Let $H$ be a Hilbert space. 
Show that a continuous linear
map $T:H\rightarrow H$ is compact, if given any $x_{n}\in H$
with $\|x_{n}\|\leq 1$,
we can find $n(j)\rightarrow\infty$ and a $y\in H$ such that
\[\|Tx_{n(j)}-y\|_{2}\rightarrow 0.\]
\end{exercise}
Exercise~\ref{E;compact as limit} gives some insight into 
what the compact operators\footnote{A continuous linear
map $T:H\rightarrow H$ is called an operator.} look like.
 
We state the theorem which we wish to prove.
\begin{theorem}{\bf [The spectral theorem]}\label{T;spectral}
Let $H$ be a Hilbert space. If
$T:H\rightarrow H$ is a continuous linear compact self-adjoint map,
we can find an orthonormal basis $e_{n}$ of eigenvectors
whose associated eigenvalues $\lambda_{n}$ are real and
satisfy the condition
$\lambda_{n}\rightarrow 0$ as $n\rightarrow\infty$.
\end{theorem}
\begin{exercise} Show that the following is an equivalent
statement of Theorem~\ref{T;spectral}.

Let $H$ be a Hilbert space. If
$T:H\rightarrow H$ is a continuous linear compact self-adjoint map
we can find an orthonormal basis $e_{n}$ and a sequence $\lambda_{n}$
of real numbers with $\lambda_{n}\rightarrow 0$ such that
\[Tu=\sum_{j=1}^{\infty}\lambda_{j}\langle u,e_{j}\rangle e_{j}\]
for all $u\in H$.
\end{exercise}
We give yet another equivalent form in 
Exercise~\ref{E;spectral projection}.
\begin{exercise} Let $H$ be a Hilbert space.
with orthonormal basis $e_{n}$. If $\lambda_{n}$ is a sequence
of real numbers with $\lambda_{n}\rightarrow 0$, then
the equation
\[Tu=\sum_{j=1}^{\infty}\lambda_{j}\langle u,e_{j}\rangle e_{j}\]
defines a  continuous linear compact self-adjoint map
$T:H\rightarrow H$.
\end{exercise}
The proof of Theorem~\ref{T;spectral} parallels its finite dimensional
analogue but additional work is required. For the rest of the section
we work in a Hilbert space $H$ and `$T$ is an operator'
will mean that $T:H\rightarrow H$ is a continuous linear map.
\begin{lemma} If $T$ is a self-adjoint operator, then
\[\sup_{\|x\|_{2}=1}|\langle x,Tx\rangle|=\|T\|.\]
\end{lemma}
\begin{lemma} If $T$ is a compact self-adjoint operator, then
at least one of $\|T\|$ or $-\|T\|$ is an eigenvalue.
\end{lemma}
The next result recalls 1B Linear Algebra.
\begin{exercise} If $T$ is a self-adjoint operator
and $e$ is an eigenvector for $T$, then, writing
\[e^{\perp}=\{f\in H\,:\,\langle e,f\rangle=0\},\]
we know that $T(e^{\perp})\subseteq e^{\perp}$.
The map $T|_{e^{\perp}}:e^{\perp}\rightarrow e^{\perp}$
is self-adjoint and, if $T$ is compact, so is $T|_{e^{\perp}}$.
\end{exercise}
\begin{lemma} If $T$ is a compact operator then, given any
$\epsilon>0$, $T$ has only finitely many
orthonormal eigenvectors with associated eigenvalues
having absolute values greater than $\epsilon$.
\end{lemma}
Putting these these results together we obtain
the spectral theorem for compact self-adjoint operators.
\begin{exercise} Suppose that $T$ is a compact self-adjoint
(ie Hermitian) operator.
Consider the following properties which $T$ may or may not have.

(A) $T^{-1}(0)=\{0\}$.

(B) $T^{-1}(0)$ has dimension $r$ for some $r\geq 1$.

(C) $T^{-1}(0)$ has infinite dimension.

(a) $T$ has infinitely many eigenvalues.

(b) $T$ has $s$ eigenvalues for some $s\geq 1$.

Which of the pairs ($X$,$y$) can be true of $T$ and which cannot?
Give reasons or examples.
\end{exercise}
This completes the course, but I have added two extra sections.
The first is an extended exercise on the use of the spectral
theorem which is strongly recommended to the reader.
\section{Using the spectral theorem}\label{S;use spectral}
In mathematical methods you studied
Sturm--Liouville equations
\[\frac{d\ }{dt}\biggl(p(t)y'(t)\biggr)+q(t)y(t)=f(t)\]
on an interval $[a,b]$ subject to conditions
\[A_{1}y(a)+A_{2}y'(a)=0,\ B_{1}y(b)+B_{2}y'(b)=0\]
with $(A_{1},A_{2})\neq(0,0)$, $(B_{1},B_{2})\neq(0,0)$,
$p$ continuously differentiable, $f$, $q$ continuous and $p(t)>0$
for all $t\in[a,b]$. You showed that it is generally possible
to find a continuous \emph{Green's function} 
$G:[a,b]^{2}\rightarrow{\mathbb R}$ 
with $G(s,t)=G(t,s)$ such that
\[y(t)=\int_{a}^{b}G(s,t)f(s)\,ds\]
solves the given problem.

We shall not into the details here. (They are in~\cite{Pryce}~\S19
and in~\cite{Young}.)
The next exercise gives a particular case.
\begin{exercise}\label{E;Green sine}\label{D1} 
(i) If $G:{\mathbb R}^{2}\rightarrow{\mathbb R}$
is differentiable and $g(t)=G(t,t)$, write down $g'(t)$.

(ii) By using the fundamental theorem of the calculus
and differentiation under the integral show, that under
conditions on $F$ that you should specify,
\[\frac{d\ }{dt}\int_{a}^{t}F(s,t)\,ds
=F(t,t)+\int_{a}^{t}\frac{\partial F}{\partial t}(s,t)\,ds.\]

(iii) Show that, if
\[G(s,t)=
\begin{cases}(1-s)t&\text{if $1\geq t\geq s\geq 0$},\\
s(1-t)&\text{if $1\geq s>t\geq 0$},
\end{cases}\]
then, if $f:[0,1]\rightarrow{\mathbb R}$ is continuous,
\[y(t)=\int_{0}^{1}f(s)G(s,t)\,ds\]
defines a twice differentiable function with $y(0)=y(1)=0$ and
\[y''(t)=f(t)\]
for $t\in[0,1]$.
\end{exercise}

We now investigate the equation
\[y(t)=\int_{a}^{b}G(s,t)f(s)\,ds\]
using the methods of linear analysis.
\begin{exercise}\label{D2} Suppose that $G:[a,b]^{2}\rightarrow{\mathbb R}$
is continuous. Show that, if $f:[a,b]\rightarrow{\mathbb C}$
is continuous, then $Lf:[a,b]\rightarrow{\mathbb C}$ given by
\[Lf(t)=\int_{a}^{b}G(s,t)f(s)\,ds\]
is continuous.
\end{exercise}
\begin{exercise}\label{D3} (This is a reprise
of parts Exercises~\ref{E;inner continuous}
and~\ref{E;continuous square not complete}.)
Show that, if we set
\[\langle f,g\rangle=\int_{a}^{b}f(t)g(t)^{*}\,dt,\]
we obtain $C([a,b])$ as an inner product space.
Show that $C([a,b])$ is an infinite dimensional
separable inner product space but is not complete.
\end{exercise}


\begin{exercise}\label{D4} 
We consider $C([a,b])$ both with the
uniform norm $\|.\|_{\infty}$ and the inner product derived
norm $\|.\|_{2}$. We shall use the Cauchy--Schwarz inequality
for integrals repeatedly.
 
(i) Show that
\[L:(C([a,b]),\|.\|_{2})\rightarrow (C([a,b]),\|.\|_{\infty})\]
is a continuous linear map.

(ii) Show that the collection of $Lf$ such that $f\in C([a,b])$
and $\|f\|_{2}\leq 1$ is equicontinuous.

(iii) Show (Exercise~\ref{E;Fubini} is relevant) that,
if $G(s,t)=G(t,s)$ for all $t,\,s\in[a,b]$, then
\[\langle Lf,g\rangle=\langle f,Lg\rangle\]
for all $f,\,g\in (C([a,b])$.
\end{exercise}
We know that $C([a,b])$ is not a complete inner product space, so we cannot
apply the spectral theorem directly. However,
Exercise~\ref{E;completion} tells us that 
there exists an inner product preserving 
linear map
$J:C([a,b])\rightarrow H$ of $U$ into the Hilbert space $H$
such that $J(C([a,b]))$ is dense in $H$.
\begin{exercise}\label{D5} The results of this exercise are not hard
but the reader should not sleep walk through them.

(i) Show that, if $u\in H$, $u_{n}\in C([a,b])$ and 
$\|Ju_{n}-u\|_{2}\rightarrow 0$, then $Lu_{n}$ converges
uniformly in $C([a,b])$ to a continuous function $g$ say.
Show that if $v_{n}\in C([a,b])$ and 
$\|Jv_{n}-u\|_{2}\rightarrow 0$, then
\[\|Lv_{n}-g\|_{\infty}\rightarrow 0.\]
Thus we can write $\tilde{L}u=g$.

(ii) Show that $\tilde{L}$ is a well defined function
$\tilde{L}:H\rightarrow C([a,b])$.

(iii) Show that
\[\tilde{L}:H\rightarrow (C([a,b]),\|.\|_{\infty})\]
is a continuous linear map.

(iv) Show that the collection of $\tilde{L}f$ such that $f\in H$
and $\|f\|_{2}\leq 1$ is equicontinuous.
\end{exercise} 
\begin{exercise}\label{D6} We now define $\breve{L}=J\tilde{L}$.

(i) Show that
\[\breve{L}:H\rightarrow H.\]
is a continuous linear map.

(ii) Show that $\breve{L}$ is compact.

(iii) From now on we suppose $G(s,t)=G(t,s)$ for all $s,\,t\in[a,b]$.
Show that $\breve{L}$ is self-adjoint.

(iv) Deduce that 
we can find an orthonormal basis $w_{n}$ and a sequence $\lambda_{n}$
of real numbers with $\lambda_{n}\rightarrow 0$ such that
\[\breve{L}u=\sum_{j=1}^{\infty}\lambda_{j}\langle u,w_{j}\rangle w_{j}\]
for all $u\in H$.
\end{exercise}
The result of the previous exercise tells us something about $\breve{L}$,
which is an operator on $H$, and we are interested in $L$, 
which is an operator
on $C([a,b])$. However, this is soon remedied.
\begin{exercise}\label{D7} (i) If $\lambda_{j}\neq 0$, use the fact that
$\lambda_{j}w_{j}=\breve{L}w_{j}$ to show that $w_{j}=Je_{j}$
for some $e_{j}\in C([a,b])$.

(ii) Conclude that, if $G:[a,b]^{2}\rightarrow{\mathbb R}$
is continuous and $G(s,t)=G(t,s)$, then 
\emph{either} we can find an orthonormal
sequence $v_{j}$ in $C([a,b])$ and a sequence $\zeta_{j}$
of non-zero real numbers with $\zeta_{j}\rightarrow 0$
having the property
\[\left\|\int_{a}^{b}f(s)G(s,.)\,ds-\sum_{j=1}^{N}\zeta_{j}
\langle f,v_{j}\rangle v_{j}\right\|_{2}\rightarrow 0\]
as $N\rightarrow\infty$,
\emph{or} we can find a finite orthonormal collection
$v_{j}$ in $C([a,b])$ and $\zeta_{j}$
non-zero real numbers with
\[\int_{a}^{b}f(s)G(s,t)\,ds=\sum \zeta_{j}
\langle f,v_{j}\rangle v_{j}(t).\]

(iii) Show that $\langle f,v_{j}\rangle =0$ for all $j$ implies
$f=0$ whenever $f\in C([a,b])$
if and only if
\[\int_{a}^{b}f(t)G(s,t)\,dt=0\]
for all $s$ implies $f=0$ whenever $f\in C([a,b])$.
\end{exercise}
\begin{exercise}\label{D8} Briefly identify the `eigenfunctions'
$v_{j}$ associated with non-zero 
eigenvalues in the case of Exercise~\ref{E;Green sine}.
\end{exercise}
\section{Where next?} In this section which will neither be examined
nor lectured, I look at the different ways in which the ideas
of this course can be developed.

\noindent\emph{Measure theory} Measure theory interacts
with linear analysis in many ways. 

(1) We have seen that $C([a,b])$
with the usual inner product can be identified with a dense
subset of of a complete inner product space. It is a surprising
fact that we can realise this complete inner product space
as a space of functions $L^{2}([a,b])$ on $[a,b]$
by using Lebesgue integration. 

In much the same way, it can be shown
that, if we write
\[\|f\|_{p}=\left(\int_{a}^{b}|f(x)|^{p}\,dx\right)^{1/p},\]
then $\|.\|_{p}$ is a norm on $C([a,b])$ 
(see Exercise~\ref{E;integral Holder}) and the completion
gives rise in a natural manner to a space of
$L^{p}([a,b])$ on $[a,b]$. These spaces are natural subjects
for linear analysis.

(2) Although we have studied the space $C([a,b])$ with
the the uniform norm, we did not try to identify its dual
(for some members of the dual see Exercise~\ref{E;some dual continuous}).
It is not hard to show that that the dual space can be
identified with the space of \emph{Borel measures}.

(3) The theory of compact self-adjoint operators that
we have developed on Hilbert space corresponds to
the theory of Fourier sums
\[f(t)\sim\sum_{j=-\infty}^{\infty}\hat{f}(j)\exp(ijt).\]
If we are to get something like the theory of Fourier
transforms with the putative inversion formula
\[f(t)\sim\frac{1}{2\pi}\int_{-\infty}^{\infty}\hat{f}(x)\exp(ixt)\,dx\]
we need to extend our notions of integration.
(In fact, we only need to extend the ideas of Riemann integration,
but, nonetheless, the extension is quite subtle.)

\noindent\emph{Examples} Spaces like $l^{p}$ and 
$C([a,b]),\|.\|_{\infty})$ are good examples of Banach spaces
to start with because they have a great deal of structure.
For the same reason, they are inadequate if we wish to
understand what a general Banach space might look like.

As a result of seventy years of hard work 
(including that of our own Professor Gowers)
we know that the good behaviour of $l^{p}$,
$C([a,b]),\|.\|_{\infty})$ and similar spaces
is not typical of Banach spaces in general. 
The study of Banach spaces (like the study of
most general mathematical objects) requires 
a plentiful stock of examples.

\noindent\emph{The axiom of choice} The reader will be aware
of a principle called the \emph{axiom of choice}\footnote{For
historical reasons this axiom has acquired an air of glamour and mystery
which it it hardly deserves.}.
This asserts that, given a non-empty collection ${\mathcal A}$
of non-empty sets, we can find a function
\[f:{\mathcal A}\rightarrow\bigcup_{A\in{\mathcal A}}A\]
such that $f(A)\in A$. (That is to say, $f$ \emph{chooses}
an element $f(A)$ from each $A\in{\mathcal A}$.)
Mathematical logicians have shown that, if the ordinary
axioms for set theory are consistent, then they remain
consistent if we add the axiom of choice but that the
axiom of choice is not implied by the ordinary axioms.

It turns out that the general study of Banach spaces
takes a more elegant form if we assume the
axiom of choice and, for this reason, it is customary
to assume it. Here are some consequences of this assumption.
\begin{theorem}\label{T;basis} 
Assuming the axiom of choice, every vector
space $V$ over  ${\mathbb F}$ has an algebraic basis 
(that is to say a subset $E$ such that any $v\in V$ may be written
uniquely as a finite sum
\[v=\sum_{g\in G}\lambda_{g}g\]
with $G$ a finite subset of $E$ and $\lambda_{g}\in{\mathbb F}$).
\end{theorem}   
Using this theorem, it is easy to prove the following result which
reinforces the lesson of Exercise~\ref{E;not continuous linear}.
\begin{theorem} Assuming the axiom of choice,
if $U$ is any infinite dimensional normed vector space
over  ${\mathbb F}$, then there exists a linear map
$\alpha:U\rightarrow{\mathbb F}$ which is not continuous.
\end{theorem}
We can also prove the following supplement to
Exercise~\ref{E;many norms}.
\begin{theorem} Assuming the axiom of choice,
we can find an infinite dimensional vector space
$U$ and two complete norms $\|.\|_{A}$ $\|.\|_{B}$
on $U$ such that
\[\sup_{u\neq 0}\frac{\|u\|_{A}}{\|u\|_{B}}
=\sup_{u\neq 0}\frac{\|u\|_{B}}{\|u\|_{A}}=\infty.\]
\end{theorem} 
(For example we can set up an algebraic isomorphism 
between $l^{2}$ and $l^{\infty}$.)

The axiom of choice also enables us to prove a beautiful
result of Hahn--Banach. We shall not discuss this but
here are some of its consequence. The first result
sheds light on the paragraph following
Theorem~\ref{T;dual p}.
\begin{lemma}\label{L;big dual} 
We work in $l^{\infty}$ and define 
${\mathbf e}_{n}\in l^{\infty}$ by
\[e_{nj}= 
\begin{cases}
1&\text{if $j=n$,}\\
0&\text{otherwise.}
\end{cases}
\]
Assuming the axiom of choice,
there there exists a non-zero continuous linear functional
$T:l^{\infty}\rightarrow{\mathbb C}$ such that
\[T{\mathbf e}_{n}=0\]
for all $n$.
\end{lemma} 
The second consequence was already stated in the discussion
of Definition~\ref{D;sufficiently rich}.
\begin{theorem} Assuming the axiom of choice, every Banach
space has a sufficiently rich dual.
\end{theorem}

The strengths and weaknesses of linear analysis
using the axiom of choice are well illustrated
by Lemma~\ref{L;big dual}. On the one hand, it asserts
the existence of an object $T$ without giving any clue
as to what it looks like. On the other hand, if we
did not know the result of Lemma~\ref{L;big dual},
we could waste an awful lot of time trying to show
that no such object exists.
\section{Books} There are many excellent introductions
to linear analysis. The book of Bollob\'as~\cite{Bolobas}
has the advantage of being based on this course and
a subsequent Part~III course. I think that~\cite{Pryce}
and~\cite{Gofman} are nice and reasonably simple.

If you wish to learn more about Hilbert space then~\cite{Young}
is an  excellent introduction and, if you simply want
to learn more analysis in a non-exam driven way, then
Rudin's \emph{Real and Complex Analysis} is a masterpiece. 
\begin{thebibliography}{9}
\bibitem{Bolobas} B.~Bollob\'as, 
\emph{Linear Analysis}, CUP, 1991.

\bibitem{Gofman} C.~Gofman and G.~Pedrick,
\emph{A First Course in Functional Analysis},
Prentice Hall, 1965. (This is now reissued by
AMS, Chelsea)

\bibitem{Pryce}J.~D.~Pryce 
\emph{Basic Methods of Linear Functional Analysis}, 
Hutchinson, 1973. (Out of print but should be
in college libraries.)


\bibitem{Rudin}
 W.Rudin \emph{Real and Complex Analysis}, McGraw Hill, 2nd Ed, 1974.

\bibitem{Young} N.~Young \emph{An Introduction to Hilbert Space},
CUP, 1988.
\end{thebibliography}

\section{First example sheet} 
Students who are unsure of  their ground should
check that they can do the exercises in the main text. 
Strong students should at least glance at the supplementary
example sheet. The order of the exercises roughly follows the
order of the lectures.

\begin{exercise}\label{E;integral Holder}\label{C1.1} In 
this exercise, $\infty>p>1$
and $p^{-1}+q^{-1}=1$.
We work with the space $C([a,b])$ of continuous functions
on $[a,b]$.

(i) Prove H{\"o}lder's integral inequality
\[\int_{a}^{b}|f(t)g(t)|\,dt\leq 
\left(\int_{a}^{b}|f(t)|^{p}\,dt\right)^{1/p}
\left(\int_{a}^{b}|g(t)|^{q}\,dt\right)^{1/q}\]
for all $f,\,g\in C([a,b])$.

(ii) State and prove an appropriate reverse form of
H{\"o}lder's integral inequality.

(iii) Show that
\[\|f\|_{p}=\left(\int_{a}^{b}|f(t)|^{p}\,dt\right)^{1/p}\]
defines a norm on $C([a,b])$.

(iv) Show that $(C([a,b]),\|.\|_{p})$ is not complete.
(We shall consider the particular case $p=2$
in Exercise~\ref{E;continuous square not complete}.) 

(v) By applying  H{\"o}lder's integral inequality
with $g=1$, $p=v/u$, or otherwise, show that
\[\|F\|_{u}\leq (b-a)^{(u^{-1}-v^{-1})}\|F\|_{v}\]
when $\infty>v>u>1$. 

(vi) Show that, if  $\infty>v>u>1$, then, given
any $K>0$, we can find an $f\in C([a,b])$ such that
\[\|f\|_{v}>K\|f\|_{u}.\]

\noindent[Note that the inequalities in (v) and (vi)
run in the {\bf opposite} way to the $l^{p}$ case.]

(vii) [Optional extra] Show that, if $\infty>v>u>1$, then 
given $K>0$, we can find continuous functions
$f$ and $g$ which are zero outside some interval
such that
\begin{gather*}
\left(\int_{-\infty}^{\infty}|f(x)|^{u}\,dx\right)^{1/u}
>K\left(\int_{-\infty}^{\infty}|f(x)|^{v}\,dx\right)^{1/v},\\
\left(\int_{-\infty}^{\infty}|g(x)|^{v}\,dx\right)^{1/v}
>K\left(\int_{-\infty}^{\infty}|g(x)|^{u}\,dx\right)^{1/u}.
\end{gather*}
\end{exercise}
\begin{exercise}\label{C1.2} Suppose $1>p>0$.

(a) Find $(x_{1},x_{2}),\,(y_{1},y_{2})\in{\mathbb R}^{2}$
such that
\[\big((x_{1}+y_{1})^{p}+(x_{2}+y_{2})^{p})\big)^{1/p}
> (x_{1}^{p}+x_{2}^{p})^{1/p}+(y_{1}^{p}+y_{2}^{p})^{1/p}.\]

(b) Show, by considering the behaviour of
$1+t^{p}-(1+t)^{p}$, or otherwise, that, if $a,\,b\geq 0$,
then 
\[a^{p}+b^{p}\geq(a+b)^{p}.\]

(c) Show that, if we write $l^{p}$ for the space of complex
sequences ${\mathbf a}$ with
\[\sum_{j=1}^{\infty}|a_{j}|^{p}<\infty,\]
then $l^{p}$ can be made into a vector space in the standard way.

Show that, if we set
\[d({\mathbf a},{\mathbf b})=\sum_{j=1}^{\infty}|a_{j}-b_{j}|^{p},\]
then $d$ is a complete metric on $l^{p}$.
\end{exercise}
\begin{exercise}\label{C1.3} 
Show that we can find a constant $A_{n}$ such that
\[\sup_{t\in[0,1]}|p'(t)|\leq A_{n}\sup_{t\in[0,1]}|p(t)|\]
for every real polynomial of degree $n$ or less.
\end{exercise}
\begin{exercise}\label{C1.4} Let $E$ and $F$ be normed spaces. 
Let $A$ be a dense subset of $E$,
and let
$T_{n}:E\rightarrow F$ be a continuous linear map for each $n\geq 1$.
Show that if

(a) there exists a $K$ with $\|T_{n}\|\leq K$ for all $n$, and

(b) $T_{n}(a)\rightarrow 0$ for all $a\in A$,

\noindent then $T_{n}(e)\rightarrow 0$ for all $e\in E$.

Is the  result true if condition (a) is  dropped? Give a proof or
a counterexample.

If (a) and (b) hold, does it follow that $\|T_{n}\|\rightarrow 0$
as $n\rightarrow\infty$? Give a proof or
a counterexample.
\end{exercise}
\begin{exercise}\label{C1.5}  
(A useful fact.) Let $(V,\|.\|)$ be a normed space.
Show that it is a Banach space if and only if $\sum_{j=1}^{\infty}x_{j}$
converges whenever
$\sum_{j=1}^{\infty}\|x_{j}\|$ converges.

In the special case when $V={\mathbb C}$ and $\|z\|=|z|$ deduce
that absolute convergence implies convergence.
\end{exercise} 

\begin{exercise}\label{C1.6} (i) Consider $C([0,1])$ with the uniform norm.
Show that
\[E=\{f\in C([0,1])\,:\,f(0)=0\}\]
is a closed subspace of $C([0,1])$ and explain why this
means that $E$ is a Banach space under the uniform norm.

Show that
\[F=\left\{f\in E\,:\,\int_{0}^{1}f(t)\,dt=0\right\}\]
is a closed subspace of $E$. Show that there does
not exist a $g\in E$ such that $\|g\|_{\infty}=1$
and 
\[\|g-f\|_{\infty}\geq 1\]
for all $f\in F$.

Thus Theorem~\ref{T;Riesz} cannot be improved
in general.

(ii) Show, however, that,
if $F$ is a subspace of a  finite dimensional
normed space space $(E,\|.\|)$
and $F\neq E$, then we can find
an ${\mathbf e}\in E$ with $\|{\mathbf e}\|=1$ such that
\[\|{\mathbf e}- {\mathbf f}\|\geq 1.\]
for all ${\mathbf f}\in F$.
\end{exercise}
\begin{exercise}\label{E;some dual continuous}\label{C1.7} 
In this question we work with real valued continuous functions although
similar results hold for the complex valued case.

(i) Consider the space $(C([a,b]),\|.\|_{\infty})$.
Show that, if $s\in[a,b]$ and
\[\delta_{s}(g)=g(s),\]
then $\delta_{s}\in  C([a,b])'$. What is $\|\delta_{s}\|$
and why? Can you find a $g\in C([a,b])$ with 
$\|g\|_{\infty}=1$ and $\delta_{s}(g)=\|\delta_{s}\|$?
Give reasons.
Show that $C([a,b])$ has a sufficiently rich dual in the sense of
Definition~\ref{D;sufficiently rich}.

(ii) Consider the space $(C([a,b]),\|.\|_{\infty})$.
If $F\in C([a,b])$ set
\[T_{F}(g)=\int_{a}^{b}F(t)g(t)\,dt.\]
Show that $T_{F}\in  C([a,b])'$. What is $\|T_{F}\|$
and why? Can you always find a $g\in C([a,b])$ with 
$\|g\|_{\infty}=1$ and $T_{F}(g)=\|T_{F}\|$?
Give reasons.

(iii) Consider the space $(C([a,b]),\|.\|_{1})$
where, as usual,
\[\|g\|_{1}=\int_{a}^{b}|g(t)|\,dt.\]
If $\delta_{s}$ and $T_{F}$ are defined as before, show that
$\delta_{s}$ is not continuous, but $T_{F}$ is.

(iv) [Optional extra] Continuing with the ideas of (iii),
find  $\|T_{F}\|$ and prove your answer.
\end{exercise}
\begin{exercise}\label{E;more dual complete}\label{C1.8}
(i) Show that if $(U,\|.\|_{U})$ is a normed space and
$(V,\|.\|_{V})$ is a Banach space, then
$({\mathcal L}(U,V),\|.\|)$ is a Banach space.

(ii) Consider $c_{00}$ (the space of sequences with 
all but finitely many terms zero) with the norm
\[\|{\mathbf a}\|_{*}=\sum_{j=1}^{\infty}|a_{j}|\]
and the space $l^{1}$ with its usual norm. Let
${\mathcal L}(l^{1},C_{00})$
be defined as in Theorem~\ref{T;dual complete}.

If we
set
\[T_{n}({\mathbf a})=(a_{1},2^{-1}a_{2},3^{-1}a_{3},\dots,n^{-1}a_{n},
0,0,\dots),\]
show that
$T_{n}\in {\mathcal L}(l^{1},c_{00})$. Show that the $T_{n}$ form a Cauchy sequence
in ${\mathcal L}(l^{1},c_{00})$ with no limit point.

Thus Theorem~\ref{T;dual complete} may fail if $(V,\|.\|)$ is
not complete.
\end{exercise}
\begin{exercise}\label{E;not $l^{1}$}\label{C1.9}
If $T:U\rightarrow V$ is an isomorphism
between the Banach spaces $U$ and $V$ (that is to say, a linear
bijection such that $T$ and $T^{-1}$ are continuous),
show that the map $T':V'\rightarrow U'$ between the
dual spaces given by
\[T'(v')u=v'(Tu)\]
for all $v'\in V'$ and $u\in U$ is a well defined isomorphism between
$V'$ and $U'$. (Observe, that, on general grounds, the verification
must consist of routine and rather easy steps.)

Deduce that $l^{1}$ cannot be isomorphic to $l^{p}$ for any $p>1$.
\end{exercise} 


\begin{exercise}\label{C1.10} Suppose that $X$, $Y$ and $Z$ are Banach spaces.
Suppose that $F:X\times Y\rightarrow Z$ is linear and continuous
in each variable separately,
that is to say that, if $y$ is fixed,
\[F(.,y)):X\rightarrow Z\]
is a continuous linear map and, if $x$ is fixed,
\[F(x,.):Y\rightarrow Z\]
is a continuous linear map. Show, by using the principle
of uniform boundedness, that there exists an $M$ such that
\[\|F(x,y)\|_{Z}\leq M\|x\|_{X}\|y\|_{Y}\]
for all $x\in X,\, y\in Y$. Deduce that
$F$ is continuous.
\end{exercise}
\begin{exercise}\label{E;equivalent norms}\label{C1.11} 
Suppose that $U$
is a vector space with two complete norms $\|.\|_{A}$ and
$\|.\|_{B}$. By applying the open mapping theorem to
an appropriate linear map, show that if there exists a $K$ such that
\[K \|u\|_{A}\geq  \|u\|_{B}\]
for all $u\in U$, then there exists a $K'$ such that
\[K' \|u\|_{B}\geq  \|u\|_{A}\]
for all $u\in U$. Thus comparable complete norms are equivalent.

\noindent[We could also use the inverse mapping theorem but this
comes to much the same thing.]
\end{exercise}
\begin{exercise}\label{C1.12}  (i) (Dini's theorem) Let $(X,d)$ 
be a compact metric space. Suppose $f_{n}:X\rightarrow{\mathbb R}$
is a sequence of continuous functions such that,
for each fixed $x\in X$, $f_{n}(x)$ is a decreasing sequence
with $f_{n}(x)\rightarrow 0$ as $n\rightarrow\infty$.
By considering
\[B_{n}=\{x\,:\,f_{n}(x)<\epsilon\}\]
for any fixed $\epsilon>0$ show that $f_{n}\rightarrow 0$
uniformly on $X$.

(ii) Show, by means of an example,
that the condition $(X,d)$ compact cannot be dropped.
Show, by means of an example,
that the condition $f_{n}$ decreasing cannot be dropped.
Show, by means of an example,
that the condition $f_{n}$ continuous cannot be dropped.


(iii) Set $p_{0}=0$ and 
$p_{n+1}(x)=\tfrac{1}{2}x^{2}+p_{n}(x)-\tfrac{1}{2}p_{n}(x)^{2}$.
Explain why $p_{n}$ is a polynomial. Show that
\[p_{n}(x)\leq p_{n+1}(x)\leq |x|\]
and all $n\geq 0$ for all $x\in [0,1]$. 
Hence deduce 
that $p_{n}(x)\rightarrow |x|$ as $n\rightarrow\infty$ for all $x\in[0,1]$.
Now use Dini's theorem to show that the convergence is uniform.

Explain how to use this result as a replacement for
Lemma~\ref{L;Taylor for Stone} in the proof of the Stone--Weierstrass
theorem.
\end{exercise}
\section{Second example sheet}
Students who are unsure of  their ground should
check that they can do the exercises in the main text. 
Strong students should at least glance at the supplementary
example sheet. The order of the exercises roughly follows the
order of the lectures.

\begin{exercise}\label{C2.1}
(i) Here is a typical use of the Stone--Weierstrass
theorem. If $f\in C[0,1]$, we say that $f$ has $n$th moment
\[E_{n}(f)=\int_{0}^{1}f(t)t^{n}\,dt.\]

Show that, if all the moments of $f$ vanish, then
\[\int_{0}^{1}f(t)P(t)\,dt=0\]
for all polynomials. Use the Stone--Weierstrass
theorem to deduce that
\[\int_{0}^{1}f(t)g(t)\,dt=0\]
for all $g\in C[0,1]$. Deduce that $f=0$.

(ii) (Optional) Let $\omega=\exp(i\pi/4)$. Show
that
\[\int_{0}^{\infty}y^{n}e^{-\omega y}\,dy=n!\omega^{-n-1}\]
and deduce that
\[\int_{0}^{\infty}y^{4n+3}\exp(-2^{-1/2}y)\sin(2^{-1/2}y)\,dy=0.\]
By making the substitution $x=y^{4}/4$, show that
\[\int_{0}^{\infty}x^{n}\exp(-x^{1/4})\sin(x^{1/4})\,dy=0\]
for all $n$ although $x\mapsto\exp(-x^{1/4})\sin(x^{1/4})$
is a well behaved non-zero continuous function. Why does
the
argument of part~(i) fail?

\noindent[Both parts have obvious relevance to the question
of what we can say about a random variable $X$ from knowledge
of its moments.]
\end{exercise}
\begin{exercise}\label{E:Riemann--Lebesgue}\label{C2.2} 
(The Riemann--Lebesgue lemma)
(i) The Riemann--Lebesgue lemma tells us that, if $f\in C({\mathbb T})$,
then $\hat{f}(n)\rightarrow 0$ as $|n|\rightarrow\infty$. There
are many ways of proving this but you are asked to prove it by
finding a dense subalgebra of $C({\mathbb T})$ for which
the result is true `for obvious reasons' and then using a density
argument to extend the result to all of $C({\mathbb T})$.

(ii) (Optional) Suppose that $\phi(n)>0$ and $\phi(n)\rightarrow 0$.
Show that we can find $0<n_{1}<n_{2}<\dots$ such that
$\sum_{j=1}^{\infty}\phi(n_{j})$ converges. By considering
$\sum_{j=1}^{\infty}\phi(n_{j})\cos n_{j}x$, or otherwise,
show that there exists an $f\in  C({\mathbb T})$ such that
$\phi(n)^{-1}\hat{f}(n)\nrightarrow 0$. Thus part~(i) cannot be improved.
\end{exercise}
\begin{exercise}\label{E;Fubini}\label{C2.3} In this question
you may use results about Riemann integration in
one dimension but not in higher dimensions.

(i) Suppose that $f:[0,1]^{2}\rightarrow{\mathbb R}$
is continuous. Explain why the function
$g:[0,1]\rightarrow{\mathbb R}$ defined by
\[g(x)=\int_{0}^{1}f(x,y)\,dy\]
is continuous and so
\[\int_{0}^{1}\left(\int_{0}^{1}f(x,y)\,dy\right)\,dx\]
exists.

(ii) Show that, if $u,\,v:[0,1]\rightarrow{\mathbb R}$
are continuous,
\[\int_{0}^{1}\left(\int_{0}^{1}u(x)v(y)\,dy\right)\,dx
=\int_{0}^{1}\left(\int_{0}^{1}u(x)v(y)\,dx\right)\,dy.\]

(iii) Use the theorem of Stone--Weierstrass to show that the
collection of functions $h$ of the form
\[h(x,y)=\sum_{j=1}^{n}u_{j}(x)v_{j}(y)\]
with $u_{j},\,v_{j}:[0,1]\rightarrow{\mathbb R}$
continuous, is uniformly dense in $C([0,1]^{2})$.

(iv) Deduce that
\[\int_{0}^{1}\left(\int_{0}^{1}f(x,y)\,dy\right)\,dx
=\int_{0}^{1}\left(\int_{0}^{1}f(x,y)\,dx\right)\,dy\]
for all $f\in C([0,1]^{2})$.
\end{exercise}
\begin{exercise}\label{C2.4} 
(i) Show that we cannot find
a sequence $P_{n}$ of polynomials such that
$P_{n}(x)\rightarrow \exp(x)$ uniformly on $[0,\infty)$
as $n\rightarrow\infty$. 
Show that we cannot find
a sequence $Q_{n}$ of polynomials such that
$Q_{n}(x)\rightarrow \exp(-x)$ uniformly on $[0,\infty)$
as $n\rightarrow\infty$.

Why do these results not
contradict the Stone--Weierstrass theorem?

(ii) Let $A_{0}$ be the subset of $C_{\mathbb R}([0,1])$
consisting of polynomials of the form 
$\sum_{j=1}^{n}a_{j}x^{j}$. Show that $A_{0}$ is not
uniformly dense in $C_{\mathbb R}([0,1])$.Why does this not
contradict the Stone--Weierstrass theorem?
Identify the uniform closure of $A_{0}$
and prove your statement.

(iii) Let $A_{1}$ be the subset of $C_{\mathbb R}([0,1])$
consisting of polynomials of the form
$\sum_{j=0}^{n}a_{j}x^{2j}$. Show that $A_{1}$ is
uniformly dense in $C_{\mathbb R}([0,1])$.

(iii) Let $A_{2}$ be the subset of $C_{\mathbb R}([-1,1])$
consisting of polynomials of the form 
$\sum_{j=0}^{n}a_{j}x^{2j}$. Show that $A_{2}$ is not
uniformly dense in $C_{\mathbb R}([-1,1])$.Why does this not
contradict the Stone--Weierstrass theorem?
Identify the uniform closure of $A_{2}$
and prove your statement.
\end{exercise}
\begin{exercise}\label{E;Baire compact}\label{C2.5}
(i) Show that a topological space $(X,\tau)$ 
is compact if and only if
it satisfies the following condition:--

If ${\mathcal F}$ is a non-empty collection of closed sets
such that, whenever 
\[F_{1},\,F_{2},\,\dots,\,F_{n}\in{\mathcal F},\]
we have $\bigcap_{j=1}^{n}F_{j}\neq\emptyset$, then
$\bigcap_{F\in{\mathcal F}}F\neq\emptyset$.

(ii) By quoting the appropriate theorem or (if you need the
practice, reproving the result which is used in obtaining that
theorem) show that if $(X,\tau)$ is compact and Hausdorff,
then, given $V\in\tau$ and $x\in V$, we can find a closed set $F$
and an open set $U$ with
\[x\in U\subseteq F\subseteq V.\]

(iii) (Baire's theorem for compact Hausdorff spaces.)
If $(X,\tau)$ is a compact Hausdorff topological space
and $U_{j}$ is an open subset of $X$ 
with the property that $U_{j}\cap V\neq\emptyset$  for all 
non-empty open $V$ $[j\geq 1]$, show that
$\bigcap_{j=1}^{\infty}U_{j}\neq\emptyset$.
\end{exercise}
\begin{exercise}\label{C2.6} (i) If $A$ is a set of first category
in a complete metric space $(X,d)$ without isolated points
show that $X\setminus A$ is uncountable.

(ii) If $A$ is a set of first category
in a complete metric space $(X,d)$ without isolated points,
show, by considering the subspace
\[\Cl\{x\,:\,d(x,y)<\delta\},\]
or otherwise, that
\[\{x\,:\,d(x,y)\leq \delta\}\setminus A\]
is uncountable for all $y\in X$ and $\delta>0$.

\noindent[Be careful. A subspace may have isolated points even if 
the original  space does not.]
\end{exercise}
\begin{exercise}\label{E;smooth non-analytic}\label{C2.7} 
Consider the space
$C_{\mathbb R}^{\infty}([0,1])$ of infinitely
differentiable real functions on $[0,1]$ (with appropriate conventions
about left and right derivatives at end points).

(i) Show that, if we define
\[d(f,g)=\sum_{j=0}^{\infty}\min(2^{-j},\|f^{(j)}-g^{(j)}\|_{\infty}),\]
then $d$ is a complete metric on $C_{\mathbb R}^{\infty}([0,1])$.

(ii) Suppose $q$ is a point of $[0,1]$. Show that
\[E_{m}(q)=\{f\in C_{\mathbb R}^{\infty}([0,1])
\,:\,|f^{(j)}(q)|\leq m \times j!\times j^{j}
\ \text{for all $j\geq 0$}\}\]
is closed in the $d$ metric and its complement is dense.

(iii) Deduce that we can find an $F\notin\bigcup_{m=1}^{\infty}E_{m}(q)$.
Show that 
\[\limsup_{j\rightarrow\infty}\frac{|F^{(j)}(q)||y|^{j}}{j!}=\infty\]
for all $y\neq 0$. Deduce that the Taylor expansion of $F$ around $q$
diverges except at $q$  (ie has zero radius of convergence).

(iv) If $F$ is as in~(iii), explain why $F$ cannot have a power series
expansion valid in any open interval containing $q$.

(v) Up to now we have kept $q$ fixed. Extend our argument to show the
existence of an infinitely differentiable 
function $G:[0,1]\rightarrow{\mathbb R}$
which does have a power series
expansion valid in any open interval.
\end{exercise}


\begin{exercise}\label{C2.8} The object of this exercise is to
prove the following form of the Tietze extension theorem.

If $(X,\tau)$ is a topological space and $Y$
is a compact subset of $X$ such that, given any
$x,\,y\in Y$ with $x\neq y$,
we can find a $g\in C_{\mathbb R}(X)$
with $g(x)\neq g(y)$ then,  given any $f\in C_{\mathbb R}(Y)$,
we can find
an $F\in C_{\mathbb R}(X)$ such that $F(y)=f(y)$ for all $y\in Y$.

(i) Prove Tietze's extension theorem in the form
given in Theorem~\ref{T;Tietze} using Urysohn's lemma
and the result just stated.

(ii) From now on, we assume that $(X,\tau)$ and $Y$
satisfy the hypotheses of our theorem. Let us write
$A$ for the collection of $f \in C_{\mathbb R}(Y)$
such that we can find
an $F\in C_{\mathbb R}(X)$ with $F(y)=f(y)$ for all $y\in Y$.
Use the Stone--Weierstrass theorem to show that $A$
is uniformly dense in $C_{\mathbb R}(Y)$.

(iii) Show that, if $f \in A$,
we can find $\tilde{f}\in C_{\mathbb R}(X)$
with $\tilde{f}(y)=f(y)$ for all $y\in Y$
and $\|\tilde{f}\|_{\infty}=\|f\|_{\infty}$.
Use this result to show that $A$ is uniformly
closed in $C_{\mathbb R}(Y)$. Deduce that
$A=C_{\mathbb R}(Y)$ and our theorem holds.
\end{exercise}
\begin{exercise}\label{C2.9} We work in ${\mathbb C}$ and write
\[\partial D=\{z\in {\mathbb C}\,:\,|z|=1\}.\]
Suppose $K$ is a compact subset of ${\mathbb C}$
and $\phi:K\rightarrow \partial D$ is continuous.
By considering functions of the form $\Phi(z)/|\Phi(z)|$,
or otherwise, show that there is an open set $\Omega\supseteq K$
and a continuous function $\tilde{\phi}:\Omega\rightarrow \partial D$.

Explain, giving an explicit $K$ and $\phi$ and 
using ideas from complex analysis or topology
(you are not asked for proofs), why we cannot always take
$\Omega={\mathbb C}$.
\end{exercise}
\begin{exercise}\label{C2.10} 
We work with continuous functions on $[0,1]$.

(i) Show that, if $f_{n}\rightarrow f$, uniformly then
the set ${\mathcal F}=\{f_{n}:n\geq 1\}$ is equicontinuous.

(ii) Show that, if ${\mathcal F}=\{f_{n}:n\geq 1\}$ is equicontinuous
and $f_{n}(q)\rightarrow f(q)$ as $n\rightarrow\infty$ at each rational
point $q\in [0,1]$, then $f_{n}\rightarrow f$ uniformly.
\end{exercise}
\begin{exercise}\label{E;alternate Ascoli}\label{C2.11}
Recall the definition of \emph{total boundedness}
from the metric and topological spaces course.
Let $(X,\tau)$ be a compact Hausdorff space.
Suppose  that ${\mathcal F}$ is a subset of $C(X)$.

(i) Explain why  ${\mathcal F}$ is complete
under the natural metric $d$ (that is the restriction of the
metric induced by $\|.\|_{\infty}$) if and only if
${\mathcal F}$ is closed in $C(X)$.
Explain why this means that ${\mathcal F}$
is compact if and only if ${\mathcal F}$
is closed and totally bounded.

(ii) Show that  ${\mathcal F}$
is totally bounded if and only if it
is uniformly bounded and equicontinuous.

(iii) Deduce the Ascoli--Arzel{\`a} theorem (Theorem~\ref{T;Ascoli}).
\end{exercise}
\begin{exercise}\label{C2.12} 
Find all continuously differentiable functions
$x:{\mathbb R}\rightarrow{\mathbb R}$ with
\[x'(t)=3x(t)^{2/3}\ \text{for all $t$}\]
subject to the condition $x(0)=0$ and show 
(formally or informally according to taste) 
that you have found them all.

\noindent[Note that $x\mapsto 2x^{2/3}$ is differentiable
(and so satisfies a Lipschitz condition) for $x\neq 0$.]
\end{exercise}
\section{Third example sheet}
Students who are unsure of  their ground should
check that they can do the exercises in the main text. 
Strong students should at least glance at the supplementary
example sheet. The order of the exercises roughly follows the
order of the lectures.

\begin{exercise}\label{C3.1} 
We work with the set $A(D)$ of functions
analytic on
\[D=\{z\in{\mathbb C}\,:\,|z|<1\}.\]

(i) Let $1>R>r$. By considering Cauchy's formula with the
contour $z=Re^{i\theta}$, or otherwise, show that
\[{\mathcal F}_{R}(M)=
\{f\in A(D)\,:\,|f(z)|\leq M\ \text{for all $|z|\leq R$}\}\]
is equicontinuous at every point $z$ with $|z|\leq r$.

(ii) If $f_{n}\in {\mathcal F}$ and $|f_{n}(z)|\leq M$ for all $|z|\leq R$,
show that we can find a continuous function
\[F:\{z\in{\mathbb C}\,:\,|z|\leq r\}\rightarrow{\mathbb C}\]
and $n(j)\rightarrow\infty$ such that $f_{n(j)}(z)\rightarrow F(z)$
uniformly for $|z|\leq r$. Explain why $F$ is analytic on 
$D_{r}=\{z\,:\,|z|<r\}$.

(iii) Show that the formula
\[d(f,g)=\sum_{j=1}^{\infty}
2^{-j}\min(1,\sup_{|z|\leq 1-2^{-j}}|f(z)-g(z)|)\]
gives a complete metric on $A(D)$.  Show that the set
\[{\mathcal F}(M)=\{f\in A(D)\,:\,|f(z)|\leq M\ \text{for all $|z|<1$}\}\]
is compact in this metric.

\noindent[Simple developments of these ideas produce the powerful 
method of \emph{normal families} in complex variable theory.]
\end{exercise}
\begin{exercise}\label{C3.2} Consider the space of continuous functions
$C([0,1])$ with the uniform norm. Suppose that
$E$ is a closed subspace consisting of continuously
differentiable functions.

(i) By applying the closed graph theorem, or otherwise, show that the
differentiation map $D:E\rightarrow  C([0,1])$ 
(given by $Df=f'$) is continuous.


(ii) Deduce, using the  Ascoli--Arzel{\`a} theorem, or otherwise that
the unit ball in $E$  is compact.

(iii) Deduce, quoting the appropriate theorem, that $E$ is finite dimensional.
\end{exercise} 
\begin{exercise}\label{C3.3} Suppose $K$ is a compact Hausdorff space
and $C(K)$ is the countable union of equicontinuous sets.
Show that $C(K)$ is the countable union of 
bounded and closed (in the uniform norm) equicontinuous sets.
Deduce, using three of the `big theorems' of the course,
or otherwise,
that $C(K)$ is finite dimensional and so $K$ is finite.
\end{exercise}
\begin{exercise}\label{E;closest convex}\label{C3.4}
(i) Let $U$ be a complete inner product space.
If $F$ is closed convex subset of $U$ and $a\in U$, 
show that  we can find a unique
$f_{0}\in F$ such that
\[\|a-f_{0}\|_{2}\leq\|a-f\|_{2}\]
for all $f\in F$.

\noindent[A set $F$ is convex if, whenever $x,\,y\in F$ 
and $1\geq \lambda\geq 0$, it follows that $\lambda x+(1-\lambda) y\in F$.]

(ii)  Give an example of a closed convex subset $F$ of $l^{\infty}$
and a point ${\mathbf a}\in l^{\infty}$
with the property that the equation
\[\|{\mathbf a}-{\mathbf g}\|_{\infty}=
\inf_{{\mathbf f}\in F}\|{\mathbf a}-{\mathbf f}\|_{\infty}\]
has more than one solution with ${\mathbf g}\in F$.
\end{exercise}
\begin{exercise}\label{E;not closed closest}\label{C3.5}
We work in $l^{2}$. Let $F$ be the collection of
${\mathbf x}\in l^{2}$ such that all but finitely many
$x_{j}$ are zero and let 
\[{\mathbf a}=(1,1/2,1/3,\dots).\]
Show that $F$ is a subspace of $l^{2}$, but,
given  ${\mathbf f}\in F$,
we can always find a ${\mathbf g}\in F$ such that
\[\|{\mathbf a}-{\mathbf f}\|_{2}>\|{\mathbf a}-{\mathbf g}\|_{2}.\]
Why does this not contradict Theorem~\ref{T;closest Hilbert}?
\end{exercise}
\begin{exercise}\label{C3.6} 
(i) Use Theorem~\ref{T;closest Hilbert}
and Lemma~\ref{L;closest orthogonal} to show that,
if $U$ is complete separable inner product space 
and $F$ is closed subspace of $U$, then $F$, considered
as a space in its own right (inheriting the inner product from $U$),
is itself a complete separable inner product space.

(ii) Suppose that $U$ is an inner product space 
and $F$ is closed subspace of $U$. Suppose further
that $F$ considered as
a space in its own right has an orthonormal basis $e_{1}$,
$e_{2}$, \dots. Use the arguments which we used to establish Bessel's
inequality to show that, if $a\in U$ and 
$\sum_{j=1}^{\infty}|x_{j}|^{2}<\infty$, then
\[\left\|\sum_{j=1}^{\infty}x_{j}e_{j}-a\right\|_{2}
\geq \left\|\sum_{j=1}^{\infty}\langle a,e_{j}\rangle e_{j}-a\right\|_{2}\]
with equality only if $x_{j}=\langle a,e_{j}\rangle$. Deduce
the conclusions of
Theorem~\ref{T;closest Hilbert}
and Lemma~\ref{L;closest orthogonal} in this case.
\end{exercise}
\begin{exercise}\label{C3.7} (i) Consider the inner product space $c_{00}$ 
of sequences with only finitely many non-zero terms and inner
product
\[\langle {\mathbf a},{\mathbf b}\rangle=
\sum_{j=1}^{\infty}a_{j}b_{j}^{*}.\]
Show that the formula
\[T{\mathbf u}=\sum_{j=1}^{\infty}j^{-1}u_{j}\]
defines a continuous linear map $T:c_{00}\rightarrow{\mathbb C}$
but there does not exist a ${\mathbf w}\in  c_{00}$ such that
\[T{\mathbf u}=\langle {\mathbf u},{\mathbf w}\rangle.\]
Why does this not contradict the Riesz representation theorem
(Theorem~\ref{T;Riesz representation})?

(ii) Find a closed subspace $F$ of $c_{00}$
and an ${\mathbf a}\in c_{00}$ such that,
given  ${\mathbf f}\in F$,
we can always find a ${\mathbf g}\in F$ such that
\[\|{\mathbf a}-{\mathbf f}\|_{2}>\|{\mathbf a}-{\mathbf g}\|_{2}.\]
Why does this not contradict Theorem~\ref{T;closest Hilbert}?
\end{exercise}
\begin{exercise}\label{C3.8} Suppose that $U$ is an inner product space
and $e_{1}$, $e_{2}$, \dots are orthonormal. Show that
\[\langle f,e_{n}\rangle\rightarrow 0\]
as $n\rightarrow\infty$. 

Use this result to give another proof of the Riemann--Lebesgue
lemma (see Exercise~\ref{E:Riemann--Lebesgue}).
\end{exercise}
\begin{exercise}\label{C3.9} 
Suppose that $e_{1}$, $e_{2}$, \dots is an orthonormal
basis for an inner product space $E$. Show reasonably carefully that
\[\langle x,y\rangle=\sum_{j=1}^{\infty}
\langle x,e_{j}\rangle\langle y,e_{j}\rangle^{*}.\]
\end{exercise}
\begin{exercise}\label{C3.10} Consider
a Hilbert space $H$ with a closed subspace 
$E$ and $T:E\rightarrow{\mathbb C}$ a continuous linear
functional. Show that there exists a continuous linear
functional $\tilde{T}:H\rightarrow{\mathbb C}$ such that
$\tilde{T}|_{E}=T$ and $\|\tilde{T}\|=\|T\|$.

\noindent[You can prove this using bases but well brought up people
do not use bases unless they have to. (However, if they have to, they
instantly forget their upbringing.)]
\end{exercise}
\begin{exercise}\label{C3.11} Let $E$ be an inner product space.
We say that a non-zero continuous linear map $P:E\rightarrow E$
is a \emph{projection} if $P^{2}=P$. We say that $P$
is an \emph{orthogonal projection} if $P(U)\perp P^{-1}(0)$.

(i) Give a example of an orthogonal projection $Q:l^{2}\rightarrow l^{2}$
such that both $Q(U)$ and $Q^{-1}(0)$ are infinite dimensional.

(ii)  Give a example of a projection 
$R:{\mathbb C}^{2}\rightarrow {\mathbb C}^{2}$ which is 
not an orthogonal projection.

(iii) If $E$ is an inner product space and $P:E\rightarrow E$ 
is a projection, show that $P$ is an orthogonal projection
if and only if $\|P\|=1$.
\end{exercise}
\begin{exercise}\label{C3.12} Suppose that $H$ is a Hilbert space.
If $C$ is a subset (NB no algebraic structure is implied)
of $H$ we write
\[C^{\perp}=\{g\in H\,:\,\langle g,f\rangle=0\ \text{for all $f\in C$}\}.\]

(i) Show that $C^{\perp}$ is a closed subspace of $H$.

(ii) Show that $C^{\perp\perp}=C$ if and only if $C$ is
a closed subspace of $H$.

(iii) Show that
$C^{\perp\perp}=\Cl\Span C$.
\end{exercise}
\section{Fourth example sheet}
Students who are unsure of  their ground should
check that they can do the exercises in the main text. 
Strong students should at least glance at the supplementary
example sheet. The order of the exercises roughly follows the
order of the lectures.

\begin{exercise}\label{C4.1} Let $e_{1}$, $e_{2}$, \dots be an orthonormal basis
for the Hilbert space $H$. Write
\begin{align*}
u_{n}&=e_{2n-1}\\
v_{n}&=e_{2n-1}+n^{-1}e_{2n}
\end{align*}
and let $U$ be the closed subspace generated by the $u_{n}$
(ie the smallest closed subspace containing the $u_{n}$)
and $V$ be the closed subspace generated by the $v_{n}$.

(i) Show that $U\cap V=\{0\}$.

(ii) Show that every $x\in U+V$ has a unique expression
as $x=Px+Qx$ with $Px\in U$ and $Qx=V$.
Show that $P:U+V\rightarrow U$ is linear and
satisfies $P^{2}=P$ (so $P$ is a projection)
but that $P$ is not continuous.

(iii) Show that $U+V$ is dense in $H$ but  $U+V\neq H$.
\end{exercise}
\begin{exercise}\label{C4.2} Let $H$ be Hilbert space. Suppose that
$T:H\rightarrow H$ is linear and self-adjoint 
(that is to say $\langle Tx,y\rangle=\langle x,Ty\rangle$
for all $x,\,y\in H$). Use the the principle of uniform
boundedness to show that $T$ is continuous.
\end{exercise}
\begin{exercise}\label{C4.3} Let $U$ be a normed space and $T:U\rightarrow U$
a continuous linear map. Explain why
\[\rho(T)=\liminf_{n\rightarrow\infty}\|T^{n}\|^{1/n}\]
is well defined.
 
By making the observation that
\[\|T^{Nk+r}\|\leq \|T^{N}\|^{k}\|T\|^{r},\]
or otherwise, show that, in fact
\[\|T^{n}\|^{1/n}\rightarrow\rho(T)\]
as $n\rightarrow\infty$. (We call $\rho(T)$ the spectral
radius of $T$.)

Now suppose that $U$ is a Banach space.
Show that $\lambda I-T$ is invertible 
(that is to say, has continuous inverse) for all $|\lambda|>\rho(T)$.

By considering maps of the form
\[(x_{1},x_{2},x_{3},\dots)\mapsto(0,c_{1}x_{1},c_{2}x_{2},\dots)\]
with appropriate $c_{j}$, show that we can find $T:l^{2}\rightarrow l^{2}$
such that $T^{n}$ is injective (so, in particular,
$T^{n}\neq 0$) for all $n\geq 1$ but $\rho(T)=0$.
\end{exercise}
\begin{exercise}\label{C4.4}\label{E;dense surjective} 
Let $U$  and $V$ be Banach spaces.
Suppose $T:U\rightarrow V$ is a continuous injective 
linear map
such that $T(U)$ is dense in $V$.
Show that the following three statements are equivalent.

(i) There exists
a $c>0$ with the property that
\[\|Tu\|_{V}\geq c\|u\|_{U}\]
for all $u\in U$.

(ii) $T$ is surjective.

(iii) $T$ is invertible (that is to say, has continuous inverse).
\end{exercise}

\begin{exercise}\label{C4.5}\label{E;approximate eigenvalue} 
(i) Consider the the shift map $S:l^{2}\rightarrow l^{2}$
given by
\[S(a_{1},a_{2},\dots)=(0,a_{1},a_{2},\dots).\] 
Show that $S$ is a well defined continuous linear map
and find its adjoint.

(ii) Let $U$ be a finite dimensional space and $T:U\rightarrow U$
a linear map. Explain why $\lambda I -T$ fails to be invertible
if and only if there exists a $u\neq 0$ such that $Tu=\lambda u$.

(iii) Give an example of a 
continuous linear map $R_{1}:l^{2}\rightarrow l^{2}$
which is surjective but not bijective.
Give an example of a 
continuous linear map $R_{2}:l^{2}\rightarrow l^{2}$
which is injective but not bijective.

(iv) By using Exercise~\ref{E;dense surjective}, or otherwise, show that,
if $U$ is a Banach space and $T:U\rightarrow U$
is a continuous linear map, then $\lambda\in\sigma(T)$ if and only if
at least one of the  following conditions hold.

(A) $\lambda$ is an eigenvalue of $T$, that is to say, there exists
a $u\neq 0$ such that $Tu=\lambda u$.

(B) $\lambda$ is not an eigenvalue of $T$ but
$\lambda$ is an \emph{approximate eigenvalue} of $T$,
that is to say, there exist
$u_{j}\in U$ with $\|u_{j}\|=1$ such that 
$\|Tu_{j}-\lambda u_{j}\|\rightarrow 0$.

(C) $(\lambda I-T)(U)$ is not dense in $V$.

(v) Give an example of a continuous linear map
$T:l^{2}\rightarrow l^{2}$ such that $\sigma(T)=\{0\}$ but $0$
is not an eigenvalue of $T$.
 
\noindent[The moral of this question is that the spectrum 
is much  more a complicated
than at first appears. We find an interesting spectrum in 
Exercise~\ref{E;shift spectrum}.]
\end{exercise}
\begin{exercise}\label{C4.6} 
(i) Let  $U$ be a Banach space and $T:U\rightarrow U$
a continuous linear map.
Suppose that $\lambda$ is in the \emph{frontier}
of $\sigma(T)$ (that is to say $\lambda$ lies in $\sigma(T)$
and in the closure of the complement of  $\sigma(T)$). By
examining possibility (C) in Exercise~\ref{E;approximate eigenvalue},
show that $\lambda$ is an approximate eigenvalue
(this includes the possibility that $\lambda$ is an eigenvalue).
In other words, the frontier of the spectrum is composed
of approximate eigenvalues. 

(ii) Now suppose $T:H\rightarrow H$ is a continuous
Hermitian (that is to say, self-adjoint) linear map.
Show that the approximate eigenvalues of $T$ are real.
Deduce from~(i) that the spectrum of $T$ consists of
real numbers.
\end{exercise}
\begin{exercise}\label{E;compact as limit}\label{C4.7} 
(This quite long, though instructive. Those who want a shorter,
thought still non-trivial, question should ignore parts~(i)
and~(iii), and
do that part of~(ii) which says that the limit of finite rank operators
is compact.)

Consider
Hilbert space $H$. 
Let us write
$B(H)$ for the space of continuous linear maps
$T:H\rightarrow H$.

(i) Show that the following statements about
a bounded sequence $x_{n}$ are equivalent.

\ \ (A) Every subsequence of the sequence has a convergent 
subsequence.

\ \ (B) Given any $\epsilon>0$ we can find a finite 
dimensional subspace $E$ such that, if $P$ is the orthogonal
projection onto $E$, we have $\|(I-P)x_{n}\|_{2}<\epsilon$. 

(ii) We say that an $S\in B(H)$ is of finite rank if
the image space $S(H)$ is finite  dimensional.
Show that $T\in B(H)$ is compact if and only if
we can find finite rank $S_{n}$ such that
$\|S_{n}-T\|\rightarrow 0$ as $n\rightarrow\infty$.

\noindent[It may be helpful to recall that $H$ is separable]

(iii) Show that the collection of compact operators is a closed
nowhere dense (ie having dense complement)
subset of $B(H)$.
\end{exercise}
\begin{exercise}\label{C4.8} 
(i) We work on the Hilbert space $l^{2}$. Show that
\[S(a_{1},a_{2},a_{3},\dots)=(a_{2},a_{3},a_{4}\dots)\]
is a continuous linear map from $l^{2}$ to $l^{2}$
and find its adjoint map $S^{*}$. Find $\|S\|$ and $\|S^{*}\|$.

(ii) We use the notation of Exercise~\ref{E;compact as limit}.
Show that the collection of self-adjoint operators is a closed
nowhere dense subset of $B(H)$.
\end{exercise}
\begin{exercise}\label{E;spectral projection}\label{C4.9} 
Check that the following statement is equivalent to the spectral
theorem for compact self-adjoint operators
(Theorem~\ref{T;spectral}).
Let $H$ be a Hilbert space. If
$T:H\rightarrow H$ is a continuous linear compact self-adjoint map
we can find a finite or infinite sequence $P_{1}$, $P_{2}$,
\dots of orthogonal projections (that is to say
continuous linear maps $P_{j}:H\rightarrow H$ with
$P_{j}^{-1}(\{0\})\perp P_{j}(H)$) which are mutually
orthogonal
(that is to say $P_{j}(H)\perp P_{k}(H)$ for $j\neq k$)
and have finite dimensional non-trivial images (that is to say
$1\leq \dim P_{j}(H)<\infty$) together with
distinct non-zero real $\lambda_{j}$, with
$\lambda_{n}\rightarrow 0$ as $n\rightarrow\infty$
if the sequence is infinite, such that
\[T=\sum_{j}\lambda_{j}P_{j}.\]

\noindent[Although this is just a simple rewrite of
Theorem~\ref{T;spectral} it provides a better jumping off point
for generalisation.]
\end{exercise}
\begin{exercise}\label{C4.10} Suppose
$T:H\rightarrow H$ is a compact self-adjoint operator
on a Hilbert space $H$.
Let $\lambda$ be a non-zero real number and let 
${\mathbf y}\in H$.
Consider the two equations
\begin{gather*}
T{\mathbf x}=\lambda{\mathbf x},\tag{1}\\
T{\mathbf x}=\lambda{\mathbf x}+{\mathbf y}.\tag{2}
\end{gather*}
Prove the \emph{Fredholm alternative} which states that 
\emph{either} (1) has no non-zero solutions and (2) has
a unique solution \emph{or} the space of solutions
$E_{\lambda}$ of (1) contains non-zero vectors and (2) has solutions
if and only if ${\mathbf y}\perp E_{\lambda}$.
In the second case, if ${\mathbf z}$ is a solution of (2),
the set of solutions is
\[\{{\mathbf z}+{\mathbf w}\,:\,{\mathbf w}\in E_{\lambda}\}.\]

\noindent[Hint: If this was a 1B mathematical methods question you
would consider it rather easy.]   
\end{exercise}
\begin{exercise}\label{C4.11} 
Work through Section~\ref{S;use spectral}.
\end{exercise}
\begin{exercise}\label{C4.12}
In this exercise we work in a Hilbert space $H$.

(i) Suppose that $T$ and $S$ are commuting compact self-adjoint
continuous maps from $H$ to $H$. If we write
\[E_{\lambda}=\{x\in H\,:Tx=\lambda x\}\]
show that $S(E_{\lambda})\subseteq E_{\lambda}$. By repeated
use of Theorem~\ref{T;spectral}, or otherwise, show that
we can find an orthonormal basis $f_{1}$, $f_{2}$,
$f_{3}$, \dots such that each
$f_{j}$ is an eigenvector of both $T$ and $S$.

(ii) A continuous linear map $R:H\rightarrow H$ is called \emph{normal}
if $RR^{*}=R^{*}R$. By considering $T=(R+R^{*})/2$
and $S=(R-R^{*})/(2i)$, or otherwise show that,
if $R$ a compact normal continuous linear map
from $H$ to $H$,  we can find
an orthonormal basis $f_{j}$ of $H$ and $\lambda_{j}\in{\mathbb C}$
with $\lambda_{j}\rightarrow 0$ such that
\[Ru=\sum_{j=1}^{\infty}\lambda_{j}\langle u,f_{j}\rangle f_{j}\]
for all $u\in U$.

(iii) Show conversely that if $f_{j}$ is any orthonormal basis 
of of $H$ and $\lambda_{j}$ any sequence in ${\mathbb C}$ with
$\lambda_{j}\rightarrow 0$ then the formula
\[Ru=\sum_{j=1}^{\infty}\lambda_{j}\langle u,f_{j}\rangle f_{j}\]
for all $u\in U$ defines a continuous linear map $R:H\rightarrow H$
which is compact and normal.
\end{exercise} 
\section{Supplementary exercises} These exercises are not intended to be harder
but are less closely linked 
to the immediate needs of the course. 
They do, however,
provide background and anyone intending to do more analysis should at least
glance at them.

\begin{exercise}\label{C5.1} (Revision of 1A)
We say that a function 
$f:[a,b]\rightarrow{\mathbb R}$ is strictly concave if, whenever
$a\leq t<s\leq b$ and $0<\lambda<1$, it follows that
\[\lambda f(t)+(1-\lambda) f(s)<f\big(\lambda t+(1-\lambda)s).\]

(i) Show that, if $g:[a,b]\rightarrow{\mathbb R}$ is twice differentiable
with $g''(t)<0$ for all $t\in(a,b)$, then $g$ is strictly concave.

(ii) Give an example of a strictly concave function
$g:[-1,1]\rightarrow{\mathbb R}$ which is not differentiable
at $0$.

(iii) Suppose that $f:[a,b]\rightarrow{\mathbb R}$ is strictly concave.
By using induction, or otherwise, prove
Jensen's inequality which states that
if $x_{1}$, $x_{2}$, \dots, $x_{n}$ are distinct points of
$[a,b]$ and $\lambda_{1}$, $\lambda_{2}$, \dots, $\lambda_{n}$
are strictly positive real numbers with 
$\sum_{j=1}^{n}\lambda_{j}=1$, then
\[f\left(\sum_{j=1}^{n}\lambda_{j}x_{j}\right)
>\sum_{j=1}^{n}\lambda_{j}f(x_{j}).\]

Deduce that if $x_{1}$, $x_{2}$, \dots, $x_{n}$ are points of
$[a,b]$ and $\lambda_{1}$, $\lambda_{2}$, \dots, $\lambda_{n}$
are strictly positive real numbers with
$\sum_{j=1}^{n}\lambda_{j}=1$, then
\[f\left(\sum_{j=1}^{n}\lambda_{j}x_{j}\right)
=\sum_{j=1}^{n}\lambda_{j}f(x_{j})\]
if and only if $x_{1}=x_{2}=\dots=x_{n}$.

(iv) Use Jensen's inequality to show that, if $a_{j}>0$, then
\[(a_{1}a_{2}\dots a_{n})^{1/n}\leq \frac{a_{1}+a_{2}+\dots+a_{n}}{n}.\]
(This is Cauchy's arithmetic-geometric inequality.)
What are the conditions for equality?

 
(v) Suppose that $p>1$ and let $g(x)=(1+x^{1/p})^{p}$.
Show that $g$ is a concave function.

Suppose that $a_{1},\,a_{2},\,\dots,\,a_{n}>0$,
$\sum_{j=1}^{n}a_{j}^{p}=1$
and $b_{1},\,b_{2},\,\dots,\,b_{n}>0$.
By applying Jensen's inequality with
$x_{k}=b_{k}^{p}/a_{k}^{p}$ and $\lambda_{k}$
chosen appropriately, prove Minkowski's inequality.
\[\left(\sum_{j=1}^{n}(a_{j}+b_{j})^{p}\right)^{1/p}
\leq \left(\sum_{j=1}^{n}a_{j}^{p}\right)^{1/p}
+\left(\sum_{j=1}^{n}b_{j}^{p}\right)^{1/p}\]
and obtain the conditions for equality.
Why does the result 
fllow for general values of $\sum_{j=1}^{n}a_{j}^{p}$?
\end{exercise}
\begin{exercise}\label{C5.2} Obtain Minkowski's inequality
by applying H{\"o}lder's inequality to the observation
\[\sum_{j=1}^{n}|x_{j}+y_{j}|^{p}
\leq \sum_{j=1}^{n}|x_{j}||x_{j}+y_{j}|^{p-1}
+\sum_{j=1}^{n}|y_{j}||x_{j}+y_{j}|^{p-1}.\]
Is this really a different proof to the one given
in the lectures using the reverse H{\"o}lder inequality?
\end{exercise}
\begin{exercise}\label{C5.3} The results of Exercise~\ref{E;increase norm}
depend on clever inequalities\footnote{To the writer
all inequalities seem clever.} but there are other
ways of arriving at the results.

Let $\infty\geq s>r\geq 1$.
Investigate maxima and minima of $\sum_{j=1}^{n}x_{j}^{s}$
subject to $x_{j}\geq 0$,
$\sum_{j=1}^{n}x_{j}^{r}=1$ using the calculus of variations.
(Unless we take care, which you are not asked to do,
the results will not be rigorous
but, once we know what is happening, it is much easier
to prove that it happens by some other technique.)
\end{exercise}
\begin{exercise}\label{C5.4} 
If $V$ is a vector space over ${\mathbb F}$,
we say that $E$ is an \emph{algebraic basis}
(that is to say a basis in the sense of~1B algebra)
if every $v\in V$ can be written uniquely as a finite sum
\[v=\sum_{j=1}^{n}\lambda_{j}e_{j}\]
with $e_{1}$, $e_{2}$, \dots, $e_{n}$ distinct
elements of $E$ and $\lambda_{j}\in{\mathbb F}$.
The collection $V^{*}$ of linear maps $\alpha:V\rightarrow{\mathbb F}$
is called the \emph{algebraic dual} 
(that is to say, the dual space in the sense of~1B algebra).
The proofs of ~1B algebra show that $U^{*}$ can be given the
structure of a vector space.

Let $c_{00}$ be the vector space
of complex sequences with only a finite number of 
non-zero terms. Explain why $c_{00}$ has a countable basis.
Identify $c_{00}^{*}$ in a natural manner with the
space ${\mathbb C}^{\mathbb N}$ of all 
complex sequences. Show that $c_{00}^{*}$
does not have a countable basis. (The argument
is not difficult but you should not sleep walk
through it.)

Although this question deals only with one space the
reader should require little convincing that, if we
only deal with algebraic duals, the algebraic dual of
an infinite dimensional space will be very much bigger
than the the space (and the dual of the dual will be even bigger).
\end{exercise}  
\begin{exercise}\label{E;not Hilbert}\label{C5.5} 
(i) Prove the parallelogram law
\[\|{\mathbf a}+{\mathbf b}\|_{2}^{2}
+\|{\mathbf a}-{\mathbf b}\|_{2}^{2}
=2(\|{\mathbf a}\|_{2}+{\mathbf b}\|_{2}^{2})\]
for all ${\mathbf a},\,{\mathbf b}\in l^{2}$.

(ii) Use induction to show that, for each $n$, we can find
$\zeta_{jk}(n)=\pm 1$ such that
\[\sum_{j=1}^{2^{n}}\left\|\sum_{k=1}^{2^{n}}
\zeta_{jk}(n){\mathbf a}(k)\right\|_{2}^{2}
=2^{n}\sum_{k=1}^{2^{n}}\|{\mathbf a}(k)\|_{2}^{2}\]
for all ${\mathbf a}(k)\in l^{2}$.

(iii) If $(U,\|.\|)$ is isomorphic to $(l^{2},\|.\|_{2})$
explain why there is a constant $K$ independent of $n$
such that
\[K\sum_{j=1}^{2^{n}}\left\|\sum_{k=1}^{2^{n}}
\zeta_{jk}(n){\mathbf u}(k)\right\|_{U}^{2}
\geq 2^{n}\sum_{j=1}^{2^{n}}\|{\mathbf u}(k)\|_{U}^{2}
\geq K^{-1}\sum_{j=1}^{2^{n}}\left\|\sum_{k=1}^{2^{n}}
\zeta_{jk}(n){\mathbf u}(k)\right\|_{U}^{2}\]
for all ${\mathbf u}(k)\in U$.

(iv) Show that $l^{2}$ is not isomorphic to $l^{p}$
when $p\neq 2$.
\end{exercise}
\begin{exercise}\label{C5.6}
Consider the space of $n\times n$ complex
matrices with the operator norm.
Prove the Cayley--Hamilton theorem by using
the fact proved in Exercise~\ref{E;dense matrices}
that the set of of matrices 
with $n$ distinct eigenvalues
is dense.
\end{exercise}
\begin{exercise}~\label{E;kernel 2}\label{C5.7} 
Suppose that 
$g_{n}:{\mathbb T}\rightarrow{\mathbb R}$
is continuous and satisfies the conditions
set out in Exercise~\ref{E;kernel 1} as follows.

(i) There exists a constant $K$
such that
\[\frac{1}{2\pi}\int_{\mathbb T}|g_{n}(t)|\,dt\leq K\]
for all $n\geq 1$.

(ii) If $\delta>0$ and $f$ is a continuous function
with $f(t)=0$ for $|t|<\delta$, then
\[\frac{1}{2\pi}\int_{\mathbb T}f(t)g_{n}(t)\,dt\rightarrow 0\]
as $n\rightarrow \infty$.

(iii) We have
\[\frac{1}{2\pi}\int_{\mathbb T}g_{n}(t)\,dt\rightarrow 1\]
as $n\rightarrow\infty$.

If $f:{\mathbb T}\rightarrow{\mathbb R}$ is continuous
and $\delta>0$ observe that
\begin{align*}
\frac{1}{2\pi}&\int_{\mathbb T}g_{n}(t)f(t)\,dt-f(0)
=\frac{1}{2\pi}\int_{|t|<\delta}g_{n}(t)(f(t)-f(0))\,dt\\
&+\frac{1}{2\pi}\int_{|t|\geq\delta}g_{n}(t)(f(t)-f(0))\,dt
+\left(\frac{1}{2\pi}\int_{\mathbb T}g_{n}(t)\,dt-1\right)\times f(0),
\end{align*}
and, by estimating the three terms separately,
show that
\[\frac{1}{2\pi}\int_{\mathbb T}g_{n}(t)f(t)\,dt
\rightarrow f(0)\]
as $n\rightarrow\infty$.

Show that condition~(ii) is implied by

(ii)$'$ If $\delta>0$ then $g_{n}(t)\rightarrow 0$ uniformly
for $t\notin(-\delta,\delta)$.

\noindent Show also, by considering the Riemann--Lebesgue lemma
or otherwise, that condition~(ii) does not imply (ii)$'$.



\end{exercise}
\begin{exercise}\label{C5.8}\label{E;kernel 3} Let 
$f:{\mathbb T}\rightarrow{\mathbb C}$
be continuous. We write
\[\hat{f}(n)=\frac{1}{2\pi}\int_{\mathbb T}f(t)\exp(-int)\,dt.\]

(i) Show that $\sum_{n=-N}^{N}r^{|n|}\exp(int)$ converges
uniformly, to $P_{r}(t)$ say, for $t\in{\mathbb T}$ as $N\rightarrow\infty$
for each fixed $r$ with $0<r<1$. Deduce that
\[\sum_{n=-\infty}^{\infty}r^{|n|}\hat{f}(n)
=\frac{1}{2\pi}\int_{\mathbb T}f(t)P_{r}(-t)\,dt
=\frac{1}{2\pi}\int_{\mathbb T}f(t)P_{r}(t)\,dt.\]
Show that
\[P_{r}(t)=\frac{1-r^{2}}{1-2r\cos t+r^{2}}.\]

(ii) Show that

(A) $P_{r}(t)\geq 0$ for all $t$.

(B) ${\displaystyle\frac{1}{2\pi}\int_{\mathbb T}P_{r}(t)\,dt=1}$.

(C) $P_{r}(t)\rightarrow 0$ uniformly as $r\rightarrow 1-$ 
for $t\notin(-\delta,\delta)$.

(iii) Deduce, by the arguments of Exercise~\ref{E;kernel 2},
or otherwise, that
\[\sum_{n=-\infty}^{\infty}r^{n}\hat{f}(n)\rightarrow f(0)\]
as $r\rightarrow 1-$. By considering the Fourier coefficients
of $f_{a}$ given by $f_{a}(t)=f(t-a)$, or otherwise, show that
\[\sum_{n=-\infty}^{\infty}r^{n}\hat{f}(n)\exp(int)\rightarrow f(t)\]
as $r\rightarrow 1-$.
\end{exercise}
\begin{exercise}\label{E;kernel 4}\label{C5.9} 
Let $f:{\mathbb T}\rightarrow{\mathbb C}$
be continuous. We write
\[\hat{f}(n)=\frac{1}{2\pi}\int_{\mathbb T}f(t)\exp(-int)\,dt.\]

(i) We write $D_{n}(t)=\sum_{j=-n}^{n}\exp(ijt)$.
Show that, if $t\neq 0$,
\[D_{n}(t)=\frac{\sin\big((n+\tfrac{1}{2})t\big)}
{\sin\tfrac{t}{2}}.\]
What is the value of $D_{n}(0)$?

(ii) Show that
\[\sum_{j=-n}^{n}\hat{f}(j)
=\frac{1}{2\pi}\int_{\mathbb T}f(t)D_{n}(-t)\,dt
=\frac{1}{2\pi}\int_{\mathbb T}f(t)D_{n}(t)\,dt.\]

(iii) Show that
\[|D_{n}(t)|\geq \frac{A(n+\tfrac{1}{2})}{r+1}
|\sin\big((n+\tfrac{1}{2})t\big)|\]
for all $r\pi/(n+\tfrac{1}{2})\leq |t|\leq (r+1)\pi/(n+\tfrac{1}{2})$
and $1\leq r\leq n-1$ for some $A>0$ (to be chosen to fit your convenience)
independent of $n$.

(iv) Deduce that
\[\frac{1}{2\pi}\int_{\mathbb T}|D_{n}(t)|\,dt\geq B\log n\]
for all $n\geq N$ and some $B>0$ (where both $N$ and $B$
are chosen to fit your convenience).

(v) Use Exercise~\ref{E;kernel 1} to show that
there exists a continuous function $f$ whose Fourier sum
does not converge to $f$ at $0$.
\end{exercise}


\begin{exercise}\label{C5.10} Let $(X,\tau)$ be a topological space.
Write $x\sim y$ if $f(x)=f(y)$ whenever $f\in C(X)$ the space
of real valued continuous functions on $X$.

(i) Show that $\sim$ is an equivalence relation.

(ii) Show that, if we give $X/\!\sim$ the quotient topology
$\tau/\!\sim$, then the function $J:C(X)\rightarrow C(X/\!\sim)$
given by $J(f)[x]=f(x)$ (where $[x]$ denotes the
equivalence class of $x$) is a well defined linear isometry.

(iii) Show that  $(X/\!\sim,\tau/\!\sim)$ is Hausdorff.

(iv) If $(X,\tau)$ is compact, show that $(X/\!\sim,\tau/\!\sim)$
is also compact.
\end{exercise}
\begin{exercise}\label{E;normal one}\label{C5.11} 
In this exercise we discuss various 
\emph{separation axioms} for topological spaces $(X,\tau)$.


A topological space $(X,\tau)$
is said to be $T_{0}$ if, given
$a,\,b\in X$ with $a\neq b$,
at least one of these statements is true.
($\alpha$) There exists a $U\in\tau$ such that $a\in U$
and $b\notin U$. ($\beta$) There exists a $V\in\tau$ such that $a\notin V$
and $b\in V$. A topological space $(X,\tau)$ is called $T_{1}$
if given
$a,\,b\in X$ with $a\neq b$,
at both of these statement are true.

(i) Show that $(X,\tau)$ is $T_{1}$ if and only if 
every singleton set $\{x\}$ is closed.

(ii) Consider ${\mathbb N}$. If the elements of $\sigma_{0}$ are
$\emptyset$ and sets of the form
$\{m\in{\mathbb N}\,:\,m\geq n\}$ show that $\sigma_{1}$
is a topology and $({\mathbb N},\sigma_{1})$ is $T_{0}$
but not $T_{1}$.

(iii) Consider ${\mathbb N}$. If the elements of $U\in\sigma_{1}$
if and only if $A=\emptyset$ or ${\mathbb N}\setminus A$ is finite
show that $\sigma_{1}$
is a topology and $({\mathbb N},\sigma_{1})$ is $T_{1}$
but not Hausdorff. (In this classification, Hausdorff spaces
are
called $T_{2}$.)

(iv) Show that every $T_{1}$ normal space is Hausdorff
but give an example of a normal space which is not
Hausdorff. (Recall that a topological space is said to be
normal if,
given $A$ and $B$ non-empty disjoint closed sets,
we can find disjoint open sets $U$ and $V$ such that
$A\subseteq U$ and $B\subseteq V$.)

\noindent[There are very small and simple examples.]

(v) Carry out us much of the proof of Urysohn's lemma
and the Tietze extension theorem as you need\footnote{Either
do the whole thing as a revision exercise or just cast your eyes
over your notes.}
to show that Urysohn's lemma holds with the hypothesis `compact and Hausdorff'
replaced by `normal' and the Tietze extension theorem
holds in the following form:-

If $Y$ is closed subset of a normal topological space
$(X,\tau)$, then, given any bounded continuous real valued
function $f$ on $Y$
(where $Y$ has the subspace topology), we can find
a bounded continuous real valued
function $F$ on $X$ such that $F(y)=f(y)$ for all $y\in Y$.
\end{exercise}
\begin{exercise}\label{E;parallelepiped}\label{C5.12} 
We work with a real 
vector space to simplify the algebra but similar
results can be obtained in the complex case.

(i) Consider a real inner product space. 
Obtain the formulae
\[4\langle u,v\rangle=\|u+v\|_{2}^{2}-\|u-v\|_{2}^{2}\]
and
\[\|u+v\|_{2}^{2}+\|u-v\|_{2}^{2}
=2\big(\|u\|_{2}^{2}+\|v\|_{2}^{2}\big).\]

(ii) Suppose that $V$ is a real normed vector space
such that
\[\|u+v\|^{2}+\|u-v\|^{2}=2\big(\|u\|^{2}+\|v\|^{2}\big).\]
Show that if we set
\[p(u,v)=(\|u+v\|^{2}-\|u-v\|^{2})/4\]
then
\[p(u+w,v)=p(u,v)+p(w,v).\]

\noindent[It may or may not be helpful to consider
the parallelepiped associated with $u$, $v$ and $w$.]

(iii) Continuing with the hypotheses and notation
of (ii), show that 
\[p(nu,v)=np(u,v)\] 
for $n$ a strictly positive integer.
Deduce that $p(\lambda u,v)=\lambda p(u,v)$ when $\lambda$ is rational.
Show that $p$ is an inner product and $\|.\|$ is the norm associated 
with $p$.
\end{exercise}
\begin{exercise}\label{C5.13}
In some of the exercises we used metrics rather than norms.
This exercise
shows why the need may arise. 
We work in $C^{\infty}([0,1])$.

(i) Show that 
$d(f,g)=\sum_{j=0}^{\infty}2^{-j}\min(1,\|f^{(j)}-g^{(j)}\|_{\infty})$
is a metric. Show that $f_{n}^{(j)}\rightarrow f^{(j)}$
uniformly on $[0,1]$ for each $j$ if and only if  $d(f_{n},f)\rightarrow 0$.
Show that $d$ is a complete metric.

(ii) The object of the rest of this question is to show that there is no norm
with this property. To this end, suppose 
that $\|.\|$ is a norm on $C^{\infty}([0,1])$.
such that  $\|f_{n}-f\|\rightarrow 0$ implies $f_{n}^{(j)}\rightarrow f^{(j)}$
uniformly on $[0,1]$ for each $j$. Show, by reductio ad absurdum, or otherwise,
that there must exist $\epsilon_{j}>0$ with 
the property that $\|f^{(j)}\|_{\infty}\geq 1$
implies $\|f\|\geq\epsilon_{j}$.

(iii) Show that we can find a sequence of $f_{n}\in C^{\infty}([0,1])$
such that
\begin{align*}
\|f^{(j)}_{n}\|_{\infty}&\leq 2^{-n}\ \text{for $0\leq j\leq n-1$}\\
\|f^{(n)}_{n}\|_{\infty}&\geq 2^{n}\epsilon_{n}^{-1}.
\end{align*}
Show that  $f_{n}^{(j)}\rightarrow f^{(j)}$
uniformly on $[0,1]$ for each $j$ but $\|f_{n}\|\rightarrow \infty$.
\end{exercise}
\begin{exercise}\label{C5.14}
Suppose that we have a sequence of metrics $d_{j}$ on a space $X$.

(i) Show that
\[d(x,y)=\sum_{j=1}^{\infty}
2^{-j}\min\big(1,d_{j}(x,y)\big)\]
is a metric on $X$ such that
the identity map $\iota:(X,d)\rightarrow (X,d_{j})$ is continuous
for each $j$.

(ii) Show that if $\rho$ is a metric on $X$ such that
the identity map $\iota:(X,\rho)\rightarrow (X,d_{j})$ is continuous
for each $j$ then $\iota:(X,\rho)\rightarrow (X,d)$ 
is continuous.
\end{exercise}
\begin{exercise}\label{C5.15}\label{E;anti}
(This easy exercise discusses anti-isomorphism\footnote{And shows that
there is not much to discuss}.)
If $(U,+,.,{\mathbb C})$ is a vector space show that, if we write
\[\lambda\bullet u=\lambda^{*}u,\]
then $(U,+,\bullet,{\mathbb C})$ is a vector space. Show that
$(U,+,.,{\mathbb C})$ and $(U,+,\bullet,{\mathbb C})$
are algebraically anti-isomorphic.
\end{exercise}

\begin{exercise}\label{C5.16}\label{E;shift spectrum} 
(i) (This is about difference equations in the sense of
the 1A differential equations course.)
If $\lambda\in{\mathbb C}$, $\lambda\neq 0$,
find the general solution of the difference equation
\[\lambda u_{n}-u_{n+1}=0 \ [n\in{\mathbb Z}].\]

(ii) Find the general solution of the
the difference equation.
\[\lambda u_{n}-u_{n+1}=0\ [n\in{\mathbb Z},\ n\neq 0].\]
(Note that there are two arbitrary constants.)

(iii) Find the general solution of the
the difference equation.
\[
\lambda u_{n}-u_{n+1}
=\begin{cases}0&\text{if $n\neq 0$}\\
1&\text{if $n=0$.}
\end{cases}
\]

(iv) From now on we work in $l^{2}({\mathbb Z})$ the vector
space of \emph{two sided} square summable sequences of complex numbers
\[{\mathbf a}=(\dots,a_{-2},a_{-1},a_{0},a_{1},a_{2},\dots)\]
with norm 
\[\|{\mathbf a}\|_{2}
=\left(\sum_{j=-\infty}^{\infty}|a_{j}|^{2}\right)^{1/2}.\]
Show that, if $0<|\lambda|<1$,
there is exactly one ${\mathbf u}\in l^{2}({\mathbb Z})$
such that
\[
\lambda u_{n}-u_{n+1}
=\begin{cases}0&\text{if $n\neq 0$}\\
1&\text{if $n=0$},
\end{cases}
\] 
and that the same is true if $|\lambda|>1$.

(iv) Show that the shift operator $S$ defined by $S{\mathbf a}={\mathbf b}$
with $b_{j}=a_{j+1}$ is a well defined continuous linear map from
$l^{2}({\mathbb Z})$ to itself. Show that, if $|\lambda|\neq 1$,
then $\lambda\notin\sigma(S)$.

(v) Show that $1$ is not an eigenvalue of $S$. By considering
${\mathbf v}$ with
\[
v_{n}=
\begin{cases}N^{-1/2}&\text{if $1\leq n\leq N$}\\
0&\text{otherwise,}
\end{cases}
\] 
show that $1$ is an approximate eigenvalue of $S$.

(vi) Show that
\[\sigma(S)=\{\lambda\in{\mathbb C}\,:\,|\lambda|=1\},\]
that all the points of $\sigma(S)$ are approximate eigenvalues
and that none of them are eigenvalues.
\end{exercise}
\begin{exercise}\label{C5.17}\label{E;non-separable Hilbert}
The object of this exercise is to exhibit a complete inner product
space which is not separable.

(i) Let $X$ be a non-empty set and $a:X\rightarrow{\mathbb R}$
a function such that $a(x)\geq 0$ for all $x\in X$ and
such that there exists a $K$ with
\[\sum_{x\in F}a(x)\leq K\]
whenever $F$ is a finite subset of $X$. Show that
\[Y=\{x\in X\,:\,a(x)\neq 0\}\]
is countable.

Using the fact that the sum of a sequence of positive
numbers is unaffected by the order of summation, this means
that we can define
\[\sum_{x\in X}a(x)=\sum_{x\in Y}a(x).\]

Suppose $b:X\rightarrow{\mathbb R}$
a function such that $b(x)\geq 0$ for all $x\in X$.
In what follows we say that $\sum_{x\in X}b(x)$ converges
if $\sum_{x\in F}b(x)\leq K$
whenever $F$ is a finite subset of $X$ and we write
\[\sum_{x\in X}b(x)=\sum_{b(x)\neq 0}b(x).\]

(ii) If $X$ is a non-empty set and $a:X\rightarrow{\mathbb C}$,
we say that $a\in l^{2}(X)$ if $\sum_{x\in X}|a(x)|^{2}$ converges.
Show that  $l^{2}(X)$ is a vector space. Show further that
\[\langle a,b\rangle=\sum_{x\in X}a(x)b(x)^{*}\]
gives a well defined inner product on $l^{2}(X)$.

(iii) Show that $l^{2}(X)$ is separable if and only if $X$
is countable.

(iv) Show that $l^{2}(X)$ is complete.
\end{exercise}
