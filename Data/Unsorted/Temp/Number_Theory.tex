\section{Introduction and divisibility}

\subsection{Introduction and examples}

We first look at some examples of conjectures:
\begin{enumerate}
\item Let $N$ be any integer $\ge$ 1 of the form $8n+5,8n+6,8n+7$. Then there always exists a right angle triangle of area $N$, all of whose size has rational length.
\item If $x$ is a rational number, $x \ge 2$, write $\pi(x)=$ number of primes $\le x, li(x)=\int_2^x \frac{dt}{\log{t}}$. Then $\forall x \ge 3, |\pi(x) - li(x)| \le \sqrt{(x)} \log{x}$.
\item There are infinitely many primes $p$ such that $p+2$ is also a prime.
\end{enumerate}
A famous example which has only been solved in the last twenty years is Fermat's Last theorem, by Andrew Wiles, which says that $\forall n \ge 3, x^n + y^n =z^n$ has no non-obvious integer solutions. i.e. one of $x, y, z$ must be $0$.

These are all examples of number theory which may or may not require a large amount of knowledge from other area of mathematics, e.g. Elliptic Curve, Modular Form, etc.

Now for the first few Chapters, we will concentrate more on analytic number theory and basic background of group theory is needed. Details will be explained later.

\subsection{Divisibility}

\begin{theorem} 
Let $a,b \in \mathbb{Z}, a>0$, then $\exists q,r \in \mathbb{Z}$ such that $b=aq+r$, where $0 \le r <a$.
\end{theorem}

\begin{proof}[\bf Proof] Let $S =\{b - na: n \in \mathbb{Z}\}$. S contains a least non-zero integer, say $r$, and let $r=b-qa$. Then $r < a$ because if $r \ge a$ then $r-a =b - (q+1)a \in S$, which contradicts $r$ being least in $S$. This is a special case of Euclidean algorithm in group theory, which we will discuss later.
\end{proof}

\begin{corollary} 
Given integers $a_1,a_2 \ldots a_n$, not all $0$. Let $I =\{a_1 x_1 + \ldots +a_n x_n: x_i \in \mathbb{Z}\}$. Then there exists $d>0$, such that $I = d \mathbb{Z}$.
\end{corollary}

\begin{proof}[\bf Proof] 
It is easy to see that $I$ is an ideal. $I$ contains integer $>0$. Let $d$ be the least integer $>0$ in $I$, and claim that $I= d \mathbb{Z}$. Take any $c$ in $I$, we have $c= qd + r, 0 \le r < d$ by Euclidean algorithm. Then $r = c- qd \in I$, and hence $r=0$ by definition of $d$. Hence $I =d \mathbb{Z}$.
\end{proof}

\begin{definition} 
We call such $d$ above the greatest common divisor of $a_1,a_2 \ldots a_n$, and write $d=(a_1, \ldots a_n)$.
\end{definition}

\begin{flushleft}The efficient way of calculating $d=(a,b), b>a>0$ is as follows:\\
$b=aq_1 + r_1, 0 \le r_1 <a$.\\
$a=r_1 q_2 + r_2, 0 \le r_2 < r_1$.\\
$\vdots$\\
Until $\exists n$, $r_{n-1}=r_n q_{n+1}$.\\
Then it is easy to check that $r_n=(a,b)=d$.
\end{flushleft}

\begin{example} Find the greatest common divisor of $45$ and $36$. we have
$45 = 36 + 9$, $36 = 9 \cdot 4$. And hence the greatest common divisor is $9$.
\end{example}

Hence we may use Euclidean algorithm to find solutions of equations of the form:
\be
a \cdot x + b \cdot y =c.
\ee

In the example above, $a = 45, b = 36$. Let $c = 9$ then we have $x = 1, y = -1$. In fact:

\begin{theorem} Suppose $n \neq 0$, then the equation $ax+by=n$ is soluble if and only if $(a,b)|n$. Moreover, if $(a,b) | n$. The general solution of the equation $ax+by=n$ has the form $\{x_0+ \frac{tb}{(a,b)}, y_0 - \frac{ta}{(a,b)}\}$, where $t \in \mathbb{Z}$ and $\{x_0,y_0\}$ is one solution of the equation above.
\end{theorem}

\begin{proof}[\bf Proof] 
Suppose $(a,b) \nmid n$, then we have no solutions. Suppose now $(a,b)|n$, as by Euclid's algorithm we can find integers $x,y$ such that $ax_1+by_1=(a,b)$. Then let $x=x_1 \frac{n}{(a,b)},y=y_1 \frac{n}{(a,b)}$, and so we have a solution.

Now it is clear that integers in the above set are indeed solutions of the equation. Now suppose $\{x_1,y_1\}, \{x_1+u, y_1 -v\}$ are both solutions of the equations. Then $au=bv$, and so $\frac{a}{(a,b)}u=\frac{b}{(a,b)}v$. Hence $\frac{b}{(a,b)}|u$. Thus $u \ge \frac{b}{(a,b)}$. And hence the result.
\end{proof}

\begin{lemma} 
Let $p$ be a prime. If $p | a b$, then $p | a$ or $p | b$.
\end{lemma}

\begin{proof}[\bf Proof] 
Suppose $p \nmid a$. Then $(a, p)=1$, and so $\exists x, y$ such that $a  x + p  y =1$. Then $a b x + p b y =b$.\\
And so as $p | a b$ and $p |a b y$. Therefore, $p$ divides RHS i.e. $p | b$.
\end{proof}

\subsection{Fundamental theorem}

\begin{theorem}[Fundamental Theorem]
Every integer $N > 1$ can be written as a product of primes, and this computation is unique up to reordering.
\end{theorem}

For example, we have $24 = 2 \cdot 2 \cdot 2 \cdot 3, 34 = 2 \cdot 17$.

\begin{proof}[\bf Proof]
We need to prove both existence and uniqueness of the theorem.
\begin{enumerate}
\item Existence: Let $p_1 \ldots p_k$ be the list of primes less than or equal to $\sqrt{N}$. $p_i < p_{i+1}$. To find a factorisation, start with $p_1$ and for each $p_i$, if $p_i$ divides $N$, then it is a factor of $N$ and then consider the new $N$ to be $\frac{N}{p_i}$, and continue again from $p_i$. If $p_i$ does not divide $N$, then move on to $p_{i+1}$. Continue this until we have run though every $p_i$.
\item Uniqueness: Suppose $N = p_1 p_2 \ldots p_r = q_1 q_2 \ldots q_s$, where $p_i, q_j$ are primes. Now use the lemma above, start from $p_1$, as $p_1 | q_1 (q_2 \cdots q_s)$, so $p_1 | q_1$ or $p_1 |q_2 \cdots q_s$.
Continue this and so we have $p_1 | q_j$ for some $j$. WLOG, we reorder so that $p_1 |q_1$. But $q_1$ is prime, and so we must have $p_1 = q_1$. Now remove $p_1, q_1$ in the above equation and continue this. Hence we conclude that $r = s$ and $p_i = q_i$ after suitable reordering.
\end{enumerate}
\end{proof}

The fundamental theorem gives a trivial way of finding the factorisation of $N$, but it is not efficient. {\bf Elementary Operation} is add or multiply any two elements of the set $\{0,1 \ldots 9\}$, carrying a remainder if necessary. For example, for any integer $M$ of $m$ digits, $R$ of $r$ digits, the multiplication $MR$ requires $2mr$ elementary operations. The algorithm is polynomial if the number of operations required is at most $a(\log N)^b, a ,b >0$. And it is still unknown whether there is algorithm which factorsises $N$ in polynomial time.

\subsection{Exercises}

\begin{enumerate}
\item Find integers $x, y$ such that $25x + 42y=1$.
\item Find integers $x, y, z$ such that $6x + 10y +15z =1$.
\item Let $n>1$ and $S_n = \frac{1}{2} + \frac{1}{3} + \cdots +\frac{1}{n}$, show that $S_n$ cannot be an integer.
\item Let $n$ be a positive integer, prove that $n$ can be written as a sum of (at least two) consecutive positive integers if and only if $n \neq 2^k$ for some $k$.
\item Let $a, b$ be positive integer such that $(a,b)=1$. Then show that $ax + by=n$ is soluble in non-negative integer $x$ and $y$ whenever $n \ge (a-1)(b-1)$.
\item Show that if $m > n$, then $2^{2^n}+1$ divides $2^{2^m}-1$ and so $(2^{2^m}+1, 2^{2^n} +1)=1$.\\
\end{enumerate}


\section{Arithmetical functions}

\subsection{Multiplicative functions}

We begin the section with some basic definitions

\begin{definition} 
A function that maps the positive integers into an integral domain or field is called a {\bf number theoretic} function.
\end{definition}

\begin{definition} 
A number theoretic function $f$ is said to be {\bf multiplicative} if $f(mn)=f(m)f(n)$ whenever $(m,n)=1$.
\end{definition}

\begin{definition} 
A number theoretic function $g$ is said to be {\bf completely multiplicative} if $g(mn)=g(m)g(n)$ without restriction.
\end{definition}

We will see several examples of multiplicative functions later.

\begin{theorem} 
If $f$ is a multiplicative function and \begin{equation*} g(n)=\sum_{d|n}f(d) \end{equation*} ,then $g$ is also multiplicative.
\end{theorem}

\begin{proof}[\bf Proof] 
Let $(m,n)=1$. A divisor of $mn$ can be uniquely written as $d=d_1 d_2$, where $d_1 |m, d_2 |n$ and also $(d_1,d_2)=1$ as $(m,n)=1$. Then we have:
\be
g(mn) = \sum_{d|mn}f(d) =  \sum_{d_1 |m, d_2 |n}f(d_1d_2) = \sum_{d_1 |m, d_2 |n}f(d_1)f(d_2)  =  \sum_{d_1|m}f(d_1)~\sum_{d_2|n}f(d_2) =  g(m)g(n)
\ee
\end{proof}

\begin{definition} 
The function $\tau$ and $\sigma$ are given by:
\begin{center}
$\tau(n) = \text{the number of positive divisors of n}$.\\
$\sigma(n) = \text{the sum of the positive divisors of n}$.
\end{center}
\end{definition}

\begin{lemma} 
$\tau$ and $\sigma$ are both multiplicative.
\end{lemma}

\begin{proof}[\bf Proof] Consider that \begin{equation*} \tau(n)=\sum_{d|n}1 \end{equation*} and as the constant function $1$ is multiplicative, and hence by Theorem 1.1 we have $\tau$ multiplicative. Similarly, as the identity function is multiplicative, and \begin{equation*} \sigma(n)=\sum_{d|n}d \end{equation*} so $\sigma$ is also multiplicative.
\end{proof}
One of the advantages of multiplicative functions is that we can use the property of it to give a general formula for the function itself. For example, let $p$ be a prime, and let $\alpha \ge 0$, we have:
\begin{equation*}
\tau(p^\alpha) = 1 + \alpha \text{ and } \sigma(p^\alpha)=1+p+p^2+\cdots +p^\alpha = \frac{p^{\alpha+1}-1}{p-1}
\end{equation*}
Hence for any $n$, we have $n=p_1^{\alpha_1} \ldots p_n^{\alpha_n}$, and use the lemma above, we have:
\begin{equation*}
\tau(n)=\prod_{i=1}^{k}(1+\alpha_i) \text{ and } \sigma(n) = \prod_{i=1}^{k} \frac{p_i^{\alpha_i+1} -1}{p_i-1}.
\end{equation*}
\begin{example} As an example, we calculate $\tau(30)$ and $\sigma(30)$.\\
Apply the formula above, we have $30=2 \cdot 3 \cdot 5$. Hence $\tau(30)= 2 \cdot 2 \cdot 2 =8$, and $\sigma(30) = 3 \cdot 4 \cdot 6 = 72$.
\end{example}
\subsection{Perfect number}
\begin{definition} A natural number $n$ is said to be a perfect number if $\sigma(n)=2n$
\end{definition}
Let's list the first several perfect numbers: 6, 28, 496, 8128. We will see the result below which characterises the even prefect numbers, yet it is still unknown whether there are infinitely many perfect numbers or if any odd perfect numbers exists.
\begin{lemma} Let $n$ be an even integer. Then $n$ is a perfect number if and only if $2^{p-1}(2^p-1)$ where both $p$ and $2^p -1$ are prime.
\end{lemma}
\begin{proof}[\bf Proof] If $n = 2^{p-1}(2^p-1)$ such that both $p$ and $2^p-1$ are primes, then we have $\sigma(n)=\sigma(2^{p-1}) \sigma(2^p-1) = (2^p -1)2^p = n$. So $n$ is a perfect number.

Conversely, suppose $n$ is even and is a perfect number, WLOG, let $n=2^{k-1}m$, where $m$ is odd and $k \ge 2$. Now

\be
\sigma(n) = \sigma(2^{k-1}) \sigma(m) =   (2^k-1) \sigma(m)\\
\ee
But by definition,
\be
\sigma(n) = 2n = 2^k m
\ee

So we have $(2^k - 1) \sigma(m) =2^k m$ and so as $2^k-1$ is odd, we have $2^k-1$ divides $m$. Write $m = (2^k-1)t$ and so we have $\sigma(m)=2^k t$. But as $m = (2^k-1)t$, if $t \neq 1$, we have $\sigma(m) \ge 1 + t + 2^k -1 + m > 2^k t$, which is a contradiction. Hence we have $t = 1$ and so $m = 2^k -1$ and $\sigma(m)=2^k t$. So $m$ must be a prime. And $m=2^k-1$ prime implies that $k$ must be a prime.
\end{proof}
\subsection{Inverse formula}
\begin{definition} The M$\ddot{\text{o}}$bius function $\mu$ is the number theoretic function given by:
\begin{flushleft}
$\mu(1)=1$\\
$\mu(n)=0$ if $n$ has a squared factor\\
$\mu(p_1 p_2 \ldots p_n)=(-1)^k$, where $p_1 \cdots p_k$ are distinct primes.
\end{flushleft}
\end{definition}
\begin{lemma} $\mu$ is multiplicative and moreover:
\begin{equation*}
\sum_{d|n}\mu(d)= \left\{
\begin{array}{ll}
1 & \text{if } n=1\\
0 & \text{if } n>1\\
\end{array} \right.
\end{equation*}
\end{lemma}
\begin{proof}[\bf Proof] $\mu$ is clearly multiplicative by considering $m,n$ as a product of primes and consider them separately. Now let \begin{equation*} s(n)=\sum_{d|n}\mu(n) \end{equation*} And if consider $s(p^\alpha)$. If $\alpha=0$, then $s(1)=1$. If $\alpha > 0$, then $s(p^\alpha)=1 - 1 + 0 +0 \cdots =0$. Now we know by previous theorem that $s$ is multiplicative, as $\mu$ is, and so the result follows.
\end{proof}
The function $\mu$ is extremely useful when we prove the inverse formula below:
\begin{theorem}{\bf The Inversion Formula}\label{I;Inversion} If $f$ is a number theoretic function and $g$ is defined by
\begin{equation*} g(n)=\sum_{d|n}f(d) \end{equation*}
Then \begin{equation*} f(n)=\sum_{d|n}g(d) \mu \left(\frac{n}{d}\right)=\sum_{d|n}g \left(\frac{n}{d}\right) \mu(d) \end{equation*}
\end{theorem}
\begin{proof}[\bf Proof] Consider:
\begin{eqnarray*}
\sum_{d|n}g \left(\frac{n}{d}\right) \mu(d) & = & \sum_{d_1 d_2 =n} g(d_1) \mu(d_2) =\sum_{d_1 d_2 =n} [\mu(d_2) \sum_{d|d_1}f(d)]\\
& = & \sum_{d_1 d |n}\mu(d_2) f(d) = \sum_{d|n}f(d) \sum_{d_2 |\frac{n}{d}} \mu(d_2)  =  f(n)
\end{eqnarray*}
Using that \begin{equation*} \sum_{d_2|\frac{n}{d}}\mu(d_1) \text{ is only non-zero when $n =d$ by previous Lemma} \end{equation*}
Thus we have \begin{equation*} f(n)=\sum_{d|n} g \left(\frac{n}{d}\right)\mu(d) \end{equation*}
And the second equality is obvious, as summing over $d|n$ is the same as over $\frac{n}{d} | n$
\end{proof}
\begin{theorem} If $g$ is a multiplicative function and it satisfies
\begin{equation*} g(n)=\sum_{d|n}f(d) \end{equation*} for some number theoretic function $f$, then $f$ is also multiplicative.
\end{theorem}
\begin{proof}[\bf Proof] We use the above theorem, and so
\begin{equation*} f(mn)=\sum_{d_1|m, d_2 |n}g(mn)\mu \left(\frac{mn}{d_1d_2}\right) \end{equation*}
Now use $g$ is multiplicative and $\mu$ is multiplicative and so we have the result.
\end{proof}
\subsection{Euler's totient function $\phi$(n)}
\begin{definition} The Euler's $totient$ function $\phi(n)$ is the number of integers less than or equal to $n$ which are coprime to $n$.
\end{definition}
For example $\phi(1)=\phi(2)=1, \phi(10)=4$. In particular, $\phi(p)=p-1$ whenever $p$ is a prime.
\begin{lemma} \begin{equation*} \sum_{d|n}\phi(d)=n \end{equation*}
\end{lemma}
\begin{proof}[\bf Proof] For each divisor $d$ of $n$, let $X_d=\{m: m \le d, (m,d)=1\}$, and so $|X_d|=\phi(d)$. Let $Y_d=\{\frac{nm}{d}: m \le d, (m,d)=1\}$. Clearly, $|Y_d| = |X_d| =\phi(d)$. Now we claim that each $Y_d$ are disjoint and the union of them is $\{1,2,\ldots,n\}$. Let $d_1,d_2|n$, and suppose $\frac{nm_1}{d_1}=\frac{nm_2}{d_2}$, where $m_1 \le d_1, (m_1,d_1)=1$ and $m_2 \le d_2, (m_2,d_2)=1$. Then we have $d_1 m_2 =d_2 m_1$. But $(d_1,m_1)=1$ so $d_1 | d_2$ and $(d_2,m_2)=1$ so $d_2 | d_1$. Therefore, $d_1=d_2$, and so if $d_1 \neq d_2$ then $Y_{d_1}$ is disjoint from $Y_{d_2}$. Now for any $k \in \{1,2,\ldots n\}$. Let $t=(k,n)$, and so $(\frac{k}{t},\frac{n}{t})=1$. As $\frac{n}{t}$ is a divisor of $n$, so we have $k=\frac{n}{\frac{n}{t}} \cdot \frac{k}{t}$, where $\frac{k}{t} \le \frac{n}{t}$ and they are coprime. Therefore, we have that the union of $Y_d$'s is $\{1,2,\ldots n\}$ and hence the result.
\end{proof}
\begin{theorem} The function $\phi$ is multiplicative, and moreover, if $n=p_1^{\alpha_1} \ldots p_k^{\alpha_k}$, then
\begin{equation*} \phi(n)=\prod_{p_i|n}\left(p_i^{\alpha_i} - p_i^{\alpha_i -1}\right)=n \prod_{p_i|n}(1-\frac{1}{p_i}) \end{equation*}
\end{theorem}
\begin{proof}[\bf Proof] Use Theorem 2.13 and Lemma 2.15 and the fact that identity function is multiplicative. Then observe that, if $p$ is a prime, then for $\alpha \ge 1$, $\phi(p^\alpha)=p^\alpha-p^{\alpha-1}$. Then use that $\phi$ is multiplicative
\end{proof}
\begin{lemma} \begin{equation*} \phi(n)=n \sum_{d|n}\frac{\mu(d)}{d} \end{equation*}
\end{lemma}
\begin{proof}[\bf Proof] Apply inverse formula
\end{proof}
\subsection{Liouville function $\lambda$}
\begin{definition} Let $\nu$ denote the number of (not necessarily distinct) prime factors of $n$.
\end{definition}
For example, $\nu(10)=2, \nu(24)=4$.
\begin{definition} The number theoretic function $\lambda$ is defined to be $\lambda(n)=(-1)^{\nu(n)}$.
\end{definition}
Note that $\nu(mn)=\nu(m)+\nu(n)$, for every $m,n >0$. And hence that the function $\lambda$ is $completely$ multiplicative.
\begin{lemma}
\begin{equation*}
\sum_{d|n}\lambda= \left\{
\begin{array}{ll}
1 & \text{if n is a square}\\
0 & \text{otherwise } \\
\end{array} \right.
\end{equation*}
\end{lemma}
\begin{proof}[\bf Proof] Let $s(n)=\sum_{d|n}\lambda(d)$, and so by earlier result, $s(n)$ is multiplicative. Now let $p$ be a prime, and $\alpha \ge 0$, then
\begin{equation*}
s(p^\alpha)=1 - 1 + 1 - 1 \cdots= \left\{
\begin{array}{ll}
1 & \text{if } \alpha \text{ is even}\\
0 & \text{if } \alpha \text{ is odd} \\
\end{array} \right.
\end{equation*}
Then use the fact that $s$ is multiplicative.
\end{proof}
\subsection{Average order}
We begin with the definition of Riemann Zeta function.
\begin{definition} The Riemann Zeta function $\zeta(s)$ is defined to be:
\begin{equation*} \zeta(s)=\sum_{n=0}^{\infty}\frac{1}{n^s} \end{equation*}
\end{definition}
\begin{flushleft} Deed discussion of the function will be in later chapters. \end{flushleft}
It is often in interest to determine the magnitude on average of arithmetic functions $f$, that is, to find estimates for sums of the form $\sum f(n)$ with $n \le x$, where $x$ is a largest real number.
Firstly, we consider the function $\tau$.
\begin{equation*}
\sum_{n \le x}\tau(n)=\sum_{n \le x} \sum_{d|n}1=\sum_{d \le x} \sum_{m \le \frac{x}{d}}1=\sum_{d \le x}\left(\frac{x}{d}\right)
\end{equation*}
We shall use the fact that
\begin{equation*} \sum_{d \le x}\frac{1}{d}=\log{x}+O(1) \end{equation*}
and hence we have
\begin{equation*} \sum_{n \le x}\tau(n)=x \log{x} + O(x) \end{equation*}
and so we see that asymptotically
\begin{equation*} \frac{1}{x} \sum_{n \le x}\tau(n) \sim \log{x} \text{ as } x \to \infty \end{equation*}

Now, for $\sigma(n)$, we have:
\begin{equation*} \sum_{n \le x}\sigma(n) = \sum_{n \le x}\sum_{d|n}\left(\frac{n}{d}\right)=\sum_{d \le x} \sum_{m \le \frac{x}{d}}m \end{equation*}
which gives
\begin{equation*} \frac{1}{2}\left[\frac{x}{d}\right]\left(\left[\frac{x}{d}\right]+1 \right) =\frac{1}{2}\left(\frac{x}{d}\right)^2 + O\left(\frac{x}{d}\right)
\end{equation*}
Now we shall use the fact that
\begin{equation*}
\sum_{d \le x}\frac{1}{d^2}=\sum_{d=1}^{\infty}\frac{1}{d^2} + O \left(\frac{1}{x}\right) \\
\text{and } \sum_{d=1}^{\infty}\frac{1}{d^2}=\frac{\pi^2}{6}
\end{equation*}
Thus we conclude that
\begin{equation*}
\sum_{n \le x}\sigma(n)=\frac{1}{12} \pi^2 x^2 + O\left(x \log{x} \right)
\end{equation*}

Finally, we shall consider the estimate for $\phi$. We have
\begin{equation*} \sum_{n \le x}\phi(n)=\sum_{n \le x} \sum_{d|n} \mu(d) \left(\frac{n}{d}\right) =\sum_{d \le x}\mu(d) \sum_{m \le \frac{x}{d}} m
\end{equation*}
Now consider that, as
\begin{equation*} \sum_{m \le \frac{x}{d}} m = \frac{1}{2}\left(\frac{x}{d}\right) + O\left(\frac{x}{d}\right)
\end{equation*}
we have,
\begin{equation*} \sum_{d \le x}\frac{\mu(d)}{d^2}=\sum_{d=1}^{\infty} \frac{\mu(d)}{d^2}+O\left(\frac{1}{x}\right)
\end{equation*}
and then we require the following lemma:
\begin{lemma} \begin{equation*} \sum_{n=1}^{\infty}\frac{\mu(n)}{n^2}=\frac{6}{\pi^2} \end{equation*}
\end{lemma}
\begin{proof}[\bf Proof] We use the fact that $\zeta(2)=\sum_{n=1}^{\infty}\frac{1}{n^2}=\frac{\pi^2}{6}$, and so we will check that $\frac{1}{\zeta(s)}=\sum_{n=1}^{\infty}\frac{\mu(n)}{n^s}$, whenever $s>1$.\\
We shall assume for now (which will be explained in later chapters) that the series converges absolutely when $s>1$, and so we have:
\begin{equation*}
\sum_{m=1}^{\infty}\frac{1}{m^s} \sum_{n=1}^{\infty}\frac{\mu(n)}{n^s} = \sum_{m,n=1}^{\infty} \frac{\mu(n)}{(mn)^s}=\sum_{d=1}^{\infty}\left(\frac{1}{d^s} \sum_{k|d}\mu(k)\right)=1
\end{equation*}
using Lemma 2.11
\end{proof}
Hence we conclude that:
\begin{equation*} \sum_{n \le x}\phi(n)=\left(\frac{3}{\pi^2}\right) x^2 + O(x \log{x})
\end{equation*}
To finish the chapter, we have the following surprising result:
\begin{corollary} The probability that any two positive integers are coprime is $\frac{6}{\pi^2}$.
\end{corollary}
\begin{proof}[\bf Proof] Consider that the number of pairs $\{r,s\}$ satisfying $1 \le r \le s \le n$ is $\frac{n(n+1)}{2}$, and $\sum_{d \le n} \phi(d)$ counts the number of such pairs which are also coprime.
Therefore, let $n$ runs through every natural number, and so the probability of any two positive integers being coprime is:
\begin{equation*}
\lim_{n \to \infty} \frac{\sum_{d \le n}\phi(d)}{\frac{n(n+1)}{2}} = \frac{6}{\pi^2}
\end{equation*}
by previous result.
\end{proof}
\subsection{Exercises}
\begin{enumerate}
\item Show that $\prod_{d|n}d = n^{\frac{\tau(n)}{2}}$
\item Show that the number of ordered pairs of positive integers whose least common multiple is $n$ equals $\tau(n^2)$.
\item Show that $\sum_{d^2|n}\mu(d)=|\mu(n)|$
\item Show that if $\sigma(n)$ is odd then $n$ is a square or twice a square.
\item[5$^\star$.] Let $det(i,j)$ be the determinant of $n \times n$ matrix whose i,j entry is (i,j). Prove that
\begin{equation*} det(i,j)=\prod_{k=1}^{n}\phi(k) \end{equation*}
\item[6.] Let
\begin{equation*}
\Lambda(n)= \left\{
\begin{array}{ll}
\log{p} & \text{if } n \text{ is a power of prime }p \\
0 & \text{otherwise } \\
\end{array} \right.
\end{equation*}
Evaluate $\sum_{d|n}\Lambda(d)$. Express $\sum_{n=0}^{\infty}\frac{\Lambda(n)}{n^s}$ in terms of $\zeta(s)$.\\
(You may use $\prod_{p \text{ prime}}(1-p^{-s})^{-1} = \zeta(s))$.\\
\item[7.] Let $a$ run through the integers with $1 \le a \le n$ and $(a,n)=1$. Show that $f(n)=\frac{1}{n} \sum a$ satisfies $\sum_{d|n}f(d)=\frac{1}{2}(n+1)$. Hence prove that $f(n)=\frac{1}{2} \phi(n)$ for $n>1$
\item[8.] If $f(s)=\sum a_n n^{-s}$ and $g(s)=\sum b_n n^{-s}$. Show that $g(s)=f(s)\zeta(s)$ if and only if $\sum a_n \left(\frac{x^n}{1-x^n}\right)=\sum b_n x^n$ with $0 \le x <1$. Hence show that $\sum_{n=1}^{\infty}\frac{\phi(n)x^n}{1-x^n}=\frac{x}{(1-x)^2}$\\
(Hint: You may need the identity that $\sum_{n=1}^{\infty}nx^n =  \frac{x}{(1-x)^2}$).
\end{enumerate}

\section{Prime numbers and congruence}
\subsection{Prime numbers}
\begin{theorem}{\bf [Euclid]} There are infinitely many prime numbers.
\end{theorem}
\begin{proof}[\bf Proof] Suppose there are finitely many. Let $p_1, p_2 \ldots p_n$ be the list of all primes. Then consider $N=\prod_{i \le n}p_i +1$. This is not divisible by any prime $p_i$ and hence it must have a prime factor different from $p_i$, which is a contradiction.
\end{proof}
\begin{definition} A prime number is called Mersenne Prime if it is a prime of the form $2^n-1$ for some integer $n$. For example, $7=2^3-1, 31=2^5-1$.
\end{definition}
\begin{remark} Clearly, if $2^n-1$ is prime, $n$ must be prime, otherwise, let $d|n$, then $2^d -1$ is a proper factor.
\end{remark}
Here is a big conjecture of prime numbers: The number of primes $q \le x$ such that $2^q -1$ is also a prime is asymptotically $\sim c \log{x}$, where $c=\frac{e^\gamma}{\log{2}}, \gamma=\lim_{n \to \infty}(1+ \cdots \frac{1}{n})-\log{n}$. $\gamma$ is called Euler constant.
\begin{proposition} Let $N$ be any integer $\ge 2$, then there is a block of $N+1$ consecutive composite numbers.
\end{proposition}
\begin{proof}[\bf Proof] Pick $p = N+2$, and let $M=p!$, then the integers $M+2,M+3 \ldots M+p$ are all composite.
\end{proof}
\begin{definition} For any  $x \ge 2$, write $\pi(x)=card\{p: p \text{ prime}, p \le x\}$.
\end{definition}
Numerically, $\pi(10^2)=25, \pi(10^3)=168, \pi(10^4)=1229, \pi(10^6)=78498$.
\begin{definition} We write $f_S(x)$ to be the number of positive integer $\le x$, composed of primes in the set $S$.
\end{definition}
\begin{proposition} $\forall x \ge 2, f_S(x) \le \sqrt{x} \cdot 2^{\card(s)}$.
\end{proposition}
\begin{proof}[\bf Proof] Take any $n \le x$, composed only of primes in $S$. We can write $n=m^2 r$, where $r$ is square free. Then as $n \le x$ and so $m \le \sqrt{x}$. So we have at most $\sqrt{x}$ choices for $m$. Then for $r$, we write $r=p_1^{\epsilon_1} \cdots p_k^{\epsilon_k}$, where $p_i \in S$ and $\epsilon_i \in \{0,1\}$ because $r$ is square free. And so we have $2^k=2^{\card(s)}$ choices for $r$. Therefore, we conclude that $f_S(x) \le \sqrt{x} \cdot 2^{\card(s)}$.
\end{proof}
\begin{corollary} $\forall x \ge 2, \pi(x) \ge \frac{\log{x}}{2 \log{2}}$.
\end{corollary}
\begin{proof}[\bf Proof] Take $S$ to be the set of all primes $\le x$. Then we have, by definition that $f_S(x)=x$, and $\card(S)=\pi(x)$. Then by the above proposition, we have that $x \le \sqrt{x} \cdot 2^{\pi(x)}$, and so $\pi(x) \ge \frac{\log{x}}{2 \log{2}}$.
\end{proof}
\subsection{Congruence}
\begin{definition} If $a,b \in \mathbb{Z}$, we say $a \equiv b$ (mod $m$) if $m | a-b$.
\end{definition}
\begin{remark}
We can easily check that this is an equivalence relation, and the equivalence classes are of the form $a+m \mathbb{Z}$. The set of equivalence classes is $\mathbb{Z}/m \mathbb{Z}$ (Quotient ring).
\end{remark}
\begin{lemma} If we write $[a]_m$ for the equivalence class of $a$ mod $m$, then we have:
\begin{equation*} [a]_m+[b]_m=[a+b]_m, [a]_m \cdot [b]_m=[ab]_m \end{equation*}
\end{lemma}
\begin{proof}[\bf Proof] It is enough to check that, if $a \equiv a'$ mod $m$, $b \equiv b'$ mod $m$, then:
\begin{equation*} a+b \equiv a'+b' ~(\text{mod } m), a b \equiv a'  b' ~(\text{mod } m)
\end{equation*}
But, we know that $m| a-a', m|b-b'$, and so $m|(a+b)-(a'+b')$. Also as $a b-a' b'=(a-a')b+a'(b-b')$. Therefore, $m|a b-a' b'$.
\end{proof}
It is also easy to check that if $x \equiv a$ (mod $p$) and $x \equiv a$ (mod $q$), where $(p,q)=1$. Then we have $x \equiv a$ (mod $pq$).

We shall now use the notation that $(\mathbb{Z}/m \mathbb{Z})^{*}$ to mean the multiplicative group of $\mathbb{Z}/m \mathbb{Z}$. Sometimes, we write $(\mathbb{Z}/(m))^{*}$
\begin{lemma} $a+m\mathbb{Z} \in (\mathbb{Z}/m \mathbb{Z})^{*} \iff (a,m)=1$.
\end{lemma}
\begin{proof}[\bf Proof] $a+m\mathbb{Z}$ is in the group if and only if it has an inverse element, say $b+m\mathbb{Z}$. And so we have proved that $[a]_m [b]_m=[ab]_m$. So the inverse element exists if and only if $\exists b,c$ such that $ab+cm=1$. And this implies that $(a,m)=1$.
\end{proof}
\begin{corollary} Let $p$ be a prime, then $\mathbb{Z}/p\mathbb{Z}$ is a field with $p$ elements.
\end{corollary}
\begin{proof}[\bf Proof] By Lemma 3.12, we have $(\mathbb{Z}/p\mathbb{Z})^{*}=\mathbb{Z}/p\mathbb{Z} - [0]_p$ because every positive integer less than $p$ is prime to $p$.
\end{proof}
\begin{theorem} Let $a,b$ and $m$ be integers with $m>0$. The congruence
\begin{equation*} ax \equiv b ~(\text{mod } m) \end{equation*}
is soluble if and only if $(a,m)|b$. If $x_0$ os a solution, then there are exactly $(a,m)$ solutions given by $\{x_0 + \frac{tm}{(a,m)}\}$ where $t=0,1 \ldots (a,m)-1$.
\end{theorem}
\begin{proof}[\bf Proof] This is just a restatement of Theorem 1.5 and use the definition of congruence.
\end{proof}
\begin{corollary} If $(a,m)=1$ then the congruence
\begin{equation*} ax \equiv b~(\text{mod } m) \end{equation*}
has exactly one solution.
\end{corollary}
\begin{proof}[\bf Proof] Use Theorem 3.14.
\end{proof}
\subsection{Fermat's little theorem and Euler's theorem}
Fermat's theorem (Often called Fermat's little theorem) states that if $a$ is any natural number and $p$ is any prime, then $a^p \equiv a$ (mod $p$). The theorem was first announced in 1640 by Fermat but without proof. Euler, however, gave a first demonstration in 1760 and established a more general result:
\begin{theorem}{\bf [Euler's Theorem]}\label{E;Euler} Let $a,n$ be any positive integers such that $(a,n)=1$. Then
\begin{equation*} a^{\phi(n)} \equiv 1 ~(\text{mod }n) \end{equation*}
\end{theorem}
\begin{proof}[\bf Proof] Clearly, Fermat's little theorem is a special case of Euler's theorem. One may apply Lagrange theorem in group theory that the order of any element in the group divides the order of the group, and hence the result follows immediately. But here is a proof without Lagrange theorem:

Let $S$ be the set of integers less than $n$ which are prime to $n$, say $S=\{p_1,p_2,\ldots p_{\phi(n)}\}$. Then consider that for each $p_i a = b_i$ (mod $n$), $b_i$ must also be prime to $n$. And if $p_i a = p_j a$ (mod $n$), then $a(p_i - p_j) = 0$ (mod $n$), and so $n|p_i-p_j$. But $(a,n)=1$ and $-n < p_i -p_j <n$, so this is only possible when $p_i - p_j=0$. i.e. $p_i=p_j$.

Use the above fact and by congruence multiplication, we have:
\begin{equation*} p_1 a \cdot p_2 a \cdots p_{\phi(n)} a \equiv b_1 b_2 \cdots b_{\phi(n)} ~(\text{mod } n)
\end{equation*}
where $\{b_1,b_2 \ldots b_{\phi(n)}\}$ is a rearrangement of $\{p_1 \ldots p_{\phi(n)}\}$. Hence we can cancel both sides by $p_1 p_2 \cdots p_{\phi(n)}$, and so
\begin{equation*} a^{\phi(n)} \equiv 1~(\text{mod } n) \end{equation*}
\end{proof}
\subsection{Wilson's theorem}
\begin{theorem}{\bf [Wilson's Theorem]}\label{W;Wilson} Let $p$ be any prime, then
\begin{equation*} (p-1)! \equiv -1 ~(\text{mod }p) \end{equation*}
\end{theorem}
The theorem was known to Ibn al-Haytham (also known as Alhazen) in Medieval Europe) circa 1000 AD, but it is named after John Wilson, a student of Waring. It is first published by Waring in his Meditationes algebraicae of 1770, but he nor Wilson could prove it. Lagrange gave the first proof in 1773.
\begin{proof}[\bf Proof] It suffices to assume that $p>2$ as the case when $p=2$ is trivially true. Now assume $p>2$, then let $b=1$ and for each $a < p$, apply Corollary 3.15, we have that each $ax \equiv 1$ (mod $p$) has only one solution. And we know by Corollary 3.13 and uniqueness of inverse in group, we can pair each $(a,x)$ such that $ax \equiv 1$ (mod $p$). The inverse of $1$ is itself, and the inverse of $p-1$ is also itself, and there are only one element which has order $2$, because if $x^2 \equiv 1$ (mod $p$), we have $p|(x-1)(x+1)$ but as $p$ is prime, so $p|x-1$ or $p|x+1$ and hence $x=1$ or $x=p-1$. Therefore, other integers are paired up and so the product is $1$ mod $p$. Therefore, we have that:
\begin{equation*} (p-1)! \equiv (p-1) \equiv -1 ~(\text{mod } p) \end{equation*}
\end{proof}
\subsection{Chinese remainder theorem}
The theorem was first stated by Chinese mathematician Sun Tzu in first century AD, and was later republished in a 1247 book by Qin jiushao, the Shushu jiuzhang. It concerns about simultaneous congruences.

For example, how can we solve a simultaneous congruence like:
\begin{equation*} x \equiv 2~(\text{mod } 3), x \equiv 2~(\text{mod } 5), x \equiv 6~(\text{mod } 7)
\end{equation*}
\begin{theorem}{\bf [Chinese remainder theorem]}\label{C;CRT} Assume $(m_i,m_j)=1$ for $i \neq j$, and that $M=m_1m_2 \ldots m_k$, and $a_1, a_2 \ldots a_k$ are arbitary integers, then there exists $x$ in $\mathbb{Z}$ satisfying $x \equiv a_1$ (mod $m_1$), $\ldots$, $x=a_k$ (mod $m_k$). Moreover, any two solutions are congruent mod $M$.
\end{theorem}
\begin{proof}[\bf Proof] Define $M_i=\frac{M}{m_i}$. Then we have $(m_i,M_i)=1$. Hence $\exists u_i$ such that $u_i M_i \equiv 1$ (mod $m_i$). Then we set $x=\sum_{j=1}^{k}a_ju_jM_j$. Then we can check that as $M_i \equiv 0$ (mod $m_j$) for any $i \neq j$. Hence $x \equiv a_i$ (mod $m_i$) and so it is a solution.

Finally, to check that any two solutions are congruent mod $M$, let $x_1,x_2$ be two solutions. Then $x_1 \equiv x_2$ (mod $m_i$) $\forall i$. And as $(m_i,m_j)=1$, so $x_1 \equiv x_2$ (mod $\prod m_i$), i.e. $x_1 \equiv x_2$ (mod $M$).
\end{proof}
We have another group theoretic of Chinese remainder theorem.
\begin{definition} $\mathbb{Z}/m_1\mathbb{Z} \times \ldots \mathbb{Z}/m_k\mathbb{Z} =\{(r_1, r_2,\ldots r_k): r_i \in \mathbb{Z}/m_i\mathbb{Z}\}$ is a ring given by pointwise operation.
\end{definition}
\begin{theorem} Assume $(m_i,m_j)=1, i \neq j$ and $M=m_1m_2 \ldots m_k$. Then the map:
\begin{equation*} \theta: \mathbb{Z}/M\mathbb{Z} \longmapsto \mathbb{Z}/m_1\mathbb{Z} \times \ldots \mathbb{Z}/m_k\mathbb{Z}
\end{equation*}
by $\theta(a+M\mathbb{Z})=(a+m_1\mathbb{Z}, \ldots,a+m_k\mathbb{Z})$ is an isomorphism.
\end{theorem}
\begin{proof}[\bf Proof] It is clear that $\theta$ is an homomorphism, by definition of congruence operation. $\theta$ is clearly injective, since if $\theta(a+M\mathbb{Z})=0$, so have:
\begin{equation*} a \equiv 0~(\text{mod } m_i) \forall i \end{equation*}
As $(m_i,m_j)=1$, so $a \equiv 0$ (mod $M$). Now consider cardinality of $\mathbb{Z}/M\mathbb{Z}$ and the cardinality of $\mathbb{Z}/m_1\mathbb{Z} \times \cdots \times \mathbb{Z}/m_k\mathbb{Z}$. They are the same, hence it is bijective. So $\theta$ is isomorphism.
\end{proof}
This has an immediate corollary:
\begin{corollary} $\phi$ is multiplicative. (This is another proof)
\end{corollary}
\begin{proof}[\bf Proof] Use Theorem 3.20
\end{proof}
\subsection{Polynomial and polynomial congruence}
The polynomial congruence investigates the solutions to the congruence of the form $f(x) \equiv 0$ (mod $m$), where $f(x) \in \mathbb{Z}[x]$. For example, $f(x)=x^2-1,m=8$ has four solutions, $x=1,3,5,7$.
\begin{definition} Let $R$ be commutative ring. $R[x]=\{\sum_{i=1}^{n}a_ix^i: a_i \in R, n \in \mathbb{N}\}$. We call the {\bf evaluation} of $\alpha \in R$ if $f(x)=\sum_{i=0}^n a_ix^i, f(\alpha)=\sum_{i=0}^{n}a_i \alpha^i \in R$.
\end{definition}
\begin{lemma} For each $f(x)$ in $R[x]$ and each $\alpha \in R$, there exists $h(x)$ in $R[x]$ such that $f(x)-f(\alpha)=(x-\alpha)h(x)$.
\end{lemma}
\begin{proof}[\bf Proof] $f(x)-f(\alpha)=a_1(x-\alpha)+\cdots a_n(x^n-\alpha^n)$. Then
$$x^k-\alpha^k=(x-\alpha)(x^{k-1} + \cdots +\alpha^{k-1})$$ So $f(x)-f(\alpha)=(x-\alpha)h(x)$ for some $h(x)$.
\end{proof}
\begin{corollary} For any $f(x)$ in $R[x]$ and any $\alpha \in R$, we have $f(\alpha)=0$ if and only if $f(x)=(x-\alpha)h(x)$ for some $h(x)$ in $R[x]$.
\end{corollary}
\begin{proof}[\bf Proof] Use Lemma 3.23.
\end{proof}
\begin{definition} $R$ is an {\bf integral domain} if $R$ has no zero divisor, i.e. $\alpha \beta=0 \Rightarrow \alpha =0$ or $\beta=0$.
\end{definition}
\begin{proposition} Let $R$ be an integral domain. Let $f(x) \in R[x]$ have distinct root $\alpha_1, \ldots, \alpha_k$. Then we can write:
\begin{equation*} f(x)=(x-\alpha_1) \cdots (x-\alpha_k)h(x) \end{equation*}
for some $h(x) \in R[x]$.
\end{proposition}
\begin{proof}[\bf Proof] Induction on $k$. It is true for $k=1$ by applying Lemma 3.23 with $f(\alpha)=0$.
Assume it is true for $k=s$. Then for $s+1$, we have:
\begin{equation*}
0=f(\alpha_{s+1})=(\alpha_{s+1}-\alpha_1) \cdots (\alpha_{s+1} -\alpha_s)h(\alpha_{s+1})
\end{equation*}
by inductive hypothesis, where $\alpha_{s+1} \neq \alpha_i, i=1,2 \ldots, s$. Hence $h(\alpha_{s+1})=0$ because $R$ is integral domain. And so we apply the case when $k=1$ to write $h(x)=(x-\alpha_{s+1})h_1(x)$. The result follows
\end{proof}
\begin{corollary}{\bf [Lagrange]}\label{L;Lagrange} Let $R$ be an integral domain, let $f$ be a non-zero polynomial in one variable of degree $n$. Then $f$ has at most $n$ distinct roots in $R$.
\end{corollary}
\begin{proof}[\bf Proof] The proof is again induction on $n$. When $n=1$, we have $f(x)=ax+b, a \neq 0$, say. Then $ax+b=0 \iff ax=-b$. As $a \neq 0$ and so we have an inverse of $a$. Therefore, $x=-a^{-1}b$ is the only solution.
\end{proof}
Here is an immediate application of Lagrange's theorem, which is another proof of Wilson's theorem (Theorem 3.17):

Again the case $p=2$ is trivial so let $p>2$. Consider the equation $f(x)=x^{p-1}-1 -(x-1)(x-2) \ldots(x-p+1)$ in the ring $\mathbb{Z}/p\mathbb{Z}$. Then by Fermat's little theorem, any $x \in \{0,1 \ldots p-1\}$ is a solution, but the term $x^{p-1}$ is canceled and so degree of $f(x)$ is at most $p-2$. Therefore, by Lagrange's theorem, $f(x)$ must be identically zero in the ring $\mathbb{Z}/p\mathbb{Z}$. Hence, we consider the constant coefficient, we have that
\begin{equation*} -(-1)^{p-1} (p-1)! -1 \equiv 0~(\text{mod } p) \end{equation*}
Hence the result follows.

Surprisingly, the converse is also true and the proof is left as an exercise (see question 6)
\subsection{Chevalley's theorem}
We may briefly discuss the general case in this subsection and congruences in more than more variable will be considered. Throughout the section, let $p$ be a fixed prime, and $f,g$ and $h$ be polynomials defined over $\mathbb{Z}/p\mathbb{Z}$, and let deg$f$ denote the total degree of $f$. We begin with the following definitions:
\begin{definition} If $f$ and $g$ are n-variable polynomials, we say $f$ is {\bf equivalent} to $g$, written $f \equiv g$, if for all sets $\{a_1, \ldots a_n\}$ of elements of $\mathbb{Z}/p\mathbb{Z}$ we have
\begin{equation*} f(a_1, \ldots ,a_n) \equiv g(a_1, \ldots a_n)~(\text{mod }p) \end{equation*}
\end{definition}
\begin{definition} We say $f$ is {\bf congruent} to $g$, written as $f \sim g$, if all coefficients of corresponding monomials of $f$ and $g$ are congruent mod $p$. (monomials are of the form $x_1^{\alpha_1} \ldots x_n^{\alpha_n}$)
\end{definition}
\begin{definition} $f$ is called reduced if $f$ has degree less than $p$ in each of its variables.
\end{definition}
\begin{remark} If $f$ is congruent to $g$, then $f$ is equivalent to $g$, as the coefficient of each monomial is congruent mod $p$. But the converse is not true. For example, $x^p \equiv x$, but $x^p \not \sim x$ as the coefficient of $x$ in $x^p$ is $0$ but is $1$ in $x$. Note that $x^p$ is not reduced.
\end{remark}
\begin{lemma} Given a polynomial $f$ we can find a reduced polynomial $f'$ such that deg$f' \le$ deg$f$ and $f' \equiv f$.
\end{lemma}
\begin{proof}[\bf Proof] Use Fermat's little theorem and so for each term $x^k, k \ge p$ in $f$. Write $k=ap+b$, where $b<p$. Then $x^k \equiv x^{a+b}$ (mod $p$) and $a + b <k$. If $a+b<p$ then we are done with this term, and if not we apply the above to $x^{a+b}$ again until the index is less than $p$. Now apply this to every term $x^k, k>p$
\end{proof}
\begin{lemma} If $f$ and $g$ are reduced polynomials and $f \equiv g$, then $f \sim g$.
\end{lemma}
\begin{proof}[\bf Proof] It is easy to check from definition that $f \equiv g \iff f-g \equiv 0$ and $f \sim g \iff f-g \sim 0$. So WLOG, we can assume that $g$ is zero polynomial.

Let $f$ be reduced, $f \equiv 0$. If $n=1$, as $f$ has degree less than $p$ but any $a \in \mathbb{Z}/p\mathbb{Z}$ is a root, so by Lagrange theorem (Corollary \ref{L;Lagrange}) we must have that $f \sim g$.

If $n \ge 2$, write $f=f(x_1,\ldots x_n)$. Then we may fix $a_2,a_3 \ldots a_n$ and treat $f$ as a polynomial in one variable $x_1$. Then as $f(a_1, \ldots, a_n) \equiv 0$ for any set $\{a_1,a_2 \ldots a_n\}$, it is equivalent to say that the polynomial $f$ in the variable $x_1$ has roots $\{0,1,2, \ldots p-1\}=\mathbb{Z}/p\mathbb{Z}$. But $f$ has degree less than $p$ in $x_1$, and hence by Lagrange theorem, we have that each coefficient is congruent to $0$.

Now each coefficient is a polynomial in n-1-variables, i.e. $x_2,x_3 \ldots x_n$. And we can continue this argument inductively, and hence we conclude that $f \sim 0$.
\end{proof}
\begin{theorem}{\bf [Chevalley's Theorem]}\label{C;Chevalley} Suppose $f$ and $g$ are n-variable polynomials with degree less than $n$.
\begin{enumerate}
\item[(i)] If the congreunce
\begin{equation*} f(x_1, \ldots, x_n) \equiv 0~(\text{mod } p) \end{equation*}
is soluble, then it has at least two solutions.
\item[(ii)] Suppose $g$ is a homogeneous polynomial then the congruence
$$g(x_1,\ldots x_n) \equiv 0~(\text{mod } p)$$ has a non-trivial solution.
\end{enumerate}
\end{theorem}
\begin{proof}[\bf Proof] (ii) follows from (i) because clearly that in the homogenous case, $\{0,0, \ldots 0\}$ is always a solution and hence by (i) we have another non-trivial solution.\\
For (i) assume that deg$f = r$ and we have only one solution $x_i=a_i$ (mod $p$), $i=1,2 \ldots n$. Let $h$ be given by:
\begin{equation*} h(x_1, \ldots, x_n)=1-f(x_1,\ldots ,x_n)^{p-1} \end{equation*}
then by assumption we have $h(x_1, \ldots ,x_n) \equiv 1$ (mod $p$) if $x_1 \equiv a_i$ (mod $p$) for all $i$, and by Fermat's little theorem, $h(x_1, \ldots ,x_n) \equiv 0$ (mod $p$) for any other $x_i$.

Now by Lemma 3.32 we have some $h'$ which is a reduced polynomial such that $h' \equiv h$. Then we define:
\begin{equation*} h''(x_1, \ldots x_n)=\prod_{i=1}^{n}\left(1-(x_i-a_i)^{p-1}\right) \end{equation*}
Clearly, $h'' \equiv h$, and hence we have $h'' \equiv h'$. But both $h'$ and $h''$ are reduced, then by Lemma 3.33, $h' \sim h''$. By Lemma 3.32, deg$h' \le$ deg$h = r(p-1)$. But $h''$ has degree $n(p-1)$, and this is impossible as we assume that $r<n$, which is a contradiction. So we have at least two solutions.
\end{proof}
\subsection{Primitive root and non-linear congruence}
In this subsection, we will investigate $\mathbb{Z}/{p^\alpha \mathbb{Z}}$ for any odd prime $p$ and
$\alpha \ge 1$.
\begin{lemma} Let $G$ be a cyclic group of order $d>1$, then $G$ has $\phi(d)$ generators.
\end{lemma}
\begin{proof}[\bf Proof] Write $G=<g>$. Then each element $g^n$ has order $d$ if and only if $(n,d)=1$.
\end{proof}
\begin{lemma} Let $H$ be any finite group of order $n$. Suppose for all divisor of $n$, the set of $x$ in $H$ with $x^d=1$ has cardinality $\le d$, then $H$ is cyclic.
\end{lemma}
\begin{proof}[\bf Proof] For each $d|n$, let $W_d=\{x \in H: x$ has exact order $d \}$. We check that $W_d$ is non-empty for each $d$. Suppose it is, then by construction, we have:
\begin{equation*} n=\sum_{d|n} \card(W_d) \end{equation*}
and if $\exists x, x$ has order $d$, then consider $G = <x>$ and apply Lemma 3.35, we have $\phi(d)$ elements of order $d$. Therefore, we know that $\card(W_d)=\phi(d)$ or $0$. But by lemma 2.15
\begin{equation*} n=\sum_{d|n}\phi(d) \end{equation*}
and hence $W_d$ can never be empty.

Now as each $W_d$ is non-empty, take $y \in W_d$, and consider the set:
$$ \{1,y,y^2,\ldots, y^{d-1}\} $$
Each element in the set satisfies $x^d=1$. Now by assumption we have at most $d$ elements satisfying this for each $d$. Hence we conclude these are all solutions. In particular, take $d=n$, then every element in the group satisfies $x^n=1$ and $W_n$ is non-empty, i.e. we have some element whose exact order is $n$ and so the group is cyclic.
\end{proof}
\begin{theorem} Let $F$ be any finite field. Then $F^*$ ($F-\{0\}$ as a group under $*$) is cyclic.
\end{theorem}
\begin{proof}[\bf Proof] Let $H=F^*$, and as $q=\card(F)$. Then $F*$ has $q-1$ elements. If $d|q-1$, then $x^d-1$ has at most $d$ distinct roots by Lagrange's theorem (\ref{L;Lagrange}) as a field must be an integral domain. Then the set of $x \in H$ with $x^d=1$ has cardinality $\le d$, by Lemma 3.36, $F^*=H$ is cyclic.
\end{proof}
The main theorem of the subsection is:
\begin{theorem} If $p$ is an odd prime, then $(\mathbb{Z}/p^n\mathbb{Z})^*$ is cyclic for all $n \ge 1$.
\end{theorem}
\begin{remark} The condition that $p$ is an odd prime is necessary. It does not work for $p=2$. For example, $(\mathbb{Z}/8\mathbb{Z})^* = C_2 \times C_2$. But in fact, we can prove that $(\mathbb{Z}/2^n\mathbb{Z})^*$ is isomorphic to $C_2 \times C_{2^{n-2}}$.
\end{remark}
To prove the theorem, we need the followings:
\begin{lemma} There is a primitive root $g$ mod $p$ (i.e. $g$ has exact order $p-1$) such that $g^{p-1}=1+bp$, where $(b,p)=1$
\end{lemma}
\begin{proof}[\bf Proof] Take any primitive root $g_1$ as we know the group $(\mathbb{Z}/p\mathbb{Z})^*$ is cyclic.
Let $g_1^{p-1}=1+b_1p$. If $(b_1,p)=1$ then we take $g_1$. Otherwise, we have $p|b$, and we take $g=g_1+p$. Then $g^p -1 = (g_1+p)^{p-1}-1 = g_1^{p-1}-1 + (p-1)g_1^{p-2}p+cp^2$.\\

So we have $g^p-1 = (b_1p+cp^2)+(p-1)g_1^{p-2}p = p(b_1+cp+(p-1)g_1^{p-2})$, and as we have $p|b_1$, so $p|b_1+cp$ and so $p \nmid b_1+cp+(p-1)g_1^{p-2}$ and so $g^p=1+bp$, where $(b,p)=1$.
\end{proof}
\begin{lemma} Let $p$ be an odd prime and $w$ be any integer of the form $w=1+bp$, where $(b,p)=1$. Then $\forall n \ge 0$, we have $w^{p^n}=1+b_np^{n+1}$ with $(b_n,p)=1$.
\end{lemma}
\begin{proof}[\bf Proof] Use induction on n. It is true for $n=0$ by assumption. Suppose it is true for n, then
$w^{p^{n+1}}=(1+b_np^{n+1})^p=1+b_np^{n+2}+\sum_{i=2}^p \binom{p}{i}{b_n}^i p^{(n+1)i}$
As $p$ is an odd prime, so $p|\binom{p}{i}~\forall 1 \le i \le p-1$.\\

So we have $1+b_np^{n+2}+cp^{2n+3} = 1+b_{n+1}p^{n+2}$, where $b_{n+1}=b_n+cp^{n+1}$. Therefore, $(b_{n+1},p)=1$
\end{proof}
Now we can prove theorem 3.38
\begin{proof}[\bf Proof] Take $g$ to be any primitive root with $(b,p)=1$, as in Lemma 3.40. Then $g+p^n\mathbb{Z}$ has exact order $p^{n-1}(p-1)~\forall n \ge 0$, by using Lemma 3.41. So $g$ is a generator.
\end{proof}
\begin{remark} This means that, to find the generator of $(\mathbb{Z}/p^n\mathbb{Z})^*$, we just need to check $(\mathbb{Z}/p\mathbb{Z})^*, (\mathbb{Z}/p^2\mathbb{Z})^*$.
\end{remark}
~\\

Finally, we briefly discuss some non-linear congruences. Consider $f(x)$, a polynomial in one variable, and we want to find all solutions to $$f(x) \equiv 0~(\text{mod } m)$$ Then by Chinese remainder theorem, the problem is reduced to solving $f(x) \equiv 0$ (mod $p^\alpha$) for some prime $p$ and $\alpha \ge 1$.
Consider that, suppose we have a solution $x_0$ of the congruence
$$f(x) \equiv 0~(\text{mod } p^\alpha)$$ Then we use a version of Taylor's theorem, so that:
$$f(x_0+tp^\alpha) = f(x_0)+tp^\alpha f'(x_0) +\ldots + \frac{(tp^\alpha)^n}{n!}f^{(n)}(x_0)$$
where $n$ is the degree of $f$ and $t$ is an integer. As $f$ is a polynomial, so that consider the expansion of $f(x_0 +tp^\alpha)$, we have that the $k^{th}$ derivative of $f$ is divisible by $k!$. So that $f(x_0+tp^\alpha) \equiv 0$ (mod $p^{\alpha+1}$) if and only if
$$ f(x_0)+tp^\alpha f'(x_0) \equiv 0~(\text{mod } p^{\alpha+1})$$ which is equivalent to
$$tf'(x_0)=-f(x_0)/p^\alpha~(\text{mod } p)$$

Now if $p \nmid f'(x_0)$, then it has a unique $t$ and so the solution $x_0$ to $f(x_0) \equiv 0$ (mod $p^\alpha$) gives rise to a unique solution $x_0+tp^\alpha$. If $p|f'(x_0)$, then we are in the case
$$f(x_0+tp^\alpha) \equiv f(x_0)~(\text{mod } p^{\alpha+1})$$ and so either $x_0+tp^\alpha$ is a solution to
$$f(x) \equiv 0~(\text{mod } p^{\alpha+1})$$ for all $t$, or we have no solution corresponding to $x_0$.

\begin{example} Consider the congruence $f(x) \equiv 0$ (mod $27$) where $f(x)=x^3-2x^2+3x+9$.\\
STEP(i): As $27=3^3$, we start with $3$. Clearly, we have two solutions, $x=0,2$.\\
STEP(ii): Congruence mod 9.
\begin{enumerate}
\item[(a)] $x=0$. We have $f(0)=9$ and $f'(0)=3$, so $0$ is a solution mod $0$, and both $3$ and $6$ are solutions (mod 9).
\item[(b)] $x=2$. We have $f(2)=15$ and $f'(2)=7$, and the congruence:
$$ 7t \equiv -15/3~(\text{mod } 3)$$ has a solution $t$1. Hence $2+3=5$ is another solution (mod $9$).
\end{enumerate}
STEP(iii): Congruence mod $27$.
\begin{enumerate}
\item[(a)] $x=0$. As $3|f'(0), 27 \nmid f(0)$, so we have no solution.
\item[(b)] $x=3$. As $3|f'(3), 27|f(3)$, and so we have solutions $3,12,21$.
\item[(c)] $x=6$. As $3|f'(6), 27 \nmid f(6)$, so we have no solution.
\item[(d)] $x=5$. $f(5)=99, f'(5)=58, 3 \nmid 58$ so the congruence
$$58t \equiv -11~(\text{mod } 3)$$ has a solution $t=1$ and so $14$ is a solution.
\end{enumerate}
To conclude, we have solutions $3,12,14,21$ (mod $27$).
\end{example}
\subsection{Exercises}
\begin{enumerate}
\item If $x$ is an integer $\ge 2$, by considering the fundamental theorem, show that
$$x \le \left(1+\frac{log{x}}{log{2}}\right)^{\pi(x)}$$ 
\item Let $p_n$ denote the $n^{th}$ prime. Deduce that $p_{n+1} \le 2^{2^n}+1$, whence $\pi(x) \ge \log{\log{x}}$ for $x \ge 2$.
\item Let $q$ be an odd prime. Prove that every prime factor of $2^q-1$ must be congruent to $1$ mod $q$.
\item Prove that there exists infinitely many primes of the form $4k+3$.
\item Prove that if $(a,m)=(a-1,m)=1$, then
$$1+ a +a^2 + \ldots +a^{\phi(m)-1} \equiv 0~(\text{mod } m)$$
and deduce that every prime other than $2$ or $5$ divides infinitely many of the integers,$1,11,111,1111, \ldots$.
\item Show that if $m>4$ then $(m-1)! \equiv -1$ (mod $m$) if and only if $m$ is prime. (This proves the converse of Wilson's theorem)\\
Show further that if $p$ is an odd prime and $0<k<p$, then
    $$(p-k)!(k-1)! \equiv (-1)^k~(\text{mod } p)$$
\item Show that if $(m,n)=1$, then $m^{\phi(n)}+n^{\phi(m)} \equiv 1$ (mod $mn$)
\item Show that the congruence $x^{p-1}-1 \equiv 0$ (mod $p^j$) has just $p-1$ solutions (mod $p^j$) for every prime power $p^j$.
\item Solve the simultaneous congruences:
$$x \equiv 5~(\text{mod } 6)$$
$$7x \equiv 5~(\text{mod } 12)$$
$$ 17x \equiv 19~(\text{mod } 30)$$
\item Show that
$$x \equiv a~(\text{mod } m) \text{ and } x \equiv b~(\text{mod } n)$$
have a common solution if and only if $(m,n)|b-a$, and the solution is unique modulo the least common multiple of $m$ and $n$.
\item Show that $2$ and $3$ are both generators of $(\mathbb{Z}/5^n\mathbb{Z})^*$. Prove that, for every natural number $n$, either there is no primitive root (mod $n$) or there are $\phi(\phi(n))$ primitive roots (mod $n$).
\item Prove that, for any prime $p$, the sum of all the distinct primitive roots (mod $p$) is congruent to $\mu(p-1)$ (mod $p$).
\item Solve the congruence if possible:
$$ x^3 -2x +4 \equiv 0~(\text{mod } 125)$$
\item[14.] Suppose $a,n$ are integers $\ge 2$, and put $N=a^n-1$. Show that the order of $a+N\mathbb{Z}$ in $(\mathbb{Z}/N\mathbb{Z})^*$ is exactly $n$, and deduce that $n$ divides $\phi(N)$. If $n$ is a prime, deduce that there exists infinitely many primes $q$ such that $q \equiv 1$ (mod $n$).
\item[15.] Let $p$ be an odd prime. By considering the polynomial $x^{p-1}-1 - (x-1)(x-2)\ldots(x-p+1)$, show that if $u_p=1+\frac{1}{2}+\ldots \frac{1}{p-1}$, then the numerator of $u_p$ is divisible by $p$.$^{\star}$ Show further that it is also divisible by $p^2$.
\end{enumerate}

\section{Quadratic residue}
In the end of last chapter, we have seen some simple examples of non-linear congruence. We shall continue the development of congruence theory. In this chapter we shall focus on the quadratic case in detail and prove Gauss's famous law of quadratic reciprocity. A general quadratic congruence is of the form
$$ax^2+bx+c \equiv 0~(\text{mod } m) \text{ where } m \nmid a$$
But it can be reduced to the basic form $$x^2 \equiv a~(\text{mod m})$$
Firstly, by using Chinese Remainder theorem, and the last subsection of chapter 3, we may reduce it to the case when $m=p$ is a prime. As $p \nmid a$, we have some $a'$ such that $2aa' \equiv 1$ (mod $p$).
Then we multiply both sides by $2a'$, we have
$$x^2+2b'x+c' \equiv 0~(\text{mod } p)$$ where $b'=a'b,c'=2a'c$.
Completing the square, we have
$$(x+b')^2 \equiv {b'}^2-c'~(\text{mod } p)$$ which is of form $x^2 \equiv a$ (mod $p$).
Therefore, we shall focus on the congruence $x^2 \equiv a$ (mod $p$)
\subsection{Legendre symbol}
\begin{definition} Let $(a,p)=1$, we say $a$ is a {\bf quadratic residue} mod $p$ if $a+p\mathbb{Z}$ is a square in $\mathbb{Z}/p\mathbb{Z}$. In other words, $\exists x$ such that $x^2 \equiv a$ (mod $p$)
\end{definition}
\begin{lemma} Let $p$ be an odd prime, then the multiplicative group $(\mathbb{Z}/p\mathbb{Z})^*$ has precisely $\frac{p-1}{2}$ quadratic residues.
\end{lemma}
\begin{proof}[\bf Proof] Pick a primitive root $g$ mod $p$. Then each element can be written as $g^i$ for some $i$. If $i$ is even, then $g^i$ is a quadratic residue. Conversely, suppose $g^i$ is a quadratic residue, then we have $g^i=g^{2j}$ for some $j$, and so $2j \equiv i$ (mod $p-1$). As $p-1$ is even, so $i$ must be even. So $g^i$ is a quadratic residue if and only if $i$ is even, and there are $\frac{p-1}{2}$ of them.
\end{proof}
\begin{definition} Let $p$ be an odd prime. The Legendre symbol $\left(\frac{a}{p}\right)$ is defined as
\begin{equation*}
\left(\frac{a}{p}\right)= \left\{
\begin{array}{ll}
0 & \text{if } p|a\\
1 & \text{if } (a,p)=1, a \text{ is a quadratic residue} \\
-1 & \text{if } (a,p)=1, a \text{ is not a quadratic residue} \\
\end{array} \right.
\end{equation*}
\end{definition}
\subsection{Euler's criterion}
\begin{lemma}{\bf [Euler]} $(\frac{a}{p}) \equiv a^{\frac{p-1}{2}}$ (mod $p$)
\end{lemma}
\begin{proof}[\bf Proof] Write $P=\frac{p-1}{2}$. If $p|a$, then $(\frac{a}{p})=0 \equiv a^{P}$ (mod $p$).

If not, $a^{p-1}-1 \equiv (a^P-1)(a^P+1) \equiv 0$ (mod $p$).
Then $p|a^{p-1}-1 \Rightarrow p|a^P-1$ or $p|a^P+1$. Let $g$ be a primitive root mod $p$. WLOG, $a=g^k$ (mod $p$) for some $k \in \mathbb{Z}$. By Lemma 4.2, $(\frac{a}{p})=1$ if and only if $k$ is even.

Now if $k$ is even, then $a^P=g^{kP}=g^{\frac{k}{2} (p-1)} \equiv 1$ (mod $p$). If $k$ is odd, then $a^P \equiv -1$ (mod $p$). (This is because, if $x=a^P$, then $x^2 \equiv 1$ (mod $p$) and as $p$ is prime, we can only have $x \equiv \pm 1$ (mod $p$) ). Therefore, $(\frac{a}{p}) \equiv a^P$ (mod $p$).
\end{proof}
\begin{corollary} $(\frac{ab}{p}) = (\frac{a}{p}) (\frac{b}{p})$,
$(\frac{-1}{p})=(-1)^{\frac{p-1}{2}}$.
\end{corollary}
\begin{proof}[\bf Proof] Apply Lemma 4.4
\end{proof}
\subsection{Gauss's lemma}
It is often interesting in mathematics that an mathematical expression can be written in two different ways. Here is another expression of $(\frac{a}{p})$ by Gauss.
\begin{lemma}{\bf [Gauss]}\label{G;Gauss's lemma} Let $p$ be an odd prime, $(a,p)=1$, write $P=\frac{p-1}{2}$. For $j=1,2, \ldots P$, define $a_j \equiv j \cdot a$ (mod $p$) and $-\frac{p}{2} < a_j < \frac{p}{2}$.
Let $\nu(a)$ be the number of $a_j$ with $a_j<0$. Then $(\frac{a}{p})=(-1)^{\nu(a)}$.
\end{lemma}
\begin{proof}[\bf Proof] Let $S=\{\pm 1, \ldots \pm P\}$ and so $a_j \in S$. The subset given by $a_j: j=1,2 \ldots P$ consists of every integer $1,2 \ldots P$ with a unique sign.

Indeed, if $a_{j_1}=a_{j_2}$, then $j_1 a \equiv j_2 a$ (mod $p$), and so $j_1 \equiv j_2$ (mod $p$), $j_1=j_2$.
If $a_{j_1}=-a_{j_2}$, then $(j_1+j_2)a \equiv 0$ (mod $p$) and so $j_1 +j_2 \equiv 0$ (mod $p$), but $j_1,j_2 \le P=\frac{p-1}{2}$, and so this is impossible.\\
Hence
$$a \cdot 2a \ldots Pa \equiv a_1 \ldots a_P~(\text{mod } p)$$
is congruent to $1 \cdot 2 \ldots P (-1)^{\nu(a)}$ (mod $p$).
By Euler (Lemma 4.4), and we cancel $1 \cdot 2 \ldots P$ on both sides, we conclude that
$(\frac{a}{p} \equiv (-1)^{\nu(a)}$ (mod $p$), but both of them can only take values $\pm 1$, so we have $(\frac{a}{p})=(-1)^{\nu(a)}$.
\end{proof}
\begin{example} Let $a=2$, and let $p=8k+r,r=1,3,5,7$. We count the number of integers $x$ such that $\frac{p}{2} < 2x <p$, i.e. $\frac{p}{4} < x < \frac{p}{2}$.
\begin{enumerate}
\item $r=1$. We have $x=2k+1, \ldots 4k$, and so $\nu(2)=2k$.\\
\item $r=3$. We have $x=2k+1, \ldots 4k+1$, and so $\nu(2)=2k+1$.\\
\item $r=5$. We have $x=2k+2, \ldots 4k+2$, and so $\nu(2)=2k+1$.\\
\item $r=7$. We have $x=2k+2, \ldots 4k+3$, and so $\nu(2)=2k+2$.
\end{enumerate}
Now apply Lemma $4.7$, so we conclude that
\begin{equation*}
\left(\frac{2}{p}\right)= \left\{
\begin{array}{ll}
1 & \text{if } p \equiv \pm 1~(\text{mod } 8)\\
-1 & \text{if } p \equiv \pm 3~(\text{mod } 8)\\
\end{array} \right.
\end{equation*}
\end{example}
\subsection{Law of quadratic reciprocity}
The main theorem of the subsection is to prove the law of quadratic reciprocity, and we shall begin with the following idea.\\
Write $p=4am_p+r_p$, where $r_p$ satisfies $0<r_p<4a$, $(a,r_p)=1$.
\begin{theorem} Let $p_1$ and $p_2$ be any two odd primes with $(p_1,a)=(p_2,a)=1$ then:
(i) If $p_1 \equiv p_2$ (mod $4a$), we have $(\frac{a}{p_1})=(\frac{a}{p_2})$.\\
(ii) If $p_1 \equiv -p_2$ (mod $4a$), then $(\frac{a}{p_1})=(\frac{a}{p_2})$.\
\end{theorem}
To prove this, we need:
\begin{lemma} Assume $\alpha,\beta \in \mathbb{R}, \alpha \le \beta, \alpha \not \in \mathbb{Z}$. Then:\\
(i) The number of integers in $[\alpha,\beta]$ is $[\beta]-[\alpha]$.\\
(ii) If $n \in \mathbb{Z}$. $[n+\beta]=n+[\beta]$.\\
(iii) If $n_1,n_2$ are in $\mathbb{Z}$, $n_1 \le n_2$, then the number of integers in $[\alpha,\beta]$ and in $[2n_1+\alpha,2n_2 + \beta]$ have the same parity.
\end{lemma}
\begin{proof}(i) Since $\alpha \not \in \mathbb{Z}. [\alpha]+1,\ldots [\beta]$ has $[\beta]-[\alpha]$ integers.\\
(ii) Let $\beta_1=\beta-[\beta]$. Then $n+\beta=(n+[\beta])+\beta_1$, and so $[n+\beta]=n+[\beta]$.\\
(iii) The first interval has $[\beta]-[\alpha]$ integers, and the second one has $[\beta+2n_2]-[\alpha+2n_1] = [\beta]+2n_2-[\alpha]-2n_1$, by using the previous parts. Hence they have the same parity.
\end{proof}
Now we can prove Theorem 4.8:
\begin{proof}[\bf Proof] Consider $a,2a,\ldots Pa$, where $P=\frac{p-1}{2}$. Note that $Pa$ is the largest multiple of $a<\frac{pa}{2}$. The integers in the set $\{a,\ldots, Pa\}$ with a negative $a_j$ are those lying in the interval $[\frac{p}{2},p],[\frac{3p}{2},2p] \ldots [(b-\frac{1}{2})p,bp]$. We write $p=4m_p a +r_p$ and then $b=\frac{a}{2}$ if $a$ is even, and $b=\frac{a+1}{2}$ if $a$ is odd.

But the number of integers in the above intervals are precisely the number of integers in the interval
$$\left[\frac{p}{2a},\frac{p}{a}\right],\ldots \left[\frac{(2b-1)p}{2a},\frac{bp}{a}\right]$$
which is $$\left[2m_p+\frac{r_p}{2a},4m_p+\frac{r_p}{a}\right], \ldots \left[2m_p(2b-1)+\frac{(2b-1)r_p}{2a},4m_pb+\frac{b r_p}{a}\right]$$
As $(a,r_p)=1$, so none of the above boundaries is integer. Apply Lemma 4.9, it has the same parity as
$$\left[\frac{r_p}{2a},\frac{r_p}{a}\right] \ldots \left[\frac{(2b-1)r_p}{2a},\frac{br_p}{a}\right]$$
Recall from Gauss's lemma, that $(\frac{a}{p})=(-1)^{\nu(a)}$.
So for part (i) of the theorem, if $p_1 \equiv p_2$ (mod $4a$), i.e. $r_{p_1} =r_{p_2}$, then
$(\frac{a}{p_1})=(\frac{a}{p_2})$.

For (ii), $p_1 \equiv -p_2$ (mod $4a$), so $r_{p_2}=4a-r_{p_1}$. Substitute this into the expression,
so consider $$\left[2-\frac{r_{p_1}}{2a},4-\frac{r_{p_1}}{a}\right] \ldots \left[4b-2-\frac{(2b-1)r_{p_1}}{2a},4b-\frac{br_{p_1}}{a}\right]$$
If $\alpha \not \in \mathbb{Z}$, then $[-\alpha]=-[\alpha]-1$, so $[4-\frac{r_{p_1}}{a}]-[2-\frac{r_{p_1}}{2a}]=2-[\frac{r_{p_1}}{a}]+[\frac{r_{p_1}}{2a}]$ etc.

Therefore, $[\frac{p_i}{2a},\frac{p_i}{a}],\ldots [\frac{(2b-1)p}{2a},\frac{bp}{a}]$  have the same parity, and so we again have $(\frac{a}{p_1})=(\frac{a}{p_2})$.
\end{proof}

\begin{theorem}{\bf [Law of quadratic reciprocity]}\label{R;Reciprocity} Let $p,q$ be distinct odd primes, then $(\frac{p}{q})(\frac{q}{p}) = (-1)^{\frac{p-1}{2} \frac{q-1}{2}}$. In particular,
$(\frac{p}{q})=(\frac{q}{p})$ when at least one of $p,q \equiv 1$ (mod $4$), and $(\frac{p}{q}) = -(\frac{q}{p})$ if both $p,q \equiv 3$ (mod $4$).
\end{theorem}
\begin{proof}[\bf Proof] We have two cases:
\begin{enumerate}
\item $p \equiv q$ (mod $4$). WLOG, let $p>q$. Then $p=q+4a$ for some $a$, and then we write $p=4am_p+r_p, q= 4am_q+r_q, r_p,r_q <4a$, we have $4a(m_q+1)+r_q=4am_p+r_p$. Hence $r_p=r_q$, and so $p \equiv q$ (mod $4a$). By Theorem 4.8(i), we have $(\frac{a}{p})=(\frac{a}{q})$.
    Now
    $$\left(\frac{p}{q}\right)=\left(\frac{q+4a}{q}\right)=\left(\frac{4a}{q}\right)
    =\left(\frac{a}{q}\right)$$
    and
    $$\left(\frac{q}{p}\right)=\left(\frac{p-4a}{p}\right)
    =\left(\frac{-1}{p}\right)\left(\frac{a}{p}\right)=(-1)^{\frac{p-1}{2}}\left(\frac{a}{p}\right)$$
    by Corollary 4.5.

    Hence $(\frac{p}{q})=(-1)^{\frac{p-1}{2}}(\frac{q}{p})$ when $p \equiv q$ (mod $4$), which is the same as $(\frac{p}{q})(\frac{q}{p})=(-1)^{\frac{p-1}{2} \frac{q-1}{2}}$.\\
\item $p \not \equiv q$ (mod $4$), then $p \equiv -q$ (mod $4$). So we can write $p+q=4a$ for some $a$, and so $p \equiv -q$ (mod $4a$), by Theorem 4.8(ii), we again have $(\frac{a}{p})=(\frac{a}{q})$.
    Now
    $$\left(\frac{p}{q}\right)=\left(\frac{4a-q}{q}\right)=\left(\frac{a}{q}\right)$$
    and
    $$\left(\frac{q}{p}\right)=\left(\frac{4a-p}{p}\right)=\left(\frac{a}{p}\right)$$
    So we have $(\frac{q}{p})=(\frac{q}{p})$ when $p \equiv -q$ (mod $4$), i.e. at least one of them is $1$ mod $4$ and hence
    $$\left(\frac{p}{q}\right)\left(\frac{q}{p}\right)=1=(-1)^{\frac{p-1}{2} \frac{q-1}{2}}$$
\end{enumerate}
\end{proof}
Now the calculation of the Legendre symbol is much easier.
\begin{example}
$(\frac{7411}{9283})=-(\frac{9283}{7411})=-(\frac{1872}{7411})$\\
~\\
$1872=2^4 3^2$, so we have $-(\frac{13}{7411})=-(\frac{7411}{13})=-(\frac{1}{13})=-1$.
\end{example}
\subsection{Jacobi symbol}
Let $n$ be any {\bf odd} integer $>1$, and write $n=q_1 q_2 \ldots q_r$, where $q_i$ are prime but not necessarily distinct.
\begin{definition} If $a \in \mathbb{Z}$, the Jacobi symbol is defined as
$$\left(\frac{a}{n}\right)=\prod_{i=1}^{r}\left(\frac{a}{q_i}\right)$$
\end{definition}
\begin{remark}
\begin{enumerate}
\item $(\frac{a}{n})$ only depends on $a+n\mathbb{Z}$ because if $a \equiv b$ (mod $n$), then
$a \equiv b$ (mod $p_i$) $\forall i$.
\item $(\frac{ab}{n})=(\frac{a}{n})(\frac{b}{n})$ by Corollary $4.5$.
\item $(\frac{a}{mn})=(\frac{a}{m})(\frac{a}{n})$ by definition.
\end{enumerate}
\end{remark}
\begin{lemma} $(\frac{-1}{n})=(-1)^{\frac{n-1}{2}},(\frac{2}{n})=(-1)^{\frac{n^2-1}{8}}$.
\end{lemma}
\begin{proof}[\bf Proof] For the first identity, we shall apply Corollary 4.5 and check the fact that for any two odd integers $n_1,n_2$, we have $n_1 n_2 \equiv n_1+n_2-1$ (mod $4$). This is because
$(n_1-1)(n_2)-1 \equiv 0$ (mod $4$). So that
$$\left(\frac{-1}{n_1}\right)\left(\frac{-1}{n_2}\right)=(-1)^{\frac{n_1-1}{2}}(-1)^{\frac{n_2-1}{2}}
=(-1)^{\frac{n_1 n_2 -1}{2}}$$
For the second identity, we have $({n_1}^2-1)({n_2}^2-1) \equiv 0$ (mod $16$). Then we have
${n_1}^2{n_2}^2-1 \equiv {n_1}^2+{n_2}^2-2$ (mod $16$).
Using Example 4.7, we have
$$\left(\frac{2}{n_1}\right)\left(\frac{2}{n_2}\right)
=(-1)^{\frac{{n_1}^2-1}{8}}(-1)^{\frac{{n_1}^2-1}{8}}=(-1)^{\frac{{n_1}^2{n_2}^2-1}{8}}$$
\end{proof}
\begin{theorem} If $m,n$ are odd positive integers. then
$$\left(\frac{m}{n}\right)=\left(\frac{n}{m}\right)(-1)^{\frac{m-1}{2} \frac{n-1}{2}}$$
\end{theorem}
\begin{proof}[\bf Proof] The case when $(m,n)=1$ is clear, as both sides are $0$. So assume now $(m,n)=1$.
Let $m=p_1 \ldots p_r,n=q_1 \ldots q_s$, where $p_i,q_j$ are prime but not necessarily distinct.
Now
$$\left(\frac{m}{n}\right)=\prod_{i=1}^{r} \prod_{j=1}^{s}\left(\frac{p_i}{q_j}\right)
\text{ and } \left(\frac{n}{m}\right)=\prod_{i=1}^{r} \prod_{j=1}^{s} \left(\frac{q_j}{p_i}\right)$$
Let $u$ be the number of $p_i$ which is congruent to $3$ mod $4$, and $v$ be the number of $q_j$ which is congruent to $3$ mod $4$. We know by Theorem 4.10, $(\frac{p_i}{q_j})=-(\frac{q_j}{p_i})$ if and only if both $p_i,q_j \equiv 3$ (mod $4$), and we have $uv$ pairs of such $p_i,q_j$.

Therefore, $(\frac{m}{n})=(\frac{n}{m})(-1)^{uv}$. Finally we check that the parity of $uv$ is the same as $\frac{n-1}{2}\frac{m-1}{2}$. This is because that each odd prime is either $1$ or $3$ mod $4$,
and so $u$ is odd if and only if $m \equiv 3$ (mod $4$). Similarly, $v$ is odd if and only if
$n \equiv 3$ (mod $4$). Hence, they have the same parity and the result follows.
\end{proof}
\begin{remark}
The definition of Jacobi symbol has nothing to do with quadratic residues. For example,
let $n=15$, then we can check that $x^2 \equiv 2$ (mod $15$) is not soluble. But
$(\frac{2}{15})=(\frac{2}{3})(\frac{2}{5})=(-1)(-1)=1$.
\end{remark}
\subsection{Kronecker symbol}
\begin{definition} Let $d$ satisfy: $d \equiv 0$ or $1$ (mod $4$), and $d$ is not square. The Kronecker symbol $(\frac{d}{n})$ is defined when $n>0$ by:
\begin{enumerate}
\item[(i)] $(\frac{d}{n})=0$ if $(d,n)>1$;
\item[(ii)] $(\frac{d}{1})=1$;
\item[(iii)] If $d$ is odd, $(\frac{d}{2})=(\frac{2}{|d|})$, a Jacobi symbol, so
\begin{equation*}
\left(\frac{d}{2}\right)= \left\{
\begin{array}{ll}
1 & \text{if } d \equiv 1~(\text{mod } 8)\\
-1 & \text{if } d \equiv 5~(\text{mod } 8)\\
\end{array} \right.
\end{equation*}\\
\item[(iv)] If $n=\prod_{i=1}^{r}p_i$, then $(\frac{d}{n})=\prod_{i=1}^{r}(\frac{d}{p_i})$
\end{enumerate}
So it is a product of Legendre symbols, and the symbol $(\frac{d}{2})$ if $n$ is even.
\end{definition}
Here is an equivalent definition:
\begin{theorem} Let $n>0$ and $(d,n)=1$ with $d$ as above. Then
(i) If $d$ is odd, we have $(\frac{d}{n})=(\frac{n}{|d|})$, a Jacobi symbol.\\
(ii) If $d=2^s t$, where $t$ is odd and $s>0$, we have:
\begin{center}
$(\frac{d}{n})=(\frac{2}{n})^s (-1)^{\frac{(t-1)(n-1)}{4}}(\frac{n}{|t|})$
\end{center}
a product of Jacobi symbols.
\end{theorem}
\begin{proof}
(i)Consider the case $d>0, n=2^u v$ where $v$ is odd. As $d$ is odd so $d \equiv 1$ (mod $4$),
and so
$$\left(\frac{d}{n}\right)=\left(\frac{d}{2^u v}\right)
=\left(\frac{d}{2}\right)^u \left(\frac{d}{v}\right)
=\left(\frac{2}{d}\right)^u \left(\frac{v}{d}\right)=\left(\frac{n}{d}\right)$$
For the case $d<0$, we use the fact that:
\begin{enumerate}
\item $(\frac{d}{2})=(\frac{2}{|d|})$.\\
\item $(\frac{d}{v})=(\frac{-1}{v})(\frac{-d}{v})=(\frac{-1}{v})(\frac{|d|}{v})$.\\
\item $-d \equiv 3$ (mod $4$) and so $(\frac{-d}{v})=(\frac{v}{d}) (-1)^{\frac{v-1}{2}}$.
\end{enumerate}
(ii) $d=2^s t$, where $t$ is odd and $s>0$, if $n$ is even, then $(d,n)>1$ and so both sides are $0$. So assume $n$ is odd and we have $(\frac{d}{n})$ is already a Jacobi symbol and so:
$$\left(\frac{d}{n}\right)=\left(\frac{2^s}{n}\right)\left(\frac{t}{n}\right)$$
By definition, as $n>0$ and is odd, we have $(\frac{2^s}{n})=(\frac{n}{2^s})=(\frac{n}{2})^s=(\frac{2}{n})^s$. Then:\\
If $t>0$, we have
$$\left(\frac{t}{n}\right)=(-1)^{\frac{t-1}{2}\frac{n-1}{2}} \left(\frac{n}{t}\right)$$
and so the result follows.\\
If $t<0$, we have $t=-|t|$, and so
$$\left(\frac{t}{n}\right)=(-1)^{\frac{n-1}{2}}(-1)^{\frac{|t|-1}{2}\frac{n-1}{2}} \left(\frac{n}{|t|}\right)$$ Hence the result follows.
\end{proof}
\begin{theorem} If $m \equiv n$ (mod $|d|$), then $(\frac{d}{m})=(\frac{d}{n})$.
\end{theorem}
\begin{proof}[\bf Proof] Assume $(m,n)=1$, otherwise both sides are $0$ and the result is trivial.
If $d$ is odd, then use Theorem 4.18, we have
$$\left(\frac{d}{m}\right)=\left(\frac{m}{|d|}\right)=\left(\frac{n}{|d|}\right)=\left(\frac{d}{n}\right)$$
If $d$ is even, let $d=2^s t$. Then we have
$$\left(\frac{d}{m}\right)=\left(\frac{2}{m}\right)^s (-1)^{\frac{x(m-1)}{2}}\left(\frac{m}{|t|}\right)$$
$$\left(\frac{d}{n}\right)=\left(\frac{2}{n}\right)^s (-1)^{\frac{x(n-1)}{2}}\left(\frac{n}{|t|}\right)$$
where $x=\frac{t-1}{2}$. Now we again we $(\frac{m}{|b|})=(\frac{n}{|b|})$, and as $d \equiv 0$ (mod $4$), we have
$$(-1)^{\frac{x(m-1)}{2}}=(-1)^{\frac{x(n-1)}{2}}$$
and also
$$\left(\frac{2}{m}\right)^s=\left(\frac{2}{n}\right)^s$$
by Lemma 4.14. Note that we must have $s \ge 2$ as $d \equiv 0$ (mod $4$) and if $s \ge 2$, then $m \equiv n$ (mod $4$).
\end{proof}

\subsection{Pseudo prime}

With the notion of quadratic residues, we are now able to introduce the concept of pseudo primes. Throughout the subsection, $N$ is odd.
\begin{definition} We say an integer $b$ is a base for $N$, if $(b,N)=1$. Usually, we take
$b \in \{1,2,\ldots N-1\}$.
\end{definition}
\begin{definition} We say $N$ is a {\bf pseudo prime} to the base $b$ if $b^{N-1} \equiv 1$ (mod $N$).
\end{definition}
\begin{remark} By Fermat's little theorem, $b^{p-1} \equiv 1$ (mod $p$) and so $p$ is a pseudo prime for every base.
\end{remark}
But if a number $N$ is a pseudo prime for every base $b$, $N$ needs not be a prime.

\begin{example} 
$N=561=3 \cdot 10 \cdot 17$, $N-1=560$, which is divisible by $2,10,16$.
So let $b$ be a base of $561$, then $b^{560} \equiv 1$ (mod $3,11,17$) and so by Chinese Remainder theorem, $b^{561} \equiv 1$ (mod $561$). But clearly $561$ is not a prime.
\end{example}

\begin{definition} We say an odd {\bf composite} integer $n>1$ is a {\bf Carmichael number}\label{C;Carmichael}, if $b^{N-1} \equiv 1$ (mod $N$) for every integer $(b,N)=1$.
\end{definition}
Here is an interesting result of Carmichael number which we do not give the proof in this book.
\begin{theorem} There are infinitely many Carmichael number. Let $C(x)$ be the number of Carmichael number $<x$, then $c(x)> x^{\frac{2}{7}}$, where $x >> 0$.
\end{theorem}
\begin{theorem} Assume $b$ is any integer $>1$, then there exists infinitely many composite number $N$, such that $b^{N-1} \equiv 1$ (mod $N$).
\end{theorem}
\begin{proof}[\bf Proof] Take any prime $p$ which satisfies: $p>2$, $(p,b)=1$ and $(p, b^2-1)=1$.
Let $N=\frac{b^{2p}-1}{b^2-1}$. Clearly, $N$ is an positive integer. We firstly check that $N$ is composite. We can write $N$ as:
$$N=\frac{b^p-1}{b-1} \frac{b^p+1}{b+1}$$ and both are integers greater than $1$, and so $N$ is composite.
Then we have $b^p \equiv b$ (mod $p$) by Fermat's little theorem, and so $b^{2p} \equiv b^2$ (mod $p$).
As we have chosen $(p,b^2-1)=1$, so that $b^2-1 \nmid p$, and so
$$N-1=\frac{b^{2p}-b^2}{b^2-1} \equiv 0~(\text{mod } p)$$
and so $p|N-1$.
Now
$$N=\frac{b^{2p}-1}{b^2-1} = b^{2p-2}+\ldots b^2+1$$
we have $p$ terms in the summation and as $p-1$ is even so $b^{2p-2}+\ldots b^2$ must be even.
So $N$ is odd and $N-1$ is even. Let $N-1=2pm$. Then
$$b^{2p}-1=(b^2-1)N$$
Thus,
$$b^{2p} \equiv 1~(\text{mod } N) \text{ and so } b^{N-1} \equiv b^{(2p)m} \equiv 1~(\text{mod } N)$$
Thus, as we have infinitely many $p$ which satisfies the condition at the beginning, so we can construct infinitely many such $N$.
\end{proof}
\begin{definition} We say $N$ is an {\bf Euler pseudo prime}\label{E;Euler pseudo} to the base $b$, if
$$b^{\frac{N-1}{2}} \equiv \left(\frac{b}{N}\right)~(\text{mod } N)$$
\end{definition}
\begin{lemma} Assume $N$ is prime and $b$ is any base, $(b,N)=1$. Write $N-1=2^s t$, where $s \ge 1, t$ is odd. Then either $b^t \equiv 1$ (mod $N$) or $\exists r$ with $0 \le r <s$ such that $b^{2^r t} \equiv -1 $ (mod $N$).
\end{lemma}
\begin{proof}[\bf Proof] If $b^t \equiv 1$ (mod $N$), then we have the first case.
If not, we have $b^t \not \equiv 1$ (mod $N$). But $b^{2^s t} \equiv 1$ (mod $N$) by Fermat's little theorem. Then $\exists k$ such that $k$ is the largest integer that $b^{2^k t} \not \equiv 1$ (mod $N$).
So $b^{2^{k+1} t} \equiv 1$ (mod $N$), i.e. $(b^{2^k t})^2 \equiv 1$ (mod $N$). As $N$ is prime, so we have at most two solutions, and so we must have $b^{2^k t} \equiv -1$ (mod $N$), which is the second case.
\end{proof}
\begin{definition} Assume $(b,N)=1$. We say $N$ is a {\bf strong pseudo prime}\label{S;Strong pseudo} if either $b^t \equiv 1$ (mod $N$) or $\exists r$ with $0 \le r < s$, such that $b^{2^r t} \equiv -1$ (mod $N$), where $N-1=2^s t$.
\end{definition}
\begin{definition} $S(N)=\{b: 1 \le b < N, (b,N)=1, N$  is a strong pseudo prime to the base $ b\}$
\end{definition}
Clearly $\card(S(N)) \le \phi(N)$, and by Lemma 4.28, equality holds when $N$ is a prime.
\begin{theorem}{\bf [Mercer]}\label{M;Mercer} Assume $N$ is an odd integer $>9$, then
$$\card(S(N)) \le \frac{1}{4} \phi(N)$$
\end{theorem}
We shall prove several lemmas before we prove the theorem.
\begin{definition} Let $u(N)$ denote the largest integer such that $2^{u(N)}$ divides $p-1$ for {\bf every} prime divisor $p$ of $N$. Let $U(N)=\{b: 1 \le b \le N, (b,N)=1 \text{ and } b^{2^{u(N)-1} t} \equiv \pm 1 \text{ mod } N\}$
\end{definition}
\begin{lemma} $S(N) \subset U(N)$.
\end{lemma}
\begin{proof}[\bf Proof] Take $b \in S(N)$. If $b^t \equiv 1$ (mod $N$), then clearly
$$b^{2^{u(N)-1} t} \equiv 1~(\text{mod } N) $$ and so $b \in U(N)$.

Now assume $b^{2^r t} \equiv -1$ (mod $N$), for some $0 \le r <s$. Take any prime $p|N$. Define $k_p=$ order of $b$ in $(\mathbb{Z}/p\mathbb{Z})^*$. As
$$b^{2^{r+1}t} = (b^{2^r t})^2 \equiv 1~(\text{mod } N\}$$
we have, $b^{2^{r+1}t} \equiv 1$ (mod $p$). So $k_p | 2^{r+1}t$, and $k_p \nmid 2^r t$. Hence $k_p$ must take the factor $2^{r+1}$.

Then we can write $k_p=2^{r+1}k_p'$, by Fermat's little theorem, we have $k_p |p-1$, and so
$2^{r+1} | p-1$, and so $r+1 \le u(N)$. Now we have two cases.
\begin{enumerate}
\item[(i)] If $r+1=u(N)$, then we have $b^{2^{u(N)-1}t} \equiv -1$ (mod $N$) because $u(N)-1 =r$ and so $b \in U(N)$.\\
\item[(ii)] If $r+1 < u(N)$, and so $u(N)-1 \ge r+1$. So $b^{2^{u(N)-1}t} \equiv 1$ (mod $N$) and so again $b \in U(N)$.
\end{enumerate}
Hence $S(N) \subset U(N)$.
\end{proof}
\begin{definition} Let $w(N)=$ number of distinct prime factors of $N$.
\end{definition}
\begin{theorem} $\card(U(N)) = 2 \cdot 2^{(u(N)-1)w(N)}\prod_{p \text{ prime}|N}(t,p-1)$
\end{theorem}
\begin{proof}[\bf Proof] Define $m=2^{u(N)-1}t$. We must count the number of $b$ mod $N$ satisfying either
$x^m \equiv 1$ (mod $N$) or $x^m \equiv -1$ (mod $N$). Let $N={p_1}^{j_1} \ldots {p_k}^{j_k}$, where $j_i \ge 1$, and $p_i$ distinct prime. So $w(N)=k$.

For $x^m \equiv 1$ (mod $N$). By Chinese Remainder theorem, we count the solutions to $x^m \equiv 1$ (mod ${p_i}^{j_i}$) for each $i$. We know that $(\mathbb{Z})/{p}^{j}\mathbb{Z})^*$ is cyclic. So let $g$ be the generator. Consider that if $(g^i)^m=1$, and as the order of the group is $p^{j-1}(p-1)$, so $im=kp^{j-1}(p-1)$ for some $k$, then we have exactly $d$ solutions, where $d=(m,p^{j-1}(p-1))$.
But $p|N, (2t,N)=1$ so that $(m,p^{j-1})=1$ and so we have $(m,p^{j-1}(p-1)=(m,p-1)$.
But $2^{u(N)} | p-1$, so $(m,p-1)=2^{u(N)-1}(t,p-1)$. Thus for each $p_i$, we have these solutions, and so in total, using Chinese Remainder theorem, we have $$(2^{u(N)-1})^{w(N)}\prod_{p \text{ prime}|N}(t,p-1)$$

For $x^m \equiv -1$ (mod $N$). We consider the solutions to $x^{2m} \equiv 1$ (mod ${p_i}^{j_i}$) for each $i$, then exclude those solutions which also satisfy $x^{m} \equiv 1$ (mod ${p_i}^{j_i}$). This is using $p^j|(x^m-1)(x^m+1) \Rightarrow p^j |x^m-1$ or $p|x^m+1$, because $p$ is prime.
Then we run through the same argument as above, we shall get the number of solutions to $x^m \equiv -1$ (mod $N$) is also $$(2^{u(N)-1})^{w(N)} \prod_{p \text{ prime}|N}(t,p-1)$$

Hence the result follows by combining both parts.
\end{proof}
\begin{corollary} If $N$ is composite and $N \ge 9$, then $\card(U(N)) \le \frac{1}{4} \phi(N)$
\end{corollary}
\begin{proof}[\bf Proof] Let $\delta(N)=\frac{\phi(N)}{\card(U(N))}$, and we can write
$$\phi(N)=\prod_{p \text{ prime}|N}p^{j-1}(p-1)$$
By Theorem 4.35, we have
$$\delta(N)=\frac{1}{2} \frac{\prod_{p|N}p^{j-1}(p-1)}{\prod_{p|N}2^{u(N)-1}(t,p-1)}
=\frac{1}{2}\prod_{p^j||N}\frac{p^{j-1}(p-1)}{2^{u(N)-1}(t,p-1)}$$
where $p^j||N$ means $p^j$ exactly divides $N$, i.e. $p^{j+1} \nmid N$.

If $w(N) \ge 2$ and $j \ge 2$ for some $p|N$, then $\delta(N) \ge 6$, and so $\delta(N) \ge 4$, because
$\frac{p-1}{2^{u(N)-1}} \ge 2$ and $p \ge 3$ as $p$ must be odd prime.
So it is $\ge \frac{1}{2} 2^2 \cdot 3 =6$. Similarly, if $w(N) \ge 3$ then $\delta(N) \ge 2 \cdot 2 =4$.
Hence the result follows.

So we only need to deal with some other cases.
\begin{enumerate}
\item $w(N)=2,N=pq$, where $p,q$ distinct prime, and WLOG, assume $p<q$.
$$\delta(N)=\frac{1}{2} \frac{(p-1)(q-1)}{2^{u(N)-1}(t,p-1)2^{u(N)-1}(t,q-1)}$$
where $2^{u(N)}|p-1,q-1$. \\
~\\
If $2^{u(N)+1}|q-1$, then $\frac{q-1}{2^{u(N)-1}(t,q-1)} \ge 4$ and so the result follows.
If not, we have $2^{u(N)}||q-1$, but $$N-1=pq-1 \equiv p-1~(\text{mod } q-1)$$ So $2^{u(N)}|N-1$.
But $q-1 \nmid N-1$, so there must be some odd factor $r |q-1$ but $r \nmid N-1$. $r \ge 3$, and so
$\frac{q-1}{(t,q-1)} \ge r \ge 3$ and again $\delta(N) \ge 6$. Hence the result follows.\\
\item $N=p^j$ for some $j \ge 2$, because $N$ is composite. Then
$$\delta(N)=\frac{p^{j-1}(p-1)}{2^{u(N)-1}(t,p-1)} \ge p^{j-1}$$
$p$ is odd so $p \ge 3$. Either $p=3$, then $j \ge 3$ because we assume that $N \ge 9$; Or $p \ge 4$. But in both cases we have $\delta(N) \ge 4$.
\end{enumerate}
Combining all these cases, we have proved the assertion.
\end{proof}
Finally, to prove Theorem 4.31, we simply apply Corollary 4.36 and Lemma 4.33.
\subsection{Exercises}
\begin{enumerate}
\item Determine the primes $p$ for which $5$ is  quadratic residue (mod $p$).
\item Let $p$ be a prime such that $p \equiv 1$ (mod $4$). Show that the sum of quadratic residues in the interval $[1,p-1]$ is equal to the sum of quadratic non-residues in the interval.
\item Show that if $p$ is an odd prime then the product $P$ of all the quadratic residues mod $p$ satisfies
    $$P \equiv (-1)^{\frac{1}{2} (p+1)}~(\text{mod p})$$
\item Show that, for any integer $d$ and any odd prime $p$, then number of solutions of the congruence $x^2 \equiv d$ (mod $p$) is $1+(\frac{d}{p})$.
\item Show that if $p$ is an odd prime, then the number of quadratic residues mod $p$ is equal to the quadratic non-residues. Evaluate
    $$\sum_{a=1}^{p-2}\left(\frac{a}{p}\right)\left(\frac{a+1}{p}\right)$$
\item Show that if $q$ is a prime, $p>3$ with $p \equiv 3$ (mod $4$) such that $2p+1$ is also a prime number, then $2^p-1$ is composite.
\item Let $N$ be a positive odd integer and assume that $N$ is divisible by a square of a prime number $p$. Prove that there exists an integer $z$ such that $z^p \equiv 1$ (mod $N$) but $z \not \equiv 1$ (mod $N$). Hence prove that there exists an integer $a$ with $(a,n)=1$ such that the congruence
    $$a^{\frac{n-1}{2}} \equiv \left(\frac{a}{N}\right)~(\text{mod } n)$$
    does not hold. Show further that this congruence fails to hold for at least half of all
    relatively prime residue classes modulo $N$.
\item Show that an integer $a$ is a square if and only if the congruence
     $$x^2 \equiv a~(\text{mod } p)$$
     is soluble for every prime $p$.
\item Let $f(x)=ax^2+bx+c$, where $a,b,c$ are integers, and let $p$ be an odd prime that does not divide $a$. Further let $d=b^2-4ac$. Show that, if $p$ does not divide $d$, then
    $$\sum_{x=1}^{p} \left(\frac{f(x)}{p}\right)=-\left(\frac{a}{p}\right)$$
    Evaluate the sum when $p$ divides $d$.\\
    (Hint: You may use the fact that when $p \nmid d$, the number of solutions to the congruence
    $$x^2-y^2 \equiv d~(\text{mod } p)$$
    is $p-1$. We shall prove this later by Jacobi sum).
\item If $p$ is a prime $>3$ with $p \equiv 3$ (mod $8$), prove that
$$\prod_{r=1}^{\frac{p-1}{2}}(\frac{r}{p})$$ is always divisible by $3$.
\item Using the Kronecker symbol show that $(\frac{d}{|d|-1})=sgn(d)$ where $sgn$ is the sign function ($sgn(d)=1$ if $d>0$ and $sgn(d)=-1$ if $d<0$). Deduce that if $m>0,n>0$, and $m+n \equiv 0$ (mod $|d|$) then $(\frac{d}{n})=(\frac{d}{m})sgn(d)$.
\item Let $p$ be a prime number such that $2p-1$ is also prime. Show that $N=p(2p-1)$ is a pseudo prime for precisely half of all bases.
\item (i) Let $p$ be a prime $>5$. Prove that $N=\frac{4^p+1}{5}$ is always a composite integer. Show further that $N$ is always a strong pseudo prime to the base $2$. \\
     (ii) Now Let $M$ be an odd integer greater than $1$ such that $M$ is a pseudo prime to the base $2$. Prove that $2^M -1$ is always a strong pseudo prime to the base $2$.
\item[$^{\star}$14.] This is a short, but tricky proof of quadratic reciprocity which uses Chinese Remainder theorem in place of Gauss's Lemma.
    \begin{enumerate}
    \item[(i)] Suppose $p$ and $q$ are distinct odd primes. By considering the group
    $(\mathbb{Z}/p\mathbb{Z})^* \times (\mathbb{Z}/q\mathbb{Z})^*$ and its subgroup
    $U=\{(1,1),(p-1,q-1)\}$, show that
    $$K=\{(i,j): i=1,\ldots p-1;j=1,\ldots \frac{q-1}{2}\}$$
    is a complete system of coset representatives for $U$ in $(\mathbb{Z}/p\mathbb{Z})^* \times (\mathbb{Z}/q\mathbb{Z})^*$, and the product of all elements of $K$ equals
    $$\Pi= ((p-1)!^{\frac{q-1}{2}},(q-1)!^{\frac{p-1}{2}}(-1)^{\frac{(p-1)(q-1)}{4}}).$$\\
    \item[(ii)] Secondly consider the set
    $$L=\{(t_p,t_q): t=1, \ldots \frac{pq-1}{2} \text{ and } (t,pq)=1\}$$
    where $t_p \equiv t$ (mod $p$) and $0 \le t_p <p$, and $t_q$ is defined similarly. Using the ring version of Chinese Remainder theorem, show that this is also a complete system of coset representatives for $U$ in $(\mathbb{Z}/p\mathbb{Z})^* \times (\mathbb{Z}/q\mathbb{Z})^*$. Now calculate the product $\Phi$ of all elements of $L$ as follows; show that the first entry of $\Phi$ is:
    $$\frac{1 \ldots (p-1)(p+1)\ldots(2p-1)\ldots
    (\frac{(q-1)p}{2}+1) \ldots (\frac{(q-1)p}{2}+\frac{p-1}{2})}
    {q \cdot 2q \ldots \cdot \frac{(p-1)q}{2}}$$
    which is congruent to $\frac{(p-1)!^{\frac{q-1}{2}}}{q^{\frac{p-1}{2}}}$ mod $p$. Repeat this calculation for the second entry and hence show that
    $$\Phi=((p-1)!^{\frac{q-1}{2}}\left(\frac{q}{p}\right),
    (q-1)!^{\frac{p-1}{2}}\left(\frac{p}{q}\right))$$\\
    \item[(iii)] Prove law of reciprocity by nothing that $\Pi = \Phi$ in the quotient group of $U$.
    \end{enumerate}
\end{enumerate}

\section{Continued fraction}
\subsection{Basic properties}
We have seen many rational approximations of the irrational $\pi$. For example,
$\frac{22}{7}, \frac{355}{113}$, and in fact these are in some sense the best approximations of $\pi$.
In this chapter we are going to introduce the concept of continued fraction to construct these magic fractions which are going to make very good approximations of many irrationals.
We begin with the following definition:
\begin{definition} A continued fraction is an expression of the form
$$a_0+\frac{1}{a_1+\frac{1}{a_2+\ldots}}$$ with either a finite or infinite number of entries $a_i$,where $a_i$ are positive integers. We shall use the notation $[a_0,a_1,a_2,\ldots]$ for this expression.

Formally, to obtain the continued fraction for a real number $\alpha$.
let $a_0 = [\alpha]$. If $a_0 = \alpha$ then we stop and $\alpha=[a_0]$.
If not, then $\alpha=a_0+\frac{1}{\alpha_1},\alpha_1>1$ and let $a_1=[\alpha_1]$.
If $a_1=[\alpha_1]$ then we stop and $\alpha=[a_0,a_1]$, and if not, we continue as in the above
procedure. We call these $a_i$ the {\bf partial quotients} of $\alpha$.
\end{definition}
\begin{lemma} The series $a_i$ is finite if and only if $\alpha$ is rational.
\end{lemma}
\begin{proof}[\bf Proof] If $a_i$ is finite then clearly $\alpha$ is rational.
Conversely, if $\alpha$ is rational, let $\alpha=\frac{a}{b}$, and WLOG, $b>0$, $(a,b)=1$.

Let $a=ba_0+r_1$, and so $\alpha=a_0+\frac{r_1}{b}=a_0+\frac{1}{\alpha_1}$.
Now as $r_1,b>0$ then we can apply Euclidean algorithm to $b$ and $r_1$,
$b=r_1a_1+r_2$ etc. and then the partial quotients $a_i$ come from the sequence $r_i$, but as Euclidean
algorithm terminates, so the series of partial quotients must be finite.
\end{proof}

It is not of great interest to find the continued fraction of $\alpha$ when it is rational, thus we shall assume from now on that $\alpha$ is irrational.

\begin{example}
$\pi=[3,7,15,1,\ldots]$\\
$e=[2,1,2,1,1,4,1,1,6,\ldots]$\\
$\sqrt[3]{2}=[1,3,1,5,1,1,4,\ldots]$\\
$\sqrt{41}=[6,2,2,1,2,2,2,1,\ldots]$\\
\end{example}

\begin{definition} The $n^{th}$ {\bf convergent} of $\alpha$, $\frac{p_n}{q_n}$ is defined as:\\
$p_0=a_0,p_1=a_1a_0+1$, and $p_n=a_np_{n-1}+p_{n-2}~\forall n \ge 2$.\\
$q_0=1,p_1=a_1$, and $q_n=a_nq_{n-1}+q_{n-2}~\forall n \ge 2$.\\
\end{definition}
For example, $\alpha=\pi$, we have $p_0=3,q_0=1$, $p_1=22,q_1=7$, $p_2=333,q_2=106$, $p_3=355,q_3=113$.
And we might intuitively guess that these rational approximations come from the convergent.
\begin{lemma} For each $\beta>0$, and for each $n \ge 2$, we have that
$$[a_0,\ldots a_{n-1},\beta]=\frac{\beta p_{n-1}+p_{n-2}}{\beta q_{n-1}+q_{n-2}}$$
\end{lemma}
\begin{proof}[\bf Proof] We use induction on $n$. By definition of $p_i,q_i$, it is easily checked that
$$[a_0,a_1,\beta]=\frac{\beta p_1+p_0}{\beta q_1 +q_0}$$ and so it is true for $n=2$.
Suppose it is true for $n \ge 2$, that
$$[a_0,\ldots a_{n-1},\beta]=\frac{\beta p_{n-1}+p_{n-2}}{\beta q_{n-1}+q_{n-2}}$$
Then as
$$[a_0,\ldots a_n,\beta]=[a_0, \ldots a_{n-1},a_n+\frac{1}{\beta}]$$ by definition, and $\beta>0$.
So we use induction that
$$[a_0,\ldots a_n,\beta]=
\frac{(a_n+\frac{1}{\beta})p_{n-1}+p_{n-2}}{(a_n+\frac{1}{\beta})q_{n-1}+q_{n-2}}
=\frac{\beta p_n+p_{n-1}}{\beta q_n+q_{n-1}}$$
as $p_n=a_n p_{n-1}+p_{n-2}$ and $q_n=a_n q_{n-1}+q_{n-2}$.
\end{proof}
\begin{corollary} $\forall n \ge 0$, we have
$$[a_0,\ldots a_n]=\frac{p_n}{q_n}$$
\end{corollary}
\begin{proof}[\bf Proof] Check for the case when $n=0,1$, and then put $\beta=a_n$ in Lemma 5.5
\end{proof}
\begin{remark} Sometimes in some other book, reader might find that the definition of convergent is given by Corollary 5.6. This is an equivalent definition as we can deduce our definition by Corollary 5.6
\end{remark}
\begin{lemma} Corollary 5.6 is an equivalent definition of convergent.
\end{lemma}
\begin{proof}[\bf Proof] Let $\alpha=[a_0,\ldots a_n$ and $\frac{r_k}{s_k}=[a_k,\ldots a_n]$.
Thus we have $$\frac{r_k}{s_k}=a_k+\frac{1}{\frac{r_{k+1}}{s_{k+1}}}$$
Equate both sides and let $r_k=a_k r_k+s_{k+1},s_k=r_{k+1}$.
Then
$$\begin{pmatrix} r_k \\ s_k \end{pmatrix} = \begin{pmatrix} a_k & 1\\ 1 & 0 \end{pmatrix}
\begin{pmatrix} r_{k+1}\\ s_{k+1} \end{pmatrix}$$
and so
$$\begin{pmatrix} r_0 \\ s_0 \end{pmatrix} = \begin{pmatrix} a_0 & 1\\ 1 & 0 \end{pmatrix}
\begin{pmatrix} a_1 & 1\\ 1 & 0 \end{pmatrix} \cdots \begin{pmatrix} a_n \\ 1 \end{pmatrix}$$
Now by definition $p_n=r_0,q_n=s_0$, so that
$$\begin{pmatrix} p_n \\ q_n \end{pmatrix} = \begin{pmatrix} a_0 & 1\\ 1 & 0 \end{pmatrix}
\begin{pmatrix} a_1 & 1\\ 1 & 0 \end{pmatrix} \cdots
\begin{pmatrix} a_n & 1\\ 1&0 \end{pmatrix} \begin{pmatrix} 1 \\ 0 \end{pmatrix}$$

$$\begin{pmatrix} p_{n+1} \\ q_{n+1} \end{pmatrix} = \begin{pmatrix} a_0 & 1\\ 1 & 0 \end{pmatrix}
\begin{pmatrix} a_1 & 1\\ 1 & 0 \end{pmatrix} \cdots
\begin{pmatrix} a_n & 1 \\ 1 & 0 \end{pmatrix} \begin{pmatrix} a_{n+1} \\ 1 \end{pmatrix}$$
And so
$$\begin{pmatrix}p_{n+1} & p_n\\q_{n+1}&q_n \end{pmatrix}=
\begin{pmatrix} a_0&1\\1&0 \end{pmatrix} \cdots
\begin{pmatrix} a_{n+1} & 1\\1&0 \end{pmatrix}$$
which gives
$$\begin{pmatrix}p_{n+1} & p_n\\q_{n+1}&q_n \end{pmatrix}=
\begin{pmatrix}p_n & p_{n-1}\\q_n&q_{n-1} \end{pmatrix}
\begin{pmatrix}a_{n+1}&1\\1&0 \end{pmatrix}$$
Compare the first column, and the result follows.
\end{proof}
\begin{lemma} $\forall n \ge 1$, we have
$$p_n q_{n-1} - p_{n-1}q_n =(-1)^{n-1}$$ and in particular, $p_n,q_n$ are coprime.
\end{lemma}
\begin{proof}[\bf Proof] We use induction on $n$. The case $n=1$ follows immediately from the definition.
Suppose it is true for $n-1$, that
$$p_{n-1}q_{n-2}-q_{n-1}p_{n-2}=(-1)^{n-2}$$
then we have
\begin{eqnarray*}
p_n q_{n-1}-p_{n-1}q_n & = &q_{n-1}(a_n p_{n-1}+p_{n-2})-p_{n-1}(a_n q_{n-1}+q_{n-2})\\
&=&-(p_{n-1}q_{n-2}-p_{n-2}q_{n-1})\\
&=&(-1)^{n-1}
\end{eqnarray*}
Hence it holds for all $n \ge 1$.
\end{proof}
\begin{lemma} $\forall n \ge 2$, we have
$$p_n q_{n-2}-p_{n-2}q_n = (-1)^n a_n$$
\end{lemma}
\begin{proof}[\bf Proof] Expand $p_n$ and $q_n$ by definition so we have
\begin{eqnarray*}
p_n q_{n-2}-p_{n-2}q_n &= &q_{n-2}(a_n p_{n-1} +p_{n-2})-p_{n-2}(a_n q_{n-1}+q_{n-2})\\
&=&a_n (p_{n-1}q_{n-2}-p_{n-2}q_{n-1})\\
&=&a_n(-1)^{n}
\end{eqnarray*}
where we applied Lemma 5.9 and use $(-1)^{n-2}=(-1)^n$.
\end{proof}
\begin{lemma} The sequence $\frac{p_{2n}}{q_{2n}}$ is strictly increasing for $n \ge 0$ while the sequence $\frac{p_{2n-1}}{q_{2n-1}}$ is strictly decreasing.
\end{lemma}
\begin{proof}[\bf Proof] We have, by Lemma 5.10, that
$$\frac{p_{2n}}{q_{2n}}-\frac{p_{2n-2}}{q_{2n-2}}=(-1)^{2n} \frac{a_{2n}}{q_{2n}q_{2n-2}} >0$$
and similarly for the other one.
\end{proof}
\begin{theorem} Assume that $\alpha$ is irrational. Then $\forall n \ge 0$, we have
$$\left|\alpha-\frac{p_n}{q_n}\right|<\frac{1}{q_n q_{n+1}} < \frac{1}{q^2_n}$$
\end{theorem}
\begin{proof}[\bf Proof] Use Lemma 5.5, we have
$$\alpha=[a_0,\ldots, a_n,\alpha_{n+1}]=\frac{\alpha_{n+1}p_n+p_{n-1}}{\alpha_{n+1}q_n+q_{n-1}}$$
and so we have
\be
\alpha-\frac{p_n}{q_n} = \frac{-p_n q_{n-1}+q_n p_{n-1}}{q_n(\alpha_{n+1}q_n+q_{n-1})} = \frac{(-1)^n}{q_n(\alpha_{n+1}q_n +q_{n-1})}
\ee
and so
\be
\left|\alpha-\frac{p_n}{q_n}\right| =\frac{1}{q_n(\alpha_{n+1}q_n+q_{n-1})} < \frac{1}{q_n(a_{n+1}q_n+q_{n-1})} \le \frac{1}{q_n q_{n+1}} < \frac{1}{q^2_n}
\ee
where we used $a_{n+1} \le \alpha_{n+1}$ and $a_{n+1} \ge 1$.
\end{proof}
\begin{theorem} Assume $\alpha$ is irrational. Let $p,q$ be any two integers such that
$0<q<q_{n+1}$,  $(n \ge 0)$ then
$$\left|q\alpha-p\right| \ge \left|q_n \alpha -p_n\right|$$
\end{theorem}
\begin{proof}[\bf Proof] Since we know that $p_n q_{n+1}-p_{n+1}q_n = \pm 1$
so we can find integers $u,v$ such that
$$p=u p_n+vp_{n+1}$$
$$q=u q_n+vq_{n+1}$$
Since $0<q<q_{n+1}$, we have $u \neq 0$.

Suppose $v=0$, then $\frac{p}{q}=\frac{p_n}{q_n}$, but $(p_n,q_n)=1$ and so equality holds in the theorem. Hence we assume $v \neq 0$, as $0<q<q_{n+1}$, $u$ and $v$ must have opposite sign. By Lemma 5.21 and we know that $\frac{p_n}{q_n} \to \alpha$ by definition, so that
$\alpha -\frac{p_n}{q_n}$ and $\alpha-\frac{p_{n+1}}{q_{n+1}}$ must have opposite sign and so
combing these two, we have that $u(q_n \alpha-p_n)$ and $v(q_{n+1}\alpha-p_{n+1})$ have the same sign.
Therefore,
$$\left|q\alpha-p\right|=\left|u(q_n (\alpha-p_n)+v(q_{n+1}-p_{n+1})\right|$$
and so
$$\left|q\alpha-p\right| \ge |u|\left|q_n \alpha-p_n\right| \ge \left|q_n\alpha-p_n\right|$$
because $u \neq 0$.
\end{proof}
\begin{corollary} Assume $\alpha$ is irrational. Let $0<q$ such that
$$\left|\alpha-\frac{p}{q}\right|<\left|\alpha-\frac{p_n}{q_n}\right|$$
for some $n$, then $q>q_n$.
\end{corollary}
\begin{proof}[\bf Proof] Assume $q \le q_n < q_{n+1}$, then we have
\be
q\left|\alpha-\frac{p}{q}\right| = \left|q\alpha-p\right| \ge \left|q_n \alpha -p_n\right| = q_n \left|\alpha-\frac{p_n}{q_n}\right|
\ee
by using Theorem 5.13.
So we have
$$\left|\alpha-\frac{p}{q}\right| \ge \frac{q_n}{q}\left|\alpha-\frac{p_n}{q_n}\right| \ge \left|\alpha-\frac{p_n}{q_n}\right|$$
which is a contradiction.
\end{proof}
\begin{theorem} Assume $\alpha$ is irrational, then at least one of any two successive convergents satisfy
$$\left|\alpha-\frac{p}{q}\right|<\frac{1}{2q^2}$$
Moreover, if $\frac{p}{q}$ is arbitrary rational number with $q>0$ satisfying
$$\left|\alpha-\frac{p}{q}\right| <\frac{1}{2q^2}$$ then we must have
$\frac{p}{q}=\frac{p_n}{q_n}$ for some $n$. i.e, this says that the convergents are the best approximation of $\alpha$.
\end{theorem}
\begin{proof}[\bf Proof] Take any two $\frac{p_n}{q_n},\frac{p_{n+1}}{q_{n+1}}$. Then by Theorem 5.12 we have
$$0<\left|\alpha-\frac{p_n}{q_n}\right|<\frac{1}{q_n q_{n+1}}$$
and also $\alpha-\frac{p_{n+1}}{q_{n+1}}$, $\alpha-\frac{p_n}{q_n}$ have opposite sign.
So we have
$$\left|\alpha-\frac{p_n}{q_n}\right|+\left|\alpha-\frac{p_{n+1}}{q_{n+1}}\right|
=\left|\frac{p_n}{q_n}-\frac{p_{n+1}}{q_{n+1}}\right| = \frac{1}{q_n q_{n+1}}$$
Apply AM-GM inequality, this is $< \frac{1}{2{q_n}^2}+\frac{1}{2{q_{n+1}}^2}$
And so one of them must satisfy
$$\left|\alpha-\frac{p}{q}\right| < \frac{1}{2q^2}$$
Moreover, assume now $|\alpha-\frac{p}{q}| < \frac{1}{2q^2}$. We choose $n$ such that
$q_n \le q < q_{n+1}$ (Because $q_n$ is unbounded). Apply triangle inequality we have,
$$\left|\frac{pq_n-qp_n}{qq_n}\right|
=\left|\frac{p}{q}-\frac{p_n}{q_n}\right| \le \left|\alpha - \frac{p}{q}\right|
+\left|\alpha-\frac{p_n}{q_n}\right| =\frac{1}{q}\left|q\alpha -p\right|+
\frac{1}{q_n}\left|q_n \alpha -p_n\right|$$
As $q<q_{n+1}$, by Theorem 5.13, we have $|q\alpha-p| \ge |q_n \alpha-p_n|$. Hence
$$\left|\frac{p}{q}-\frac{p_n}{q_n}\right| \le \left(\frac{1}{q}+\frac{1}{q_n}\right)\left|q\alpha-p\right|
< \frac{1}{2q^2}+\frac{1}{2qq_n}$$
by assumption.

Since $q \ge q_n$, so
$$\left|\frac{p}{q}-\frac{p_n}{q_n}\right| < \frac{1}{2qq_n}+\frac{1}{2qq_n}
=\frac{1}{qq_n}$$
Hence we have
$$\left|\frac{pq_n-qp_n}{qq_n}\right| < \frac{1}{qq_n}$$
and so
$$\left|pq_n-qp_n\right|<1$$ But this is an integer and so it must be $0$.
Therefore, we have
$$\frac{p}{q}=\frac{p_n}{q_n}$$
\end{proof}
The theorem above also says that there are infinitely many convergents which satisfy
$$\left|\alpha-\frac{p_n}{q_n}\right| \le \frac{1}{2q^2_n}$$
But the fraction $\frac{1}{2}$ is not best possible and we can in fact improve this by:
\begin{theorem}{\bf [Hurwitz]}\label{H;Hurwitz} There are infinitely many convergents of an irrational
$\alpha$ which satisfies
$$\left|\alpha-\frac{p_n}{q_n}\right| < \frac{1}{b^2_n \sqrt{5}}$$
\end{theorem}
The proof is similar to Theorem 5.15 and is left an exercise to the reader (see Exercise 4 of this chapter). But the fraction $\sqrt{5}$ in the theorem can not be improved as we have the following theorem:
\begin{theorem} There exists a real number $\gamma$ with the property: if $\theta > \sqrt{5}$ then the inequality
$$\left|\gamma-\frac{p}{q}\right| <\frac{1}{q^2 \theta}$$
has only finitely many rational solutions $\frac{p}{q}$.
\end{theorem}
\begin{proof}[\bf Proof] Set $\gamma=\frac{-1+\sqrt{5}}{2}$ and suppose the solutions to the above inequality exist with arbitrarily large $q$. We may rewrite the above as:
$$\frac{-1+\sqrt{5}}{2} = \frac{p}{q}+\frac{\delta}{p^2}~\text{ where }
|\delta|<\frac{1}{\theta}<\frac{1}{\sqrt{5}}$$
That is
$$-\frac{\delta}{q}+\frac{\sqrt{5}}{2}q = \frac{q}{2}+p$$
Squaring both sides we have
$$\frac{\delta^2}{q^2} - \delta \sqrt{5} = p^2 +pq -q^2$$
Now, as $-1 < -\frac{\delta}{5} <1$, we have for sufficiently large $q$,
$$-1 < \frac{\delta^2}{q^2} - \frac{\delta}{\sqrt{5}}<1$$
and hence $p^2+pq-q^2=0$ as it is an integer between $-1$ and $1$.
Multiply both sides by $4$ and we have $(2p+q)^2 = 5q^2$ which is impossible as LHS is a perfect square,
but RHS is not. Therefore, we can only have finitely many such $q$ satisfying the inequality, and for each $q$, it is clear that we only have finitely many such $p$.
\end{proof}
The continued fraction of $\frac{-1+\sqrt{5}}{2}$ is $[0,1,1,\ldots]$. In fact, any real number whose continued fraction is {\bf eventually} all ones can be used for $\gamma$ in Theorem 5.17. Further, it can be shown that, for all numbers not of this type, we may replace $\sqrt{5}$ in Theorem 5.16 by $\sqrt{8}$ and this is the best possible when $\gamma=\sqrt{2}$ or is equal to any number whose continued fraction is eventually all twos. An analogue to Theorem 5.17 is the following: it can be shown that there are uncountably many real numbers $\beta$ such that the inequality
$$\left|\beta-\frac{p}{q}\right| < \frac{1}{3b^2}$$
has infinitely many rational solutions $\frac{p}{q}$, but only finitely many solutions if the number
$3$ is replaced by any large number.
\subsection{Pell's equation}
The Diophantine equation $x^2-dy^2=1$, usually known as Pell's equation, is of considerable importance in number theory. The main applications are in quadratic form theory and in the unit problem. The case when $d$ is a square only have trivial solution and so throughout the subsection we shall assume that $d>0$
and is not a square.
\begin{theorem} Any solution to
$$x^2-dy^2=1$$
for $x,y>0$ must have $x=p_n,y=q_n$ for some $n$, where $\frac{p_n}{q_n}$ is convergent of $\sqrt{d}$.
\end{theorem}
\begin{proof}[\bf Proof] As $d$ is square free, so $\sqrt{d}$ is irrational, and $d>1$.
Let $x=p,y=q$ be a solution to $x^2-dy^2=1$. We have
$$p^2-dq^2=(p-\sqrt{d}q)(p+\sqrt{d}q)=1$$
and as $p+\sqrt{d}q>0$, we must have $p-\sqrt{d}q>0$. So $p>q\sqrt{d}$.
Then
$$0<p-q\sqrt{d}=\frac{1}{p+q\sqrt{d}}<\frac{1}{2q\sqrt{d}}$$
and so
$$\left|\frac{p}{q}-\sqrt{d}\right|<\frac{1}{2q^2 \sqrt{d}} <\frac{1}{2q^2}$$
and so $\frac{p}{q}=\frac{p_n}{q_n}$ for some $n$ by Theorem 5.15.
\end{proof}
The converse of the theorem above is the following, which proves the existence of the solution:
\begin{theorem} The Pell's equation $x^2-dy^2=1$ is soluble in integers $x$ and $y$.
\end{theorem}
Before we prove the theorem, we shall introduce a very important concept
\begin{definition} We say $\alpha$ has a period continued fraction if there exists $\omega>0$ such that
$a_{n+\omega}=a_n$ for all $n \ge b_0$ for some $n_0$.
\end{definition}
For example, $\sqrt{43}=[6,\underbrace{1,1,3,1,5,,3,1,1,1,2}_{\omega=10,n_0=1},1,1\ldots]$
\begin{definition} We say an irrational number $\alpha$ is a quadratic irrationality if there exists integers $a,b,c$, not all $0$ with $a\alpha^2+b\alpha+c=0$
\end{definition}
\begin{theorem}{\bf [Lagrange]}\label{L;Lagrange continued fraction} Assume that
$\alpha$ is irrational. Then $\alpha$ is a quadratic irrationality if and only if $\alpha$ has a periodic continued fraction.
\end{theorem}
\begin{proof}[\bf Proof] We introduce the notation $[a_0,\ldots a_{n-1},a_n^*]$
for $[a_0,\ldots a_{n-1},a_n,a_n\ldots]$
Suppose that $\alpha$ has a periodic continued fraction.
$$\alpha=[a_0,\ldots a_{n-1},a_n^*,\ldots a_{n+m-1}^*]$$
Then we have, by Lemma 5.5
$$\alpha=[a_0,\ldots a_{n-1},\alpha_n]=\frac{\alpha_n p_{n-1}+p_{n-2}}{\alpha_n q_{n-1}+q{n-2}}$$
and as $[a_n,\ldots a_{n+m-1}]$ is repeated
$$\alpha_n=[a_n,\ldots a_{n+m-1},\alpha_n]=
\frac{\alpha_n p'_{m-1}+p'_{m-2}}{\alpha_n q'_{m-1}+q'_{m-2}}$$
where $\frac{p'_m}{q'_m}$ is the $m^{th}$ convergent of $\alpha_n$.
From the second equality we have a quadratic equation for $\alpha_n$ as $p'_m,q'_m$ are integers.
Then from the first equality we can write $\alpha_n$ in the form $\frac{a\alpha+b}{c\alpha+d}$.
Substitute this into the quadratic equation of $\alpha_n$ and clear the denominators, and we have a quadratic equation for $\alpha$ after rearrange the equation.

Conversely, suppose $\alpha$ is positive and $\alpha=\alpha_0=\frac{c_0+\sqrt{d}}{e_0}$, where
$c_0,d_0$ and $e_0$ are integers and $d_0$ is not a square, $e_0 \neq 0$. We may rewrite $\alpha_0$ as
$\frac{c_0e_0+\sqrt{d{e_0}^2}}{{e_0}^2}$ if necessary so that we have $e_0|d-{c_0}^2$.
By definition of continued fraction of $\alpha$, we can define $c_i$ and $e_i$ by:
$$c_{i+1}=a_i e_i- c_i,~e_{i+1}=\frac{d-{c_{i+1}}^2}{e_i}$$
It is a simple matter to check by induction that $c_i$ and $e_i$ are integers and that
$e_i \neq 0$ and $e_i|d-{c_i}^2$, using the equations
$$e_{i+1}=\frac{d-{c_{i+1}}^2}{e_i}=\frac{d-{c_i}^2}{e_i}+2a_ic_i-{a_i}^2e_i$$
And so we conclude that
$$\alpha_i=\frac{c_i+\sqrt{d}}{e_i}$$
by induction using the equation $\alpha_i-a_i=\frac{1}{\alpha_{i+1}}$ and above.

Further, let $\alpha'_i=\frac{c_i-\sqrt{d}}{e_i}$. By Lemma 5.5, and rearrange the equation, we have
$$\alpha_i=-\frac{\alpha_0 q_{i-2}-p_{i-2}}{\alpha_0 q_{i-1}-p_{i-1}}$$
and $\alpha'_i$ is the conjugate of $\alpha_i$, so taking conjugate, we have
$$\alpha'_i=-\frac{\alpha'_0 q_{i-2}-p_{i-2}}{\alpha'_0 q_{i-1} - p_{i-1}}$$
which we may rewrite as
$$\alpha'_i=-\frac{q_{i-2}}{q_{i-1}}\left(\frac{\alpha'_0 -\frac{p_{i-2}}{q_{i-2}}}
{\alpha'_0 -\frac{p_{i-1}}{q_{i-1}}}\right)$$
As the convergent tends to $\alpha$ as $i$ tends to infinity, and so the term in parentheses tends to $1$, and so there is an $n_0$ such that $\alpha'_n <0$ if $n > n_0$. Hence
$\alpha_n-\alpha'_n=\frac{2\sqrt{d}}{e_n}>0$, and thus $e_n >0$ if $n > n_0$. Also
as $e_n$ is integer, so that
$$e_n \le e_n e_{m+1}=d-c^2_{n+1}< d$$
and
$$c^2_{n+1} < c^2_{n+1} + e_n e_{n+1}=d$$
Hence if $n>n_0$ there can be only finitely many distinct pairs $\{c_n,e_n\}$ as they are integers bounded by $|d|$, and so there is a $k>0$, such that if $n>n_0$, $a_{n+k}=a_n$ because each pair $\{c_n,e_n\}$ determines a unique $\alpha_i$, and each $\alpha_i$ determines $a_i$. Then use
$\alpha_i-a_i=\frac{1}{\alpha_{i+1}}$, we have $a_{n+t}=a_{n+k+t}$ where $t \ge 0$, and the result follows.
\end{proof}
Now we can prove Theorem 5.19.
\begin{proof}[\bf Proof] Let $\alpha=\sqrt{d}$, by Theorem 5.22 we have $\alpha_n=\frac{c_n+\sqrt{d}}{e_n}$ and so
by Lemma 5.5 we have
$$\sqrt{d}=\frac{\alpha_n p_{n-1}+p_{n-2}}{\alpha_n q_{n-1}+q_{n-2}}
=\frac{(c_n+\sqrt{d})p_{n-1}+e_np_{n-2}}{(c_n+\sqrt{d})q_{n-1}+e_nq_{n-2}}$$
Equating irrational and rational parts, we have
$$q_{n-1}c_n+q{n-2}e_n=p_{n-1} \text{ and } p_{n-1}c_n+p_{n-2}e_n=dq_{n-1}$$
and multiplying by $p_{n-1}$ and $q_{n-1}$ respectively, and subtract, we have
$$p^2_{n-1}-dq^2_{n-1}=(p_{n-1}q_{n-2}-p_{n-2}q_{n-1}) e_n=(-1)^n e_n$$
by Lemma 5.9.

Now let $k$ be the period of the continued fraction of $\sqrt{d}$, and $t$ be any positive integer.
Then by Theorem 5.22 we have
$$\frac{c_n+\sqrt{d}}{e_n}=\frac{c_{n+kt}+\sqrt{d}}{e_{n+kt}}$$. As in the proof of Theorem 5.22,
we have $e_n \le d$ and so will repeat. Then there is a non-zero integer $e$ such that
$$u^2-dv^2=e$$ has infinitely many solutions $\{u,v\}$.
(This is from the equation $p^2_{n-1}-dq^2_{n-1}=(-1)^n e_n$ above as $p_n,q_n$ are monotonically increasing.
We may assume that $u$ and $v$ are positive and we partition this set of solutions into classes by:
$\{u,v\}$ is in the same set as $\{u',v'\}$ if and only if
$u \equiv u'$ and $v \equiv v'$ (mod $|e|$). At least one of these classes has two distinct
(in fact infinitely many as the pair $(u,v)$ (mod $|e|$) is finite) members $\{u_1,v_1\}$ and $\{u_2,v_2\}$. Now let $x$ and $y$ be given by:
$$x=\frac{u_1 u_2 -d v_1 v_2}{e}, y=\frac{u_1v_2-u_2 v_1}{e}$$
We claim that $x$ and $y$ are integers and they are solutions to the original Pell's equation.
Indeed, we have
$$u_1 u_2-dv_1v_2 \equiv u^2_1 -d^2_1 \equiv 0~(\text{mod } |e|)$$
$$0 \neq u_1v_2-u_2v_1 \equiv 0~(\text{mod } |e|)$$
and
$$e^2(x^2-dy^2)=(u^2_1-dv^2_1)(u^2_2-dv^2_2)=e^2$$
and so we have
$$x^2-dy^2=1$$
\end{proof}
\begin{theorem} Suppose $\{x_0,y_0\}$ is the solution of Pell's equation with $x_0,y_0$ positive and
$D=x_0+y_0\sqrt{d}$ minimal, then $\{x,y\}$ is a solution of the Pell's equation if and only if
$x$ and $y$ satisfy
$$x+y\sqrt{d}=\pm (x_0+y_0\sqrt{d})^n$$
for some integer $n$ which is positive, negative, or zero. We call such solution $\{x_0,y_0$\} the fundamental solution.
\end{theorem}
\begin{proof}[\bf Proof] Suppose that $\{x_1,y_1\}$ and $\{x_2,y_2\}$ are solutions of the Pell's equation and $x_3,y_3$ are given by
$$x_3+y_3\sqrt{d}=(x_1+y\sqrt{d})(x_2+y\sqrt{d})$$
then $\{x_3,y_3\}$ is a solution of Pell's equation because
$$x^2_3-dy^2_3=(x_1x_2+dy_1y_2)^2-d(x_1y_2+x_2y_1)^2=(x^2_1-dy^2_1)(x^2_2-dy^2_2)=1$$
Then, note that if $n=0$ we have $x=1, y=0$, the trivial solution.\\
Also
$$(x_1+y_1 \sqrt{d})^{-1} =\frac{x_1-y_1\sqrt{d}}{x^2_1-dy^2_1}=x_1+(-y_1)\sqrt{d}$$
It follows by considering conjugate that if $\{x_1,y_1\}$ is a solution and $x,y$ are given by
$$x+y\sqrt{d}=\pm(x_1+y_1 \sqrt{d})^n$$ then $\{x,y\}$ is also a solution for every integer $n$.

Now suppose $\{t,u\}$ is a solution with $t,u>0$. Choose $m$ to satisfy
$$D^m\le t+u\sqrt{d} < D^{m+1}$$ that is
$$1 \le (t+u\sqrt{d})D^{-m} < D$$
By the argument above, if $(t+u\sqrt{d})D^{-m}= X+Y\sqrt{d}$, then $\{X,Y\}$ is a solution with
$X,Y$ non-negative, because we have $1 \le X+Y\sqrt{d} <D$ and
$0< X- Y\sqrt{d} \le 1$ (as $X^2-dY^2=1)$
and so $2X \ge 1$ and $2Y\sqrt{d} \ge 0$. It follows by minimality of
$D$ that $X=1$ and $Y=0$. Hence $t+u\sqrt{d}=D^m$ and the proof is complete.
\end{proof}
Let's briefly discuss the relation between continued fraction and Pell's equation. If $k$ is the period of continued fraction of $\sqrt{d}$ and $k$ is even, then $\{a_{k-1},b_{k-1}\}$ is a solution of Pell's equation (See in the proof of Theorem 5.19, we had $p^2_{n-1}-dq^2_{n-1}=(-1)^n e_n$); and using a similar argument, it follows that equation $x^2-dy^2=-1$ is insoluble. If $k$ is odd the situation is slightly different: $\{p_{k-1},q_{k-1}\}$ is a solution to the equation $x^2-dy^2=-1$, and
$\{p_{2k-1},q_{2k-1}\}$ is a solution of $x^2-dy^2=1$. We shall left these as an exercise (see Exercise $10$ of this chapter).

As an example, consider $d=13$ and $d=14$. In the first case we have $\sqrt{13}=[3,1^*,1,1,1,6^*]$ with period $5$.
$[3,1,1,1,1]=\frac{18}{5}$ and we can check that $x=18,y=5$ is a solution of
$x^2-13y^2=-1$. And as $649+180\sqrt{3}=(18+5\sqrt{13})^2$, $\{649,180\}$ is a solution
of Pell's equation. Now for the second case $\sqrt{14}=[3,1^*,2,1,6^*]$ with period $4$, and
$[3,1,2,1]=\frac{15}{4}$. Hence $\{15,4\}$ is a solution to the Pell's equation. The equation $x^2-14^2=-1$ is not soluble by considering the equation modulo
$7$ ($-1$ is not a square mod $7$).
\begin{theorem} If the equation $$u^2-dv^2=m$$ has a solution $\{u,v\}$, then it has
infinitely many solutions, at least one of which satisfies
$$0<u<\sqrt{\frac{(x_0+1)|m|}{2}}$$
where $\{x_0,y_0\}$ is the fundamental solution of Pell's equation in Theorem 5.23.
\end{theorem}
\begin{proof}[\bf Proof] Consider the case when $m>0$; a virtually identical argument can be used if $m<0$.
Assume that the equation is soluble and all solutions $\{u,v\}$ are such that
$$u \ge \sqrt{\frac{(x_0+1)m}{2}}$$
Choose $u$ and $v$ so that $\{u,v\}$ is a solution with $u,v>0$, and $u$ is minimal. Define
$u_1$ and $v_1$ by
$$u_1+v_1 \sqrt{d}=(u+v\sqrt{d})(x_0-y_0\sqrt{d})$$
As above $\{u_1,v_1\}$ is also a solution by considering the conjugate, and use
$$x^2_0-dy^2_0=1 \text{ and } 1-\frac{m}{u^2}=d\frac{v^2}{u^2}$$
Now $u_1=ux_0-dvy_0 >u$ so divide both sides by $u$, we have
$$1<x_0-\frac{dvy_0}{u}$$
We write $\frac{dvy_0}{y}$ as $(\sqrt{d}y_0) \frac{\sqrt{d} v}{u}$
and use:
$$\sqrt{d}y_0=\sqrt{x^2_0-1}$$
and
$$\frac{\sqrt{d}v}{u}=\sqrt{1-\frac{m^2}{u^2}}$$
so we have
$$1<x_0-\sqrt{x^2_0-1}\sqrt{1-\frac{m^2}{u^2}}$$
Rearrange and square both sides, we have
$$1-\frac{m^2}{u^2}<\frac{x_0-1}{x_0+1}=1-\frac{2}{x_0+1}$$ by noticing that $x_0 \ge 1$.
Rearrange again, we have
$$(x_0+1)m^2>2u^2$$ and hence the result follows.
\end{proof}


\subsection{A set of real numbers modulo 1}
The last topic of this section is Chebysheve's theorem, which concerns the distribution of
$(n\alpha)$ for $n \ge 1$ and $\alpha$ any fixed irrational number, where $(a)$ denotes the fractional part of $a$.
We shall prove that the set $\{(n\alpha): n=1,2, \ldots\}$ is dense in $[0,1]$ and is uniformly distributed.
\begin{theorem} If $\alpha$ is an irrational number $\beta$ is a real number, then there are infinitely many pairs of integers $x$ and $y$ such that
$$\left|x\alpha-y-\beta\right|<\frac{3}{x}$$
\end{theorem}
\begin{proof}[\bf Proof] By Theorem 5.16 (Hurwitz), there are infinitely many integers $a$ and $b$ satisfying
$(a,b)=1$ and
$$\alpha=\frac{a}{b}+\frac{\gamma}{b}$$
for some $\gamma$ with $|\gamma|<1$ (In Theorem 5.16 we took $\gamma$ to be $\frac{1}{\sqrt{5}}$).
Also let $c$ be the integer closet to $b\beta$, so
$$\beta=\frac{c}{b}+\frac{\delta}{2b}$$
where $|\delta| \le 1$. Further, as $(a,b)=1$, integers $x$ and $y$ can be found to satisfy
$ax-by=c$ and $b \le 2x < 3b$.
Using these, we have
\begin{eqnarray*}
\left|x\alpha-y-\beta\right|&=
&\left|\frac{xa}{b}+\frac{x\gamma}{b^2}-y-\frac{c}{b}-\frac{\delta}{2b}\right| = \left|\frac{x\gamma}{b^2}-\frac{\delta}{2b}\right|\\
&<&\frac{x}{b^2}+\frac{1}{2b} < \frac{9}{4x}+\frac{3}{4x}=\frac{3}{x}
\end{eqnarray*}
as $b>\frac{2x}{3}$. The result follows because $x \ge \frac{b}{2}$ and $b$ can be arbitrarily large.
\end{proof}
\begin{definition} Let $R=\{\alpha_0,\alpha_1,\ldots \}$ be a set of real numbers
in the unit interval and let
$0 \le \beta_1 < \beta_2 \le 1$. Define $T$ by:
$T(n,\beta_1,\beta_2)$ equals the number of elements $\alpha_i$ of $R$ satisfying
$i \le n$ and $\beta_1 \le \alpha_i \le \beta_2$. $R$ is said to be {\bf uniformly distributed} in the unit interval if and only if
$$\lim_{n \to \infty} \frac{T(n,\beta_1,\beta_2)}{n} =\beta_2 -\beta_1$$
holds for all $\beta_1$ and $\beta_2$.
\end{definition}
\begin{theorem} If $\alpha$ is irrational, then the set
$$S=\{(n\alpha):n=1,2,\ldots\}$$ is uniformly distributed in the unit interval.
\end{theorem}
\begin{proof}[\bf Proof] By Theorem 5.16 we have infinitely many $a$ and $b$ with $b>0$ such that
$$\alpha=\frac{a}{b}+\frac{\delta}{b^2}, |\delta|<1,(a,b)=1$$
For $b$ to be chosen, let $J_j$ be the set $\{jb+k:k=0,1,\ldots b-1\}$ and
consider the set $K_j=\{(i\alpha): i \in J_j\}$. We have
$$\alpha(jb+k)=\left(\frac{a}{b}+\frac{\delta}{b^2}\right)(jb+k)=ja+\frac{ka+j\delta}{b}
+\frac{k\delta}{b^2}$$
and so as $k<b$
$$(\alpha(jb+k))=\left(\frac{ka+[j\delta]}{b}+\frac{\gamma}{b}\right)$$
where $|\gamma|<2$ (We can write $\gamma$ explicitly as
$\frac{\frac{k\delta}{b}+(j\delta)}{b}$).
$[j\delta]$ is independent of $k$ and so, as $k$ ranges over a complete system of residues modulo $b$, $ka+[jb]$ ranges over this system. It follows that
$$K_j=\left\{\left(\frac{k+\gamma}{b}\right): k=0,1,\ldots b-1\right\}$$

Let $[\beta_1,\beta_2]$ be a subinterval of the unit interval and let $b$ be sufficiently large so that
we can choose integers $v$ and $w$ to satisfy
$$v-1<b\beta_1\le v <w \le b\beta_2 <w+1$$
Let the number of elements of $K_j$ in the interval $[\beta_1,\beta_2]$ be denoted by
$W_j$. The $W_j$ is bounded below by the number of elements of $K_j$ in the interval
$(\frac{v}{b},\frac{w}{b})$; that is, use the above conclusion for $K_j$,
 $W_j$ is bounded below by the number of integers $k$ which satisfy
$$v \le k+\gamma \le w$$
It follows that $W_j \ge w-v-4$ as $|\gamma| <2$. \\
Similarly by considering
$$v-1 \le k+\gamma \le w+1$$
$W_j \le w-v+6$; note that
these bounds are independent of $j$.

Let $n=rb+s$ where $0 \le s <b$ and let $T(n,\beta_1,\beta_2)$ be the number of
reals $(x\alpha)$ in the interval $[\beta_1,\beta_2]$ for $x \le n$.
Then by definition of $J_j$, we have
$$\sum_{j=0}^{r-1}W_j \le T(n,\beta_1,\beta_2) \le \sum_{j=0}^{r}W_j$$
and use the inequalities above,we have
$$r(w-v-4) \le T(n,\beta_1,\beta_2) \le (r+1)(w-v+6)$$
But provided $r>1$, use $w-v+2 \ge b(\beta_2-\beta_1)$ and $s<b$
$$r(w-v-4)=\frac{n-s}{b}(w-v-4)\ge n(\beta_2-\beta_1)-\frac{6n}{b}-(w-v-4)$$
and by $w-v \le b(\beta_2-\beta_1)$ and $s \ge 0$,
$$(r+1)(w-v+6)\le \left(\frac{n}{b}+1\right)(w-v+6)\le n(\beta_2-\beta_1)+\frac{6n}{b}+(w-v+6)$$
Given $\epsilon>0$ choose $b$ such that $b>\frac{1}{\epsilon}$, and $n$ such that
$\frac{b(\beta_2-\beta_1)}{n}<\epsilon$ (We choose $b$ according to this $\epsilon$, and the previous paragraph is based
on any $v,w$ satisfying the condition).
This gives that $\frac{w-v}{n} <\epsilon$ and
$\frac{1}{n}\le \frac{w-v}{n}<\epsilon$ and so $1<n\epsilon$.

Combining the inequalities above we have
$$n(\beta_2-\beta_1)-7n\epsilon \le T(n,\beta_1,\beta_2) \le n(\beta_2-\beta_1)+13n\epsilon$$
(Note in both inequalities, we use $\frac{6n}{b}<6n\epsilon$ as $b>\frac{1}{\epsilon}$)
Hence
$$\lim_{n \to \infty}\frac{1}{n} T(n,\beta_1,\beta_2)=\beta_2-\beta_1$$
\end{proof}
\begin{remark} The last inequality in the proof might not be the best bounds but is sufficient for the proof.
\end{remark}

\subsection{Exercises}
\begin{enumerate}
\item Let $N$ and $M$ be positive integers such that $N$ is not a square and $M \le \sqrt{N}$.
      If $x,y$ are positive integers satisfyig
      $$x^2-Ny^2=M$$
      Prove that $\frac{x}{y}$ is a convergent of $\sqrt{N}$.
\item Let $\alpha=[a_0,a_1,\ldots]$ and $\frac{p_n}{q_n}$ be convergent. Show that
      $$p_n=det \begin{pmatrix} a_0 &-1& 0& \ldots&0&0\\ 1&a_1& -1 &\ldots & 0 &0\\.&.&.&.&.&.\\
      0&0&0&\ldots&1&a_n \end{pmatrix}$$
      Find a similar expression for $q_n$.\\
      Show also that $\frac{p_{n+1}}{p_n}=[a_{n+1},\ldots,a_0]$ and
      $\frac{b_{n+1}}{b_n}=[a_{n+1},\ldots,a_1]$
\item Let $u_n$ be the $n^{th}$ term in the Fibonacci sequence. i.e.
     $$u_1=1,u_2=1 \text{ and }u_{n+2}=u_{n+1}+u_n~\forall n \ge 0$$ Let $\alpha=\frac{1+\sqrt{5}}{2}$.
     Show that:
     \begin{enumerate}
     \item[(i)]$\frac{u_{n+2}}{u_{n+1}}$ is the $n^{th}$ convergent of $\alpha$.
     \item[(ii)] If $p$ is a prime, then $p|u_{p-1}$ if $p \equiv \pm 1$ (mod $5$), and
     $p|u_{p+1}$ if $p \equiv \pm 2$ (mod $5$), and hence show that every prime divides
     infinitely many Fibonacci numbers.\\
     Hint: you may consider $2^{p-1}u_p$ mod $p$ and use the fact that
     $$u_n=\frac{\alpha^n-\alpha'^n}{\sqrt{5}}$$
     \end{enumerate}
\item Prove Theorem 5.16. (Hurwitz)\\
     (Hint: Consider any three consecutive convergents. If we write $\beta_n=\frac{q_{n-2}}{q_{n-1}}$, try to prove that
     $$\beta_n+\frac{1}{\beta_n} < \sqrt{5}$$
     and hence get a contradiction of the fact $a_n \ge 1$)
\item Suppose $\alpha$ is a real number in the interval $(0,1)$ and $\alpha=[0,a_1,a_2,\ldots]$
     Evaluate the probability that $a_1 \ge m$ and $a_1 = m$
     Further, show that the probability that $a_2 \ge m$ is
     $$\sum_{a_1=1}^{\infty} \frac{1}{a_1(ma_1+1)}$$
     Hence, find the probability that $a_2=1$.
\item Assume that $\alpha$ is a quadratic irrational number which satisfies $\alpha>1$, and
     $-1<\alpha'<0$ where $\alpha'$ is the conjugate of $\alpha$; that is, $\alpha'=\frac{a-b\sqrt{d}}{c}$, if $\alpha=\frac{a+b\sqrt{d}}{c}$. Show that
     \begin{enumerate}
     \item[(i)] $-1<\alpha'_{n}<0$, where $\alpha_n$ satisfies
     $\alpha=[a_0,a_1,\ldots a_{n-1},\alpha_n]$.
     \item[(ii)] $a_n=[\frac{-1}{\alpha'_{n+1}}]$.
     \item[(iii)] $\alpha$ is purely periodic (that is, the first period of the continued fraction of $\alpha$ begins with $a_0$).
     \end{enumerate}
\item Assume that $\alpha>1$ is irrational and is purely periodic defined in the above question.
     Let $\frac{p_n}{q_n}$ be convergent.
     Show that $\alpha$ satisfies the polynomial equation $f(\alpha)=0$ where
     $$f(x)=x^2q_{n-1}+x(q_{n-2}-p_{n-1})-p_{n-2}$$
     for suitably chosen $n$. Hence show that the second root $\alpha'$ of this equation
     satisfies $-1<\alpha'<0$.
     Combine question $6$ and $7$, it follows that the continued fraction of a quadratic irrational
     number $\alpha$ is purely periodic if and only if $\alpha>1$ and $-1<\alpha'<0$.
\item By writing $x^3-dy^3$ as a product of two factors, with one of which being
     $\frac{x}{y}-d^{\frac{1}{3}}$, show that if $\{x_0,y_0\}$ is a solution of the equation
     $x^3-dy^3=n$ then $\frac{x_0}{y_0}$ is a convergent of $d^{\frac{1}{3}}$ provided that
     $y_0 > \frac{8|n|}{3d^{\frac{2}{3}}}$ and $d$ is positive and not a cube.
\item Suppose $d$ is positive and is not square. Show that the equation
     $$v^2-dw^2=4$$
     is always soluble and there is a fundamental solution $\{v_0,w_0\}$ from which all other solutions are generated by $(n \in \mathbb{Z})$
     $$\frac{v+w\sqrt{d}}{2}=\pm \left(\frac{v_0+w_0\sqrt{d}}{2}\right)^n$$
\item[$^\star$ 10.] Let $k$ be the period of continued fraction of $\sqrt{d}$ where $d>0$ and is not square. Suppose $\alpha=[\sqrt{d}]+\sqrt{d}$, $c_0=[\sqrt{d}]$, $e_0=1$ and $c_i$ and $e_i$, $i=1,2,\ldots$ are as defined in Theorem 5.22 (on page 55). Show that:
    \begin{enumerate}
    \item[(i)] $e_i=1$ if and only if $k|i$.
    \item[(ii)] $e_j \neq -1$ for all $j$.
    \item[(iii)] $a_i < \sqrt{d}$ for $i=1,2,\ldots,k-1$.
    \end{enumerate}
    Further, Show that the continued fraction of $\sqrt{d}$ has the form
    $$[a_0,a_1^*,\ldots,a_{k-1},2a_0^*] \text{ where } a_0=[\sqrt{d}] \text{ and } a_i=a_{k-i}$$
    for $i=1,2,\ldots k-1$. (You may use question $2$ and $6$ in this question).
\item[11.] Show that there are infinitely many integers $n$ such that the sum
    $$1+2+\ldots +n$$ is a square.
\end{enumerate}

\section{Distribution of prime numbers and Riemannn zeta function}
The study of the primes is one of the most important branches of number theory. The literature on the prime numbers is very extensive beginning withy Pythagoras and Euclid; even so, many problems remain.
There are two pre-eminent results: Dirichlet's theorem and the prime number theorem(PNT). Dirichlet's theorem states that every arithmetic progression $\{an+b: n=0,1,2,\ldots\}$, where $(a,b)=1$ contains infinitely many primes. The PNT was postulated by both Legendre and Gauss and it was finally published by Hadamard, which states that
$$\pi(x) \sim \frac{x}{\log{x}}$$
A more precise version of this result is
$$\pi(x) \sim li(x)=\int_2^{x}\frac{dt}{\log{t}}$$
\subsection{Basic properties}\
\begin{theorem} Let $P$ be the set of all prime numbers. Then
$$\sum_{p \in P} \frac{1}{p} \text{ and } \prod_{p \in P}\left(1-\frac{1}{p}\right)^{-1}$$
both diverge to $\infty$.
\end{theorem}
\begin{proof}[\bf Proof] Let $x \ge 2$. We write
$$S(x)=\sum_{p \in P,p \le x }\frac{1}{p} \text{ and }
P(x)=\prod_{p \in P,p \le x} \left(1-\frac{1}{p}\right)^{-1}$$
Let $p_1,p_2,\ldots p_r$ be all prime numbers less than or equal to $x$. Then
\begin{eqnarray*}
P(x)&=&\prod_{i=1}^{r}\left(1-\frac{1}{p_i}\right)^{-1} = \prod_{i=1}^{r}\left(1+p^{-1}_i+p^{-2}_i+\ldots \right)\\
&=&\sum_{a_1=0}^{\infty} \cdots \sum_{a_r=0}^{\infty}p^{-a_1}_1 \cdots p^{-a_r}_r >  \sum_{n=1}^{\infty}\frac{1}{n} \to{\infty} \text{ as } x \to{\infty}
\end{eqnarray*}
As each integer $n$ can be written as a product of prime factors.

\be
\log{P(x)} = -\sum_{i=1}^{r}\log{\left(1-\frac{1}{p_i}\right)} = \sum_{i=1}^{r} \sum_{n=1}^{\infty} \frac{1}{np^n_i} = S(x)+\sum_{i=1}^{r}\sum_{n=2}^{\infty}\left(np^n_i\right)^{-1}
\ee
For each $i$, we have
$$\sum_{n=2}^{\infty}\left(np^n_i\right)^{-1} < \sum_{n=2}^{\infty}\frac{1}{p^n_i}
=\frac{p^{-2}_i}{1-\frac{1}{p_i}} \le 2p^{-2}_i$$
So
$$\sum_{i=1}^r \sum_{n=2}^{\infty}\left(np^{-n}_i\right) \le \sum_{i=1}^r 2p^{-2}_i
\le \sum_{n=1}^{\infty} \frac{1}{n^2}$$ which converges. Hence,
as $\log{P(x)}$ diverges to infinity, we have $S(x) \to \infty$.
\end{proof}

\begin{theorem} $\forall x \ge 2$, we have $P(x) \ge \log{x}$ and $S(x) \ge \log{\log{x}}-\frac{1}{2}$
\end{theorem}
In fact we will strengthen the result in the next subsection.
\begin{proof}
$$P(x) > \sum_{n=1}^{[x]}\frac{1}{n} >\int_1^{[x]}\frac{1}{n}dn=\log{([x]+1)} >\log{x}$$
As $0<u<1$, we have
$$-\log{(1-u)}-u=\frac{u^2}{2}\left(1+\frac{u}{\frac{3}{2}}+\ldots\right)
<\frac{u^2}{2}\left(1+u+u^2+\ldots\right)=\frac{u^2}{2(1-u)}$$
Apply the above with $u=\frac{1}{p}$ and sum over all prime less than or equal to $x$, we have
$$0 \le \log{P(x)}-S(x)<\frac{1}{2}\sum_{p \in P,p \le x}\frac{1}{p(p-1)}<\frac{1}{2}\sum_{n=2}^{\infty}\frac{1}{n(n-1)}=\frac{1}{2}$$
So $S(x) \ge \log{\log{x}}-\frac{1}{2}$
\end{proof}

\subsection{Legendre formulae}
\begin{definition} Let $p_n$ be the $n^{th}$ prime. For $x \ge 2, r \ge 1$, we define
$N_r(x)$ to be the number of integers $n$ with $1 \le n \le x$ which are not divisible by any of the
first $r$ primes.
\end{definition}
Intuitively, we would guess by inclusion exclusion that
$$N_r(x)=[x]-\sum_{i=1}^r\left[\frac{x}{p_i}\right] +\ldots +(-1)^r\left[\frac{x}{p_1 \ldots p_r}\right]$$

Here is one special case: Let $r=\pi(\sqrt{x})$. Every composite in $[2,x]$ must be divisible by some prime number $p_i, i \le r$ where $p_1,p_2,\ldots,p_r$ are all prime numbers $\le \sqrt{x}$.
Then we have (Remember to include $1$)
$$N_r(x)=\pi(x)-\pi(\sqrt{x})+1$$
which is the number of prime in $[\sqrt{x},x]$ plus $1$. Hence
$$1+\pi(x)=\pi(\sqrt{x})+[x]+\ldots +(-1)^r\left[\frac{x}{p_1 \ldots p_r}\right]$$
\begin{definition} We extend Definition 6.3 such that we write $N_r(x,h)$ to be the number of integers
$\le x$, not divisible by $p_1,\ldots,p_r$ but divisible by $h$.
For example, $N_0(x,h)$ is the number of integers $\le x$ which are divisible by $h$, and clearly,
$N_0(x,h)=[\frac{x}{h}]$.
\end{definition}
\begin{lemma} Assume that $r \ge 1$ and $(h,p_r)=1$. Then
$$N_{r-1}(x,h)=N_r(x,h)+N_{r-1}(x,p_r h)$$
\end{lemma}
\begin{proof}[\bf Proof] $N_{r-1}(x,h)$ counts the number of integers in the set
$$\{n: 1 \le n \le x \text{ with } h|n \text{ and } (n,p_1p_2\ldots p_{r-1})=1\}$$
The set above can be written as a disjointunion of two sets, which are:
$$\{n: 1 \le n \le x \text{ with } h|n \text{ and } (n,p_1p_2\ldots p_r)=1\}$$
and
$$\{n: 1 \le n \le x \text{ with } h|n,p_r|n \text{ and } (n,p_1p_2 \ldots p_r)=1\}$$
As $(h,p_r)=1$, $h|n,p_r | n \iff hp_r |n$
and so the result follows.
\end{proof}
So we have
$$N_1(x,h)=N_0(x,h)-N_0(x,2h)$$
$$.~.~.~.~.~.~.~.~.~.~. $$
$$N_r(x,h)=N_0(x,h)- \sum_{i} N_0(x,p_i h)+ \ldots +(-1)^{r} N_0(x,p_1p_2\ldots p_r h)$$
This proves Legendre's formulae by setting $h=1$ and using $N_0(y,m)=[\frac{y}{m}]$.
Here is an application of Legendre's formulae.

\begin{theorem} $\lim_{x \to \infty} \frac{\pi(x)}{x}=0$
\end{theorem}
\begin{proof}[\bf Proof] Let $\zeta$ e any real number btween $2$ and $x$ for a fixed $x$. Let $r(\zeta)$ be an integer that $p_{r(\zeta)} \le \zeta <p_{r(\zeta)+1}$. We shall firstly prove that
$$\pi(x) < r(\zeta)+N_{r(\zeta)}(x)$$
Let $q$ be a prime $\le x$, then either $q \in \{p_1,\ldots p_{r(\zeta)}\}$ or it is not divisible
by any $p_i,i \le r(\zeta)$. But there are other composite number which is not divisible by these
$p_i$. Counting in this way, we have the above inequality.

Now we look for an upper bound for $N_{r(\zeta)}(x)$. Apply Legendre's formulae
$$N_r(x)=[x]-\sum_{i=1}^r\left[\frac{x}{p_i}\right]+\ldots +(-1)^r\left[\frac{x}{p_1\ldots p_r}\right]$$
and the error from dropping $[\cdot]$ is less than
$$1+\binom{r}{1}+\binom{r}{2}+\ldots +\binom{r}{r}=2^r$$ by considering the number of terms in each sum.
Hence we have
$$N_{r(\zeta)}(x) \le 2^{r(\zeta)}+x-\sum_{i=1}^{r(\zeta)}\frac{x}{p_i}+\ldots
=2^{r(\zeta)}+x \prod_{i=1}^{r(\zeta)} \left(1-\frac{1}{p_i}\right)$$
Use the result above, we have
$$\pi(x) \le r(\zeta)+2^{r(\zeta)}+x\prod_{i=1}^{r(\zeta)} \left(1-\frac{1}{p_i}\right)
\le 2\cdot 2^{r(\zeta)} + \prod_{i=1}^{r(\zeta)} \left(1-\frac{1}{p_i}\right)$$
because $r(\zeta) \le 2^{r(\zeta)}$. By Theorem 6.2,we have
$$\prod_{i=1}^{r(\zeta)} \left(1-\frac{1}{p_i}\right)^{-1} > \log{\zeta}$$
because $p_1,\ldots,p_{r(\zeta)}$ are all primes less than or equal to $\zeta$.
and so
$$\prod_{i=1}^{r(\zeta)} \left(1-\frac{1}{p_i}\right)^< \frac{1}{\log{\zeta}}$$
Therefore,
$$\frac{\pi(x)}{x} \le \frac{2^{r(\zeta)+1}}{x}+\frac{1}{\log{\zeta}}
=2e^{r(\zeta)\log{2}-\log{x}}+\frac{1}{\log{\zeta}}$$
Now we pick $\zeta=\frac{\log{x}}{2\log{2}}$ and use $r(\zeta) < \zeta$, so that
$$\frac{\pi(x)}{x} < 2e^{-\frac{\log{x}}{2}}+\frac{1}{\log{(\frac{\log{x}}{2\log{2}})}}$$
which tends to $0$ as $x \to \infty$.
\end{proof}




\subsection{The result of Chebyshev}
We begin the subsection by some basic properties:
\begin{lemma} The exact order of each prime factor of $n!$ is
$$\sum_{i=1}^{\infty} \left[\frac{n}{p^i}\right]$$
\end{lemma}
\begin{proof}[\bf Proof] The number of integers less than or equal to $n$ which are divisible by $p^i$
is $[\frac{n}{p^i}]$. The result follows by counting the number from each $i$.
\end{proof}
\begin{lemma} Let $n \ge 1$ throughout
\begin{enumerate}
\item[(i)] $2^n \le \binom{2n}{n} <2^{2n}$.
\item[(ii)]$\prod_{n<p \le 2n}p \big| \binom{2n}{n}$, where $p$ in the product runs through prime only.
\item[(iii)] Let $r(p)$ satisfies $p^{r(p)} \le 2n <p^{r(p)+1}$, then
$$\binom{2n}{n} \big| \prod_{p \le 2n}p^{r(p)}$$
\item[(iv)] If $n \ge 2$ and $\frac{2n}{3} < p \le n$, then $p \nmid \binom{2n}{n}$.
\item[(v)] $\prod_{p \le n}p < 4^n$.
\end{enumerate}
\end{lemma}
\begin{proof}
\begin{enumerate}
\item[(i)] $2^{2n}=(1+1)^{2n}$, and the expansion contains the term $\binom{2n}{n}$, and so
$2^{2n} \ge \binom{2n}{n}$. Also, $\binom{2n}{n}=\frac{(2n)(2n-1)\ldots(n+1)}{n(n-1)\ldots 1}$
and for each $i$, $\frac{2n-i}{n-i}\ge 2$, so that $\binom{2n}{n} \ge 2^n$.
\item[(ii)] Let $p$ be a prime such that $n<p \le 2n$. Then $p \nmid n!$ but $p \big|(2n)!$.
Let $k=\binom{2n}{n}=\frac{(2n)!}{(n!)^2}$, and so $(n!)^2 k=(2n)!$. Then
$p \big| (2n)! \Rightarrow p\big| (n!)^2$ or $p \big| k$. But $p \nmid n!$ and so $p \big| k$.
As any two $(p_i,p_j)=1$, therefore, the product of them also divides $k$.
\item[(iii)] By Lemma 6.7, we have the exact power of $p$ dividing $\binom{2n}{n}$ is
$$\sum_{j=1}^{\infty} \left(\left[\frac{2n}{p^j}\right]-2 \left[\frac{n}{p^j}\right]\right)
=\sum_{j=1}^{r(p)}\left(\left[\frac{2n}{p^j}\right]-2\left[\frac{n}{p^j}\right]\right)
\le \sum_{j=1}^{r(p)} 1 =r(p)$$
Hence $\binom{2n}{n} \big| \prod_{p \le 2n}p^{r(p)}$.
\item[(iv)] Let $\frac{2n}{3} < p <n$.  Then the factor $p$ occurs once only in $n!$ because
$2p > \frac{4n}{3} >n$. Also $p$ occurs exactly twice in $(2n)!$ since $3p>2n$ and $2p <2n$.
Therefore the factor $p$ has been canceled in $\frac{(2n)!}{(n!)^2}$, and so the result follows.
\item[(v)] Apply induction on $n$. It is easy to check that the result holds for $n=1,2,3$.
Let $m>1$, suppose it is true for $2m-1$, then it is also true for $2m$ as
$$\prod_{p \le 2m}p=\prod_{p \le 2m-1}p <4^{2m-1}<4^{2m}$$
because $2m$ is not prime. So it remains to check that, if it is true for $m+1$, then it is also true for $2m+1$, and hence complete the proof. By (ii), every prime in the interval $[m+2,2m+1]$ divides
$\binom{2m+1}{m}$. So we have
$$\prod_{p \le 2m+1}p \le \left(\prod_{p \le m+1}p\right) \binom{2m+1}{m}
<4^{m+1} \binom{2m+1}{m}$$
Now $\binom{2m+1}{m}=\binom{2m+1}{m+1}$, and they both occur in the expansion
$(1+1)^{2m+1}$, so $\binom{2m+1}{m} \le \frac{2^{m+1}}{2}=4^m$.
So $$\prod_{p \le 2m+1} p < 4^{m+1}4^m=4^{2m+1}.$$
\end{enumerate}
\end{proof}
Now we come to a version of Chebyshev's result; note that the constants can be improved by using more
precise inequalities.
\begin{theorem}{\bf [Chebyshev]}\label{C;Chebyshev}
If $n>1$, then
$$\frac{n}{8\log{n}}<\pi(n)<\frac{6n}{\log{n}}$$
\end{theorem}
\begin{proof}[\bf Proof] By Lemma 6.8, $\binom{2n}{n} \big| \prod_{p \le 2n}p^{r(p)}$ and each $p>n$ divides
$\binom{2n}{n}$. So
\begin{equation}
n^{\pi(2n)-\pi(n)}<\prod_{n<p \le 2n}p \le \binom{2n}{n} \le \prod_{p \le 2n}p^{r(p)}
\le (2n)^{\pi(2n)} \tag{$\star$}
\end{equation}
using $p^{r(p)} \le 2n$. Also by Lemma 6.8, $\binom{2n}{n} \le 2^{2n}$, so we have
\begin{equation}
n^{\pi(2n)-\pi(n)}<\binom{2n}{n} \le 2^{2n} \tag{A}
\end{equation}
Also by Lemma 6.8, $\binom{2n}{n} >2^n$, and so by $(\star)$
\begin{equation}
(2n)^{\pi(2n)} \ge \binom{2n}{n} > 2^n \tag{B}
\end{equation}
Now, let $n=2^k$ then for $(A)$, we have (Compare the exponent)
$$k\left(\pi(2^{k+1})-\pi(2^k)\right) <2^{k+1}$$
and for $(B)$
$$(k+1)\pi(2^{k+1}) > 2^k$$
Rewrite $(A)$ as
$$(k+1)\pi(2^{k+1})-k\pi(2^k) <2^{k+1} +\pi(2^{k+1}) \le 3 \cdot 2^k$$
because $\pi(2^{k+1}) \le 2^k$ as every even number apart from $2$ is not prime, and $1$ is not prime.
So we can sum over the above from $k=0$ to $m$, that
$$\sum_{k=0}^{m}(k+1)\pi(2^{k+1})-k\pi(2^k)<\sum_{k=0}^{m}3 \cdot 2^k$$
But we can simplify LHS and use $\pi(1)=0$, so we have
$$(m+1)\pi(2^{m+1}) < 3(2^{m+1}-1)<3 \cdot 2^{m+1} $$
\begin{equation}
and so
\pi(2^{m+1}) <\frac{3 \cdot 2^{m+1}}{m+1} \tag{$A'$}
\end{equation}
By $(B)$, we have $(m+1)\pi(2^{m+1}) > 2^m$ and so
\begin{equation}
\pi(2^{m+1}) >\frac{2^m}{m+1}=2^{m+1}{2(m+1)} \tag{$B'$}
\end{equation}
So given any $n$, choose $m$ such that $2^{m+1} \le n < 2^{m+2}$.
Since $\frac{1}{2} < \log{2} <1$, we have
$$\frac{t}{2} < \log{2^t} <t ~\forall t \ge 1$$
Now by $(A')$,
$$\pi(n) \le \pi(2^{m+2}) <\frac{3 \cdot 2^{m+2}}{m+2} <\frac{3 \cdot 2^{m+2}}{\log{2^{m+2}}}
=\frac{6 \cdot 2^{m+1}}{\log{2^{m+2}}} \le \frac{6n}{\log{n}}$$
and by $(B')$,
$$\pi(n) \ge \pi(2^{m+1}) > \frac{2^{m+1}}{2(m+1)} >\frac{2^{m+2}}{8\left(\frac{m+1}{2}\right)}
>\frac{2^{m+2}}{8\log{2^{m+1}}} > \frac{n}{8\log{n}}$$
\end{proof}
\subsection{Bertrand's postulate and some other results}
We are going to strengthen some of the pervious results in this subsection. But before that, here is
a simple but useful theorem about prime numbers by Bertrand:
\begin{theorem}{\bf [Bertrand's postulate]}\label{B;Bertrand} For every integer $n \ge 1$, there exists
a prime $p$ such that $n < p \le 2n$.
\end{theorem}
\begin{proof}[\bf Proof] We know there is no prime $p$ dividing $\binom{2n}{n}$ for $\frac{2n}{3} <p \le n$.
Suppose we have no prime between $n$ and $2n$,
then the only prime divisors of $\binom{2n}{n}$ are those $\le \frac{2n}{3}$.
Let $S(p)$ be the exact power of $p$ dividing $\binom{2n}{n}$.
By Lemma 6.8, $\binom{2n}{n} \big| \prod_{p \le 2n}p^{r(p)}$ and so $p^{S(p)} \le 2n$ because
$p^{r(p)} \le 2n$.

Suppose $S(p)>1$, then $p \le \sqrt{2n}$, and hence no more than $[\sqrt{2n}]$ primes can divide
$\binom{2n}{n}$ more than once, and $p^{S(p)} \le 2n$ so we have
$$\binom{2n}{n} \le \left(2n\right)^{[\sqrt{2n}]}\prod_{p \le \frac{2n}{3}}p$$

Also, $\binom{2n}{n}$ is the largest term in the expansion $(1+1)^{2n}=4^n$ and there are in total
$2n+1$ terms, so we have $\binom{2n}{n} >\frac{4^n}{2n+1}$ and so
$$(2n)^{\sqrt{2n}} \prod_{p \le \frac{2n}{3}}p > \frac{4^n}{2n+1}$$
Since $\sqrt{2n}>[\sqrt{2n}]$.

Now by Lemma 6.8, $\prod_{p \le m}p <4^m$ and so
$$4^{\frac{2n}{3}}\left(2n\right)^{\sqrt{2n}}>\frac{4^n}{2n+1}$$ by above. Also $(2n+1)<(2n)^2$,
so $$4^{\frac{n}{3}}<\left(2n\right)^{2+\sqrt{2n}}$$
and so
$$\frac{n\log{4}}{3}<\left(2+\sqrt{2n}\right)\log{2n}$$
Numerically, this gives a contradiction when
$n \ge 750$. For the case when $n<750$, we check manually than there is prime
between $n$ and $2n$. Hence the result follows.
\end{proof}
Now we shall evaluate two important series defined on prims, they play a vital role in number theory.
We shall denote for any real $x$, $\sum_{p \le x}f(p)$ as the sum $f(2)+f(3)+\ldots f(p)$, where the
value $p$ takes over all primes less than or equal to $x$.
\begin{theorem}
$$\sum_{p \le x} \frac{\log{p}}{p}=\log{x}+O(1)$$
\end{theorem}
\begin{proof}[\bf Proof] This is derived for an integer variable first by estimating $\log{n!}$ in two distinct ways. By Lemma 6.7, we have
\be
\log{n!} = \log{\prod_{p \le n}p^{\left(\left[\frac{n}{p}\right]+\left[\frac{n}{p^2}\right]+\ldots \right)}} = \log{p}\left(\sum_{p \le n}\left[\frac{n}{p}\right]\right) + \log{p} \left(\sum_{p \le n} \sum_{i \ge 2} \left[\frac{n}{p^i}\right]\right)
\ee
We consider the sums separately. Clearly
$$\log{p} \sum_{p \le n}\left[\frac{n}{p}\right] \le \sum_{p \le n} \frac{n \log{p}}{p}$$
and
\begin{eqnarray*}
\log{p} \sum_{p \le n}\left[\frac{n}{p}\right] &>& \log{p} \sum_{p \le n}\left(\frac{n}{p}-1\right) = \sum_{p \le n} \frac{n \log{p}}{p} -\sum_{p \le n} \log{p}\\
&>&\sum_{p \le n} \frac{n \log{p}}{p} -\pi(n)\log{n} > \sum_{p \le n}\frac{n \log{p}}{p} -6n
\end{eqnarray*}
where the last inequality came from Theorem 6.9. For the second sum,
\be
\log{p}\sum_{p \le n}\sum_{i=2}^{\infty}\left[\frac{n}{p^i}\right]  \le  n \log{p} \sum_{p \le n} \sum_{i=2}^{\infty}\left(\frac{1}{p^i}\right) = n\sum_{p \le n} \frac{\log{p}}{p(p-1)} < n\sum_{j \le n} \frac{\log{j}}{j(j-1)}=c n
\ee
for some constant $c$ as the last series is convergent. Combining these inequalities we have
$$n\sum_{p \le n}\frac{\log{p}}{p}-6n \le \log{n!} \le n\sum_{p \le n}\frac{\log{p}}{p}+cn$$
Now we estimate $\log{n!}$ in a different way. We have $\log{n!}=\sum_{j \le n}\log{j}$ and
$$\sum_{j \le n}\log{j} > \int_1^n\log{y}dy=n\log{n}-n+1$$
Also $n! <n^n$, so we have
$$n\log{n}-2n< n\log{n}-n+1 <\log{n!}<n\log{n}$$
Now combine the two estimation, we have by the first estimation that
$$\log{n!}-cn \le n \sum_{p \le n} \frac{\log{p}}{p} \le \log{n!}+6n$$
and use the second estimation, we have
$$n\log{n}-(2+c)n < n\sum_{p \le n}\frac{\log{p}}{p} < n\log{n}+6n$$
which is the same as
$$\sum_{p \le n}\frac{\log{p}}{p}=\log{n}+O(1)$$
Finally, if replace the integer variable $n$ by a real variable $x$ we have, if $n \le x<n+1$,
$$\sum_{p \le x}\frac{\log{p}}{p}=\sum_{p \le n}\frac{\log{p}}{p}
=\log{n}+O(1)=\log{x}+O(1)$$
as $\log{x}-\log{n}<1$.
\end{proof}

For the our second result we need the following lemma:
\begin{lemma} Let $t_1,t_2,\ldots$ be a non-decreasing sequence of real numbers with limit infinity,
let $z_1,z_2,\ldots$ be any sequence of real numbers, and suppose $f$ is a real function with a
continuous derivative for arguments greater than or equal to $t_1$. Define $Z$ by
$$Z(x)=\sum_{t_n \le x}z_n$$
the sum of all $z_n$ for those $n$ which satisfy $t_n \le x$. Then
$$\sum_{t_n \le x}z_nf(t_n)=Z(x)f(x)-\int_{t_1}^x Z(y)f'(y)dy$$
\end{lemma}
\begin{proof}[\bf Proof] As $Z(t_{n+1})-Z(t_n)=z_{n+1}$ because $t_n$ is non-decreasing, we have, if $m$ is the largest subscript such that $t_m \le x$,
\begin{eqnarray*}
\sum_{t_n \le x}z_nf(t_n)&=&Z(t_1)f(t_1)+(Z(t_2)-Z(t_1))f(t_2) + \ldots (Z(t_m)-Z(t_{m-1}))f(t_m)\\
&=&Z(t_1)(f(t_1)-f(t_2))+\ldots+Z(t_{m-1})(f(t_{m-1})-f(t_m)) +Z(t_m)(f(t_m)-f(x))+Z(x)f(x)
\end{eqnarray*}
By definition of $m$, $Z(t_m)=Z(x)$. Now because $Z$ is constant on the interval
$[t_i,t_{i+1})$ we have
$$Z(t_i)(f(t_i)-f(t_{i+1}))=-\int_{t_i}^{t_{i+1}}Z(y)f'(y)dy$$
Combining these equations we have
\begin{eqnarray*}
\sum_{t_n \le x}z_n f(t_n)&=&Z(x)f(x)-\left\{\sum_{i=1}^{m-1}\int_{t_i}^{t_{i+1}}+\int_{t_m}^x\right\}
Z(y)f'(y)dy\\
&=&Z(x)f(x)-\int_{t_1}^x Z(y)f'(y)dy
\end{eqnarray*}
\end{proof}
The method employed at the beginning of the proof is called {\bf partial summation}; We shall use it often in the following pages.
We shall finish the subsection by a stronger version of Theorem 6.2.
\begin{theorem}
$$\sum_{p \le x}\frac{1}{p} = \log{\log{x}}+c+O\left(\frac{1}{\log{x}}\right)$$
for some constant $c$.
\end{theorem}
\begin{proof}[\bf Proof] In Lemma 6.12 $t_n=p_n$ the $n^{th}$ prime, $z_n=\frac{\log{p_n}}{p_n}$ and
$f(y)=\frac{1}{\log{y}}$, so we have if $x \ge 2$
\begin{eqnarray*}
\sum_{p \le x}\frac{1}{p}&=&\sum_{p \le x}\frac{\log{p}}{p}\frac{1}{\log{p}}
=\frac{1}{\log{x}}\sum_{p \le x}\frac{\log{p}}{p}+\int_2^x \left(\sum_{p \le t}\frac{\log{p}}{p}\right)
\frac{dt}{t \log^2{t}}\\
&=&\frac{1}{\log{x}}(\log{x}+O(1))+\int_2^x\frac{\log{t}dt}{t\log^2{t}}+
\int_2^x\left(\sum_{p \le t}\frac{\log{p}}{p}-\log{t}\right)\frac{dt}{t\log^2{t}}\\
&=&1+O\left(\frac{1}{\log{x}}\right)+\log{\log{x}}-\log{\log{2}}+
\left(\int_2^\infty -\int_x^\infty \right)\frac{O(1)dt}{t\log^2{t}}\\
&=&\log{\log{x}}+c+O\left(\frac{1}{\log{x}}\right)
\end{eqnarray*}
where we use Theorem 6.11 to evaluate $\sum_{p \le x}\frac{\log{p}}{p}$. The constant $c$ exists as the first integral in the line above converges and the second has order $O(\frac{1}{\log{x}})$.
\end{proof}







\subsection{Riemann zeta function}
We have mentioned Riemann zeta function in Chapter $2$, and now we are going to discuss in detail
about some very important properties of Riemann zeta function. The function is given by:
$$\zeta(s)=\sum_{n=1}^\infty n^{-s}$$
for $s$ a complex variable. If we write $s=\sigma+it (\sigma,t$ real$)$, then it is convergent for
$\sigma >1, -\infty < t <\infty$. There are lots of connections between Riemann zeta function and the primes. We shall begin with:
\begin{theorem}
\begin{enumerate}
\item[(i)] If $\sigma \ge 1+\epsilon$, for $\epsilon > 0$, then the series
$$\sum_{n=1}^{\infty}n^{-s}$$
is absolutely and uniformly convergent.
\item[(ii)] For $\sigma>1$,
$$\zeta(s)=\prod_p \left(1-p^{-s}\right)^{-1}$$
where the product is taken over all prime numbers $p$.
\end{enumerate}
\end{theorem}
\begin{proof}
\begin{enumerate}
\item[(i)] For an integer $j>1$. we have
$$\left|\sum_{n=j}^{\infty} n^{-s}\right| \le \sum_{n=j}^\infty\left|n^{-s}\right|
=\sum_{n=j}^\infty n^{-\sigma} \le \int_{j-1}^\infty x^{-\sigma}dx
\le \frac{(j-1)^{1-\sigma}}{\sigma -1} <\frac{j^{-\epsilon}}{\epsilon}$$
Hence the result follows.
\item[(ii)] Let $P$ be a fixed prime. Then
$$\prod_{p \le P}\left(1-p^{-s}\right)^{-1}=\sum_{p \le P}(1+p^{-s}+p^{-2s}+\ldots)
=\sum_{n \in X}n^{-s}$$
where $X$ contains all integer composed of primes $\le P$. So
$$\left|\zeta(s)-\prod_{p \le P}\left(1-p^{-s}\right)^{-1}\right|
=\left|\sum_{n \not \in X}n^{-s}\right| \le \sum_{n \not \in X} n^{-\sigma}
\le \sum_{n=P+1}^\infty n^{-\sigma}$$
as $n \not \in X \Rightarrow n \ge P+1$. The last sum tends to $0$ as $P$ tends to infinity as
$\sigma >1$.
\end{enumerate}
\end{proof}
\begin{corollary} $\zeta(s)$ has no zero when $\sigma>1$.
\end{corollary}
\begin{proof}[\bf Proof] Let $P$ be a fixed prime.
$$\prod_{p \le P}\left(1-p^{-s}\right)\zeta(s)=1+m^{-s}_1+m^{-s}_2+\ldots$$
where $m_i$ are integers which have at least one prime factor $>P$.
Then
\be
\left|\prod_{p \le P}\left(1-p^{-s}\right)\zeta(s)\right|  \ge 1-m^{-\sigma}_1-m^{-\sigma}_2-\ldots \ge 1- \left(p+1\right)^{-s}-\left(p+2\right)^{-\sigma}-\ldots \ge 1-\frac{1}{2} =\frac{1}{2}
\ee
for $P$ large enough. Therefore,
$$\left|\prod_{p \le P}(1-p^{-s})\zeta(s)\right| \ge \frac{1}{2}$$
and $1-p^{-s} \neq 0$ for $\sigma >1$ since each $1-p^{-s} \neq 0$, and so $\zeta(s) \neq 0$.
\end{proof}
\begin{lemma} Let $\{a_n\},\{b_n\}$ be any sequence in $\mathbb{C}$ satisfying:
\begin{enumerate}
\item[(i)] $\sum_{n=1}^\infty a_n b_n$ converges.
\item[(ii)] $A_n b_n \to 0$, where $A_n=a_1+\ldots a_n$.
\end{enumerate}
Then $\sum_{n=1}^\infty A_n(b_n-b_{n+1})$ converges to $\sum_{n=1}^\infty a_n b_n$.
\end{lemma}
\begin{proof}[\bf Proof] Define $A_0=0$. $S_N=\sum_{n=1}^N a_nb_n$ Then
\begin{eqnarray*}
S_N&=&\sum_{n=1}^N(A_n-A_{n-1})b_n=\sum_{n=1}^{N}(A_nb_n-A_{n-1}b_n)\\
&=&\sum_{n=1}A_nb_n-\sum_{n=0}^{N-1}A_n b_{n+1} = A_Nb_N+\sum_{n=0}^{N-1}A_n(b_n-b_{n+1})
\end{eqnarray*}
As $N \to \infty$, as $S_N$ converges and $A_Nb_N \to 0$, so
$$\sum_{n=0}^{N-1}A_n(b_n-b_{n+1}) \to S_N$$
as $N \to \infty$.
\end{proof}
The following theorem extends the domain of $\zeta$:
\begin{theorem} $\zeta(s)-\frac{1}{s-1}$ has an analytic continuation in the half plane
$\mathcal{R}e(s)>0$.
\end{theorem}
\begin{proof}[\bf Proof] Apply Lemma 6.16 to $\zeta(s)$ with $a_n=1,b_n=n^{-s}$
So $\sum a_nb_n$ converges, and $A_nb_n=n n^{-s} \to 0$. Then we have
$$\sum A_n(b_n-b_{n+1}) \to \sum a_n b_n=\zeta(s)$$
Write
\begin{eqnarray*}
\zeta(s)&=&\sum_{n=1}^{\infty}n(n^{-s}-(n+1)^{-s}) = s\sum_{n=1}^\infty n\int_{n}^{n+1} x^{-s-1}dx\\
&=&s\sum_{n=1}^\infty \int_n^{n+1} [x] x^{-s-1} dx = s \int_1^\infty [x] x^{-s-1} dx\\
&=&s \left(x-(x) \right)x^{-s-1}dx = s\int_1^\infty x^{-s}-(x)x^{-s-1}dx\\
&=&\frac{s}{s-1}-s\int_1^{\infty} x^{-s-1}(x) dx 
\end{eqnarray*}
Therefore,
$$\zeta(s)-\frac{1}{s-1}=1-s\int_1^\infty x^{-s-1}(x) dx$$
and the second term converges absolutely. Therefore we have an analytic continuation.
\end{proof}
For $\sigma \le 0$, $\zeta(s)$ can be defined using the {\bf functional equation} for $\zeta$ as follows.
Let $\Gamma$, the $gamma$ function, be given by (for $\sigma>0$)
$$\Gamma(s)=\int_0^\infty e^{-x}x^{s-1}dx$$
Note that $\gamma(s+1)=s\gamma(s)$ (integrate by parts), and so $\gamma(m)=(m-1)!$ for positive integer $m$.
This relation can also be used to extend the domain of definition of $\gamma$ to the whole plane. $\gamma$ has a simple pole at $s=0,-1,-2,\ldots$ and nowhere else.

Riemann showed that, if $\sigma<0$,
$$\zeta(s)=2^s \pi^{s-1} \Gamma(1-s)\zeta(1-s)\sin{\left(\frac{\pi s}{2}\right)}$$
We have many important properties of the Riemann zeta functions, for which we shall only state the results and will not go through any detail.
\begin{theorem} Define
$$\Lambda = \pi^{-\frac{s}{2}}\Gamma\left(\frac{s}{2}\right)\zeta(s)$$
for $\sigma >0$ where $s=\sigma+it$. Then
$$\Lambda(s)=\Lambda(1-s)$$
\end{theorem}
This theorem implies that the zeros of $\zeta(s)$ are symmetric about the line $\mathcal(R)e(s)=\frac{1}{2}$.
We also have $\zeta(s) <0$ when $s \in [0,1)$ and that $\zeta(s)$ has at most a simple $0$ on the line
$\mathcal{R}e(s)=1$ (See Exercise 7). In fact we shall prove later that $\zeta(s)$ has no zero on the line $\mathcal(R)e(s)=1$.
We also have
\begin{theorem}{\bf [Hardy]}\label{H;Hardy} There are infinitely many zeros on the line $\mathcal{R}e(s)=\frac{1}{2}$.
\end{theorem}
Riemann's hypothesis states that every $0$ of $\zeta(s)$ (for $\mathcal{R}e(s)>0)$ lies on the line $\mathcal{R}e(s)=\frac{1}{2}$. Numerically, the first $10^9$ zeros lies on this line.
\begin{definition} Dirchlet Character mod $N$ is a homomorphism
$$\chi: (\mathbb{Z}/N\mathbb{Z})^* \rightarrow \mathbb{C}^*$$
and $\bar{\chi}: \mathbb{Z} \rightarrow \mathbb{C}$ by
\begin{equation*}
\bar{\chi}(n)= \left\{
\begin{array}{ll}
0 & \text{if } (n,N)>1\\
\chi(n+N\mathbb{Z}) & \text{if } (n,N)=1\\
\end{array} \right.
\end{equation*}
and We define $L(\chi,S)=\sum_{n=1}^\infty \frac{\bar{\chi}}{n^s}$
\end{definition}
The general Riemann's hypothesis states that fir every $\chi$, every zero of $L(\chi,s)$ in the region
$\mathcal{R}e(s) \ge 0$ satisfies $\mathcal{R}e(s)=\frac{1}{2}$. The general Riemann's hypothesis plays a very important role in number theory.

\begin{flushleft}
$Applications$
\end{flushleft}
The first application is a strong version of one of the facts we stated earlier.
\begin{theorem} For all real $t$, $\zeta(1+it) \neq 0$
\end{theorem}
\begin{proof}
By Theorem 6.14(ii)
$$\zeta(s)=\prod_p (1-p^{-s})^{-1}$$
so
$$\log{\zeta(s)}=-\sum_p \log{(1-p^{-s})}=\sum_p \sum_n \frac{p^{-ns}}{n}$$
Therefore,
$$\zeta(s)=\exp{\left(\sum_p \sum_n \frac{p^{-ns}}{n}\right)}$$
Let $s=\sigma+it$. Then
$$\left|\zeta(s)\right|=\exp{\left(\sum_{p,n}\frac{\cos{(nt\log{p})}}{np^{n\sigma}}\right)}$$
because the modulus of $e^{z}$ is $e^{\mathcal{R}e(z)}$ and we write
$$p^{-ns}=p^{-n\sigma}e^{-int\log{p}}$$
Recall $\cos{(2\phi)}=2\cos^2{\phi}-1$ and that
$$3+4\cos{\phi}+cos{(2\phi)}=2(1+\cos{\phi})^2 \ge 0$$
Therefore,
$$|\zeta^3(\sigma)\zeta^4(\sigma+it)\zeta(\sigma+2it)|
=\exp{\left(\sum_{p,n}\frac{3+4\cos{(nt\log{p})}+\cos{(2nt\log{p})}}{np^{n\sigma}}\right)}$$
and the numerator is $\ge 0$ by above.
Therefore,
$$|\zeta^3(\sigma)\zeta^4(\sigma+it)\zeta(\sigma+2it)| \ge 1$$
for $\sigma > 1$.

Now suppose $\zeta(1+it)=0$. By Theorem 6.17 we know that $\zeta$ has a pole at $1$. So we look at
$\sigma+it$ and let $\sigma \to 1^+$. As $\sigma \to 1^+$, $\zeta^3(\sigma)$ has triple pole, and
by assumption $\zeta^4(\sigma+it)$ has quadruple zero. So we have
$$|\zeta^3(\sigma)\zeta^4(\sigma+it)\zeta(\sigma+2it)| \to 0$$
as $\sigma \to 1$. But we have shown above that this expression is $\ge 1$, which is a contradiction.
\end{proof}
A typical analytic proof of PNT begins with the result:
\begin{theorem} For $\sigma >1$,
$$\log{\zeta(s)}=s \int_2^\infty \frac{\pi(x)}{x(x^s-1)} dx$$
\end{theorem}
\begin{proof}[\bf Proof] By Theorem 6.14(ii), using partial summation we have
\begin{eqnarray*}
\log{\zeta(s)}&=&-\sum_p \log{(1-p^{-s})}=-\lim_{k \to \infty}\sum_{n=2}^{k}(\pi(n)-\pi(n-1))
\log{(1-n^{-s})}\\
&=&\lim_{k \to \infty}\left[\sum_{n=2}^{k-1}\pi(n)(\log{(1-(n+1)^{-s})}-\log{(1-n^{-s})})-\pi(k)\log{(1-k^{-s})}\right]
\end{eqnarray*}
Now for $\sigma >1$ the final term in
the limit above tends to zero as $k \to \infty$, and so by the
fundamental theorem of the calculus
\begin{eqnarray*}
\log{\zeta(s)}&=&\sum_{n=2}^\infty \pi(n)\int_n^{n+1} \frac{d}{dx}\log{(1-x^{-s})}dx\\
&=&s \sum_{n=2}^\infty \int_n^{n+1} \frac{\pi(x)}{x(x^s-1)}dx=s\int_2^\infty \frac{\pi(x)}{x(x^s-1)}dx
\end{eqnarray*}
\end{proof}
Now PNT can be derived by solving this equation for $\pi(x)$, it uses a version of the Meliin transform. The accuracy of the estimate for $\pi(x)$ given by PNT is limited by Theorem 6.21, this estimate could be greatly improved if the Riemann hypothesis was available. We shall give another proof (in detail) of PNT
later.

In the second application we define the function $M$ by
$$M(x)=\sum_{n \le x}\mu(n)$$
where $\mu$ is the M$\ddot{\text{o}}$bius function. As $|\mu(n)| \le 1$ holds for all $n$, we have immediately $M(x)=O(x)$. The question arises: what is the true order of $M(x)?$. In 1897 Mertens conjectured that, for all $\epsilon >0$ and $x>0$,
$$M(x) = o\left(x^{\frac{1}{2}+\epsilon}\right)$$
This is equivalent to Riemann hypothesis. One part of the equivalence can be demonstrated easily.
In Lemma 2.22, we have shown that for $\sigma >1$,
$$\frac{1}{\zeta(s)}=\sum_{n=1}^\infty \frac{\mu(n)}{n^s}$$
By partial summation we derive, as $M(0)=0$,
\be
\frac{1}{\zeta(s)} = \sum_{n=1}^{\infty} \frac{M(n)-M(n-1)}{n^s} =\sum_{n=1}^\infty M(n)\left\{\frac{1}{n^s}-\frac{1}{(n+1)^s}\right\}=s\int_0^{\infty}M(x)x^{-s-1}dx
\ee
It is a simple matter to show that if $M(n)$ grows less faster than $x^c$ than the integral
above converges when $\sigma >c$. So to prove the Riemann hypothesis it is sufficient to derive
$$M(x)=o\left(x^{\frac{1}{2}+\epsilon}\right)$$

\subsection{Bernoulli numbers}
The last subsection of the chapter is an introduction to Bernoulli numbers and we shall use them to
evaluate the zeta function at some integer arguments.
\begin{definition} The sequence $B_0,B_1,B_2,\ldots$ of rational numbers, called the {\bf Bernoulli numbers}, is given by $B_0=1$, and if $m>0$,
$$(m+1)B_m=-\sum_{j=0}^{m-1}\binom{m+1}{j}B_j$$
\end{definition}
So we have
$$2B_1=-1, 3B_2=-3B_1-1$$ and so on, and we may use this to calculate the first few $B_i$:
$$B_1=-\frac{1}{2},B_2=\frac{1}{6},B_3=0,B_4=-\frac{1}{30},B_5=0,B_6=\frac{1}{42},\ldots$$
We shall see below that, for all positive integers $n$, $B_{2n+1}=0$ and $B_2n$ alternates in sign. First we give an equivalence definition:
\begin{lemma} If we expand $\frac{t}{e^t-1}$ as a power series in $t$, then
$$\frac{t}{e^t-1}=\sum_{n=0}^\infty \frac{B_nt^n}{n!}$$
\end{lemma}
The proof is left as an exercise. (Exercise 11).
\begin{corollary} If $n>0$, then $B_{2n+1}=0$.
\end{corollary}
\begin{proof}[\bf Proof] As $B_1 =-\frac{1}{2}$ we have by Lemma 6.24
$$\frac{t}{e^t-1}+\frac{t}{2}=1+\sum_{n=2}^\infty \frac{B_nt^n}{n!}$$
But
$$\frac{t}{e^t-1}+\frac{t}{2}=\frac{t(e^t+1)}{2(e^t-1)}$$
which is an even function of $t$. Hence the odd power coefficients $\frac{B_3}{3!},\frac{B_5}{5!},\ldots$
are all zero.
\end{proof}
In the next result we evaluate $\zeta(2t)$ in terms of $B_{2t}$ where $t$ is a positive integer.
\begin{theorem}{\bf [Euler]}\label{E;Euler, zeta} If $t$ is a positive integer, then
$$2(2t)!\zeta(2t)=(-1)^{t+1}(2\pi)^{2t}B_{2t}$$
\end{theorem}
\begin{proof}[\bf Proof] We quote a standard result from analysis:
$$\sin{x}=x\prod_{n=1}^\infty \left(1-\frac{x^2}{n^2 \pi^2}\right)$$
(Intuitively, we came up with this expression by expanding $\sin{x}$ as a series in $x$ and so
we may regard $\frac{\sin{x}}{x}$ as an infinite polynomial in $x$ which, may be written as
$$\left(1-\frac{x}{r_1}\right)\left(1-\frac{x}{r_2}\right) \cdots$$
because $0$ is not a root of the polynomial and the constant term is $1$ by taking the limit of $x \to 0$. The roots of $\frac{\sin{x}}{x}$ are $\pm \pi, \pm 2\pi, \ldots$ and so we may write it intuitively as
$$\frac{\sin{x}}{x}=\prod_{n=1}^\infty \left(1-\frac{x}{n \pi}\right)\left(1-\frac{x}{-n \pi}\right)$$
which is what we want. But in order to write down the expression as an infinite product, we must show that the product is absolutely convergent and so we may swap to order of terms. We shall not explain in detail of the proof. Taking the logarithm of the product, we have an infinite sum, and it is equivalent to show the infinite sum is absolutely convergent. But for each $x$, there is $n$ such that
$\frac{x}{n}$ is small enough and the expansion of $\log{1-t}$ is roughly $-t$ for $t$ small but positive, and we use the fact that $\sum_{n=1}^\infty \frac{1}{n^2}$ converges.)

Now take the logarithmic derivative we have
$$\cot{x}=\frac{1}{x}-2\sum_{n=1}^\infty \frac{x}{n^2\pi^2-x^2}$$
and so, expanding each term in this sum as a geometric series,
$$x\cot{x}=1-2\sum_{n=1}^\infty \sum_{t=1}^\infty \left(\frac{x}{n\pi}\right)^{2t}
=1-2\sum_{t=1}^\infty\zeta(2t)\left(\frac{x}{\pi}\right)^{2t}$$
Further, as $\cot{x}=i\frac{(e^{ix}+e^{-ix})}{e^{ix}-e^{-ix}}$, we also have
\be
x\cot{x} = ix \frac{e^{2ix}+1}{e^{2ix}-1} = ix+\frac{2ix}{e^{2ix}-1} = 1+\sum_{t=2}^\infty \frac{B_t(2ix)^t}{t!}
\ee
by Lemma 6.24. Equating coefficient of $x^{2t}$ $(t>0)$ these equations yield
$$-\frac{2\zeta(2t)}{\pi^{2t}}=\left(-1\right)^t 2^{2t}\frac{B_{2t}}{(2t)!}$$
as required.
\end{proof}
\begin{corollary}
\begin{enumerate}
\item[(i)] If $t>0$, $(-1)^{t+1}B_{2t} >0$
\item[(ii)] $|B_{2t}|>2(\frac{t}{\pi e})^{2t}$
\end{enumerate}
\end{corollary}
\begin{proof}
\begin{enumerate}
\item[(i)] By Theorem 6.26, when $t>0$,
$$(-1)^{t+1}B_{2t}=\frac{2(2t)!\zeta(2t)}{(2\pi)^{2t}}>0$$
\item[(ii)] Take modulus of the above expression, we have
$$|B_{2t}|=\frac{2(2t)!\zeta(2t)}{(2\pi)^{2t}}>2\frac{(2t)!}{(2\pi)^{2t}}$$
as $\zeta(2t)>1$. Now we use
$$\log{n!} > n\log{n}-n$$ by considering the area bounded by $\log{x}$ from $1$ to $n$.
So that
$$e^{n}>\frac{n^n}{n!}$$
and hence the result follows by setting $n=2t$ above.
\end{enumerate}
\end{proof}
Here is another immediate useful consequence:
\begin{corollary} $\zeta(2)=\frac{\pi^2}{6}$
\end{corollary}
\begin{proof}[\bf Proof] Put $t=1$ in Theorem 6.26 and use $B_2=\frac{1}{6}$.
\end{proof}
In fact, we can also calculate that $\zeta(4)=\frac{\pi^4}{90}, \zeta(6)=\frac{\pi^6}{945}$ and so on
by using Theorem 6.26. No similar expression is known for $\zeta(2n+1)$. Note further that if we accept
the functional equation
$$\zeta(s)=2^s \pi^{s-1}\Gamma(1-s)\zeta(1-s)\sin{(\frac{\pi s}{2})}$$
we can calculate $\zeta(-n)$ for $n \ge 0$. We can show that $\zeta(-2n)=0$ if $n>0$ (Exercise 12)
 and the functional equation and Theorem 6.26 give, again if $n>0$
\begin{eqnarray*}
\zeta(1-2n)&=&2^{1-2n}\pi^{-2n}\zeta(2n)\Gamma(2n)\sin{(\frac{\pi}{2}-n\pi)}\\
&=&2^{1-2n}\pi^{-2n}(-1)^{(n-1)}\frac{2^{2n}\pi^{2n}}{2(2n)!}B_{2n}(2n-1)!(-1)^n=-\frac{B_{2n}}{2n}
\end{eqnarray*}
We shall see some further properties of Bernoulli numbers in the exercise.




\subsection{Exercises}
\begin{enumerate}
\item Let $N$ be a positive integer. Using $N! >(\frac{N}{e})^{N}$, prove that
$$\prod_{p \le N}\frac{\log{p}}{p-1} > \log{N}-1$$
\item Prove that every non-constant polynomial with integer coefficients assumes infinitely many composite values.
\item Prove that every integer $N>6$ can be expressed as a sum of distinct primes.
\item \begin{enumerate}
      \item[(i)] By considering the case $k<n$ and $k \ge n$ separately, show that
$$\frac{1}{n}+\frac{1}{n+1}+\ldots+\frac{1}{n+k}=m$$ is not soluble.
      \item[(ii)]Show that the equation
      $$n!=m^k$$
      is only soluble if at least one of $k,m$ or $n$ is $1$.
      \end{enumerate}
\item \begin{enumerate}
      \item[(i)] Let $p_n$ be the $n^{th}$ prime, show that there exist constants $c$ and $c'$ such that
      for $n>1$
      $$cn \log{n} < p_n < c'n \log{n}$$
      (Hint: Use the result $\log{n}=o(\sqrt{x})$).)
      \item[(ii)] Hence show that consecutive primes can be arbitrarily far apart.
      \end{enumerate}
\item Recall from question 6 in Exercise 2, that
\begin{equation*}
\Lambda(n)= \left\{
\begin{array}{ll}
\log{p} & \text{if } n \text{ is a power of prime } p\\
0 & \text{otherwise } \\
\end{array} \right.
\end{equation*}
Prove that
$$\sum_{n \le x}\frac{\Lambda(n)}{n}=\log{x}+O(1)$$
\item Prove that if there is a zero of $\zeta(s)$ on the line $\mathcal{R}e(s)=1$, then
      it must be a simple zero.
\item Use Lemma 6.12 to prove
      \begin{enumerate}
      \item[(i)]$\sum_{n \le x}\log{n}=x\log{x}-x+O(\log{x})$.
      \item[(ii)]$\sum_{n \le x}\log^2{n}=x \log^2{x}-2x\log{x}+2xO(\log^2{x})$.
      \end{enumerate}
\item \begin{enumerate}
      \item[(i)] Show that
      $$\sum_{p \le x}\left(\log{\left(1-\frac{1}{p}\right)}+\frac{1}{p}\right)=O(1)$$
      \item[(ii)] Hence show that
      $$\phi(n)>\frac{cn}{\log{\log{n}}}$$ for some constant $c$, by considering the primes less than
      $\log{n}$ which divides $n$.
      \end{enumerate}
\item \begin{enumerate}
\item[(i)] Find $$\lim_{s \to 1^+}(s-1)\zeta(s)$$
\item[(ii)]
Use the functional equation to calculate $\zeta(0)$.
\end{enumerate}
\item Prove Lemma 6.24.
\item Show that $\zeta(-2n)=0$, where $n$ is a positive integer.
\item Let $S_k(m)=\sum_{n=1}^{m-1} n^k$. Show that
$$\sum_{j=0}^{m-1}e^{jt}=\sum_{k=0}^{\infty} S_k(m)\frac{t^k}{k!}$$
      and so derive the formula
$$(k+1)S_k(m)=\sum_{j=0}^k \binom{k+1}{j}B_j m^{k+1-j}$$
      Deduce that $S_3(m)=\frac{m^2(m-1)^2}{4}=(S_1(m))^2$.
\item[$^\star$ 14.] In this question we are going to derive the Clausen-von Staudt theorem which determines the denominators of the Bernoulli numbers. Let $f(t)=\sum_{n=0}^\infty \frac{a_nt^n}{n!}$ and $g(t)=\sum_{n=0}^\infty \frac{b_nt^n}{n!}$. The series $f$ is called integral if each $a_n$ is
    an integer, and we write $f \equiv g$ (mod $m$) when $a_n \equiv b_n$ (mod $m$) for all $n$, and
    $f$ and $g$ are integral. It is a simple matter to show that if $f$ and $g$ are both integral, so are $f'$, $\int_0^tf(x)dx$ and $fg$. Using these, prove that:
    \begin{enumerate}
    \item[(i)] If $f$ is integral, so is $\frac{f^m}{m!}$ provided that $f(0)=0$.
    \item[(ii)] Deduce that if $m>4$ and is composite, then
    $$(e^t-1)^{m-1} \equiv 0~(\text{mod } m)$$
    and
    $$(e^t-1)^3 \equiv 2\sum_{k=1}^\infty \frac{t^{2k+1}}{(2k+1)!}~(\text{mod } 4)$$
    and if $p$ is an odd prime, then
    $$(e^t-1)^{p-1} \equiv -\sum_{k=1}^\infty \frac{t^{kp-k}}{(kp-k)!}~(\text{mod } p)$$
    \item[(iii)] Write $t=\log{(1+(e^t-1))}$ and so expand $\frac{t}{e^t-1}$ as a power series in
    $e^t-1$. Further, use the previous parts, show that
    $$\sum_{p-1 \big|2n} \frac{1}{p}+B_{2n}$$ is an integer, where the sum is taken over all primes $p$ satisfying $p-1 \big|2n$. This determines the denominator of $B_{2n}$.
    \item[(iv)] Deduce that the denominator of the Bernoulli number $B_{2n+1}$ is divisible by $6$.
    \end{enumerate}
\end{enumerate}

\section{Algebraic background-Integral domain}

The next few chapters are intended as  detailed explanations in the ring theory and modules for those beginner of algebra. Readers with strong background of ring theory may skip those chapters. Basic properties and definition of the group theory will be assumed.\\
Sometimes we shall use the notation $\mathbb{Z}[a]$ to mean $\mathbb{Z}+\mathbb{Z}a$.

\subsection{Basic properties}

\begin{definition} 
An {\bf integral domain} is a commutative ring that has a multiplicative identity but no divisor of zero. In other words, if $ab=0$, then $a=0$ or $b=0$.

An integral domain $D$ is called a field if for each $a \in D$, $a \neq 0$, there exists $b \in D$ with $ab=1$.
\end{definition}

\begin{example} 
The ring $\mathbb{Z}=\{0,\pm1,\pm2,\ldots\}$ of all integers is an integral domain.
\end{example}

\begin{example} 
The Gaussian integer is of the form $\mathbb{Z}+\mathbb{Z}i$, which is an integral domain and is $\mathbb{Z}+\mathbb{Z}i$ is called the {\bf Gaussian domain}.
\end{example}

\begin{example} The set $\mathbb{Z}+\mathbb{Z}\omega=\{a+b\omega: a,b \in Z\}$, where $\omega$ is the complex cube root given by $\omega=\frac{-1+\sqrt{-3}}{2}$, is an integral domain, and the element in the set is called {\bf Eisenstein integers}, and the set is called {\bf Eisenstein domain}.
\end{example}

\begin{example} $\mathbb{Z}+\mathbb{Z}\sqrt{m} =\{a+b\sqrt{m}: a,b \in \mathbb{Z}\}$, where $m$ is a positive or negative integer that is not a perfect square, is an integral domain. $\mathbb{Z}+\mathbb{Z}\sqrt{m}$ is called a {\bf quadratic domain}.
\end{example}

\begin{example} $F[X]=$ the ring of polynomials in the indeterminate $x$ with coefficients from a field $F$ is an integral domain. Similarly, $F[x,y]=$ the ring of polynomials in the two indeterminates $x$ and $y$ with coefficients from the field $F$ is an integral domain.
\end{example}

\begin{example} 
$\mathbb{Z}+\mathbb{Z}\theta+\mathbb{Z}\theta^2=\{a+b\theta+c\theta^2: a,b,c \in \mathbb{Z}\}$, where $\theta$ is a root of the cubic equation $\theta^3+\theta+1=0$, is an integral domain. It is called a cubic domain.
\end{example}

\begin{lemma} Let $D$ be a integral domain. Then:
\begin{enumerate}
\item[(i)] The identity element of $D$ is unique.
\item[(ii)] $D$ possesses a left cancellation law, that is:
$$ab=ac, a \neq 0 \Rightarrow b=c$$ as well as a right cancellation law $$ac=bc,c \neq 0 \Rightarrow a=b$$
\item[(iii)] There exists a field $F$, called the field of quotients of $D$ or the quotient field of $D$, that contains an isomorphic copy $D'$ of $D$.
\end{enumerate}
\end{lemma}

\begin{proof}
\begin{enumerate}
\item[(i)] Let $1$,$1'$ both be identities for $D$ then
$$1=1\cdot1'=1'$$
\item[(ii)] $$ab=ac \Rightarrow a(b-c)=0 \Rightarrow a=0 \text{ or } b-c=0$$
But $a \neq 0$ and so $b=c$. The other case is similar.
\item[(iii)]
\end{enumerate}

\end{proof}
\begin{definition} Let $a$ and $b$ belong to the integral domain $D$. The element $a$ is said to be a {\bf divisor} of $b$ (or $a$ divides $b$), written $a | b$ if there exists an element $c$ of $D$ such that $b=ac$. If $a$ is not a divisor of $b$, we write $a \nmid b$.
\end{definition}
\begin{example} In $\mathbb{Z}+\mathbb{Z}i$, $1+i | 2$ as $2=(1+i)(1-i)$.
\end{example}

\begin{center} 
{\bf Properties of divisors}
\end{center}

\begin{enumerate}
\item[(i)] $a|a$ (reflexive).
\item[(ii)] $a|b$ and $b|c$ implies $a|c$ (transitivity).
\item[(iii)] $a|b$ and $a|c$ implies that $a|xb+yc$ for any $x,y \in D$
\item[(iv)] $a|b \Rightarrow ac|bc$.
\item[(v)] $ac|bc$ and $c \neq 0$ implies $a|b$, because we have some $d \in D$ such that $acd=bc \Rightarrow ad=b$
\item[(vi)] $1|a$.
\item[(vii)] $a|0$.
\item[(viii)] $0|a \Rightarrow a=0$.
\end{enumerate}
\begin{definition} An element $a$ of an integral domain $D$ is called a {\bf unit} if $a|1$. The set of units of $D$ is denoted by $U(D)$.
\end{definition}
\begin{center} {\bf Properties of Units}
\end{center}
\begin{enumerate}
\item[(i)] $\pm 1 \in U(D)$.
\item[(ii)] If $a \in U(D)$ then $-a \in U(D)$.
\item[(iii)] If $a \in U(D)$ then $a^{-1} \in U(D)$ because if we have $ab=1$, then $a^{-1}b^{-1}=1$
\item[(iv)] If $a \in U(D)$ and $b \in U(D)$ then $ab \in U(D)$.
\item[(v)] If $a \in U(D)$ then $\pm a^n \in U(D)$ for any $n \in Z$.
\end{enumerate}
\begin{example} $i$ is a unit in $\mathbb{Z}+\mathbb{Z}i$ because $i \cdot (-i)=1$.
\end{example}
\begin{theorem} If $D$ is an integral domain then $U(D)$ is an Abelian group with respect to multiplication.
\end{theorem}
\begin{proof}[\bf Proof] By property (iv) $U(D)$ is closed under multiplication. As $D$ is an integral domain, so multiplication is both associative and commutative. The identity element is $1 \in U(D)$. By property (iii) every element in $U(D)$ has an inverse.
\end{proof}
\begin{definition} Two non-zero elements $a$ and $b$ of an integral domain $D$d are associates, or said to be associated, if each divides the other. If $a$ and $b$ are associates, we write $a \sim b$, and if they are not associates, we write $a \not \sim b$.
\end{definition}
\begin{center} {\bf Properties of Associates}
\end{center}
Let $a,b,c \in D^*=D \backslash \{0\}$, where $D$ is an integral domain. Then:
\begin{enumerate}
\item[(i)] $a \sim a$.
\item[(ii)] $a \sim b \Rightarrow b \sim a$.
\item[(iii)] $a \sim b$ and $b \sim c$ implies $a \sim c$.
\item[(iv)] $a \sim b$ if and only if $ab^{-1} \in U(D)$.
\item[(v)] $a \sim 1$ if and only if $a$ is a unit.
\end{enumerate}
For (iv), suppose $ab^{-1} \in U(D)$ then we have some $c$ such that $ab^{-1}c=1$ and so $ac=b$ and $a|b$.
Similarly, $ba^{-1} \in U(D)$ as it is a group and so we have $b|a$.
Conversely, suppose $a \sim b$, then we can write $a=bc,b=ad$ for some $c,d$ and so
$cd=1$. Hence $ab^{-1}=c \in U(D)$. (v) is a consequence of (iv).
\begin{example} In $D=\mathbb{Z}+\mathbb{Z}i$ we have $1+i \sim 1-i$ as $\frac{1+i}{1-i}=i \in U(D)$.
\end{example}

\subsection{Irreducible and Primes}
\begin{definition} A non-zero, non-unit element $a$ of an integral domain $D$ is called an {\bf irreducible}, or said to be irreducible, if $a=bc$ where $b,c \in D$ implies either $b$ or $c$ is a unit.

A non-zero non-unit element which is not irreducible is said to be reducible.
\end{definition}
\begin{example} $2$ is irreducible in $\mathbb{Z}$, but is reducible in $\mathbb{Z}+\mathbb{Z}i$ because $2=(1+i)(1-i)$ and $1 \pm i$ are not units.
\end{example}
\begin{definition} A non-zero, non-unit element $p$ of an integral domain $D$ is called a prime if
$p | ab$, where $a,b \in D$ implies $p | a$ or $p|b$.
\end{definition}
\begin{example} $2$ is prime in $\mathbb{Z}$, but it is not prime in $\mathbb{Z}+\mathbb{Z}i$, because
$2|2i=(1+i)(1+i)$ but $2 \nmid 1+i$.
\end{example}
\begin{theorem} In any integral domain $D$,a prime is irreducible.
\end{theorem}
\begin{proof}[\bf Proof] Let $D$ be an integral domain and suppose that $p$ is a prime and $p=ab$. Then $p|p$ and so $p|ab$. By definition, $p|a$ or $p|b$. WLOG, we may assume that $p|a$ and so $a=pc$.

Then substitute this into $p=ab$ we have $p=pbc \Rightarrow bc=1$. Then $b$ is a unit and so $p$ is irreducible.
\end{proof}
\begin{remark} The converse is not true. For example, $2$ is irreducible in $\mathbb{Z}+\mathbb{Z}\sqrt{-5}$ but not prime. (see Exercise 2).
\end{remark}
But we have the following theorem which gives a class of integral domain in which every irreducible is prime
\begin{theorem} Let $D$ be an integral domain which has the following property:
Every quadratic polynomial in $D[X]$ having roots in the quotient field $F$ of $D$ is a product of
linear polynomials in $D[X]$.
Then every irreducible is prime.
\end{theorem}
\begin{proof}[\bf Proof] Let $p$ be an irreducible element in $D$, which is not prime. Then there exists $a,b \in D$ such that
$$p|ab, p \nmid a,b$$
Let $pr=ab$ and consider the quadratic polynomial
$$f(X)=pX^2-(a+b)X+r$$
In $F(X)$ we have
$$f(X)=p\left(X-\frac{a}{p}\right)\left(X-\frac{b}{p}\right)$$
Suppose that $f(X)=(cX+s)(dX+t)$ which are linear factors in $D[X]$. Then we have $cd=p$. As $p$ is irreducible, then one of $c$ and $d$ is unit, and WLOG, say $d$. So we have $c=d^{-1}p$. Then the roots of $f(X)$ in $F$ are $-\frac{ds}{p}$ and $-\frac{t}{d}$. But $-\frac{t}{d} \in D$, and so we know that one of the root is in $D$. But the two roots are $\frac{a}{p},\frac{b}{p}$ and so neither of them lies in $D$. Hence we have no such factorisation, which contradicts the assumption. Therefore, $p$ is prime.
\end{proof}

\subsection{Ideals}
\begin{definition} An {\bf ideal} $I$ of an integral domain $D$ is a non-empty subset of $D$ having the following properties:
\begin{enumerate}
\item[(i)] $a \in I, b \in I \Rightarrow a+b \in I$
\item[(ii)] $a \in I, r \in D \Rightarrow ra \in I$
\end{enumerate}
\end{definition}
It follows immediately from the definition that if $a_1,a_2,\ldots,a_n \in I$ then
$r_1a_1+\ldots r_na_n \in I$ for any $r_1,\ldots,r_n \in D$. $0$ is always in $I$ and most importantly,
if $1 \in I$ then $D=I$.
\begin{definition} An ideal $I$ of an integral domain $D$ is called a principal ideal if there exists an element $a \in I$ such that $I=\langle a \rangle$. The element $a$ is called a generator of the ideal $I$.
\end{definition}
If $D$ is an integral domain the principal ideal $\langle a \rangle$ generated by $a \in D$ is just the set $\{ra: r \in D\}$. The principle ideal $\langle 1 \rangle$ is $D$.
\begin{definition} An ideal $I$ of an integral domain $D$ is called a proper ideal of $D$ if $I \neq \langle 0 \rangle, \langle 1 \rangle$.
\end{definition}
\begin{remark}
The generator of the ideal is not necessarily unique. For example, in $D=\mathbb{Z}$, the ideal
$k\mathbb{Z}=\langle k \rangle=\langle -k \rangle$.
\end{remark}
\begin{theorem} Let $D$ be an integral domain and let $a,b \in D^*=D \backslash \{0\}$. Then
$$\langle a \rangle=\langle b \rangle \iff ab^{-1} \in U(D)$$
\end{theorem}
\begin{proof}[\bf Proof] Suppose $ab^{-1} \in U(D)$ then $a=bu$ for some $u \in U(D)$. Let $x \in \langle a\rangle$. Then $x=ac$ for some $c \in D$ and so $x=bcu \in \langle b \rangle$. Similarly,  we can write $b=a^{-1}u$ and so
for any $x \in \langle b \rangle, x \in \langle a \rangle$. Therefore $\langle a \rangle =\langle b \rangle$.

Conversely, suppose $\langle a \rangle = \langle b \rangle$, then $a=bc$ for some $c \in D$ and $b=ad$ for some $d \in D$. Hence $b=bd$, and as $b \neq 0$ we have $1=cd$ and so $c=ab^{-1} \in U(D)$.
\end{proof}
\subsection{Principal Ideal Domains}
\begin{definition} An integral domain $D$ is {\bf principal ideal domain} if every ideal in $D$ is principal.
\end{definition}
\begin{theorem} $\mathbb{Z}$ is a principal ideal domain.
\end{theorem}
\begin{proof}[\bf Proof] Let $I$ be a non-zero ideal of $\mathbb{Z}$. Let $a$ be the least positive element in $I$ (it must exist) and we shall prove that $I=\langle a \rangle$. Let $b \in I$, and so we can write
$b=ap+r$ with $0 \le r <a$. By definition of $a$, $r=0$ and so $b \in \langle a \rangle$. Therefore $I=\langle a \rangle$, and so it is a principal ideal domain.
\end{proof}
\begin{theorem} Let $D$ be a principal ideal domain. Then any irreducible element is prime.
\end{theorem}
\begin{proof}[\bf Proof] Let $p$ be irreducible in $D$ and suppose $p|ab$, with $a,b \in D$. Suppose $p \nmid a$,
then consider the ideal $I$ generated by $a$ and $p$. As $D$ is principal ideal domain, so we have some $d$ such that
$\langle d \rangle =\langle a,p \rangle$. Therefore, we have $x,y \in D$ such that $ax+yp=d$. Also we have $d|p$, and so we have some $c$ such that $p=cd$. But $p$ is irreducible, so $c$ or $d$ is unit.
Suppose $c$ is unit then $d=pc^{-1}$ and so $pc^{-1}=ax+py$ then we have $p|a$ which is a contradiction. Therefore, $d$ is a unit.

Then let $de=1$ and so $ax'+py'=1$, where $x'=xe,y'=ye$ and so
$$abx'+pby'=b$$
As $p|ab$, therefore we have $p|b$ and so $p$ is prime.
\end{proof}
Therefore, we conclude that:
\begin{theorem} In a principal ideal domain, an element is prime if and only if it is irreducible.
\end{theorem}
\begin{proof}[\bf Proof] Combine Theorem 7.20 and 7.30.
\end{proof}
\begin{definition} Let $D$ be a principal ideal domain and let $\{a_1,\ldots,a_n\}$ be a set of elements of $D$. Then the ideal $\langle a_1,\ldots,a_n \rangle$ is a principal ideal. A generator of this ideal is called a {\bf greatest common divisor} of $a_1,\ldots,a_n$.
\end{definition}
Let $D$ be a principal ideal domain. If $a$ and $b$ are greatest common divisors of $a_1,\ldots,a_n \in D$ then
$$\langle a \rangle = \langle a_1,\ldots,a_n \rangle =\langle b \rangle$$
So that $\frac{a}{b} \in U(D)$ by Theorem 7.27 (equivalently, $a \sim b$). We write $(a_1,\ldots,a_n)$ for a greatest common divisor of $a_1,\ldots,a_n$, understanding that $(a_1,\ldots,a_n)$ is only defined up to a unit. Note that $(a_1,\ldots,a_n)=(a_1,\ldots,a_{n-1})$ if $a_n=0$.

Further more, we can justify the definition by the following:
$$a \in \langle a \rangle =\langle a_1,\ldots,a_n \rangle$$
so that $\exists r,1,\ldots,r_n \in D$ such that
$$a=r_1a_1+\ldots r_n a_n$$ Thus, if $c \in D$ such that $c|a_j~\forall j$, then $c|a$, which justifies that $a$ is the greatest common divisor in the usual sense.
\begin{definition} The elements $a_1,\ldots,a_n$ are called relatively prime if $(a_1,\ldots,a_n)$ is a unit, and so $$\langle a_1,\ldots,a_n \rangle = \langle 1 \rangle =D$$
\end{definition}
\subsection{Maximal ideals and prime ideals}
\begin{definition} A proper ideal $M$ of an integral domain $D$ is called a {\bf maximal ideal} if whenever $I$ is an ideal of $D$ such that $M \subseteq I \subseteq D$, then $I=M$ or $I=D$.
\end{definition}
\begin{example} The ideal $\langle x^2+1 \rangle$ is maximal in $D=\mathbb{R}[x]$. To show shit, suppose we have an ideal $I$ which properly contains $\langle x^2+1 \rangle$, then we have some $f(x) \in I$ but $f(x) \not \in \langle x^2+1 \rangle$, apply division algorithm so we may write:
$$f(x)=(x^2+1)q(x)+r(x)$$ with the degree of $r(x)$ being $0$ or $1$ and we may assume in general that
$r(x)=ax+b$. Now $(x^2+1)q(x),f(x) \in I$, so $r(x)=f(x)-(x^2+1)q(x) \in I$. Then
$$a^2x-b^2=(ax+b)(ax-b)=r(ax-b) \in I$$
and
$$a^2(x^2+1) \in I$$
Therefore, we have $a^2+b^2 \in I$ but $a,b$ are not both $0$ and so $a^2+b^2$ is a unit $R[x]$. Hence $1 \in I$ and so $I=D$.
\end{example}
\begin{theorem} Let $D$ be an integral domain. Let $a \in D$ be a non-zero and non-unit element. If $\langle a\rangle$ is a maximal ideal, then $a$ is irreducible.
\end{theorem}
\begin{proof}[\bf Proof] Suppose that $a$ is not an irreducible element of $D$. Then, aa $a$ is neither $0$ nor a unit, it must be reducible and we may write $a=bc$ for some $b,c \in D$. Thus, we have
$$\langle a \rangle \subset \langle b \rangle \subset D$$
and so $a$ is not maximal.
\end{proof}
\begin{remark} The converse of Theorem 7.36 is not true. For example, $x$ is irreducible in $mathbb{Z}[x]$ but $\langle x \rangle$ is not maximal as we have
$$\langle x \rangle \subset \langle 2,x \rangle \subset \mathbb{Z}[x]$$
\end{remark}
\begin{theorem} Let $D$ be a principal ideal domain. Let $a \in D$ be a non-zero and non-unit element.
Then $\langle a \rangle$ is a maximal ideal if and only if $\langle a \rangle$ is irreducible in $D$.
\end{theorem}
\begin{proof}[\bf Proof] One direction follows from Theorem 7.36. For the other direction, suppose $a$ is irreducible but $\langle a \rangle$ is not a maximal ideal. Then we have some intermediate ideal $I$ between $\langle a \rangle$ and $D$ and we have some $b$ such that $I=\langle b \rangle$ because $D$ is a principal ideal domain. Therefore, we have $a=bc$ for some $c \in D$. But $a$ is irreducible and so one of $b$ and $c$ is unit. $b$ cannot be unit as we assume that $I$ is not $D$, so $c$ is unit. But then
$a \sim b$ and so $\langle a \rangle=\langle b \rangle=I$, which is a contradiction.
\end{proof}
\begin{theorem} Let $D$ be an integral domain and let $I$ be an ideal of $D$. Then $D/I$ is a field if and only if $I$ is maximal.
\end{theorem}
\begin{proof}[\bf Proof] Suppose $D/I$ is a field and let $I \subseteq J \subseteq D$. Then as $J$ is an ideal of $D$, it is a simple matter to check that $J/I \subseteq D/I$. But $D/I$ is a field and so the only ideals in $D/I$ is $0$ and $D/I$ itself. Clearly $J/I$ is not $0$ and so it must be $D/I$, and so $J=D$.

Conversely, suppose $I$ is a maximal ideal. Take any $b \not \in I$ and so $b+I \neq I$. Consider the set $$B=\{x \in D: x=by+w \text{ for some } y \in D \text{ and some } w \in I\}$$
Then it is clear that $B$ is an ideal of $B$. We can also show that $I \subset B$ (see Exercise 7). Since $I$ is maximal and so we have $B=D$ and so we have some $y \in D$ and $w \in I$ such that
$$1=by+w$$
Then $$(b+I)(y+I)=by+I=1-w+I=1+I$$ which has an inverse and so $D/I$ is a field.
\end{proof}
\begin{definition} A proper ideal $P$ of an integral domain $D$ is called a {\bf prime ideal} if
$$a,b \in D \text{ and } ab \in P \Rightarrow a \in P \text{ or } b \in P$$
\end{definition}
\begin{example} The principal ideal $I=\langle x^2+1 \rangle$ is not a prime ideal of $\mathbb{C}[x]$ as
$x \pm i \in \mathbb{C}[x],(x+i)(x-i)=x^2+1 \in I$ but $x \pm i \not \in I$.
\end{example}
\begin{theorem} Let $D$ be an integral domain and let $a$ be a non-zero non-unit element in $D$. Then
$\langle a \rangle$ is a prime ideal of $D$ if and only if $a$ is prime in $D$.
\end{theorem}
\begin{proof}[\bf Proof] Suppose that $\langle a \rangle$ is a prime ideal of $D$. Let $a|bc$ for some $b,c \in D$ and then $bc \in \langle a \rangle$, but $\langle a \rangle$ is prime and so either $b$ or $c$ is in $\langle a \rangle$, which means $a$ divides either $b$ or $c$.

Conversely, suppose $a$ is prime, and let $bc \in \langle a \rangle$. Then $a|bc$ and $a$ is prime, so $a|b$ or $a|c$, which means either $b$ or $c$ is in $\langle a \rangle$.
\end{proof}
\begin{theorem} Let $D$ be an integral domain and let $I$ be an ideal of $D$. Then
$D/I$ is an integral domain if and only if $I$ is prime.
\end{theorem}
\begin{proof}
Suppose $D/I$ is an integral domain, and let $ab \in I$. Then $(a+I)(b+I)=I$, and so $a+I$ or $b+I=I$, which means either $a$ or $b$ is in $I$.

Conversely, suppose $I$ is prime, and let $(a+I)(b+I)=I$. Then we have $ab \in I$ and so $a \in I$ or $b \in I$ which means either $a+I=I$ or $b+I=I$. Hence $D/I$ is an integral domain.
\end{proof}
\begin{theorem} Let $D$ be an integral domain. Let $I$ be a maximal ideal of $D$. Then $I$ is a prime ideal of $D$.
\end{theorem}
\begin{proof}[\bf Proof] $I$ is maximal, and so $D/I$ is a field, by theorem 7.39 and hence an integral domain. Then by Theorem 7.43.
\end{proof}
\begin{theorem} Let $D$ be a principal ideal domain and $I$ a proper ideal of $D$. Then $I$ is maximal if and only if $I$ is prime.
\end{theorem}
\begin{proof}[\bf Proof] One direction follows from Theorem 7.44. For the other direction, suppose we have some prime ideal $I$ which is not maximal in $D$. Then we have some $J$ such that
$$I \subset J \subset D$$ As $D$ is a principal ideal domain so we may assume, WLOG, that
$I=\langle a \rangle$ and $J=\langle b \rangle$. Then we have $b|a$ and so we have some $c$ such that
$a=bc$. But $a \in I$ and $I$ is prime so either $b$ or $c$ is in $I$. We assume that $J$ properly contains $I$ so that $b \not \in I$. Then $c \in I$ and so we have some $d$ such that $c=ad$. Therefore,
$$a=bc=abd$$ and so $bd=1$ which means that $b$ is a unit. Then $J=D$, which is a contradiction. Hence $I$ is maximal.
\end{proof}



\subsection{Sum and products of ideals}
\begin{definition} Let $I$ and $J$ be ideals of an integral domain $D$. The sum of $I$ and $J$, written
$I+J$ is defined by:
$$I+J=\{i+j: i \in I, j \in J\}$$
\end{definition}
It is easy to check that the sum of ideals is an ideal. Notice that, as $0 \in I$ we have
$I+ \langle 1 \rangle=\langle 1 \rangle$.
\begin{definition} Let $I$ and $J$ be ideals of an integral domain $D$. The product of $I$ and $J$, written $IJ$ is defined by:
\begin{eqnarray*}
I J&=&\{x \in D: x=i_1j_1+\ldots i_r j_r: \text{ for some } r \in N, \\
&~&\text{ some } i_1,\ldots,i_r \in I, \text{ some }  j_1,\ldots, j_r \in J\}
\end{eqnarray*}
and it is not hard to check this is an ideal of $D$.
\end{definition}
\begin{lemma} For any ideals $I,J,K$ of an integral domain $D$, we have
$$(I+J)K=IK+JK$$
\end{lemma}
\begin{proof}[\bf Proof] Take any element on LHS, it has the form
$$a_1k_1+\ldots a_r k_r$$ where $a_i=b_i+c_i$ with $b_i \in I,c_i \in J$.
It can be rearranged as
$$(b_1k_1+\ldots b_r k_r)+(c_1k_1+\ldots c_r k_r)$$ which has the form $IK+JK$ and so
$(I+J)K \subset IK+JK$.
Similarly we can show $IK+JK \subset (I+J)K$ and so they are equal.
\end{proof}
Similarly we can also show that, if $I= \langle a \rangle$ and $J= \langle b \rangle$ then
$IJ= \langle ab \rangle$.
\begin{theorem} Let $P$ be a proper ideal of an integral domain $D$. Then $P$ is a prime ideal if and only if for any two ideals $A$ and $B$ of $D$ with $AB \subset P$, we have either $A \subset P$ or $B \subset P$.
\end{theorem}
\begin{proof}[\bf Proof] Suppose we have such $P$ satisfying the above property, and let $ab \in P$. Then
the ideal generated by $ab$ lies in $P$ and hence
$$ \langle ab \rangle= \langle a \rangle  \langle b \rangle \subset P$$ and so
we have $ \langle a \rangle \subset P$ or $ \langle b \rangle \subset P$ and so
either $a \in P$ or $b \in P$.

Conversely, suppose we have some $A$ and $B$ such that $AB \subset P$ but $A,B \not \subset P$.
Then we have some $a \in A,b \in B$ such that $a,b \not \in P$. But $ab \in P$ and so $P$ is not prime.
\end{proof}
\begin{theorem} Let $D$ and $D_1$ be integral domain satisfying $D \subset D_1$. Let $P$ be a prime ideal of $D_1$ such that $P \cap D \neq \{0\},D$. Then $P \cap D$ is a prime ideal of $D$.
\end{theorem}
\begin{proof}[\bf Proof] We firstly check that $P \cap D$ is an ideal of $D$. Let $a,b \in P \cap D$. Then
as $P$ is an ideal of $D_1$, we have $a+b \in P$ and as $a,b \in D$ so $a+b \in D$. So $a+b \in P \cap D$.
Let $r \in D$. Then $ar \in D$. Also $r \in D \Rightarrow r \in D_1$, and as $P$ is an ideal of $D_1$ so we have $ar \in P$. Therefore, $ar \in P \cap D$. Hence $P \cap D$ is an ideal of $D$.

Then we check that $P \cap D$ is prime. Let $ab \in P \cap D$ and as $P$ is prime, so WLOG we assume that $a \in P$. But $a \in D$ so $a \in P \cap D$.
\end{proof}
\subsection{Exercises}
\begin{enumerate}
\item Show that $1+i$ is a prime in $\mathbb{Z}+\mathbb{Z}i$.
\item Show that in $\mathbb{Z}[\sqrt{-5}]$, $2$ is irreducible but not prime.
\item Show that the ideal $\langle 2,1+\sqrt{-7} \rangle$ is not principal in $\mathbb{Z}[\sqrt{-7}]$.
\item Let $A$ and $B$ be ideals of an integral domain $D$. Prove that $AB \subseteq A \cap B$.
\item Let $A$ and $B$ be ideals of an integral domain $D$. Show that
$$(A \cap B)(A+B) \subseteq AB$$
\item Let $p$ be a prime. Let $m$ be an integer with $m \le -(p+1)$. Prove that $p$ is irreducible in $\mathbb{Z}[\sqrt{m}]$.
\item Prove that the set $B$ defined in Theorem 7.39 is an ideal and $I \subset B$.
\item Let $P$ be a prime ideal of an integral domain $D$. Let $A_1,\ldots,A_k$ be ideals of $D$ such that $P \supseteq A_1\cdots A_k$. Prove that $P \supseteq A_i$ for some $i \in \{1,2,\ldots,k\}$.
\item Show that any finite integral domain is a field.
\item Let $D$ be an integral domain and $I, J$ be ideals of $D$ such that
$$I=\langle x_1,\ldots,x_m \rangle, J=\langle y_1,\ldots,y_n \rangle$$
Show that
$$IJ=\langle x_i y_j: 1 \le i \le m, 1 \le j \le n \rangle$$
(where some of the generators might be redundant.)
\end{enumerate}





\section{Algebraic background-Euclidean domain}
\subsection{Euclidean Domains}
\begin{definition} Let $D$ be an integral domain. A function $\phi$ is called an {\bf Euclidean function} on $D$ if it satisfies the following:
\begin{enumerate}
\item $\phi: D \rightarrow \mathbb{Z}$. (In other words, $\phi$ takes values in integers.
\item $\phi(ab) \ge \phi(a)$, for all $a,b \in D$ with $b \neq 0$.
\item If $a,b \in D$ with $b \neq 0$, then there exists $q,r \in D$ such that
$a=qb+r$ and $\phi(r) < \phi(b)$
\end{enumerate}
\end{definition}
\begin{example} Let $D=F[x]$, where $F$ is a field. Let $p(x) \in D$, then
\begin{equation*}
\phi(p(x))= \left\{
\begin{array}{ll}
deg(p(x)) & \text{if } p(x) \neq 0\\
-1 & \text{if } p(x)=0\\
\end{array} \right.
\end{equation*}
is a Euclidean function on $D$.
\end{example}
\begin{lemma} Let $D$ be an integral domain which possesses a Euclidean function $\phi$. Let $a,b \in D$, then:
\begin{enumerate}
\item[(i)] $a \sim b \Rightarrow \phi(a)=\phi(b)$.
\item[(ii)] $a|b$ and $\phi(a)=\phi(b) \Rightarrow a \sim b$.
\item[(iii)] $a \in U(D) \iff \phi(a)=\phi(1)$.
\item[(iv)] $\phi(a) >\phi(0)$ if $a \neq 0$.
\end{enumerate}
\end{lemma}
\begin{proof}
\begin{enumerate}
\item[(i)] We have $a=bp$ and $b=aq$ for some $p,q \in D$ and so we have
$$\phi(a)=\phi(bp) \ge \phi(b) \text{ and }\phi(b)=\phi(aq) \ge \phi(a)$$
   Therefore, $\phi(a)=\phi(b)$.
\item[(ii)] To show $a \sim b$, we have already got $a|b$, so we only need to check $b|a$. We can write $a=qb+r$ with $\phi(r) < \phi(b)$. Suppose $r \neq 0$, then $a|b \Rightarrow a|r$.
    and so$r=ac$ for some $c$. So $\phi(r) \ge \phi(a)=\phi(b)$, which is a contradiction.
\item[(iii)] If $a \in U(D)$, then $a \sim 1$ and so by (i), $\phi(a)=\phi(1)$. Conversely, if $\phi(a)=\phi(1)$, and clearly $1 |a$, so by (ii), $a \sim U(1)$ and so $a \in U(D)$.
\item[(iv)] We have $0=qa+r$ for some $q,r \in D$ with $\phi(r)<\phi(a)$. Suppose $r \neq 0$, then
$-qa=r$ and so $\phi(r) \ge \phi(a)$ which is a contradiction. Hence $r=0$ and so $\phi(0) <\phi(a)$.
\end{enumerate}
\end{proof}
\begin{definition} Let $D$ be an integral domain. If $D$ possesses a Euclidean function $\phi$, then $D$ is called a {\bf Euclidean domain} with respect to $\phi$.
\end{definition}
\begin{theorem} If $D$ is a Euclidean domain, then $D$ is a principal ideal domain.
\end{theorem}
\begin{proof}[\bf Proof] Let $\phi$ be the Euclidean function, and $I$ be an ideal in $D$. Let $a$ be the non-zero element such that $\phi(a)$ is the smallest integer greater than $\phi(0)$. Take any $b \in I$, then we may write $b=qa+r$ with $\phi(r)<\phi(a)$. But by the choice of $a$, we must have $r=0$. Therefore, $a|b$ for any $b \in I$ and so $I=\langle a \rangle$.
\end{proof}
\begin{remark} The converse of Theorem 8.5 is not true.
\end{remark}
Also, we shall extend the Euclidean algorithm in Chapter $1$ to every Euclidean domain. (In $\mathbb{Z}$, the Euclidean function is $\phi(a)=|a|$, and we can use exactly the same method to prove that Euclidean algorithm gives a way to calculate the greatest common divisor of two elements.
We shall see many examples of Euclidean domains in the exercises.
\subsection{Almost Euclidean domains}
\begin{definition} Let $D$ be an integral domain, a function $\phi:D \rightarrow \mathbb{N} \cup \{0\}$
is called an {\bf almost Euclidean function} on $D$ if it satisfies the following:
\begin{enumerate}
\item $\phi(0)=0$.
\item $\phi(a) >0$ for all $a \in D$ with $a \neq 0$.
\item $\phi(ab) \ge \phi(a)$.
\item if $a,b \in D$ with $b \neq 0$ then either $a=bq$ for some $q \in D$ or
$$0<\phi(ax+by)<\phi(b)$$ for some $x,y \in D$.
\end{enumerate}
\end{definition}
It is clear from the definition that if $\phi$ is a Euclidean function satisfying $\phi(0)=0$ then $\phi$ is an almost Euclidean function.
\begin{definition} Let $D$ be an integral domain. If $D$ possesses an almost Euclidean function $\phi$ then $D$ is called an {\bf almost Euclidean domain} with respect to $\phi$.
\end{definition}
\begin{theorem} An almost Euclidean domain is a principal ideal domain.
\end{theorem}
\begin{proof}[\bf Proof] Let $D$ be an almost Euclidean domain and $\phi$ be an almost Euclidean function. Let $I$ be an ideal of $D$ and take $b$ to be a non-zero element in $I$ such that $\phi(b)$ is the least positive integer. Let $a \in I$, then either $a=qb$ for some $q$ or we have
$$0<\phi(ax+by)<\phi(b)$$
But this is impossible because $ax+by \in I$, which contradicts the definition of $b$. So we have $b|a$ for any $a \in I$ and so $I=\langle a \rangle$.
\end{proof}
Now we see that an almost Euclidean domain is always a principal ideal domain. In many other context, the definition of Euclidean domain is different from the one we give, but the properties are almost the same.
\begin{definition} Let $D$ be an integral domain. A function $\phi:D \backslash\{0\} \rightarrow \mathbb{N}$ is called a {\bf Euclidean function} if it satisfies the following:
\begin{enumerate}
\item For all non-zero $a,b$ we have $\phi(a) \le \phi(ab)$.
\item  If $a$ and $b$ are in $D$ and $b$ is non-zero, then there are $q,r \in D$ such that $a = bq + r$ and either $r = 0$ or $f(r) < f(b)$.
\end{enumerate}
\end{definition}
It is clear that Definition 8.1 is a more general version, but with not much difference. Readers may use either definition where appropriate.
\subsection{Exercises}
\begin{enumerate}
\item Show that $\mathbb{Z}[\sqrt{-2}]$ is a Euclidean domain.
\item Show that $\mathbb{Z}[\sqrt{2}]$ is a Euclidean domain.
~\\
In the remaining questions, we are going to discuss some properties of the norm function.
Formally, Let $m$ be a square free integer, we define the norm $\phi: \mathbb{Q}[\sqrt{m}] \rightarrow \mathbb{Q}$ by:
    $$\phi(a+b\sqrt{m})=|a^2-mb^2|$$
    for all $a,b \in \mathbb{Q}$. (Use definition 8.1 in the following question).
\item Show that:
    \begin{enumerate}
    \item[(i)] $\phi(\alpha \beta)=\phi(\alpha)\phi(\beta)$ for all $\alpha,\beta \in \mathbb{Q}[\sqrt{m}]$.
    \item[(ii)] Let $\alpha \in \mathbb{Q}[\sqrt{m}]$. Then $\phi(\alpha)=0$ if and only if $\alpha=0$.
    \item[(iii)] $\phi(\alpha \beta) \ge \phi(\alpha)$ for all $\alpha,\beta \in \mathbb{Z}[\sqrt{m}]$ with $\beta \neq 0$.
    \end{enumerate}
\item Let $m \equiv 1$ (mod $4$). Show that
    \begin{enumerate}
    \item[(i)] $\phi: \mathbb{Z}[\frac{1+\sqrt{m}}{2}] \rightarrow \mathbb{N} \cup \{0\}$.
    \item[(ii)] $\phi(\alpha\beta) \ge \phi(\alpha)$ for all $\alpha,\beta \in \mathbb{Z}[\frac{1+\sqrt{m}}{2}]$ with $\beta \neq 0$.
    \end{enumerate}
\item Let $m$ be a square free integer. Then the integral domain $\mathbb{Z}[\sqrt{m}]$ is Euclidean with respect to $\phi$ if and only if for all $x, y \in \mathbb{Q}$, there exists $a.b \in \mathbb{Z}$ such that
    $$\phi((x+y\sqrt{m})-(a+b\sqrt{m}))<1$$
\item Using question 5, deduce that if $m$ is a negative square free integer, then $\mathbb{Z}[\sqrt{m}]$ is Euclidean with respect to $\phi$ if and only if $m=-1,-2$.
\item Let $m \equiv 1$ (mod $4$). Show that $\mathbb{Z}[\frac{1+\sqrt{m}}{2}]$ is Euclidean with
    respect to $\phi$ if and only if for all $x,y \in \mathbb{Q}$, there exist $a,b \in \mathbb{Z}$ such that
    $$\phi\left((x+y\sqrt{m})-\left(a+b\left(\frac{1+\sqrt{m}}{2}\right)\right)\right)<1$$
    Hence, prove that if $m$ is a negative square free integer, then $\mathbb{Z}[\frac{1+\sqrt{m}}{2}]$ is Euclidean with respect to $\phi$  only if $m=-3,-7,11$.\\
    (The converse is also true, and is not hard to check, but it takes some time).
\end{enumerate}




\section{Algebraic background-Noetherian domain}
\subsection{Noetherian domains}
Suppose we have an integral domain $D$, and let $a_i$ be non-zero elements.
Let $A_i$ be the ideals generated by $\{a_1,\ldots,a_i\}$. Clearly, we have
$$A_1 \subseteq A_2 \ldots \subseteq A_n \subseteq \ldots$$
Is it going to stop at some stage? Or it goes on to infinity?
\begin{definition} An infinite sequence of ideals $\{I_n: n=1,2,\ldots\}$ in an integral domain is said to be an {\bf ascending chain} if
$$I_1 \subseteq I_2 \subseteq \ldots \subseteq I_n \subseteq \ldots$$
The chain is said to be a strictly ascending chain if
$$I_1 \subset I_2 \subset \ldots \subset I_n \subset \ldots$$
\end{definition}
\begin{definition} An ascending chain of ideals
$$I_1 \subseteq I_2 \subseteq \ldots \subseteq I_n \subseteq \ldots$$
in an integral domain is said to {\bf terminate} if there exists a positive integer $n_0$ such that
$$I_n = I_{n_0} \text{ for all } n \ge n_0$$
\end{definition}
\begin{definition} An integral domain $D$ is said to satisfy the {\bf ascending chain condition} if every ascending chain of ideals in $D$ terminates. Equivalently, $D$ does not contain a strictly ascending chain of ideals.
\end{definition}
\begin{definition} An integral domain which satisfies the ascending chain condition is called a
{\bf Noethetian domain}. More generally, we define a Noetherian ring to be a ring $R$ in which every ascending chain of ideals in $R$ terminates.
\end{definition}
\begin{example} $\mathbb{Z}$ is a Noetherian domain because it is a Euclidean domain and hence principal ideal domain. Also $\langle a \rangle \subseteq \langle b \rangle$ if and only if $b|a$, and if
$$I_1 \subseteq I_2 \subseteq \ldots \subseteq I_n \subseteq \ldots$$
we have only finitely many divisors of $I_1$ and so the chain must terminate.
\end{example}
\begin{theorem} Let $D$ be an integral domain. Then $D$ is Noetherian if and only if every ideal of $D$ is finitely generated.
\end{theorem}
\begin{proof}[\bf Proof] Let $D$ be a Noetherian domain. Suppose we have some ideal $I$ in $D$ which is not finitely generated and we may pick any non-zero $a_1 \in I$ and let $I_1$ in $D$. For $n \ge 2$, having got $I_{n-1}$, we pick $a_n$ to be the element in $I$ such that $a_n \not \in I_{n-1}$ and let
the ideal $I_n$ contains both $I_{n-1}$ and $a_n$. In other words, $I_n=\langle a_1,\ldots,a_n \rangle$.
Thus we have
$$I_1 \subset I_2 \subset \ldots \subset I_n \subset \ldots$$
which is an strictly ascending chain and so $D$ is not Noetherian, which is a contradiction.

Conversely, suppose every ideal in $D$ is finitely generated. Then let
$$I_1 \subseteq I_2 \subseteq \ldots \subseteq I_n \subseteq \ldots$$
be an ascending chain of ideals in $D$. Let $I=\bigcup_{i=1}^\infty I_i$ and it is easily checked that this is an ideal of $D$ and by assumption this is finitely generated, say by $a_1,\ldots,a_N$, and let
$a_i \in I_{n_i}$ and set $n_0= \text{max}_i(n_i)$. Then for any $n \ge n_0$, we have
$$I=\langle a_1,\ldots,a_N \rangle \subseteq I_n \subseteq I$$ and so $I_n=I=I_{n_0}$ for all $n \ge n_0$.
\end{proof}
\begin{theorem} Let $D$ be a principal ideal domain, then $D$ is a Noetherian domain.
\end{theorem}
\begin{proof}[\bf Proof] This follows immediately from Theorem 9.6 as every ideal is generated by one element.
\end{proof}
\begin{definition} An integral domain $D$ is said to satisfy the {\bf maximal condition} if every non-empty set $S$ of ideals in $D$, there exists an ideal $I$ such that for all $J \in S$, if $I \subseteq J$, then $I=J$.
\end{definition}
\begin{theorem} Let $D$  be an integral domain. Then $D$ is Noetherian if and only if it satisfies the maximal condition.
\end{theorem}
\begin{proof}[\bf Proof] Suppose $D$ does not satisfy the maximal condition. Then we have some set $S$ of ideals in $D$ and for each $I \in S$, we have some $J \in S$ such that $I \subset J$. Then we may construct an infinite strictly increasing chain and so $D$ is not Noetherian.

Conversely, let $D$ be an integral domain which satisfies the maximal condition. Let
$$I_1 \subseteq I_2 \subseteq \ldots$$
be an ascending chain of ideals in $D$. Let $S$ be the set which contains every $I_n$.  Then $S$ contains some $I_{n_0}$ such that if $I_{n_0} \subseteq I_j$ then $I_{n_0}=I_j$. Therefore, we have
$I_n=I_{n_0}$ for all $n \ge n_0$.
\end{proof}
We end the subsection by a famous theorem:
\begin{theorem}{\bf Hilbert Basis Theorem}\label{H;Hilbert}
Let $R$ be a Noetherian ring, then $R[x]$ is also Noetherian.
\end{theorem}
\begin{proof}
\end{proof}
\begin{corollary} Let $R$ be a Noetherian ring, then $R[X_1,X_2,\ldots]$ is also Noetherian.
\end{corollary}
\begin{proof}[\bf Proof] We apply induction on $n$. For $n=1$, use Theorem 9.10. Suppose true for $n-1$, $n \ge 2$, then $R[X_1,\ldots,X_n]=R[X_1,\ldots,X_{n-1}][X_n]$ and we know $R[X_1,\ldots,X_n]=R[X_1,\ldots,X_{n-1}]$ is Noetherian, and so by the case when $n=1$, we have $R[X_1,\ldots,X_n]=R[X_1,\ldots,X_{n-1}][X_n]$ Noetherian.
\end{proof}
\subsection{Factorisation domain and unique factorisation domains}
Let $D$ be a field. Then $D$ does not contain any irreducible element as every non-zero element is a unit. What about integral domain?
Does every integral domain which is not a field necessarily have at least one irreducible element?
We will see (exercise 4) some examples of integral domain, which is not a field, yet does not contain any irreducible element. But we have:
\begin{theorem} Let $D$ be an Noetherian domain, which is not a field. Then $D$ contains elements that are irreducible.
\end{theorem}
\begin{proof}[\bf Proof] Suppose not. $D$ contains no irreducible element. As we assume that $D$ is not a field, so $D$ contains non-zero, non-unit elements. Let $a$ be one of these. Then $a$ is reducible, say
$a=a_1 b_1$. By assumption, $a_1$ is not irreducible and is not unit. So $a_1$ is reducible, say
$a_1=a_2 b_2$ and so on. This process will never terminate. Also $a \nmid a_1$ because $a_2$ is not unit. Then we have a strictly ascending chain
$$\langle a \rangle \subset \langle a_1 \rangle \subset \langle a_2 \rangle \ldots$$
which is a contradiction as $D$ is Noetherian.
\end{proof}
\begin{definition} Let $D$ be an integral domain. Then $D$ is called a {\bf factorisation domain} if every non-zero, non-unit element of $D$ can be expressed as a finite product of irreducible elements of $D$.
\end{definition}
\begin{theorem} Let $D$ be a Noetherian domain. Then $D$ is a factorisation domain.
\end{theorem}
\begin{proof}[\bf Proof] Suppose $D$ is Noetherian. Take any non-zero, non-unit element $a$. If $a$ is irreducible, then we are done. If not, we may write $a=a_1b_1$, both of which are non-unit. If $a_1,b_1$ can both be written as a finite product of irreducibles, so is $a$. If not, say $a_1$ cannot be written as a finite product of irreducibles, then $a_1$ itself is not irreducible and $a_1=a_2 b_2$. Notice that $a \nmid a_1$ as $b_1$ is not unit. Continue this and so at each stage if we assume $a_i$ is not a finite product of irreducibles, then the process will not terminate and
$$\langle a \rangle \subset \langle a_1 \rangle \subset \langle a_2 \rangle \ldots$$
is a strictly ascending chain, which contradicts $D$ being Noetherian. Therefore, the original assumption for $a_1,b_1$ is not true and so $a$ is a finite product of irreducibles.
\end{proof}
\begin{theorem} Let $D$ be a principal ideal domain. Then $D$ is a factorisation domain.
\end{theorem}
\begin{proof}[\bf Proof] If $D$ is a principal ideal domain, then it is Noetherian and so is a factorisation domain by Theorem 9.14.
\end{proof}
In $\mathbb{Z}$, we know that every positive integer greater than $1$ can be written uniquely as a product of prime numbers. Is this necessarily true in any factorisation domain? Is every factorisation necessarily unique? Here is an example
\begin{example} In $\mathbb{Z}[\sqrt{-5}]$ we have $6=2\cdot3=(1+\sqrt{-5})(1-\sqrt{-5})$ and
$2,3,1 \pm \sqrt{-5}$ are all irreducibles by considering the norm.
\end{example}
Thus, we may introduce the following concept:
\begin{definition} Let $D$ be a factorisation domain. Suppose that every non-zero, non-unit element in $D$ can be uniquely (up to modulo units) expressed into a finite product of irreducibles, then $D$ is called a
{\bf unique factorisation domain}.
\end{definition}
\begin{theorem} Let $D$ be a principal ideal domain. Then $D$ is a unique factorisation domain.
\end{theorem}
\begin{proof}[\bf Proof] Suppose $D$ is a principal ideal domain, then it is a factorisation domain by Theorem 9.15. So we want to show the factorisation is unique. Let $a=p_1 p_2\ldots p_n =q_1q_2\ldots q_m$ be factorisations into irreducibles. Then by theorem 7.30, any irreducible element in a principal ideal domain is prime. Thus we have $p_1 |q_1q_2\ldots q_m$ and so $p_1 | q_j$ for some $j$.
But $q_j$ is irreducible and so writing $q_j=p_1 u_j$ we have $u_j \in U(D)$ and then we may rearrange $q_i$ so that $q_j$ is $q_1$ above. Hence
$$p_2p_3\ldots p_n=q_2q_3\ldots q_m u_1$$
Continue this, and so we must have $m=n$ because, if not, say $n>m$, we will have
$$1=u_1\ldots u_n q_{n+1}\ldots q_m$$ which is impossible because this implies that $q_j$ is unit.
Then $m=n$ and $p_i \sim q_i$ and so the factorisation is unique up to modulo units.
\end{proof}
\begin{remark} The converse is not true. For example, $\mathbb{Z}[X]$ is a unique factorisation domain but it contains $\langle 2,X \rangle$ which is not principal.
\end{remark}
\begin{theorem} Let $D$ be a unique factorisation domain. Then an element of $D$ is irreducible if and only if it is prime.
\end{theorem}
\begin{proof}[\bf Proof] A prime is always irreducible by Theorem 7.20. Now let $p$ be irreducible in $D$. Let $p|ab$ then we may write $ab=pc$ for some $c \in D$. Then as $D$ is a unique factorisation domain, we may write
$$a=a_1a_2 \ldots a_k,b=b_1b_2\ldots b_m,c=c_1c_2\ldots c_n$$ where the factorisation is unique up to modulo units. Then we must have $p \sim a_i$ or $p \sim b_j$ for some $a_i$ or $b_j$ because $p$ is irreducible. WLOG, say $p \sim a_i$, then $p|a_i$ and $a_i|a$ so we have $p|a$ and hence $p$ is prime.
\end{proof}

What does a factorisation tell us? Consider $a_1,\ldots,a_n$ non-zero elements in a unique factorisaiton domain. Let $\{p_1,\ldots,p_k\}$ be a set of irreducibles such that:
\begin{enumerate}
\item Each $p_i$ divides at least one of $a_1,\ldots,a_n$.
\item $p_i \not \sim p_j$.
\item If $p$ is any irreducible which divides at least one of $a_1,\ldots,a_n$, then $p \sim p_i$ for some $i$.
\end{enumerate}
In other words, these are all possible irreducible factors of $a_1,\ldots,a_n$ modulo units.
Then we have, for each $i$, that
$$a_i=u_i \prod_{j=1}^{k} p_i^{e_{ij}} \text{ where } e_{ij} \ge 0$$
and let $e_j= \min_{1 \le i \le n}e_{ij}, j=1,2\ldots,k$. Define
$$a=\prod_{j=1}^k p_j^{e_j}$$
By construction, we have $a | a_i$ for all $i$, and if $b|a_i$ for all $i$, we have
$b|a$. Thus we shall see that $a$ is the greatest common divisor of $a_1,\ldots,a_k$, and since
the factorisation is defined up to modulo units, so the greatest common divisor is also unique up to
modulo units.

In a principal ideal domain $D$, a greatest common divisor $a$ of $a_1,\ldots,a_k$ is always
a linear combination of $a_1,\ldots,a_k$ with coefficients in $D$ (This is because the generator of the ideal containing these elements is the greatest common divisor). But this is not necessarily the case in a unique factorisation domain.
\begin{example} Let $D=\mathbb{Z}[X]$, which is a unique factorisation domain. The greatest common divisor of $2x$ and $3x^2$ is $x$ but we are not able to write $x$ as a linear combination of $2x$ and $3x^2$ with coefficients in $\mathbb{Z}[X]$.
\end{example}
\begin{theorem} A principal ideal domain is an almost Euclidean domain.
\end{theorem}
\begin{proof}[\bf Proof] Let $D$ be a principal ideal domain. Then it is a unique factorisation domain by Theorem
9.18. We shall define the function $\phi$ as follows and check it is an almost Euclidean function.
\begin{equation*}
\phi(a)= \left\{
\begin{array}{ll}
0 & \text{if } a=0\\
1 & \text{if } a \in U(D)\\
2^n &\text{if } a \not \in U(D) \text{ and } a=p_1\ldots p_n \text{ where } p_1,\ldots,p_n \text{ are irreducibles}
\end{array} \right.
\end{equation*}
It is clear that $\phi$ is well-defined because the factorisation is unique up to modulo units and it is easy to check that $\phi(ab)=\phi(a)\phi(b)$ for any $a,b \in D$. Therefore, $\phi(ab) \ge \phi(a)$ if
$b \neq 0$. So it remains to check that for each $a,b \in D$, $b \neq 0$, one of the following holds:
\begin{enumerate}
\item $a=bq$ for some $q \in D$
\item There exists $x,y \in D$ such that
$$0<\phi(ax+by)<\phi(b)$$
\end{enumerate}
Take any $a,b \in D$ with $b \neq 0$ and let $I=\langle a,b \rangle=\langle c \rangle$ for some $c$ because $D$ is a principal ideal domain. Suppose $a \neq bq$ for any $q \in D$, then $I \neq \langle b\rangle$.
Since $b \in I$, so $b=cd$ for some $d \not \in U(D)$ because $I \neq \langle b\rangle$. Thus we have
$$\phi(b)=\phi(cd)=\phi(c)\phi(d)>\phi(c)$$
But $c \in I$ and so there exist $x,y$ such that $ax+by=c$ and so we have $x,y$ such that
$$0<\phi(ax+by)=\phi(c)<\phi(b)$$
\end{proof}
\begin{theorem}{\bf [Greene's Theorem]}\label{G;Greene} An integral domain is a principal ideal domain if and only if it is an almost Euclidean domain.
\end{theorem}
\begin{proof}[\bf Proof] Combine Theorem 8.9 and Theorem 9.22
\end{proof}
\subsection{Exercises}
\begin{enumerate}
\item
\end{enumerate}




\section{Algebraic background-Modules}
\subsection{Modules}
\begin{definition} Let $R$ be a ring with identity and $M$ and additive Abelian group. A function
$f: R \times M \rightarrow M$ is called an {\bf $R$-action} on $M$ if it has the following properties:
\begin{enumerate}
\item $f(r+s,m)=f(r,m)+f(s,m)$.
\item $f(r,m+n)=f(r,m)+f(r,n)$.
\item $f(r,f(s,m))=f(rs,m)$.
\item $f(1,m)=m$.
for all $r,s \in R$ and $m,n \in M$.
\end{enumerate}
\end{definition}
\begin{definition} Let $R$ a ring with identity. An additive Abelian group $M$ together with an $R$-action on $M$ is called an {\bf $R$-module}.
\end{definition}
In fact, what we defined above is a left $R$-module since we can see the action is on the left. We can define the right $R$-module in a similar notion. Unless otherwise stated, the $R$-action in this book will be defined on the left. To make the notation simpler, we may write $rm$ as $f(r,m)$.
\begin{lemma} $(-r)m=-(rm)=r(-m)$
\end{lemma}
\begin{proof}[\bf Proof] $(-r)m+rm=(-r + r )m=0 \cdot m=0$. So $(-r)m=-(rm)$. Similarly,
$r(-m)=r(-1 \cdot m)=(r \cdot (-1))m=(-r)m$
\end{proof}
\begin{example} Let $F$ be a field. Then an $F$-module is the same as a vector space over $F$.
\end{example}
\begin{example} Any additive Abelian group $A$ can be thought of as a $\mathbb{Z}$-module in a nartural way. The $\mathbb{Z}$-action on $A$ is just the map $(n,a) \rightarrow na$ from $\mathbb{Z} \times A$ to $A$.
\end{example}
\begin{example} Any ring $R$ with identity can be thought of as a module over itself in a natural way. The $R$-action is just the mao $(a,b) \rightarrow ab$ from $R \times R$ to $R$.
\end{example}
\begin{definition} Let $M$ be an $R$-module. A subgroup $N$ of $M$ is called a {\bf submodule} of $M$ if $rn \in N$ for all $r \in R$, $n \in N$.
\end{definition}
\begin{example} Let $V$ be a vector space over $F$ and so $V$ is an $F$-module. Then any subspace of $V$ is a submodule of $V$.
\end{example}
\begin{definition} If $X$ is a subset of an $R$-module M then the submodule generated by $X$ is the smallest submodule of $M$ containing $X$. The definition is well-defined because any intersection of submodules is still a submodule.
\end{definition}
\begin{lemma} Let $X$ be a non-empty subset of $M$. The submodule of $M$ generated by $X$ has the form
$$\left\{\sum_{i=1}^n r_i x_i: r_i \in R, x_i \in X, n \ge 1\right\}$$
\end{lemma}
\begin{proof}[\bf Proof] This is clearly a submodule of $M$ and it contains the set $X$. Now Let $Y$ be the submodule which contains $X$. Then it contains every element of the form
$$\sum_{i=1}^n r_i x_i: r_i \in R, x_i \in X, n \ge 1$$ because $Y$ is a submodule. Therefore the submodule generated by $X$ has the above form.
\end{proof}
\begin{definition} An $R$-module $M$ is called {\bf finitely generated} if $M$ is generated by some finite set of elements of $M$.
\end{definition}

\section{Field extension}
Now we can start the algebraic topics of Number theory. Why is algebraic number theory helpful? What is the main difference between algebraic number theory and analytic number theory? We shall begin with the following example:\\
Find all integer solutions to the equation:
$$x^2 + 2 = y^3$$
It is not hard to see (probably after staring the equation for seconds) that $x=\pm 5, y=3$ are both solutions. Are there any other solution? Readers might have tried lots of other values of $x$ and $y$ and may be convinced numerically that $x=\pm5, y=3$ are the only solutions. How can we prove this?
The algebraic number theory studies the structure theorem about the ring of integers (which we will define later) and use those to find the solution to equations such as $y^3=x^2+a$ etc.
This section is intended as an introduction to some algebraic concept in number theory.
Any basic ring theory will be assumed.
\subsection{Basic properties}
\begin{definition} When a subring $K$ of a field $F$ is a field. We call $K$ a subfield of $F$ and $F$ an extension of $K$. Usually, the pair of $K$ and its extension $F$ is called an extension $F/K$. A field $L$ satisfying
$$K \subset L \subset F$$ is called an intermediate field of the extension $F/K$, and $L/K$ is called a subextension of $K$.
\end{definition}
\begin{definition} Let $F/K$ be a field extension. The {\bf degree} of the extension is the dimension of $F$ as a $K$-vector space and is denoted by
$[F:K]$. When $[F:K]=n$ for some $n \in \mathbb{N}$, we say $F/K$ is a {\bf finite} extension (of degree $n$), and when $[F:K]=\infty$, we say $F/K$ is an infinite extension.
\end{definition}
\begin{remark} It is clear that $[F:K]=1$ if and only if $F=K$.
\end{remark}
\begin{proposition}{\bf [Tower Law]}\label{T;Tower} Let $K \subset L \subset F$. Then $[F:K]=[F:L][L:K]$.
\end{proposition}
\begin{proof}[\bf Proof] This is clear if at least one of $[F:L]$ or $[L:K]=\infty$. So we may assume that both of them are finite, say $[F:L]=m,[L:K]=n$. Then we may pick an $L$-basis $\{l_1,\ldots,l_m\}$ for $F$ and a $K$-basis
$\{k_1,\ldots,k_n\}$ for $L$. Let $S=\{l_i k_j: 1 \le i \le m, 1 \le j \le n\}$ and we check that $S$ is a $K$-basis for $F$ and so the result follows.\\
Linear independence: Let $\sum_{i,j}\lambda_{i,j}l_i k_j=0$, and so we have
$\sum_{i}\lambda l_i \sum_j \lambda_{i,j} k_j=0$. As $\{l_1,\ldots,l_m\}$ is linearly independent, so for each $i$ we
have
$$\sum_j \lambda_{i,j} k_j=0$$ which implies $\lambda_{i,j}=0$ for each $i,j$ because $\{k_1,\ldots,k_n\}$ is linearly independent.\\
Spanning: Let $f \in F$ and so we have some $\lambda_i \in L$ such that
$$f=\sum_{i=1}^m \lambda_i l_i$$
For each $i$, we have $\lambda_{i,j}$ such that
$$\lambda_i =\sum_{j=1}^n \lambda_{i,j}k_j$$
and hence
$$f=\sum_{i=1}^m\sum_{j=1}^n \lambda_{i,j}l_i k_j$$.
\end{proof}
\begin{proposition} Let $P$ be a monic irreducible polynomial over a field $K$. Then $[K_P:K]=\deg{P}$.
\end{proposition}
\begin{proof}[\bf Proof] Let $d=\deg{P}$ and $S=\{1,\bar{x},\ldots,\bar{x}^{d-1}\}$ where $\bar{f}$ means $f$ modulo $P$.  the set $S$ is a $K$-basis for $K_p$.
\end{proof}
\subsection{K-Homomorphism}
This short subsection will introduce the concept of $K$-homomorphism.
\begin{definition} Let $F,F'$ be two field extensions of $K$. A ring homomorphism
$$f: F \rightarrow F'$$ is a {\bf $K$-homomorphism} if it satisfies $f|_K=id$.

The set of all $K$-homomorphism from $F$ to $F'$ is denoted by Hom$_K(F,F')$. A bijective $K$-homomorphism is called a $K$-isomorphism. We say $F$ and $F'$ are $K$-isomorphic if there exists a $K$-isomorphism between $F$ and $F'$ and write $F \cong F'$.
In particular, a $K$-isomorphism $F \rightarrow F$ is called a $K$-automorphism of $F$, and the group consisting
of all $K$-automorphisms of F is denoted by $Aut_K(F)$.
\end{definition}
\begin{example} If $F \cong F'$, then $[F:K]=[F':K]$. because if $S$ be a $K$-basis for $F$ and let $f$ be a $K$-isomorphism from $F$ to $F'$. Then it is easy to check that
$f(S)$ is a $K$-basis for $F'$.
\end{example}
\begin{remark} We know that the kernel $\ker{f}$ is always an ideal of $F$ and so it must be $0$ or $F$. But $1 \in K$ and $f(1)=1$ so that every $K$-homomorphism $f$ is injective. And the image is a subextension of $F'/K$.
\end{remark}
\begin{lemma} If $[F:K]=[F':K]$ then any $K$-homomorphism from $F$ to $F'$ must be a $K$-isomorphism.
\end{lemma}
\begin{proof}[\bf Proof] Let $f$ be any $K$-homomorphism from $F$ to $F'$ and so we know $f$ is injective. But as $K$-vector spaces,
the dimension of $F$ is the same as the dimension of $F'$. Then by {\bf Rank-Nullity} of Linear Algebra, $f$ must be surjective. Hence it is an isomorphism.
\end{proof}
\subsection{Algebraic extension}
\begin{definition} Let $F$ be a field and $K$ be a subfield of $F$. An element $\alpha$ of $F$ is called an {\bf algebraic element} over $K$ or said to be {\bf algebraic} over $K$ if there exists a polynomial $f(x) \in K[x]$ such that $f(\alpha)=0$. An element $\alpha$ is said to be {\bf transcendental} over $K$ if it is not algebraic.
\end{definition}
\begin{definition} Let $F/K$ be a field extension and $\alpha \in F$. The {\bf substitution map} $f_\alpha$ is defined as $$f_\alpha: K[x] \ni P(x) \rightarrow P(\alpha) \in F$$
\end{definition}
Then in the sense of field extension, an element $\alpha$ is said to be algebraic if we have the substitution map $f_\alpha$ with $\ker{f_\alpha} \neq 0$ and transcendental if $\ker{f_\alpha}=0$.
\begin{definition} Let $F/K$ be a field extension. We say $F/K$ is algebraic if $\alpha$ is algebraic for every $\alpha \in F$.
\end{definition}
\begin{proposition} Finite extensions are algebraic.
\end{proposition}
\begin{proof}[\bf Proof] Let $F/K$ be a finite extension. Pick any $\alpha \in F$, and consider the substitution map $f_\alpha$.
It is clear that $f_\alpha$ is linear. But the dimension of $K[x]$ over $K$ is infinite, while the dimension of $F$ over $K$ is finite. By Rank-Nullity, we have $\ker{f} \neq 0$.
\end{proof}
Let $F$ be a field and $K$ be a subfield of $F$. We know that if $K$ is a field then clearly $K[x]$ is a Euclidean domain and hence a principal ideal domain. Let $\alpha$ be algebraic over $K$. Consider the set $$I_K(\alpha)=\{f(x) \in K[x]: f(\alpha)=0\}$$
It is easy to check that $I_K(\alpha)$ is an ideal of $K[x]$ and thus is generated by a single element. The generator is unique up to modulo units because we know that if $a \sim b$, then $\langle a \rangle=
\langle b \rangle$. As $K$ is a field, so we may assume the generating polynomial is monic.
Therefore we may introduce the following concept:
\begin{definition} Let $F/K$ be a field extension. Let $\alpha \in F$ be algebraic over $K$.
The {\bf minimal polynomial} of $\alpha$ over $K$ is the unique monic polynomial $p(x) \in K[x]$ such that
$$\langle p(x) \rangle =I_K(\alpha)$$
and is denoted by $\min_{K,\alpha}(x)$.
\end{definition}
Equivalently, in the sense of field extension, the minimal polynomial $\min_{K,\alpha}(x)$ is defined by the generator of the ideal $\ker{f_\alpha}$
\begin{lemma} Let $F/K$ be a field extension. Let $\alpha \in F$ be algebraic over $K$. Then
$\min_{K,\alpha}(x)$ is irreducible in $K[x]$.
\end{lemma}
\begin{proof}[\bf Proof] Suppose not, say $\min_{K,\alpha}(x)=f(x)g(x)$ where $f(x),g(x) \in K[x]$. Then
As $min_{K,\alpha}(\alpha)=0$, we have $f(\alpha)$ or $g(\alpha)=0$, and WLOG, say $f$. Thus we have
$f|min_{K,\alpha}(x)$ but $\min_{K,\alpha}(x) \nmid f$ and so the minimal polynomial is $f$, which is a contradiction.
\end{proof}
\begin{definition}
Let $K$ be a field and let $P$ be a monic irreducible polynomials in $K[X]$ and so the ideal
$\langle P \rangle$ is maximal. Hence the quotient ring
$$K_P=K[X]/\langle P \rangle$$ is a field. We say $K_P$ is an field extension of $K$ by {\bf adjoining} a root of P and we write $K \subset K_P$.
\end{definition}
Equivalently,
\begin{definition} Let $F/K$ be a field extension. Let $\alpha \in F$ be algebraic over $K$. The subextension Im$f_\alpha$ of $F/K$ is denoted by $K(\alpha)$, and called the field generated by $\alpha$ over $K$. A finite extension $F/K$ is called a simple extension if $F=K(\alpha)$ for some $\alpha \in F$. In this case we have:
$$K_P \cong F$$ for $P=\min_{K,\alpha}(x)$.
\end{definition}
We shall extend Definition 11.17 as follows:
\begin{definition} Let $F/K$ be a field extension and $\alpha_1, \ldots,\alpha_n \in F$ be algebraic over $K$. We define the subextension $K(\alpha_1,\ldots,\alpha_n)$ inductively by:
$$K_0=K; K_{i+1}=K_i(\alpha_{i+1}), 0 \le i \le n-1; K_n=K(\alpha_1,\ldots,\alpha_n)$$
This is the smallest subfield of $F$ which contains $\alpha_1,\ldots,\alpha_n$. The definition is independent of the ordering of $\alpha_i$.
\end{definition}
\begin{proposition} For $F/K$ a finite extension, there exists $\alpha_1,\ldots,\alpha_n$ such that
$F=K(\alpha_1,\ldots,\alpha_n)$.
\end{proposition}
\begin{proof}[\bf Proof] We can find these $\alpha_i$ inductively by the following procedure. Let $\alpha_1 \in F$ be any element not in $K$. Define $K_1=K(\alpha_1)$. For $i \ge 1$, having defined $K_i$, let $K_{i+1}=K_i(\alpha_{i+1})$ for some $\alpha_{i+1} \in F$ not in $K_i$. This process must terminate because by Tower Law (Proposition 11.4) we have
$$[K_n:K]=[K_n:K_{n-1}]\ldots [K_1:K]$$ and each $[K_i:K_{i-1}] \ge 2$ by construction. Therefore, as $F/K$ is finite, the process will terminate.
\end{proof}
\subsection{Automorphism group}
Let $F/K$ be a field extension and $\alpha \in F$ be algebraic over $K$ and let $P=\min_{K,\alpha}(x)$ be its minimal polynomial. Let $\sigma \in \text{Hom}_K(K_P,F)$ be a $K$-homomorphism. What can $\sigma(\alpha)$ be?

Consider $f(x) \in K[x]$ be any polynomials with coefficients in $K$. Then as $\sigma$ is a ring homomorphism and identity on $K$, we have
$$\sigma(f(\alpha))=f(\sigma(\alpha))$$
Thus, if we take $f=P$, then we have $P(\alpha)=0$ and $\sigma(0)=0$. Therefore, we have
$$P(\sigma(\alpha))=0$$ In other words, $\sigma(\alpha)$ must also be a root of the minimal polynomial $P$.
Therefore, $\sigma$ may be viewed as a permutation of the roots of $P$.
Also, the structure of $K_P$ does not depend on which root of $P$ we took in the first place and $K_P \cong K(\alpha)$. Hence, the field extension $K_P/K$ is generated by a single element (some root of $P$) and therefore the homomorphism
$\sigma$ is determined by the image of the generator.
\begin{proposition} Let $R(P)$ be the set of roots of $P$. Then the following maps are inverse to each other:
$$R(P) \ni \alpha \rightarrow f_\alpha \in \text{Hom}_K(K_P,F)$$
$$\text{Hom}_K(K_P,F) \ni \sigma \rightarrow \sigma(\bar{x}) \in R(P)$$
where $\bar{x}$ is $x$ modulo $P$ in the quotient ring.\\
In particular, $\left|\text{Hom}_K(K_P,F)\right|=|R(P)| \le \deg{P} =[K_P:K]$
\end{proposition}
\begin{proof}[\bf Proof] It is clear that $\bar{x}$ is a root of $P$ and so by the previous observation, $\sigma(\bar{x}) \in R(P)$.
Let $\alpha \in R(P)$, and so clearly $f_\alpha(\bar{x})=\alpha$.
\end{proof}
If in particular $F/K$ is a simple extension, say $F=K(\alpha)$. Then we can interprete the permutations of roots as $K$-automorphisms of a simple extension. In
general, the group of $K$-automorphisms $Aut_K(F)$ of $F$ acts on the set $Hom_K(K_P,F)$
as follows:
$$\text{Aut}_K(F) \times \text{Hom}_K(K_P ,F) \ni (\sigma, f) \rightarrow \sigma \circ f \in \text{Hom}_K(K_P , F)$$
Thus we have:
\begin{proposition} Let $F \cong K_P$. For any root $\alpha \in R(P)$, the map:
$$\text{Aut}_K(F) \ni \sigma \rightarrow \sigma(\alpha) \in R(P)$$ is bijective. In particular,
$\text{Aut}_K(F)| =|R(P)| \le [F:K]$
\end{proposition}
\begin{proof}[\bf Proof] As $[F:K]=[K_P:K]$, the map $f_\alpha \in \text{Hom}_K(K_P,F)$ is a $K$-isomorphism. Therefore, we have a bijection:
$$\text{Aut}_K(F) \ni \sigma \rightarrow \sigma \circ f_\alpha \in \text{Hom}_K(K_P,F)$$
Now use Proposition 11.20, the map
$$\text{Hom}_K(K_P,F) \ni f \rightarrow f(\bar{x}) \in R(P)$$ is a bijection and hence the composition is also bijective
and use $f_\alpha(\bar{x})=\alpha$.
\end{proof}
\subsection{Splitting field and separability}
We shall very briefly introduce the concepts of splitting field and separability here. Detailed explanations will not be given in this book.
\begin{definition} Let $K$ be a field and $P(x) \in K[x]$. We say $P$ splits in a field extension $F$ of $K$ if $P$ can be written as a linear product in $F[x]$. The splitting field of $P$ is the smallest field extension $F$ such that $P$ splits in $F$.
\end{definition}
In fact, this definition can be generalised if we modify and extend the definition of field extension as follows
\begin{definition} Let $K$ be a field. An extension $F_\tau/K$ is defined as a pair $(F, \tau)$ of a
field $F$ and a ring homomorphism $\tau: K \rightarrow F$. A $K$-homomorphism from $F_\tau$ to $F'_{\tau'}$
is a ring homomorphism $\psi : F \rightarrow F'$ such that $\tau' = \psi \circ \tau$. By a subextension
$L_\tau/K$ of $F_\tau/K$, we mean an intermediate field $L$ of $F/\tau(K)$ and $\tau : K \rightarrow L$.
\end{definition}
But we shall not give details in this book.
\begin{example} The splitting field of $P(x)=x^2-2$ in $\mathbb{Q}[x]$ is $\mathbb{Q}(\sqrt{2})$. The splitting field of
$P(x)=x^4+2$ in $\mathbb{Q}[x]$ is $\mathbb{Q}(i,\sqrt[4]{2})$.
\end{example}
(In the sense of field extension, Let $P \in K[x]\backslash K$. If $P$ splits in an extension $F'$ if and only if
Hom$_K(F,F')\neq \emptyset$, then we call $F$ a splitting field of $P$ over $K$).
\begin{definition}
\begin{enumerate}
\item[(i)] A polynomial $P \in K[X]\backslash K$ is called separable if $|R(P,E)| =
\deg{P}$ for some extension $E/K$.
\item[(ii)] Let $F/K$ be an algebraic extension and an element $\alpha \in F$ is separable if the minimal polynomial
$\min_{K,\alpha}(x)$ is separable.
\item[(iii)] A finite extension $F/K$ is called separable if $|\text{Hom}_K(F,E)| = [F : K]$ for some
extension $E/K$.
\end{enumerate}
\end{definition}
To be concrete, a polynomial is separable if it has no repeated roots in the splitting field. For example,
$P(x)=(x-3)^2$ is not separable in any field. $P(x)=x(x-1)(x-2)$ is separable in $\mathbb{Q}[x]$ but is not separable in a field with characteristic $2$ because it is the same as $P(x)=x^2(x-1)$.
\begin{lemma} Let $F/L/K$ be finite with $F = L(\alpha)$, and $E/K$ arbitrary. Let $P$ be the
minimal polynomial of $\alpha$ over $L$. Then $$|\text{Hom}_K(F,E)| \le |\text{Hom}_K(L,E)| \cdot [F : L]$$ and
the equality holds if and only if $|R(\tau P,E)| = \deg{P}$ for all $\tau \in \text{Hom}_K(L,E)$ where
$R(\tau P,E)$ is the root of $\tau P$ in $E$.
\end{lemma}
\begin{proof}
It is not hard to check that:
$$\text{Hom}_K(F,E) =\bigsqcup_{\tau \in \text{Hom}_K(L,E)}\text{Hom}_L(F,E_\tau)$$
Use Proposition 11.20 that $|\text{Hom}_L(F,E_\tau )|= |R(\tau P,E)|$ and also $\deg{P} =[F:L]$.
\end{proof}
\begin{lemma} Let $F/K$ be a finite separable extension. Let $L,L'$ be subextensions of $F/K$ with $L \subset L'$. Then
$$\left|\text{Hom}_L(L,E)\right|=[L':L]$$
\end{lemma}
\begin{proof}[\bf Proof] WLOG, we can write $F=K(\alpha_1,\ldots,\alpha_m)$ for some $\alpha_i \in F$.
Let $(L_i)_{0 \le i \le n}$ be a chain of subextensions such that:
$$L_0=K, L_n=F, L_{i+1}=L_i(\alpha_{i+1})$$
Let $P_i$ be the minimal polynomial of $\alpha_i$ over $L_{i-1}$. Then the equality in Lemma 11.26 holds by using Tower Law and the definition of separability. Hence by Lemma 11.26
$$\left|R(\tau P_i,E)\right| = \deg{P_i} \text{ for all } \tau \in \text{Hom}_K(L_{i-1},E)$$
Then we let $L=L_i,L'=L_j$ for some $i,j$ in the chain. Then at each stage, the above equation holds, and so the equality in Lemma 11.26 holds and hence the result follows.
\end{proof}
The most important theorem we are going to prove in this subsection which will be quoted frequently later is:
\begin{theorem}{\bf [Primitive Element Theorem]}\label{P;Primitive element} Every separable finite extension is simple.
\end{theorem}
\begin{proof}[\bf Proof] Let $F/K$ be a finite extension. If $K$ is finite then so is $F$ and so the group $F^*$ is cyclic.
Assume now $K$ is infinite. Let $F=K(\alpha_1,\ldots,\alpha_m)$ and we prove the case when $m=2$ first. So let
$F=K(\alpha,\beta)$, $[F:K]=n$ and Hom$_K(F,E)=\{\sigma_1,\ldots,\sigma_n\}$. for some extension $E/K$. Each homomorphism $\sigma_i$ is determined by the images of $\alpha$ and $\beta$. So we have, for $\i \neq j$, $\sigma_i(\alpha) \neq \sigma_j (\alpha)$ or $\sigma_i(\alpha) \neq \sigma_j(\beta)$. Consider the polynomial:
$$P(x)=\prod_{i \neq j}\left((\sigma_i(\alpha)-\sigma_j(\alpha))x+(\sigma_i(\beta)-\sigma_j(\beta))\right)$$
As $K$ is infinite, there exists $\gamma \in K$ which is not a root of $P(x)$. Let $\delta=\alpha\gamma+\beta \in F$.
Then
$$0 \neq P(\gamma)=\prod_{i \neq j}(\sigma_i(\delta)-\sigma_j(\delta))$$
Thus $\sigma_i(\delta) \neq \sigma_j(\delta)$ whenever $i \neq j$.
Therefore, $\sigma_1,\ldots,\sigma_n$ are $n$ different $K$-homomorphisms from $K(\delta)$ to $E$.
So
$$n \le \left|\text{Hom}_K(K(\delta),E)\right| \le [K(\delta):K]$$
where the second inequality came from proposition 11.20. But $[K(\delta):K] \le [F:K]=n$. Therefore, the equality holds and so $K(\delta)=F$. This completes the proof for the case $m=2$.

For $m \ge 2$, observe that we have
$$K(\alpha_1,\alpha_2,\ldots,\alpha_m)=K(\alpha_1,\alpha_2)(\alpha_3,\ldots,\alpha_m)
=K(\beta,\alpha_3)(\alpha_4,\ldots,\alpha_m)$$
for some suitable $\beta$ by the case $m=2$. Since each subextension is separable by Lemma 11.27, we can apply this argument repeatedly and hence the result follows.
\end{proof}
\begin{theorem} Any finite extension of $\mathbb{Q}$ is separable. (This is one of the important facts we shall quote and use in the next chapter)
\end{theorem}
\begin{proof}[\bf Proof] Let $K/\mathbb{Q}$ be a finite extension. We will show firstly that if $P(x) \in \mathbb{Q}[x]$ is irreducible, then $P(x)$ is separable. Clearly, $P(x)$ is non constant and so the differential $D(P) \neq 0$. As
$\deg{D(P)} <\deg{P}$ so $D(P) \not \subseteq \langle P \rangle$. Thus, the sum
$$\langle P \rangle + \langle D(P) \rangle =\mathbb{Q}[x]$$ because $P(x)$ irreducible implies that
$\langle P(x) \rangle$ is a maximal ideal. So there exists $f(x),g(x) \in \mathbb{Q}[x]$ such that
$$f(x)P+g(x)D(P)=1$$ and so the linear factor of $P(x)$ does not divide $D(P)$. Hence we have no repeated roots and so $P(x)$ is separable. Thus, if $K=\mathbb{Q}(\alpha)$, with $\alpha$ algebraic, then $K/\mathbb{Q}$ is separable and similarly, if $K=\mathbb{Q}(\alpha_1,\ldots,\alpha_n)$ is a finite extension of $\mathbb{Q}$, then $K/\mathbb{Q}$ is separable.
\end{proof}
\subsection{Exercises}
\begin{enumerate}
\item \begin{enumerate}
      \item[(i)] Let $F/K$ be a field extension whose degree is prime. Show that there is no intermediate extension
$$F \not \supseteq K' \not \supseteq K$$
      \item[(ii)] Let $F/K$ be an extension and suppose that $\alpha \in F$ be algebraic over $K$ of odd degree, i.e. $[K(\alpha):K]$ is odd. Show that $K(\alpha)=K(\alpha^2)$.
      \item[(iii)] Let $F=K(\alpha,\beta)$ with $[K(\alpha):K]=m, [K(\beta):K]=n$ and $(m,n)=1$. Show that
      $[F:K]=mn$.
      \end{enumerate}
\item Let $F/K$ be an extension of degree $2$. Show that if the characteristic of $K$ is not $2$, then
      $F=K(\alpha)=\{a+b\alpha: a,b \in K\}$ for some $\alpha \in F$ with $\alpha^2 \in K$. Show that if the characteristic is $2$, then either $F=K(\alpha)$ with $\alpha^2 \in K$ or $F=K(\alpha)$ with $\alpha^2+\alpha \in K$.
\item Prove that $[\mathbb{Q}(\sqrt{2},\sqrt{3},\sqrt{5}):\mathbb{Q}]=8$.
\item Prove that $\mathbb{Q}(\sqrt{2},i)=\mathbb{Q}(\sqrt{2}+i)$.
\item Let $F/K$ be a finite extension and $P(x) \in K[x]$. an irreducible polynomial of degree $d>1$. Show that
      if $(d,[F:K])=1$, then $P$ has no roots in $F$.
\item Let $L/K$ be a field extension. Show that $\alpha$ is algebraic over $K$ if and only if the field $K(\alpha)$
       is finite dimensional as a vector space over $K$. Let $F/L$,$L/K$ be algebraic extensions. Show that $F/K$ is also algebraic.
\item Let $F/K$ be a field extension and let $\alpha,\beta \in F$. Show that $\alpha+\beta$ and $\alpha \beta$ are algebraic over $K$ if and only if $\alpha$ and $\beta$ are algebraic.
\item Let $F/K$ be an extension and $\alpha,\beta \in F$ be transcendental over $K$. Show that $\alpha$ is algebraic over $K(\beta)$ if and only if $\beta$ is algebraic over $K(\alpha)$.
\item Let $\alpha$ be algebraic over $K$. Show that there is only a finite number of intermediate fields $K \subset K' \subset K(\alpha)$. (Hint: consider the minimal polynomial $P$ of $\alpha$ over $K$ and show that $P$ determines $K'$.) Now let $F/K$ be a finite extension for which there exists only finitely many intermediate subfields $K \subset K' \subset F$, then $F=K(\alpha)$ for some $\alpha \in F$. (Hint: Discuss the case when $K$ is finite and the case when $K$ is infinite separately)
\item Show that if $F$ is a splitting field over $K$ for $P \in K[x]$ of degree $n$, then $[F:K] \le n!$.
\item Let $K$ be a field and $\alpha \in K$. If $m,n \in \mathbb{N}$ with $(m,n)=1$. Show that $x^{mn}-\alpha$
    is irreducible if and only if $x^m -\alpha $ and $x^n-\alpha$ are irreducible.
\item Let $a,b$ be square free positive integer $>1$. Show that the minimal polynomial of $\sqrt{a}+\sqrt{b}$ over $\mathbb{Q}$ is reducible mod $p$ for all primes $p$. (Hint: You may find it helpful to show firstly that for every prime $p$, one of $a,b$ or $ab$ is a square in $\mathbb{F}_p=\{0,1,\ldots,p-1\}$)
\end{enumerate}




\section{Algebraic Number field}
\subsection{Basic concepts}
We have seen in the previous chapter the definition of algebraic element. We shall enhance this definition in this chapter.
\begin{definition} A {\bf number field} is a field $K$ that is a finite extension of $\mathbb{Q}$.
For example, $K=\mathbb{Q}[\sqrt{2}] \cong \mathbb{Q}[x]/\langle x^2-2 \rangle$.
\end{definition}
\begin{definition} Let $K$ be a number field. An element $\alpha \in K$ is called an {\bf algebraic number} if it satisfies a non-trivial polynomial with coefficients in $\mathbb{Q}$. For example, $\sqrt{2},\sqrt[3]{2}$ are algebraic numbers.
\end{definition}
\begin{definition} Let $K$ be a number field. An element $\alpha \in K$ is called an {\bf algebraic
integer} if it satisfies a non-trivial {\bf monic} polynomial with {\bf integer} coefficients. For example, $\sqrt{2}$ is an algebraic integer, while $\frac{1}{\sqrt{2}}$ is not an algebraic integer.

More generally, let $A \subset B$ be two rings and $x \in B$. We say $x$ is integral over $A$ if there exists a monic polynomial $P$ with coefficients in $A$ such that $P(x)=0$.
\end{definition}
\begin{remark} To check whether $\alpha$ is an algebraic integer, we only need to check whether the minimal polynomial has coefficients in $\mathbb{Z}$. Suppose $\alpha$ satisfies $g(x)$, which is monic and has integer coefficients. Let $f$ be the minimal polynomial of $\alpha$ over $\mathbb{Q}$ and we assume that $g \neq f$. So $f|g$ and so by Gauss's lemma $g$ is reducible in $\mathbb{Q}[X]$ and so must be reducible in $\mathbb{Z}[X]$ and so $f$ has integer coefficients.
\end{remark}
\begin{definition} Let $K$ be a number field. The {\bf ring of integers} is defined by the set of algebraic integers in $K$, and is denoted by $\mathcal{O}_K$
\end{definition}
The main aim of this subsection is to prove that $\mathcal{O}_K$ is a ring for any number field $K$.
\subsection{Cayley-Hamilton Theorem}
\begin{theorem}{\bf [Cayley-Hamilton theorem]}\label{C;Cayley} Let $A$ be a ring and $I$ be an ideal of $A$. Let $M$ be an $A$-module generated by $m_1,\ldots,m_n$ and $\phi: M \rightarrow M$ is an $A$-endomorphism such that
$\phi(M) \subseteq IM$. Then $\phi$ satisfies an equation
$$\phi^n+a_{n-1}\phi^{n-1}+\ldots+a_0=0$$
for some $a_i \in I$.
\end{theorem}
\begin{proof}[\bf Proof] Let $\phi(m_i)=\sum_{j=1}^n a_{ij}m_j$ for some $a_{ij} \in I$ because $\phi(M) \subseteq IM$.
Hence
$$\sum_{j=1}^n (\delta_{ij}\phi-a_{ij})m_j=0 \text{ for all } i$$
Let End$_A(M)$ be the set of $A$-endomorphism from $M$ to $M$, then $A[\phi] \subseteq \text{End}_A(M)$. Since
End$_A(M)$ is a commutative ring. So $M$ is an $A[\phi]$-module. Thus we can write
\begin{equation*} P\cdot \begin{pmatrix} m_1 \\ .\\.\\m_n \end{pmatrix} = \begin{pmatrix} 0\\.\\.\\0\end{pmatrix}
\end{equation*}
Where $P_{ij}=\delta_{ij}\phi - a_{ij}$ which is an $n \times n$ matrix.
Let $P^{adj}$ be the adjoint matrix of $P$ and so
$$P^{adj}P=(\det{P})I_n$$
and so left multiply by $P^{adj}$ in the above matrix equation, we have
\begin{equation*} (\det{P}) I_n \cdot \begin{pmatrix} m_1\\.\\.\\m_n \end{pmatrix}
=\begin{pmatrix} 0\\.\\.\\0 \end{pmatrix} \end{equation*}
Thus we have $\det{P}=0$, because $m_i \neq 0$. Therefore,
$$\det{P}=\phi^n+a_{n-1}\phi^{n-1}+\ldots +a_0$$ for some $a_i$ with each $a_i$ a polynomial in $a_{ij}$.
As $a_{ij} \in I$ for all $i,j$, so $a_i \in I$ for all $i$.
\end{proof}
\begin{corollary}{\bf [Nakayama's Lemma]}\label{N;Nakayama} Let $A$ be a ring and $I$ be an ideal of $A$. $M$ is an
$A$-module generated by $m_1,\ldots,m_n$. If $IM=M$, then there exists $x \in A$ such that
$x \equiv 1$ (mod) $I$ and $xM=0$.
\end{corollary}
\begin{proof}[\bf Proof] In Theorem 12.6, we take $\phi$ to be the identity map, and so all conditions in Theorem 12.6 are satisfied. Then we have some $a_i \in I$ such that
$$\phi^n+a_{n-1}\phi^{n-1}+\ldots+a_0=0$$
Take $x=1+a_{n-1}+\ldots+a_0$. So $x \equiv 1$ (mod) $I$, and let $m \in M$, so we have
$$xm=(1+a_{n-1}+\ldots+a_0)m=(\phi^n+a_{n-1}\phi^{n-1}+\ldots+a_0)m =0$$ because it is a zero map.
\end{proof}
\subsection{Ring of integers}
\begin{definition} An $A$-module $M$ is {\bf faithful} if for all $a \in A$ with $a \neq 0$, there exists $m \in M$ such that $am \neq 0$. Equivalently, for all $a \neq b \in A$, there exists $m \in M$ such that $am \neq bm$.
\end{definition}
\begin{lemma} Let $A,B$ be two rings such that $A \subseteq B$. Let $x \in B$. Then the following are equivalent:
\begin{enumerate}
\item[(i)] $x$ is integral over $A$.
\item[(ii)] $A[x]$ is a finitely generated $A$-module.
\item[(iii)] $A[x]$ is contained in a subring $C$ of $B$ such that $C$ is finitely generated $A$-module.
\item[(iv)] There exists a faithful $A[x]$-module $M$ that is finitely generated as an $A$-module.
\end{enumerate}
\end{lemma}
\begin{proof}[\bf Proof] The proof has the following four steps:
\begin{enumerate}
\item (i) $\Rightarrow$ (ii): As $x$ is integral over $A$, then there exists $a_i \in A$ such that
$$x^n+a_{n-1}x^{n-1}+\ldots+a_0 =0$$
Then $A[x]$ is finitely generated by $\{1,x,x^2,\ldots,x^{n-1}\}$ as an $A$-module.
\item (ii) $\Rightarrow$ (iii): Take $C=A[x]$.
\item (iii) $\Rightarrow$ (iv): Take $M=C$ and clearly the subring $C$ is faithful.
\item (iv) $\Rightarrow$ (i): Define the map $\phi: M \rightarrow M$ by left multiplication by $x$. In other words, $\phi(m)=xm$. Let $I=A$. Then use the same notation as in Theorem 12.6, we have $\phi(M) \subset IM$ because $M$ is a finitely generated $A$-module. Hence by Theorem 12.6, we have some $a_i \in I=A$ such that
    $$\phi^n+a_{n-1}\phi^{n-1}+\ldots+a_0=0$$
    and so apply this to each $m \in M$, we have
    $$P(x)m=(x^n +a_{n-1} x^{n-1}+\ldots+a_0)m=0$$
    This implies that $P(x)=0$ because if not, then $0 \neq P(x) \in A[x]$. As $M$ is a faithful $A[x]$-module, so we must have some $m$ such that $P(x) m \neq 0$, which gives a contradiction. Hence $P(x)=0$ and so
    $x$ is an algebraic integer over $A$.
\end{enumerate}
\end{proof}
\begin{corollary} Let $x_1,\ldots,x_n \in B$ which are algebraic integers over $A$. Then $A[x_1,\ldots,x_n]$ is a finitely generated $A$-module.
\end{corollary}
\begin{proof}[\bf Proof] Use induction on $n$. When $n=1$, use (i) $\Rightarrow$ (ii) in Lemma 12.9.
Suppose true for $n-1, n \ge 2$ then $A[x_1,\ldots,x_n]=A[x_1,\ldots,x_{n-1}][x_n]$, and use the case $n=1$, we conclude that $A[x_1,\ldots,x_n]$ is a finitely generated $A[x_1,\ldots,x_n]$-module.
Also, by inductive hypothesis, $A[x_1,\ldots,x_{n-1}]$ is a finitely generated $A$-module, and so combine these two, we have $A[x_1,\ldots,x_n]$ is a finitely generated $A$-module.
\end{proof}
\begin{corollary} $D=\{x \in B: x \text{ is integral over } A\}$ is a subring of $B$. In particular, take $A=\mathbb{Q}$, $B$ any number field, then the ring of integers $\mathcal{O}_K$ is a ring.
\end{corollary}
\begin{proof}[\bf Proof] Suppose $y,z \in D$. Then $C=A[y,z]$ is a finitely generated $A$-module by Corollary 12.10. Then use
(iii) $\Rightarrow$ (i) in Lemma 12.9, since $A[y+z],A[yz] \subseteq A[y,z]$, we have $y+z, yz$ integral over $A$.
Hence $D$ is a subring.
\end{proof}
\subsection{Integral elements and integral closure}
In this subsection we shall derive some basic properties of integral elements.
\begin{definition} Let $A \subseteq B$ be two rings. We say $B$ is {\bf integral} over $A$ if every element in $B$ is integral over $A$.
\end{definition}
\begin{proposition} Let $A \subseteq B \subseteq C$ be a tower of rings. If $x \in C$ is integral over $A$ then $c$ is integral over $B$.
\end{proposition}
\begin{proof}[\bf Proof] As $c \in C$ is integral over $A$ there exist $a_i \in A, n \in \mathbb{N}$ such that
$$c^n+a_{n-1}c^{n-1}+\ldots+a_0=0$$
As $A \subseteq B$ so $a_i \in B$ and hence the result follows.
\end{proof}
\begin{theorem} Let $A \subseteq B \subseteq C$ be a tower of rings. Suppose $B$ is integral over $A$, and $x \in C$ is integral over $B$. Then $x$ is integral over $A$.
\end{theorem}
\begin{proof}[\bf Proof] As $x$ is integral over $B$, there exist $b_i \in B$ and $n \in \mathbb{N}$ such that
$$x^n + b_{n-1}x^{n-1}+\ldots+b_0=0$$
Let $R=A[b_0,\ldots,b_{n-1}$. Since $B$ is integral over $A$ then by Corollary 12.10, $R$ is a finitely generated $A$-module, say by $r_1,\ldots,r_m$. Also $R[x]$ is a finitely generated $R$-module by $\{1,x,\ldots,x^{n-1}\}$. So $R[x]$ is a finitely generated $A$-module by $\{r_ix^j: 1 \le i \le m, 1 \le j \le n-1\}$. Now apply
(iii) $\Rightarrow$ (i) in Lemma 12.9 to conclude that $x$ is integral over $A$.
\end{proof}
\begin{definition} Let $A \subseteq B$ be two rings. The {\bf integral closure} of $A$ in $B$ is the set of elements in $B$ which are integral over $A$. Usually, we denote the integral closure of $A$ in $B$ by $A^B$.
\end{definition}
\begin{corollary} Let $A \subseteq B \subseteq C$ be a tower of rings. Suppose $B$ is integral over $A$. Then the integral closure of $A$ in $C$ and the integral closure of $B$ in $C$ are the same.
\end{corollary}
\begin{proof}[\bf Proof] Combine Proposition 12.13 and Theorem 12.14.
\end{proof}
\begin{theorem} Let $D$ be a unique factorisation domain. Let $F$ be the field of quotients of $D$. Then
$c \in F$ is integral over $D$ if and only if $c \in D$.
\end{theorem}
\begin{proof}[\bf Proof] If $c \in D$ then $c$ satisfies $x-c$ and so is integral over $D$ and also $c \in F$.
Conversely, suppose $c \in F$ is integral over $D$. Then there exist $a_i \in D$, $n \in \mathbb{N}$ such that
$$c^n+a_{n-1}c^{n-1}+\ldots+a_0=0$$
As $c \in F$ so we have $c=\frac{r}{s}$ for some $r,s \in D$ with $(r,s)=1$ and $s \neq 0$. Substitute this into the equation and
clear the denominator, we have
$$r^n+a_{n-1}r^{n-1}s+\ldots+a_1rs^{n-1}+a_0s^n=0$$
Since $D$ is a unique factorisation domain, so each element can be expressed uniquely as a finite product of irreducibles. But each irreducible is prime in unique factorisation domain. So if $s$ is not a unit, we have some prime $p|s$. Then $p|r^n$ and hence $p|r$ as $p$ is prime, which contradicts $(r,s)=1$. So $s$ must be a unit and so $c \in D$.
\end{proof}
\begin{corollary} Let $q \in \mathbb{Q}$. Then $q$ is an algebraic integer if and only if $q \in \mathbb{Z}$.
\end{corollary}
\begin{proof}[\bf Proof] Clearly every integer is an algebraic integer. Conversely, as $\mathbb{Z}$ is a unique factorisation
domain and $\mathbb{Q}$ is its field of quotient. Thus by Theorem 12.17, if $q \in \mathbb{Q}$ is an algebraic integer (integral over $\mathbb{Z}$), then $q \in \mathbb{Z}$.
\end{proof}
\begin{proposition} Let $A=\mathbb{Z}$ and $B=\mathbb{Q}(i)$. Then the integral closure of $A$ in $B$ is
$$A^B=\mathbb{Z}[i]$$
\end{proposition}
\begin{proof}[\bf Proof] By Remark 12.4, it is enough to check the minimal polynomials. Let $\alpha=u+iv \in B$. Then if $v \neq 0$ then the minimal polynomial of $\alpha$ over $\mathbb{Z}$ is
$$P(x)=x^2-2ux+u^2+v^2$$
Then we must have $2u \in \mathbb{Z}$ and $u^2+v^2 \in \mathbb{Z}$. If $u$ has denominator $2$, then it is impossible for $v \in \mathbb{Q}$ such that $u^2+v^2 \in \mathbb{Z}$. Hence $u \in \mathbb{Z}$ and $v \in \mathbb{Z}$.

If now $v=0$ then the only algebraic integer in $\mathbb{Q}$ is integer by Corollary 12.18. Hence we conclude that
$A^B=\mathbb{Z}[i]$.
\end{proof}
\begin{definition} An integral domain $D$ is said to be {\bf integrally closed} or {\bf normal} if and only if elements of its field of quotient that are integral over $D$ are those of $D$ itself.
\end{definition}
\begin{example} $D=\mathbb{Z}[\sqrt{-3}]$ is not integrally closed. The field of quotient is $F=\mathbb{Q}[\sqrt{-3}]$. Let $\alpha=\frac{1+\sqrt{-3}}{2} \in F$ and $\alpha \not \in D$. Moreover, $\alpha$ satisfies the equation
$$x^2-x+1=0$$
and so it is not integrally closed. Hence by Theorem 12.17, it is not a unique factorisation domain.
\end{example}
\begin{theorem} Every algebraic number is of the form $\frac{a}{b}$, where $a$ is an algebraic integer and $b$ is a non-zero ordinary integer.
\end{theorem}
\begin{proof}[\bf Proof] Let $c$ be an algebraic number. Then there exist $a_i \in \mathbb{Q}$ and $n \in \mathbb{N}$ such that
$$c^n+a_{n-1}c^{n-1}+\ldots+a_1 c+a_0=0$$
Let $b$ be the least common multiple of the denominator of $a_i$. Thus $b \in \mathbb{N}$ and $ba_i \in \mathbb{Z}$. Then clear the denominators, we have
$$(bc)^n+(ba_{n-1})(bc)^{n-1}+\ldots+(b^{n-1}a_1)(bc)+(b^na_0)=0$$
which shows that $bc$ is an algebraic integer, say $a$. Thus $c=\frac{a}{b}$ is of the required form.
\end{proof}
\begin{theorem} Let $K$ be a number field. Then the field of quotient of $\mathcal{O}_K$ is $K$.
\end{theorem}
\begin{proof}[\bf Proof] Let $F$ be the field of quotient of $\mathcal{O}_K$. Take $\alpha \in F$, then $\alpha=\frac{x}{y}$ for some $x,y \in \mathbb{O}_K$ and so $x,y \in K$, which means $\alpha \in K$. Hence $F \subseteq K$.

Now take any $\alpha \in K$. Since $K$ is a finite extension, hence algebraic, and so $\alpha$ is algebraic and therefore $\alpha=\frac{x}{y}$ for some algebraic integer $x$ and ordinary integer $y$ by Theorem 12.22. Also
$x =\alpha y \in K$ and so $x \in \mathcal{O}_K$. Then $\alpha=\frac{x}{y} \in F$ and so $K \subset F$. Therefore, $F=K$.
\end{proof}
\begin{theorem} Let $K$ be a number field then $\mathcal{O}_K$ is integrally closed.
\end{theorem}
\begin{proof}[\bf Proof] Let $\alpha$ be integral over $\mathcal{O}_K$. Since $\mathcal{O}_K$ is integral over $\mathbb{Z}$, so $\alpha$ is integral over $\mathbb{Z}$ by Theorem 12.14. So $\alpha$ must be an algebraic integer. By Theorem 12.23, the field of quotient of $\mathcal{O}_K$ is $K$ and so the result follows.
\end{proof}
\subsection{Trace and norm}
\begin{definition}
Let $L/K$ be a finite field extension, define for any $x \in L$ the map:
$$\phi_x: L \rightarrow L$$ by
$$\phi_x(y)=xy$$
Clearly, this is a $K$-linear map. Then
\begin{enumerate}
\item[(i)] Denote the trace of $\phi_x$ by $T_{L/K}(x)$.
\item[(ii)] The bilinear form $T: L \times L \rightarrow K$ is defined by $T(x,y)=T_{L/K}(xy)$.
\item[(iii)] Denote the norm (determinant) of $\phi_x$ by $N_{L/K}(x)$.
\end{enumerate}
\end{definition}
\begin{theorem}{\bf [Dedekind]}\label{D; Dedekind hom} Let $F,E$ be fields and $\sigma_1,\ldots,\sigma_n$ be $n$ distinct field homomorphisms from $F$ to $E$. Then they are linearly independent over $E$ in the $E$-vector
space of all additive group homomorphisms from $F$ to $E$. In other words,
if $c_1,\ldots,c_n \in E$, with $\sum_{i=1}^n c_i \sigma_i(x)=0$ for all $x \in F$. Then $c_i=0 ~\forall i$.
\end{theorem}
\begin{proof}[\bf Proof] Suppose not and let $k$ be the minimal integer for which $\{\sigma_1,\ldots,\sigma_k\}$ is linearly depend. So there exist $c_i \in E$ such that $\sum_{i=1}^k c_i \sigma_i(x)=0$ and $c_k \neq 0$. As $\sigma_k \neq 0$, there is some $t \le k$ such that $c_t \neq 0$. As $\sigma_t \neq \sigma_k$, choose $x \in F$ with $\sigma_t(x) \neq \sigma_k(x)$. For all $y \in F$, we have
$$\sum_{i=1}^kc_j \sigma_j(x)\sigma_j(y)=\sum_{j=1}^k c_j \sigma(xy)=0$$
and hence
$$\sum_{j=1}^k c_j\sigma_j(x)\sigma_j=0$$ Therefore,
$$\sum_{j=1}^{k-1} c_j (\sigma_j(x)-\sigma_k(x))\sigma_j=\sum_{j=1}^k c_j \sigma_j(x)\sigma_j-\sigma_k(x)\sum_{j=1}^k c_j\sigma_j=0$$
As $\sigma_t(x) \neq \sigma_k(x)$, and $c_t \neq 0$, we conclude that $\{\sigma_1,\ldots,\sigma_{k-1}\}$ is
linearly dependent, which is a contradiction.
\end{proof}
\begin{lemma} Let $F/K$ a finite separable extension with $[F : K] = n$, and
$$\text{Hom}_K(F,E) =\{\sigma_1,\ldots , \sigma_n\}$$ for an extension $E/K$.
Then a subset $X = {x_1, \ldots , x_n} \subset F$ is a K-basis
of F if and only if the matrix $A \in GL_n(E)$, where $A_{ij}=\sigma_i(x_j)$
\end{lemma}
\begin{proof}[\bf Proof] If $X$ is a basis for of $F$. Let $X^*=\{x^*_1,\ldots,x^*_n\}$ be the dual basis. Then it is easy toe check that $X^*$ forms a $E$-basis for the set of $K$-linear maps from $F$ to $E$. Also, by Theorem 12.24, the set
Hom$_K(F,E)=\{\sigma_1,\ldots,\sigma_,n\}$ also forms a $E$-basis for the set of $K$-linear maps from $F$ to $E$.
Thus we have
$$\sigma_i=\sum_{j=1}^n \sigma_i(x_j)x^*_j$$ and as we can write $x^*_j$ in terms of linear combination of $\sigma_i$'s, the matrix $A_{ij}=\sigma_i(x_j)$ must be invertible and hence $\in GL_n(E)$.

Conversely, suppose the matrix $A_{ij}=\sigma_i(x_j) \in GL_n(E)$ and let
$$\sum_{j=1}^n c_j x_j=0 \text{ for some } c_j \in K$$ Then $\sum_{j=1}^n c_j \sigma_i(x_j)=0$ for all $i$. Thus we may invert the matrix $A$ and we have $c_i=0$ for all $i$ and since $[F:K]=n$, so linear independence implies that $\{x_1,\ldots,x_n\}$ is a $K$-basis for $F$.
\end{proof}
\begin{theorem} Let $F/K$ be separable and $[L:K]=n$. Then
$$T_{F/K}(x)=\sum_{i=1}^n \sigma_i(x), N_{F/K}(x)=\prod_{i=1}^n \sigma_i(x)$$ where
Hom$_K(F,E)=\{\sigma_1,\ldots,\sigma_n\}$ for an extension $E$. In particular, $T_{F/K}(x)$ is not a zero map by Theorem 12.24 and hence surjective (as the dimension of $K$ as a $K$-vector space is $1$.)
\end{theorem}
\begin{proof}[\bf Proof] The trace and norm is independent of the choice of basis. Let $X=\{x_1,\ldots,x_n\}$ be a $K$-basis of $F$, and let $X^*=\{x^*_1,\ldots,x^*_n\}$ be the dual basis. $\phi_x$ is a $K$-linear map from $F$ to $F$. By a simple fact of Linear Algebra, if $A$ is the matrix for $\phi_(x)$ with respect to the basis $X$, and $\phi^*_x$ is the dual map of $\phi_x$, then the matrix $B$ for $\phi^*_x$ with respect to the basis $X^*$ is the transpose of $A$. Thus, they have the same trace and determinant. Hence we may calculate the trace and norm of $\phi^*_x$.
By Theorem 12.24, Hom$_K(F,E)=\{\sigma_1,\ldots,\sigma_n\}$ is a $E$-basis for the $K$-linear maps from $F$ to $E$.
Also,
$$\phi^*_x(\sigma_j)(x_i)=\sigma_j m_x(x_i)=\sigma_j(xx_i)=\sigma_j(x)\sigma_j(x_i)$$
and so $\phi^*_x(\sigma_j)=\sigma_j(x)\sigma_j$. Therefore, the linear map $\phi^*_x$ is diagonal with entries
$\sigma_j(x)$ with respect to the basis $\sigma_j$. Hence the result follows.
\end{proof}

\subsection{Conjugates}

We define the conjugates of an element over a subfield of $\mathbb{C}$

\begin{definition} Let $\alpha \in \mathbb{C}$ be algebraic over a subfield $K$ of $\mathbb{C}$. The {\bf conjugates} of $\alpha$ over $K$ are the roots of $\min_{K,\alpha}(x)$ in $\mathbb{C}$.
\end{definition}

\begin{example} The conjugates of $\sqrt{2}$ are $\pm \sqrt{2}$. The conjugates of $\frac{1+i}{\sqrt{2}}$ are
$\frac{1 \pm i}{\sqrt{2}},\frac{-1 \pm i}{\sqrt{2}}$.
\end{example}

\begin{remark} If $K$ is a number field, then $K/\mathbb{Q}$ is separable and so the conjugates must be distinct.
\end{remark}

\begin{lemma} The conjugates of an algebraic integer is algebraic integers.
\end{lemma}

\begin{proof}[\bf Proof] Let $\alpha$ be an algebraic integer and let $h(x)$ be a monic polynomial in $\mathbb{Z}[x]$ such that $h(\alpha)=0$. Then $min_{\mathbb{Q},\alpha}(x) \big|h(x)$ and so the conjugates of $\alpha$ are roots of $h(x)$ and so they are algebraic integers.
\end{proof}

By using the notation of conjugates, we may reprove an important fact as stated in Remark 12.4 without using Gauss's Lemma:

\begin{theorem} 
Let $\alpha$ be an algebraic integer. Then $\min_{\mathbb{Q},\alpha}(x) \in \mathbb{Z}[x]$.
\end{theorem}

\begin{proof}[\bf Proof] 
Let $\alpha_1=\alpha,\alpha_2,\ldots,\alpha_n$ be the conjugates of $\alpha$. Then
\begin{eqnarray*}
\min_{\mathbb{Q},\alpha}(x)&=&\prod_{i=1}^n (x-\alpha_i)\\
&=&x^n-(\alpha_1+\ldots \alpha_n)x^{n-1}+(\alpha_1\alpha_2+\ldots+\alpha_{n-1}\alpha_n)x^{n-2}\\
&~&+\ldots+(-1)^n\alpha_1\ldots\alpha_n
\end{eqnarray*}
Let $e_i$ be the $i^{th}$ elementary symmetric function of $\alpha_j$'s. In other words,
$$e_i=\sum_{j_k}\prod_{k=1}^i \alpha_{j_k}$$
where the sum runs through all possible ordered $i$-subsets of $\{1,\ldots,n\}$. Then $e_i \in \mathbb{Q}$ as
$\min_{\mathbb{Q},\alpha}(x) \in \mathbb{Q}[x]$.

As the conjugates of an algebraic integer must also be algebraic integers, and the ring of integer is a ring. Hence the sum, product of algebraic integers are algebraic integers and so $e_i$ are algebraic integers. But $e_i \in \mathbb{Q}$ and hence $e_i \in \mathbb{Z}$.
\end{proof}

\subsection{Complex embeddings}

Let $K$ be a number field. $[K:\mathbb{Q}]=n$. We can write $n=r+2s$ in such a way that there are exactly $r$ embeddings $$\sigma_1,\ldots,\sigma_r :K \rightarrow \mathbb{R}$$ and $2s$ complex embeddings
$$\sigma_{r+1},\ldots,\sigma_{2s+r} :K \rightarrow \mathbb{C}$$
To be precise, we start with some examples.
\begin{example}
\begin{enumerate}
\item $K=\mathbb{Q}[x]/\langle x^2-5 \rangle$ where $n=2,r=2,s=0$. $K=\mathbb{Q}(\bar{x})$, where
$\bar{x}=x$ (mod $x^2-5$). We have $\sigma_1(\bar{x})=\sqrt{5},\sigma_2(\bar{x})=-\sqrt{5}$.
\item $K=\mathbb{Q}[x]/\langle x^2+5 \rangle$ where $n=2,r=0,s=2$. $K=\mathbb{Q}(\bar{x})$, where
$\bar{x}=x$ (mod $x^2+5$). We have $\sigma_1(\bar{x})=i\sqrt{5},\sigma_2(\bar{x})=-i\sqrt{5}$.
\item $K=\mathbb{Q}[x]/\langle x^3-2 \rangle$ where $n=3,r=1,s=2$. $K=\mathbb{Q}(\theta)$, where
$\theta=x$ (mod $x^3-2$). We have $\sigma_1(\theta)=\sqrt[3]{2},\sigma_2(\theta)=e^{\frac{2\pi i}{3}}\sqrt[3]{2}, \sigma_3(\theta)=e^{-\frac{2\pi i}{3}} \sqrt[3]{2}$
\end{enumerate}
\end{example}
We see from the above that these embeddings are just sending $x$ to roots of the polynomials. Because when we define the number field $K=\mathbb{Q}[x]/\langle P \rangle$ for a monic irreducible polynomial $P$, we do not identify the roots of $P$. In other words, every root of $P$ has the equal weight and so these embeddings make the extension more concrete.

Now we shall justify what we did above is valid. By the primitive element theorem, we may assume $K=\mathbb{Q}(\alpha)$ for some $\alpha$ and let $P=\min_{\mathbb{Q},\alpha}(x)$. Then $K \cong \mathbb{Q}[x]/\langle P \rangle$. Let $\alpha_1=\alpha, \alpha_2,\ldots, \alpha_n$ be the conjugates of $\alpha$. Define:
$$\sigma_i: K \ni \bar{x} \rightarrow \alpha_i \mathbb{C}$$
and extend linearly so that $\sigma_i$ is an embedding for all $i$. Let $r$ be the number of real roots and $2s$ be the number of complex roots (as they appear as conjugate pair), then we have $r$ real embeddings and $2s$ complex embeddings.

Then, we need to check that the definition of $\sigma_i$ is independent of the choice $\alpha$.
Suppose $K=\mathbb{Q}(\beta)=\mathbb{Q}(\alpha)$. Let $Q(x)=\min_{K,\beta}(x)$ and so we have
$$\mathbb{Q}[x]/\langle P \rangle \cong \mathbb{Q}[x]/\langle Q \rangle$$
and let $\phi$ be the ring homomorphism from  $\mathbb{Q}[x]/\langle P \rangle$ to $\mathbb{Q}[x]/\langle Q \rangle$ such that $\phi(\lambda)=\lambda$ for all $\lambda \in \mathbb{Q}$.
As each real number is a limit point of a rational sequence, thus $\phi$ extends to an isomorphism
$$\phi: \mathbb{R}[x]/\langle P \rangle \rightarrow \mathbb{R}[x]/\langle Q \rangle$$
and we now factorise $P$ and $Q$ in $\mathbb{R}[x]$.
Use the fact that the only finite extension of $\mathbb{R}$ is $\mathbb{C}=\mathbb{R}[i]$. So each irreducible factors in $\mathbb{R}[x]$ has degree $1$ or $2$. Thus, writing $P=P_1\ldots P_m$ with $P_i$ irreducible and let $r$be the number of $P_i$ of degree $1$ and $s$ be the number of $P_i$ of degree $2$.
Thus by Chinese Remainder Theorem, we have
$$\mathbb{R}[x]/\langle P \rangle \cong \mathbb{R}[x]/\langle P_1 \rangle \times \ldots \mathbb{R}[x]/\langle P_m \rangle$$
Similarly, factorise $Q$ into irreducible factors $Q_1,\ldots,Q_k$ and let $r'$ be the number of $Q_i$ of degree $1$ and $s$ be the number of $Q_i$ of degree $2$. Then as $\phi$ is isomorphism, we have
$$\mathbb{R}[x]/\langle P_1\rangle \times \ldots \times \mathbb{R}[x]/\langle P_m \rangle
\cong \mathbb{R}[x]/\langle Q_1 \rangle \times \ldots \times \mathbb{R}[x]/\langle Q_k \rangle$$
Thus we have $r+2s=r'+2s'=n$ because the degree of the extension must be the same and it is equal to the degree of the polynomial.

Also, consider the number of solutions to $x(x^2+1)=x^3+x=0$ on both sides. Then the quotient ring
$\mathbb{R}[x]/\langle P_i \rangle$ with $P_i$ of degree $1$ has only one root $0$ and in each complex quotient ring, we have some element corresponding to the complex number $i$ because $\mathbb{R}[\alpha]=\mathbb{R}[i]$ if $\alpha \not \in \mathbb{R}$. So each $\mathbb{R}[x]/\langle P_i \rangle$ with $P_i$ of degree $2$ has three roots. Therefore, the number of roots is $3^s$ on the left hand side. Similarly, the number of roots is $3^{s'}$ on the right hand side. As they are isomorphic, we have $3^s=3^{s'}$ and so $s=s'$. Therefore, $r=r', s=s'$ and each
$\mathbb{R}[x]/\langle P_i \rangle \cong \mathbb{R}[x]/\langle Q_j \rangle$ for some $j$ and hence the embedding defined above is independent of the choice of $\alpha$ for which $K=\mathbb{Q}(\alpha)$.

Use this notation, we may define the conjugates of any $\alpha \in K$ to be $\sigma_1(\alpha),\ldots,\sigma_{r+2s}(\alpha)$. Here, we slightly modify the definition of conjugates in the sense that if $\alpha$ has minimal polynomial of degree less than $n$, then some of the above might be the same. But we still take those as a complete set of conjugates. For example, if $\alpha \in \mathbb{Q}$, then the conjugates of $\alpha$ is $\alpha,\alpha,\ldots,\alpha$ and we have $n$ of them.

\begin{proposition} Let $x,y$ be algebraic integers. Then each conjugate of $x+y,x-y,xy,xy^{-1}$ are of the form
$x'+y',x'-y',x'y',x'y'^{-1}$ where $x',y'$ are conjugates of $x,y$ respectively.
\end{proposition}
The proof is left as an exercise (see exercise 9).
\subsection{$\mathcal{O}_K$ as finitely generated $\mathbb{Z}$-module}
We proved earlier that $\mathcal{O}_K$ is a ring for any number field $K$. The last topic in the chapter is to prove that $\mathcal{O}_K$ is a finitely generated $\mathbb{Z}$-module.
\begin{lemma} The bilinear form $T$ defined in Definition 12.25 is non-degenerate if $F/K$ is separable.
\end{lemma}
\begin{proof}[\bf Proof] Let $F=K(x)$ for some $x \in F$ and so $\{1,x,\ldots,x^{n-1}\}$ is a basis. Let $x_1=x,\ldots,x_n$ be the conjugates of $x$. Then by Proposition 12.35, the conjugates of $x^r$ are $x^r_1,\ldots,x^r_n$.
Thus the $i,j$ entry of the matrix $T$ is $T(x^{i-1},x^{j-1})$, which is precisely $T_{F/K}(x^{i+j-2})=\sum_{q=1}^n x^{i+j-2}_q$. Let $V$ be the Vandermonde  matrix

\begin{equation*} \begin{pmatrix} 1&1&\ldots&1 \\ x_1 &x_2 &\ldots& x_n \\ .&.&\ldots&. \\ x^{n-1}_1 &x^{n-1}_2 &\ldots & x^{n-1}_n \end{pmatrix} \end{equation*}
Then $T=V^t V$ and so the determinant
$$\det{T}=(\det{V^t})(\det{V})=(\det{V})^2$$
But $x_i \neq x_j$ as $L/K$ separable, so $\det{V} \neq 0$ and so $\det{T} \neq 0$. Therefore, it is non-degenerate.
\end{proof}
\begin{lemma} Suppose $A \subseteq B$ be integral domains and $A$ is integrally closed (normal). Let $F$ be the field of quotient of $A$ and $z \in B$ be integral over $A$. Then the minimal polynomial $P$ of $z$ over $F[x]$ lies in $A[x]$.
\end{lemma}
\begin{proof}[\bf Proof] WLOG, let $P(x)=x^n+e_{n-1}x^{n-1}+\ldots+e_1x+e_0$ with $e_i \in F$. Now $e_i$ is the elementary function of conjugates of $z$ (see Theorem 12.33). As $z$ is integral over $A$, so then $z$ satisfies a monic polynomial $f \in A[x]$, such that $f(z)=0$ and hence $f(z_i)=0$ for all $z_i$ conjugate of $z$ because
$P|f$. Thus $z_i$ is integral over $A$. Hence, the sum and product of $z_i$ are integral over $A$ and so $e_i$ is integral over $A$ for all $i$. But $A$ is integrally closed. Therefore $e_i \in A$ for all $i$ and so $P \in A[x]$.
\end{proof}
\begin{theorem} Let $A$ be a Noetherian domain which is integrally closed. Let $F$ be the field of quotient of $A$ and $K/F$ a finite separable extension. Then the integral closure $A^K$ of $A$ in $K$ is a finitely generated $A$-module. In particular, by taking $A=\mathbb{Z},F=\mathbb{Q}$ and $K$ be any number field, the ring of integers $\mathcal{O}_K$ is a finitely generated $Z$-module.
\end{theorem}
\begin{proof}[\bf Proof] As $K/F$ is finite, it is algebraic and so let $u \in K$, there exist $q_i \in F$ and $n \in \mathbb{N}$ such that
$$u^n+q_{n-1}u^{n-1}+\ldots+q_1u+q_0=0$$
Clearing the denominators, we have $a_i \in A$ such that
$$a_nu^n+a_{n-1}u^{n-1}+\ldots+a_1u+a_0=0$$
Multiply both sides by $a^{n-1}_n$, so
$$(a_n u)^n+(a_n u)^{n-1} +\ldots+(a_n u)a{n-2}_na_1+a^{n-1}_n a_0=0$$
Thus, $a_n u$ is integral over $A$. (This is an analogue of Theorem 12.22.) So $a_n u \in A^K$.
Hence, given any $F$-basis $V \subset K$ of $K$ we can multiply them by some suitable elements so that it is an $F$-basis $V'$ of $K$ such that $V' \subset A^K$.

By Lemma 12.36, the bilinear form $T$ is non-degenerate as $K/F$ separable.
So we can choose a basis $U=\{u_1,\ldots,u_n\} \subset K$ such that
$$T(u_i,u_j)=\delta_{ij}$$
and a basis $V=\{v_1,\ldots,v_n\}$ such that $v_i=a_i u_i$ for some $a_i$ that $v_i \in A^K$. Then
$$T(v_i,v_j)=a_i a_j \delta_{ij}$$
Now for any $x \in A^K$ we have $x=\sum_{j=1}^n x_j v_j$ for some $x_j \in F$ and $v_j \in A^K$.
As $A^K$ is a ring, $xv_i \in A^K$ for all $i$. By Lemma 12.37, as $A$ is integrally closed, so
$\min_{F,xv_i}(x) \in A[x]$, in particular, $T_{K/F}(xv_i) \in A$ because it is the coefficient of subleading term
of $\min_{F,xv_i}(x)$. But
$$T_{K/F}(xv_i)=T_{K/F}(\sum_{j=1}^n x_jv_jv_i)=\sum_{j=1}^n x_j T(v_i,v_j)=a_i x_i$$
So $a_i x_i \in A$ and since $a_i \in A$, we conclude that $x_i \in A$. Thus, $A^K$ is contained in an
$A$-module generated by $v_1,\ldots,v_n$. As $A$ is Noetherian, so $A^K$ is finitely generated.
\end{proof}
\begin{corollary} Let $K$ be a number field. Then $\mathcal{O}_K$ is a Noetherian domain.
\end{corollary}
\begin{proof}[\bf Proof] By Theorem 12.38, $\mathcal{O}_K$ is a finitely generated $Z$-module. Then as $Z$ is Noetherian,
$\mathcal{O}_K$ is Noetherian.
\end{proof}
\subsection{Exercises}
\begin{enumerate}
\item Let $\theta$ be aroot of $x^3+6x+34$. Prove that $\mathbb{Z}[\theta]$ is not integrally closed.
\item Prove that $\mathbb{Z}[\sqrt{2},\sqrt{5}]$ is not integrally closed. Show further that $\mathbb{Z}[\sqrt{2},\sqrt{a}]$ is not integrally closed whenever $a$ is a square free odd integer and hence show that it holds for every square free integer $a$.
\item Let $K=\mathbb{Q}(\sqrt{5})$. Prove that the ring of integer $\mathcal{O}_K=\mathbb{Z}[\frac{1+\sqrt{5}}{2}]$.
\item Let $K=\mathbb{Q}(\sqrt{3})$. Prove that the ring of integer $\mathcal{O}_K=\mathbb{Z}[\sqrt{3}]$.
\item Let $A$ and $B$ be integral domains with $A \subseteq B$ and $B$ integral over $A$. If $I$ is a non-zero ideal of $B$, prove that $I \cap A$ is a non-zero ideal of $A$.
\item Prove that $\frac{10^{\frac{2}{3}}-1}{\sqrt{-3}}$ is an algebraic integer.
\item Let $m$ be a square free integer, and $m \equiv 1$ (mod $4$). Prove that $\mathbb{Z}[\sqrt{m}]$ is not integrally closed.
\item[$^\star$ 8.] Let $F/K$ be a cyclic extension of degree $n$, i.e. $[F:K]=n$ and Hom$_K(F,F)=\{\sigma,\sigma^2,\ldots,\sigma^n=id\}$. Prove that $T_{F/K}(\sigma(x)-x) = 0$ for all $x \in F$. Deduce that if $y \in F$ then $T_{F/K}(y) = 0$
if and only if $y = \sigma(x) - x$ for some $x \in F$. The last part of this question may be referred to {\bf Hilbert's Theorem 90}.
\item[9] Prove Proposition 12.35
\end{enumerate}

\section{Dedekind Domain}
\subsection{Dedekind domains}
We have proved that if $K$ is a number field, then by Theorem 12.24 and Theorem 12.39, $\mathcal{O}_K$ is integrally closed and Noetherian. We shall prove firstly some other important facts of $\mathcal{O}_K$.
\begin{lemma} Let $K$ be a number field. Let $I$ be a non-zero ideal of $\mathcal{O}_K$. Then $I \cap \mathbb{Z} \neq \emptyset$ and $\mathcal{O}_K/I$ is finite.
\end{lemma}
\begin{proof}[\bf Proof] Let $0 \neq x \in I$. Then there exist $a_i \in \mathbb{Z}$ and $n \in \mathbb{N}$ such htat
$$x^n+a_{n-1}x^{n-1}+\ldots+a_1x+a_0=0$$
and we may assume $a_0 \neq 0$ (otherwise we divide the equation by $x$). Then
$$a_0 = -x^n-a_{n-1}x^{n-1}-\ldots-a_1x \in I$$
and so $I \cap \mathbb{Z} \neq \emptyset$.

Now let $\{e_1,\ldots,e_n\}$ generates $\mathcal{O}_K$ as a $\mathbb{Z}$-module and let $\bar{e_i}=e_i$ (mod $I$).
Then $\{\bar{e_1},\ldots,\bar{e_n}\}$ generates $\mathcal{O}_K/I$ over the ring $\mathbb{Z}/I \cap \mathbb{Z}$.
As $\mathbb{Z}$ is a principal ideal domain and so we may assume $I \cap \mathbb{Z}=\langle \alpha \rangle$ and so
$\mathbb{Z}/I \cap \mathbb{Z} \cong \{0,1,\ldots,\alpha-1\}$ is finite. Hence $\mathcal{O}_K/I$ a is finitely generated module by a finite ring, which must be finite.
\end{proof}
\begin{theorem} Every non-zero prime ideal $P \subseteq \mathcal{O}_K$ is maximal.
\end{theorem}
\begin{proof}Since $P$ is prime, $\mathcal{O}_K/P$ is an integral domain and it is finite by Lemma 13.1. But any finite integral domain is a field. Thus $\mathcal{O}_K/P$ is a field and so $P$ is maximal.
\end{proof}
\begin{theorem} For all primes $p \in \mathbb{Z}$, there exists a prime ideal $Q \subset \mathcal{O}_K$ such that
$Q \cap \mathbb{Z} = p\mathbb{Z}$. Moreover, there exists only finitely many such $Q$.
\end{theorem}
\begin{proof}[\bf Proof] If $\langle p \rangle \mathcal{O}_K \neq \mathcal{O}_K$, then pick any maximal ideal $Q$ in $\mathcal{O}_K$ containing $\langle p \rangle \mathcal{O}_K$ and hence prime and clearly $Q \cap \mathbb{Z}=p\mathbb{Z}$ because $p$ is a prime number so $Q$ cannot contain any other integer which is not multiple of $p$, otherwise $Q \cap \mathbb{Z}=\mathbb{Z}$, which is impossible. If $\langle p \rangle \mathcal{O}_K=\mathcal{O}_K$. Then as $1 \in \mathcal{O}_K$, so we must have $\frac{1}{p} \in \mathcal{O}_K$, which is impossible.

Moreover, we look at the prime ideals $Q$ in $\mathcal{O}_K$ containing $\langle p \rangle \mathcal{O}_K$.
But $\mathcal{O}_K/\langle p \rangle \mathcal{O}_K$ is finite by Lemma 13.1. So each prime ideal $Q$ containing
$\langle p \rangle \mathcal{O}_K$ corresponds to a prime ideal in $\mathcal{O}_K/\langle p \rangle \mathcal{O}_K$ and so we must have finitely many of them.
\end{proof}
These give all the properties we need so far for $\mathbb{O}_K$. An integral domain with these properties is called a {\bf Dedekind} domain named after Richard Dedekind, the creator of modern theory of ideals. Formally,
\begin{definition} An integral domain $D$ which satisfies:
\begin{enumerate}
\item $D$ is a Noetherian domain.
\item $D$ is integrally closed.
\item each prime ideal of $D$ is a maximal ideal
\end{enumerate}
is called a {\bf Dedekind} domain.
\end{definition}
Therefore, we have shown that $\mathcal{O}_K$ is a Dedekind domain.
\begin{theorem} Let $D$ be a principal ideal domain. Then $D$ is a Dedekind domain.
\end{theorem}
\begin{proof}[\bf Proof] $D$ is a principal ideal domain and hence Noetherian. Also a unique factorisation domain and hence integrally closed by Theorem 12.17. Finally, each prime ideal in a principal ideal domain is maximal. Hence $D$ is a Dedekind domain.
\end{proof}
The main objective of this chapter is to show that every non-zero proper ideal in a Dedekind domain can be expressed uniquely as a product of prime ideals.
\subsection{Ideals in a Dedekind domain}
The first step of our objective is to show that every proper ideal contains a product of prime ideals.
\begin{theorem} In a Noetherian domain, every non-zero ideal contains a product of prime ideals.
\end{theorem}
\begin{proof}[\bf Proof] Suppose not. Let $D$ be Noetherian which has at least one non-zero ideal that does not contain a product of prime ideals. Let $S$ be the set of such ideals and so $S$ is non-empty. As $D$ is Noetherian, we have an ideal $A$ which is maximal with the above property, i.e. $A \in S$ and if $B \in S$, then $B \subseteq A$. Clearly, $A$ is not prime and so (by Theorem 7.49) we have ideals $B$ and $C$ such that
$$BC \subseteq A, B \not \subseteq A, C \not \subseteq A$$
Define the ideals $B_1$ and $C_1$ of $D$ by
$$B_1=A+B,C_1=A+C$$
and so $A \subset B_1, A\subset C_1$. So $B_1,C_1 \not \in S$ by maximality of $A$.
Then we have prime ideals $P_1,\ldots,P_m$, $Q_1,\ldots,Q_n$ such that
$$P_1\cdots P_m \subseteq B_1,Q_1\cdots Q_n \subseteq C_1$$
But $B_1C_1=(A+B)(A+C) \subseteq A$ because $BC \subseteq A$. So
$$P_1\cdots P_m Q_1\cdots Q_n \subseteq A$$
contradicting $A \in S$.
\end{proof}
As a Dedekind domain is Neotherian, so we have an immediate consequence:
\begin{theorem} In a Dedekind domain every non-zero ideal contains a product of prime ideals.
\end{theorem}
We shall extend the definition of ideals in an integral domain.
\begin{definition} Let $D$ be an integral domain. Let $K$ be the field of quotient of $D$. A non-empty set $I$ of $K$ with the following properties:
\begin{enumerate}
\item[(i)] $\alpha \in I,\beta \in I \Rightarrow \alpha+\beta \in I$.
\item[(ii)] $\alpha \in I, r \in D \Rightarrow r \alpha \in I$.
\item[(iii)] there exists $\gamma \in D$ with $\gamma \neq 0$ such that $\gamma I \subseteq D$
\end{enumerate}
is called a {\bf fractional ideal} of $D$. Equivalently, it is a $D$-submodule of $K$ such that there exists $0 \neq \gamma$ with $\gamma I \subseteq D$. The element $\gamma$ can be thought of as a  `common denominator'.
\end{definition}
\begin{example} The set
$$I=\left\{\frac{n}{25}: n \in \mathbb{Z} \right\}$$
is a fractional ideal of $\mathbb{Z}$ because clearly it satisfies (i) and (ii) and $25I \subseteq \mathbb{Z}$.
\end{example}
\begin{example} The set
$$I=\left\{\frac{n}{3^m}: n \in \mathbb{Z}, m \in \mathbb{N} \cup {0} \right\}$$
is not a fractional ideal of $\mathbb{Z}$ because we have no such $\gamma$ with $\gamma I \subseteq \mathbb{Z}$.
\end{example}
It is clear that the usual ideal of $D$ is also a fractional ideal of $D$ by taking $\gamma =1$. We often refer to the usual ideals of $D$ in the ordinary sense as {\bf integral ideal}. It is easy to check that if $I$ is a fractional ideal and $\gamma$ is the `common denominator', then $\gamma I$ is an integral ideal of $D$. Therefore,
if $I$ is a fractional ideal, then
$$I=\frac{1}{\gamma}I'$$
for some integral ideal $I'$ of $D$. Note this representation is non unique as we have
$$\frac{1}{\gamma}I'=\frac{1}{\gamma \delta}(\delta I')$$

In a Dedekind domain $D$, as $D$ is Noetherian, so every ideal is finitely generated, say
$$I'=\langle \alpha_1,\ldots,\alpha_n \rangle$$
Then
$$I=\frac{1}{\gamma}{I'}=\frac{1}{\gamma} \langle \alpha_1,\ldots,\alpha_n \rangle$$
that is, every fractional ideal $I$ of $D$ is also finitely generated. We can also extend the definition of
sum and product of ideals to fractional ideals. It is easily checked that (see exercise 2) if $I_1,I_2$ are fractional ideals of $D$, so are $I_1+I_2,I_1I_2$.
\begin{definition} Let $D$ be an integral domain and $K$ the field of quotient. For any fractional ideal $I$ of $D$, define
$$I^{-1}=\{x \in K: xI \subseteq D\}$$
\end{definition}
\begin{lemma} Let $D$ be an integral domain and $K$ the field of quotient. Then $I^{-1}$ is a fractional ideal for any fractional ideal $I$ of $D$.
\end{lemma}
\begin{proof}[\bf Proof] We check that $I^{-1}$ satisfies the properties of a fractional ideal.
\begin{enumerate}
\item[(i)] Let $\alpha,\beta \in I^{-1}$, then $(\alpha+\beta) x =\alpha x + \beta x \in D$ for all $x \in I$ because $D$ is a ring and so $\alpha+\beta \in I^{-1}$.
\item[(ii)] Let $\alpha \in I^{-1}$ then $\alpha I \subset D$ and so $r\alpha I \subset D$ for all $r \in D$.
\item[(iii)] Pick any $0 \neq \gamma \in I$ and $\alpha \in I^{-1}$ then $\gamma \alpha \in D$ because $\alpha I \subseteq D$. Therefore, $\gamma I^{-1} \subseteq D$
\end{enumerate}
Thus, $I^{-1}$ is a fractional ideal of $D$.
\end{proof}
The notation of $I^{-1}$ suggests that it is in some sense the inverse of $I$. We need several steps to prove this. We shall firstly deduce that
\begin{theorem} Let $D$ be a Dedekind domain. Let $P$ be a prime ideal of $D$. Then $P P^{-1}=D$.
\end{theorem}
\begin{proof}[\bf Proof] As $P$ is an integral ideal of $D$, so $1 \in P^{-1}$ and so $P \subseteq PP^{-1}$. But in a Dedekind domain, every prime ideal is maximal, so $PP^{-1}=P$ or $D$. Further, if $\alpha \in D$, then $\alpha P \subseteq D$ , and so we have $D \subseteq P^{-1}$.

Next we show that $D$ is strictly contained in $P^{-1}$. Let $0 \neq \beta \in P$. By Theorem 13.7 there exist
prime ideals $P_1,\ldots,P_k$ $(k \ge 1$) such that
$$P \subseteq \langle \beta \rangle \supseteq P_1\cdots P_k$$
Let $k$ be the least integer such that this inclusion holds. As $P$ is a prime ideal, so we have
by Theorem 7.49 that $P_i \subseteq P$ for some $i$ and we may relabel $P_1$ as $P_1$. then we have
$P_1 \subseteq P$. But $P_1$ is prime and $D$ is a Dedekind domain, so $P_1$ is maximal and so
$$P_1=P$$

If $k=1$ then $P_1=P=\langle \beta \rangle$ and $\beta$ is not a unit.
Let $\gamma =\frac{1}{\beta} \in K$ so that
$$\gamma P=\frac{1}{\beta} \langle \beta \rangle = \langle 1 \rangle =D$$
and so $\gamma \in P^{-1}$. But $\gamma \not \in D$ because $\beta$ is not a unit. Thus $D \subset P^{-1}$ in this case.

If $k \ge 2$, then by minimality of $k$, we have
$$P_2 \cdots P_k \not \subseteq \langle \beta \rangle$$
Hence there exists $\delta \in P_2\cdots P_k$ such that $\delta \not \in \langle \beta \rangle$.
Define $\gamma =\frac{\delta}{\beta} \in K$, and so $\gamma \not \in D$ because $\delta \not \in \langle \beta \rangle$. Then
$$P \langle \delta \rangle =P_1 \langle \delta \rangle \subseteq P_1 \cdots P_k \subset \langle \beta \rangle$$
so
$$P\gamma=P \frac{\delta}{\beta} \subseteq D$$
and thus $\gamma \in P^{-1}$. So in both cases we have proved that there exists some $\gamma \in P^{-1}$ but not in $D$. Hence $D \subset P^{-1}$.

Finally, we show that $PP^{-1} \neq P$ and so we have $PP^{-1}=D$. Suppose $PP^{-1}=P$.
Then if $\alpha \in P^{-1}, \beta \in P^{-1}$, we have
$$\alpha P \subseteq P, \beta P \subseteq P$$
and so
$$\alpha \beta P \subseteq \alpha P \subseteq P \subset D$$
Thus, $\alpha \beta \in P^{-1}$ and so $P^{-1}$ is an integral domain in $D$. As $D$ is Noetherian, $P^{-1}$ is finitely generated $D$-module. Then in Lemma 12.9, take $A$ to be $D$ and $B$ to be $P^{-1}$ here. Let $C=B=P^{-1}$ in Lemma 12.9 (iii) then (iii) $\Rightarrow$ (i) implies that for any $x \in P^{-1}$, $x$ is integral over $D$ and hence $P^{-1}$ is integral over $D$. But $D$ is integrally closed so we have $D=P^{-1}$ which contradicts $D \subset P$. Therefore, $PP^{-1}=D$.
\end{proof}
\subsection{Factorisation into prime ideals}
In this subsection we shall prove the fundamental property of a Dedekind domain $D$, namely, that every proper
integral ideal of $D$ can be expressed uniquely up to reordering as a product of prime ideals.\\
Note: There is a proof using the knowledge of Artinian ring which will be given later when we introduce the concept of Artianian ring. The proof given here is however, based on the Theorem 13.13 and is elementary.
\begin{theorem} Let $D$ be a Dedekind domain. Then every integral ideal is a product of prime ideals and this factorisation is unique up to reordering. In other words, if
$$P_1\cdots P_k=Q_1 \cdots Q_l$$
Then $k=l$ and $P_i=Q_i$ after relabeling.
\end{theorem}
\begin{proof}[\bf Proof] Suppose there exist integral ideals of $D$ which are not products of prime ideals. Let $S$ be the set of such ideals. As $D$ is Dedekind, it is Noetherian, and so there exists an ideal $A$ which is maximal with such property, i.e. $A \in S$ and $J \subseteq A$ if $J \in S$ and since $D$ is Noetherian, there exists a maximal ideal $B$ containing $A$. Let $C$ be the inverse of $B$, $C=B^{-1}$. Then as $B$ is an integral ideal, it is clear that
$D \subseteq C$ and so
$$A=AD \subseteq AC \subseteq BC =D$$
Therefore, from above $AC$ is an integral ideal of $D$.

Now as $B$ is a maximal ideal, so $B$ is prime and so from the proof of Theorem 13.13, we have $D \subset C$.
Thus $A \subset AC$ and so by maximality of $A$, there exist prime ideals $P_1,\ldots,P_k$ ($k \ge 1$) such that
$$AC=P_1 \cdots P_k$$
Multiply both sides by $B$ so we have
$$A=P_1 \cdots P_k B$$ which is a product of prime ideals and so $A \not \in S$, which is a contradiction.

To prove uniqueness, suppose we have two prime factorisation
$$P_1 \cdots P_k = Q_1 \cdots Q_l$$
Then
$$P_1 \cdots P_k \subseteq Q_1$$
and so as $Q_1$ is prime, we have
$$P_i \subseteq Q_1$$ for some $i$. We may relabel so that $P_1 \subseteq Q_1$.
As $P_1$ is prime and so $P_1$ is maximal so $P_1=Q_1$ since $Q_1 \neq D$.
Thus we may multiply the inverse $P^{-1}_1$ on both sides so that
$$P_2 \cdots P_k=Q_2 \cdots Q_l$$
Continue the above process inductively and thus we conclude that $k=l$ and $P_i =Q_i$ after relabeling.
\end{proof}
If we use the notation of exponential, i.e. $P^k$ is the the multiplication of $k$ lots of $P$, and
$P^{-k}$ means $(P^{-1})^k$ for $k >0$, then we have
\begin{theorem} Let $D$ be a Dedekind domain. Then every fractional ideal $I$ can be factorised uniquely into the form
$$\prod_{i=1}^k P_i^{e_i}$$
for some $k$ and $P_1,\ldots,P_k$, where $P_i$ is prime ideal and $e_i \neq 0$ for all $i$.
\end{theorem}
\begin{proof}[\bf Proof] The case when $I$ is an integral ideal is proved in Theorem 13.14. To prove it also holds for fractional ideal, we simply notice that there exists $\gamma$ such that $\gamma I \subseteq D$. So we have a
factorisation for $\gamma I$ and hence for $I$ since we have $I=\frac{1}{\gamma}(\gamma I)$.
The uniqueness is proved in a similar manner. If
$$P_1^{e_1} \cdots P_k^{e_k}=Q_1^{f_1} \cdots Q_l^{f_l}$$
Then we multiply both sides by suitable $P_i^{e_i}$ and $Q_j^{f_j}$ to delete the negative exponential. Then the proof is identical to the integral ideal case.
\end{proof}
\begin{example} Let $D=\mathbb{Z}[\sqrt{-5}]$, as $D$ is $\mathbb{O}_K$ for $K=\mathbb{Q}[\sqrt{-5}]$ so it is a
Dedekind domain. $D$ is not a unique factorisation domain as $6=2 \cdot 3=(1+\sqrt{-5})(1-\sqrt{-5})$. But by Theorem 13.15 the ideal $\langle 6 \rangle$ can be uniquely factorised into prime ideals. We shall see how this happens.
Let
$$P_1=\langle 2,1+\sqrt{-5} \rangle, P_2=\langle 3,1+\sqrt{-5} \rangle, P_3=\langle 3,1-\sqrt{-5} \rangle$$
Then we have
\begin{eqnarray*}
P_2P_3&=&\langle 3,1+\sqrt{-5} \rangle \langle 3,1-\sqrt{-5} \rangle\\
&=& \langle 9, 3(1+\sqrt{-5}),3(1-\sqrt{-5}),6 \rangle\\
&=& \langle 3 \rangle \langle 3,1+\sqrt{-5},1-\sqrt{-5},2 \rangle\\
&=&\langle 3 \rangle \langle 1 \rangle =\langle 3 \rangle
\end{eqnarray*}
and clearly $P_1=\langle 2,1+\sqrt{-5} \rangle=\langle 2,1-\sqrt{-5} \rangle$ so
\begin{eqnarray*}
P^2_1&=&\langle 2,1+\sqrt{-5} \rangle \langle 2,1-\sqrt{-5} \rangle\\
&=&\langle 4,2(1+\sqrt{-5}),2(1-\sqrt{-5}),6 \rangle\\
&=&\langle 2\rangle \langle 2,1+\sqrt{-5}, 1-\sqrt{-5},3\rangle \\
&=&\langle 2\rangle \langle 1 \rangle= \langle 2 \rangle
\end{eqnarray*}
Similarly,
\begin{eqnarray*}
P_1P_2&=&\langle 2,1+\sqrt{-5} \rangle \langle 3,1+\sqrt{-5} \rangle\\
&=&\langle 6,2(1+\sqrt{-5}),2(1-\sqrt{-5},(1+\sqrt{-5})^2 \rangle\\
&=&\langle 1+\sqrt{-5} \rangle \langle 1-\sqrt{-5},2,3,1+\sqrt{-5} \rangle\\
&=&\langle 1+\sqrt{-5} \rangle \langle 1\rangle=\langle 1+\sqrt{-5} \rangle
\end{eqnarray*}
and
\begin{eqnarray*}
P_1P_3&=&\langle 2,1-\sqrt{-5} \rangle \langle 3,1-\sqrt{-5} \rangle\\
&=&\langle 6,2(1-\sqrt{-5}),3(1-\sqrt{-5}),(1-\sqrt{-5})^2 \rangle\\
&=&\langle 1-\sqrt{-5} \rangle \langle 1+\sqrt{-5},2,3,1-\sqrt{-5} \rangle\\
&=&\langle 1-\sqrt{-5} \rangle \langle 1\rangle=\langle 1-\sqrt{-5} \rangle
\end{eqnarray*}
Thus, we see that
$$\langle 2 \rangle =P^2_1, \langle 3 \rangle =P_2P_3,
\langle 1+\sqrt{-5} \rangle =P_1P_2, \langle 1-\sqrt{-5} \rangle =P_1P_3$$
and so we have
$$\langle 6 \rangle =\langle 2 \rangle \langle 3 \rangle =\langle 1+\sqrt{-5} \rangle \langle 1-\sqrt{-5} \rangle
=P^2_1P_2P_3$$
\end{example}
\begin{definition} Let $D$ be a Dedekind domain. Let $A$ and $B$  be non-zero integral ideals of $D$. We say $A$
{\bf divides} $B$, written $A|B$, if there exists an integral ideal $C$ of $D$ such that $B=AC$.
\end{definition}
It is clear that once we have unique factorisation of each integral ideal, we have many similar properties of
the ideals in Dedekind domain as integers in $\mathbb{Z}$. For example, Let
$$A=\prod_{i=1}^n P_i^{e_i} \text{ and } B=\prod_{i=1}^n P_i ^{f_i}$$
$e_i,f_i \ge 0$. Then $A|B \iff e_i \le f_i$ for all $i$.
More importantly, the factorisation of fractional ideal suggests that the set of non-zero fractional ideals form a group under multiplication. The only property we need to check now is that, for every fractional ideal $I$, the inverse exists and is $I^{-1}$. We have proved this for prime ideal, and so it is also true for any non-zero fractional ideal by writing out the prime factorisation and take inverse for each prime ideal, formally
\begin{theorem} The set of all non-zero integral and fractional ideals of a Dedekind domain $D$ forms an
Abelian group under multiplication. The identity of the group is $D$ and the inverse of $A=\prod_{i=1}^nP_i^{e_i}$
, where $P_1,\ldots,P_n$ distinct and $e_i \mathbb{Z}$ is
$$A^{-1}=\prod_{i=1}^n P_i ^{-e_i}$$
\end{theorem}
\subsection{Order of an ideal}
We shall extend the definition of divisibility of integral ideal to fractional ideal.
\begin{definition} Let $D$ be a Dedekind domain. Let $A$ and $B$ be non-zero fractional ideals of $D$. We say $A$ divides $B$, written $A | B$ if there exists an integral ideal $C$ of $D$ such that $B=AC$.
\end{definition}
Further, we shall introduce the notation:
\begin{definition} Let $D$ be a Dedekind domain, and $A=\prod_{i=1}^n P_i^{e_i}$ a fractional ideal of $D$.
Then we write $ord_{P_i}(A)=e_i$, which is called the {\bf order} of $A$ with respect to $P_i$.
\end{definition}
Then clearly, if $A|B$ then $ord_{P_i}(A) \le ord_{P_i}(B)$ for any $P_i$ in the factorisation.
The next theorem gives a necessary and sufficient condition for an ideal $A$ to divide an ideal $B$.
\begin{proposition} Let $D$ be a Dedekind domain. Let $A$ and $B$ be any non-zero fractional ideal of $D$. Then
$$A|B \iff A \supseteq B$$
\end{proposition}
\begin{proof}[\bf Proof] $A|B \iff BA^{-1}$ is an integral ideal and so we have
$$A|B \iff BA^{-1} \subseteq D \iff B \subseteq A$$
\end{proof}
The two basic properties of order is given in the next theorem.
\begin{proposition} Let $D$ be a Dedekind domain and $P$ be a prime ideal of $D$. Let $A$ and $B$ be non-zero fractional ideals of $D$. Then
\begin{enumerate}
\item[(i)] $ord_P(AB)=ord_P(A)+ord_P(B)$.
\item[(ii)] $ord_P(A+B)=\min{(ord_P(A),ord_P(B))}$.
\end{enumerate}
\end{proposition}

\begin{proof}

Let $$A=P^{e} \prod_{i=1}^nP_i^{e_i},B=P^{f} \prod_{i=1}^m Q_i$$ where $P_i, Q_i \neq P$. Then
\begin{enumerate}
\item[(i)]
$ord_P(AB)=e+f=ord_P(A)+ord_P(B)$.
\item[(ii)] Let $C=A+B$. Then $AC^{-1}+BC^{-1}=D$. Hence $AC^{-1},BC^{-1} \subseteq D$ and so both of them are integral. Also if $AC^{-1},BC^{-1} \subseteq P$ then $D \subseteq P$ which is impossible. So we may assume
    $AC^{-1} \not \subseteq P$ and so $\min{(ord_P(AC^{-1}),ord_P(BC^{-1}))}=0$. Thus by (i),
    $$\min{(ord_P(A),ord_P(B))}=ord_P(C)=ord_P(A+B)$$
\end{enumerate}
\end{proof}
We extend the order of an ideal to order of an element.
\begin{definition} Let $D$ be a Dedekind domain with field of quotient $K$. For $0 \neq \alpha \in K$, define
$$ord_P(\alpha)=ord_P(\langle \alpha \rangle)$$
the {\bf order} of $\alpha$ with respect to a prime ideal $P$.
\end{definition}
\begin{lemma} Let $D$ be a Dedekind domain with field of quotient $K$. Let $A$ be a fractional ideal of $D$ and $0 \neq \alpha \in K$. Then
$$\alpha \in A \iff ord_P(\alpha) \ge ord_P(A) \text{ for all prime ideals } P \text{ of } D$$
\end{lemma}
\begin{proof}
$$\alpha \in A \iff \langle \alpha \rangle \subseteq A \iff ord_P(\alpha) \le ord_P(A)$$
The last step is clear if $P$ divides $\alpha$ or $A$. If $P$ does not divide either of them, then the order is $0$ for both of them.
\end{proof}
The following proposition extends Proposition 13.22
\begin{lemma} Let $D$ be a Dedekind domain with quotient field $K$. Let $P$ be a prime ideal of $D$. Then
for any non-zero $\alpha,\beta \in K$
\begin{enumerate}
\item[(i)] $ord_P(\alpha \beta)=ord_P(\alpha)+ord_P(\beta)$.
\item[(ii)] If $\alpha+\beta \neq 0$, then
$$ord_P(\alpha+\beta) \ge \min{(ord_P(\alpha),ord_P(\beta))}$$
\item[(iii)] If $\alpha+\beta \neq 0$, and $ord_P(\alpha) \neq ord_P(\beta)$, then
equality holds in (ii).
\end{enumerate}
\end{lemma}
\begin{proof}
\begin{enumerate}
\item[(i)] This follows from proposition 13.22 as $ord_P(\alpha \beta)=ord_P(\langle \alpha \beta \rangle)$.
\item[(ii)] Since $\alpha+\beta \in \langle \alpha \rangle+ \langle \beta \rangle$, thus by Lemma 13.24, we have
    $$ord_P(\alpha+\beta) \ge ord_P(\langle \alpha \rangle +\langle \beta \rangle)$$
    Hence the result follows by Proposition 13.22(ii)
\item[(iii)] WLOG, assume that $ord_P(\alpha) > ord_P(\beta)$.
Then we have
$$ord_P(\beta)=ord_P((\alpha+\beta)-\alpha) \ge \min{(ord_P(\alpha+\beta),ord_P(\alpha))}$$
Now since $ord_P(\beta) < ord_P(\alpha)$ so the above can only hold if
$$\min{(ord_P(\alpha+\beta),ord_P(\alpha))}=ord_P(\alpha+\beta)$$
Then
$$ord_P(\beta) \le ord_P(\alpha+\beta) \ge ord_P(\beta)$$
and so the equalities must hold and so
$$ord_P(\alpha+\beta)=ord_P(\beta)=\min{(ord_P(\alpha),ord_P(\beta))}$$
\end{enumerate}
\end{proof}
\begin{example} We give an example to show that the inequality can actually be strict in Proposition 13.25(ii).
Let $D=\mathbb{Z}, \alpha=1,\beta=2, P=\langle 3 \rangle$. Then
$$ord_P(\alpha)=0,ord_P(\beta)=0, ord_P(\alpha+\beta)=1$$
\end{example}
\begin{theorem} Let $D$ be a Dedekind domain and $K$ be the field of quotient. Given any finite set of prime ideal
$P_1,\ldots,P_k$ of $D$ and a set of integers $e_1,\ldots,e_k$ then there exists $\alpha \in K$ such that
$$ord_{P_i}(\alpha)=e_i, i=1,2,\ldots,k$$
and
$$ord_P(\alpha) \ge 0 \text{ for any prime ideal distinct from } P_i,i=1,2,\ldots,k$$
\end{theorem}
\begin{proof}
For the prime ideal $P_i$ and integer $e_i$. Consider
$$I_i=P^{e_i}_i \prod_{j \neq i}P^{e_j+1}_j, J_i= \prod_{j=1}^k P^{e_j+1}_j$$
$I_i |J_i$ and so by Proposition 13.21, $J_i \subseteq I_i$.
By unique factorisation, we have $J_i \neq I_i$ and so $J_i \subset I_i$. Hence there exists
$\alpha_i$ for each $i$ such that
$$\alpha_i \in I_i \text{ but } \alpha_i \not \in J_i$$
Then $ord_{P_i}(\alpha_i)=e_i$ by Lemma 13.24. Also $ord_{P_j}(\alpha_i) \ge e_j+1$ for $j \neq i$, again by Lemma 13.24. Now define
$$\alpha=\alpha_1+\ldots+\alpha_k \in K$$
By repeatedly applying Lemma 13.25(ii), we have
$$ord_{P_i}\left(\sum_{j \neq i}\alpha_j\right) \ge \min_{j \neq i}(ord_{P_i}(\alpha_j)) \ge e_i+1>ord_{P_i}\left(\alpha_i\right)$$
Therefore, using $\sum_{j=1}^k \alpha_j=\alpha_i +(\sum_{j \neq i} \alpha_j)$, we have
$$ord_{P_i}\left(\alpha\right)=ord_{P_i}\left(\sum_{j=1}^k \alpha_j\right)=\min{\left(ord_{P_i}(\alpha_i),ord_{P_i}(\sum_{j \neq i}\alpha_j)\right)}
=ord_{P_i}\left(\alpha_i\right)$$
by using Lemma 13.25(iii). Hence we have
$$ord_{P_i}(\alpha)=e_i$$
Finally, if $P \neq P_i$ for any $i$, then
$$ord_P(\alpha_i) \ge 0$$
and so
$$ord_P(\alpha) \ge 0$$
\end{proof}
\subsection{Chinese Remainder Theorem}
The last topic of this subsection is an analogue of Chinese Remainder Theorem in Dedekind domain.
Let $D$ be a Dedekind domain and $K$ be the field of quotient. Let $I$ be a fractional ideal. We shall write
for any $a,b \in I$
$$a \equiv b~(\text{mod } I) \iff I| \langle a-b \rangle$$
It is easy to check that
$$I| \langle a-b \rangle \iff \langle a-b \rangle \subseteq I \iff a-b \in I \iff a+I=b+I$$
and if $c \in I$ then
$$ac \equiv bc~(\text{mod } I)$$
\begin{theorem}{\bf [Chinese Remainder Theorem]}\label{C;Chinese Remainder theorem Dedekind}
Let $D$ be a Dedekind domain.
\begin{enumerate}
\item[(i)] Let $P_1,\ldots,P_k$ be distintct prime ideals in $D$. Let $e_1,\ldots,e_k$ be positive integers and $\alpha_1,\ldots,\alpha_k \in D$. Then there exists $\alpha \in D$ such that $$\alpha \equiv \alpha_i~(\text{mod }P^{e_i}_i), i=1,2,\ldots,k$$
\item[(ii)] Let $I_1,\ldots,I_k$ be pairwise relatively prime ideals of $D$. (in the sense of factorisation into prime ideals.) \ Let $\alpha_1,\ldots,\alpha_k$ be elements of $D$. Then there exists $\alpha \in D$ such that
$$\alpha \equiv \alpha_i~(\text{mod }I_i),i=1,2,\ldots,k$$
\end{enumerate}
\end{theorem}
\begin{proof}
\begin{enumerate}
\item[(i)] Let $A_i=P^{e_i}_i, B_i=\sum_{j \neq i} P^{e_j}_j$, then define
$$Q_i=A_i+B_i$$
Suppose $P$ is any prime ideal dividing $Q_i$. Since $A_i,B_i \subseteq Q_i$, then by Proposition
13.21, we have
$$Q_i \big| A_i \text{ and } Q_i \big| B_i$$
and so
$$P \big|A_i \text{ and } P\big| B_i$$
But $A_i=P^{e_i}_i$ and since $P_i$ is prime. Then $P \big|P_i$, and so by maximality of prime ideal in Dedekind domain, we have $P=P_i$. Thus $P_i \big| B_i$ is a contradiction. Hence there exists no prime ideal dividing $Q_i$
and so $Q_i=D$ for each $i$.
Thus we have
$$x_i \in A_i, y_i \in B_i$$
such that
$$x_i+y_i=1$$
which implies that
$$y_i \equiv 1~(\text{mod } P^{e_i}_i) \text{ and } y_i \equiv 0~(\text{mod } P^{e_j}_j), j \neq i$$
Now define
$$\alpha=\sum_{i=1}^k e_iy_i$$
and so
$$\alpha \equiv e_i ~(\text{mod } P^{e_i}_i$$
\item[(ii)] This follows from the unique factorisation of each $I_i$ into prime ideals and apply (i) since $I_i$ are relatively prime to each other.
\end{enumerate}
\end{proof}
We shall see many other properties of a Dedekind domain in the exercises.
\subsection{Exercises}
\begin{enumerate}
\item Let $D$ be a Dedekind domain. Let $A$ and $B$ be proper integral ideals of $D$. Show that $AB \neq D$.
\item Let $D$ be a Dedekind domain. Let $I$ and $J$ be fractional ideals. Show that $I+J$ and $IJ$ are also fractional ideals.
\item Let $D$ be a principal ideal domain with field of quotient $K$. Prove that every fractional ideal
of $D$ is of the form $\{d\alpha: d \in D\}$ for some $\alpha \in K$.
\item Let $K$ be a number field and $\mathcal{O}_K$ the ring of integer. If $[K:\mathbb{Q}]=n$, show that every ordinary integer $\alpha \in \mathbb{Z}$ lies in at most $\alpha^n$ integral ideals of $\mathcal{O}_K$.
\item Let $K=\mathbb{Q}[\sqrt{-5}]$, find the inverse of $\langle 2, 1+\sqrt{-5} \rangle$.
\item Let $D$ be a Dedekind domain. Show that if $D$ is a unique factorisation domain, then $D$ is a principal ideal domain.
\item Let $D$ be a Dedekind domain. Let $I$ and $J$ be integral ideals of $D$ and let
 $$I=\prod_{i=1}^k P^{e_i}_i, J=\prod_{i=1}^k P^{f_i}_i$$
where $P_1,\ldots,P_k$ are all possible divisors of $I$ or $J$ and $e_i, f_i$ are non-negative integers.
Define the greatest common divisor $(I,J)$ and the least common multiple $\{I,J\}$ by:
$$(I,J)=\prod_{i=1}^k P^{\min{(e_i,f_i)}}_i, \{I,J\}=\prod_{i=1}^k P^{\max{(e_i,f_i)}}_i$$
Show that
$$(I,J)=I+J \text{ and } \{I,J\}=I \cap J$$
\item  Prove that the cancelation law holds in Dedekind domain. That is, if $D$ is a Dedekind domain, and
$I,J,K$ are ideals of $D$ such that $I \neq 0$. Then
$$IJ=IK \iff J=K$$
\item Let $K$ be a number field. Prove that $\mathcal{O}_K$ contains infinitely many prime ideals.
\item Prove the ring version of Chinese Remainder Theorem in Dedekind domain. That is, if $D$ is a Dedekind domain, then
$$D/I \cong D/P^{e_1}_1 \times \cdots \times D/P^{e_k}_k$$
where
$$I=\prod_{i=1}^k P^{e_i}_i, e_i \ge 1$$
\item Let $K$ be a number field and $\mathcal{O}_K$ be its ring of integers.
      \begin{enumerate}
      \item[(i)] Let $P$ be a prime ideal. Show that the only non-zero proper ideals in the ring
      $\mathcal{O}_K/P^n$ are $P^{j}\mathcal{O}_K/P^n$, $j=1,2,\ldots,n-1$.
      \item[(ii)] Show that if $\alpha \in P \backslash P^2$, then
      $$P^i=\langle \alpha^i \rangle +P^n, i=1,2,\ldots,n$$
      \item[(iii)] Hence show that any ideal of the ring $\mathcal{O}_K/I$ can be generated by one element where $I$ is any non-zero integral ideal of $\mathcal{O}_K$.
      \item[(iv)] Show further that any fractional ideal in $\mathcal{O}_K$ can be generated by at most two elements.
      \end{enumerate}
\item[$^\star$12.] Let $D$ be a Dedekind domain and $I$ any non-zero ideal of $D$ such that $I \neq D$. Let $0 \neq \alpha \in I$.
      \begin{enumerate}
      \item[(i)] Show that $\langle \alpha \rangle = IJ$ for some integral ideal $J$. Let $P_1,\ldots,P_k$ be all possible prime ideals such that either
          $$ord_{P_i}(I) \neq 0 \text{ or } ord_{P_i}(IJ) \neq 0$$
      Prove that that there exists $\beta \in K$, the field of quotient of $D$ such that
      $$ord_{P_i}(\beta)=ord_{P_i}(I), i=1,2,\ldots,k$$
      and
      $$ord_P(\beta) \ge 0, P \neq P_1,\ldots,P_k$$
      \item[(ii)] Show further that
      $$ord_P(I)=ord_P(\langle \beta \rangle +IJ) \text{ for all prime ideals } P$$
      \item[(iii)] Hence conclude that $I$ is generated by at most two elements. Thus, we have proved that in any Dedekind domain $D$, any ideal of $D$ is generated by at most two elements in $K$. We see that question 11 is a special case.
      \end{enumerate}
\end{enumerate}

\section{Discriminant and norm}
\subsection{Discriminant}
We know $\mathcal{O}_K$ is a subring of the number field $K$. How large is $\mathcal{O}_K$ in terms of $K$? We shall introduce the concept of discriminant to measure the `size' of $\mathcal{O}_K$.
\begin{definition} Suppose $M$ is a finitely generated free $\mathbb{Z}$-module, and
$$T: M \times M \rightarrow \mathbb{Q}$$
a bilinear form. Pick a basis $\{e_1,\ldots,e_n\}$ of $M$. Let $A$ be the matrix such that $A_{ij}=T(e_i,e_j)$. Then the {\bf discriminant} of $M$ with respect to the bilinear form $T$ is defined as
$$D(M)=disc(M)=\det{(A)}$$
When $M=\mathcal{O}_K$ for any number field $K$, we take $T$ be the bilinear form defined in Definition 12.25, that $T_(x,y)=T_{K/\mathbb{Q}}(xy)$.
\end{definition}
Note the definition makes sense as we have proved that $\mathcal{O}_K$ is a finitely generated $\mathbb{Z}$-module. Further, in the proof of Theorem 12.38, if $[K:\mathbb{Q}]=n$, then $\mathcal{O}_K$ is generated by $n$ elements.
Thus $A$ is an $n \times n$ matrix. But suppose we pick another basis for $M$, does it change the discriminant?
\begin{lemma} $disc(M)$ is independent of the choice of $\mathbb{Z}$-basis.
\end{lemma}
\begin{proof}[\bf Proof] Let $\{e_1,\ldots,e_n\}$ be a basis and $\{f_1,\ldots,f_n\}$ be another basis. Let $P$ be the change of basis matrix and $B_{ij}=T(f_i,f_j)$. Then
$$B=P A P^t, \det{B}=(\det{P})^2 \det{A}$$
But $P$ is the change of basis matrix so $\det{P}$ is invertible in $\mathbb{Z}$. Hence $\det{P}= \pm 1$ and so
$(\det{P})^2=1$. Hence $\det{B}=\det{A}$.
\end{proof}
Observe that for $x,y \in \mathbb{Z}$, $T(x,y) \in \mathbb{Z}$. For shorthand, write $\mathcal{D}_k=disc(\mathcal{O}_K)$. We can also define discriminant for other finitely generated $Z$-module, for example, the ideals. Since each ideal in $\mathcal{O}_K$ is a submodule of $\mathcal{O}_K$ and hence finitely generated $\mathbb{Z}$-module.
\begin{lemma} Let $K$ be a number field and $[K:\mathbb{Q}]=n$. Let $\{x_1,\ldots,x_n\}$ be a $\mathbb{Z}$-basis of $M$. Let $x_{i,1}=x_i,\ldots,x_{i,n}$ be conjugates of $x_i,i=1,\ldots,n$. Then
$$disc(M)=(\det{X})^2$$ where $X_{ij}=x_{ij}$.
\end{lemma}
\begin{proof}[\bf Proof] Let $A_{ij}=T(x_i,x_j)$. So the conjugates of $x_ix_j$ are $x_{ip}x_{jp}$, $p=1,\ldots,n$. But
$$T(x_i,x_j)=T_{K/\mathbb{Q}}(x_ix_j)=\sum_{p=1}^n x_{ip}x_{jp}$$
Define the matrix $X_{ip}=x_{ij}$. Then
$$A=XX^t \text{ and so } \det{A}=(\det{X})^2$$
\end{proof}
Now, let $e_1,\ldots,e_n$ be $n$ elements in $K$ and let
$$e^{(1)}_i=\sigma_1(e_i),e^{(2)}_i=\sigma_2(e_i),\ldots,e^{(n)}_i=\sigma_n(e_i)$$
For convention, we define
\begin{equation*}
D(e_1,\ldots,e_n)= \begin{vmatrix} e^{(1)}_1 & \ldots & e^{(1)}_n\\ \vdots & \ldots &\vdots\\
e^{(n)}_1 & \ldots & e^{(n)}_n \end{vmatrix}^2
\end{equation*}
and for any element $\alpha \in K$, we define $D(\alpha)=D(1,\alpha,\ldots,\alpha^n)$.
\begin{example} Let $m$ be a non-zero square free integer and $m \neq 1$ and $K=\mathbb{Q}[\sqrt{m}]$. It is easy to check that if $m \not \equiv 1$ (mod $4$) then $\mathcal{O}_K=\mathbb{Z}[\sqrt{m}]$ and if $m \equiv 1$ (mod $4$),
then $\mathcal{O}_K=\mathbb{Z}[\frac{1+\sqrt{m}}{2}]$.

If $m \not \equiv 1$ (mod $4$), then $\{1,\sqrt{m}\}$ is a basis for $\mathcal{O}_K$ and so
\begin{equation*} \mathcal{D}_K = \left(\begin{vmatrix} 1&\sqrt{m}\\1&-\sqrt{m} \end{vmatrix}\right)^2
=(-2\sqrt{m})^2=4m
\end{equation*}

If $m \equiv 1$ (mod $4$), then $\{1,\frac{1+\sqrt{m}}{2}\}$ and so
\begin{equation*} \mathcal{D}_K = \left(\begin{vmatrix} 1&\frac{1+\sqrt{m}}{2}\\1&\frac{1-\sqrt{m}}{2} \end{vmatrix}\right)^2
=(\sqrt{m})^2=m
\end{equation*}
We can see that $\mathbb{Z}[\sqrt{m}] \subseteq \mathbb{Z}[\frac{1+\sqrt{m}}{2}]$ and so this indicates that the larger the $\mathbb{Z}$-module, the smaller the discriminant.
\end{example}
\begin{theorem} Suppose $N \subseteq M$ is a $\mathbb{Z}$-submodule which has index $r$. In other words,
$M/N$ is a finite Abelian group of order $r$. Then
$$disc(N)=r^2 disc(M)$$
\end{theorem}
\begin{proof}[\bf Proof] Let $\{e_1,\ldots,e_n\}$ and $\{f_1,\ldots,f_n\}$ be bases for $M$ and $N$ respectively.
Since $N \subseteq M$, so there exist $a_{ij} \in \mathbb{Z}$ such that
$$f_i=\sum_{j=1}^n a_{ij}e_j,i=1,\ldots,n$$
By structure theorem for finitely generated module over $\mathbb{Z}$ (Smith Normal Form), we can obtain
$x_1,\ldots,x_n \in M$ and $r_i \in \mathbb{Z}$ such that $\{x_1,\ldots,x_n\}$ is a basis for $M$ and
$\{r_1x_1,\ldots,r_nx_n\}$ is a basis for $N$. Then it is easy to see that
$$\{x_1z_1+\ldots x_nz_n: z_1=0,1,\ldots,|r_1|-1;\ldots;z_n=0,1,\ldots,|r_n|-1\}$$
is a complete set of coset representatives for $M/N$. So
$r=r_1\cdots r_n$. (Moreover, by the algorithm, we have
$r=\det{(a_{ij})}$.)
Hence $$disc(N)=(\det{X})^2=r^2 disc(M)$$
where $X_{ij}=\text{the } j^{th} \text{ conjugate of }r_ix_i=r_ix_{ij}$ because $r_i \in \mathbb{Z}$.
Hence $\det{X}=\prod_i r_i \det{(x_{ij})}$ and so $\det{X}=r^2 disc(M)$.
\end{proof}
\begin{corollary} If $N \subseteq \mathcal{O}_K$ has $disc(N)$ square free. Then $N=\mathcal{O}_K$.
\end{corollary}
\begin{proof}[\bf Proof] By Theorem 14.5, the index of $N$ in $\mathcal{O}_K$ must be $1$ since $disc(N)$ is square free.
Hence $\mathcal{O}_K/N$ is a group of order $1$ and so $\mathcal{O}_K=N$.
\end{proof}
Theorem 14.5 also tells that if $N \subseteq M$ and $disc(N)=disc(M)$, then $N=M$.
\subsection{Discriminant of a polynomial}
When we solve the quadratic equation
$$ax^2+bx+c=0$$ we usually define the discriminant $d$ to be
$b^2-4ac$. Then the two roots are
$$\frac{-b \pm \sqrt{d}}{2a}$$
What about the discriminant of general polynomial?
\begin{definition} Let
$$f(x)=a_nx^n+a_{n-1}x^{n-1}+\ldots+a_1x+a_0 \in \mathbb{C}[x]$$
where $n \in \mathbb{N}$ and $a_n \neq 0$. Let $x_1,\ldots,x_n$ be the roots of $f(x)$. The discriminant of $f(x)$ is defined by
$$disc(f(x))=a^{2n-2}_n\prod_{1 \le i <j \le n}(x_i-x_j)^2$$
\end{definition}
\begin{theorem} Let $K$ be a number field od degree $n$. Let $\alpha \in K$. Then
$$D(\alpha)=disc(\min_{\mathbb{Q},\alpha}(x)^d)$$
where $d$ is an integer such that $d e=n, e=\deg{\min_{\mathbb{Q},\alpha}(x)}$.
\end{theorem}
\begin{proof}[\bf Proof] By Vanmdermonde determinant, it is easy to check that
$$D(\alpha)=\prod_{1 \le i <j \le n}\left(\alpha^{(i)}-\alpha^{(j)}\right)^2$$
where $\alpha^{(1)},\ldots,\alpha^{(n)}$ are conjugates of $\alpha$ in $K$.
So the roots of $\min_{\mathbb{Q},\alpha}(x)^d$ are $\alpha^{(1)},\ldots,\alpha^{(n)}$, and so the discriminant
$$disc(\min_{\mathbb{Q},\alpha}(x)^d)=\prod_{1 \le i <j \le n}\left(\alpha^{(i)}-\alpha^{(j)}\right)^2=D(\alpha)$$
because the leading coefficient is $1$.
\end{proof}
The following theorem gives an criteria for the generator of the number field $K$.
\begin{theorem} Let $K$ be a number field of degree $n$. Let $\alpha \in K$. Then
$$K=\mathbb{Q}(\alpha) \iff D(\alpha) \neq 0$$
\end{theorem}
\begin{proof}
$$K=\mathbb{Q}(\alpha) \iff \min_{\mathbb{Q},\alpha}(x) \text{ has } n \text{ distinct root } \iff D(\alpha)\neq 0$$
by using Vandermonde determinant.
\end{proof}
\begin{theorem} Let $K$ be a number field of degree $n$.
\begin{enumerate}
\item[(i)] If $e_1,\ldots,e_n \in K$, then $$D(e_1,\ldots,e_n) \in \mathbb{Q}$$
\item[(ii)] If $e_1,\ldots,e_n \in \mathcal{O}_K$, then $$D(e_1,\ldots,e_n) \in \mathbb{Z}$$
In particular, $\mathcal{D}_K \in \mathbb{Z}$.
\item[(iii)] If $e_1,\ldots,e_n \in K$, then
$$D(e_1,\ldots,e_n) \neq 0 \iff e_1,\ldots,e_n \text{ are linearly independent over } \mathbb{Q}$$
\end{enumerate}
\end{theorem}
\begin{proof}
\begin{enumerate}
\item[(i)] Let $K=\mathbb{Q}(\alpha)$ for some $\alpha \in K$. Let $e^{(j)}_i$ be the conjugates of $e_i$,
$i=1,\ldots,n; j=1,\ldots,n$. Then there exist $a_{ij} \in \mathbb{Q}$ such that
$$e^{(j)}_i=a_{0i}+a_{1i}\alpha_j+\cdots+a_{n-1 i}\alpha^{n-1}_j$$
Then by definition, $D(e_1,\ldots,e_n)$ is the square of the determinant of the matrix whose entries are now of the form
$$a_{0i}+a_{1i}\alpha_j+\cdots+a_{n-1 i}\alpha^{n-1}_j$$
for $1 \le i,j \le n$. If we swap $i$ and $j$ then it is same as swapping rows and columns of the matrix, which may only change the sign of the determinant. But we take the square of the determinant so it then will be the same if we swap $i$ and $j$. Hence it is a symmetric function in $\alpha_1,\ldots,\alpha_n$. Now let
$$(x-\alpha_1)\cdots(x-\alpha_n)=x^n+b_{n-1}x^{n-1}+\cdots+b_1x+b_0$$
where $b_0,\ldots,b_n \in \mathbb{Q}$.
Then the symmetric function in $\alpha_1,\ldots,\alpha_n$ can be expressed in terms of $b_0,\ldots,b_n$ and so it must be rational. So $D(e_1,\ldots,e_n) \in \mathbb{Q}$.
\item[(ii)] Since $e_1,\ldots,e_n \in \mathcal{O}_K$ and $\mathcal{O}_K$ is a ring, so $D(e_1,\ldots,e_n)$ is obtained by the determinant of the matrix whose entries are algebraic integers. But the determinant is obtained by a series of addition and multiplication. Hence it is still an algebraic integer. By (i) it also rational, hence must be an integer.
\item[(iii)]  Let $K=\mathbb{Q}(\alpha)$ Suppose $e_1,\ldots,e_n$ are linearly independent over $\mathbb{Q}$, then $\{e_1,\ldots,e_n\}$ is a basis and so there exists $a_{ij} \in \mathbb{Q}$ such that
$$\alpha^{i-1}=\sum_{j=1}^n a_{ij}e_j$$
Let $\alpha_1=\alpha,\ldots,\alpha_n$ be the conjugates of $\alpha$ and
$e^{(k)}_j, k=1,\ldots,n$ be the conjugates of $e_j$ for each $j$.
Then
$$\alpha^{i-1}_k=\sum_{j=1}^n a_{ij}e^{(k)}_j$$
Therefore, we have
$$D(\alpha)=|\det{(a_{ij})}|^2 D(e_1,\ldots,e_n)$$
By Theorem 14,9, $D(\alpha) \neq 0$ and so $D(e_1,\ldots,e_n) \neq 0$.

Conversely, suppose $e_1,\ldots,e_n$ are linearly dependent. Then there exist $a_1,\ldots,a_n \in \mathbb{Q}$
such that
$$a_1e_1+\cdots a_ne_n=0$$
and so
$$\sum_{i=1}^n a_i e^{(k)}_i=0$$
where $e^{(k)}_i, k=1,\ldots,n$ are conjugates of $e_i$ for each $i$.
This is a system of $n$ linear equations in $n$ quantities $c_1,\ldots,c_n$ and has a non-trivial solution. Therefore, we must have the determinant
$$D(e_1,\ldots,e_n)=0$$
\end{enumerate}
\end{proof}
\subsection{Norm of an ideal}
\begin{definition} Let $I$ be an integral ideal of $\mathcal{O}_K$. The {\bf norm} of $I$, written $N(I)$ is defined as
$$N(I)=\sqrt{\frac{D(I)}{\mathcal{D}_K}}$$
\end{definition}
The main results we are going to deduce in this subsection is
$$N(I)=\card(\mathcal{O}_K/I)$$
\begin{theorem} Let $G$ be a free $\mathbb{Z}$-module generated by $e_1,\ldots,e_n$. Let $H \subseteq G$ generated by
$f_1,\ldots,f_n$ so that
$$H=\{y_1f_1+\ldots+y_nf_n: y_1,\ldots,y_n \in \mathbb{Z}\}$$
As each $f_i \in H \subseteq G$, we have
$$f_i =\sum_{j=1}^n a_{ij}e_i \text{ where } a_{ij} \in \mathbb{Z}$$
Let $A$ be the matrix such that $A_{ij}=a_ij$. Then
\begin{equation*}
[G:H]= \left\{
\begin{array}{ll}
|\det{A}| & \text{if } \det{A} \neq 0\\
\infty & \text{if } \det{A}=0\\
\end{array} \right.
\end{equation*}
\end{theorem}
\begin{proof}[\bf Proof] This is in fact a restatement of Theorem 14.5. We can find a $\mathbb{Z}$-basis $\{x_1,\ldots,x_n\}$ for $G$ such that there exists $r_i \in \mathbb{Z}$ such that $\{r_1x_1,\ldots,r_nx_n\}$ is a basis for $H$ by applying elementary operations to the matrix $A$ to the diagonal matrix $D=diag(r_1,\ldots,r_n)$. So
$\det{A}=\det{D}=d_1\cdots d_n$. So it is clear that if $r_i \neq 0$ for each $i$, then
$$z_1x_1+\ldots z_nx_n \in H \iff z_i=y_ir_i, y_i \in \mathbb{Z}, i=1,\ldots,n \iff r_i |z_i$$
and so in this case $\det{A} \neq 0$ and we can check that
$\{z_1x_1+\ldots+z_nx_n: z_1=0,1,\ldots,|r_1|-1; \ldots; z_n=0,1\ldots,|r_n|-1\}$ form a complete set of coset representatives for $G/H$. Thus
$$[G:H]=|r_1| \cdots |r_n|=|\det{A}|$$
If $\det{A}=0$, i.e. $r_i=0$ for some $i$. Then $kx_i+H, k=1,2,\ldots$ are distinct cosets of $H$ in $G$ so that
$[G:H]=\infty$.
\end{proof}
\begin{theorem} Let $K$ be a number field with $[K:\mathbb{Q}]=n$. Let $I$ be a non-zero integral ideal of $\mathcal{O}_K$. Then
$$N(I)=\card(\mathcal{O}_K/I)$$
\end{theorem}
\begin{proof}[\bf Proof] Let $\{e_1,\ldots,e_n\}$ be a $\mathbb{Z}$-basis for $\mathcal{O}_K$ and $\{f_1,\ldots,f_n\}$ be a
$\mathbb{Z}$-basis for $I$. Then there exists $a_{ij} \in \mathbb{Z}$ such that
$$f_i=\sum_{j=1}^n a_{ij}e_i, i=1,2,\ldots,n$$
Therefore, by Theorem 14.8,
$$N(I)=\sqrt{\frac{D(I)}{\mathcal{D}_K}}=r=\det{A}=[\mathcal{O}_K:I]=\card(\mathcal{O}_K/I)$$
where $r$ is the index of $I$ in $\mathcal{O}_K$ in Theorem 14.5.
\end{proof}
\subsection{Norm of a product of ideals}
The main result we are going to deduce n the subsection is $N(IJ)=N(I)N(J)$ for any integral ideals $I,J$. We shall give two proofs. The first proof requires some ring theory and Chinese Remainder Theorem.
\begin{lemma} Let $K$ be a number field and $P$ a prime ideal of $\mathcal{O}_K$. Then, for any $n \ge 1$,
$$P^n\mathcal{O}_K/P^{n+1} \cong \mathcal{O}_K/P$$
\end{lemma}
\begin{proof}[\bf Proof] Since $P^n \supset P^{n+1}$, so there exists $\alpha \in P^n$ and $\alpha \not \in P^{n+1}$.
Define a ring homomorphism:
$$f: \mathcal{O}_K \rightarrow P^n\mathcal{O}_K/P^{n+1}$$
by $f(x) \equiv x\alpha$ (mod $P^{n+1}$). Then
$$f(x)=0 \iff x\alpha \equiv 0~(\text{mod } P^{n+1}) \iff x \in P$$
So the kernel of $f$ is $P$.

Now suppose we have $c \in P^n$. Then by Chinese Remainder Theorem, we have $\beta \in \mathbb{O}_K$ such that
$$\beta \equiv c~(\text{mod } P^{n+1}) \text { and } \beta \equiv 0~(\text{mod } I)$$
where $IP^n=\langle \alpha \rangle$, because $P^n \big| \langle \alpha \rangle$.
Clearly, since $c \in P^n$ so if $\beta-c \in P^{n+1}$ then $\beta=c+d$ for some $d \in P^{n+1}$ and so
$$\beta \in P^n+P^{n+1} \in P^n$$ Then $P^n \big| \langle \beta \rangle$.
Also $\beta \in I$ and so $I \big| \langle \beta \rangle$. But $P \nmid I$ because $\alpha \not \in P^{n+1}$, so
$P^n$ and $I$ are coprime. So
$$\langle \alpha \rangle=P^n I \big| \langle \beta \rangle$$
and so $\frac{\beta}{\alpha} \in \mathcal{O}_K$. Further,
$$f\left(\frac{\beta}{\alpha}\right)=\beta \equiv c~(\text{mod } P^{n+1})$$
Hence, the map is also surjective. The result follows by using isomorphism theorem.
\end{proof}
\begin{theorem} Let $K$ be a number field. Let $I$ and $J$ be non-zero integral ideals of $\mathcal{O}_K$. Then
$$N(I)N(J)=N(IJ)$$
\end{theorem}
\begin{proof}[\bf Proof] Suppose $I$ and $J$ are coprime, then the result follows by the ring version of Chinese Remainder Theorem (exercise 10 of chapter 13). Also, since each ideal can be factorised into product of prime ideals, so we only need to check for the case $I=P^m,J=P^n$ for some $m,n \ge 1$ and $P$ prime ideals.

Now since
$$\mathcal{O}_K/P^n \cong \mathcal{O}_K/P \times P\mathcal{O}_K/P^2 \times \cdots \times P^{n-1}\mathcal{O}_K/P^n$$
and for each $i$,
$$P^i \mathcal{O}_K/P^{i+1} \cong \mathcal{O}_K/P$$
by Lemma 14.14. Therefore,
$$\card(\mathcal{O}_K/P^n)=\left(\card(\mathcal{O}_K/P)\right)^n$$
Use this, we have
$$\card(\mathcal{O}_K/P^{m+n})=\card(\mathcal{O}_K/P^n)\card(\mathcal{O}_K/P^m)
=\left(\card(\mathcal{O}_K/P)\right)^{m+n}$$
This proves the remaining case and hence the result follows.
\end{proof}
The second proof is independent of the first one, It uses the order of ideals, together with Chinese Remainder Theorem.
We shall begin with:
\begin{lemma} Let $I$ be a non-zero fractional ideal of $\mathcal{O}_K$ with $I \neq \mathcal{O}_K$. Let $J$ be a non-zero integral ideal of $\mathcal{O}_K$ with $J \neq \mathcal{O}_K$. Then there exists $\gamma \in I$, such that
$$I=\langle \gamma \rangle +IJ$$
\end{lemma}
\begin{proof}[\bf Proof] Let $P_1,\ldots,P_n$ be all possible prime ideals such that either
$$ord_{P_i}(I) \neq 0 \text{ or } ord_{P_i}(IJ) \neq 0$$
This is non-empty since $I \neq \mathcal{O}_K$. Then by Theorem 13.27, there exists $\gamma \in K$ such that
$$ord_{P_i}(\gamma)=ord_{P_i}(I), i=1,2,\ldots,n$$
and
$$ord_P(\gamma) \ge 0 \text{ for all prime ideals } P \neq P_1,\ldots,P_n$$
Then it is clear that
$$ord_P(\gamma) \ge ord_P(I) \text{ for all prime ideals } P$$
and so $\gamma \in I$ by Lemma 13.24.
Now for $i=1,2,\ldots,n$ we have
\begin{eqnarray*}
ord_{P_i}(\langle \gamma \rangle +IJ)&=&\min{(ord_{P_i}(\langle \gamma \rangle),ord_{P_i}(IJ))}\\
&=&\min{(ord_{P_i}(I),ord_{P_i}(IJ))}\\
&=&ord_{P_i}(I)
\end{eqnarray*}
because $J$ is an integral ideal. For any prime ideal $P \neq P_1,\ldots,P_n$, we have $ord_P(IJ)=ord_P(I)=0$, and so
$$ord_P(\langle \gamma \rangle+IJ)=\min{(ord_P(\langle \gamma \rangle),ord_P(IJ))}=0=ord_P(I)$$
Hence $ord_P(\langle \gamma \rangle +IJ)=ord_P(I)$ for all prime ideals $P$ and so
$$I=\langle \gamma \rangle +IJ$$
\end{proof}
\begin{theorem} For any non-zero integral ideals $I,J \in \mathcal{O}_K$, we have
$$N(IJ)=N(I)N(J)$$
\end{theorem}
\begin{proof}[\bf Proof] This is clear if $I$ or $J=\mathcal{O}_K$. Hence we may assume $I,J \neq \mathcal{O}_K$ and let
$m=N(I),n=N(J)$. Then by Theorem 14.13, $\card(\mathcal{O}_K/I)=m,\card(\mathcal{O}_K/J)=n$, where $m,n >1$.
Let
$$\mathcal{O}_K/I=\alpha_1+I,\ldots,\alpha_m+I$$
and
$$\mathcal{O}_K/J=\beta_1+J,\ldots,\beta_n+J$$
By Lemma 14.12, there exists $\gamma \in I$ such that
$$I=\langle \gamma \rangle +IJ$$
and $\gamma \neq 0$ otherwise $J=\mathcal{O}_K$. For any $x \in \mathcal{O}_K$, we have some $i$ such that
$$x \equiv \alpha_i~(\text{mod } I)$$
Thus, $x-\alpha_i \in I=\langle \gamma \rangle +IJ$.
So there exists $d \in \mathcal{O}_K$ and $y \in IJ$ such that
$$x -\alpha_i=\gamma d+ y$$
Then there exists some $j$ such that
$$d \equiv \beta_j~(\text{mod } J)$$
which implies $d-\beta_j \in J$.
As $\gamma \in A$, we have
$$(d-\beta_j)\gamma \in IJ$$
and so
$$x=\alpha_i+\gamma d+y=\alpha_i+\beta_j \gamma +(d-\beta_j)\gamma +y \equiv \alpha_i+\beta_j \gamma
~(\text{mod }IJ)$$
This shows that $\alpha_i+\beta_j \gamma +IJ$ for ms a complete set of coset representatives of $\mathcal{O}_K/IJ$.
Now we shall show they are distinct.
Suppose $$\alpha_i+\beta_j\gamma+IJ=\alpha_p+\beta_q\gamma+IJ$$
Then
$$\alpha_i+\beta_j\gamma \equiv \alpha_p+\beta_q\gamma ~(\text{mod } IJ)$$
and hence
$$\alpha_i-\alpha_p \equiv (\beta_q-\beta_j)\gamma~(\text{mod }IJ)$$
But $IJ \subseteq I$ and $\gamma \in I$ so $\alpha_i-\alpha_p \in I$ and so $i=p$.
So we have
$$(\beta_j-\beta_q)\gamma \in IJ$$
Let $P$ be any prime ideal which divides $J$. Then
\begin{eqnarray*}
ord_P(I)&=&ord_P(\langle \gamma \rangle +IJ)\\
&=&\min{(ord_P(\gamma),ord_P(IJ))}\\
&=&\min{(ord_P(\gamma),ord_P(I)+ord_P(J))}
\end{eqnarray*}
Since $ord_P(J) \ge 1$ because $J$ is integral. So the above implies that
$$ord_P(I)=ord_P(\gamma) \text{ for any prime ideal } P \text{ which divides } J$$
Suppose $\beta_j \neq \beta_q$, then $(\beta_j-\beta_q)\gamma$ is a non-zero element of $IJ$, and so for any $P$ which divides $J$,  we have
$$ord_P((\beta_j-\beta_q)\gamma) \ge ord_P(IJ)$$
and so
$$ord_P(\beta_j-\beta_q)+ord_P(\gamma) \ge ord_P(I)+ord_P(J)$$
which implies that
$$ord_P(\beta_j-\beta_q) \ge ord_P(J)$$
Also if $P$ does not divide $J$ then $ord_P(J)=0$ but $ord_P(\beta_j-\beta_q) \ge 0$.
So $\beta_j-\beta_q \in J$ by Lemma 13.24, contradicting $\beta_j \neq \beta_q$.
Therefore, $\beta_j=\beta_q$. This proves that each
$\alpha_i+\beta_j\gamma+IJ$ is distinct from others and so
$$\card(\mathcal{O}_K/IJ)=mn$$
\end{proof}
\subsection{Norm of an element}
We have seen the definition of norm of an element in Definition 12.35. In this subsection we are going to discuss the relation between $N(\alpha)$ and $N(\langle \alpha \rangle)$ for any $\alpha \in \mathcal{O}_K$.
Recall that the norm of an element $\alpha$ is defined by
$$N(\alpha)=\prod_{i=1}^n \sigma_i(\alpha)$$
and by previous knowledge, we know that $\sigma_1(\alpha),\ldots,\sigma_n(\alpha)$ are conjugates of $\alpha$.
\begin{theorem} Let $K$ be a number field of degree $n$. Let $\alpha \in \mathcal{O}_K$. Then
$$N(\langle \alpha \rangle)=|N(\alpha)|$$
\end{theorem}
\begin{proof}[\bf Proof] Let $\{e_1,\ldots,e_n\}$ be a basis for $\mathcal{O}_K$ and so $\{\alpha e_1,\ldots,\alpha e_n\}$ is a basis for $\langle \alpha \rangle$. Then the discriminant
\begin{equation*} D(\langle \alpha \rangle)= \begin{vmatrix} \sigma_1(\alpha e_1) &\ldots& \sigma_1(\alpha e_n)\\
\vdots &\ldots& \vdots\\
\sigma_n(\alpha e_1)&\ldots &\sigma_n(\alpha e_n) \end{vmatrix}^2
\end{equation*}
But since $\sigma$ is homomorphism, so $\sigma(\alpha e_i)=\sigma(\alpha) \sigma(e_i)$. Hence
\begin{equation*} D(\langle \alpha \rangle)= \prod_{i=1}^n (\sigma_i(\alpha))^2 \begin{vmatrix} \sigma_1(e_1) &\ldots& \sigma_1(e_n)\\
\vdots &\ldots& \vdots\\
\sigma_n(e_1)&\ldots& \sigma_n(e_n) \end{vmatrix}^2=N(\alpha)^2 \mathcal{D}_K
\end{equation*}
Therefore,
$$N(\langle \alpha \rangle)=\sqrt{\frac{D(\langle \alpha \rangle)}{\mathcal{D}_K}}=\sqrt{N(\alpha)^2}
=|N(\alpha)|$$
\end{proof}
\begin{theorem} Let $K$ be a number field of degree $n$. Let $\alpha \in K$. Let
$$\min_{\mathbb{Q},\alpha}(x)=x^m+a_{m-1}x^{m-1}+\ldots+a_1x+a_0 \in \mathbb{Q}[x]$$
Then
$$N(\langle \alpha \rangle)=|a_0|^{\frac{n}{m}}$$
\end{theorem}
\begin{proof}[\bf Proof] Clearly, as $\mathbb{Q}(\alpha) \subseteq K$, so $m|n$. Let $\alpha_1=\alpha,\ldots,\alpha_m$ be conjugates of $\alpha$ in $\mathbb{Q}(\alpha)$, and so
$$\min_{\mathbb{Q},\alpha}(x)=\prod_{i=1}^m (x-\alpha_i)$$
and so
$$a_0=(-1)^m \prod_{i=1}^m \alpha_i$$
Further, the conjugates of $\alpha \in K$ are:
$$\alpha,\alpha,\ldots,\alpha;\alpha_1,\ldots,\alpha_1;\ldots,\alpha_m,\ldots,\alpha_m$$
with each $\alpha_i$ repeated $\frac{n}{m}$ times.
Thus, by Theorem 14.18,
$$N(\langle \alpha \rangle)=|N(\alpha)|=|(-1)^n a^{\frac{n}{m}}_0|=|a_0|^{\frac{n}{m}}$$
\end{proof}
\subsection{Norm of fractional ideal}
In the last subsection of the chapter, we shall extend the definition of norm of an integral ideal to the norm of fractional ideal.
\begin{definition} Let $K$ be a number field and $\mathcal{O}_K$ be the ring of integers. Let $J$ be a fractional ideal and so there exists $0 \neq \gamma \in \mathcal{O}_K$ such that
$$J=\frac{1}{\gamma}I$$ for some integral ideal $I$.
We shall define the {\bf norm} of $J$ by
$$N(J)=\frac{N(I)}{N(\langle \gamma \rangle)}$$
\end{definition}
\begin{proposition} Definition 14.20 is well-defined. In other words, if
$$J=\frac{1}{\alpha}I_1=\frac{1}{\beta}I_2$$
for integral ideals $I_1,I_2$ and $0 \neq \alpha,\beta \in \mathcal{O}_K$, then
the norm of $J$ is independent of whether we choose $\alpha$ or $\beta$ in the first place.
\end{proposition}
\begin{proof}[\bf Proof] Let
$$J=\frac{1}{\alpha}I_1=\frac{1}{\beta}I_2$$
Then
$$\alpha I_2=\beta I_1 \Rightarrow \langle \alpha \rangle I_2 =\langle \beta \rangle I_1$$
as they are all integral ideals, then by Theorem 14.11, we have
$$N(\langle \alpha \rangle)N(I_2)=N(\langle \beta \rangle)N(I_1)$$
and hence
$$\frac{N(I_1)}{N(\langle \alpha \rangle)}=\frac{N(I_2)}{N(\langle \beta \rangle)}$$
Therefore, the norm of $J$ is independent of the choice of denominator.
\end{proof}
The next theorem shows that the multiplicative property for the norm of integral ideals carries over to fractional ideals.
\begin{theorem} Let $K$ be a number field. Let $I$ and $J$ be fractional ideals of $\mathcal{O}_K$. Then
$$N(IJ)=N(I)N(J)$$
\end{theorem}
\begin{proof}[\bf Proof] There exist $0 \neq \alpha,\beta \in \mathcal{O}_K$ such that
$$I=\frac{1}{\alpha} K_1, J=\frac{1}{\beta} K_2$$
for some integral ideals $K_1,K_2$ of $\mathcal{O}_K$.
Then
$$IJ=\frac{1}{\alpha \beta}K_1 K_2$$
and so
$$N(IJ)=\frac{N(K_1 K_2)}{N(\langle \alpha \beta \rangle)}$$
By Theorem 14.15,
$$N(K_1K_2)=N(K_1)N(K_2)$$
and $\langle \alpha \beta \rangle =\langle \alpha \rangle \langle \beta \rangle$. Hence
$$N(IJ)=\frac{N(K_1)}{N(\langle \alpha \rangle)} \frac{N(K_2)}{N(\langle \beta \rangle)}=N(I)N(J)$$
\end{proof}
Here is a useful remark.
\begin{remark} For any integral ideal $I \subseteq \mathcal{O}_K$, we have
$$[I^{-1}: \mathcal{O}_K]=[\mathcal{O}_K:I]$$
We shall prove this by establishing a map:
$$f: I^{-1} \rightarrow \mathcal{O}_K/I$$
by
$$f(x) \equiv xy~(\text{mod } I) \text{ where } y \in I \text{ is fixed }$$
It is clear that $f$ is a module homomorphism. Now suppose $xy \in I$. If we have some $x \not \in \mathcal{O}_K$
such that $xy \in I$, then $I |\langle x \rangle \langle y \rangle$ which is impossible by considering the factorisation of ideals. Hence the kernel is precisely $\mathcal{O}_K$. Then by module isomorphism theorem, we have
$$I^{-1}/\mathcal{O}_K \cong \mathcal{O}_K/I$$
and hence the result follows.
\end{remark}
\subsection{Exercises}
\begin{enumerate}
\item Let $K$ be a number field of degree $n=r+2s$, where $r$ is the number of real embeddings and $2s$ is the number of complex embeddings. Show that the sign of discriminant $\mathcal{D}_K$ is $(-1)^s$.
\item Calculate the discriminant of the polynomial $f(x)=x^3+ax+b$ where $a,b \in \mathbb{R}$. Hence or otherwise, show that $f(x)$ has three real distinct roots if $-4a^3-27b^2>0$, one real root, two non-real complex roots if $-4a^3-27b^2<0$ and at least two equal real roots if $-4a^3-27b^2=0$.
\item Let $K=\mathbb{Q}(\alpha)$, where $\alpha^3-4\alpha+2=0$. Let $\theta=\alpha+\alpha^2$. Compute $D(\theta)$. Is $K=\mathbb{Q}(\theta)?$
\item Let $K$ be a number field of degree $n$ and let $K=\mathbb{Q}(\alpha)$. Suppose that every root of
$\min_{\mathbb{Q},\alpha}(x)$ lies in $K$.
\begin{enumerate}
\item[(i)] Let $\mathcal{D}_K=d^2$, where $d=\det{(\sigma_i(x_j))}$, $\sigma_1,\ldots,\sigma_n$ are embeddings and $\{x_1,\ldots,x_n\}$ is a basis for $\mathcal{O}_K$. Recall from the definition of determinant. For any matrix $A$,
    $$\det{A}=\sum_{\tau \in S_n} sgn(\tau)\prod_{i=1}^n A_{i \tau(i)}$$
     Let $d=P-N$ where $P$ is the sum in the determinant corresponds to even permutations in $S_n$ and $N$
     corresponds to odd permutations. Show that $P+N,PN \in \mathbb{Z}$
\item[(ii)] Hence prove that
$$\mathcal{D}_K \equiv 0,1~(\text{mod } 4)$$
The result is true in general and is known as Stickelberger's criteria, that
$$\mathcal{D}_K \equiv 0,1~(\text{mod } 4)$$
which can be proved by using what we have done and taking {\bf Galois closure} for the general number field $K$.
\end{enumerate}
\item In this question we are going to deduce the discriminant of the polynomial which has the form
$$f(x)=x^n+px+q$$
\begin{enumerate}
\item[(i)] Consider the polynomial
$$P(x)=\prod_{i=1}^n (x-x_i)$$
Show that
$$disc(P(x))=(-1)^{\frac{n(n-1)}{2}}\prod_{i=1}^nP'(x_i)$$
where $P'(x)$ is the derivative of $P(x)$.
\item[(ii)] Let $f(x)=x^n+px+q$. Show that
$$x_i f'(x_i)=(n-1)p \left(-\frac{nq}{(n-1)p}-x_i \right)$$
and deduce that
$$disc(f)=(-1)^{\frac{n(n-1)}{2}}((1-n)^{n-1}p^n+n^n q^{n-1})$$
\end{enumerate}
\item Let $K=\mathbb{Q}(\sqrt{p})$
\begin{enumerate}
\item[(i)] Suppose $p$ is a prime such that $p \equiv \pm 3$ (mod $8$). Prove that there does not exist an element $\alpha \in \mathcal{O}_K$ such that $N(\alpha)=2$.
\item[(ii)] Suppose $p$ is a prime such that $p \equiv 5 \text{ or } 7$ (mod $8$). Prove that there does not exist an element $\alpha \in \mathcal{O}_K$ such that $N(\alpha)=-2$.
\end{enumerate}
\item Let $K$ be a number field and let $I$ be a proper integral ideal. Show that
\begin{enumerate}
\item[(i)] If $N(I)$ is a prime, then $I$ is a prime ideal.
\item[(ii)] $\langle N(I) \rangle \subseteq I$.
\end{enumerate}
\item Let $K$ be a number field and $P$ a prime ideal of $\mathcal{O}_K$. Prove that
$$G=\{a+P: a \in \mathcal{O}_K, a \not \in P\}$$
is a cyclic group under multiplication. What is the order of $G$?
\item Let $K$ be a number filed and $I$ an integral ideal of $\mathcal{O}_K$ such that
$N(I)=|N(\alpha)|$ for some $\alpha \in I$. Prove that $I=\langle \alpha \rangle$.
\item Let $K$ be a number field and $P$ a prime ideal of $\mathcal{O}_K$. Show that $P \cap \mathbb{Z}=\langle p \rangle$ for some prime number $p$.
\item Let $K$ be a number field. Let $n$ be a given positive integer. Prove that there are only finitely many integral ideals $I$ of $\mathcal{O}_K$ such that $N(I)=n$.
\item Let $K$ be a number field and $P$ a prime ideal of $\mathcal{O}_K$. Let $a \in \mathcal{O}_K$ such that
$P \nmid \langle a \rangle$. Prove that
$$a^{N(P)}-1 \equiv 0~(\text{mod } P)$$
\item Let $K$ be a number field and $I$ an integral ideal of $\mathcal{O}_K$ such that $pq \big|N(I)$, where
$p$ and $q$ are distinct prime numbers. Prove that $I$ is not a prime ideal.
\item Let $K$ be a number field of degree 2. Let $\alpha \in \mathcal{O}_K$ such that $|N(\alpha)|=pq$, where
$p$ and $q$ are coprime positive integers. Prove that
$$\langle \alpha,p \rangle \langle \alpha,q \rangle =\langle \alpha \rangle$$
\item[$^\star$ 15.] Let $K$ be a number field. Define the {\bf inverse different} $\mathcal{R}^{-1}_K$ by
$$\mathcal{R}^{-1}_K= \{x \in K : T_{K/\mathbb{Q}}(xy) \in Z, \forall y \in \mathcal{O}_K\}$$
\begin{enumerate}
\item[(i)] Show that $\mathcal{R}^{-1}_K$ is a fractional ideal.
Now define the {\bf different} to be the inverse of $\mathcal{R}^{-1}_K$, written $\mathcal{R}_K$
\item[(ii)] Show that $\mathcal{R}_K$ is an integral ideal of $\mathcal{O}_K$.
\item[(iii)] Show that $N(\mathcal{R}_K)=|\mathcal{D}_K|$.
\item[(iv)] Assume $\mathcal{O}_K=\mathbb{Z}[x]$ for some $x$ and let $f$ be the minimal polynomial of $x$ over $\mathbb{Q}$. Suppose $x_1=x,\ldots,x_n$ are conjugates of $x$. Show that
$$\frac{1}{f(T)}=\sum_{i=1}^n\frac{1}{f'(x_i)(T-x_i)}$$
\item[(v)] Deduce that \begin{equation*}
T_{K/\mathbb{Q}}\left(\frac{x^r}{f'(x)}\right)= \left\{
\begin{array}{ll}
0 & \text{if } 0 \le r <n-1\\
1 & \text{if } r=n-1\\
\end{array} \right.
\end{equation*}
\item[(vi)] Deduce that $$\mathcal{R}_K =\langle f'(x) \rangle$$
\end{enumerate}
\item[$^\star$ 16.] Let $K$ be a number field. Define the Dedekind $\zeta$ function $\zeta_K(s)$ by
$$\zeta_K(s)=\sum_{0 \neq I \subseteq O_K}\frac{1}{N(I)^s}$$
summing over all non-zero integral ideals.
\begin{enumerate}
\item[(i)] Show that
$$\zeta_K(s)=\prod_P \frac{1}{1-N(P)^{-s}}$$
where the sum is taking over all non-zero prime ideals $P$. Show that the product converges on the region
$\mathcal{R}e(s)>1$.
\item[(ii)] We have seen that the primes in $\mathbb{Z}[i]$ are (up to associate):\\
$1\pm i$;\\
$p$ with $p \equiv 3$ (mod $4$);\\
$a+ib$ with $a^2+b^2=p$ where $p$ is a prime number with $p \equiv 1$ (mod $4$).\\
Let $K=\mathbb{Q}(i)$ and define the function $L$ by:
$$L(\chi,s)=\prod_p \frac{1}{1-\chi(p)p^{-s}}$$ where
$\chi(p)=(-1)^{\frac{p-1}{2}}$ and the product is running over all odd prime numbers $p$. Show that
$$\zeta_K(s)=\zeta_\mathbb{Q}(s) \cdot L(\chi,s)$$
\item[(iii)] Show that
$$L(\chi,s)=\sum_{n=0}^\infty \frac{(-1)^n}{(2n+1)^s}=1-\frac{1}{3^s}+\frac{1}{5^s}-\frac{1}{7^s}\pm\cdots$$
\end{enumerate}
\end{enumerate}

\section{Factorisation of prime numbers}
In this section we shall give many examples and deduce some useful results about the factorisation of ordinary prime numbers in the ring of integers $\mathcal{O}_K$ for some number field $K$.
\subsection{Basic property}
We shall begin with some restriction of the norm of prime ideals.
\begin{lemma} Let $K$ be a number field. Let $P$ be a prime ideal of $\mathcal{O}_K$. Then there exists a {\bf unique} prime number $p \in \mathbb{Z}$ such that
$$P \big| \langle p \rangle$$
\end{lemma}
\begin{proof}[\bf Proof] It is clear that $P \cap \mathbb{Z}$ is a prime ideal and hence there exists a prime number $p$ such
that $$P \cap \mathbb{Z}=\langle p \rangle$$ since $\mathbb{Z}$ is a principal ideal domain.
Now suppose there is a prime $q$ such that $$P \big| \langle q \rangle$$
Then $q \in P \cap \mathbb{Z}$ and so $q \in \langle p \rangle$. Then $p|q$ and so $q=p$. This proves the uniqueness.
\end{proof}
\begin{theorem} Let $K$ be a number field with $[K:\mathbb{Q}]=n$. Let $P$ be a prime ideal of $\mathcal{O}_K$.
Let $p$ be a prime number which lies in $P$ (which exists and is unique by Lemma 15.1.) Then
$$N(P)=p^f$$ for some $f \le n$.
\end{theorem}
\begin{proof}[\bf Proof] By Lemma 15.1, we have  $P\big| \langle p \rangle$ and hence
$$N(P) \big| N(\langle p \rangle)=|N(P)|=p^n$$
Therefore, $N(P)=p^f$ for some $f \le n$ because $p$ is a prime.
\end{proof}
\begin{definition} Let $K$ be a number field with $[K:\mathbb{Q}]=n$. Let $P$ be a prime ideal and $p$ be a prime number lies in $P$. Then By Theorem 15.2, then the positive integer $f$ such that
$$N(P)=p^f$$
is called the {\bf inertial degree} of $P$ in $\mathcal{O}_K$ and is denoted by $f_K(P)$.
\end{definition}
\begin{proposition} Let $K$ be a number field and $P$ a prime ideal of $\mathcal{O}_K.$ Let $p$ be a prime number
in $P$. Let $a,b \in \mathbb{Z}$. Then $$a \equiv b~(\text{mod }p) \iff a \equiv b~(\text{mod }P)$$
\end{proposition}
\begin{proof}[\bf Proof] If $a \equiv b$ (mod $p$) then $a-b \in \langle p \rangle$ and so $a-b \in P$ because
$$P \big| \langle p \rangle$$
Conversely, if $a \equiv b$ (mod $P$) then $P \big| \langle a-b \rangle$. Then
$$p^f \big| (a-b)^n=N(\langle a-b \rangle)$$
and so
$$(a-b)^n \equiv 0~(\text{mod } p) \Rightarrow a \equiv b~(text{mod } p)$$
\end{proof}
\begin{theorem} Let $K$ be a number field with $[K:\mathbb{Q}]=n$. Let $p$ be a prime number. Suppose that the
principal ideal $\langle p \rangle$ factors in $\mathcal{O}_K$ in the form
$$\langle p \rangle=\prod_{i=1}^g P^{e_i}_i$$
where $P_1,\ldots,P_g$ are distinct prime ideals of $\mathcal{O}_K$ and $e_1,\ldots,e_g$ are positive integers.
Suppose $f_i$ is the inertia degree of $P_i$ in $K$, then
$$\sum_{i=1}^g e_i f_i=n$$
\end{theorem}
\begin{proof}[\bf Proof] This follows immediately by taking the norm of $\langle p \rangle$, so
$$N(\langle p \rangle)=p^n=\prod_{i=1}^g (p^{f_i})^{e_i}$$
and so
$$\sum_{i=1}^g e_i f_i=n$$
\end{proof}
\begin{corollary} Let $K$ be a number field with $[K:\mathbb{Q}]=n$ and $P$ a prime ideal in $\mathcal{O}_K$. Let $p$ be a prime number in $P$. If $P^n \big| \langle p \rangle$, then
$$P^n=\langle p \rangle$$
\end{corollary}
\begin{proof}[\bf Proof] $N(P^n)=N(P)^n$. Let $f_K(P)=f$. Then we have
$$fn \le n$$
which says $f \le 1$ and so $f=1$. So $N(P^n)=p^n$ and so $P^n=\langle p \rangle$.
\end{proof}
\begin{definition} Let $K$ be a number field. Let $p$ be a prime number. Then the number of distinct prime ideals of $\mathcal{O}_K$ which divide $p$ is called the {\bf decomposition number} of $p$ in $K$, written $g_K(p)$.
\end{definition}
It follows immediately from Theorem 15.5 that the decomposition number $g_K(p) \le n$ if $[K:\mathbb{Q}]=n$.
\begin{definition} Let $K$ be a number field and $P$ a prime ideal in $\mathcal{O}_K$. Let $p$ be the prime number
in $P$. Then the unique positive integer $e$ such that
$$P^e \big| \langle p \rangle, P^{e+1} \nmid \langle p \rangle$$
is called the {\bf ramification index} of $P$ in $K$ and is written $e_K(P)$. It is clear that $e_K(P) \le n$.
\end{definition}
In the notation of Theorem 15.5, we have
$$e_K(P_i)=e_i,f_K(P_i)=f_i,g_K(p)=g$$
\subsection{Ramification}
\begin{definition} Let $K$ be a number field of degree $n$. Let $p$ be a prime number. Let $P_1,\ldots,P_{g_K(p)}$ be the prime ideals which divide $\langle p \rangle$, and let $e_i=e_K(P_i)$.
If $e_i >1$ for some $i \in \{1,\ldots,g\}$, then $p$ is said to {\bf ramify} in $K$. If $e_i=1$ for all $i$, then
we say $p$ is {\bf unramified} in $K$.
\end{definition}
The main result we are going to deduce in this subsection is to show that a prime $p$ ramifies in $K$ if and only if $p |\mathcal{D}_K$. We shall begin with:
\begin{theorem} $p$ ramifies in $K$ if and only if $\mathcal{O}_K/\langle p \rangle$ has a nilpotent element.
\end{theorem}
\begin{proof}
By Chinese Remainder Theorem, suppose that
$$\langle p \rangle=\prod_{i=1}^g P^{e_i}_i$$
So we have
$$\mathcal{O}_K/\langle p \rangle \cong \mathcal{O}_K/P^{e_1}_1 \times \cdots \times \mathcal{O}_K/P^{e_k}_k$$
Now for each $i$, consider $\mathcal{O}_K/P^{e_i}_i$. Either $e_i=1$, so that it is a finite field because $P_i$
is prime and hence maximal; Or $e_i \ge 2$, and it has a non-zero nilpotent element, because we could take
$\alpha \in P_i$ but $\alpha \not \in P^2_i$, so we have $0 \neq \alpha+P^{e_i}_i$ in $\mathcal{O}_K/P^{e_i}_i$,
such that
$$(\alpha+P^{e_i}_i)^{e_i}=0$$ because $\alpha^{e_i} \in P^{e_i}_i$.

So in the first case, $\mathcal{O}_K$ is a product of finite field, which contains no nilpotent element, and in the second case, we have a nilpotent element in the subring and so a nilpotent element in $\mathcal{O}_K$.
\end{proof}
\begin{theorem} Let $p$ be a prime number. Then $p$ ramifies in $K$ if and only if $p|\mathcal{D}_K$.
\end{theorem}
\begin{proof}[\bf Proof] Recall that the discriminant is the determinant of the matrix which represents the trace pairing. Let $T_{ij}=T(e_i,e_j)$ where $\{e_1,\ldots,e_n\}$ is a $\mathbb{Z}$-basis for $\mathcal{O}_K$.
Then
$$p|\mathcal{D}_K \iff p | \det{T} \iff \det{T} \equiv 0~(\text{mod } p)$$
Consider the matrix $\bar{T} = T$ modulo $p$.
Then
$$\bar{T}: \mathcal{O}_K/\langle p \rangle \times \mathcal{O}_K/\langle p \rangle \rightarrow \mathbb{Z}/\langle
p \rangle$$
and so by above
$$p|\mathcal{D}_K \iff \det{\bar{T}}=0 \iff \bar{T} \text{ is degenerate }$$

Now as $p$ is a prime, so $\mathbb{Z}/p\mathbb{Z}$ is a field. Also $p \in P^{e_i}_i$ for each $i$, so
$\mathcal{O}_K/P^{e_i}_i$ is a $\mathbb{Z}/p\mathbb{Z}$-vector space. Then we restrict $T$ on each
$\mathcal{O}_K/P^{e_i}_i$. If $e_i=1$, as in the proof of Theorem 15.10,$\mathcal{O}_K/P_i$ is a field and is finite. So it is a finite extension of field $\mathbb{F}_p$, which is separable. Therefore,
if $e_i=1$ for all $i$, then $T$ is non-degenerate on each subspace and so is non-degenerate on the whole vector space. If $e_i \ge 2$ for some $i$. Then again as in the proof of Theorem 15.10, we have a nilpotent element, say $\alpha$, in $\mathcal{O}_K/\langle p \rangle$ and let $\alpha^n=0$. Then $(\alpha x)^n=0$ for all $x \in \mathcal{O}_K/\langle p \rangle$. In particular, the linear map $\phi_{\alpha x}$ (Definition 12.25) is nilpotent. So every eigenvalue is $0$ and since the trace is the sum of eigenvalues, so we conclude that
$$\bar{T}(\alpha x)=0~\forall x \in \mathcal{O}_K/\langle p \rangle$$
Therefore, the bilinear form $\bar{T}$ is degenerate.

Combine the first and second paragraph, we conclude that
$$p|\mathcal{D}_K \iff p \text{ ramifies in } K$$
\end{proof}


\subsection{Application in a quadratic field}
In this subsection we are going to deduce some properties of the factorisation of primes in a quadratic field, when
$[K:\mathbb{Q}]=2$.
The decomposition number is then $1$ or $2$.

If $g=2$ then we have
$$e_1f_1+e_2f_2=2$$
so that $$e_1=e_2=f_1=f_2=1$$
If $g=1$ then $e_1f_1=2$. So we have
$$e_1=2,f_1=1 \text{ or } e_1=1,f_1=2$$
Then we have three cases in total. So we may get the factorisation of $\langle p \rangle$ in one of the following forms:
\begin{enumerate}
\item[(i)] $$\langle p \rangle=P_1P_2, N(P_1)=N(P_2)=p,P_1 \neq P_2$$
\item[(ii)] $$\langle p \rangle=P^2, N(P)=p$$
\item[(iii)] $$\langle p \rangle=P, N(P)=p^2$$
\end{enumerate}
The next theorem gives a necessary and sufficient condisions for each of (i),(ii) and (iii) to occur, using Legendre symbol.
\begin{theorem} Let $K$ be a quadratic field so that we may write $K=\mathbb{Q}(\sqrt{m})$ for some square free
integer $m$. Let $p$ be a prime number. Then we have the following conclusions.
When $p>2$,
\begin{enumerate}
\item[(i)] if $(\frac{m}{p})=1$, then $\langle p \rangle=P_1P_2$, where $P_1$ and $P_2$ are prime.
\item[(ii)] if $p|m$, then $\langle p \rangle=P^2$, where $P$ is prime.
\item[(iii)] if $(\frac{m}{p})=-1$, then $\langle p \rangle=P$, where $P$ is prime.
\end{enumerate}
When $p=2$,
\begin{enumerate}
\item[(i)] if $m \equiv 1$ (mod $8$), then $\langle 2 \rangle=P_1P_2$ where $P_1$ and $P_2$ are prime.
\item[(ii)] if $m \equiv 2$ or $3$ (mod $4$), then $\langle 2 \rangle =P^2$ where $P$ is prime.
\item[(iii)] if $m \equiv 5$ (mod $8$), then $\langle 2 \rangle=P$ where $P$ is prime.
\end{enumerate}
\end{theorem}
\begin{proof}[\bf Proof] When $p>2$,
\begin{enumerate}
\item[(i)] if $(\frac{m}{p})=1$, then there exists $a$ such that $a^2 \equiv m$ (mod $p$), and $p \nmid a$ because $p \nmid m$. Consider the ideals
$$P_1=\langle p, a+\sqrt{m} \rangle, P_2=\langle p,a-\sqrt{m} \rangle$$
Then
$$P_1P_2=\langle p \rangle
\langle p,a+\sqrt{m},a-\sqrt{m},\frac{a^2-m}{p} \rangle=\langle p \rangle$$
because $a^2-m$ is divisible by $p$, and also $(2a,p)=1$, so the second ideal in the product above is $\mathcal{O}_K$.
Also $P_1 \neq P_2$ because if so then $2a \in P_1$ and $p \in P_1$ then $P_1=\langle 1 \rangle$, which is a contradiction.\\
\item[(ii)] if $p|m$, then $p|\mathcal{D}_K$ and so by Theorem 15.11, $p$ ramifies in $K$ so $p=\langle P\rangle^2$. We can write out explicitly the factorisation. Consider the ideal
$$P=\langle p,\sqrt{m} \rangle$$
Then
$$P^2=\langle p \rangle \langle p,\sqrt{m},\frac{m}{p}\rangle$$
Since $m$ is square free so $(p,\frac{m}{p})=1$ and so the second ideal in the product above is $\mathcal{O}_K$.\\
\item[(iii)] if $(\frac{m}{p})=-1$, suppose $\langle p \rangle$ is not prime. Then let $P$ be a prime ideal, $P \neq \langle p \rangle$ and $P|\langle p \rangle$. Then since $N(p)=p^2$, we must have $N(P)=p$. Then it is easy to check that
    $$\{P,1+P,\ldots,p-1+P\} \cong \mathbb{Z}/p\mathbb{Z}$$
    is the field $\mathcal{O}_K/P$. Since $\sqrt{m} \in \mathcal{O}_K$, so there exists
    $a \in \{0,1,\ldots,p-1\}$ such that
    $$\sqrt{m} \equiv a~(\text{mod } P) \Rightarrow m \equiv a^2~(\text{mod } P)$$
    Thus, by Proposition 15.4, we have
    $$m \equiv a^2~(\text{mod } p)$$
    which is a contradiction. Hence $\langle p \rangle$ is a prime.
\end{enumerate}
When $p=2$,
\begin{enumerate}
\item[(i)] if $m \equiv 1$ (mod $8$), then $\frac{1+\sqrt{m}}{2} \in \mathcal{O}_K$, so we consider the two ideals
    $$P_1=\langle 2,\frac{1+\sqrt{m}}{2}\ rangle, P_2=\langle 2,\frac{1-\sqrt{m}}{2} \rangle$$
    Then
    $$P_1P_2=\langle 2 \rangle \langle 2,\frac{1+\sqrt{m}}{2},\frac{1-\sqrt{m}}{2},\frac{1-m}{8} \rangle$$
    Is is clear that $\frac{1+\sqrt{m}}{2}+\frac{1-\sqrt{m}}{2}=1$ and so the second ideal in the above product
    is
    $\mathcal{O}_K$. So $$P_1P_2=\langle 2 \rangle$$
    Also $P_1 \neq P_2$, because if $P_1=P_2$, then $1 \in P_1$, which is a contradiction.
\item[(ii)] if $m \equiv 2$ or $3$ (mod $4$), then $\mathcal{D}_K=4m$ and since $2|4m$ so $2$ ramifies in $K$.
    To be more precise, if $m \equiv 2$ (mod $4$), then we consider the ideal
    $$P=\langle 2,\sqrt{m} \rangle$$
    Then
    $$P^2=\langle 2 \rangle \langle 2,\sqrt{m},\frac{m^2}{4} \rangle$$
    Since $m \equiv 2$ (mod $4$) so $\frac{m^2}{4}$ is odd and hence the second ideal in the product above is
    $\mathcal{O}_K$ and so $P^2=\langle 2 \rangle$.\\
    If $m \equiv 3$ (mod $4$), then we consider the ideal
    $$P=\langle 2,1+\sqrt{m} \rangle$$
    Then
    $$P^2=\langle 2 \rangle \langle 2,1+\sqrt{m},\sqrt{m}+\frac{1+m}{2} \rangle$$
    Since $m \equiv 3$ (mod $4$) so $\frac{1+m}{2}$ is even and since
    $$(m-1)=\sqrt{m}+\frac{1+m}{2}-(1+\sqrt{m})$$
    is odd and lies in the second ideal in the product above, so it is $\mathcal{O}_K$ and
    $$P^2=\langle 2 \rangle$$
\item[(iii)] if $m \equiv 5$ (mod $8$). Suppose $\langle 2 \rangle$ is not prime, and let $P_1$ be a prime ideal which divides $\langle 2 \rangle$ and $N(P)=2$. So $\mathcal{O}_K/P=\{P,1+P\}$. Since $m \equiv 1$ (mod $4$), so $\frac{1+\sqrt{m}}{2} \in \mathcal{O}_K$, then
    there exists $a=0,1$ such that
    $$\frac{1+\sqrt{m}}{2} \equiv a~(\text{mod } P)$$
    and
    $$\frac{1-\sqrt{m}}{2}=1-\frac{1+\sqrt{m}}{2} \equiv 1-a~(\text{mod }P)$$
    So
    $$\frac{1-m}{4} \equiv a(1-a)~(\text{mod } P)$$
    and so by Proposition 15.4, and $a=0$ or $1$, we have
    $$\frac{1-m}{4} \equiv 0~(\text{mod } 2)$$
    which is a contradiction because $m \equiv 5$ (mod $8$) and so $\frac{1-m}{4}$ is odd.
\end{enumerate}
\end{proof}
Thus from the proof we can also conclude that:
\begin{equation*}
\langle 2 \rangle= \left\{
\begin{array}{ll}
\langle 2 \rangle & \text{ if } m \equiv 5~(\text{mod } 8)\\
\langle 2,\frac{1}{2}(1+\sqrt{m})\rangle \langle 2,\frac{1}{2}(1-\sqrt{m})\rangle & \text{ if } m \equiv 1~(\text{mod }8)\\
\langle 2,1+\sqrt{m} \rangle^2 &\text{ if } m \equiv 3~(\text{mod } 4)\\
\langle 2,\sqrt{m} \rangle^2 &\text{ if } m \equiv 2~(\text{mod } 4)
\end{array} \right.
\end{equation*}
and for $p>2$,
\begin{equation*}
\langle p \rangle= \left\{
\begin{array}{ll}
\langle p \rangle & \text{ if } \left(\frac{m}{p}\right)=-1\\
\langle p,x+\sqrt{m} \rangle \langle p,x-\sqrt{m} \rangle & \text{ if } \left(\frac{m}{p}\right)=1 \text{ and }
x^2 \equiv m~(\text{mod } p)\\
\langle p,\sqrt{m} \rangle^2 & \text{ if } p|m\\
\end{array} \right.
\end{equation*}
Further, we say $\langle p \rangle$ splits if $\langle p \rangle=P_1P_2$ for different prime ideals $P_1,P_2$, and use
Kronecker symbol (Definition 4.17) we have
\begin{corollary} Let $K$ be a number field. Let $d=\mathcal{D}_K$ and $p$ be a prime number. Then
\begin{enumerate}
\item[(i)] $\langle p \rangle$ splits if and only if $(\frac{d}{p})=1$.
\item[(ii)] $\langle p \rangle$ ramifies if and only if $(\frac{d}{p})=0$.
\item[(iii)] $\langle p \rangle$ is a prime if and only if $(\frac{d}{p})=-1$.
\end{enumerate}
\end{corollary}
The proof is just a restatement of Theorem 15.12 using Kronecker symbol.

Now if $K$ is a quadratic field, then let $\sigma$ be the generator of the automorphism group. We write
$\bar{\alpha}=\sigma(\alpha)$ for all $\alpha \in K$. In other words, if
$K=\mathbb{Q}(\sqrt{m})$ then $\bar{\sqrt{m}}=-\sqrt{m}$ and $\bar{q}=\sigma(q)=q$ for all $q \in \mathbb{Q}$.
For any ideal $I=\langle \alpha_1,\ldots,\alpha_n \rangle$, we write
$$\bar{I}=\langle \bar{\alpha}_1,\ldots,\bar{\alpha}_n \rangle$$
It is clear that for any ideals $I$ and $J$, since $\sigma$ is a homomorphism, so
$$\overline{IJ}=\bar{I}\bar{J}$$
The next proposition gives another expression for the norm of ideal in quadratic field.
\begin{proposition} Let $K$ be a quadratic field. Let $I$ be a ideal of $\mathcal{O}_K$. Then
$$\langle N(I) \rangle=I \bar{I}$$
\end{proposition}
\begin{proof}[\bf Proof] This is clear if $I$ is zero or $\mathcal{O}_K$. If not, we have
$$I=\prod_{i=1}^m P^{a_i}_i \bar{P}^{b_i}_i \prod_{j=1}^n Q^{c_j}_j  \prod_{k=1}^l R^{d_k}_k$$
where $P_i,Q_j,R_k$ are pairwise distinct prime ideals. Also by Theorem 15.12, we know that each prime ideal
which contains a prime number is one of the three following types:
\begin{enumerate}
\item[(i)] $P_i \bar{P}_i= \langle p_i \rangle$ where $p_i$ is a prime number in $P_i$ and $P_i \neq \bar{P_i}$.\\
\item[(ii)] $Q^2_j=\langle q_j \rangle$ where $q_j$ is a prime number in $Q_j$ and $Q_j=\bar{Q}_j$.\\
\item[(iii)] $R_k=\langle r_k \rangle$ where $r_k$ is a prime number in $R_k$ and $R_j=\bar{R}_k$.
\end{enumerate}
Then since $\bar{\bar{I}}=I$, for all $I$, we have
$$\bar{I}=\prod_{i=1}^m P^{b_i}_i\sigma(P)^{a_i}_i) \prod_{j=1}^n Q^{c_j}_j \prod_{k=1}^l R^{d_k}_k$$
and so
$$I\bar{I}=\langle \prod_{i=1}^m p^{a_i+b_i}_i \prod_{j=1}^n q^{c_j}_j \prod_{k=1}^l r^{2d_k}_k\rangle$$
Now we consider the norm, the norm of $P_i$ is $p_i$, the norm of $Q_j$ is $q_j$ and the norm of
$R_k$ is $r^2_k$. Hence
$$N(I)=\prod_{i=1}^m p^{a_i+b_i}_i \prod_{j=1}^n q^{c_j}_j \prod_{k=1}^l r^{2d_k}_k$$
and the result follows.
\end{proof}
Now since $\bar{\bar{I}}=I$ so we also have
$$\langle N(\bar{I}) \rangle=I\bar{I}$$
by Proposition 15.14, and so $N(I)=uN(\bar{I})$ for some $u \in \mathcal{O}_K$ which is a unit. But
since $N(I),N(\bar{I})$ are both positive integers, so $u=1$, and so
$$N(I)=N(\bar{I})$$
\begin{definition} Let $K$ be a quadratic field. We say an ideal $I$ is {\bf self-conjugate} if $I=\bar(I)$.
\end{definition}



\subsection{Kummer-Dedekind theorem}

We shall prove a very important result concerning the factorisation of prime numbers in a number field $K$ such that the ring of integer $\mathcal{O}_K$ has a power basis (In other words, $\mathcal{O}_K=\mathbb{Z}[\theta]$ for some $\theta \in \mathcal{O}_K$.) It was originally proved by Dedekind in 1878, and is known as Kummer-Dedekind Theorem.

\begin{theorem}{\bf [Kummer-Dedekind]}\label{K;Kummer-Dedekind} 
Let $K=\mathbb{Q}(\theta)$ be a number field of
degree $n$ such that $\mathcal{O}_K=\mathbb{Z}[\theta]$. Let $p$ be a prime number.
Let $$f(x)=\min_{\mathbb{Q},\theta}(x)$$ and $\bar{f(x)}$ be the reduction of $f(x)$ modulo $p$.
Let $$\bar{f}(x)=\prod_{i=1}^r g_i(x)^{e_i}$$ where $g_1(x),\ldots,g_r(x)$ are distinct monic irreducible polynomials in $\mathbb{F}_p[x]$ and $e_1,\ldots,e_r$ are positive integers. For each $i$, let $f_i(x)$ be any
monic polynomial of $\mathbb{Z}[x]$ such that $\bar{f_i(x)}=g_i(x)$. Now set
$$P_i=\langle p, f_i(\theta)\rangle$$
for each $i$. Then
$$\langle p \rangle = \prod_{i=1}^r P^{e_i}_i$$
and $N(P_i)=p^{\deg{f_i}}$ for each $i$.
\end{theorem}

\begin{proof}[\bf Proof] For each $i$, let $\theta_i$ be a root of $g_i$ in some suitable extension of $\mathbb{F}_p$.
Then since $g_i$ is irreducible, we have
$$\mathbb{F}_p[\theta_i] \cong \mathbb{F}_p[x]/\langle g_i(x) \rangle$$
and since $g_i(x)$ is irreducible, hence maximal and so $\mathbb{F}_p[\theta_i]$ is a field for each $i$.
Now define the map:
$$\phi_i: \mathbb{Z}[\theta] \rightarrow \mathbb{F}_p[\theta_i]$$
by
$$\phi_i(h(\theta))=\bar{h}(\theta_i)$$
Then it is clear that $\phi_i(\theta)=\theta_i$ and so the map is surjective, and so
$$\mathcal{O}_K/\ker{(\phi_i)}=\mathbb{Z}[\theta]/\ker{(\phi_i)} \cong \mathbb{F}_p[\theta_i]$$
is a field. So we conclude that $\ker{(\phi_i)}$ is a maximal and hence prime ideal in $\mathcal{O}_K$.

Next, we show that the kernel of $\phi_i$ is
$$P_i=\langle p,f_i(\theta)\rangle$$
It is clear that
$$\phi_i(p)=0 \text{ and } \phi_i(f_i(\theta))=\bar{f_i}(\theta_i)=g_i(\theta)=0$$
and so $P_i \subseteq \ker{(\phi_i)}$.

Conversely, let $g(\theta) \in \ker{(\phi_i)}$, then
$$\phi_i(g(\theta))=\bar{g}(\theta_i)=0$$
Then since $g_i$ is the minimal polynomial of $\theta_i$ in $\mathbb{F}_p[x]$, then
we have $g_i(x)|\bar{g}(x)$ in $\mathbb{F}_p[x]$ and so we have
$$\bar{g}(x)=\bar{f_i}(x)\bar{h_i}(x) \text{ for some } \bar{h} \in \mathbb{F}_p[x]$$
Therefore, $g(x)-f_i(x)h_i(x)$ is divisible by $p$. Hence,
$$g(\theta)=(g(\theta)-f_i(\theta)h_i(\theta))+f_i(\theta)h_i(\theta) \in \langle p, f_i(\theta)\rangle$$
This proves that
$$\ker{(\phi_i)}=\langle p,f_i(\theta)\rangle=P_i$$
and hence $P_i$ is prime for each $i$.

Then we show $P_i \neq P_j$ if $i \neq j$. Suppose $P_i =P_j$,
then we have $g(\theta),h(\theta) \in \mathbb{Z}[\theta]$ such that
$$f_i(\theta)=g(\theta)p+h(\theta)f_j(\theta)$$
Apply $\phi_j$, so we have
$$g_i(\theta_j)=\bar{h}(\theta_j)g_j(\theta_j)=0$$
Since $g_j$ is the minimal polynomial of $\theta_j$ in $\mathbb{F}_p[x]$, so $g_j(x)|g_i(x)$.
But $g_i(x)$ is irreducible and monic, so $g_i=g_j$ and so $i=j$.

We now show that
$$\langle p \rangle=\prod_{i=1}^r P^{e_i}_i$$
It is clear that
\be
\prod_{i=1}^r P^{e_i}_i = \prod_{i=1}^r \langle p,f_i(\theta)\rangle^{e_i} = \prod_{i=1}^r (\langle p \rangle +\langle f_i(\theta) \rangle)^{e_i} \subseteq \langle p \rangle+\prod_{i=1}^r \langle f_i(\theta) \rangle^{e_i} = \langle p \rangle +\langle f(\theta) \rangle = \langle p \rangle
\ee
because for any ideals $A,B_1$ and $B_2$, we have
$$(A+B_1)(A+B_2) \subseteq A+B_1B_2$$
and $f(\theta)=0$.
So we have
$$\langle p \rangle \big|\prod_{i=1}^r P^{e_i}_i$$
Now it is clear that $p \in P_i$ for each $i$ and so $P_i|\langle p \rangle$.
So we write
$$\langle p \rangle=\prod_{i=1}^r P^{k_i}_i$$
where $k_i \in \{1,2,\ldots,e_i\},i=1,2,\ldots,r.$

Now we consider that
$$\mathcal{O}_K/P_i =\mathbb{Z}[\theta]/P_i \cong \mathbb{F}_p[\theta_i]$$
so that
$$N(P_i)=\card(\mathcal{O}_K/P_i)=\card(\mathbb{F}_p[\theta_i])=p^{\deg{g_i}}$$
Hence we have
$$p^n=N(\langle p \rangle)=\prod_{i=1}^r N(P_i)^{k_i}=p^{\sum_{i=1}^r \deg{g_i}k_i}$$
and so
$$n=\sum_{i=1}^r k_i\deg{g_i}$$
But since $[K:\mathbb{Q}]=n=\deg{f}=\deg{\bar{f}}$, and
$$\bar{f}(x)=\prod_{i=1}^r g^{e_i}_i$$
Comparing the degree on both sides, we have
$$n=\sum_{i=1}^r e_i \deg{g_i}$$
Hence
$$\sum_{i=1}^r e_i \deg{g_i}=\sum_{i=1}^r k_i \deg{g_i}$$
and since $k_i \le e_i$ for each $i$, so we conclude that
$$e_i=k_i$$ for each $i$.
Therefore,
$$\langle p \rangle=\prod_{i=1}^r P^{e_i}_i$$
and
$$N(P_i)=p^{\deg{g_i}}=p^{\deg{f_i}}$$
\end{proof}
The assumption that $\mathcal{O}_K$ has a power basis is essential in this theorem.



\subsection{Generalisation of Kummer-Dedekind theorem}
The following is a generalisation of Kummer-Dedekind theorem.
\begin{theorem} Let $K=\mathbb{Q}(\theta)$ be a number field with $\theta \in \mathcal{O}_K$. Let $p$ be a prime number. Let
$$f(x)=\min_{\mathbb{Q},\theta}(x) \in \mathbb{Z}[x]$$
Let $\bar{f(x)}$ be the reduction of $f(x)$ modulo $p$ for any $f(x)$. Let
$$\bar{f}(x)=\prod_{i=1}^r g^{e_i}_i$$
where $g_1(x),\ldots,g_r(x)$ are distinct monic irreducible polynomials in $\mathbb{F}_p[x]$ and $e_1,\ldots,e_r$ are positive integers. For each $i$, let $f_i(x)$ be any monic polynomial of $\mathbb{Z}[x]$ such that
$\bar{f_i}(x)=g_i(x)$. Set
$$P_i=\langle p, f_i(\theta) \rangle$$
for each $i$. If the index of $\mathbb{Z}[\theta]$ in $\mathcal{O}_K$ is not $0$ modulo $p$, then
$P_1,\ldots,P_r$ are distinct prime ideals and
$$\langle p \rangle=\prod_{i=1}^r P^{e_i}_i$$
and
$$N(P_i)=p^{\deg{f}}$$ for each $i$.
\end{theorem}
\begin{proof}[\bf Proof] The proof can be modeled on Theorem 15.16 with slight modification. For simplicity, write
$$A=\mathbb{Z}[\theta],I_i=\langle p,f_i(\theta) \rangle \mathbb{Z}[\theta]$$
In Theorem 15.16, we proved that $I_i$ is a prime ideal for each $i$ and $I_i \neq I_j$ if $i \neq j$.
Define the map $\phi$ by:
$$\phi: A \rightarrow \mathcal{O}_K/P_i$$
by $\phi(x)=x$ mod $P_i$. This is clear a ring homomorphism and it is clear that $I_i \in \ker{(\phi)}$.
Conversely, suppose $h(\theta) \in \ker{(\phi)}$, then $h(\theta)=0$ mod $P_i$. Hence $h(\theta) \in P_i$ and so
$h(\theta) \in I_i$ because $I_i \subseteq P_i$.

Also, let $[\mathcal{O}_K:A]=n$, then for any $\alpha \in \mathcal{O}_K$, $n\alpha \in A$.
Then since $(n,p)=1$, so pick $m \in \mathbb{Z}$ such that
$$mn \equiv 1~(\text{mod } p)$$
Then $m(n\alpha) \in A$ and
$$\phi(m (n\alpha))= mn \alpha~(\text{mod } P_i)$$
but $mn-1 \in \langle p \rangle \subseteq P_i$, so $mn \alpha \equiv \alpha$ (mod $P_i$). Hence
$$\phi(m (n\alpha))=\alpha~(\text{mod } P_i)$$
and so the map is surjective.
Therefore, we have
$$A/I_i \cong \mathcal{O}_K/\langle p,f_i(\theta) \rangle$$
and we have proved in Theorem 15.16 that $A/I_i$ is a field, hence $P_i$ is maximal and hence prime in $\mathcal{O}_K$.

Now use the above isomorphism, suppose $P_i =P_j$, then $I_i=I_j$ and so $i=j$ by Theorem 15.6.
Finally, the proof of
$$\langle p \rangle \prod_{i=1}^r P^{k_i}_i$$
is the same as before. We then use
$$\mathcal{O}_K/P_i \cong A/I_i$$
to conclude that
$$N(P_i)=\card(\mathcal{O}_K/P_i)=\card(A/I_i)=\card(\mathbb{F}_p[\theta_i])=p^{d_i}$$
where $d_i=\deg{g_i}=\deg{\bar{f}_i}$ as in Theorem 15.6. The rest of the proof is then identical to that in
Theorem 15.6.
\end{proof}

\subsection{Exercise}
You may assume $\mathcal{D}_K \equiv 1$ or $0$ (mod $4$) for any number field $K$.
\begin{enumerate}
\item Factorise $\langle 2 \rangle$ into prime ideals in $K=\mathbb{Q}[\sqrt{47}]$.
\item Let $K=\mathbb{Q}(\sqrt{26})$. Factorise the principal ideals
$$\langle 2 \rangle, \langle 5 \rangle, \langle 6+\sqrt{26} \rangle$$
\item Let $K=\mathbb{Q}(\sqrt{35})$. Show that
$$\langle 2 \rangle=\langle 2,1+\sqrt{35} \rangle^2, \langle 5 \rangle =\langle 5,\sqrt{35} \rangle^2,
\langle 5+\sqrt{35} \rangle=\langle 2,1+\sqrt{35} \rangle \langle 5,\sqrt{35} \rangle$$
\item Let $K=\mathbb{Q}(\sqrt[3]{10})$. Does $\mathcal{O}_K$ has a power basis?
\item Let $K=\mathbb{Q}(\sqrt[3]{3})$.
\begin{enumerate}
\item[(i)] Deduce that $\langle 3 \rangle =P^3$, where $P=\langle \sqrt[3]{3} \rangle$ is a prime ideal. Find its norm.
\item[(ii)] Are there any prime numbers $p \neq 3$ such that
$$\langle p \rangle=Q^3$$
in $\mathcal{O}_K$ for some prime ideal $Q$?
\end{enumerate}
\item Let $\zeta_m$ be an $m^{th}$ primitive root of $1$. Let $K=\mathbb{Q}(\zeta_m)$. Show that
$N(\zeta_m)=1$.\\
\item Let $K$ be a number field with $[K:\mathbb{Q}]=n$. A prime number $p$ is called {\bf totally ramified} if $\langle p \rangle=P^n$ for some prime ideal $P$.
\begin{enumerate}
\item[(i)] Let $K$ be a number field. Show that if a prime number $p$ is ramified in $K$, then it is
ramified in all intermediate field $M$.\\
\item[(ii)] Let $K$ and $L$ be number fields. Suppose that the prime $p$ is totally ramified in $\mathcal{O}_{K}$ and unramified in $\mathcal{O}_L$. Prove that $K \cap L=\mathbb{Q}$.\\
\end{enumerate}
\item Let $K=\mathbb{Q}(\theta)$, where $\theta^3-\theta-1=0$. Prove that $$\sqrt{\theta} \not \in \mathbb{Q}(\theta)$$
\item
\begin{enumerate}
\item[(i)] Let $K=\mathbb{Q}(\theta)$, where $\theta^3-\theta+4=0$. Show that the ideal $I=\langle 2,\theta \rangle$ is principal in $\mathcal{O}_K$ and is generated by $\theta+2$.
\item[(ii)] Hence or otherwise, show that the norm $N(2+\theta)=2$ or $-2$.
\end{enumerate}
\item[$^\star$ 10.] In this question, we are going to deduce the factorisation of prime numbers in cyclotomic extension. We call $K$ a cyclotomic extension of $\mathbb{Q}$ if
$$K=\mathbb{Q}(\zeta_m)$$
for some $m>1$ where $\zeta_m$ is an $m^{th}$ primitive root of unity.
\begin{enumerate}
\item[(i)] Let $p$ be a prime number. Let $K=\mathbb{Q}(\zeta_m)$. Write $m=p^r n$ where $r \ge 0$ and
$p \nmid n$. Show that in the subfield $L=\mathbb{Q}(\zeta_{p^r})$, $p$ is totally ramified. Now let
$M=\mathbb{Q}(\zeta_n)$. Show that $p$ is unramified in $M$.
\item
\end{enumerate}
\end{enumerate}

\section{The ideal class group}
\subsection{Basic concept}
We have already seen the factorisation of non-zero ideals of a Dedekind domain $D$ Let $K$ be the field of quotient of $D$. The ideals under multiplication form a group $I(K)$.  The principal ideal in $D$ has the form  $\langle \alpha \rangle=\{r\alpha: r \in D\}$ for some $\alpha \in K \backslash\{0\}$.
In particular, the set of principal ideals $P(K)$ in forms a subgroup of the group $I(K)$. Further, since
the multiplication is commutative, and so $I(K)$ is an Abelian group. Hence $P(K)$ is a normal subgroup.
In this section, we are going to investigate the properties of the quotient group $I(K)/P(K)$.
\begin{definition} Let $K$ be a number field. Let $I(K)$ be the group of non-zero ideals of $D=\mathcal{O}_K$.
Let $P(K)$ be the subgroup of principal ideals of $I(K)$. Then the quotient group $I(K)/P(K)$ is called
the {\bf ideal class group} of $K$ and is denoted by $Cl(K)$ (or $H(K)$ in some context.)
\end{definition}
We shall see later (with proof) that the ideal class group is always finite.
\begin{definition} Let $K$ be a number field and $D=\mathcal{O}_K$.  The order of the class group $Cl(K)$ (or $H(K)$) is called the {\bf class number} of $K$ and is denoted by $h(K)$.
\end{definition}
\begin{proposition} Let $K$ be a number field. Then $h(K)=1$ if and only if $\mathcal{O}_K$ is a principal ideal domain.
\end{proposition}
\begin{proof}[\bf Proof] $$h(K)=1 \iff [I(K):P(K)]=1 \iff I(K)=P(K)$$
and $I(K)=P(K)$ if and only if every ideal of $\mathcal{O}_K$ is principal, and hence $\mathcal{O}_K$ is a principal ideal domain.
\end{proof}

\subsection{Minkowski's convex body theorem}
In this subsection, we are going to prove Minkowski's theorem from which we can deduce that the ideal class group is finite. We shall begin with some commutative algebra.
\begin{definition} A subset $H \subseteq V$ is {\bf discrete} if for every compact subset $X \subseteq V$,
$H \cap X$ is finite.
\end{definition}
In $\mathbb{R}^n$, we shall use the fact that a subset $H$ is compact if and only if it is closed and bounded.
\begin{lemma} Let $V$ be a $\mathbb{Z}$-module, which is a subset of $\mathbb{R}^n$. Suppose $H \subseteq V$ is a submodule. Then the followings are equivalent:
\begin{enumerate}
\item $H$ is discrete in $V$.
\item $H$ is finitely generated over $\mathbb{Z}$ and some generating set is linearly independent over $\mathbb{R}$.
\item $H$ is finitely generated over $\mathbb{Z}$ and every $\mathbb{Z}$-basis is linearly independent over
$\mathbb{R}$.
\end{enumerate}
\end{lemma}
\begin{proof}
\begin{enumerate}
\item[(a)] 1 $\Rightarrow$ 2: Let $e_1,\ldots,e_r \in H$ be linearly independent over $\mathbb{R}$ with $r$ maximal.
It is clear that $r \le n$. Let $P$ be the set
$$P=\left\{\sum_{i=1}^r a_ie_i: a_i \in [0,1]\right\}$$ It is clear that $P$ is compact, and $P \cap H$ is finite by assumption.

Take $x \in H$. As $r$ is maximal, so $x=\sum_{i=1}^r b_i e_i$, where $b_i \in \mathbb{R}$.
Write $b_i=[b_i]+a_i e_i$, where $0 \le a_i <1$ and so we have
$$x=\sum_{i=1}^r \left([b_i]+a_ie_i\right)$$
Since $H$ is a $\mathbb{Z}$-module, so that $[b_i]e_i \in H$ and $x \in H$ and so
$\sum_{i=1}^r a_ie_i \in H$, but this is also clearly in $P$ by definition. Hence
$$\sum_{i=1}^r a_ie_i \in P \cap H$$
Therefore, $H$ is generated as a $\mathbb{Z}$-module by $e_1,\ldots,e_r$ and $P \cap H$.

For each integer $j$, define
$$x_j=jx-\sum_{i=1}^r [jb_i]e_i=\sum_{i=1}^r \left(jb_i-[jb_i]\right)e_i\in P$$
But $jx \in H$ and $\sum_{i=1}^r[jb_i]e_i \in H$, so $x_j \in H \cap P$.
Therefore, there exists $j \neq k$ such that $x_j=x_k$ because $H \cap P$ is finite.
Then we have
$$(j-k)b_i=[jb_i]-[kb_i]$$
for each $i$ because $e_1,\ldots,e_r$ are linearly independent.
So
$$b_i=\frac{[jb_i]-[kb_i]}{j-k} \in \mathbb{Q}$$
and $b_i$ is independent of $x$. Let $b_i=\frac{B_i}{N}$ where $B_i \in \mathbb{Z}$.
Then
$$x=\sum_{i=1}^r b_ie_i=\sum_{i=1}^r\frac{B_ie_i}{N}$$
So $H$ is a $\mathbb{Z}$ module generated by $\{\frac{e_1}{N},\ldots,\frac{e_r}{N}\}$ and it is clear that
they are linearly independent over $\mathbb{R}$ since $e_1,\ldots,e_r$ are linearly independent.
\item[(b)] 2 $\Rightarrow$ 3: It is clear that $H$ is finitely generated over $\mathbb{Z}$. There is a set
$\{e_1,\ldots,e_r\}$ which is a $\mathbb{Z}$-basis for $H$ and is linearly independent over $\mathbb{R}$, then for any $\mathbb{Z}$-basis $\{f_1,\ldots,f_r\}$, write $f_i$ in terms of $e_i$.
$$f_i=\sum_{j=1}^r a_{ij}e_j$$
where $a_{ij} \in \mathbb{Z}$. Then suppose
$$\sum_{i=1}^r b_if_i=0$$
then
$$\sum_{i=1}^r b_i\sum{j=1}^r a_{ij}e_j=0$$
But since $e_1,\ldots,e_r$ is linearly independent over $\mathbb{R}$ so we have
$$\sum_{i=1}^r b_i a_{ij}=0$$ for each $j$.
But the matrix $A_{ij}=a_{ij}$ is invertible since we can also write $e_i$ in terms of $f_i$. Hence
we have $b_i=0$ for all $i$.
\item[(c)] 3 $\Rightarrow$ 1: Suppose $\{e_1,\ldots,e_r\}$ is a $\mathbb{Z}$-basis of $H$, which is also linearly independent over $\mathbb{R}$. So we can extend it to a $\mathbb{R}$-basis and add $e_{r+1},\ldots,e_n$ such that $\{e_1,\ldots,e_n\}$ is a $\mathbb{R}$ of $\mathbb{R}^n$. Suppose $\{f_1,\ldots,f_n\}$ is the standard orthonormal basis for $\mathbb{R}^n$. Then there is a $\mathbb{R}$ linear transformation:
$$L: V \rightarrow V$$
such that $L(e_i)=f_i$. It is clear that both $L$ and $L^{-1}$ are continuous. Suppose $X \subseteq V$ is compact, then $L(X)$ is also compact in $V$. Then there is a ball $B \subseteq V$ centered at $0$ such that
$$L(X) \subseteq B$$ Let $B$ have radius $R$ and so $B=B_R(0)$.
Now observe that $L(H) \subseteq \mathbb{Z}f_1+\cdots+\mathbb{Z}f_n$. Then
$L(H) \cap B$ is finite because there are only finitely many integer vectors
$(m_1,\ldots,m_r)$ with $\sum_{i=1}^r m^2_i \le R^2$.
Since $L$ is bijective, we have
$$H \cap L^{-1}(B)=L^{-1}(L(H) \cap B)$$
finite. But $X \subseteq L^{-1}(B)$ because $L(X) \subseteq B$ and so $H \cap X$ is finite. Hence $H$ is discrete.
\end{enumerate}
\end{proof}
\begin{definition} Let $V$ be a $\mathbb{Z}$ module in $\mathbb{R}^n$. A {\bf lattice} in $V$ is an discrete additive subgroup (i.e. a $\mathbb{Z}$ submodule) $H \subseteq V$ such that $rank_\mathbb{R}H=n$.
\end{definition}
In other words, it is discrete subset in $V$ which achieves the full rank.
\begin{definition} Let $H$ be a lattice in $V$, then by Lemma 16.5, let $\{e_1,\ldots,e_n\}$ be a $\mathbb{Z}$-basis for $H$. Let
$$P=\left\{\sum_{i=1}^n a_ie_i: a_i \in [0,1)\right\}$$
The {\bf covolume} of $H$ is defined as the volume of $P$, and is written as $cov(H)$.
\end{definition}
\begin{proposition} The definition of $cov(H)$ is independent of the choice of basis.
\end{proposition}
\begin{proof}[\bf Proof] Let $\{e_1,\ldots,e_n\}$ and $\{f_1,\ldots,f_n\}$ be two $\mathbb{Z}$-bases for $H$. Let
$A$ be the change of basis matrix. Define
$$P=\left\{\sum_{i=1}^n a_ie_i: a_i \in [0,1)\right\}, Q=\left\{\sum_{i=1}^n b_ie_i: b_i \in [0,1)\right\}$$
Then since $A$ has entries in $\mathbb{Z}$ and is invertible, so $\det(A)=\pm 1$. Therefore,
$$V(Q)=|\det{A}|V(P)=V(P)$$
the volume of $Q$ is the same as $P$ and so the definition is independent of the choice of basis.
\end{proof}
\begin{definition} A set $S \in \mathbb{R}^n$ is {\bf convex} if $\forall x,y \in S$,
$\lambda x+(1-x)\lambda(y) \in S$ for all $\lambda \in [0,1]$. A set $S$ is called a {\bf convex body} if
it is compact and convex.
\end{definition}
\begin{definition} Let $P$ be a subset of $\mathbb{R}^n$. Let $h \in \mathbb{R}^n$, the set
$$P_h=\{h+p: p \in P\}$$
is called the {\bf translate} of $P$.
\end{definition}
\begin{remark}
It is clear that if $H \subseteq V$ is a lattice with basis $\{e_1,\ldots,e_n\}$. Let
$$P=\left\{\sum_{i=1}^n a_ie_i: a_i \in [0,1)\right\}$$
then $V=\cup_{h \in H}P_h$. Moreover, the union is disjoint.
\end{remark}
\begin{definition} A set $S \subseteq \mathbb{R}^n$ is called {\bf symmetric} if $-\alpha \in S$ for any
$\alpha \in S$.
\end{definition}
\begin{theorem}{\bf [Minkowski's convex body theorem]}\label{M;Minkowski convex}
Suppose $H \subseteq V$ is a lattice and $S \subseteq V$ is measurable ($S$ has a volume), and the measure of
$S$ is denoted by $m(S)$, then
\begin{enumerate}
\item Suppose $m(S)> cov(H)$. Then there exists $x,y \in S$ such that $0 \neq x-y \in H$.
\item If $m(S)>2^n cov(H)$, and $S$ is convex and symmetric, then there exists $x \neq 0$, such that
$x \in S \cap H$.
\item If $m(S) \ge 2^n cov(H)$, and $S$ is a convex body and symmetric, then there exists $x \neq 0$ such that
$x \in S \cap H$.
\end{enumerate}
\end{theorem}
\begin{proof}[\bf Proof] Let $\{e_1,\ldots,e_n\}$ be a $\mathbb{Z}$-basis for $H$ and let
$$P=\left\{\sum_{i=1}^n a_ie_i: a_i \in [0,1)\right\}$$
\begin{enumerate}
\item Observe that the linear transformation does not change the measure (or volume) so we have
$$m(S \cap P_k)=m(S_{-k} \cap P) \text{ where } P_k=\{k+p: p \in P\}$$
Since $V=\sqcup_{k \in H}P_k$, we have
$$S=\sqcup_{k \in H}(S \cap P_k) \text{ and so } m(S)=\sum_{k \in H}m(S \cap P_k)$$
Suppose $(S_{-k} \cap P) \cap (S_{-h} \cap P)=\emptyset$ for all $h \neq h \in H$, then
$$\sum_{k \in H}m(S_{-k} \cap P) \le m(P) =cov(H)$$
and so
$$m(S)=\sum_{k \in H}m(S \cap P_k)=\sum_{k \in H}m(S_{-h} \cap P) \le cov(H)$$
which is a contradiction.

Therefore, there exists $h \neq k \in H$ such that
$$(S_{-h} \cap P) \cap (S_{-k} \cap P) \neq \emptyset$$
Hence there exist $x,y \in S$ such that
$$x-h=y-k \text{ and so } 0\neq x-y=h-k \in H$$
\item Let $S'=\frac{1}{2}S=\{\frac{1}{2}s: s \in S\}$. Then $m(S')=\frac{1}{2^n}m(S)$ and so
$$m(S')>cov(H)$$
and so by previous part we have distinct $y,z \in S'$ such that $y-z \in H$.
Then $2y,2z \in S$ and so $-2z \in S$. Then
$$x=y-z=\frac{1}{2}\left(2y +(-2z)\right) \in S$$
because $S$ is convex. Thus $0 \neq x \in H$.
\item For any integer $m$, define the set
$$S_m=\left(1+\frac{1}{m}\right)S$$
It is clear that for all $m$,
$$S \subset S_m \subset S_{m-1} \subset S_1$$
Thus, $m(S_m) >2^n cov(H)$ for all $m$. By previous part, we can find
$x_m \in S_m \cap H \subseteq S_1 \cap H$. Since $S_1$ is compact, and $H$ is a lattice, so we have
$S_1 \cap H$ finite and so there exists $x$ such that $x_m=x$ for infinitely many $m$.
Hence $x \in \bigcap_m S_m=S$ because $S$ is compact.
\end{enumerate}
\end{proof}
\subsection{Finiteness of the ideal class group}
\begin{definition} Let $K$ be a number field with $[K:\mathbb{Q}]=n=r+2s$ in the usual notation. Then
define the map
$$\sigma=(\sigma_1,\ldots,\sigma_{r+s}): K \rightarrow \mathbb{R}^r \times \mathbb{C}^s$$
and for each complex pair, only one of the embedding appears. Another view of the map is
$$\sigma=(\sigma_1,\ldots,\mathcal{R}e\sigma_{r+1},\mathcal{I}m\sigma_{r+1},\ldots,\mathcal{R}e\sigma_{r+s},
\mathcal{I}m\sigma_{r+s}): K \rightarrow \mathbb{R}^n$$
\end{definition}
Now we shall use Minkowski's convex body theorem to deduce that the ideal class group is finite.
\begin{proposition} Let $K$ be a number field of degree $n$. Then $\sigma(\mathcal{O}_K$ is a lattice of $\mathbb{R}^n$ with covolume $2^{-s}\sqrt{|\mathcal{D}_K|}$.
Moreover,If $I$ is an integral ideal of $\mathcal{O}_K$, then
$\sigma(I)$ is a lattice of $\mathbb{R}^n$ with covolum $2^{-s} \sqrt{|\mathcal{D}_K|}N(I)$
\end{proposition}
\begin{proof}[\bf Proof] Let $\{e_1,\ldots,e_n\}$ be a $\mathbb{Z}$-basis of $\mathcal{O}_K$. For any $x \in K$,
$$\sigma(x)=(\sigma_1(x),\ldots,\sigma_r(x),\mathcal{R}e\sigma_{r+1}(x),\mathcal{I}m\sigma_{r+1}(x),\ldots,
\mathcal{R}e\sigma_{r+s}(x),\mathcal{I}m\sigma_{r+s}(x)\}$$
Let $X$ be the matrix such that the $i^{th}$ row is $\sigma(e_i)$.
Recall that for any complex number $z$, let $z^*$ be the conjugate of $z$, we have
$$\mathcal{R}e(z)=\frac{z+z^*}{2},\mathcal{I}m(z)=\frac{z-z^*}{2i}$$
Consider the entries
$$a=\frac{\sigma_j(e_i)+(\sigma_j(e_i))^*}{2}, b=\frac{\sigma_j(e_i)-(\sigma_j(e_i))^*}{2i}$$
We have $\sigma_j(e_i)=a+ib$, and then $(\sigma_j(e_i))^*=\sigma(e_i)-2ib$.
But the complex conjugate $(\sigma_j)^*$ is the complex pair of the embedding $\sigma_j$.
Hence we can apply column operations so that the determinant
$$d=\frac{1}{(2i)^s}\det{(\sigma_{j}(e_i))}$$
and so
$$d^2=\frac{1}{(2i)^{2s}}\det{(\sigma_j(e_i))}^2=\frac{1}{(2i)^{2s}}\mathcal{D}_K \neq 0$$
So the vectors $\sigma_(x_1),\ldots,\sigma_(x_n)$ are linearly independent, which is a basis and a
$\mathbb{Z}$-basis for $\sigma(\mathcal{O}_K)$ because $\mathcal{O}_K$ is a $\mathbb{Z}$-module.
Hence it is a lattice. By definition of determinant in real spaces, we have
$$cov(\mathcal{O}_K)=|d|=2^{-s} \sqrt{|\mathcal{D}_K|}$$

Now let $I$ be an integral ideal. Then we have, by the above argument, we have
$$cov(I)=2^{-s}\sqrt{|D(I)|}=2^{-s} \sqrt{|\mathcal{D}_K|}N(I)$$
\end{proof}
\begin{definition} Let $K$ be a number field of degree $n=r+2s$. Define the {\bf Minkowski's bound} by
$$c_K=\left(\frac{4}{\pi}\right)^s \frac{n!}{n^n} \sqrt{|\mathcal{D}_K|}$$
\end{definition}
\begin{theorem} Let $K$ be a number field. Let $I$ be an integral ideal of $\mathcal{O}_K$. Then there exists
$0 \neq x \in I$ such that
$$|N(x)| \le N(I)c_K$$
\end{theorem}
\begin{proof}[\bf Proof] For any positive real number $t$, define the set
$$B(r,s)_t=B_t=\{(y_1,\ldots,y_r,z_1,\ldots,z_s) \in \mathbb{R}^r \times \mathbb{C}^s: \sum_{i=1}^r|y_i|+2\sum_{j=1}^s|z_j| \le t\}$$
It is clear that the region $B(r,s)_t$ is symmetric and by triangle inequality the region is convex. It is also clear that the region is compact. So $B(r,s)_t$ is a convex body.
We shall firstly show that the measure (volume) of $B(r,s)_t$ is
$$m(B(r,s)_t)=2^r \left(\frac{\pi}{2}\right)^s \frac{t^n}{n!}$$
We shall prove this by double induction.
When $r=1,s=0$, it is clear that the volume is
$$\int_{-t}^t dy=2t$$
When $s=1,r=0$, let $z=re^{i\theta}$ then $dz=2\pi r dr$ since the volume is independent of $\theta$.
$$\int_{0}^{\frac{t}{2}}  2\pi r dr=\frac{\pi t^2}{4}=\frac{\pi}{2} \frac{t^2}{2}$$
So both of them satisfy the formula given above.
Now suppose it is true for $(r,s)$. Then for $(r+1,s)$, we have
$$\sum_{i=1}^r |y_i|+2\sum_{j=1}^s |z_j| \le t-|y_{r+1}|$$
and so the volume is, by induction
$$\int_{-t}^{t} 2^r \left(\frac{\pi}{2}\right)^s \frac{(t-|y_{r+1|})^n}{n!}dy_{r+1}=2^{r+1} \left(\frac{\pi}{2}\right)^s \frac{t^{n+1}}{(n+1)!}$$
Similarly, for $(r,s+1)$, let $z=r e^{i\theta}$ and so $dz=2\pi r dr$, so the volume is
\begin{eqnarray*}
V&=&\int_{|z_{s+1} \le \frac{t}{2}} \le 2^r \left(\frac{\pi}{2}\right)^s \frac{(t-\frac{|z_{s+1}|}{2})^n}{n!}dz_{s+1}\\
&=&\int_{0}^{\frac{t}{2}} 2^r \left(\frac{\pi}{2}\right)^s \frac{(t-\frac{r}{2})^n}{n!}2\pi r dr\\
&=&2^r\left(\frac{\pi}{2}\right)^{s+1} \frac{t^{n+2}}{(n+2)!}
\end{eqnarray*}
Hence it is true for all $(r,s)$.

Then since we have proved that $B(r,s)_t$ is a convex body and we pick some $t$ such that
$$m(B(r,s)_t)=2^n cov(\sigma(I))$$
and so
$$t^n=\left(\frac{4}{\pi}\right)^s n! \sqrt{|\mathcal{D}_K|}N(I)$$
Apply Minkowski's convex body theorem (Theorem 16.13) there exists a non-zero $y \in B(r,s)_t \cap \sigma(I)$.
Let $\sigma(x)=y=(y_1,\ldots,y_r,z_1,\ldots,z_s) \in B(r,s)_t$. Then
$$|N(x)|=\left|\prod_{i=1}^r y_i\right| \left|\prod_{j=1}^s z_jz^*_j\right|=\prod_{i=1}^r|y_i| \prod_{j=1}^s |z_j|^2$$
because only one of the conjugate pair appears. Then apply AM-GM inequality, we have
$$\left|N(x)\right|^{\frac{1}{n}} \le \frac{1}{n}\left(\sum_{i=1}^r |y_i|+2\sum_{j=1}^s|z_j|\right) \le \frac{t}{n}$$
So
$$|N(x)| \le \left(\frac{t}{n}\right)^n =\left(\frac{4}{\pi}\right)^s \frac{n!}{n^n}\sqrt{|\mathcal{D}_K|}N(I)$$
\end{proof}
\begin{definition} In the ideal class group, the {\bf ideal class} $[I]$ containing $I$ is the set of ideals
which contains $I$ and $I_1,I_2 \in [I]$ if $I_1P(K)=I_2P(K)$ in the ideal class group.
\end{definition}
\begin{corollary} Every ideal class in $\mathcal{O}_K$ has an integral ideal representative $J$ such that
$N(J) \le c_K$.
\end{corollary}
\begin{proof}[\bf Proof] Let $M$ be any ideal in the given class. Then it is clear that $\alpha M$ is also in the given class for any $\alpha \in K\backslash \{0\}$ because
$$\alpha M P(K)=MP(K)$$
Then we may rescale $M$ by multiplying $M$ by suitable $\alpha$ such that $\alpha M \supseteq \mathcal{O}_K$
(For example, multiply $M$ by $\alpha$ so that one of the generator is $1$.) Thus, we may assume that the given
$M \supseteq \mathcal{O}_K$ (as we can always pick one in the given class.) and so $M^{-1}$ is integral.
Then by Theorem 16.17, there exists $0 \neq x \in M^{-1}$ such that $|N(x)|<N(M^{-1})c_K$.
By multiplicative property of ideal norm, we have
$$N(xM) \le c_K$$
and it is clear that $xM$ is in the given class.
\end{proof}
\begin{proposition} Let $K$ be a number field. Let $M$ be a positive integer. Then there are only finitely many integral ideals $I$ of $\mathcal{O}_K$ such that $N(I)=M$.
\end{proposition}
\begin{proof}[\bf Proof] Let $I$ be an integral ideal with $N(I)=M$. Then we have
$$I \big|\langle M \rangle$$ by exercise 7 of chapter 14.
Let
$$\langle M \rangle=\prod_{i=1}^r P^{e_i}_i$$
where $e_i \ge 1$. Then we have
$$I=\prod_{i=1}^r P^{f_i}_i$$
where $0 \le f_i \le e_i$. Hence there are at most $(e_1+1)\cdots(e_r)+1$ such $I$.
\end{proof}
\begin{theorem} Let $K$ be a number field. Then the ideal class group $Cl(K)$ is finite.
\end{theorem}
\begin{proof}[\bf Proof]  By Corollary 16.19, each class contains an integral ideal whose norm is less than or equal to $c_K$.
So the class number must be less than or equal to the number of integral ideals with norm less than or equal to $c_K$. By Proposition 16.20, there are only finitely many integral ideals with norm equal to $M$ for each $M \le c_K$ and hence finitely many integral ideals with norm less than or equal to $c_K$.
Hence the ideal class group is finite.
\end{proof}

\subsection{Further properties}

We shall deduce some useful result using the Minkowski's bound.

\begin{theorem} 
Suppose $n gr 2$, then $|\mathcal{D}_K| \ge \frac{\pi}{3}(\frac{3\pi}{4})^{n-1}>1$.
\end{theorem}

\begin{proof}[\bf Proof] There exists some ideal $I$ such that $N(I) \le c_K$ by Corollary 16.19. But $1 \le N(I)$ and so
$1 \le c_K$. Therefore, we have
$$|\mathcal{D}_K|^{\frac{1}{2}} \ge \left(\frac{\pi}{4}\right)^s \frac{n^n}{n!}$$
But since $2s \le n$ and $\frac{\pi}{4} \le 1$, so
$$|\mathcal{D}_K| \ge \left(\frac{\pi}{4}\right)^n \frac{n^{2n}}{(n!)^2}$$
For $n \ge 2$, define
$$a_n=\left(\frac{\pi}{4}\right)^n \frac{n^{2n}}{(n!)^2}$$
It is clear that $a_2=\frac{\pi^2}{4} >1$.
For $n>2$,
\be
\frac{a_{n+1}}{a_n} = \frac{\pi}{4}\left(1+\frac{1}{n}\right)^{2n} = \frac{\pi}{4}\sum_{i=0}^2n \binom{2n}{i} \left(\frac{1}{n}\right)^i > \frac{3\pi}{4}
\ee
by taking the first two terms in the expansion.
Hence
$$|\mathcal{D}_K| >\frac{\pi}{3}\left(\frac{3\pi}{4}\right)^{n-1}$$
\end{proof}
\begin{corollary} Let $K$ be a number field with degree $n \ge 2$. Then there exists a prime number $p$ which ramifies in $K$.
\end{corollary}
\begin{proof}[\bf Proof] By Theorem 16.22, $|\mathcal{D}_K| >1$ and we know $|\mathcal{D}_K|$ is an integer. Therefore,
it must have a prime factor $p$ and so $p$ ramifies in $K$ by Theorem 15.11.
\end{proof}
\begin{theorem}{\bf [Hermite]}\label{H;Hermite} There are only finitely many number fields $K$ with $|\mathcal{D}_K|$ below a certain bound.
\end{theorem}
\begin{proof}[\bf Proof] Let $A$ be a constant bound and let $K$ be a number field such that $|\mathcal{D}_K|<A$.
So we show prove that the number of $K$ such that $|\mathcal{D}_K|=M$ for each $M<A$ is finite.
By Theorem 16.22,
$$M=|\mathcal{D}_K| >\frac{\pi}{3}\left(\frac{3\pi}{4}\right)^{n-1}$$
and so the degree of extension $n$ is bounded. So we may check that for each $M$ and $n$, there are only finitely many number fields $K$ such that $|\mathcal{D}_K|=M$ and $[K:\mathbb{Q}]=n$. If $n$ is fixed, then there are only finitely many pairs $(r,s)$ and so we may then check that for each pair $(r,s)$ there are finitely many number fields with fixed discriminant $|\mathcal{D}_K|=M$. We treat the cases $r=0$ and $r \ge 1$ separately.
If $r\ge 1$, consider the region
\be
B = \left\{(y_1,\ldots,y_r,z_1,\ldots,z_s) \in \mathbb{R}^r \times \mathbb{C}^s: |y_1|\le 2^{n-1}\left(\frac{\pi}{2}\right)^{-s}\sqrt{|\mathcal{D}_K|} |y_i|,|z_j| \le \frac{1}{2} \forall i \neq 1,\forall j\right\}
\ee

If $r=0$, consider the region
\be
C = \left\{(y_1,\ldots,y_r,z_1,\ldots,z_s) \in \mathbb{R}^r \times \mathbb{C}^s: \mathcal{R}e|z_1| \le 2^{n-2} \Im|z_1| \le \frac{\pi}{4}\left(\frac{\pi}{2}\right)^{-s}\sqrt{|\mathcal{D}_K|},\text{ and } |z_j| \le \frac{1}{2}~\forall j \neq 1\right\}
\ee
It is clear that both $B$ and $C$ are convex bodies. The volumes
$$V(B)=V(C)=2^n2^{-s}\sqrt{|\mathcal{D}_K|}=2^n cov(\sigma(\mathcal{O}_K))$$
Therefore in both cases, by Minkowski's convex body theorem, there exists $x \in \mathcal{O}_K$, such that
$$\sigma(x) \in \sigma(\mathcal{O}_K) \cap B$$
We shall check that $K=\mathbb{Q}(x)$.

If $r>0$, then $|\sigma_j(x)| \le \frac{1}{2}$ for all $j \neq 1$ and since $|N(x)|$ is an integer, we conclude that $|\sigma_1(x)| >1$. Hence it is distinct from any other $\sigma_j(x)$ and so $K=\mathbb{Q}(x)$.
If $r=0$, then $\sigma_1(x)$ is the only element lying outside the unit sphere and so $K=\mathbb{Q}(x)$ as before.

Then by construction of the regions $B$ and $C$, each $|\sigma_i(x)|$ is bounded and since the coefficients of
minimal polynomials of $x$ are the elementary functions of $|\sigma_i(x)|$ and so they are bounded. But they are
integers so there are finitely many such polynomials and hence finitely many such $x$ and hence finitely many
such number fields $K=\mathbb{Q}(x)$. Therefore, we conclude that there are only finitely many number fields whose discriminant is under a certain bound.
\end{proof}
\begin{theorem} There are only finitely many number fields $K$ where the degree $n$ is bounded and the set of ramified prime $p\in Z$ is bounded. (This is a more useful version of theorem 16.24).
\end{theorem}
\begin{proof}[\bf Proof] By Theorem 16.24, it is enough to bound $|\mathcal{D}_K|$ by $n$ and the set $S$ of ramified primes. We say $v_p(n)=r$ if the order of $p$ in $n$ is $r$.
Then we quote a result from [Serre, Corps Locaux, III.6, p58]:
$$v_p(\mathcal{D}_K)\le n+n\cdot \log_p{n}-1$$
Since the set $S$ is bounded and $n$ is bounded, hence $\mathcal{D}_K$ is bounded.
\end{proof}



\subsection{Algorithm to determine the ideal class group}
The previous results give us a method of determining all the ideal classes of a given number field.
To determine the representatives of the ideal classes, we need only look at the integral ideals of $\mathcal{O}_K$ less than or equal to $c_K$, the Minkowksi bound. Also, by unique factorisation, if $I$ is one such ideal, then
$N(P) \le c_K$ for all prime ideals which divide $I$. Now we have seen $N(P)=p^f$ for some prime number $p$ and inertia degree $f$, so the prime ideals in various integral ideals in $I$ are all factors of $p \le c_K$. Thus if we take each prime number $p \le c_K$, look at how $\langle p \rangle$ factorises in $\mathcal{O}_K$, and form all
possible products of the prime ideals of these prime numbers that yield ideals with norm $\le c_K$ then we can have at least one representatives of every ideal class.

In particular, if every prime number $\le c_K$ factorises into a product of prime ideals of $\mathcal{O}_K$, each
of which is a principal ideal, then we can conclude that the class number is $1$. To sum up, we shall give the following algorithm:
\begin{flushleft}
{\bf Algorithm to find the ideal class group of a number field $K$}
\end{flushleft}
\begin{enumerate}
\item Determine $n=[K:\mathbb{Q}]$ and $r,s$ in the usual notation such that $r+2s=n$.
\item Compute the discriminant $\mathcal{D}_K$ and hence the Minkowski bound
$$c_K=\left(\frac{4}{\pi}\right)^s \frac{n!}{n^n}\sqrt{|\mathcal{D}_K|}$$
\item For each prime number $p \le c_K$, find the factorisation of $\langle p \rangle$. Take the list of prime ideals which contain $p$ for each $p$.
\item Consider all possible product of these prime ideals with norm $\le c_K$. Determine the generators of
the ideal class group $Cl(K)$.
\end{enumerate}
We shall apply the above algorithm to several examples. In general, we do not need to find precisely the generators
of the prime ideals which contain $p$ for each $p$. In fact, in many cases, we only need to consider the norm.
We shall make this concrete in the following examples.
\begin{example} Compute the ideal class group of $K=\mathbb{Q}(\sqrt{-19})$. We have
$$n=2,r=0,s=1,\mathcal{D}_K=-19$$
The Minkowski bound is
$$c_K=\left(\frac{4}{\pi}\right)\frac{2!}{2^2}\sqrt{19}=\frac{2}{\pi}\sqrt{19}<4$$
Consider $p=2,3$. But since
$$\left(\frac{-19}{2}\right)=\left(\frac{-19}{3}\right)=-1$$
so $\langle 2 \rangle, \langle 3 \rangle$ are both prime ideals. Hence $Cl(K)$ is the trivial group.
\end{example}
\begin{example} Compute the ideal class group of $K=\mathbb{Q}(\sqrt{-14})$. We have
$$n=2,r=0,s=1,\mathcal{D}_K=-56$$
The Minkowski bound is
$$c_K=\left(\frac{4}{\pi}\right)\frac{2!}{2^2}\sqrt{56}<5$$
Consider $p=2,3$. But $2|-56$ and so $2$ ramifies. Let
$$\langle 2 \rangle =P^2 \text{ where } N(P)=2$$
and since $(\frac{-56}{3})1$ so
$$\langle 3 \rangle =QR \text{ where } N(Q)=N(R)=3$$
It is clear that $P,Q$ and $R$ are not principal because we have no elements in $\mathcal{O}_K$ with norm $2$ or $3$. So the order of $[P]$ is $2$ in the class group since $[P]^2=[P^2]=[2]$ which is principal.
Let $\alpha \in \mathcal{O}_K$, then $\alpha=a+b\sqrt{-14}$ for some $a,b \in \mathbb{Z}$.
$N(\alpha)=a^2+14b^2$. Consider $\alpha=2+\sqrt{-14}$ then $N(\alpha)=18=2\cdot 3^2$.
Since the norm of prime ideal must be a power of prime. Hence we know that
$\langle \alpha \rangle$ must be divisible by a prime ideal of norm $2$ and prime ideals of norm $3$ or $9$.
But $3$ splits in $\mathcal{O}_K$, so it is divisible by a prime ideal of norm $2$ and two prime ideals of
norm $3$. So
$$\langle \alpha \rangle=PQ^2,PQR \text{ or } PR^2$$
But it cannot be $PQR$ because if so then since $QR=\langle 3 \rangle$, then
$3|\alpha$ in $\mathcal{O}_K$, which is impossible. Then it is $PQ^2$ or $PR^2$. But these two cases can be treated in the same way. Suppose it is $PQ^2$ then we know that
$$[PQ^2]=[\langle \alpha \rangle]$$ which is identity in the class group. But since $[P]^2$ is identity, so
we square $[PQ^2]$ so that $[Q^4]$ is the identity.
It is clear that $[Q^2]$ cannot be the identity because if so then $P$ would be the identity. Hence
$[Q]$ has order $4$ in the class group. Since $[PQ]^2$ is the identity, so $[P]=[Q^2]$, and since
$[QR]=[\langle 3 \rangle]$ is the identity, we have $[R]=[Q^3]$. Finally, the only integral ideals with norm
$\le 4$ are $P^2,Q,P,PQ$ and so the class group is
$$Cl(K)=\{[Q],[Q^2]=P,[Q^3]=R,[Q^4]=[P^2]=1\}$$
which is a cyclic group of order $4$. Similarly, if we consider the case when
$\langle \alpha \rangle=PR^2$ then the group would be generated by $[R]$. But in both cases, $Cl(K)=C_4$
\end{example}
In the second example above, we see that the explicit factorisation of $\langle 3 \rangle$ is not needed. The technique to find the class group is to find some `suitable' element in the ring of integer $\mathcal{O}_K$ with norm $n$ such that $n$ contains only prime factors lies below the Minkowski bound. We shall use this technique very
often and readers will find more examples in the exercises.

We shall give a table (table 1) of non-trivial class groups $Cl(\mathbb{Q}(\sqrt{k}))$ for $|k|<30$, square free.


\begin{table}[htbp]
\begin{center}
 \caption{$K=\mathbb{Q}(\sqrt{k})$,$|k|<30$ square free.}
 \begin{tabular}{lll}
  \hline
  $k$ & $Cl(K)$&generators\\
  \hline
  10 & $C_2$&$I=[\langle 2,\sqrt{10}\rangle]$\\
  15 & $C_2$&$I=[\langle 2,1+\sqrt{15}\rangle]$\\
  26 & $C_2$&$I=[\langle 2,\sqrt{26}\rangle]$\\
  30 & $C_2$&$I=[\langle 2,\sqrt{30}\rangle]$\\
  -5 & $C_2$&$I=[\langle 2,1+\sqrt{-15}\rangle]$\\
  -6 & $C_2$&$I=[\langle 2,\sqrt{-6}\rangle]$\\
  -13 & $C_2$&$I=[\langle 2,\sqrt{-10}\rangle]$\\
  -14& $C_4$&$I=[\langle 3,1+\sqrt{-14}\rangle]$\\
  -15 & $C_2$&$I=[\langle 2,\frac{1}{2}(3+\sqrt{-15})\rangle]$\\
  -17 & $C_4$&$I=[\langle 3,1+\sqrt{-17}\rangle]$\\
  -21 & $C_2 \times C_2$&$I=[\langle 2,1+\sqrt{-21}\rangle]$,$J=[\langle 3,\sqrt{-21}\rangle]$, $Cl(K)=\{I,J,IJ,1\}$\\
  -22 & $C_2$&$I=\langle [2,\sqrt{-22}\rangle]$\\
  -23 & $C_3$&$I=[\langle 2,\frac{1}{2}(1+\sqrt{-23})\rangle]$\\
  -26 & $C_6$&$I=[\langle 5,2+\sqrt{-26}\rangle]$\\
  -29 & $C_6$&$I=[\langle 3,1+\sqrt{-29}\rangle]$\\
  -30 & $C_2\times C_2$&$I=[\langle 2,\sqrt{-30}\rangle]$,$J=[\langle 3,\sqrt{-30}\rangle]$,
$Cl(K)=\{I,J,IJ,1\}$\\
  \hline
 \end{tabular}
\end{center}
\end{table}

For a quadratic field $K$, Dirichlet has given an explicit formula for the class number $h(K)$, which states:
\begin{theorem} Let $K$ be a quadratic field of discriminant $d$. Then the class number
$$h(K)=\frac{-w(d)}{2|d|}\sum_{r=1}^{|d|-1}r\left(\frac{d}{r}\right), \text{ if } d<0$$
and
$$h(K)=\frac{-1}{\log{u}}\sum_{1 \le r<\frac{d}{2}}\left(\frac{d}{r}\right)\log{\sin{\frac{\pi r}{d}}},
\text{ if } d>0$$
where $w(d)$ denotes the number of roots of unity in $\mathcal{O}_{\mathbb{Q}(\sqrt{d})}(d<0)$ so that
\begin{equation*}
w(d)= \left\{
\begin{array}{ll}
6 & \text{if } d=-3\\
4 & \text{if } d=-4 \\
2 & \text{ if } d<-4
\end{array} \right.
\end{equation*}
and $(\frac{d}{n})$ is the Kronecker symbol and $u$ is the fundamental unit of $\mathcal{O}_{\mathbb{Q}(\sqrt{d})}$ which we shall define in the next chapter.
\end{theorem}
The proof uses the $L$-series, and we shall introduce this later.

Now we can use the ideal class group to find the integer solutions to the equation of the form
$$x^2=y^3+k$$
for $k <0$. We shall see many examples in the exercises.



\subsection{Exercises}
\begin{enumerate}
\item Compute the ideal class group of $\mathbb{Q}(\sqrt{-30})$.
\item Compute the ideal class group of $\mathbb{Q}(\sqrt{-65})$.
\item Compute the ideal class group of $\mathbb{Q}(\theta)$ where $\theta^3-4\theta+2=0$.
\item Compute the ideal class group of $\mathbb{Q}(\sqrt[4]{2})$.
The following few questions will give details in solving equations of the form
$$x^2=y^3+k$$
\item Consider the equation $x^2+13=y^3$.
\begin{enumerate}
\item[(i)] Compute the ideal class group of $\mathbb{Q}(\sqrt{-13})$.
\item[(ii)] Writing
$$\langle x+\sqrt{-13} \rangle \langle x-\sqrt{-13} \rangle =\langle y \rangle^3$$
and consider the factorisation of these ideals. Find all integer solutions to the equation.
\end{enumerate}
\item Find all integer solutions to $x^2+21=y^3$.
\item Let $m$ be a square-free even integer. Let $K$ be a number field such that the class number $h(K)$ is prime to
$3$. Show that $y^3=x^2+m$ has at most two integer solutions.
\item Let $K$ be a quadratic field. Let $I$ be an ideal of $\mathcal{O}_K$. If $I^2$ is a principal ideal, prove that
$[I]=[\bar{I}]$, where $\bar{I}$ is the conjugate of $I$.
\item Let $K$ be an imaginary quadratic field with discriminant $d<-4$ and $d$ is odd. Use Dirichlet's class number formula to prove
that
$$h(K)=\frac{1}{2-\left(\frac{d}{2}\right)}\sum_{1 \le r< \frac{|d|}{2}} \left(\frac{d}{r}\right)$$
(Hint: You may use the standard properties of Kronecker symbol from chapter $4$ and use question 11 in Exercise 4).
\item Let $p$ be a prime which is congruent to $3$ mod $4$. Use Dirichlet's class number formula to prove that
$h(\mathbb{Q}(\sqrt{-p})$ is odd.
\item Let $p$ be a prime which is congruent to $3$ mod $4$. Suppose $h(\mathbb{Q}(\sqrt{p})$ is odd. By considering the ideal $\langle 2,1+\sqrt{p} \rangle$, prove that
there exists $a$ and $b$ such that
$$a^2-pb^2=(-1)^{\frac{(p+1)}{4}} 2$$
\item Use stirling's formula
$$n!=\sqrt{2\pi n}\left(\frac{n}{e}\right)^n e^{\frac{\theta}{12n}}~(0 <\theta<1)$$
Show that as $n$ increases, the discriminant of a number field of degree $n$ converges to infinity.
\item Show that if $x^2+31=y^3$ has an integer solution $(x,y)$, then $x$ must be odd. Hence, show that
$x^2+31=y^2$ has no integer solution, given the fact that the class number of $K=\mathbb{Q}(\sqrt{-31})$ is $3$
\item Find all integer solutions to the equation
$$x^2+1175=4y^3$$
\item Let $p$ and $q$ be distinct odd primes such that $(\frac{p}{q})=(\frac{q}{p})=1$.
\begin{enumerate}
\item[(i)] Show that at least one of $p$ and $q$ is congruent to $1$ mod $4$. Let $p \equiv 1$ (mod $4$). Show also that there are integers $u,v$ such that
    $$u^2 \equiv p~(\text{mod } 4q),p|u, v^2 \equiv q~(\text{mod }p),q|v$$
\item[(ii)] Define
$$\Lambda=\{(x,y,z)\in \mathbb{Z}^3: z \equiv 0~(\text{mod }2),x \equiv uy+vz~(\text{mod }2pq)\}$$
Show that $\Lambda$ is a lattice in $\mathbb{R}^3$ and if $(x,y,z) \in \Lambda$, then
$$x^2-py^2-qz^2 \equiv 0~(\text{mod }4pq)$$
\item[(iii)] Consider the ellipsoid
$$X=\{(x,y,z)\in \mathbb{R}^3: x^2+py^2+qz^2 < 4pq\}$$
Show that
$$x^2-py^2-qz^2=0$$
has a non-trivial solution.
\end{enumerate}
\end{enumerate}








\section{Units in the ring of integers}
In this section we are going to deduce some important properties of the units in the ring of integers.
\subsection{Units in a quadratic field}
\begin{lemma} Let $K$ be a number field. Then $u$ is a unit in $\mathcal{O}_K$ if and only if $N(u)=\pm 1$.
\end{lemma}
\begin{proof}[\bf Proof] Suppose $u$ is a unit, let $v \in \mathcal{O}_K$ such that $uv=1$.
Then $N(u)N(v)=1$. But $N(u),N(v)$ are both integers, so $N(u)=\pm 1$.
Conversely, if $N(u)=\pm 1$, then $N(\langle u \rangle)=1$ and so
$$\card(\mathcal{O}_K/\langle u \rangle)=1$$
Thus $\langle u \rangle=\mathcal{O}_K$ and so $u$ is a unit.
\end{proof}
It is clear that if $K$ is a quadratic field, $K=\mathbb{Q}(\sqrt{m})$ where $m$ is square free.
If $m<0$, then the only element in $\mathcal{O}_K$ with norm $\pm 1$ are $\pm 1$. Hence
the only units in $\mathcal{O}_K$ are $\pm 1$. What about the case when $m>0$?
Let $m \equiv 2$ or $3$ (mod $4$), then $\mathcal{O}_K=\mathbb{Z}[\sqrt{m}]$. Hence an element
$\alpha=a+b\sqrt{m}$ has norm $a^2-mb^2$. Thus, this suggests that we should look at the integer solutions
to the equation
$$a^2-mb^2=\pm 1$$
For $a^2-mb^2=1$, by Theorem 5.23, we know the solutions are
$$a+b\sqrt{m}=\pm(a_0+b_0\sqrt{m})^n$$
where $(a_0,b_0)$ is the fundamental solution and $n \in \mathbb{Z}$.
Further, the equation
$a^2-mb^2=-1$ has integer solutions if and only if the period $k$ of the continued fraction $\sqrt{m}$ is odd,
and if $(a,b)$ is a solution such that $a^2-mb^2=-1$, then it is clear that
$$(a',b') \text{ where } a'+b'\sqrt{d}=(a+b\sqrt{m})^2$$
is a solution for $a^2-mb^2=1$. Hence the units are also of the form,
$$a+b\sqrt{m}=\pm(a_0+b_0\sqrt{m})^n$$
where $(a_0,b_0)$ is a fundamental solution to $a^2-mb^2=-1$.

When $m \equiv 1$ (mod $4$), $\mathcal{O}_K=\mathbb{Z}[\frac{1+\sqrt{m}}{2}]$.
Let $\alpha=a+\frac{b}{2}+\frac{b\sqrt{m}}{2}$. Then $N(\alpha)=(a+\frac{b}{2})^2-\frac{b^2m}{4}$.
Thus we may consider the equations
$$(2a+b)^2-mb^2=\pm 4$$
It is clear that each solution $(x,y)$ with $x^2-my^2=\pm 4$ corresponds to a pair $(a,b)$ with
$2a+b=x,b=y$.
Thus, we shall consider
$$x^2-my^2=\pm 4$$
By question 9 in exercise 5, the solutions are given by
$$\frac{x+y\sqrt{m}}{2}=\pm\left(\frac{x_0+y_0\sqrt{m}}{2}\right)^n$$
where $(x_0,y_0)$ is a fundamental solution to $x^2-my^2=-4$ if this is soluble, and
a fundamental solution to $x^2-my^2=4$ if $x^2-my^2=-4$ is not soluble.

Thus, in both cases, intuitively we seen that the set of units is an infinite Abelian group generated by one element together with $\pm 1$.
In fact, we have


\begin{theorem} Let $K=\mathbb{Q}(\sqrt{m})$, where $m$ is square free and positive. Let $U_K$ be the set of units in $\mathcal{O}_K$.
Then
$$U_K \cong C_2 \times C_\infty$$
\end{theorem}
We shall use the algebraic method instead of the continued fraction to deduce the above result, though this seems already true from the previous discussion by continued fraction.
\begin{proof}[\bf Proof] Define the map
$$L:\mathcal{O}_K \backslash \{0\} \rightarrow \mathbb{R}^2$$
by
$$L(x)=(\log{|x|},\log{|\sigma(x)|})=(u,v) \in \mathbb{R}^2$$
This is clearly a ring homomorphism. Restrict the map to $U_K$.
Since $N(x)=\pm 1~\forall x\in U_K$, so the image lies in the line $W$:
$$W=\{(u,v) \in \mathbb{R}^2: u+v=0\}$$
Let $B$ be a compact subset in $W$ and so it is closed and bounded. Then there exists $a \in \mathbb{R}$ such that
$$\frac{1}{a} \le |x|,|\sigma(x)| \le a$$
for all $x \in L^{-1}(B)$.
But then $|x+\sigma(x)|,|x\sigma(x)|$ are bounded and so the coefficients of the minimal polynomial of $x$
is also bounded. Hence there are only finitely many those polynomials because each coefficient is integer,
and so there are only finitely many $x$. Hence $L^{-1}(B)$ is finite. Moreover, $L^{-1}(B) \cap U_K$ is finite and
so $B \cap L(U_K)$ is finite. This show that $L(U_K)$ is discrete.

It is clear that $\ker{L} \cap U_K=\{-1,1\}$. Since the set of units
is a group under multiplication, so
$$U_K/(U_K \cap \ker{L}) \cong V \subseteq W$$
where $V$ is a discrete additive subgroup of $W$ and hence it is finitely generated over $\mathbb{Z}$,
and has a $\mathbb{Z}$-basis which is independent over $\mathbb{R}$. So we have shown that
$$U_K/\{\pm 1\} \cong \mathbb{Z}^{\oplus m}$$
where $m=0$ or $1$ and so
$$U_K \cong C_2 \times \mathbb{Z}^{\oplus m}$$
It remains to show that $m=1$.
For each $l_1 >0$, define $l_2$ such that
$$l_1 l_2=\sqrt{|\mathcal{D}_K|}$$
and define the region
$$B_l=\{(y_1,y_2): |y_i| \le l_i\}$$
It is clear that $B$ is a convex body (a rectangle). The volume of $B$ is then
$$V(B)=4l_1l_2=4\sqrt{|\mathcal{D}_K|}=2^2 cov(\tau(\mathcal{O}_K))$$
where $\tau(x)=(x,\sigma(x))$.
By Minkowski convex body theorem, there exists $0 \neq x_l \in B_l \cap \tau(\mathcal{O}_K)$.
So
$$|x_l| \le l_1,|\sigma(x_l)| \le l_2 \Rightarrow |N(x_l)| \le l_1l_2=\sqrt{|\mathcal{D}_K|}$$
Now move $l_1$ from a fixed point to $0$. Then for each $x_l$, $N(x_l) \le \sqrt{|\mathcal{D}_K|}$, and
as $l_1 \to 0$, there are infinitely many distinct $x_l$ because $x_l \neq 0$. But by Proposition 16.20, there
are only finitely many integral ideals with norm $\le \sqrt{|\mathcal{D}_K|}$. Hence there exists $l \neq k$ such
that $x_l \neq \pm x_k$ but $\langle x_l \rangle=\langle x_k \rangle$.
Then $\frac{x_l}{x_k}$ is a unit other than $\pm 1$. Therefore, $m=1$ and so
$$U_K \cong C_2 \times \mathbb{Z} \cong C_2 \times C_\infty$$
\end{proof}
\begin{definition} Let $K$ be a real quadratic field. Then $U_K \cong C_2 \times C_\infty$. The {\bf fundamental unit}
$u$ is a positive unit $>1$ in $\mathcal{O}_K$ which generates the infinite cyclic group of $U_K$. In other words,
if $v$ is a unit in $\mathcal{O}_K$, then $v=\pm u^n$ for some $n \in \mathbb{Z}$.
\end{definition}
\begin{proposition} Let $K$ be a real quadratic field. Then the fundamental unit $u \in \mathcal{O}_K$ is the smallest unit of $\mathcal{O}_K >1$.
\end{proposition}
\begin{proof}[\bf Proof] Suppose there exists a unit $\theta$ such that
$$1<\theta<u$$
Then $\theta=u^n$ for some $n$ because both of them are positive.
It is clear that $n<0$ because $u>\theta$. But $u>1$ and so $u^n<1$ for $n<0$, contradicting $\theta>1$.
\end{proof}
The following theorems characterise the norm of fundamental unit
in $\mathcal{O}_K$ where $K=\mathbb{Q}(\sqrt{p})$ for an odd prime $p$.
\begin{theorem}{\bf [Hilbert]}\label{H;Hilbert} Let $p$ be a prime with $p \equiv 1$ (mod $4$). Then the fundamental unit in $\mathcal{O}_{\mathbb{Q}(\sqrt{p})}$ has norm $-1$.
\end{theorem}
\begin{proof}[\bf Proof] Let $u$ be the fundamental unit and let $v \in \mathcal{O}_K$ with $u\bar{u}=1$.
Then since $u,\bar{u}>0$ then we have
$$u=\frac{1+u}{1+\bar{u}}$$
Let $m$ be the largest positive integer such that $m|1+u,1+\bar{u}$ in $\mathcal{O}_K$.
Define
$$\gamma=\frac{1+u}{m} \in \mathcal{O}_K,\bar{\gamma}=\frac{1+\bar{u}}{m} \in \mathcal{O}_K$$
so that $\gamma,\bar{\gamma}$ are coprime and $u=\frac{\gamma}{\bar{\gamma}}$. So
$$\gamma=u\bar{\gamma} \Rightarrow \langle \gamma \rangle=\langle \bar{\gamma} \rangle$$
Let $Q$ be a prime ideal which divides
$\langle \gamma \rangle$. Then $Q \big|\langle \bar{\gamma} \rangle$.
Let $\bar{Q}$ be the conjugate of $Q$, then
$$\bar{Q} \big| \langle \bar{\gamma}\rangle=\langle \gamma \rangle$$
Hence
$$Q,\bar{Q} \big|\langle \gamma \rangle$$
Now $Q$ is a prime ideal, and the discriminant of $\mathcal{O}_K$ is $p$, so we have either
$$Q=\bar{Q}=\langle q \rangle \text{ where } q \text{ is a prime number with }\left(\frac{p}{q}\right)=-1$$
or
$$Q \neq \bar{Q}, Q\bar{Q}=\langle q \rangle \text{ where } q \text{ is a prime number with} \left(\frac{p}{q}\right)=1$$
or
$$Q =\bar{Q}, Q^2=\langle q \rangle \text{ where } q \text{ is a prime number with } q|p$$
In the first case, we have $q|\gamma, \bar{\gamma}$ which is impossible.

In the second case, we have $Q$ and $\bar{Q}$ are distinct prime ideals which divide $\langle \gamma \rangle$, and
so
$$\langle q \rangle =Q\bar{Q} \big|\langle \gamma \rangle=\langle \bar{\gamma} \rangle$$
and again we have $q|\gamma,\bar{\gamma}$, which is impossible.

In the third case, since $q$ is also a prime number, so $q=p$ and $Q=\langle \sqrt{p} \rangle$. Hence
$\langle \sqrt{p} \rangle$ is the only prime ideal which divide $\langle \gamma \rangle$ by previous.
Thus
$$\langle \gamma \rangle=\langle \sqrt{p} \rangle^j$$
If $j \ge 2$ then $p|\gamma,\bar{\gamma}$ which is impossible. Hence $j \le 1$.
If $j=0$, then $$\langle \gamma \rangle=\langle 1 \rangle$$ and so $\gamma$ is a unit.
So the norm $N(\gamma)=\gamma \bar{\gamma}=\pm 1$.
Hence
$$u=\frac{\gamma}{\bar{\gamma}}=\frac{\gamma^2}{\gamma \bar{\gamma}}=\pm \gamma^2$$
which contradicts $u$ is the fundamental unit. If $j=1$, then
$$\langle \gamma \rangle=\langle \sqrt{p} \rangle$$
and so $u=w\sqrt{p}$ for some unit $w \in \mathcal{O}_K$.
Hence
$$u=\frac{\gamma}{\bar{\gamma}}=\frac{w \sqrt{p}}{-\bar{w}\sqrt{p}}=\frac{w}{-\bar{w}}$$
But $N(w)=w\bar{w}=\pm 1$, so
$$u=\frac{w^2}{-w\bar{w}}=\pm w^2$$
which again contradicts that $u$ is the fundamental unit. Hence, combine all three cases, the fundamental unit must have norm $-1$.
\end{proof}
\begin{theorem} Let $m$ be a positive square free integer. If there exists a prime $p \equiv 3$ (mod $4$) dividing $m$, then the fundamental unit of $\mathcal{O}_{\mathbb{Q}(\sqrt{m})}$ has norm $1$. In particular, the fundamental unit of $\mathcal{O}_{\mathbb{Q}(\sqrt{p})}$ for $p$ a prime which is $3$ mod $4$ has norm $1$.
\end{theorem}
\begin{proof}[\bf Proof] Let $u$ be the fundamental unit. Let $u=\frac{x+y\sqrt{m}}{2}$, where $x,y \in \mathbb{Z}$, such that
$$x \equiv y \equiv 0~(\text{mod }2), \text{ if } m \equiv 2,3~(\text{mod } 4)$$
and
$$x \equiv y~(\text{mod }2), \text{ if } m \equiv 1~(\text{mod } 4)$$
Suppose the fundamental unit $u$ has norm $-1$. Then
$$N(u)=\frac{x^2-my^2}{4}=-1$$
and so $$x^2-my^2=-4$$
Since $p|m$, we have
$$x^2 \equiv -4~(\text{mod } p)$$
But $p \equiv 3$ (mod $4$), and so $(\frac{-4}{p})=(\frac{-1}{p})=-1$, which is a contradiction.
Hence $N(u)=1$.
\end{proof}


\subsection{Dirichlet's unit theorem}
In the previous subsection we have seen that if $K$ is a quadratic field. Then
$$U_K \cong C_2 \times C_\infty$$
by Theorem 17.2 In fact, theorem 17.2 can be generalised to an arbitrary number field. The main result we are
going to deduce in this subsection is
$$U_K \cong \mu_K \times (C_\infty)^m$$
where $\mu_K$ is the roots of unity in $\mathcal{O}_K$ and $m=r+s-1$ with $r+2s=n$.
\begin{lemma} Let $K$ be a number field. Define the map
$$L:\mathcal{O}_K \rightarrow \mathbb{R}^{r+s}$$
by
\begin{eqnarray*}
L(x)&=&(\log{|\sigma_1(x)|},\ldots,\log{|\sigma_r(x)|},2\log{|\sigma_{r+1}(x)|},\ldots,2\log{|\sigma_{r+s}(x)|})\\
&=&(x_1,\ldots,x_{r+s})
\end{eqnarray*}
where, as usual for each pair of complex embeddings, only one of them appears.
Then for all non-zero $\alpha \in \mathcal{O}_K$, with $L(\alpha)=(\alpha_1,\ldots,\alpha_{r+s})$,
there exists a non-zero $\beta \in \mathcal{O}_K$, with $|N(\beta)| \le \frac{4}{\pi}\sqrt{|\mathcal{D}_K|}$
such that $\beta_i \le \alpha_i$ for all $i \neq k$, where $k$ is a fixed index and $L(\beta)=(\beta_1,\ldots,\beta_{r+s})$
\end{lemma}
\begin{proof}[\bf Proof] Let $k$ be the fixed index, and for each $i \neq k$, choose $c_i$ such that
$$0 <c_i<e^{\alpha_i}$$
Then define $c_k$ such that
$$\prod_{i=1}^{r+s} c_i=\left(\frac{4}{\pi}\right)^s \sqrt{|\mathcal{D}_K|}$$
Define the region
$$E=\{(y_1,\ldots,y_r,z_1,\ldots,z_s) \in \mathbb{R}^r \times \mathbb{C}^s: |y_i| \le c_i, |z_j|^2 \le c_{r+j}\}$$
It is clear that $E$ is a convex body and
$$V(E)=2^r \pi^s \prod_{i=1}^r c_i \prod_{j=1}^s c_{r+j}=2^{r+2s}\sqrt{|\mathcal{D}_K|}=2^n cov(\sigma(\mathcal{O}_K)$$
Then, by Minkowski convex body theorem, there exists a non-zero $\beta \in \mathcal{O}_K$, such that
$$\sigma(\beta) \in E \cap \sigma(\mathcal{O}_K)$$
By construction, $N(\beta) \le \prod_{i=1}^{r+s}c_i \le \frac{4}{\pi}\sqrt{|\mathcal{D}_K|}$.
Let $L(\beta)=(\beta_1,\ldots,\beta_{r+s})$. If $i \neq k$, for $1 \le i \le r$, we have
$$\beta_i=\log{|\sigma_i(\beta)|} \le \log{c_i} <\alpha_i$$
and for $r+1\le i \le s$, we have
$$\beta_i=2\log{|\sigma_i(\beta)|} \le 2\log{\sqrt{c_i}} <\alpha_i$$
\end{proof}
\begin{lemma} There exists a unit $\theta$ in $\mathcal{O}_K$, if $L(\theta)=(\theta_1,\ldots,\theta_{r+s})$, then
$\theta_i <0$ for all $i \neq k$ where $k$ is a fixed index and $L$ is the map defined in Lemma 17.7.
\end{lemma}
\begin{proof}[\bf Proof] Let $k$ be a fixed index.
Let $x_1$ be any non-zero element in $\mathcal{O}_K$. Define $x_n$ inductively using Lemma 17.7 such that
$x_{n+1}$ is the non-zero element in $\mathcal{O}_K$ such that
$$|N(x_{n+1}| \le \frac{4}{\pi}\sqrt{|\mathcal{D}_K|}$$
and the $i^{th}$ coordinate of $L(x_{n+1})$ is less than the $i^{th}$ coordinate of $x_n$ for all $i \neq k$.
Hence we have a sequence $(x_n)_{n=1}^\infty$ in $\mathcal{O}_K$ such that for all $i \neq k$, the $i^{th}$ coordinate of $L(x_n)$ is decreasing and
$|N(x_n)|$ is bounded by $\frac{4}{\pi}\sqrt{|\mathcal{D}_K|}$. By Proposition 16.20, there are only finitely many
integral ideals with norm less than or equal to a certain bound. Hence there exists $n$ and $r>0$ such that
$$\langle x_n \rangle=\langle x_{n+r} \rangle$$
and so $\frac{x_{n+r}}{x_n}$ is a unit. Let $\theta=\frac{x_{n+r}}{x_n}$.
Let $L(x_{n+r})=(y_1,\ldots,y_{r+s}),L(x_n)=(z_1,\ldots,z_{r+s})$ and $y_i \le z_i$ for all $i \neq k$.
Then
$$L(\theta)=(y_1-z_1,\ldots,y_{r+s}-z_{r+s})=(\theta_1,\ldots,\theta_{r+s})$$
and so $\theta_i<0$ for all $i \neq k$.
\end{proof}
\begin{lemma} For each index $k$, $1\le r+s$, let $u_k$ be the units in Lemma 17.8. Let $m=r+s$ and $A$ be the
$m \times m$ matrix whose $i^{th}$ row is $L(u_o)$. Then $A$ is a real matrix, and $A_{ij}<0$ for all $i \neq j$,
and $A_{ii}>0$ for all $i$. Moreover, The sum of each row is $0$ and the rank of the matrix $A$ is $m-1$.
\end{lemma}
\begin{proof}[\bf Proof] It is clear that $A$ is real, the sum of each row, is
$$\log{\prod_{i=1}^{r+s}|\sigma_i(u_k)|}=\log{|N(u_k)|}=0$$
and by construction and by Lemma 17.8, $A_{ij}<0$ for $i \neq j$, and since the sum of each row is $0$, so
$A_{ii}>0$ for all $i$. Also, since the sum of each row is $0$, so the vector
$(1,1,\ldots,1)^t$ is in the kernel. Hence the rank is at most $m-1$. We show that the rank is exactly $m-1$.

Let $V_i$ be the $i^{th}$ column. Suppose
$$t_1V_1+\cdots+t_{m-1}V_{m-1}=0$$
$t_i$ not all zero. Then pick $k \in \{1,\ldots,m-1\}$ with $t_k$ maximal. We divide both sides by $t_k$, and
rewrite the above as
$$t_1V_1+\cdots+t_{m-1}V_{m-1}=0, t_i \le t_t=1$$
Now look at the $k^{th}$ row, say $(a_{k1},\ldots,a_{km})$. Then from the above sum, we have
$$0=\sum_{i=1}^{m-1}t_i a_{ki} \ge \sum_{i=1}^{m-1}>\sum_{i=1}^m a_{ki}=0$$
using $a_{kj}<0$ if $j \neq k$, which is a contradiction.
Hence the rank is at least $m-1$ and so the rank is $m-1$.
\end{proof}
Now we shall prove the main result of the subsection.
\begin{theorem}{\bf [Dirichlet's unit theorem]}\label{D;Dirichlet's unit theorem} Let $K$ be a number field.
Let $U_K$ be the set of units in $\mathcal{O}_K$. Then
$$U_K \cong \mu_K \times (C_\infty)^{r+s-1}$$
where $\mu_K$ is the set of roots of unity in $\mathcal{O}_K$ and $n=r+2s$ in the usual notation.
\end{theorem}
\begin{proof}[\bf Proof] The first part of the proof is identical to that in the quadratic case. Let
$$L: \mathcal{O}_K \rightarrow \mathbb{R}^{r+s}$$
by
$$L(x)=(\log{|\sigma_1(x)|},\ldots,\log{|\sigma_r(x)|},2\log{|\sigma_{r+1}(x)|},\ldots,2\log{|\sigma_{r+s}(x)|})$$
It is clear that $L$ is a homomorphism. Let $x$ be a unit, then $N(x)=\pm 1$. So the image of $U_K$ under $L$
lies in the hyperplane $W$, where
$$W=\left\{(x_1,\ldots,x_r,y_1,\ldots,y_s) \in \mathbb{R}^{r+s}: \sum_{i=1}^r x_i+\sum_{j=1}^s y_j=0\right\}$$

Let $B$ be a compact subset in $W$, then there exists $a \in \mathbb{R}$ such that
$$\frac{1}{a} \le |\sigma_i(x)| \le a$$
for all $i$ and for all $x \in L^{-1}(B)$. Since $|\sigma_i(x)|$ is bounded, and so the coefficients of the
minimal polynomial of $x$ are bounded, and so there are finitely many such polynomials because the coefficients are integers. Hence there are finitely many such $x$, and so $L^{-1}(B)$ is finite and so $B \cap L(U_K)$ is finite.
This shows that $L(U_K)$ is a discrete additive subgroup of $W$. Also, since
$\{0\}$ is a compact subset, and so $\ker{L} \cap U_K$ is finite, and since $U_K$ is a multiplicative group,
and the kernel of $U_K$ under $L$ is $\ker{L} \cap U_K$, hence
$$U_K/(\ker{L} \cap U_K) \cong V \subseteq W$$
where $V$ is a discrete additive subgroup. Then by Lemma 16.5 $U_K/(\ker{L} \cap U_K)$ is finitely generated
over $\mathbb{Z}$ and some $\mathbb{Z}$-basis is linearly independent over $\mathbb{R}$. So the rank of $V$
is less than or equal to the rank of $W$, which is $r+s-1$.
So we have
$$U_K/(\ker{L} \cap U_K) \cong \mathbb{Z}^{\oplus m}$$
where $m \le r+s-1$ and so
$$U_K \cong \mathbb{Z}^{\oplus m} \oplus (\ker{L} \cap U_K)$$
It is clear that $\ker{L} \cap U_K$ is a multiplicative subgroup of $K$ and we have shown it is finite. Hence any
finite subgroup of $K^*$ is cyclic, where $K$ is a field. Thus, $\ker{L} \cap U_K$ is cyclic. Since
it contains complex numbers, and so the only finite cyclic group of complex numbers is the group consisting of
roots of unity. Hence, $\ker{L} \cap U_K=\mu_K$.

On the other hand, by Lemma 17.9, we can find $r+s-1$ units such that the image of them under $L$ are linearly independent over $\mathbb{R}$. Hence we conclude that $m=r+s-1$, and so
$$U_K \cong \mu_K \times \mathbb{Z}^{\oplus m} \cong \mu_K \times (C_\infty)^m$$
\end{proof}
Now, we have a way to determine the units in $\mathcal{O}_K$, and so we are able to find the integer solutions to the equation
$$x^2+k=y^3$$
for any integer $k$. We shall see many examples in the exercise.

\subsection{Independent units}

Dirichlet's unit theorem suggests that we shall define the following concept:

\begin{definition} Let $K$ be a number field. Let $e_1,\ldots,e_k$ be units of $\mathcal{O}_K$.
The units $e_1,\ldots,e_k$ are said to be {\bf independent} if
$$e^{r_1}_1\cdots e^{r_k}_k=1 \text{ where } r_1,\ldots,r_k \in \mathbb{Z} \Rightarrow r_1=\cdots=r_k=0$$
\end{definition}

\begin{definition} Let $K$ be a number field with degree $n=r+2s$. We say $\{e_1,\ldots,e_{r+s-1}\}$
is a {\bf fundamental system} of units in $\mathcal{O}_K$ if $e_1,\ldots,e_{r+s-1}$ are independent and
for any unit $u \in \mathcal{O}_K$, $u$ can be {\bf uniquely} expressed in the form
$$u=\zeta e^{k_1}_1\cdots e^{k_{r+s-1}}_{r+s-1}$$
where $\zeta$ is a root of unity in $\mathcal{O}_K$ and $k_1,\ldots,k_{r+s-1}$ are integers.
\end{definition}
Therefore, this allows us to redefine the fundamental unit.
\begin{definition} Let $K$ be a number field with degree $n=r+2s$. If $r+s=2$, then by Dirichlet's unit theorem, the set of unit
$$U_K \cong \mu \times C_\infty$$
Then any unit $u \in \mathcal{O}_K$ such that $\{u\}$ is a fundamental system of units for $\mathcal{O}_K$ is called a {\bf fundamental unit}.
\end{definition}
The Dirichlet's unit theorem gives the structure of the set of units in $\mathcal{O}_K$, but in general, it is hard to find the fundamental system of units. We shall give two examples in the exercise. Also, the fundamental system
of units is not unique. For example, if $u$ is in the system, then we replace $u$ by $u^{-1}$, the new set
is also a fundamental system of units.
\begin{definition} Let $K$ be a number field of degree $n=r+2s$. Let $\sigma_1,\ldots,\sigma_r$ be the real embeddings and $\sigma_{r+1},\ldots,\sigma_{r+s}$ be the complex embeddings such that
$$\sigma_{r+j} \neq \bar{\sigma}_{r+i}$$
for any $i \neq j$. Let $\{e_1,\ldots,e_{r+s-1}\}$ be a fundamental system of units of $\mathcal{O}_K$.
Let $E$ be the $(r+s-1) \times (r+s)$ matrix with
$$E_{ij}=\log{|\sigma_i(e_j)|},~i=1,2,\ldots,r+s-1,j=1,2,\ldots,r+s$$
Let $A$ be any $r+s-1 \times r+s-1$ minor of $E$. Then the non-negative real number
$$R(K)=|\det{A}|$$
is called the {\bf regulator} of $K$.
\end{definition}
The first thing to check is that the regulator is independent of the $r+s-1\times r+s-1$ minor we chose.
\begin{proposition} Let $E$ be the matrix above. Let $A_i$ be the matrices formed by $E$ with the $i^{th}$
column deleted. Then for any $i \neq j$,
$$|\det{A_i}|=|\det{A_j}|$$
\end{proposition}
\begin{proof}[\bf Proof] Since the norm of unit is $\pm1$ and so the sum of each row is $E$ is $\log{1}=0$. Let $V_1,\ldots,V_{r+s}$ be the columns of $E$. Then we can write
$$V_{r+s}=-(V_1+\cdots+V_{r+s-1})$$
Hence, the absolute value of the determinant of any $A_i$ for $i \le r+s-1$, can be written as:
$$|\det{A_i}|=|\det{(V_1|\cdots|V_{i-1}|V_{i+1}|\cdots|-V_i)}|=|\det{A_{r+s}}|$$
by swapping the $i^{th}$ column and the last column because the elementary operation does not change the determinant. Hence we have shown that for any $i$,
$$|\det{A_i}|=|\det{A_{r+s}}|$$
and so
$$|\det{A_i}|=|\det{A_{j}}|$$
for any $i \neq j$.
\end{proof}
It is nature to ask whether the regulator of $K$ depends on the choice of fundamental system of units as we have
seen that the fundamental system of units is not necessarily unique.
\begin{proposition} Let $K$ be a number field. The regulator $R(K)$ is independent of the choice of basis.
\end{proposition}
\begin{proof}[\bf Proof] Let $\{e_1,\ldots,e_{r+s-1}\}$, and $\{f_1,\ldots,f_{r+s-1}\}$ be two fundamental systems of units.
Then by definition, for any $f_j$, we have $a_j, k_{ij} \in \mathbb{Z}$ any $\zeta$ a root of unity such that
$$f_j=\zeta^{a_j}e^{k_{1j}}_1\cdots e^{k_{r+s-1 j}}_{r+s-1}$$
Also, we have $b_j, m_{ij} \in \mathbb{Z}$ such that
$$e_j= \zeta^{b_j}f^{m_{1j}}_1\cdots f^{m_{r+s-1 j}}_{r+s-1}$$
Hence we have
\be
f_j =\zeta^{a_j}\prod_{i=1}^{r+s-1} e^{k_{ij}}_i = \zeta^{a_j}\prod_{i=1}^{r+s-1}\left(\zeta^{b_i}\prod_{l=1}^{r+s-1}f^{m_{li}}_l\right)^{k_{ij}} = \zeta^{a_j+\sum_{i=1}^{r+s-1}k_{ij}b_i}\prod_{l=1}^{r+s-1} f^{\sum_{i=1}^{r+s-1} m_{li}k_{ij}}_l
\ee
Then, by uniqueness of the expression, we have
\begin{equation*}
\sum_{i=1}^{r+s-1}m_{li}k_{ij}= \left\{
\begin{array}{ll}
1 & \text{if } l=j\\
0 & \text{if } l \neq j\\
\end{array} \right.
\end{equation*}
Now define the matrices $A$ and $B$ by
$$A_{ij}=k_{ij},B_{ij}=m_{ij}$$
and so
$$AB=BA=I$$
Thus
$$\det{A}\det{B}=1$$
But the matrices $A$ and $B$ have integer entries, and so
$$|\det{A}|=|\det{B}|=1$$
Now, let $\sigma_i$ be any embeddings. Then
$$\sigma_i(f_j)=\sigma_i\left(\zeta^{a_j}\prod_{l=1}^{r+s-1}e^{k_{lj}}_l\right)=\sigma(\zeta)^{a_j}\prod_{l=1}^{r+s-1}
\sigma_i(e_l)^{k_{lj}}$$
Since the conjugate of root of unity is still a root of unity, then we have
$$|\sigma_i(f_j)|=\prod_{l=1}^{r+s-1}|\sigma_i(e_l)|^{k_{lj}}$$
and so
$$\log{|\sigma_i(f_j)|}=\sum_{l=1}^{r+s-1}k_{lj}\log{|\sigma_i(e_l)|}$$
Now let $E$ and $F$ be the $(r+s-1) \times (r+s-1)$ matrices with entries
$\log{|\sigma_i(e_j)|}$ and $\log{|\sigma_i(f_j)|}$ respectively. Then from above, we have
$$F=AE$$
and so
$$|\det{F}|=|\det{AE}|=|\det{E}|$$
Hence the determinant (regulator) is independent of the choice of fundamental system of units.
\end{proof}
Now, as in the proof of Dirichlet's unit theorem, let $e_1,\ldots,e_{r+s-1}$ be $r+s-1$ units. Then
define the matrix $E$ by
$$E_{ij}=\log{|\sigma_i(e_j)|}$$
Then if any $(r+s-1) \times (r+s-1)$ minor has non-zero determinant, then $e_1,\ldots,e_{r+s-1}$ are independent.
Thus, this gives an easier way to check the independence of the units. But in general, it is still hard to
check whether these independent units form a fundamental system of units. One way to check this is to check whether
for each $e_i$, there exists some $f_i$ such that $e_i$ is a positive power of $f_i$. If not, then
the given set forms a fundamental system of units. In fact, we could use the class number formula to find the
regulator and check whether the determinant of $E$ is equal to the regulator.
\begin{definition} Let $K$ be a number field. The {\bf Dedekind $\zeta$-function} $\zeta_K(s)$ is defined by:
$$\zeta_K(s)=\sum_{0 \neq I \subseteq \mathcal{O}_K}\frac{1}{N(I)^{s}}$$
where the summation is over all non-zero integral ideals of $\mathcal{O}_K$.
\end{definition}
\begin{theorem}{\bf [Class number formula]}\label{C;Classnumber formula} Let $K$ be a number field with discriminant $\mathcal{D}_K$. Let $R(K)$ be the regulator of $K$
and $h(K)$ be the class number. Let $n$ be the degree of $K$ over $\mathbb{Q}$ and $n=r+2s$ in the usual notation.
Let $w(K)$ be the number of roots of unity in $K$. Then
$$\lim_{s \to 1^+}(s-1)\zeta_K(s)=\frac{2^r (2\pi)^s h(K)R(K)}{\omega(K) \cdot \sqrt{|\mathcal{D}_K|}}$$
\end{theorem}
If we can approximate the left hand side well enough, then we will be able to get a decent approximation
of $h(K)R(K)$. If we can also calculate the class number, then the above formula would give a good
approximation of the regulator, which helps to check whether any give $r+s-1$ units form a fundamental system
of units. The proof of the theorem is not elementary and we shall discuss this later.
\subsection{The Diophantine equation: $x(x+1)(x+2)=y(y+1)$}
in this subsection we use the arithmetic of the cubic field
$$K=\mathbb{Q}(\theta), \theta^3-4\theta+2=0$$
to determine all the solutions in integers $x$ and $y$ of the equation
$$y(y+1)=x(x+1)(x+2)$$
this is, we determine all those integers that are simultaneously a product of two consecutive integers and a product of three consecutive integers. This problem was first proposed by Edgar Emerson to Burton W.Jones and was first solved by Louis.J.Mordell.
By question 3 in Exercise 16, we know that the $\mathcal{O}_K$ with $K$ defined above has an
integral basis $\{1,\theta,\theta^2\}$ and has class number $1$ and so it is a principal ideal domain,
hence a unique factorisation domain. The fundamental system of units of $\mathcal{O}_K$ is $\{u,v\}$ where $u=\theta-1,v=2\theta-1$ (see question 4.)
Then we start with a couple of lemmas. Throughout this subsection, let $\theta=\theta_1,\theta_2,\theta_3$ be the roots of $x^3-4x+2=0$.
\begin{lemma} $\theta$ is a prime in $\mathcal{O}_K$. Moreover, $4\theta-3$ is a prime.
\end{lemma}
\begin{proof}[\bf Proof] It is clear that
$$N(\langle \theta \rangle)=|N(\theta)|=\left|\prod_{i=1}^3 \theta_i\right|=2$$
which is a prime. Hence $\langle \theta \rangle$ is a prime ideal and hence $\theta$ is prime.

Now the conjugates of $4\theta-3$ are
$$4\theta_1-3,4\theta_2-3,4\theta_3-3$$
and so
\begin{eqnarray*}
N(4\theta-3)&=&(4\theta_1-3)(4\theta_2-3)(4\theta_3-3)\\
&=&64\theta_1\theta_2\theta_3-48(\theta_1\theta_2+\theta_1\theta_3+\theta_2\theta_3)+36(\theta_1+\theta_2+\theta_3)
-27\\
&=&64(-2)-48(-4)+36(0)-27 =37
\end{eqnarray*}
which is again a prime. Hence $4\theta-3$ is a prime.
\end{proof}
\begin{lemma} The factorisation of $\langle 2\rangle$ is:
$$\langle 2\rangle=\langle \theta \rangle^3$$
In particular, $2=\gamma\theta^3$ where $\gamma$ is a unit in $\mathcal{O}_K$.
\end{lemma}
\begin{proof}[\bf Proof] $\theta^3=4\theta-2=2(2\theta-1)$, and
$$N(2\theta-1)=N(\frac{\theta^3}{2})=\frac{N(\theta)^3}{8}=-1$$
and hence $2\theta-1$ is a unit. Then
$$\langle 2 \rangle=\langle \theta \rangle^3$$
and so $2=\gamma \theta^3$ for some unit $\gamma$.
\end{proof}
Now in the equation
$$y(y+1)=x(x+1)(x+2)$$
we set $X=2x+2,Y=2y+1$, then we have
$$2Y^2=X^3-4X+2$$
\begin{lemma} If $(X,Y)$ is a solution to $2Y^2=X^3-4X+2$, then
$$(X-\theta)(X^2+\theta X+(\theta^2-4))=\gamma \theta^3 Y^2$$
In particular
$$\langle X-\theta \rangle \langle X^2+\theta X+(\theta^2-4) \rangle = 2 \langle Y^2\rangle$$
\end{lemma}
\begin{proof}[\bf Proof] We have
\be
(X-\theta)(X^2+\theta X+(\theta^2-4)) = X^3-4X-\theta^3+4\theta = X^3-4X+2 = 2Y^2 = \gamma \theta^3 Y^2
\ee
The last part follows by $\langle \theta \rangle^3=\langle 2 \rangle$.
\end{proof}
\begin{lemma} The only possible primes in $\mathcal{O}_K$ which divides both $X-\theta$ and $X^2+\theta X+(\theta^2-4)$ are associates of $\theta$ and $4\theta-3$.
\end{lemma}
\begin{proof}[\bf Proof] Let $\alpha$ be a prime in $\mathcal{O}_K$ which divides both $X-\theta$ and $X^2+\theta X+(\theta^2-4)$. Then $\alpha$ divides
\be
(X^2+\theta X + (\theta^2-4))-(X+2\theta)(X-\theta) = 3\theta^2-4=\frac{3\theta^2-4\theta}{\theta}=\frac{8\theta-6}{\theta} = \frac{2}{\theta}(4\theta-3)=\gamma(4\theta-3)\theta^2 
\ee
by Lemma 17.20. Since $\gamma$ is a unit, and $4\theta-3$ and $\theta^2$ are both irreducible because
they are prime and $\mathcal{O}_K$ is a unique factorisation domain.
Hence $\alpha$ can only be the associates of $4\theta-3$ or $\theta^2$.
\end{proof}
\begin{lemma} $\theta$ divides both $X-\theta$ and $X^2+X\theta+(\theta^2-4)$. Further, $\theta^2 \nmid X-\theta$,
$\theta^2 |X^2+X\theta+(\theta^2-4)$ but $\theta^3 \nmid X^2+X\theta+(\theta^2-4)$.
\end{lemma}
\begin{proof}[\bf Proof] As $X=2(x+1)$ so it is even, and since $\theta|2$ (as $\langle 2\rangle=\langle \theta\rangle^3$) then $\theta|X$ and
so $\theta|X-\theta$. Similarly, $\theta|X^2+X\theta+(\theta^2-4)$ and $\theta^2 |X^2+X\theta+(\theta^2-4)$. Finally, since $\theta^2|X$ and $\theta^2 \nmid \theta$, so $\theta^2 \nmid X-\theta$, and since $\theta^3 \nmid \theta^2$, $\theta^3 \nmid X^2+X\theta+(\theta^2-4)$.
\end{proof}
\begin{theorem} The integer solutions of the equation
$$2Y^2=X^3-4X+2$$
are
$$(X,Y)=(-2,\pm1),(0,\pm1),(2,\pm1)(4,\pm5),(12,\pm29)$$
\end{theorem}
\begin{proof}[\bf Proof] Let $n$ be the order of $4\theta-3$ in $X-\theta$, namely
$$(4\theta-3)^n |X-\theta,(4\theta-3)^{n+1} \nmid X-\theta$$
Now let
$$X-\theta=\theta (4\theta-3)^n \prod_{i=1}^r P^{e_i}_i,X^2+X\theta+(\theta^2-4)=\theta^2 (4\theta-3)^m \prod_{j=1}^t Q^{f_j}_j$$
be the prime factorisations. By Lemma 17.22, $P_i \neq Q_j$ for all $i,j$. Then by Lemma 17.21, we have
\be
(X-\theta)(X^2+X\theta+(\theta^2-4))= \theta^3 (4\theta-3)^{m+n}\prod_{i=1}^r P^{e_i}_i \prod_{j=1}^t Q^{f_j}_j = \gamma\theta^3 Y^2
\ee

This shows that the indices $e_i,f_i$ must all be even, and so collect all squared terms, using the fact
$\mathcal{O}_K=\mathbb{Z}[\theta]$, we may write
$$X-\theta=\theta(4\theta-3)^A u^B v^C (a+b\theta+c\theta^2)^2$$
where $a,b,c \in \mathbb{Z}$, $A,B,C \in \{0,1\}$. Take norms of both sides, since
$$N(X-\theta)=(X-\theta_1)(X-\theta_2)(X-\theta_3)=X^4-4X+2$$
and let $Z=N(a+b\theta+c\theta^2) \in \mathbb{Z}$, and the norm of unit is $\pm 1$, we have
$$X^4-4X+2=\pm 2 \cdot 37^A \cdot Z^2$$
Since $X$ is even, let $X=2X_1$, so
$$4X^3_1-4X_1+1=4(X^2_1-1)X_1+1=\pm 37^A Z^2$$
Reduce the above equation modulo $8$, and for any integer $Z$,
$$Z^2 \equiv 0,1,4~(\text{mod } 8)$$
It is clear that $4(X^2_1-1)X_1+1 \equiv 1$ (mod $8$), and so the required $Z$ must have
$$Z^2 \equiv ~(\text{mod } 8)$$
Hence
$$1 \equiv \pm 5^A~(\text{mod } 8)$$
This shows that $A$ is not $1$ and since $A \in \{0,1\}$, so $A=0$. Hence,
$$X-\theta=\pm \theta u^B v^C (a+b\theta+c\theta^2)^2$$
and expand the square, we have
$$X-\theta=\pm \theta  u^B v^C ((a^2-4bc)+(2ab+8bc-2c^2)\theta+(2ac+b^2+4c^2)\theta^2)$$
We now discuss the four cases:
$$(B,C)=(0,0),(0,1),(1,0),(1,1)$$
\begin{enumerate}
\item[(i)] $(B,C)=(0,0)$. Then we equate the coefficients of $1,\theta,\theta^2$ on both sides above,
we have
$$4ac+2b^2+8c^2=\pm X$$
$$a^2-4bc+4b^2+8ac+16c^2=\pm 1$$
$$ab+4bc-c^2=0$$
For the second equation above, reduce both sides modulo $4$, then we have the plus sign holds. Thus,
$$X=4ac+2b^2+8c^2$$
$$(a+4c)^2+4b^2-4bc=1$$
$$b(a+4c)=c^2$$
If $b=0$, then $c=0$ and then $X=0$. This gives the solution $(0,\pm 1)$.
If $b \neq 0$, then $a+4c=\frac{c^2}{b}$ and so
$$\frac{c^4}{b^2}+4b^2-4bc=1 \iff 3b^4+c^4-4b^3c=0$$
Now since
$$(c-b)^2((c+b)^2+2b^2) \ge 0$$
and so
$$\frac{c^4}{b^2}+4b^2-4bc \ge b^2$$
this shows that $b^2 \le 1$. Then $b= \pm 1$ and the corresponding $(a,b,c)$ are
$$(a,b,c)=(1,1,-3) \text{ or } (-1,-1,3)$$
This gives $X=-2$ and $Y=\pm 1$. 
\item[(ii)] Now consider the case $(B,C)=(0,1)$. We have
$$X-\theta=\pm \theta(2\theta-1)(a+b\theta+c\theta^2)^2$$
multiply both sides by $\theta$ and collect the term $\theta^2$ into the term $(a+b\theta+c\theta^2)^2$, we
have
$$(X\theta-\theta^2)=\pm (1-2\theta)(a^2-4bc+(2ab+8bc-2c^2)\theta+(2ac+b^2+4c^2)\theta^2)$$
Equate the coefficients, we have
$$0=a^2-4bc+4(2ac+b^2+4c^2)$$
$$\mp X=2ab+8bc-2c^2-2(a^2-4bc)-8(2ac+b^2+4c^2)$$
$$\pm 1=2ac+b^2+4c^2-2(2ab+8bc-2c^2)$$
From the first equation we have $a$ is even and from the last equation we have $b$ is odd. Now consider
the reduction modulo $4$ of the last equation, we have the plus sign holds.
Hence
$$0=(a+4c)^2+4b(b-c)$$
$$1=a(2c-b)+b^2-16bc+8c^2$$
$$x=2a^2+8b^2+34c^2-2ab+16ac-16bc$$
Suppose $c=2b$, then we have $b^2=1$ from the second equation. Let $b=1$, then $c=2$ and $a=-6,-10$.
Then we have $X=4$ or $X=12$ which gives the solutions
$$(4,\pm 5),(12,\pm 29)$$
Since $-a,-b,-c$ also gives the same solution, so we do not consider the case when $b=-1$.

Suppose now $c \neq 2b$, then from the second equation above, we have
$$a=\frac{b^2-16bc+8c^2-1}{4b-2c}$$
and so
$$a+4c=\frac{b^2-1}{4b-2c}$$
Substitute this into the first equation, we have
$$\left(\frac{b^2-1}{4b-2c}\right)^2+4b(b-c)=0$$
so that
$$b^2-2b^2+1+16b(2b-c)^2(b-c)=0$$
which implies that $b|1$ and so $b=\pm 1$, and so $c=\pm 1$. Then $a=\mp 4$. We have
$$(X,Y)=(2,\pm 1)$$
\item[(iii)] Now consider the case $(B,C)=(1,0)$. We have
$$\mp(X-\theta)=(\theta-\theta)^2(a^2-4bc+(2ab+8bc-2c^2)\theta+(2ac+b^2+4c^2)\theta^2)$$
Equating the coefficients of $\theta$ and $\theta^2$ we have
$$\pm 1=(a^2-4bc)+6(2ac+b^2+4c^2)-4(2ab+8bc-2c^2)$$
$$0=(2ab+8bc-2c^2)-(a^2-4bc)-4(2ac+b^2+4c^2)$$
The first equation shows that $a$ is odd, and the second equation shows that $a$ is even, which is impossible.\\
\item[(iv)] The last case is $(B,C)=(1,1)$. We have
$$\pm(X-\theta)=\theta(1-\theta)(1-2\theta)(a+b\theta+c\theta^2)^2$$
Multiply both sides by $\theta$ and collect the term $\theta^2$ into $(a+b\theta+c\theta^2)^2$, we may write
$$\pm(X-\theta)=(1-3\theta+2\theta^2)(a^2-4bc+(2ab+8bc-2c^2)\theta+(2ac+b^2+4c^2)\theta^2)$$
Equate the coefficients of $1$ and $\theta$, we have
$$0=(a^2-4bc)+6(2ac+b^2+4c^2)-4(2ab+8bc-2c^2)$$
$$\mp1=9(2ac+b^2+4c^2)-3(2ab+8bc-2c^2)+2(a^2-4bc)$$
The first equation shows that $a$ is even and then consider the reduction modulo $4$ in the first equation,
we conclude that
$$2b^2 \equiv 0~(\text{mod } 4)$$
and so $b$ is even. But the second equation shows that $b$ is odd, which is impossible.
\end{enumerate}
This completes the proof.
\end{proof}
Therefore, from the theorem above, and $X=2x+2,Y=2y+1$, we conclude that the solutions to
$$y(y+1)=x(x+1)(x+2)$$ are
\be
(x,y) =(0,0),(0,-1),(-1,0),(-1,-1),(-2,0),(-2,-1),(1,2),(1,-3), (5,14),(5,-15)
\ee

We shall see many other examples of Diophantine equations in later chapters.
\subsection{Exercise}
\begin{enumerate}
\item Prove that $8+3\sqrt{7}$ is the fundamental solution in $\mathbb{Q}(\sqrt{7})$. Hence, find all
integer solutions to the equation
$$x^2-7y^2=2$$
\item By considering the factorsation of $2$ and $5$ in $\mathcal{O}_K$ where $K=\mathbb{Q}(\sqrt{26})$, find all integer solutions to the equations
$$x^2-26y^2=\pm 10$$
(We only need the solutions $(x,y)$ which satisfies one of the above equations.)
\item
\begin{enumerate}
\item[(i)] For all $x\in \mathbb{R}$ and all $\theta \in \mathbb{R}$. Show that
$$\sin^2{\theta}(x-2\cos{\theta})^2 <x^2+4$$
\item[(ii)]
Let $K$ be a cubic field with one real and two complex embedding. Let $u>1$ be the fundamental
unit of $\mathcal{O}_K$. If we set the conjugates of $u$ to be $re^{i\theta}$ and $re^{-i\theta}$,
show that the discriminant of $\mathbb{Z}[u]$ is
$$-4\sin^2{\theta}(r^3+r^{-3}-2\cos{\theta})^2$$
Show further that the absolute value of discriminant of $\mathbb{Z}[u]$ is less than
$$4(u^3+u^{-3}+6)$$
\item[(iii)]
If $|\mathcal{D}_K| >32$, show that
$$\left|u^3-\left(\frac{|\mathcal{D}_K|}{8}-3\right)\right| >\frac{|\mathcal{D}_K|}{8}-\frac{15}{4}$$
Hence show that
$$u^3 >\frac{|\mathcal{D}_K|-27}{4}$$
\end{enumerate}
\item Let $K=\mathbb{Q}(\theta)$ where $\theta^3-4\theta+2=0$. 
Find the factorisation of primes $2,3,5,7,11,13$ in $\mathcal{O}_K$ and hence, using question 16 of exercise 14, and the fact (which we have proved in a previous exercise) that 
$$\lim_{s \to 1^+}(s-1)\zeta(s)=1$$
estimate the limit
$$\lim_{s \to 1^+}(s-1)\zeta_K(s)$$
Hence or otherwise, prove that $\{\theta-1, 2\theta-1\}$ is a system of fundamental solutions in $K$.\\
(Hint: You may use the class number formula.)
\item Use the fact that $\mathcal{O}_K=\mathbb{Z}[\sqrt[3]{2}]$ for $K=\mathbb{Q}(\sqrt[3]{2})$ and question $3$, show that the fundamental unit of $\mathcal{O}_K$ is $1+\sqrt[3]{2}+(\sqrt[3]{2})^2$.
\item Let $\phi$ be the Euler's totient function (Definition 2.14). Show that, if $\phi(k) \le n$, then
$k \le 2n^2$. Hence show that $\mathcal{O}_K$ can only contains finitely many roots of unity for any number field $K$.\\
\item Show that if $K$ is a number field of odd degree. Then the only roots of unity in
$\mathcal{O}_K$ are $\pm 1$.\\
\item
\begin{enumerate}
\item[(i)] Let $K=\mathbb{Q}(\sqrt{3},\sqrt{-1})$. Show that $\zeta_{12} \in \mathcal{O}_K$, where
$\zeta_{12}$ is the $12^{th}$ root of unity.
\item[(ii)] Further, show that $12$ is the largest integer $m$ such that $\zeta_m \in \mathcal{O}_K$ with
$[K:\mathbb{Q}]=4$.
\end{enumerate}
\item Show that
$$\left\{1+\sqrt{2},\sqrt{2}+\sqrt{3},\frac{1}{2}\left(\sqrt{2}+\sqrt{6}\right)\right\}$$
is a fundamental system of units for $\mathcal{O}_K$, where $K=\mathbb{Q}(\sqrt{2},\sqrt{3})$.
\item Prove that $x^3-x^2+x-2$ has only one real root $\theta$. Show further that
$$1.3<\theta<1.4$$
and hence show that the fundamental unit is $1+\theta^2$.
\item Let $K$ be a number field such that $U_K$ contains a non-real root of unity. Prove that
$N(\alpha >0)$ for every $\alpha \in K\backslash\{0\}$.
\end{enumerate}
The following questions are examples of
$x^2+k=y^3$
using congruence relation.
\begin{enumerate}
\item[12.] Consider the equation
$$x^2=y^3+45$$
By considering the congruence modulo $8$, show that $x$ is even and $y$ is odd. Further, by considering the congruence modulo $4$ and treating the case
$$y \equiv 3~(\text{mod } 8)$$
and
$$y \equiv 7~(\text{mod } 8)$$
separately, show that the equation above has no integer solutions.\\
(Hint: It may be useful to observe that $45=72-27=18+27$.)\\
\item[13.] Using a similar argument as above, show that
$$x^2=y^3+46$$ has no integer solution. Further, let
$$m \equiv 4~(\text{mod } 8) \text{ and } n \equiv 1~(\text{mod } 2)$$
and $n$ contains prime factors $p$ with
$$p \equiv 1 \text{ or } 3~(\text{mod } 8)$$
Let $k=m^3-2n^2$. Show that
$$x^2=y^3+k$$ has no integer solution.
\item[14.] \begin{enumerate}
\item[(i)] Classify the prime numbers $p$ such that $(\frac{3}{p})=1$.
\item[(ii)] Let
$$m \equiv 3~(\text{mod }4) \text{ and } n \equiv \pm 2~(\text{mod }8)$$
and $n$ contains prime factors $p$ with
$$p \equiv \pm 1~(\text{mod }12)$$
Let $k=m^3+3n^2$. Show that
$$x^2=y^3+k$$ has no integer solution.
\end{enumerate}
\end{enumerate}
Finally, we give two examples of $x^2=y^3+k$ with no integer solutions, checked by using algebraic approach, together with congruence approach.
\begin{enumerate}
\item[15.] Consider the equation
$$x^2=y^3+7$$
Show that the class number $h(K)$ is prime to $3$, where $K=\mathbb{Q}(\sqrt{7})$. Show that if we have an integer solution $(x,y)$, then
$$x+\sqrt{7}=u^n (a+b\sqrt{7})^3$$
where $u=8+3\sqrt{7}$, $n=0,\pm 1$ and $a,b \in \mathbb{Z}$. Show that we have no integer solutions when $n=0$.
For the case $n=\pm 1$, by considering the congruence modulo $3$ and $9$, show that we have no integer solutions. Hence, conclude that we have no integer solutions for $x^2=y^3+7$.\\
\item[16.] Use a similar method as above, show that the equation
$$x^2=y^3+6$$
has no integer solutions.\\
(Hint: you may consider the congruence modulo $9$ at some stage.)
\item[$^\star$ 17.] This question uses a different method. We are going to show the equation
$$y^3+51=x^2$$
has no integer solution. Let $Y=y^3+54$ and $X=x^2+3$.
\begin{enumerate}
\item[(i)] Show that $2 \nmid Y$.
\item[(ii)] Show that $3 \nmid Y$.
\item[(ii)] Show that $p \nmid Y$ for any prime number $p>3$. Hence conclude that the equation has no integer solutions.\\
    (Hint: You may use the fact that for any prime $p$ with $p \equiv 1$ (mod $3$), and
    $x^3 \equiv 2$ (mod $p$) is soluble if and only if there exist integers $u$ and $v$ such that
    $p=u^2+27v^2$.)
\end{enumerate}
\begin{remark}
The same method shows that $x^2-29=y^3$ has no integer solution by observing
$$x^2+3=y^3+4\cdot 2^3$$
\end{remark}
\end{enumerate}


\section{Hints and Solutions}
\subsection{Exercises 1}
\begin{enumerate}
\item $x=-5, y=3$.
\item To find the greatest common divisor of three integers say $(a,b,c)$, we can find the greatest common divisor of two of them, say $d=(a,b)$ and then find $(c,d)$. So we have $15 - 6 \cdot 2=3$ and then we have $10 - 3 \cdot 3 =1$. Hence we conclude $10 - (15- 6 \cdot 2) \cdot 3 =1$, and so $x = 6, y =1, z= -3$.
\item Let $2^k \le n < 2^{k+1}$ for some $k$, and consider $S = S_n \cdot 2^{k-1} \cdot 3 \cdot 5 \ldots$ (we multiply every odd number less than or equal to $n$). Then suppose $S_n$ is an integer, then $S$ is also an integer. But consider each term of $S$, all of them are integers except a $\frac{1}{2}$ from $2^{k-1} \cdot \frac{1}{2^k}$, and so $S$ is not an integer, which is a contradiction.
\item Let $n = 2^k$ for some $k$ and suppose we have $a,a+1 \ldots a+l, l \ge 1$ such that the sum of these integers equals $n$. Then we have $\frac{(2a+l)(l+1)}{2}=2^k$, i.e. $(2a+l)(l+1)$ is a power of $2$. Then as $l \neq 0$, and we must have $l$ is $2^t -1$ for some $t$ but then $2a+l$ cannot be a power of $2$. Conversely, if $n$ is not a power of $2$, then let $n=2^k t$ where $t$ is odd. Then if we want $\frac{(2a+l)(l+1)}{2}=n$, i.e. $(2a+l)(l+1)=2n=2^{k+1}t$. Then we simply let $l=2^{k+1}-1$ and $2a +l = t$. This is always possible as $t$ is odd and in particular, $t \ge 3$.
\item Use Theorem 1.5 for the general solution so we can write $(x,y)=(x_0+tb,y_0-ta)$ and so WLOG, we let $(x,y)$ be the solution such that $0 \le y <a$. So $y \le a-1$ and so $ax+by=n \le ax+b(a-1)$. Now $n \ge (a-1)(b-1)$, so we have $(a-1)(b-1) \le ax+b(a-1)$ and hence we rearrange the inequality, we get $1 \le a(x+1)$. Therefore, we must have $x \ge 0$ and so both $x,y$ are non-negative.\\
\item When $m>n$, $2^{2^m}=(((2^{2^n})^2)^2)^{\ldots}$ and so clearly $2^{2^n}$ divides $2^{2^m}$. The later follows from an obvious fact that any two consecutive odd number is coprime, and use $2^{2^n}$ divides $2^{2^m}$.
\end{enumerate}
\subsection{Exercises 2}
\begin{enumerate}
\item Pair $d$ and $\frac{n}{d}$ for each divisor $d$ and so the result follows.
\item If a pair $(x,y)$ has least common multiple $p^\alpha$, for some prime $p$ and $\alpha \ge 1$. Then, we must have, $a = p^\alpha, b = p^\beta$ for some $0 \ge \beta \le \alpha$ or vice versa. So we have $2\alpha +1$ such pairs. And $\tau(p^{2\alpha})=2\alpha+1$. Now let $n=p_1^{\alpha_1} \ldots p_k^{\alpha_k}$ and so we have $\prod_{i=1}^k (2\alpha_i +1)$ such pairs, which is equal to $\tau(n^2)$ as $\tau$ is multiplicative.\\
\item Suppose $n$ has a squared factor, then $\mu(n)=0$. Let $n=a^2 b$, where $a > 1$ and $b$ is square free. Then $d^2 | n \iff d|a$ and so LHS is $\sum_{d|a}\mu(d) = 0$ by Lemma 2.11.\\
    Suppose now $n$ is square free, then $d^2 |n \iff d=1$ and so we have LHS is just $\mu(1)=1$, and RHS is also 1.
\item Write $n=p_1^{\alpha_1} \ldots p_k^{\alpha_k}$, where $p_i$ are distinct primes. Then suppose $\sigma(n)$ is odd, as $\sigma$ is multiplicative, we have that $\sigma(p_i^{\alpha_i}$ is odd for each $i$. But for $p_i \neq 2$, $\sigma(p_i^{\alpha_i})=1+p_i+ \ldots + p_i^{\alpha_i}$ is odd if and only if $\alpha_i$ is even. And $\sigma(2^t)$ is odd $\forall k \ge 1$. Therefore, we conclude that $n$ is either a square or twice a square.
\item Let $g(i,j)=1$ if $j|i$ and $0$ otherwise. Then by Lemma 2.15, we have $(i,j)=\sum_{d|(i,j)}\phi(d) =\sum_{d \le n}g(i,d)g(j,d)\phi(d)$ as $d|i,d|j \iff d|(i,j)$. Now let $A_{i,j}=(i,j)$ and let $B_{i,j}=g(i,j)$, $C_{i,j}=g(j,i)\phi(i)$. Then we have $A=BC$ and so $det(A)=det(B)det(C)$. Clearly, $det(B)=1$ (It's upper triangular and every diagonal entry is 1). For $det(C)$, $C$ is also upper triangular, and the diagonal entry $C_{i,i}=\phi(i)$. Hence $det(A)=det(C)=\prod_{d=1}^n \phi(d)$.
\item Let $n = p_1^{\alpha_1} \ldots p_k^{\alpha_k}$. Then the only contributions of $d$ in the sum are those divisors $d=p_i^{\beta_i}$, where $1 \le \beta_i \le \alpha_i$. And we have that $\Lambda(p_i^{\beta_i})= \log{p_i}$ for each $i$ and so the total contribution is $\sum \alpha_i \log{p_i} = \log{\prod p_i^{\alpha_i}} = \log{n}$.\\
    Now by definition of $\Lambda$, we have that
\begin{equation*}
\sum_{n=0}^{\infty}\frac{\Lambda(n)}{n^s}=\sum_{p \text{ prime}} \sum_{k=1}^{\infty}\frac{\log{p}}{p^{ks}} = \sum_{p \text{ prime}} \frac{p^{-s}\log{p}}{1-p^{-s}}
\end{equation*}
Now we may check that $\frac{\zeta'(s)}{\zeta(s)}=\frac{d(\log{\zeta(s)}}{ds}$, which is:
\begin{equation*} \frac{d~[\log {\prod_p (1-p^{-s})^{-1}}]}{ds}= - \sum_p \frac{d (\log{1-p^{-s}})}{ds} = - \sum_{p \text{ prime}} \frac{p^{-s}\log{p}}{1-p^{-s}}
\end{equation*}
Hence we conclude that the sum is $-\frac{\zeta'(s)}{\zeta(s)}$.
\item The first part follows from the proof of Lemma 2.15. \\
Consider $\sum_{d|n}n \cdot f(d) = \sum_{d|n} \sum_{(a,d)=1} \frac{na}{d}$, and as in the proof, we know the that the sum runs through every integer in the set $\{1,2 \ldots n\}$ and so is $\frac{n(n+1)}{2}$. Therefore, since we have multiplied it by $n$, so $\sum_{d|n}f(d)=\frac{n+1}{2}$.\\
For the second part, apply Theorem 2.12 (The inversion formula) and Lemma 2.17.
\item The series is absolutely convergent for $s>1$ and so
\begin{equation*}
f(s)\zeta(s)=\sum_{m=1}^{\infty} m^{-s} \sum_{n=1}^{\infty}a_n n^{-s} = \sum_{n=1}^{\infty} n^{-s} \sum_{d|n}a_d
\end{equation*}
Hence $f(s)\zeta(s)=g(s)$ if and only if $$ \sum_{d|n}a_d=b_n $$
And for
\begin{equation*}
\sum a_n \left(\frac{x^n}{1-x^n}\right)=\sum b_n x^n
\end{equation*}
Now we expand the fraction $\frac{x^n}{1-x^n} = x^n(1+x^n+x^{2n} +\ldots)$ (We can expand as $0 \le x<1$), we still have that the above equality holds if and only if $$ \sum_{d|n}a_d=b_n $$
Thus we have proved the first part. For the second part, use the hint and write RHS as a sum. Then use Lemma 2.15 to write $n$ as a sum. It is then easy to check that $f(s)\zeta(s)=g(s)$, as $a_n=\phi(n), b_n=n$. So as in the proof of the first part, that $$ \sum_{d|n}a_d=b_n $$
\end{enumerate}


\subsection{Exercises 3}

\begin{enumerate}
\item Let $S$ be the set of all primes less than or equal to $x$. By fundamental theorem, we may factorise any integer $n \le x$ into prime factors in $S$, say $x=p_1^{\alpha_1} \ldots p_k^{\alpha_k}$, where we have $k=\pi(x)$ and as each $p_i \ge 2$, so $\alpha_i \le \frac{\log{x}}{\log{2}}$.
    Hence the result follows.
\item Induction on $n$. It is true when $n=0$, as $p_1=2 \le 3$. Suppose it is true for all $k \le n$, then consider $p_1 \cdot p_2 \ldots p_n +1$, this number is not divisible by any of the first $n$ primes and hence the $(n+1)^{th}$ prime must be less than or equal to this number. By induction, it is $\le (2^{2^0}+1)(2^{2^1}+1) \ldots (2^{2^n}+1) \le 2^{2^{n+1}}+1$. Hence it holds for all $n$. For the second part, replace $n$ with $\pi(n)$ in the result, and by definition $p_{\pi(n)+1}$ is the least prime greater than $n$ and hence $p_{\pi(n)+1}-1 \ge n$.
    Then take logarithm on both sides, and use $2 < e$, so that we have $\pi(n) \ge \log{\log{n}}$.
\item Let $p$ be a prime factor of $2^q-1$. Then we have $p|2^q -1$ and so
$$ 2^q \equiv 1~(\text{mod } p)$$ But $p$ is also prime, and so by Fermat's little theorem
$$ 2^{p-1} \equiv 1~(\text{mod } p)$$ Let $d$ be the order of $2$ mod $p$. Then $d|q$ and $d|p-1$. But $q$ is prime, so $d=1$ or $d=q$. But $d=1$ is impossible and so $d=q$. Hence $q|p-1$.
\item Suppose there are finitely many, say $p_1,p_2 \ldots p_n$. Then consider
$$ P=4\prod_{t=1}^{n}p_i -1$$ This is an integer of the form $4k+3$, but is not divisible by any $p_i$. Now consider the prime factor of $P$. If every prime factor is of the form $4k+1$, then the product $P$ also has the form $4k+3$. Therefore, we have some prime factor, say $p$, which has the form $4k+3$ and is distinct from $p_i$, which is a contradiction.
\item As $(a,m)=1$ we have the following two congruences
$$a^{\phi(m)}-1 \equiv 0~(\text{mod } m) \text{ and } a^{\phi(m)}-1 \equiv 0~(\text{mod } a-1)$$
and use Chinese Remainder theorem, as $(a-1,m)=1$, we have that
$$a^{\phi(m)}-1 \equiv 0~(\text{mod } (a-1)m)$$ and so $m|\frac{a^{\phi(m)}-1}{a-1}$. Therefore,
$$1+a+\ldots a^{\phi(m)-1}=\frac{a^{\phi(m)}-1}{a-1} \equiv 0~(\text{mod } m)$$
Now use the first part we conclude that for each prime $m$, if $(m,10)=(m,9)=1$. (i.e. $m \neq 2,3,5)$, then we have $m| \underbrace{11\ldots 1}_{m-1}$. For example, $7|111111$. Also, we may observe that $3|111$. So we can conclude that, if $m \neq 2,5$ then there is at least one integer $n$ of the form $11\ldots 1$ such that $m|n$. \\
If $m|n$, let $k$ be an integer such that $10^{k-1} \le n <10^k$, then $m|10^k n + n$. Hence we have construct another integer $n'=10^k n + n$ of the form $11\ldots 1$ such that $m|n'$. Use this observation, we conclude that there are infinitely many of them.\\
\item If $m$ is prime, then $(m-1)! \equiv -1$ (mod $m$) by Wilson's theorem. Conversely, suppose $m$ is not prime, then if $m$ is not a square, we have $m=ab$ with $a,b <m,a \neq b$, and so $(m-1)! \equiv 0$ (mod $m$). If $m$ is a square, let $m=p^2$, then as $m>4$, we have $p > 2$ and so $2p <p^2$. So $p \cdot 2p \equiv 0$ (mod $m$) and hence $(m-1)! \equiv 0$ (mod $m$).\\
    The second part is an immediate application of Wilson's theorem. As we have
    $$(p-1)!=-1~(\text{mod } p)$$ and observe that $p-k+t \equiv (-1)(k-t)$ (mod $p$) and hence the result follows by applying Wilson's theorem and counting the number of $t, 1 \le t \le k-1$.
\item By Euler's theorem, as $(m,n)=1$, we have
$$m^{\phi(n)} \equiv 1~(\text{mod } n) \text{ and } n^{\phi(m)} \equiv 1~(\text{mod } m)$$
and so we have
$$m^{\phi(n)}+n^{\phi(m)} \equiv 1~(\text{mod } m) \text{ and } m^{\phi(n)}+n^{\phi(m)} \equiv 1~(\text{mod } n)$$
To finish the proof, apply Chinese Remainder theorem.
\item If $p=2$, then it is clear that we have only one solution. If $p>2$, then we apply Theorem 3.38. Let $g$ be a generator of $(\mathbb{Z}/p^j\mathbb{Z})^*$ and then $g^{kp^{j-1}},k=1,2,\ldots p-1$ are the only solutions.
\item We need to check that the second congruence is consistent with the first one. But indeed if we multiply both sides by $-5$, We have $x \equiv 5$ (mod $12$). And so we don't need the first congruence anymore. Now solve the third congruence, we have $x \equiv 17$ (mod $30$). Now use Chinese Remainder theorem, we have $x \equiv 1$ (mod $4$), $x \equiv 2$ (mod $3$), $x \equiv 2$ (mod $5$). Therefore, we have $x \equiv 17$ (mod $60$).
\item Let $d=(m,n), k=\frac{m}{d},l=\frac{n}{d}$. Then it has a solution if and only if
$$x \equiv a~(\text{mod } d) \text{ and } x \equiv b~(\text{mod } d)$$
by applying Chinese Remainder theorem. This is consistent if and only if $d|(b-a)$, and so the result follows. Then again by Chinese Remainder theorem, the solution is unique up to modulo $d \cdot k \cdot l$, which is the least common multiple of $m,n$.
\item By Remark 3.42, we can easily check that $2,3$ are generators of $\mathbb{Z}/5\mathbb{Z}$, $\mathbb{Z}/25\mathbb{Z}$, and so they are generators of $\mathbb{Z}/5^n\mathbb{Z}$ for all $n$. Now for a natural number $n$, suppose it has a primitive root, say $g$. The order of the cyclic group is $k=\phi(n)$, and there are $\phi(k)$ integers which are less than $k$ and prime to $k$, say $\{1=a_1,a_2,\ldots a_{\phi(k)}\}$, and so $g^{a_i}$ gives a list of primitive roots, and so we have $\phi(k)=\phi(\phi(n))$ of them.
\item If $p=2$, it is obvious. If not, let $f(d)$ be the sum of $d^{th}$ primitive root, where $d$ is a factor of $p-1$. Then consider that $$1+2+\ldots p-1=\sum_{d|p-1}f(d)$$ because each element is some $d^{th}$ primitive root. We want to calculate $f(p-1)$. Let $g(p-1)=\sum_{d|p-1}f(d)$, then we apply Theorem 2.12, the inversion formula, we get $f(p-1)=\sum_{d|p-1}g\left(\frac{p-1}{d}\right) \mu(d)$. Now $g(p-1) \equiv 0$ (mod $p$). But $g(1)=1$, so we apply the inversion formula, we have $f(p-1) \equiv \mu(p-1)$ (mod $p$).
\item STEP(i): Congruence mod $5$, we have solutions $3,4$.\\
      STEP(ii): Congruence mod $25$
      \begin{enumerate}
      \item $x=3,f(3)=20,f'(3)=25$, we have no solution.\\
      \item $x=4,f(4)=55,f'(4)=46$, so $t=4$, and so $24$ is a solution.
      \end{enumerate}
      STEP(iii): Congruence mod $125$
      $x=19,f(19)=13375,f'(19)=1081$, so $t=0$ and so $24$ is a solution.\\
      Hence the only solution is $24$.
\item The first part is clear as $a^n \equiv 1$ (mod $N$) and $a^d < N$, if $d < n$. The order of the multiplicative group is $\phi(N)$ and so the order of $a$, which is $n$, divides $\phi(N)$.

    Now fix a prime number $n$, suppose there are only finitely many, say $q_1, \ldots q_k$. Let $Q=q_1 q_2 \ldots q_k$ for any $q_i \equiv 1$ (mod $n$), and let $N=(nQ)^n-1$. WLOG, $N=p_1^{e_1}\ldots p_t^{e_t}$, and by previous part, we know $n|\phi(N)=p_1^{e_1-1} \ldots p_t^{e_t-1}(p_1-1)\ldots(p_t-1)$. But clearly, $n \nmid N$, and so $n$ cannot be any $p_i$. Thus $n|p_i-1$ for some $i$. Hence, $p_i \equiv 1$ (mod $n$), but $q_j \nmid N$ for all $j$, and so $p_i \neq q_j ~\forall i,j$. So we have constructed another prime which is congruent to $1$ mod $n$, contradicting the assumption.
\item Observe that the numerator of $u_p$ is the coefficient of the sub-leading term of the given polynomial, possibly with a negative sign, which we don't need to care about. Now, by Lagrange's theorem and Fermat's little theorem, we have that the polynomial must be $0$ congruent to $p$, because $1,2 \ldots p-1$ are all solutions but the degree of the polynomial is $p-2$ (at most). Hence the sub-leading term is divisible by $p$.
    For the second part, we can rearrange the summation, so that
    $$1+\frac{1}{2}+\ldots+\frac{1}{p-1}=(1+\frac{1}{p-1})+(\frac{1}{2}+\frac{1}{p-2})
    +\ldots+(\frac{1}{\frac{p-1}{2}}+\frac{1}{\frac{p+1}{2}})$$
    which can be simplified as
    \be
\frac{pA}{(p-1)!}\quad \text{ where }A=\sum_{a=1}^{\frac{p-1}{2}} \frac{(p-1)!}{a(p-a)}
\ee
    Now we check that, for each $a$
    $$\frac{(p-1)!}{a(p-a)} \equiv a^{-2}~(\text{mod } p)$$
    Indeed, let $x=\frac{(p-1)!}{a(p-a)}$, then $a(p-a)x=(p-1)!$, which is congruent to $-1$ mod $p$, by Wilson's theorem. And as $(p-a) \equiv -a$ (mod $p$), hence $x \equiv a^{-2}$ (mod $p$).
    Use this, we conclude that
    $$A \equiv (1)^{-2}+(2)^{-2}+\ldots + (\frac{p-1}{2})^{-2}~(\text{mod } p)$$
    Therefore,
    $$2A \equiv (1)^{-2}+(2)^{-2}+\ldots + (\frac{p-1}{2})^{-2}+(1)^{-2}+(2)^{-2}+\ldots + (\frac{p-1}{2})^{-2}~(\text{mod } p)$$
    Writing $a^{-2}$ as $(-a)^{-2}$ in the above summation, we have
    $$2A \equiv \sum_{a=1}^{p-1} a^{-2}~(\text{mod } p)$$
    But summing over $a^{-1}$ is the same as summing over $a$, hence
    $$2A \equiv \sum_{a=1}^{p-1} a^2~(\text{mod } p)$$
    Now apply
    $$\sum_{a=1}^{p-1}a^2=\frac{(p-1)p(2p-1)}{6}$$
    as $p>3$, so $6 \nmid p$, and so $2A \equiv 0$ (mod $p$) and hence $A \equiv 0$ (mod $p$).
    Therefore, as $(p-1)! \nmid p$, so the numerator $\frac{pA}{(p-1)!}$ is congruent to $0$ mod $p$.
\end{enumerate}


\subsection{Exercises 4}
\begin{enumerate}
\item We want to find $p$ such that $(\frac{5}{p})=1$. By law of reciprocity, we have
    $(\frac{5}{p})=(\frac{p}{5})$. And so we have $p \equiv 1,4$ (mod $5$). In fact we have
    $p \equiv \pm 1$ (mod $10$) because $p$ is prime.
\item $(\frac{-1}{p})=(-1)^{\frac{p-1}{2}} =1$. So we know that $-1$ is a square mod $p$. Hence we have
    $$\left(\frac{p-a}{p}\right)=\left(\frac{-1}{p}\right)\left(\frac{a}{p}\right)
    =\left(\frac{a}{p}\right)$$
    Hence, we can pair each $a$ with $p-a$, and so the sum of quadratic residues and non-residues are both $\frac{p(p-1)}{4}$.\\
\item Observe that if $ab \equiv 1$ (mod $p$), then as
$$1=\left(\frac{ab}{p}\right)=\left(\frac{a}{p}\right)\left(\frac{b}{p}\right)$$
$(a,p),(b,p)=1$, so we must have $(\frac{a}{p})=(\frac{b}{p})$. So we can pair each quadratic residue with its inverse. The only elements whose inverse is itself are $1$ and $-1$. And use
$$\left(\frac{-1}{p}\right)=(-1)^{\frac{p-1}{2}}$$ to achieve the result.
\item If $p|d$, then we have $x \equiv 0$ (mod $p$) and so we have one solution. If $d$ is not a quadratic residue, then we have no solution. Finally, if $d$ is a quadratic residue, then we have at most two solutions by Lagrange's theorem. We need to check it has precisely two solutions. But this is clear as if $x^2 \equiv d$ (mod $p$), so is $(p-x)^2$.
\item Use question $5$, we know that each quadratic residues $d$ corresponds to two solutions, and so the number of quadratic residues is $\frac{p-1}{2}$, which is equal to the number of non-residues. Hence we have
    $$\sum_{a=1}^{p-1} \left(\frac{a}{p}\right)=0$$
    and so
    $$\sum_{a=2}^{p-1} \left(\frac{a}{p}\right)=-1$$
    Now $$\sum_{a=1}^{p-2}\left(\frac{a}{p}\right)\left(\frac{a+1}{p}\right)
    =\sum_{a=1}^{p-2}\left(\frac{a^2(1+a^{-1})}{p}\right)
    =\sum_{a=1}^{p-2}\left(\frac{1+a^{-1}}{p}\right)$$
    as $a^{-1}$ are all different and $p-1$ has itself inverse, so $1+a^{-1}$ runs through $\{2,\ldots p-1\}$ as $0$ has no inverse. Thus we have
    $$\sum_{a=1}^{p-2}\left(\frac{1+a^{-1}}{p}\right)=\sum_{a=2}^{p-1}\left(\frac{a}{p}\right)=-1$$\\
\item $p \equiv 3$ (mod $4$) and so $2p+1 \equiv 7$ (mod $8$). Hence $2$ is a quadratic residue mod $2p+1$. So we may write $2 \equiv x^2$ (mod $2p+1$). By Fermat's little theorem,
    $x^{2p} \equiv 1$ (mod $2p+1$) because $2p+1$ is prime. Therefore, $2^p \equiv 1$ (mod $2p+1$), and so $2^p-1$ is divisible by $2p+1$. Now as $p >3$, $2p+1 < 2^p-1$, so it is a proper factor and so $2^p-1$ is composite.\\
\item $N$ is odd, so we use Chinese Remainder theorem (ring version) to write
     $$(\mathbb{Z}/N\mathbb{Z})^* \cong \prod_i (\mathbb{Z}/{p_i}^{j_i}\mathbb{Z})^*$$
     WLOG we can assume that $p=p_1$, and so $j_1 \ge 2$ as $p^2|N$. Now pick a primitive root $g$ of $(\mathbb{Z}/{p_1}^{j_1}\mathbb{Z})^*$, and let $h=g^{p^{j-2}(p-1)}$, so $h$ has order $p$ in the group and is non-identity. Then, pick $z$ to be the element corresponding to
     $$(h,e,\ldots,e )$$ in the ring and so this element $z$ gives $z^p \equiv 1$ (mod $N$) but
     $z \not \equiv 1$ (mod $N$).


     Now pick $a$ to be the $z$ above. Suppose the congruence holds, then we square both sides, we have
     $$a^{N-1} \equiv 1~(\text{mod } N)$$
     But $(p,N-1)=1$ and so we have $x,y$ such that $px+y(N-1)=1$. Therefore, we have
     $$a=a^{px+y(N-1)} \equiv 1~(\text{mod }N)$$
     which contradicts the way we chose $a$. So the congruence does not hold for this $a$.
     Finally, observe that $a^{\frac{N-1}{2}}b^{\frac{N-1}{2}}=(ab)^{\frac{N-1}{2}}$, and
     $(\frac{ab}{N})=(\frac{a}{N})(\frac{b}{N})$. So if we have no element for which the congruence holds, then we are done. If we have one for which the congruence holds, say $b$, then the congruence does not hold for $ab$,
     and hence the congruence does not hold for at least half of all relatively prime residue classes modulo $N$.
\item One way is clear, if $a$ is a square, then we simply take $x$ to be $\sqrt{a}$ and so the congruence is soluble.

    Conversely, suppose $a$ is not a square, then we got to prove that there exists a prime $p$ such that $(\frac{a}{p})=-1$ (In fact we can find infinitely many, the proof of this is in a similar manner). To do this, we firstly find some odd integer $n$ such that $(\frac{a}{n})=-1$.

    We firstly deal with the case when $a$ is positive. Let $a=b^2 r$ where $r$ is square free and let $r=p_1 p_2 \ldots p_k$, where $p_i$ are distinct primes. Then $(\frac{a}{n})=(\frac{r}{n})$.
    Now pick $n$ which satisfies:
    $$n \equiv 1~(\text{mod }4)$$
    $$n \equiv 1~(\text{mod }p_i)$$
    $$n \equiv z~(\text{mod }p_1)$$
    where $i \ge 2, (\frac{z}{p_1})=-1$. Such $n$ exists by Chinese Remainder theorem.

    Then by law of reciprocity  $(\frac{r}{n})=(\frac{n}{r})$ and
    $$\left(\frac{n}{r}\right)=\prod_{i=1}^{k}\left(\frac{n}{p_i}\right)=-1$$
    Hence $(\frac{a}{n})=-1$ and so there must exists a prime factor $p$ of $n$ such that
    $(\frac{a}{p})=-1$.

    Now suppose $a$ is negative, and so let $a=-b^2 r$, where $r>0$ and is square free. Again let
    $r=p_1 p_2 \ldots p_k$, where $p_i$ are distinct primes.Then $(\frac{a}{n})=(\frac{-r}{n})$, and we pick $n$ satisfying the same condition as above. Use $(\frac{-1}{n})=(-1)^{\frac{n-1}{2}}$, but $\frac{n-1}{2}$ is even as $n \equiv 1$ (mod $4$). Hence we again have $(\frac{a}{n})=-1$ and so there must exists a prime factor $p$ of $n$ such that $(\frac{a}{p})=-1$.
\item Let $p \nmid d$. So we can simplify the sum as
    $$\left(\frac{a}{p}\right)\sum_{x=1}^{p}\left(\frac{(x+b(2a)^{-1})^2-d(4a)^{-1}}{p}\right)$$
    by completing the square, and $b(2a)^{-1}$ is just a linear transformation, and so it is the same as
    $$\left(\frac{a}{p}\right)\sum_{y=1}^{p}\left(\frac{y^2+e}{p}\right)$$
    where $e=-d(4a)^{-1}$ which is not divisible by $p$.

    Now use question $4$ that the number of solutions to the congruence
    $$x^2 \equiv y^2+e~(\text{mod } p)$$
    is $1+(\frac{y^2+e}{p})$ for each fixed $y$, and so in total we have
    $$\sum_{y=1}^{p} \left(1+\left(\frac{y^2+e}{p}\right)\right)$$
    solutions. Use the hint, we know this is equal to $p-1$. So we conclude that
    $$\sum_{y=1}^{p} \left(\frac{y^2+e}{p}\right)=-1$$
    and so the result follows.

    For the case when $p|d$, the above expression is simplified to
    $$\sum_{y=1}^{p}\left(\frac{y^2}{p}\right)=\sum_{y=1}^{p-1}1=p-1$$
    and so it is $(p-1)(\frac{a}{p})$.\\
\item Consider $\sum_{r=1}^{p-1}(\frac{r}{p})r$. There are two ways to sum this.
    $$\sum_{r=1}^{p-1}r\left(\frac{r}{p}\right)=\sum_{r=1}^{\frac{p-1}{2}}(2r)\left(\frac{2r}{p}\right)
    +\sum_{r=1}^{\frac{p-1}{2}}(p-2r)\left(\frac{p-2r}{p}\right)$$
    or we have
    $$\sum_{r=1}^{p-1}r\left(\frac{r}{p}\right)=\sum_{r=1}^{p-1}r\left(\frac{r}{p}\right)
    +\sum{r=1}^{\frac{p-1}{2}}(p-r)\left(\frac{p-r}{p}\right)$$
    As $p \equiv 3$ (mod $8$) we have $(\frac{-1}{p})=(\frac{2}{p})=-1$.
    So we can equate the above two and we have
    LHS:
    $$\sum_{r=1}^{\frac{p-1}{2}}(-2r)\left(\frac{r}{p}\right)+
    \sum_{r=1}^{\frac{p-1}{2}}\left\{p\left(\frac{r}{p}\right)-2r\left(\frac{r}{p}\right)\right\}$$
    RHS:
    $$\sum_{r=1}^{\frac{p-1}{2}}r\left(\frac{r}{p}\right)
    +\sum_{r=1}^{\frac{p-1}{2}}\left\{(-p)\left(\frac{r}{p}\right)+r \left(\frac{r}{p}\right)\right\}$$
    And so rearrange the equation we have
    $$\sum_{r=1}^{\frac{p-1}{2}} 2p\left(\frac{r}{p}\right)
    =\sum_{r=1}^{\frac{p-1}{2}}6r\left(\frac{r}{p}\right)$$
    and so $$p \sum_{r=1}^{\frac{p-1}{2}}\left(\frac{r}{p}\right)
    =3 \sum_{r=1}^{\frac{p-1}{2}}r\left(\frac{r}{p}\right)$$
    As $p > 3$, so we have
    $$3|\sum_{r=1}^{\frac{p-1}{2}}\left(\frac{r}{p}\right)$$
\item If $d$ is odd then apply Lemma 4.18, we have $(\frac{d}{|d|-1})=(\frac{|d|-1}{|d|})$, and so it gives $(\frac{-1}{|d|})$, if $d>0$, by definition of Kronecker symbol, $d \equiv 1$ (mod $4$), and so it is $1$.
    If $d<0$, then $|d| \equiv 3$ (mod $4$), and so it is $-1$, which gives the result.

    Now let $d$ be even, $d=2^s t$. Then we have
    $$\left(\frac{d}{|d|-1}\right)=\left(\frac{2^s}{2^s t-1}\right)\left(\frac{t}{2^s t-1}\right)$$
    By definition of Kronecker symbol, $d \equiv 0$ (mod $4$), and so $s \ge 2$.
    If $s =2$ or $s$ is even, then $(\frac{2^s}{2^s t-1})=1$. if not $2^s t -1 \equiv 7$ (mod $8$), and so $(\frac{2^s}{2^s t-1})=(\frac{2}{2^s t -1})=1$.

    Then for $(\frac{t}{2^s t -1})$, then if $d>0$, i.e. $t>0$, by law of reciprocity we have, as
    $2^s t-1 \equiv 3$ (mod $4$), that
    $$\left(\frac{t}{2^s t -1}\right)=(-1)^{\frac{t-1}{2}}\left(\frac{2^s t-1}{t}\right)$$ which is $$(-1)^{\frac{t-1}{2}}(-1)^{\frac{t-1}{2}}=1$$
    If $d<0$, i.e. $t<0$, we have $(\frac{-t}{1-2^s t})$ which is $-1$.
    So $(\frac{d}{|d|-1})=sgn(d)$.

    For the second part, use Theorem 4.19, so we know that
    $$\left(\frac{d}{n}\right) = \left(\frac{d}{m|d|-m}\right)$$
    as $n \equiv m|d|-m$ (mod $|d|$). Now
    $$\left(\frac{d}{m|d|-m}\right)=\left(\frac{d}{m}\right)\left(\frac{d}{|d|-1}\right)$$
    and the result follows by applying the first part.
\item $\phi(N)=\phi(p)\phi(2p-1)=2(p-1)^2$. We need to check that $N$ is pseudo prime for precisely
    $(p-1)^2$ bases. We have
    $$b^{N-1} \equiv 1~(\text{mod } N) \iff b^{N-1} \equiv 1~(\text{mod } p,2p-1)$$
    $N-1=p(2p-1)-1=(2p+1)(p-1)$. By Fermat's little theorem:
    $$b^{2p-1} \equiv 1~(\text{mod }2p-1) \text{ and } b^{p-1} \equiv 1~(\text{mod }p)$$
    Now as $p-1|N-1$, so $b^{N-1} \equiv 1$ (mod $p$) for every base $b$, and this has $p-1$ solutions.

    For $2p-1$, $b^{N-1} = b^{p(2p-2) +(p-1)} \equiv b^{p-1}$ (mod $2p-1$), so we want
    $b^{p-1} \equiv 1$ (mod $2p-1$). But $b^{2p-2} = (b^{p-1})^2 \equiv 1$ (mod $2p-1$) and so we have precisely $\frac{\phi(2p-1)}{2}=p-1$ solutions satisfying this. So use Chinese Remainder theorem, we have in total $(p-1)^2$ bases which satisfy $b^{N-1} \equiv 1$ (mod $N$).\\
\item (i) $4^p+1=2^{2p}+1=(2^{p}+1)^2-2^{p+1}$. $p$ is a prime $>5$ and so is odd,
    so $2^{p+1}=(2^{\frac{p+1}{2}})^2$ and so
    $$4^p+1=(2^{p}+1)^2-(2^{\frac{p+1}{2}})^2=(2^{p}+1+2^{\frac{p+1}{2}})(2^{p}+1-2^{\frac{p+1}{2}})$$
    Clearly as $p>5$, both factors are greater than $1$ and is not $5$. Hence
    $\frac{4^p+1}{5}$ is composite.

    Let $N-1=2^s t$. We want to show that
    $$\text{Either } 2^t \equiv 1~(\text{mod } N) \text{ or } 2^{2^r t} \equiv -1~(\text{mod }N)$$
    for some $0 \le r <s$. $N-1=4 \frac{4^{p-1}}{5}$ and so $s = 2, t= \frac{4^{p-1}}{5}$.
    But $5N=4^p+1=2^{2p}+1$, so we have $$2^{2p} \equiv -1~(\text{mod }N)$$ Also as $p>5$, so
    $4^{p-1}-1 \equiv 0$ (mod $p$) and so $p|t$. Let $t=mp$ with $m$ odd.
    Therefore we have
    $$2^{2t}=2^{2mp}=(2^{2p})^{m} \equiv (-1)^m \equiv -1~(\text{mod } N)$$
    And so there exists such $r$, $r=1$ and $0 \le r < 2$ as required.\\
    (ii) Let $N=2^M-1$, and $N-1=2t$, where $t=2^{M-1}-1$. So $2^M \equiv 1$ (mod $N$).
    Also as $M$ is a pseudo prime to the base $2$, so
    $$2^{M-1} \equiv 1~(\text{mod } M)$$
    So that $M|2^{M-1}-1$ and so $2^{2^{M-1}-1}=2^t \equiv 1$ (mod $N$). Hence the result follows.
\item (i) Suppose $(a,b)U=(c,d)U$, then $(ac^{-1},bd^{-1}) \in U$. If $ac^{-1}=1$ then $bd^{-1}=1$ and so $a=c, b=d$. If $ac^{-1}=p-1=$ then $bd^{-1}=p-1$, and so $b+d \equiv 0$ (mod $p$) which is impossible. Hence we know that these cosets are all different and there are $\frac{(p-1)(q-1)}{2}$ of them, and so they form a complete system of coset representatives for $U$. The product is:
    $$\prod_{i=1}^{p-1} \prod_{j=1}^{\frac{p-1}{2}} (i,j)
    =\left((p-1)!^{\frac{q-1}{2}},(\frac{q-1}{2})!^{p-1}\right)$$
    We can write $(\frac{q-1}{2})!^{p-1}$ as $(\frac{q-1}{2})!^{2\frac{p-1}{2}}$
    and for each $k>\frac{q-1}{2}, k \equiv -(p-k)$ (mod $p$).
    Hence we have
    $$\left((\frac{q-1}{2})!^{p-1}\right)=(\frac{q-1}{2})!^{\frac{p-1}{2}}(\frac{q-1}{2})!^{\frac{p-1}{2}}
    =(q-1)!^{\frac{p-1}{2}}(-1)^{\frac{q-1}{2}\frac{p-1}{2}}$$\\
    (ii) As $(p,q)=1$, so we have, by Chinese Remainder theorem,
    $$(\mathbb{Z}/p\mathbb{Z})^* \times (\mathbb{Z}/q\mathbb{Z})^* \cong (\mathbb{Z}/pq\mathbb{Z})^*$$
    where the isomorphism is sending $t \in (\mathbb{Z}/pq\mathbb{Z})^*$ to
    $(t_p,t_q)$ in $(\mathbb{Z}/p\mathbb{Z})^* \times (\mathbb{Z}/q\mathbb{Z})^*$, with $t_p,t_q$ defined in the question.

    $U$ corresponds to $\{1,pq-1\}$ in
    $(\mathbb{Z}/pq\mathbb{Z})^*$ because $pq-1 \equiv p-1$ (mod $p$) and is $q-1$ (mod $q$). And so $\{1,2,\ldots \frac{pq-1}{2}\}$ is a complete system of coset representatives. Now reduce it to
    $(\mathbb{Z}/p\mathbb{Z})^* \times (\mathbb{Z}/q\mathbb{Z})^*$, it is just $L$.

    Then it is clear that the first entry is the product of all these, except those multiples of $p$ and $q$, and so we get the required product.

    Now the numerator is congruence to
    $(p-1)!^{\frac{q-1}{2}}(\frac{p-1}{2})!$ mod $p$ (Remember the last term terminates at a multiple of $p$ plus $\frac{p-1}{2}$.

    The denominator is congruent to $(\frac{p-1}{2})! q^{\frac{p-1}{2}}$ mod $p$, and so together we have $\frac{(p-1)!^{\frac{q-1}{2}}}{q^{\frac{p-1}{2}}}$ mod $p$.

    Recall from Lemma 4.4 (Euler's criterion) that $(\frac{q}{p}) \equiv q^{\frac{p-1}{2}}$ (mod $p$), and $(\frac{q}{p})^{-1}=(\frac{q}{p})$ so we have shown the first entry is
    $(p-1)!^{\frac{q-1}{2}}(\frac{q}{p})$. As the product is symmetric in $p$ and $q$, so in the second entry we just swap $p$ and $q$.\\
    (iii)As we proved that both $\Pi$ and $\Phi$ form complete system of coset representatives and so their products must be the same in the quotient group of $U$. This is because, as both are just product of the representatives, so it means that
    $$\Phi U=\Pi U$$
    Then $\Phi= \pm \Pi$, but in either case we have that the proportion of the first entry is equal to that of the second entry. Therefore, we have
    $$\left(\frac{q}{p}\right)=\left(\frac{p}{q}\right)(-1)^{\frac{(p-1)(q-1)}{4}}$$
    which proves the law of reciprocity.
\end{enumerate}


\subsection{Exercises 5}
\begin{enumerate}
\item $(x-\sqrt{N}y)(x+\sqrt{N}y)=M$ and so $x -\sqrt{N}y>0$. So we have$x>\sqrt{N}y$.
   Now consider that
   $$\frac{x}{y}-\sqrt{N}=\frac{M}{y(x+\sqrt{N}y)} < \frac{M}{2\sqrt{N}y^2}$$
   But $M \le \sqrt{N}$ and hence we have
   $$\left|\sqrt{N}-\frac{x}{y}\right| < \frac{1}{2y^2}$$
   By Theorem 5.15, $\frac{x}{y}$ is a convergent of $\sqrt{N}$.
\item This is clearly true for $n=0,1$. And for $n \ge 2$, we notice that
      $$det\begin{pmatrix} a_0 &-1& 0& \ldots&0&0\\ 1&a_1& -1 &\ldots & 0 &0\\.&.&.&.&.&.\\
      0&0&0&\ldots&1&a_n \end{pmatrix}$$
      is just
       $$a_n det\begin{pmatrix} a_0 &-1& 0& \ldots&0&0\\ 1&a_1& -1 &\ldots & 0 &0\\.&.&.&.&.&.\\ 0&0&0&\ldots&1&a_{n-1} \end{pmatrix} - det \begin{pmatrix} a_0 &-1& 0& \ldots&0&0\\ 1&a_1& -1 &\ldots & 0 &0\\.&.&.&.&.&a_{n-2}\\ 0&0&0&\ldots&1&-1 \end{pmatrix}$$
      Now use induction, and combine the above expression, we have exactly that
      the determinant gives $a_n p_{n-1}+p_{n-2}$ because
      $$-det \begin{pmatrix} a_0 &-1& 0& \ldots&0&0\\ 1&a_1& -1 &\ldots & 0 &0\\.&.&.&.&.&a_{n-2}\\ 0&0&0&\ldots&1&-1 \end{pmatrix}=det \begin{pmatrix} a_0 &-1& 0& \ldots&0&0\\ 1&a_1& -1 &\ldots & 0 &0\\.&.&.&.&.&a_{n-2}\\ 0&0&0&\ldots&1&p_{n-2} \end{pmatrix}$$
     Similarly, we have $q_0=1$ and
     $$q_n=det\begin{pmatrix} a_1 &-1& 0& \ldots&0&0\\ 1&a_2& -1 &\ldots & 0 &0\\.&.&.&.&.&.\\
      0&0&0&\ldots&1&a_n \end{pmatrix}$$
      for $n \ge 1$.

      For $\frac{p_{n+1}}{p_n}$, it is easy to check that this holds for $n=0$. Now use induction on $n$ and notice that $p_{n+1}=a_{n+1}p_n+p_{n-1}$, and hence
      $$\frac{p_{n+1}}{p_n}=a_{n+1}+\frac{p_{n-1}}{p_n}$$
      But assume $\frac{p_n}{p_{n-1}}=[a_n,\ldots,a_0]$ by induction, then we have
      $$a_{n+1}+\frac{1}{[a_n,\ldots,a_0]}=[a_{n+1},a_n,\ldots,a_0]$$
      Use a similar argument for $\frac{q_{n+1}}{q_n}$.
\item \begin{enumerate}
      \item[(i)] Observe that $$\frac{1+\sqrt{5}}{2}=\frac{1}{\frac{\sqrt{5}-1}{2}}$$
      Hence by induction that $\alpha_n=\frac{\sqrt{5}+1}{2}$ for all $n$ and so
      $\frac{\sqrt{5}+1}{2}=[1,1,1,\ldots]$. Therefore, as we can easily check that
      $p_0=u_2,p_1=u_1$, and $p_n=a_n p_{n-1}+p_{n-2}=p_{n-1}+p_{n-2}$. Thus
      $p_n=u_{n+2}$ and similarly $q_n=u_{n+1}$.\\
      \item[(ii)] If $p<5$, we can check directly for the first few cases. Let $p>5$, expand
      $\alpha^n-\alpha'^n$ for $n$ odd, we have
      $$2^{n-1}u_n=\left(n+\binom{n}{3}5+\ldots+5^{\frac{n-1}{2}}\right)$$
      As $p$ is a prime, and so $2^{p-1} \equiv 1$ (mod $p$), and
      $p|\binom {p}{k}$ for $1 \le k \le p-1$. Hence we have
      $$u_p \equiv 5^{\frac{p-1}{2}} \equiv \left(\frac{5}{p}\right) ~(\text{mod } p)$$
      by Euler's criterion (Lemma 4.4). By Lemma 5.9, we have
      $$p_n q_{n-1}-p_{n-1}q_n=(-1)^{n-1}$$
      Use (i), and $p$ odd, we have
      $$u^2_p-u_{p+1}u_{p-1}=1$$
      By above, $u^2_p \equiv 1$ (mod $p$) and hence $p|u_{p-1}u_{p+1}$ and so
      $$p|u_{p-1} \text{ or } p|u_{p+1}$$
      Now consider when $n$ is even, we have
      $$2^{n-1}u_n=n+\binom{n}{3}+\ldots+n 5^{\frac{n-2}{2}}$$
      Let $n=p+1$, and we have $2^p u_{p+1} \equiv 1+ (\frac{5}{p})$ (mod $p$).
      Therefore, if $(\frac{5}{p})=-1$, we have $p|u_{p+1}$. Otherwise, we have
      $p|u_{p-1}$.

      But $5 \equiv 1$ (mod $4$) and so
      $$\left(\frac{5}{p}\right)=\left(\frac{p}{5}\right)=1 \iff p \equiv \pm 2~(\text{mod }p)$$
      The result follows by combining the above arguments.
      \end{enumerate}
\item As in the proof of Theorem 5.12, we have
      $$\left|\alpha-\frac{p_n}{q_n}\right|=\frac{1}{q_n (\alpha_{n+1}q_n+q_{n-1})}$$
      and we set $\beta_{n+1}=\frac{q_{n-1}}{q_n}$, and we aim to show that
      $$\alpha_n+\beta_n \le \sqrt{5}$$ cannot hold for three consecutive convergents if $n>1$.

      Suppose the above holds for $n-1$ and $n$, then
      Use $\alpha_{n-1}=a_{n-1}+\frac{1}{\alpha_n}$, and
      $$\frac{1}{\beta_n}=\frac{q_{n-1}}{q_{n-2}}=a_{n-1}+\beta_{n-1}$$
      we have
      $$\frac{1}{\alpha_n}+\frac{1}{\beta_n}=\alpha_{n-1}+\beta_{n-1}$$
      As we assume $\alpha_n+\beta_n \le \sqrt{5}$, and use the above we have
      $$1=\frac{\alpha_n}{\alpha_n} \le \left(\frac{-1}{\beta_n} +\sqrt{5}\right)
      \left(-\beta_n+\sqrt{5}\right)$$
      Rearrange the inequality we have
      $$\beta_n+\frac{1}{\beta_n} < \sqrt{5} \text{ and so } \beta_n >\frac{\sqrt{5}-1}{2}$$
      as $0<\beta_n<1$.

      If now we assume $$\alpha_n+\beta_n \le \sqrt{5}$$ holds for $n$ and $n+1$, then the above
      holds if we replace $\beta_n$ by $\beta_{n+1}$. That is, the inequality holds for $n-1,n,n+1$/
      Then
      $$a_n=\frac{q_n-q{n-2}}{q_{n-1}}=\frac{1}{\beta_{n+1}}-\beta_n
      <(-\beta_{n+1}+\sqrt{5})-\beta_n$$
      Then use $-\beta_n,-\beta_{n+1}<-\frac{-1+\sqrt{5}}{2}$, we have
      $$a_n<\sqrt{5}-(-1+\sqrt{5})=1$$
      which contradicts $\alpha$ being irrational (as then $a_n \ge 1$).
\item $a_0=0$ and so $\alpha_1=\frac{1}{\alpha},a_1=[\frac{1}{\alpha}]$. So the probability that
      $a_1 \ge m$ is the probability that $\alpha \le \frac{1}{m}$, which is $\frac{1}{m}$. Now the
      probability that $a_1=m$ is the same as $a_1 \ge m$ but not $a_1 \ge m+1$ as $a_1$ is an integer.
      Hence it is $\frac{1}{m}-\frac{1}{m+1}=\frac{1}{m(m+1)}$.

      Now
      $$\alpha_2=\frac{1}{\frac{1}{\alpha}-a_1}$$
      So the probability of $a_2 \ge m$ is the same as $\alpha_2 \le \frac{1}{m}$. i.e. we have
      $$\frac{1}{a_1} \le \alpha \le \frac{1}{a_1+m}$$ for each $a_1$
      and so we have
      $$\sum_{a_1=1}^{\infty}\frac{1}{a_1(ma_1+1)}$$

      Put $m=2$ we have
      $$\sum_{a_1=1}^{\infty}\frac{1}{a_1(2a_1+1)}=2\sum_{n=1}^{\infty}\left(\frac{1}{2n}-
      \frac{1}{2n+1}\right)$$
      Now expand $\log{2}=\log{1+1}$, we have
      $$\log{2}=1-\frac{1}{2}+\frac{1}{3}-\frac{1}{4} \ldots$$
      and hence the summation above is equal to $2(1-\log{2})$. Therefore, the probability is
      $$2\log{2}-1$$ as the probability that $a_2 \ge 1$ is $1$.\\
\item \begin{enumerate}
      \item[(i)] We have $\frac{1}{\alpha_{n+1}}=\alpha_n-a_n$, and so we can check directly by algebra that
      the conjugates satisfies
      $$\frac{1}{\alpha'_{n+1}}=\alpha'_n-a_n$$
      Then use $-1<\alpha'_0<0$ and the above, and apply induction on $n$, we deduce that
      $-1<\alpha'_n<0$ for all $n$.\\
      \item[(ii)] We have $\frac{1}{\alpha'_{n+1}}=\alpha'_n -a_n$ and so
      $[\frac{-1}{\alpha'_{n+1}}]=[-\alpha'_n]+a_n$. Use (i), $0<-\alpha'_n<1$ and so
      $[-\alpha'_n]=0$. Therefore, we have $a_n=[\frac{-1}{\alpha'_{n+1}}]$.\\
      \item[(iii)] By Theorem 5.22, we know $a_n$ is periodic and we have
      $\alpha_j=\alpha_k$ for some $j,k$. We also have that
      $$\alpha_{j-1}=\frac{1}{\alpha_j}+a_{j-1}=\frac{1}{\alpha_j}+\left[\frac{1}{\alpha'_j}\right]$$
      by (ii). Also if $\alpha_j=\alpha_k$, then we take conjugates on both sides, we would have
      $\alpha'_j=\alpha'_k$. Therefore, use the equation above, we conclude that if
      $\alpha_j=\alpha_k$, then $\alpha_{j-1}=\alpha_{k-1}$ and hence the period starts from $a_0$.
      \end{enumerate}
\item By Lemma 5.5, we have
      $$\alpha=\frac{\alpha_n p_{n-1}+p_{n-2}}{\alpha_n q_{n-1}+q_{n-2}}$$
      Now as $\alpha$ is purely periodic, and so we have $\alpha=\alpha_n$ for some $n$. Pick that $n$,
      and we rearrange the above equation, we have
      $$\alpha^2q_{n-1}+\alpha(q_{n-2}-p_{n-1})-p_{n-2}$$
      as required.

      Now consider that $f(0)=-p_{n-2}<0$, $f(-1)=q_{n-1}-q_{n-2}+p_{n-1}-p_{n-2}>0$ and
      $f(1)=q_{n-1}+q_{n-2}-p_{n-1}-p_{n-2}<0$ and so we conclude that:
      One of the roots is greater than $1$ and the other one is between $0$ and $-1$.
      Hence the result follows.
\item We have
      $$n=x^3-dy^3=y^3\left(\left(\frac{x}{y}\right)-d^{\frac{1}{3}}\right)
      \left[\left(\frac{x}{y}+\frac{d^{\frac{1}{3}}}{2}\right)^2+\frac{3}{4}d^{\frac{2}{3}}\right]$$
      Hence we have
      $$\left|\frac{x}{y}-d^{\frac{1}{3}}\right|=\frac{|n|}{y \cdot y^2
      \left[\left(\frac{x}{y}+\frac{d^{\frac{1}{3}}}{2}\right)^2+\frac{3}{4}d^{\frac{2}{3}}\right]}
      <\frac{|n|}{y\cdot y^2 \frac{3}{4}d^{\frac{2}{3}}}$$
      Now, since $y_0>\frac{8|n|}3d^{\frac{2}{3}}$, we have that
      $$\left|\frac{x_0}{y_0}-d^{\frac{1}{3}}\right|<\frac{1}{2d^{\frac{2}{3}}}$$
      and hence $\frac{x_0}{y_0}$ is a convergent by Theorem 5.15.
\item This is always soluble as we have solutions to $x^2-dy^2=1$ and let $v=2x,w=2y$.
      Now let $D=\frac{v+w\sqrt{d}}{2}$, and let $v_0,w_0$ be a positive solution such that $D$ is minimal.
      It is easy to check that if $\{a,b\},\{v,w\}$ are solutions, then
      $\{x,y\}$ defined as
      $$\frac{x+y\sqrt{d}}{2}=\left(\frac{a+b\sqrt{d}}{2}\right)\left(\frac{v+w\sqrt{d}}{2}\right)$$
      is also a solution.

      Further, let $\{t,u\}$ be a solution and we choose $m$ so that
      $$D^m \le \frac{t+u\sqrt{d}}{2} <D^{m+1}$$
      and so we have
      $$1 \le \left(t+u\sqrt{d}\right)D^{-m} < D$$
      Write $\frac{(t+u\sqrt{d})}{2}D^{-m}=\frac{X+Y\sqrt{d}}{2}$, then $\{X,Y\}$ is a solution with $X,Y$ non-negative because we have $1 \le \frac{X+Y\sqrt{d}}{2} <D$ and
      so $0<\frac{X-Y\sqrt{d}}{2} \le 1$ and hence $X \ge 1$ and $Y\sqrt{d} \ge 0$. Then it follows by the minimality of $D$ that $X=2$ and $Y=0$. Hence $\frac{t+u\sqrt{d}}{2}=D^m$ for some $m$.
      The proof is just an analogue of Theorem 5.23.
\item \begin{enumerate}
      \item[(i)]
      $\alpha>1$, and $\alpha'=[\sqrt{d}]-\sqrt{d}$ and so $-1<\alpha'<0$. Use question $6$, we know
      that $\alpha$ is purely periodic. As $k$ is the period, so we have $\alpha=\alpha_{nk}$, and
      $\alpha \neq \alpha_i$ for $k \nmid i$.
      $$\frac{c_{nk}+\sqrt{d}}{e_{nk}}=\frac{c_0+\sqrt{d}}{e_0}=[\sqrt{d}]+\sqrt{d}$$
      Compare both sides for the irrational parts, we have $e_{nk}=1$.

      Conversely, if $e_i=1$, then we have $\alpha_i=c_i+\sqrt{d}$ and $\alpha'_i=c_i-\sqrt{d}$. By question $6$, we know that $-1<\alpha'_i<0$ and $c_i$ is integer. Hence we have $c_i=[\sqrt{d}]$.
      Therefore, $\alpha_i=\alpha$ and so $k|i$.
      \item[(ii)] As $\alpha_i>0$, suppose $e_i = -1$, then we have $c_i<-\sqrt{d}$.
      Then $\alpha'_i=-c_i+\sqrt{d}>0$, contradicts what we had in question $6$.
      \item[(iii)] In fact $e_i \ge 0$ for all $i$ in (ii), using a similar argument. Then we have by (i)
      that $e_i \ge 2$ for $i=1,2,\ldots,k-1$. Now as
      $$c_{i+1}=a_i e_i-c_i \le \alpha_i e_i-c_i =c_i+\sqrt{d}-c_i=\sqrt{d}$$
      So that
      $$\alpha_i=\frac{c_i+\sqrt{d}}{e_i} \le \frac{\sqrt{d}+\sqrt{d}}{2}$$
      and so $a_i < \sqrt{d}$.
      \end{enumerate}

     Now We have
     $$\alpha_1=\frac{1}{\sqrt{d}-[\sqrt{d}]}$$ and so
     $$\alpha'_1=\frac{1}{-\sqrt{d}-[\sqrt{d}]}$$ which lies between $-1$ and $0$.
     By question $6$, $\alpha_1$ is purely periodic and $\alpha=[a_0,\alpha_1]$, the period starts at
     $a_1$. Let the period be $k$. Also, by question $6$, we know that
     $$a_k=\left[-\frac{1}{\alpha'_{k+1}}\right]=\left[-\frac{1}{\alpha'_{1}}\right]
     =[d+\sqrt{d}]=2a_0$$ as required.

     Finally, we have $\sqrt{d}=[a_0,\ldots,\frac{1}{-a_0+\sqrt{d}}]$, and so apply Lemma 5.5, we have
     $$\sqrt{d}=\frac{\frac{1}{-a_0+\sqrt{d}}p_k+p_{k-1}}{\frac{1}{-a_0+\sqrt{d}}q_k+q_{k-1}}$$
     Rearrange the above, we have a quadratic equation for $d$, which is
     $$x^2-x\left(a_0+\frac{p_{k-1}}{q_{k-1}}-\frac{q_k}{q_{k-1}}\right)-\frac{p_k}{q_{k-1}}
     +\frac{a_0p_{k-1}}{q_{k-1}}$$
     Compare the rational part and irrational part, so the coefficient of $x$ must be $0$, and so
     we have
     $$a_0+\frac{p_{k-1}}{q_{k-1}}-\frac{q_k}{q_{k-1}}=0$$
     Now $\frac{p_{k-1}}{q_{k-1}}=[a_0,\ldots,a_{k-1}]$ and
     by question $2$ we have
     $$\frac{q_k}{q_{k-1}}=[a_k,\ldots,a_1]$$
     But $a_k=2a_0$, and so we rearrange the equation, we get
     $$a_0+\frac{1}{a_1+\frac{1}{a_2+\ldots \frac{1}{a_{k-1}}}}
     =a_0+\frac{1}{a_{k-1}+\frac{1}{a_{k-2}+\ldots \frac{1}{a_1}}}$$
     The result follows by comparing both sides.
\item We need to find $n$ and $k$ such that
     $$\frac{n(n+1)}{2}=k^2$$
     Rearrange the above, we have $n^2+n=2k^2$.
     Multiply both sides by $4$, we have $4n^2+4n=8k^2.$ completing the square, we have
     $$(2n+1)^2-8k^2=1$$
     Now, let $d=8$ which is positive but not a square.
     As we have infinitely many solutions for the Pell's equation
     $$x^2-8y^2=1$$ and each solution $\{x,y\}$ must have $x$ odd, so we have infinitely such pairs
     $\{n,k\}$.
\end{enumerate}
\subsection{Exercises 6}
\begin{enumerate}
\item It is equivalent to show:
$$\prod_{p \le N}p^{\frac{N}{p-1}} \ge \left(\frac{N}{e}\right)^N$$
By Lemma 6.7, the exact power of $p$ diving $N!$ is
$$\sum_{k=1}^{\infty}\left[\frac{N}{p^k}\right] \le N \sum_{k=1}^\infty \frac{1}{p^k}=\frac{N}{p-1}$$
and so we have
$$\prod_{p \le N}p^{\frac{N}{p-1}} \ge N! \ge \left(\frac{N}{e}\right)^N$$
\item Let $f(x)$ be a polynomial, so it has finitely many $x$ such that $f(x) \le 1$. WLOG, let
$f(a)=p >1$. Then, as $kp+a \equiv a$ (mod $p$), so that
$$f(kp+a) \equiv f(a)~(\text{mod } p)$$
because $f(b)-f(a)$ always has a factor $b-a$. Hence $p \big| f(kp+a)~\forall k$.
Now we only have finitely many $x$ such that $f(x)=p$, and finitely many $x$ such that $f(x)=2p,3p,\ldots$ and so the results follows as we can take any multiple of $p$.
\item We check manually that $7=2+5,8=3+5,9=2+7,10=3+7,11,12=5+7,13=2+11,14=3+11,15=2+13,16=3+13,
17=2+3+5+7,18=7+11,19=3+5+11,20=7+13,21=2+3+5+11,22=2+7+13,23=3+7+13,24=11+13,25=5+7+13,26=2+11+13$.
We apply induction in this way: suppose for any $k^{th}$ prime $\ge 7$, we can write any $x$ with
$p_k \le x < 2p_k$ as a sum of distinct primes without using any primes larger than $p_k$, then, consider any
$y$ with $p_{k+1} \le y < 2p_{k+1}$.
By Bertrand's postulate (Theorem 6.10), we have $p_k < p_{k+1}<2p_k$
and so
$$p_k \le y-p_{k+1} < 2p_k$$
Now by assumption we may write $y-p_{k+1}$ into sum of primes no larger than $p_k$ and so this shows that we can write any $y$ with $p_{k+1} \le y < 2p_{k+1}$ into sum of distinct primes which are no larger than $p_{k+1}$.

The above arguments proves the inductive step, and it is true for $p_k=13$ where we started with. Hence, by induction, we know that for all $k$, with $p_k \ge 7$, we can write $x$ with $p_k \le x <2p_k$ into a sum of distinct primes which are no larger than $p_k$. Also as $p_{k+1}<2p_k$,
we have $[p_k,2p_k) \cup [p_{k+1},2p_{k+1})=[p_k,2p_{k+1}]$. Hence we have covered every integer.
\item \begin{enumerate}
\item[(i)] If $k<n$, then
$$\frac{1}{n}+\ldots+\frac{1}{n+k} \le \frac{1}{n}+\ldots+\frac{1}{2n-1} <n \frac{1}{n}=1$$
and so it is not soluble

If $k \ge n$, then by Bertrand's postulate, we have a prime $p$ between $n$ and $2n$. Let $p$ be the largest prime between $n$ and $n+k$. Then we multiply both sides by $\frac{(n+k)!}{p}$. Clearly, that $m\frac{(n+k)!}{p}$ is still an integer. On the other hand, each term on the left hand side is an integer apart from the term $\frac{1}{p} \frac{(n+k)!}{p}$.  And so it is not soluble.

\item[(ii)] If $k,m$ and $n>1$, then let $p \big| n!$, and so $p<n$. Then $p|m^k$ and hence $p|m$.
So the exact order of $p$ in $n!$ is at least two. But by Bertrand's postulate, there is a prime
$p$ between $[\frac{n}{2}]$ and $n$. This prime can only occur once in $n!$, which is a contradiction. Hence at least one of $k,m$ ad $n$ is $1$.
      \end{enumerate}
\item  \begin{enumerate}
\item[(i)] By Theorem 6.9, we have $\pi(n) < \frac{6n}{\log{n}}$. Replace $n$ by $p_n$ and use $\pi(p_n)=n$. we have $p_n>\frac{1}{6}n\log{p_n}>\frac{1}{6}n\log{n}$.
    Also we have $\frac{1}{8} <\frac{n \log{p_n}}{p_n}$ by the other part of inequality in Theorem 6.9.
    Now we want to find some $n$ such that $\log{n} < \frac{\sqrt{n}}{8}$. But as $\log{n}=o(\sqrt{n})$,
    we have some $n_0$, such that $\log{n} < \frac{\sqrt{n}}{8}$ for $n \ge n_0$.
    Then for $p_n \ge n_0$, we have $\frac{p_n}{8\log{n}} >\frac{n}{\sqrt{n}}$ and hence we have
    $$p_n < n^2$$
    Therefore, we have $p_n < 8n\log{p_n} <16n\log{n}$, for $p_n \ge n_0$. And for $p_n \le n_0$, we can
    check manually and get a suitable constant if necessary. Hence, both $c$ and $c'$ exist.
\item[(ii)] Suppose we have an upper bound, say $k$, such that for all $n$, $p_{n+1}-p_n < k$.
    Then $p_{n+t}-p_n < tk$ and use (i) we have
    $$c(n+t)\log{n+t}<p_{n+t}<c'n \log{n}+kt$$
    for some constants $c$ and $c'$ and any $t \ge 1$. But this is false for $t$ large enough.
    \end{enumerate}
\item We have
$$\sum_{n \le x}\frac{\Lambda(n)}{n}=\sum_{p \le x}\frac{\log{p}}{p}+\sum_{p^2 \le x} \frac{\log{p}}{p^2}+\ldots$$
    The first term is $\log{x}+O(1)$ by Theorem 6.11. The other terms are convergent and there are only finitely many terms as we have some $k$ such that $2^k > x$. Hence the result follows by combining these two.
\item By Lemma 2.22, $$\frac{1}{\zeta(s)}=\sum_{n=0}^\infty \frac{\mu(n)}{n^s}$$
Let $s=\sigma+it$ and hence $$\left|\frac{1}{\zeta(s)}\right| \le \zeta(\sigma)$$ by
triangle inequality. If we have a double zero at $1+it$. Then we have a double pole for $\frac{1}{\zeta(s)}$.
But $\zeta(\sigma)$ has only a simple pole at $\sigma=1$, which is a contradiction.
\item \begin{enumerate}
      \item[(i)] Let $t_n=n, z_n=1,f(x)=\log{x}$ in Lemma 6.12. Then we have
      $$Z(x)=[x] \text{ and } f'(y)=\frac{1}{y}$$
      $$\sum_{n \le x} \log{n}=[x]\log{x} - \int_1^x \frac{[y]}{y} dy=x\log{x}-x+O(\log{x})$$
      using $(y) \le 1$ and so $\int_1^x \frac{1}{y}dy=\log{x}$.\\
      \item[(ii)] Let $t_n=n, z_n=1, f(x)=\log^2{x}$ in Lemma 6.12. Then we have
      $$z(x)=[x] \text{ and } f'(y)=\frac{2\log{y}}{y}$$
      \begin{eqnarray*}
      \sum_{n \le x} \log^2{n}&=&[x]\log^2{x}-\int_1^x \frac{2[y]\log{y}}{y}dy\\
      &=& x\log^2{x}-2x\log{x}+2x+O(\log^2{x})
      \end{eqnarray*}
      using $(y) \le 1$ and so $\int_1^x \frac{2\log{y}}{y}=\log^2{x}$.
      \end{enumerate}
\item \begin{enumerate}
      \item[(i)] Expand $\log{(1-x)}$ for $0<x \le 1$ and use the fact that
      $$\sum_{n=1}^\infty \frac{1}{n^s}$$ converges for $s>1$.\\
      \item[(ii)]
      $$\phi(n)=n\prod_{p | n}\left(1-\frac{1}{p}\right)$$
      Use (i), we have
      $$\log{\frac{\phi(n)}{n}}
      =-\sum_{p |n}\left(\log{\left(1-\frac{1}{p}\right)}+\frac{1}{p}\right)
      >-\sum_{p |n}\frac{1}{p}-c$$ for some constant $c$.
      Let $p_1,p_2,\ldots p_r$ be prime factors of $n$ and $p_1,\ldots,p_s$ be those less than $\log{n}$.
      Then
      $$\sum_{p|n}\frac{1}{p}=\sum_{i=1}^s \frac{1}{p_i}+\sum_{i=s+1}^r \frac{1}{p_i}
      =S_1+S_2$$
      By Theorem 6.13, we have
      $$S_1 <\log{\log{p_s}}+c<\log{\log{\log{n}}}+c$$
      \end{enumerate}
\item \begin{enumerate}
\item[(i)] By Theorem 6.17, we have
$$(s-1)(\zeta_s)=1+(s-1)(1-s\int_1^\infty x^{-s-1}(x)dx$$
and so the result follows immediately by taking the limit when $s \to 1^+$.
\item[(ii)]
The functional equation works for $\sigma <0$ and so we have
$$\zeta(0)=\lim_{s \to 0^-}2^s \pi^{s-1}\Gamma(1-s)\zeta(1-s)\sin{\left(\frac{\pi s}{2}\right)}$$
which can be rearranged as:
$$\left(\frac{1}{2}\right)\lim_{s \to 0-}2^s \Gamma(1-s)(-s)\zeta(1-s)\left[\frac{\sin{\left(\frac{\pi s}{2}\right)}}{\frac{\pi^{1-s} s}{2}}\right]$$
as $s \to 0$, $2^s \Gamma(1-s) \to 1$ and $\left[\frac{\sin{\left(\frac{\pi s}{2}\right)}}{\frac{\pi^{1-s} s}{2}}\right] \to 1$. Now consider that
$$\lim_{s \to 0^-}(-s)\zeta(1-s)=\lim_{t \to 1^+}(t-1)\zeta(t)$$
Recall by Theorem 6.17 that
$$\zeta(t) (t-1)=t -t(t-1)\int_1^\infty x^{-t-1}(x)dx$$
and so
$$\lim_{s \to 0^-}(-s)\zeta(1-s)=\lim_{t \to 1^+}(t-1)\zeta(t)=1$$
Therefore $\zeta(0)=-\frac{1}{2}$.
\item Let $\frac{t}{e^t-1}=\sum_{n=0}^\infty \frac{b_n t^n}{n!}$. Then multiply both sides by
$e^t-1$ and expand in $t$, we have
$$t=\sum_{m=1}^\infty \frac{t^m}{m!}\sum_{n=0}^\infty \frac{b_n t^n}{n!}$$
Equating the coefficient for $t$ we have (if $k+1=m+n$)
$$b_0=1 \text{ and } \frac{1}{(k+1)!}\sum_{j=0}^{k+1}\binom{k+1}{j}b_j=0$$
for each $k$. Hence we have
$$b_k(k+1)=-\sum_{j=0}^{k-1}\binom{k+1}{j} b_j$$ as required.
\item This follows immediately by substituting $-2n$ into the functional equation and use
$\sin{(-\pi n )}=0$.
\end{enumerate}
\item Expand $e^{jt}$ as $\sum_{k=0}^\infty \frac{(jt)^k}{k!}$ and then swap the summation as the series is absolutely convergent, and the result follows by definition of $S_k(m)$. Now
    $$\sum_{j=0}^{m-1}e^{jt}=\frac{e^{mt}-1}{e^t-1}=\frac{e^{mt}-1}{t}\frac{t}{e^t-1}$$
    Use Lemma 6.24, and expand $e^{mt}$, we have
    $$\left(\sum_{r=0}^\infty \frac{m^rt^{r-1}}{r!}\right)\left(\sum_{n=0}\frac{B_n t^n}{n!}\right)$$
    Use the previous part and compare the coefficients (if $r-1+n=k$), we have
    $$\sum_{r+n=k+1} \frac{m^r B_n}{r! n!}= \frac{S_k(m)}{k!}$$
    and hence $$(k+1)S_k(m)=\sum_{j=0}^k \binom{k+1}{j}B_jm^{k+1-j}$$
    The last part follows immediately by applying the above formulae and we have
    $$4S_3(m)=m^4-2m^3+m^2$$
\item \begin{enumerate}
      \item[(i)] Use induction on $m$. It is true for $m=1$.
      Suppose true for $m$, that $\frac{f^m}{m!}$ is integral, then notice that
      $$\frac{f^{m+1}}{(m+1)!}=\int_0^x \frac{f^{m}(t)}{m!}f'(t) dt$$
      We have $f'(t)$ integral and by assumption we have $\frac{f^m}{m!} f'(t)$ integral, and hence
      $$\frac{f^{m+1}}{(m+1)!}=\int_0^x \frac{f^{m}(t)}{m!}f'(t) dt$$ is integral.
      \item[(ii)] We have
      $$(e^t-1)^{m}=\sum_{k=0}^{m}(-1)^{m-1-k}e^{kt}\binom{m}{k}$$
      Then expand $e^kt$ and so we have
      $$(e^t-1)^{m}=\sum_{n=0}^\infty \left(\sum_{k=0}^{m} (-1)^{m-k}\binom{m}{k}k^n\right)\frac{t^n}{n!}$$
      Hence,
      $$(e^t-1)^3=\sum_{n=0}^\infty \left(\sum_{k=0}^3 (-1)^{3-k}\binom{3}{k}{k^n}\right)\frac{t^n}{n!}$$
      and each coefficient for $\frac{t^n}{n!}$ is
      \begin{equation*}
\sum_{k=0}^3 (-1)^{3-k}\binom{3}{k}{k^n}=0= \left\{
\begin{array}{ll}
0 & \text{if } n=0\\
3-3\cdot 2^n+3^n & \text{otherwise } \\
\end{array} \right.
     \end{equation*}
     Therefore, when $n \ge 2$, $2^n \equiv 0$ (mod $4$) and so
     $$(e^t-1)^3 \equiv \sum_{n=2}^\infty \frac{(3^n-1)t^n}{n!}~(\text{mod } 4)$$
     Now
      \begin{equation*}
3^n-1 \equiv \left\{
\begin{array}{ll}
0~(\text{mod } 4)& \text{if } n \text{ is even}\\
2~(\text{mod } 4) & \text{if } n \text{ is odd} \\
\end{array} \right.
     \end{equation*}
     So the result follows.

     For the first and last part, we use
     $$(x-1)^m= \sum_{j=0}^m (-1)^{m-j}\binom{m}{j}x^j$$
     Differentiate, and put $x=1$ we have
     $$0=\sum_{j=1}^m (-1)^{m-j}\binom{m}{j} j=\sum_{j=0}^m (-1)^{m-j} \binom{m}{j} j$$
     Further, if we differentiate the above twice, and put $x=1$, we have
     $$0=\sum_{j=2}^m (-1)^{m-j} \binom{m}{j} j(j-1)=\sum_{j=0}^m (-1)^{m-j} \binom{m}{j} j(j-1)$$
     and so we have
     $$0=\sum_{j=0}^m (-1)^{m-j} \binom{m}{j} j^2$$
     Continue this we have
     $$0=\sum_{j=0}^m (-1)^{m-j} \binom{m}{j} j^{n} \text{ for } 0 \le n \le m-1$$
     For $m$, we differentiate the above $m$ times and so we have
     $$m!=\sum_{j=0}^m (-1)^{m-j} \binom{m}{j} j^m$$

     For the first part, replace $m$ by $m-1$ above, and so, as $m$ is composite, so $(m-1)! \equiv 0$
     (mod $m$). Then we have
     $$\sum_{j=0}^{m-1} (-1)^{m-1-j} \binom{m-1}{j} j^n \equiv 0~(\text{mod } m)$$
     for $0 \le n \le m$. If $n>m$, then $n>\phi(m)$ and so we write
     $$n=k\phi(m)+r, 0 \le r <\phi(m) <m$$
     and so
     $$j^{n} \equiv j^r~(\text{mod } m)$$
     and therefore we conclude that
     $$\sum_{j=0}^{m-1} (-1)^{m-1-j} \binom{m-1}{j} j^n \equiv 0~(\text{mod } m)$$
     for all $n$.

     For the last part, we have $(p-1)! \equiv -1$ (mod $p$) and so
     the expression is $0$ mod $p$ if $0 \le n \le p-2$ and is
     $-1$ mod $p$ if $n=p-1$. Now as $\phi(p)=p-1$ and as $(\mathbb{Z}/p\mathbb{Z})^*$ so the period is
     $p-1$ and so the result follows.
     \item[(iii)] We have $\frac{t}{e^t-1}=\frac{\log{(1+y)}}{y}$ where $y=e^t-1$.
     Use the expansion of $\log{{1+t}}=\sum_{n=1}^\infty (-1)^{n-1}\frac{t^n}{n}$ and so we have
     $$\frac{\log{(1+y)}}{y}=\sum_{n=1}^\infty (-1)^{n-1}\frac{y^{n-1}}{n}
     =\sum_{n=0}^\infty (-1)^{n}\frac{y^{n}}{n+1}$$
     Now use part (ii) and
     $$(e^t-1)^m \equiv \sum_{n=0}^\infty \frac{b_n t^n}{n!}~(\text{mod }m+1)$$
     means that
     $$\frac{(e^t-1)^m}{m+1}-\frac{1}{m+1}\sum_{n=0}^\infty \frac{b_n t^n}{n!}=
     \sum_{n=0}^\infty \frac{c_n t^n}{n!}$$ with $c_n$ integers.
     Therefore,
     $$\frac{t}{e^t-1}=\sum_{k=0}^\infty \frac{a_k t^k}{k!}
     -\frac{1}{2}\sum_{k=1}^\infty \frac{t^k}{k!}-\frac{2}{4}\sum_{k=1}^\infty \frac{t^{2k+1}}{(2k+1)!}
     -\sum_{p>2}\frac{1}{p}\sum_{k=1}^\infty \frac{t^{kp-k}}{(kp-k)!}$$
     By Lemma 6.24, we have
     $$\frac{t}{e^t-1}=\sum_{n=0}^\infty \frac{B_n t^n}{n!}$$
     and so compare the coefficient of $t^2n$ on both sides for each $n$, then consider $k(p-1)=2n$,
     and so $p-1 \big| 2n$. Therefore, we have
     $$B_n+\frac{1}{2}+\sum_{p>2,p-1 \big| 2n}\frac{1}{p}$$ is an integer, in other words
     $$B_n+\sum_{p-1 \big|2n}\frac{1}{p}$$ is an integer.
     \item[(iv)] This is clear, as $2-1 \big|2n+2$ and $2=3-1 \big|2n+2$.
      \end{enumerate}
\end{enumerate}
\subsection{Exercises 7}
\begin{enumerate}
\item Suppose $1+i |(a+bi)(c+di)$ for some $a,b,c,d \in \mathbb{Z}$ and so
$(1+i)(x+iy)=(a+bi)(c+di)$ for some $x,y$. We take the modulus on both sides, so that
$$2(x^2+y^2)=(a^2+b^2)(c^2+d^2)$$
and so at least one of $a^2+b^2$ and $c^2+d^2$ is even, and WLOG, say $a^2+b^2$. If $a$ is even, so is $b$ and so we have $a=2r,b=2s$ for some $r$ and $s$ and
$$a+bi=2(r+si)=(1+i)((r+s)+(-r+s)i)$$
If $a$ and $b$ are both odd, then $a=2r+1,b=2s+1$ for some $r$ and $s$ and
$$a+bi=2(r+si)+(1+i)=(1+i)((r+s+1)+(-r+s)i)$$ Hence $1+i |a+bi$.
\item Suppose $2=xy$ in $D=\mathbb{Z}[\sqrt{-5}]$. We may define the norm of an element $a+b\sqrt{-5}$ in $D$ to be $|a+b\sqrt{-5}|=a^2+5b^2$. Then if $u$ is a unit in $D$, then $u u^{-1}=1$ and $1$ has norm $1$. It is clear that the norm is multiplicative and so the norm of $u$ must be $1$ as it is always a positive integer and divides $1$. Now as the norm of $2$ is $4$, so
    $$|x| |y|=|2|=4$$
    It is also easy to check that we have no element having norm $2$, therefore, we must have that
    either $|x|$ or $|y|$ is $1$, WLOG, say $x$. Then the only elements having norm $1$ is $\pm 1$ and so $2$ is irreducible as $\pm 1$ are units.

    To show $2$ is not prime, we have $2|6=(1+\sqrt{-5})(1-\sqrt{-5})$ but $2$ divides neither of them, because if so, we have the norm of $2$ divides the norm of $1 \pm \sqrt{-5}$ but clearly $4$ does not divide $6$.\\
\item Suppose it is principal, then there exists $a$ such that
    $$\langle a \rangle=\langle 2,1+\sqrt{-7} \rangle$$ and then $a|2,1+\sqrt{-7}$.
    We use a similar method as above: define the norm of $a+b\sqrt{-7}$ to be
    $|a+b\sqrt{-7}|=a^2+7b^2$. Then if $a|2,1+\sqrt{-7}$, we have the norm of $a$ divides the norm of $2$ and $1+\sqrt{-7}$ and so the norm of $a$ divides $4$. Then as we have no element with norm $2$ so the norm of $a$ is $4$ or $1$. If $|a|=4$, then $a = \pm 2$ but clearly, $2 \nmid 1+\sqrt{-7}$. So $|a|=1$ and so $a=\pm 1$ which is a unit and then $\langle a \rangle=D$, which is a contradiction because clearly, $3$ does not lie in $\langle 2,1+\sqrt{-7} \rangle$.\\
\item Take any element in $AB$, say $a_1b_1+\ldots a_r b_r$ for some $r$ and $a_i \in A$, $b_i \in B$. But clearly, as $A$ is an ideal, so $a_i b_i \in A$ and so the sum of them still lies in $A$. Similarly, it also lies in $B$ and so it lies in $A \cap B$. Hence $AB \subseteq A \cap B$.\\
\item Take any element in $(A \cap B)(A+B)$, which can be written as
$$x_1 (y_1+z_1)+ \ldots x_r(y_r+z_r) \text{ where } x_i \in A,B, y_i \in A \text{ and } z_i \in B$$
    Now we can expand the bracket and notice that $x_i y_1\in A$ and $1$ is always in $B$
    so that $$x_1y_1+\ldots x_r y_r \in AB$$ Also, $z_i \in B$ and so
$$x_1 z_1+\ldots x_r z_r \in AB$$ Therefore,
$$x_1 (y_1+z_1)+ \ldots x_r(y_r+z_r) \in AB$$ because $AB$ is an ideal.\\
\item As $m <0$ so we define the norm of $a+b\sqrt{m}$ to be $a^2-b^2m$. Suppose $p$ is reducible and as $p$ is prime, the norm of $p$ is $p^2$. So $p$ is reducible only if we have some element with norm $p$. But we shall prove this is impossible. Let $a^2-b^2m=p$ and since $m \le -(p+1)$, we have $p=a^2-b^2m \ge a^2+(p+1)b^2$ and so
    $$0 \ge a^2+b^2+p(b^2-1)$$ which is impossible. Therefore, it is irreducible.\\
\item The set $B=\{x \in D: x=by+w \text{ for some } y \in D \text{ and some w } \in I\}$
    Let $x_1,x_2 \in B$ and WLOG, we assume that $x_1=by_1+w_1,x_2=by_2+w_2$, the
    $$x_1+x_2=b(y_1+y_2)+(w_1+w_2)$$ which still lies in $B$ as $I$ is an ideal.
    Similarly, let $x \in B$ and $d \in D$, say $x=by+w$. Then $xd=b(yd)+wd$ and $wd \in I$, $yd \in D$. So $xd \in B$. Hence it is an ideal.
    Finally, it is easy to check that $I \subset B$ because $0 \in D$ and so for each $i \in I$ we have
    $$i=b \cdot 0 +i \in B$$
\item This is an immediate application of Theorem 7.49. Having set $A_2 \cdots A_k=B$, we have
$$A_1 B \subseteq P \Rightarrow A_1 \subseteq P \text{ or } B \subseteq P$$
    because $P$ is prime. Repeat the above and so the result follows.\\
\item It suffices to show that every non-zero element has an multiplicative inverse. Let $a \in D$ and $a \neq 0$. Define $f_a: D \rightarrow D$ with $f_a(x)=ax$ for all $x \in D$. This is clearly an ring homomorphism and is injective. But $D$ is finite and so it is bijective. Therefore, we have some $x$ such that $ax=1$.\\
\item Let $K=\langle x_i y_j: 1 \le i \le m, 1 \le j \le n \rangle$. It is clear that $K$ is an ideal and
$K \subseteq IJ$. Now let $\alpha \in IJ$ then $x=\sum r_i s_j$, $r_i \in I$, $s_j \in J$. Then each $r_i$ can be written as a linear combination of $x_i$ and each $s_j$ can be written as a linear combination of $s_j$. Thus we may recollect the coefficient of each $x_iy_j$ and so it is a linear combination of $x_iy_j$ and so $IJ \subseteq K$. Hence $K=IJ$.
\end{enumerate}
\subsection{Exercises 8}
\begin{enumerate}
\item Define the norm of $a+b\sqrt{-2}$ to be $a^2+2b^2$. Take $\phi$ to be the norm function. It is clear that $\phi$ is multiplicative and so for any non-zero elements $\alpha,\beta$, we have
    $\phi(\alpha \beta) \ge \phi(\alpha)$. Now for any $a+b\sqrt{-2}$ and $c+d\sqrt{-2}$,
\item
\item \begin{enumerate}
      \item[(i)] Let $\alpha=a+b\sqrt{m},\beta=c+d\sqrt{m}$, then $\alpha \beta=(ac+bdm)+(ad+bc)\sqrt{m}$. We have
      $$\phi(\alpha \beta)=|(ac+bdm)^2-m(ad+bc)^2|=|a^2-b^2m||c^2-d^2m|$$
      \item[(ii)] Suppose $\alpha=0$, then by definition $\phi(\alpha)=0$. Conversely, if $\alpha=a+b\sqrt{m}$ and $\phi(\alpha)=0$, then we have $a^2=mb^2$. But we assumed that
          $m$ is square free so that $a=b=0$.\\
      \item[(iii)] Let $\beta \neq 0$, then by (i)
      $$\phi(\alpha \beta) =\phi(\alpha) \phi(\beta) \ge \phi(\alpha)$$ because
      $\phi(\beta)$ is integer and by (ii) is not zero.
      \end{enumerate}
\item \begin{enumerate}
      \item[(i)] Let $\alpha=a+b(\frac{1+\sqrt{m}}{2})$ and so
      $$\phi(\alpha)=\left|\left(a+\frac{b}{2}\right)^2 -m\frac{b^2}{4}\right|$$
      But $\frac{b^2}{4}(1-\frac{m}{4})$ is an integer as $m \equiv 1$ (mod $4$). So it is clear that
      the range of $\phi$ is non-negative integer. Also $0 \in D$ and $\phi(0)=0$.\\
      \item[(ii)] Notice that $\mathbb{Z}[\frac{1+\sqrt{m}}{2}]$ is a subdomain of $\mathbb{Q}[\sqrt{m}]$ and so by the previous question, the result follows immediately.
      \end{enumerate}
\item Suppose it is an Euclidean domain with respect to $\phi$, then for any $x,y \in \mathbb{Q}$, let
      $x+y\sqrt{m}=\frac{r+s\sqrt{m}}{t}$, where $(r,s,t)=1$. Then we have some $\alpha,\beta$ such that
      $$r+s\sqrt{m}=t\alpha+\beta \text{ where } \phi(\beta) < \phi(t)$$
      Hence, writing $\alpha=a+b\sqrt{m}$, we have
      $$\phi(x+y\sqrt{m}-(a+b\sqrt{m}))=\phi\left(\frac{\beta}{t}\right)$$
      and by multiplicative property, we have $\phi(\frac{\beta}{t})<1$.

      Conversely, suppose for each $x,y \in \mathbb{Q}$, we have $a,b \in \mathbb{Z}$ such that the condition holds. Then for each pairs $\alpha=r+s\sqrt{m}$ and $\beta=u+v\sqrt{m}$ with $\beta \neq0$, let
      $$x+y\sqrt{m}=\frac{\alpha}{\beta}$$
      and so we may simplify the above:
      $$x=\frac{ru-msv}{u^2-mv^2},y=\frac{su-rv}{u^2-mv^2}$$
      We have $a,b \in \mathbb{Z}$ such that
      $$\phi(x+y\sqrt{m}-(a+b\sqrt{m}))<1$$
      Let $c,d \in \mathbb{Z}$ such that
      $$\alpha=(a+b\sqrt{m})\beta+(c+d\sqrt{m})$$
      and we may write $c+d\sqrt{m}=\beta(x+y\sqrt{m}-(a+b\sqrt{m})$.
      $$\phi(c+d\sqrt{m})=\phi(\beta)\left(\phi\left(x+y\sqrt{m}-(a+b\sqrt{m})\right)\right)<\phi(\beta)$$
      It is also clear that $\phi(\alpha \beta) \le \phi(\alpha)$ by multiplicative property.\\
\item By question $1$, $\mathbb{Z}[\sqrt{-2}]$ is a Euclidean domain. The case when $m=-1$ can be shown in a similar manner. Conversely, if it is a Euclidean domain, by question $5$, we pick $x,y=\frac{1}{2}$. Then we should have some $a,b \in \mathbb{Z}$ such that
    $$\phi\left(\frac{1}{2}+\frac{1}{2}\sqrt{m}-(a+b\sqrt{m})\right)<1$$
    then we have
    $$\left(\frac{1}{2}-a\right)^2-m\left(\frac{1}{2}-b\right)^2 <1$$
    For any integer $a$, we have $(\frac{1}{2}-a)^2 \ge \frac{1}{4}$, and as $-m>0$, we have
    $$\frac{1}{4}-\frac{1}{4}m \le \left(\frac{1}{2}-a\right)^2-m\left(\frac{1}{2}-b\right)^2 <1$$
    and so $m>-3$ we have $m=-1,-2$.\\
\item We can always write
    $$x+y\sqrt{m}=\frac{r+s\left(\frac{1+\sqrt{m}}{2}\right)}{\frac{1+\sqrt{m}}{2}t}$$
    for suitable $s,r$ and $t \in \mathbb{Z}$. The one direction of the proof follows immediately in exactly the same argument of question 6. The other half follows by writing
    $$\frac{r+s\left(\frac{1+\sqrt{m}}{2}\right)}{u+v\left(\frac{1+\sqrt{m}}{2}\right)}=x+y\sqrt{m}$$
    and define $c,d$ in a similar way as in question 6.

    Finally, pick $a=b=\frac{1}{2}$, and use the fact that $(\frac{1}{4}-x)^2 \ge \frac{1}{16}$, so we have $m >-15$ if we follow the same procedure of question 6. Use $m \equiv 1$ (mod $4$) and so the result follows.
\end{enumerate}
\subsection{Exercises 11}
\begin{enumerate}
\item \begin{enumerate}
      \item[(i)] Suppose we have some intermediate field $K'$, then by Tower Law we have
      $$[F:K'][K'K]=[F:K]$$ which is impossible as $[F:K]$ is a prime.\\
      \item[(ii)] It is clear that $K(\alpha^2) \subseteq K(\alpha)$. Now by Tower Law, we have
      $$[K(\alpha):K(\alpha^2)][K(\alpha^2):K]=[K(\alpha):K]$$
      Also, $[K(\alpha):K(\alpha^2)]$ is at most two because $\alpha$ satisfies $x^2-\alpha^2$ in $K(\alpha^2)[x]$ and so $[K(\alpha):K(\alpha^2)]=1$ or $2$. But
      $[K(\alpha):K]$ is odd and so the factor $[K(\alpha):K(\alpha^2)]$ must also be odd. Therefore,
      $[K(\alpha):K(\alpha^2)]=1$ and $K(\alpha)=K(\alpha^2$).\\
      \item[(iii)] Using Tower Law, we have
     $$[F:K(\alpha)][K(\alpha):K]=[F:K(\beta)][K(\beta):K]=[F:K]$$
     Then $m,n|[F:K]$ and since $(m,n)=1$ so $mn|[F:K]$.
     Now $[K(\alpha):K]=m$ so the minimal polynomial of $\alpha$ over $K$ has degree $m$, and as $K \subset K(\beta)$, so the minimal polynomial of $\alpha$ over $K(\beta)$ divides that over $K$. Hence
     $$[F:K(\beta)]=[K(\beta)(\alpha):K(\beta)] \le m$$
     and hence by Tower Law $[F:K] \le mn$. But $mn|[F:K]$ and so $[F:K]=mn$
      \end{enumerate}
\item WLOG, let $y \in F\backslash K$. Then we must have some $u,v \in K$ such that
     $y^2=uy+v$ because $[F:K]=2$. Then since the characteristic of $F$ is not $2$ so we may assume complete the square and so we have
     $$\alpha^2=v+\frac{u^2}{4} \in K$$ where $\alpha=y-\frac{u}{2}$. Thus, as $\alpha \not \in K$ and so
     $[K(\alpha):K]=2$ and $K(\alpha) \subseteq F$. So we have $F=K(\alpha)$.

     Now suppose the characteristic of the field is $2$. Then take any $y \in F\backslash K$, we still have some
     $u,v \in K$ such that $y^2=uy+v$. We have the following two cases:
     \begin{enumerate}
     \item[(i)] $u=0$: Then $y^2=v \in K$ and so $F=K(y)$ with $y^2 \in K$.\\
     \item[(ii)] $a \neq 0$ then $a^{-1}$ exists and so we have
     $$y^2 a^{-2}=ya^{-1}+ba^{-2}$$ But $ya^{-1}=-ya^{-1}$ because the characteristic is $2$. Hence
     $$(ya^{-1})^2+ya^{-1}=ba^{-2}$$ and let $\alpha=ya^{-1}$ then we have $\alpha^2+\alpha \in K$.
     $F=K(\alpha)$. (We cannot complete the square as $2=0$ has no inverse in the field)
     \end{enumerate}
\item Let $F=\mathbb{Q}(\sqrt{2},\sqrt{3},\sqrt{5}), L=\mathbb{Q}(\sqrt{2},\sqrt{3})$ and $K=\mathbb{Q}(\sqrt{2})$.
     Then it is easy to check that $[F:L]=[L:K]=[K:\mathbb{Q}]=2$ by considering the minimal polynomial in the corresponding field. Then use Tower Law.\\
\item It is clear that $\mathbb{Q}(\sqrt{2}+i) \subset \mathbb{Q}(\sqrt{2},i)$. Conversely, the inverse of $\sqrt{2}+i$ is $\frac{1}{3}(\sqrt{2}-i)$ and so $\sqrt{2}-i \in \mathbb{Q}(\sqrt{2}+i)$. Hence $\sqrt{2}$ and $i$ are both in $\mathbb{Q}(\sqrt{2}+i)$.\\
\item Suppose $P$ has a root in $F$, say $\alpha$. Then by Tower Law
     $$[F:K]=[F:K(\alpha)][K(\alpha):K]$$
     As $P$ is irreducible in $K$ and so $[K(\alpha):K]=d$. Then d divides $[F:K]$, which is a contradiction as
     $(d,[F:K])=1$.\\
\item Suppose $\alpha \in L$ and $[K(\alpha):K]$ is finite. Then by Proposition 11.13, any finite extension is algebraic, and in particular $\alpha$ is algebraic. Conversely, let $\alpha$ be algebraic then we have some non-trivial polynomial $P(x) \in K[x]$ such that $p(\alpha)=0$ with $\deg{P}=d$. Then $[K(\alpha):K] \le d$.

    Now take any $\alpha \in F$. Then there exist $a_0,\ldots,a_m \in L$ such that $\alpha$ is a root of
    $$P(x)=\sum_{i=0}^m a_i x^i \in L[x]$$
    Since $L/K$ is algebraic, and so by previous part that each $[K(a_i):K]$ is finite. Hence
    $K[(a_0,\ldots,a_m):K]$ is finite. Also, it is clear that
    $$[K(a_0,\ldots,a_m,\alpha):K(a_0,\ldots,a_m)] \le m$$
    Hence $[K(a_0,\ldots,a_m,\alpha):K]$ is finite, and hence algebraic by Proposition 11.13. Then $\alpha$ is algebraic over $K$. Since $\alpha$ is arbitrary, we have $F/K$ algebraic.\\
\item Let $u=\alpha+\beta,v=\alpha \beta$. Suppose $\alpha,\beta$ is algebraic, then $[K(\alpha,\beta):K]$ is finite, because $[K(\alpha):K],[K(\beta):K]$ are both finite. (because they satisfy some polynomials over $K$)
    Then $K(u),K(v) \subset K(\alpha,\beta)$ and so $[K(u):K],[K(v):K]$ are both finite and hence algebraic by Proposition 11.13.

    Conversely, suppose $u,v$ are algebraic over $K$. Observe that $\alpha,\beta$ are roots of
    $$x^2-ux+v \in K(u,v)[x]$$
    So $[K(u,v,\alpha):K(u,v)],[K(u,v,\beta):K(u,v)] \le 2$
    But $[K(u,v):K]$ is algebraic and so
    $[K(u,v,\alpha):K],[K(u,v,\beta):K]$ are both finite.
    Hence $\alpha,\beta$ are algebraic by Proposition 11.13.\\
\item Since the statement is symmetric so we only need to check one direction holds. Suppose $\alpha$ is algebraic over $K(\beta)$. Then there exists a non-trivial $P(x) \in K(\beta)[x]$ such that $P(\alpha)=0$. WLOG, let
    $$P(x)=\sum_{i=0}^m a_i x^i$$ Each $a_i \in K(\beta)$ so we have
    $$a_i=\sum_{j=0}^{m_i} b_{ij} \beta^j$$
    Thus we have
    $$\sum_{i=0}^m \sum_{j=0}^{m_i} b_{ij} \beta^j \alpha^i =0$$ which can be viewed as a polynomial in
    $K(\alpha)[x]$ which vanishes at $\beta$. We need to check this polynomial is not a zero polynomial. Indeed, if it is zero, then for each $j$, the coefficient for $\beta^j$ is zero, which is
    $$\sum_{i=1}^m \alpha^i b_{ij} \text{(if we don't have $b_{ij}$ for some $i,j$ then let $b_{ij}=0$)}$$
    But $\alpha$ is transcendental and so this is possible only if $b_{ij}=0$ for all $i,j$. But then the polynomial $P(x)$ itself is a zero polynomial, which is a contradiction. Hence $\beta$ is algebraic over $K(\alpha)$.\\
\item Let $P$ be the minimal polynomial of $\alpha$ over $K'$, where $K \subset K' \subset K(\alpha)$. WLOG, let $P(x)=x^n+a_{n-1}x^{n-1}+\ldots+a_1 x+a_0$. Let $K''=K(a_0,\ldots,a_{n-1})$.
    Then $[K''(\alpha):K'']=n$. Also $[K'(\alpha):K']=n$ by definition of $P$. But $a_i \in K'$, so $K'' \subseteq K' \subset K(\alpha)$. Hence
    $$[K(\alpha):K'']=[K(\alpha):K'][K':K'']$$
    Also as $K' \subset K(\alpha)$, so $K'(\alpha) \subset K(\alpha)$, but $K(\alpha) \subset K'(\alpha)$ as
    $K \subset K'$. So we have $K(x)=K'(x)$. Similarly, $K(x)=K''(x)$ as $K'' \subset K'$.
    Then the above equation becomes:
    $$[K''(\alpha):K'']=[K'(\alpha):K'][K':K'']$$ But $n=[K''(\alpha):K]=[K'(\alpha):K']$ and so $[K':K'']$, which implies that $K'=K''$, and so $K'$ is generated by $a_0,\ldots,a_{n-1}$. Let $Q$ be the minimal polynomial of $\alpha$ over $K$ and so $Q$ is independent of $K'$. Then $P|Q$ and so there are only finitely many possible sets of values of $\{a_0,\ldots,a_{n-1}\}$ and hence finitely many $K'$.

    If $K$ is finite, so is $F$ and so $F^*$ is generated by some element say $\alpha$ and so $F=K(\alpha)$.
    So we may assume now $K$ is infinite. One way to do this is a Linear Algebra approach. Suppose $F \neq K(\alpha)$ for all $\alpha \in F$. But we only have finitely intermediate field, so only finitely many
    $K(\alpha)$ for different $\alpha \in F$. But
    $$F=\bigcup_{\alpha \in F}K(\alpha)$$ and so the above union must be a finite union. Now $K$ is infinite and so if $F$. This is impossible as an infinite vector space cannot be written as a finite union of proper subspaces.

    The second approach is purely algebraic. Consider for two elements $\alpha,\beta \in F$, the field
    $K(\alpha+a\beta)$ for some $a \in K$. We have $K \subset K(\alpha+a\beta) \subset F$. As we have only finitely many intermediate fields, and since $K$ is infinite, so there exist $a \neq b$ such that
    $$K(\alpha+a\beta)=K(\alpha+b\beta)$$
    Also,
    $$\beta=\frac{(\alpha+a\beta)-(\alpha+b\beta)}{a-b}, \alpha=(\alpha+a\beta)-a\beta$$
    Since $\alpha+b\beta \in K(\alpha+a\beta)$, so $\beta \in K(\alpha+a\beta)$ and hence
    $\alpha \in K(\alpha+a\beta)$. Therefore, $K(\alpha,\beta)=K(\alpha+a\beta)$ for some $a$.
    Now $F/K$ is finite, so there exist $\alpha_1,\ldots,\alpha_n$ such that
    $F=K(\alpha_1,\ldots,\alpha_n)$. Apply the above $n-1$ times
    (i.e $K(\alpha_1,\alpha_2,\ldots,\alpha_n)=K(\alpha_1+a\alpha_2,\ldots,\alpha_n)$ etc)
    so we conclude that
    $F=K(\alpha)$ for some $\alpha$
\item Use induction on $n$. It is clear that if $n=1$, then $P$ is linear and so the splitting field $F=K$.
    Now suppose it is true for $n-1, n \ge 2$. Then let $\deg{P}=n$. Let $x-\alpha$ be a linear factor in the splitting field of $P$, say $F$. Then $[K(\alpha),K] \le n$. Let $P(x)=(x-\alpha)Q(x)$. By induction, the splitting field of $Q(x)$, say $F'$ has $[F':K] \le (n-1)!$. Clearly, $F$ is contained in the field
    $F'(\alpha)$ because $P$ splits in $F'(\alpha)$. Hence $[F:K] \le n!$.\\
\item If $x^m-\alpha,x^n-\alpha$ are both reducible, then $x^{mn}-\alpha=(x^{m})^n-\alpha$ is also reducible.

     Conversely, suppose $x^m-\alpha,x^n-\alpha$ are both irreducible, then let $u$ be any root of $x^{mn}-\alpha$.
     Then $u^n$ is a root of $x^m-\alpha$ and $u^m$ is a root of $x^n-\alpha$. Since $x^m-\alpha,x^n-\alpha$ both irreducible, we have $$[K(u^m):K]=n \text{ and } [K(u^n):K]=m$$
     Also $$[K(u):K]=[K(u):K(u^m)][K(u^m):K]=[K(u):K(u^n)][K(u^n):K]$$
     and so $mn$ divides $[K(u):K]$ because $(m,n)=1$.
     But $u$ is a root of $x^{mn}-\alpha$ so $[K(u):K] = mn$. Hence $x^{mn}-\alpha$ is the minimal polynomial of $u$ over $K$ and so it is irreducible.\\
\item The minimal polynomial is
     $$P(x)=x^4-2(a+b)x^2+(a-b)^2$$
     Now, using Legendre symbol, either one of  $(\frac{a}{p}),(\frac{b}{p})$ is $1$, or both of them are $-1$, which gives $(\frac{ab}{p})=1$ so one of them is a square for any prime $p$. Then denote
     $\alpha,\beta,\gamma$ by the square root of $a,b,ab$ mod $p$ respectively. (We know at least one of them exists)
     Then we can factorise $P(x)$ mod $p$ into one of the following forms:
     $$(x^2-2\alpha x+a-b)(x^2+2\alpha x+ a-b) \text{ if } a \text{ is a square}$$
     $$(x^2-2\beta x +b-a)(x^2+2\beta x +b-a) \text{ if } b \text{ is a square}$$
     $$(x^2-(a+b+2 \gamma))(x^2-(a+b-2\gamma)) \text{ if } c \text{ is a square}$$
\end{enumerate}
\subsection{Exercises 12}
\begin{enumerate}
\item Consider $\frac{1+\theta}{3} \in \mathbb{Q}[\theta]$, the field of quotient of $\mathbb{Z}[\theta]$. We have
      $$\theta^3+6\theta+34=0$$
      and so
      $$\frac{\theta^3}{27}+\frac{2\theta}{9}+\frac{34}{27}=0$$
      But this is
      $$y^3-y^2+y+1=0$$ where $y=\frac{1+\theta}{3}$ and so it is algebraic but it is not in $\mathbb{Z}[\theta]$.\\
\item $\sqrt{10}=\sqrt{2}\sqrt{5} \in \mathbb{Z}[\sqrt{2},\sqrt{5}]$, and so $y=\frac{\sqrt{2}+\sqrt{10}}{2} \in F$, the field of quotient of $\mathbb{Z}[\sqrt{2},\sqrt{5}]$. $y^2=3+\sqrt{5}$ and so
    $$y^4-6y^2+4=0$$ Hence $y$ is an algebraic integer but not in $\mathbb{Z}[\sqrt{2},\sqrt{10}]$

    Now if $a$ is square free and odd, we consider the element $y=\frac{\sqrt{2}+\sqrt{2a}}{2}$ in the field of quotient of $\mathbb{Z}[\sqrt{2},\sqrt{a}]$. Then $y^2=\frac{1+a}{2} + \sqrt{a}$, where $\frac{1+a}{2}$ is an integer because $a$ is odd. Thus,
    $$y^4-(1+a)y^2-a=0$$ and so $\mathbb{Z}[\sqrt{2},\sqrt{a}]$ is not integrally closed.

    If $a$ is square free and even, then observe that
    $$\mathbb{Z}[\sqrt{2},\sqrt{a}]=\mathbb{Z}\left[\sqrt{2},\sqrt{\frac{a}{2}}\right]$$
    where $\frac{a}{2}$ is odd as $a$ is square free, and so use the previous part.\\
\item Let $\alpha=a+b\sqrt{5}, a,b \in \mathbb{Q}$. If $b=0$ then $a \in \mathbb{Z}$. If $b \neq 0$, the minimal polynomial is $x^2-2ax+a^2-5b^2$. So we need
    $2a,a^2-5b^2 \in \mathbb{Z}$. If $a \in \mathbb{Z}$, then $b \in \mathbb{Z}$.
    If $a$ has denominator $2$, i.e. $a=\frac{c}{2}$, where $c$ is odd. Then $b$ has denominator at most $2$ and $b \not \in \mathbb{Z}$. So we may assume $b=\frac{d}{2}$ for some $d$ odd. Thus,
    $$a^2-5b^2=\frac{c^2-5d^2}{4} \in \mathbb{Z}$$ because
    $c^2,5d^2 \equiv 1$ (mod $4$). Hence, combine the case $a \in \mathbb{Z}$, and the case when $a=\frac{c}{2}$, we conclude that $\mathcal{O}_K=\mathbb{Z}[\frac{1+\sqrt{5}}{2}]$.\\
\item Let $\alpha=a+b\sqrt{3}, a,b \in \mathbb{Q}$. If $b=0$ then $a \in \mathbb{Z}$. If $b \neq 0$, the minimal polynomial is $x^2-2ax+a^2-3b^2$. Thus $2a,a^2-3b^2 \in \mathbb{Z}$. If $a \in \mathbb{Z}$, so is $b$.
    If $a$ has denominator $2$, say $a=\frac{c}{2}$, then $b$ has denominator at most $2$ and $b \not \in \mathbb{Z}$. If $b=\frac{d}{2}$ for some $d$ odd. Then we must have
    $$\frac{c^2-3d^2}{4} \in \mathbb{Z}$$
    But this is impossible, as $c^2-3d^2 \equiv 2$ (mod $4$). Hence we have $a,b \in \mathbb{Z}$ and so
    $\mathcal{O}_K=\mathbb{Z}[\sqrt{3}]$.\\
\item Pick any $0 \neq x \in I$, so $x \in B$ and there exist $a_i \in A$ and a smallest $n \in \mathbb{N}$ such that
    $$x^n+a_{n-1}x^{n-1}+\ldots+a_1 x+a_0=0$$
    and we may assume $a_0 \neq 0$ otherwise we can divide the equation by $x$ and get a lower degree.
    As $x \in I$, each $a_i x^i \in I$ and $0 \in I$, we have $a_0 \in I$. Hence $a_0 \in I \cap A$, and so $I \cap A$ is non-zero.\\
\item Let $x=\frac{10^{\frac{2}{3}}-1}{\sqrt{-3}}, y=10^{\frac{1}{3}}, z= \sqrt{-3}$. Then we have
    $$zx+1=y^2 \text{ and so } zx^3+3x^2-zx+33=0$$
    Rearrange, we have
    $$z(x^3-x)=(-3x^2-33) \text{ and so } x^6+x^4+67x^2+363=0$$ and so it is an algebraic integer.
\item Consider the element $x=\frac{1+\sqrt{m}}{2}$. Then $x^2-x+\frac{1-m}{4}=0$ and $\frac{1-m}{4}$ is an integer because $m \equiv 1$ (mod $4$).\\
\item $|\text{Hom}_K(F,F)|=n=[F:K]$, so we have $F/K$ separable and so by Theorem 12.28,
    $$T_{F/K}\left(\sigma(x)-x\right)=\sum_{i=1}^n \sigma^i\left(\sigma(x)-x \right)=0$$

    Then we need to prove the converse. Let $Tr_{F/K}(y)=0$ for some $y \in F$. By Theorem 12.28, $Tr_{F/K}$ is not a zero map and hence there exists $z \in F$ such that $T_{F/K}(z) \neq 0$.
    Define $x \in F$ by
    \begin{eqnarray*}
    -x T_{F/K}(z)&=&yz+\left(y+\sigma(y)\right)\sigma(z)+\ldots+\\
    &~&\left(y+\sigma(y)+\ldots+\sigma^{n-1}(y)\right)
    \sigma^{n-1}(z)\\
    &=&\sum_{i=0}^{n-1} \sum_{j=0}^i\sigma^{j}(y)\sigma^i(z)
    \end{eqnarray*}
    Then
    $$-\left(\sigma(x)-x\right)T_{F/K}(z)=-y\sum_{i=1}^{n-1} \sigma^{i}(z) +z\sum_{i=1}^{n-1}\sigma^i(y)
    =-yT_{F/K}(z)$$
    using $\sum_{i=0}^{n-1} \sigma^i (y)=0$. Thus, we have $y=\sigma(x)-x$.
\item It is easy to check that any embedding $\sigma$ is a homomorphism (in fact isomorphism) and hence the result follows. Suppose $K=\mathbb{Q}(\theta)$ for some $\theta$ and $\sigma(\theta)=\alpha$ for some $\alpha$. Then:
    \begin{enumerate}
    \item[(i)] Let
    $$a=a_0+a_1\theta+\ldots+a_{n-1}\theta^{n-1}, b=b_0+b_1\theta+\ldots+b_{n-1}\theta^{n-1} \in K$$
    with $a_i, b_i \in \mathbb{Q}$. Then
    $$\sigma(a+b)=a_0+b_0+(a_1+b_1)\alpha+\ldots+(a_{n-1}+b_{n-1})\alpha^{n-1}$$
    \item[(ii)] Let
    $$f(x)=a_0+a_1x+\ldots+a_{n-1}x^{n-1}, g(x)=b_0+b_1x+\ldots+b_{n-1}x^{n-1}$$
    such that $f(\theta)=a,g(\theta)=b$. Let $P(x)=\min_{\mathbb{Q},\theta}(x)$.
    Then there exist $q(x),r(x)$ such that
    $$f(x)g(x)=P(x)q(x)+r(x) \text{ with } \deg{r} < \deg{P}$$
    Let $x=\theta$, so we have $r(\theta)=ab$.
    Also
    $$P(\alpha)=P(\sigma(\theta))=\sigma(P(\theta))=0$$
    so
    \be
    \sigma(ab) = \sigma(r(\theta))=r(\sigma(\theta)=r(\alpha) = r(\alpha)+P(\alpha)q(\alpha)  = f(\alpha)g(\alpha) = \sigma(a)\sigma(b)
    \ee
    \end{enumerate}
\end{enumerate}

\subsection{Exercises 13}

\begin{enumerate}
\item Suppose $AB=D$, then $A|D$ and so we have $D \subseteq A$, which is a contradiction.
\item Let $\gamma I \subseteq D$ and $\delta J \subseteq D$ since $I$ and $J$ are fractional ideals.
      Clearly $I+J$ and $IJ$ satisfy the first two properties of a fractional ideal, just as what we checked for integral ideal. For the third property, let $i \in I$ and $j \in J$, then
$$\gamma \delta (i+j)=\delta(\gamma i)+\gamma (\delta j) \in D$$
and so $\gamma \delta (I+J)\subseteq D$. Similarly we have $\gamma \delta IJ \subseteq D$
and so $I+J$ and $IJ$ are both fractional ideals.
\item Let $I$ be a fractional ideal of $D$. Then there exists $0 \neq \gamma \in D$ such that $\gamma I \subseteq D$ is an integral ideal. Since $D$ is a principal ideal domain, we have some $\delta$ such that
    $$\gamma I=\langle \delta \rangle$$
    So
    $$I=\{\alpha d :d \in D \} \text{ where } \alpha=\frac{\delta}{\gamma} \in K$$
\item In the proof of Lemma 13.1, we see that $\mathcal{O}_K/ \langle \alpha \rangle$ has at most $\alpha^n$ elements because $\mathbb{Z}/\alpha \mathbb{Z}=\alpha$. Hence, for each integral ideal $I$ containing $\alpha$ (and hence $\langle \alpha \rangle$), it corresponds to the element $I/\langle \alpha \rangle$ in
    $\mathcal{O}_K/\langle \alpha \rangle$. Hence we have at most $\alpha^n$ such $I$.
\item Consider $\alpha= x+y\sqrt{-5}$, $x,y \in \mathbb{Q}$ such that $\alpha \langle 2,1+\sqrt{-5} \rangle \subseteq \mathcal{O}_K$.
because $\mathcal{O}_K=\mathbb{Z}[\sqrt{-5}]$. Thus we have
$$2(x+y\sqrt{-5}) \subseteq \mathbb{Z}[\sqrt{-5}], (x+y\sqrt{-5})(1+\sqrt{-5}) \subseteq \mathbb{Z}[\sqrt{-5}]$$
and so
$$2x,2y,x-5y,x+y \in \mathbb{Z}$$
If $x \in \mathbb{Z}$, so is $y$. If $x=\frac{a}{2}$ for some $a$ odd, then $y=\frac{b}{2}$ for some $b$ odd and all the above hold. Therefore, we conclude that the inverse of $\langle 2, 1+\sqrt{-5} \rangle$ is
$$\mathbb{Z}+\frac{1+\sqrt{-5}}{2}\mathbb{Z}=\frac{1}{2} \langle 2,1+\sqrt{-5}\rangle$$
\item Let $D$ be a Dedekind domain. Suppose $D$ is a unique factorisation domain. Let $P$ be any prime ideal in $D$. Take $0 \neq x \in P$. As $P \neq D$, so $x$ is not a unit and so
    $$x=p^{e_1}_1 \cdots p^{e_k}_k$$
    where $e_i \ge 1$ and $p_i$ is irreducible, $i=1,\ldots,k$. Since $P$ is a prime ideal, so
    $p_j \in P$ for some $j$ and in a unique factorisation domain, an irreducible element is prime. So $p_j$ is
     prime and so $\langle p_j \rangle$ is a prime ideal and hence maximal in a Dedekind domain.
    But $\langle p_j \rangle \subseteq P$. Thus $\langle p_j \rangle =P$ and so $P$ is principal. Finally,
     since each ideal can be factorised into prime ideals. Therefore, we conclude that every ideal is principal
    and hence $D$ is a principal ideal domain.
\item It is clear that the definition coincides with the definition of greatest common divisor and least common multiple in $\mathbb{Z}$. That is, $(I,J)$ is the common divisor of $I$ and $J$ such that if
    $K \big| I,J$ then $K\big|(I,J)$ because $K$ must be a product of $P_i$'s with exponent less than the
    minimum of $e_i$ and $f_i$. Similarly the least common multiple $\{I,J\}$ is an ideal such that
    $I,J \big| \{I,J\}$ and for any $L$ such that $I,J \big| L$, we have $\{I,J\} \big| L$. Note also that they
    are unique by the uniqueness of factorisation.

    Using the above observation, for $(I,J)$. Suppose $K \big| I,J$ then $K \supseteq I,J$ and hence
    $K \supseteq I+J$. Therefore, $K \big| I+J$ and so $(I,J)=I+J$.

    Similarly, let $I,J \big| L$ then $L \subseteq I,J$ and hence $L \subseteq I \cap J$. Therefore,
    $I \cap J \big|$ and so $\{I,J\}=I \cap J$.
\item If $J=K$ then clearly $IJ=JK$.

     Conversely, if $IJ=IK$. Then by uniqueness of the inverse of an ideal in $D$ (since $I$ is non-zero), we multiply by the inverse of $I$ and so we have $J=K$.
\item Let $p$ and $q$ be distinct prime numbers in $\mathbb{N}$. Use the notation in question 7, we have
    $$(\langle p \rangle, \langle q \rangle)=\langle p \rangle +\langle q \rangle$$
    But since $(p,q)=1$ there exists $x,y \in \mathbb{Z}$ such that
    $$px+qy=1$$
    Thus $$\langle p \rangle +\langle q \rangle=\langle 1 \rangle =\mathcal{O}_K$$
    Therefore, the prime ideal dividing $\langle p \rangle$ does not divide $\langle q \rangle$ and since there are infinitely many prime numbers in $\mathbb{N}$, so we conclude that there are infinitely many prime ideals in $\mathbb{O}_K$ because each $\langle p \rangle$ must be divisible by some prime ideals.
\item Define a ring homomorphism
    $$f: D/I \rightarrow D/P^{e_1}_1 \times \cdots d/P^{e_k}_k$$
    by
    $$f(a+I)=(a_1+P^{e_1}_1,\ldots,a_k+P^{e_k}_k) \text{ where } a_i \equiv a~(\text{mod } P^{e_i}_i)$$
    It is clearly a ring homomorphism since the congruence is additive and multiplicative. It is injective,
    because if $f(a+I)=0$, then
    $$a \equiv 0~(\text{mod } P^{e_i}_i) \text{ for all } i$$
    and so $a \in P^{e_i}_i$ for all $i$. Hence $P^{e_i}_i \big| \langle a \rangle$ and since $P_i$ are distinct, so we have $I \big| \langle a \rangle$ and so $a=0$.
    It is surjective, because there exists $\alpha$ such that
    $$\alpha \equiv \alpha_i~(\text{mod } P^{e_i}_i) \text{ for all } i$$
    by Theorem 13.28 (Chinese Remainder Theorem). Hence it is an isomorphism and so the result follows.\\
\item \begin{enumerate}
    \item[(i)] Each ideal in $\mathcal{O}_K/P^n$ corresponds to an ideal containing $P^n$. Thus let $I \supseteq P^n$, then $I \big|P^n$ and so $I=P^j,j=1,\ldots,n$. Hence the result follows.\\
    \item[(ii)] Use question 7, that
    $$\langle \alpha^i \rangle + P^n = (\langle \alpha^i \rangle, P^n)$$
    Since $\alpha \in P \backslash P^2$, so $\langle \alpha \rangle \subseteq P$ but $\not \subseteq P^2$.
    Hence
    $P \big| \langle \alpha \rangle, P^2 \big| \langle \alpha \rangle$.
    Therefore the greatest common divisor
    $$(\langle \alpha^i \rangle, P^n)=P^i \text{ for } i=1,2,\ldots,n$$
    \item[(iii)] Use (i) that any non-zero proper ideal in $\mathcal{O}_K/P^n$ has the form
    $P^j \mathcal{O}_K/P^n$. Then by (ii) we conclude that
    $$P^j \mathcal{O}_K/P^n =\langle \alpha^j+P^n \rangle \text{ for any } \alpha \in P\backslash P^2$$
    Therefore every ideal in $\mathcal{O}_K/P^n$ is generated by one element. Now use Chinese Remainder Theorem in question 10, for any ideal $I$ of $\mathcal{O}_K$, we have
    $$\mathcal{O}_K/I \cong \mathcal{O}_K/P^{e_1}_1 \times \cdots \times \mathcal{O}_K/P^{e_k}_k$$
    where $I=P^{e_1}_1 \cdots P^{e_k}_k$. Since each ideal of $\mathcal{O}_K/P^{e_j}_j$ can be generated by one element, so is the ideal of $\mathcal{O}_K/I$.\\
    \item[(iv)] Let $I$ be an integral ideal of $\mathcal{O}_K$ and let $0 \neq \alpha \in I$. Let
    $J=\langle \alpha \rangle$. Then $I/J \subseteq \mathcal{O}_K$ and so is generated by one element $\beta$, by (iii). Thus $I/J=\langle \beta +J \rangle$. Now for any $x \in I$, let $x \equiv y$ (mod $J$) and so
    $x=j+y$ for some $j \in J$. But $y \in \langle \beta \rangle$ and $j \in J=\langle \alpha \rangle$. Therefore,
    $$I=\langle \alpha, \beta \rangle$$
    Suppose more generally that $I$ is a fractional ideal. Then take $\gamma \in D$ such that
    $$\gamma I \subseteq D \text{ is an integral ideal}$$
    Thus, there exists $\alpha, \beta$ such that $\gamma I =\langle \alpha, \beta \rangle$ and hence
    $$I=\left\langle \frac{\alpha}{\gamma}, \frac{\beta}{\gamma} \right\rangle$$
    \end{enumerate}
\item \begin{enumerate}
    \item[(i)] $\alpha \in I \rightarrow \langle \alpha \rangle \subseteq I$ so $I \big| \langle \alpha \rangle$.
    So there exists an integral ideal $J$ such that $IJ=\langle \alpha \rangle$. Then apply Theorem 13.27.\\
    \item[(ii)] Now since $J$ is an integral ideal, we have, for $i=1,\ldots,k$, we have
    \begin{eqnarray*}
    ord_{P_i}(I)&=&\min{(ord_{P_i}(I),ord_{P_i}(IJ))}\\
    &=&\min{(ord_{P_i}(\beta),ord_{P_i}(IJ))}\\
    &=&\min{(ord_{P_i}(\langle \beta \rangle),ord_{P_i}(IJ))}\\
    &=&ord_{P_i}(\langle \beta \rangle + IJ)
    \end{eqnarray*}
    by using Proposition 13.22(ii). For $P \neq P_i,i=1,\ldots,k$, we have
    $$ord_{P}(I)=ord_{P}(IJ)=0$$
    and since $ord_{P}(\beta) \ge 0$, so
    $$ord_{P}(I)=\min{(ord_{P}(\beta),ord_{P}(IJ))}=ord_{P}(\langle \beta \rangle +IJ)$$
    Hence $$ord_{P}(I)=ord_{P}(\langle \beta \rangle +IJ) \text{ for all prime ideals } P$$
    \item[(iii)] Use (ii) and unique factorisation of ideals into prime ideals, we conclude that
    $$I=\langle \beta \rangle +IJ=\langle \beta \rangle +\langle \alpha \rangle =\langle \alpha,\beta \rangle$$
    \end{enumerate}
\end{enumerate}

\subsection{Exercises 14}

\begin{enumerate}
\item Let $\{x_1,\ldots,x_n\}$ be a basis for $\mathcal{O}_K$ and let $x^{(j)}_i,j=1,2,\ldots,n$ be the conjugates of $x_i$ for each $i$ and define the matrix $X$ with entry $X_{ij}=x^{(j)}_i$. Then
    $$\mathcal{D}_K=(\det{X})^2$$
      Let $\det{X}=A+iB$, where $A,B \in \mathbb{R}$. Take conjugate, so we have
    $$(\det{X})^*=A-iB=\det{X^*}$$
    But the complex embeddings appear as conjugate pairs, so taking the conjugate of the matrix is the same
    as swapping $s$ rows. So we have
    $$(-1)^s (A+iB)=A-iB$$
    If $s$ is odd, then $-A-iB=A-iB$ and so $A=0$. Then
    $$(\det{X})^2=i^2 B^2=-B^2<0$$
    If $s$ is even, then $A+iB=A-iB$ and so $B=0$. Then
    $$(\det{X})^2=A^2>0$$
\item Let $\alpha,\beta$ and $\gamma$ be the roots of $x^3+ax+b=0$. So we have
    $$\alpha+\beta+\gamma=0 \Rightarrow \alpha=-\beta-\gamma$$
    $$\alpha\beta+\alpha\gamma+\beta\gamma=a \Rightarrow \beta^2+\gamma^2+\beta\gamma=-a$$
    $$\alpha \beta \gamma =-b \Rightarrow \beta^2 \gamma+\beta \gamma^2 =b$$
    The discriminant $d^2$ is
    $$(\beta-\alpha)^2(\gamma-\beta)^2(\alpha-\gamma)^2=(2\beta+\gamma)^2(\gamma-\beta)^2(2\gamma+\beta)^2$$
    Then we have
    \begin{eqnarray*}
    d^2&=&(4\beta^2+\gamma^2+4\beta\gamma)(4\gamma^2+4\beta\gamma+\beta^2)(\beta^2+\gamma^2-2\beta\gamma)\\
    &=&(33\beta^2\gamma^2+4\gamma^4+4\beta^4+20\beta^3\gamma+20\beta\gamma^3)(\beta^2+\gamma^2-2\beta\gamma)\\
    &=&4\beta^6+4\gamma^6-3\beta^2\gamma^4-3\beta^4\gamma^2-26\beta^3\gamma^3+12\beta\gamma^5+12\beta^5\gamma
    \end{eqnarray*}
    Now we have
    $$-a^3=\beta^6+\gamma^6+3\beta^5\gamma+3\beta\gamma^5+6\beta^4\gamma^2+6\beta^2\gamma^4+7\beta^3\gamma^3$$
    and
    $$b^2=\beta^4\gamma^2+\beta^2\gamma^4+2\beta^3\gamma^3$$
    Hence the discriminant $d^2=-4a^3-27b^2$.

    The result follows by the following observation:\\
    We have two equal roots if and only if discriminant is $0$. If the discriminant is not zero, then two roots
    are complex conjugate, say $\alpha$ and $\beta$ then
    $$(\alpha-\beta)^2(\alpha-\gamma)^2(\beta-\gamma)^2<0$$
    because $$(\alpha-\gamma)^2(\beta-\gamma)^2=|\alpha-\gamma|^4>0$$
    and
    $$(\alpha-\beta)^2=4(Im(\alpha))^2<0$$
\item We have
    $$\theta^2=\alpha^4+2\alpha^3+\alpha^2=5\alpha^2+6\alpha-4$$
    $$\theta^3=22\alpha^2+30\alpha-22$$
    Hence, we have
    $$\theta^3-8\theta^2+18\theta-10=0$$
    Since the discriminant only depends on the difference of the roots, so we may shift each root by
    $\frac{8}{3}$, i.e
    let $f(x)=x^3-8x^2+18x-10$ and $y=(x-\frac{8}{3})$, then
    $$g(y)=y^3-\frac{10}{3}y+\frac{2}{27}$$
    so the discriminant of $g(y)$ is (Use question 2)
    $$-4\left(-\frac{10}{3}\right)^3-27\left(\frac{2}{27}\right)^2=148$$
    Hence the discriminant of $f(x)$ is also 148. By Theorem 14.8, we conclude $D(\theta)=148$.
    Since $D(\theta) \neq 0$, then $K=\mathbb{Q}(\theta)$.
\item \begin{enumerate}
    \item[(i)] By definition,
    $$P+N=\sum_{\tau \in S_n}\prod_{i=1}^n A_{i \tau(i)}$$
    and
    $$PN=\left(\sum_{\tau \text{ even }\in S_n}\prod_{i=1}^n A_{i \tau(i)}\right)
    \left(\sum_{\tau \text{ odd }\in S_n}\prod_{i=1}^n A_{i \tau(i)}\right)$$
    Since every root lies in the field $K$, so the embeddings $\{\sigma_1,\ldots,\sigma_n\}$ is the
    automorphism group of $K$ over $\mathbb{Q}$. (Recall that $|Aut_\mathbb{Q}(K,K)| \le [K:\mathbb{Q}]$.)
    Let $\sigma$ be any element in the automorphism group $G$. But each entry $A_{ij}=\sigma_i(x_j)$, so
    consider
    $$\sigma(P+N)=\sum_{\tau \in S_n}\prod_{i=1}^n \sigma(\sigma_i(x_{\tau(i)}))=(P+N)$$
    because for each fixed $\tau$, the product running through $\sigma\sigma_i$ is the same as the product
    running through $\sigma_i$.
    For the same reason, we have $\sigma(PN)=PN$. Hence $P+N,PN$ are invariant under the action of $G$, and
    hence is fixed by $G$. But $G=Aut_\mathbb{Q}(K)$, so $P+N,PN \in \mathbb{Q}$. Finally, since
    $\sigma_i(x_j)$ is algebraic integer for all $i,j$ and so $P$,$N$ are both algebraic integers. Therefore,
    $P+N,PN$ are rational and algebraic integers, and so they must be ordinary integers.
    $$P+N,PN \in \mathbb{Z}$$
    \item[(ii)] Observe that $d^2=(P+N)^2=(P-N)^2+4PN$. Hence
    $$\mathcal{D}_K=d^2=(P-N)^2+4PN \equiv 0,1~(\text{mod } 4)$$
    because any square is either $0$ or $1$ mod $4$.
    \end{enumerate}
\item \begin{enumerate}
    \item[(i)] For any $i$,
    $$P'(x_i)=\sum_{k=1}^n \prod_{j \neq k}(x_i-x_j)=\prod_{j \neq i}(x_i-x_j)$$
    Hence
    $$disc(P(x))=\prod{1 \le j <i \le n}(x_i-x_j)^2 = (-1)^{\frac{n(n-1)}{2}}\prod_{i=1}^n P'(x_i)$$
    because the factor $(x_i-x_j)^2$ comes from the factor $x_i-x_j$ in $P'(x_i)$ and $x_j-x_i$ in $P'(x_j)$ and
    the sign changes each time. Since we have $\binom{n}{2}=\frac{n(n-1)}{2}$ such factors, so the sign changes
    $(-1)^{\frac{n(n-1)}{2}}$ times.\\
    \item[(ii)]
    \begin{eqnarray*}
    x_if'(x_i)&=&nx^{n-1}_i+px_i\\
    &=&n(x^n_i+px_i+q)-(n-1)px_i-nq\\
    &=&-(n-1)px_i-nq
    \end{eqnarray*}
    which is what we want because $x^n_i+px_i+q=0$.
    Then
    \be
    \prod_{i=1}^n x_if'(x_i) =(-1)^n (nq+(n-1)px_1)(nq+(n-1)px_2) \cdots(nq+(n-1)px_n)
    \ee
    But
    $$x^n+px+q=(x-x_1)\cdots(x-x_n)$$
    and so the elementary functions of degree $k$ is zero except $k=n-1,n$.
    Hence in the above product, the only two non-zero terms are
    $$(nq)^n=n^n q^n \text{ and } ((n-1)p)^n \prod_{i=1}^n x_i=((n-1)p)^n(-1)^nq$$
    Therefore,
   $$\prod_{i=1}^n x_if'(x_i)=(-1)^n (n^nq^n+((n-1)p)^n (-1)^n q)$$
    and so using the previous part,
   $$disc(f(x))=(-1)^{\frac{n(n-1)}{2}} (n^n q^{n-1}+(1-n)^np^n)$$
   because $\prod_{i=1}^n x_i =(-1)^nq$.
    \end{enumerate}
\item Both questions can be done by the following observation.
    If $p \not \equiv 1$ (mod $4$), then $\mathcal{O}_K=\mathbb{Z}[\sqrt{p}]$ and so for any
    $\alpha \in \mathcal{O}_K$, we have $\alpha=a+b\sqrt{p}$, $a,b \in \mathbb{Z}$. Then
    $N(\alpha)=a^2-bp^2$. Then to check there does not exist $\alpha$ such that
    $N(\alpha)=k$, we can reduce the above modulo $p$. So check whether
    $$a^2 \equiv k~(\text{mod }p)$$ exists.

    If $p \equiv 1$ (mod $4$), then $\mathcal{O}_K=\mathbb{Z}[\frac{1+\sqrt{p}}{2}]$ and so
    for any $\alpha \in \mathcal{O}_K$, we have $\alpha=a+b(\frac{1+\sqrt{p}}{2})$. Then
    $N(\alpha)=a^2+ab+b^2(\frac{1-p}{4})$. Then if there exists $\alpha$ whose norm is $k$, then
    $$4a^2+4ab+b^2-b^2p=4k$$ and so
    $$(2a+b)^2 \equiv 4k~(\text{mod }p)$$
    But
    $$\left(\frac{4k}{p}\right)=\left(\frac{k}{p}\right)$$
    So in the first part,
    $$\left(\frac{2}{p}\right)=-1 \text{ if } p \equiv \pm 3~(\text{mod } 8)$$
    and in the second part,
    $$\left(\frac{-2}{p}\right)=-1 \text{ if } p \equiv 5,7~(\text{mod } 8)$$
    So in fact, we have proved a stronger fact.
\item \begin{enumerate}
    \item[(i)] Suppose not, then $I$ can be written as a product ideals. Say $I=JK$.
    Then $N(I)=N(J)N(K)$ and since $N(J)$ and $N(K)$ are both integers, so $N(I)$ is not prime, which is a
    contradiction.\\
    \item[(ii)] If $I=\mathcal{O}_K$, then $N(I)=1$ and $1 \in I$. Suppose $I$ is a proper integral ideal, then
    consider
    $$1+I,2+I,\ldots \in \mathcal{O}_K$$
    But $\mathcal{O}_K/I$ is finite, so there exists $k \in \mathbb{N}, k >1$ such that
    $k \in I$. Hence the order of $1+I \in \mathcal{O}_K$ is $k$. But the order of the (additive) group
    $\mathcal{O}_K$ is $N(I)$. Hence $k|N(I)$ and so $N(I) \in I$ because $k \in I$.
    \end{enumerate}
\item Since $P$ is prime, and so maximal in a Dedekind domain. Hence $\mathcal{O}_K/P$ is a field and is finite. So the multiplication group of a finite field is cyclic. The order of the group is $N(P)$.
\item $N(I)=|N(\alpha)|=N(\langle \alpha \rangle)$, and $\alpha \in I$, so $\langle \alpha \rangle \subseteq I$.
    Then
   $$N(I)^2=\frac{D(I)}{\mathcal{D}_K},N(\langle \alpha \rangle)^2=
    \frac{D(\langle \alpha \rangle)}{\mathcal{D}_K}$$
   But $N(I)=N(\langle \alpha \rangle)$, so $D(I)=D(\langle \alpha \rangle)$, and so the index of
   $\langle \alpha \rangle$ in $I$ is $1$, and so $I=\langle \alpha \rangle$.
\item Since $P$ is a prime ideal, so $P \cap \mathbb{Z}$ is a prime ideal in $\mathbb{Z}$. Since $\mathbb{Z}$ is
   a principal ideal domain, so $P \cap \mathbb{Z}=\langle \alpha \rangle$. Hence $P \cap \mathbb{Z}$ is prime, so
   $\alpha=p$ is a prime number.
\item By question 7(ii), $$I | \langle N(I) \rangle$$
   so we have
   $$I | \langle n \rangle$$
   But $n$ is fixed, so we can factorise $\langle n \rangle$ into prime ideals, and so there are only finitely many
   integral ideals $I$ such that $I| \langle n \rangle$.
\item Since $a \not \in P$, so we consider the order (under multiplication) of $a+P$ in $\mathcal{O}_K/P$. (It is a group because $P$ is prime and so maximal, so $\mathcal{O}_K/P$ is a field.)
   Since the order of the group (under multiplication) $\mathcal{O}_K/P$ is $N(P)$. So $(a+P)^{N(P)}=1+P$ and so
   $$a^{N(P)}-1 \equiv 0~(\text{mod } P)$$
\item Suppose $I$ is a prime ideal. Then $I$ is maximal and hence $\mathcal{O}_K/I$ is a finite field. Then
   $N(I)=p^k$ for some prime $p$ and $k>0$, which contradicts $pq \big|N(I)$.
\item Let
    $$I=\langle \alpha,p \rangle, J=\langle \alpha,q \rangle, K=\langle \alpha \rangle$$
    It is clear that
    $$I,J |K \text{ because } K \subseteq I,J$$
    Also, since $(p,q)=1$, so $I+J=\langle 1 \rangle=\mathcal{O}_K$. Then $I$ and $J$ are coprime and so
    $$IJ | K \Rightarrow N(I)N(J)|N(K)=pq$$
    In particular, $N(I),N(J)|pq$. Now $I|\langle p \rangle$, so
    $$N(I)| N(\langle p \rangle)=p^2 \text{ because } [K:\mathbb{Q}]=2$$
    Therefore, $N(I)|pq,p^2 \Rightarrow N(I)|p$ because $p$ and $q$ are coprime. Similarly, $N(J)|q$.
    But by question 7(ii), $N(I) \in I$ and clearly that $p$ is the least positive integer in $I$. Hence
    $N(I)=p$. Similarly, $N(J)=q$ and so $N(I)N(J)=pq=N(K)$. Therefore,
    $$IJ=K$$
    Alternatively, we can calculate the discriminant of $I,J$ and $K$ directly and conclude $IJ=K$.
\item \begin{enumerate}
    \item[(i)] If $u,v \in \mathcal{R}^{-1}_K$, then
    $$T_{K/\mathbb{Q}}((u+v)y)=T_{K/\mathbb{Q}}(vy)+T_{K/\mathbb{Q}}(vy) \in \mathbb{Z}, \forall y \in \mathcal{O}_K$$
    If $u \in \mathcal{R}^{-1}_K$, and $d \in \mathcal{O}_K$, then
    $$T_{K/\mathbb{Q}}(udy)=T_{K/\mathbb{Q}}(u(dy)) \in \mathbb{Z}, \forall y \in \mathcal{O}_K$$
    Now we need to check the third property of fractional ideal. As in the proof of
    Theorem 12.38, there exists
    $\{e_1,\ldots,e_n\} \subseteq \mathcal{O}_K$ which is a $\mathbb{Q}$-basis for $K$. Since $K/\mathbb{Q}$ is
    separable, so the trace pairing $T$ is non-degenerate. Let $T$ be the matrix such that
    $$P_{ij}=T(e_i,e_j)$$
    Let $d=\det{P}$ and so $d \neq 0$. Let $Q=P^{adj}$ so that $PQ=d I$, so
\begin{equation*}
\sum_{j=1}^n P_{ij}Q_{jk}= \left\{
\begin{array}{ll}
d & \text{if } i=k\\
0 & \text{if } i \neq k\\
\end{array} \right.
\end{equation*}

    Suppose now $x \in \mathcal{R}^{-1}_K$, then there exist $a_1,\ldots,a_n \in \mathbb{Q}$ such that
    $$x=\sum_{i=1}^n a_i e_i$$
    and so
    $$T_{K/\mathbb{Q}}(x e_j)=\sum_{i=1}^n T_{K/\mathbb{Q}}(a_i e_i e_j)=\sum_{i=1}^n a_i P_{ij}$$
    Therefore, we have
    $$da_k=\sum_{j=1}^n T_{K/\mathbb{Q}}(xe_j)Q_{jk} \in \mathcal{O}_K$$
    because the each entry of $Q_{jk}$ lies in $\mathcal{O}_K$ and $T_{K/\mathbb{Q}}(xe_j) \in \mathbb{Z}$.
    Therefore,
    $$dx=\sum_{i=1}^n da_ie_i \in \mathcal{O}_K$$
    and so
    $$d \mathcal{R}^{-1}_K \subseteq \mathcal{O}_K$$
\item[(ii)] By definition, $\mathcal{R}_K=\{x \in K: x\mathcal{R}^{-1}_K \subseteq \mathcal{O}_K\}$.
    Pick any $x \in \mathcal{R}_K$ and any $y \in \mathcal{R}^{-1}_K$, then we have
    $xy \in \mathcal{O}_K$, in particular, since $1 \in \mathcal{R}^{-1}_K$, so
    $$x=x \cdot 1 \in \mathcal{O}_K$$
    Therefore, $\mathcal{R}_K \subseteq \mathcal{O}_K$.
\item[(iii)] Let $\{e_1,\ldots,e_n\}$ be a basis for $\mathcal{O}_K$ and $\{f_1,\ldots,f_n\}$ be its dual basis with respect to the trace pairing. In other words,
    $$T(e_i,f_j)=\delta_{ij}$$
    Then for any $x \in \mathcal{R}^{-1}_K$, we claim that
    $$x=\sum_{i=1}^n T(x,e_i)f_i$$
    because we can multiply both sides by $e_j$ and take the trace. Since the trace is a non-degenerate
    linear map, so this shows that
    $$T_{K/\mathbb{Q}}(xe_j)=T(x,e_j)=T_{K/\mathbb{Q}}(xe_j)$$ for all $e_j$ and since $\{e_1,\ldots,e_n\}$ is a
    basis, so we have what we claimed.
    Since $T(x,e_i) \in \mathbb{Z}$ as $x \in \mathbb{R}^{-1}_K$, so $\{f_1,\ldots,f_n\}$ is a $\mathbb{Z}$-basis
    for $\mathcal{R}^{-1}_K$.

    Now for each $i$, write
    $$e_i=\sum_{j=1}^n a_{ij}f_j \text{ for some } a_{ij} \in \mathbb{Z}$$
    because $\mathcal{O}_K \subseteq \mathcal{R}^{-1}_K$. Then multiply both sides by $e_k$ and take trace, we have
    $$T(e_i,e_k)=a_{ik}$$
    and so the matrix $A$ has entries $a_{ij}=T(e_i,e_j)$. Hence we have
    $$T(e_i,e_j)=T(\sum_{k=1}^n a_{ik}f_k,\sum_{k=1}^n a_{jk}f_k)$$
    Taking determinant of both sides, we have
    $$\mathcal{D}_K=(\mathcal{D}_K)^2 disc(\mathcal{R}^{-1}_K)$$
    and so the index $$[\mathcal{R}^{-1}_K:\mathcal{O}_K]=|\mathcal{D}_K|$$
    Finally, by Remark 14.23, we have
    $$N(\mathcal{R}_K)=[\mathcal{O}_K:\mathcal{R}_K]=[\mathcal{R}^{-1}_K:\mathcal{O}_K]=|\mathcal{D}_K|$$
\item[(iv)] For each $i$, we have
    $$f'(x_i)=\prod_{j \neq i}(x_i-x_j)$$
    Consider
    $$\sum_{i=1}^n \frac{1}{(T-x_i)f'(x_i)}=\frac{\sum_{i=1}^n A_i(T)}{f(T)}=\frac{C(T)}{f(T)}$$
    where
    $$A_i(T)=\frac{f(T)}{f'(x_i)(T-x_i)}$$
    It is clear that $C(T)$ has degree at most $n-1$.
    But let $T=x_i$ for a fixed $i$. Then
    $$A_i(x_i)=\frac{\prod_{j \neq i}(x_i-x_j)}{f'(x_i)}=1, A_j(x_i)=0$$
    and so $C(x_i)=1$ for each $i$. Then $C(T)-1$ has $n$ zeros but $C(T)-1$ has degree at most $n-1$. So we
    conclude that $C(T)-1=0$ and so $C(T)=1$.
\item[(v)] Since the embeddings are homomorphism, so
    $$T_{K/\mathbb{Q}}\left(\frac{x^r}{f'(x)}\right)=\sum_{i=1}^n \left(\frac{x^r_i}{f'(x_i)}\right)$$
    Consider
    $$\sum_{i=1}^n \frac{x^r_i}{f'(x_i)(T-x_i)}=\frac{\sum_{i=1}^n A_i(T)}{f(T)}=\frac{C(T)}{f(T)}$$
    where
    $$A_i(T)=\frac{f(T)}{f'(x_i)(T-x_i)}x^r_i$$
    It is clear that for each $i$,
    $$A_i(x_i)=x^r_i, A_j(x_i)=0$$
    So $C(x_i)=x^r_i$. Now the degree of $C(T)-T^r$ is at most $n-1$,
    and $C(T)-T^r$ has $n$ roots, $x_1,\ldots,x_n$. so $C(T)=T^r$. Then
    $$\frac{T^r}{f(T)}=\sum_{i=1}^n \frac{x^r_i}{f'(x_i)(T-x_i)}$$
    and multiply both sides by $f(T)$, we have
    $$T^r=\sum_{i=1}^n \frac{x^r_i}{f'(x_i)} \prod_{j \neq i}(T-x_j)$$
    Hence, by comparing the coefficient of $T^{n-1}$ on both sides and since $r <n-1$, we have
    $$\sum_{i=1}^n \frac{x^r_i}{f'(x_i)}=0$$
    For $r=n-1$, we again compare the coefficient of both sides, so
    $$\sum_{i=1}^n \frac{x^{n-1}_1}{f'(x_i)}=1$$
\item[(vi)] Since $\mathcal{O}_K=\mathbb{Z}[x]$ so $\{1,x,\ldots,x^{n-1}\}$ is a basis for $\mathcal{O}_K$.
    We have
    shown above that the dual basis of $\{1,x,\ldots,x^{n-1}\}$ with respect to the trace pairing is a basis for
    $\mathcal{R}^{-1}_K$. Let the minimal polynomial $f$ be
    $$f(T)=T^n+a_{n-1}T^{n-1}+\cdots+a_1T+a_0$$
    Consider the dual element of $1$. Let $\alpha=\frac{b_{n-1}x^{n-1}+\cdots+b_1x+b_0}{f'(x)}$, and we want
    $$T(\alpha,1)=1,T(\alpha,x^r)=0$$
     Then we have $b_{n-1}=1$ by $T(\alpha,1)=1$ and use (v). By the above relation, we have
    $x^n=-a_{n-1}x^{n-1}-\cdots-a_0$ so that $T(\alpha,x)=0$ implies that
    $b_{n-2}=a_{n-1}$. Similarly, we may use the relation and continue in this way, so that we can uniquely
    determine the values of $b_i$. So we may write the dual basis as:
    $$\left\{\frac{\alpha_1}{f'(x)},\ldots \frac{\alpha_n}{f'(x)}\right\}$$
    where $\alpha_1,\ldots,\alpha_n$ is a polynomial in $x$ whose coefficients are sum and products of $a_i$ and so
    lies in $\mathcal{O}_K$. Therefore, we conclude that
    $$\mathcal{R}^{-1}_K =\frac{\mathcal{O}_K}{f'(x)}$$
    and so
    $$\mathcal{R}_K=\langle f'(x) \rangle$$
    \end{enumerate}

\item
\begin{enumerate}
\item[(i)] We firstly show that the product converges. For each $k \ge 1$ and $p$ prime,
$\frac{1}{1-(p^k)^{-s}} \le \frac{1}{1-p^{-s}}$
Also if $[K:\mathbb{Q}]=n$, then each each prime lies in at most $n$ prime ideals, and so
we have
$$\prod_{P} \frac{1}{1-N(P)^{-s}} \le \prod_{p} \frac{n}{1-p^{-s}}$$
and the right hand side converges on $\mathcal{R}e(s)>1$ (which we have seen before) so the left hand side also converges on this region. Then we can expand the product as
$$\prod_{P}\frac{1}{1-N(P)^{-s}}=\prod_{P}\left(\sum_{k=0}^\infty \frac{1}{(N(P)^s)^k}\right)$$
and since each proper integral ideal can be factorised into prime ideals, so
$$\zeta_K(s)=\sum_{0 \neq I \subseteq \mathcal{O}_K} \frac{1}{N(I)^s}=\prod_{P}\frac{1}{1-N(P)^{-s}}$$
\item[(ii)] Since $\mathcal{O}_\mathbb{Q}=\mathbb{Z}$, and each ideal is principal, so it is clear that
$\zeta_\mathbb{Q}(s)$ is just the Riemann $\zeta$ function. Also $K=\mathbb{Q}[i]$ and so
$\mathcal{O}_K=\mathbb{Z}[i]$, which is a Euclidean domain and so is a principal ideal domain. Thus, each prime ideal in $\mathbb{Z}[i]$ is generated by a prime element in $\mathbb{Z}[i]$. If the prime element is $a+ib$, where $b \neq 0$, then the norm must be a prime $p$ such that $p \equiv 1$ (mod $4$). (or $p=2$), and if
$a+ib$ is prime, so is $a-ib$. So each $p \equiv 1$ (mod $4$) appears twice in the product. On the other hand,
the prime $p \equiv 3$ (mod $4$) is itself a prime element in $\mathbb{Z}[i]$ and so the norm of it is
$p^2$. Therefore, using
$$p \equiv 1~(\text{mod } 4) \iff (-1)^{\frac{p-1}{2}}=1$$
we conclude that
\begin{eqnarray*}
\zeta_K(s)&=&\prod_{p \equiv 3~(\text{mod }4)} \frac{1}{1-p^{-2s}} \prod_{p \equiv 1~(\text{mod }4)}\frac{1}{(1-p^{-s})^2}\\
&=&\prod_{p}\frac{1}{1-p^{-s}} \prod_{p}\frac{1}{1-\chi(p)p^{-s}} = \zeta_\mathbb{Q}(s)\cdot L(\chi,s)
\end{eqnarray*}
\item[(iii)]
$$L(\chi,s)=\prod_{p \equiv 1~(\text{mod }4)}\left(\sum_{k=0}^\infty \frac{1}{(p^{-s})^k}\right)
\prod_{p \equiv 3~(\text{mod }4)}\left(\sum_{k=0}^\infty \frac{(-1)^k}{(p^{-s})^k}\right)$$
Now each odd integer can be written as a product of odd primes, and from the above expression, we can see that
the negative sign comes in if we have a prime which is $3$ modulo $4$. Hence the sign is negative if and only if it contains odd number of prime factors which are $3$ modulo $4$ and so itself is $3$ modulo $4$. Therefore,
$$L(\chi,s)=\sum_{n=0}^\infty \frac{(-1)^n}{(2n+1)^s}$$
\end{enumerate}
\end{enumerate}

\subsection{Exercises 15}
\begin{enumerate}
\item $47 \equiv 3$ mod $4$ and so by the remark following Theorem 15.12, we have
$$\langle 2 \rangle =\langle 2,1+\sqrt{47} \rangle^2$$
\item It is clear that, since $26 \equiv 2$ (mod $4$),
$$\langle 2 \rangle=\langle 2,\sqrt{26} \rangle^2$$
and since $(\frac{26}{5})=1$, and $1^2=26$ (mod $5$), so
$$\langle 5 \rangle=\langle 5, 1+\sqrt{26} \rangle \langle 5,1-\sqrt{26} \rangle$$
Now clearly, that $6+\sqrt{26} \in \langle 2,\sqrt{26} \rangle, \langle 5,1+\sqrt{26} \rangle$, and it is clear
that these two are different ideals since they contains different primes. (By Lemma 15.1, the prime is unique.)
Now $N(6+\sqrt{26})=(6+\sqrt{26})(6-\sqrt{26})=10$ and so by consider the norm, we conclude that
$$\langle 6+\sqrt{26} \rangle =\langle 2,\sqrt{26} \rangle \langle 5,1+\sqrt{26} \rangle$$
\item Using the same method, we have
$$\langle 2 \rangle=\langle 2,1+\sqrt{35} \rangle^2$$
$$\langle 5 \rangle=\langle 5,\sqrt{35} \rangle^2$$
because $35 \equiv 3$ (mod $4$) and $5|35$.

Now clearly $5+\sqrt{35} \in \langle 2,1+\sqrt{35} \rangle, \langle 5,\sqrt{35} \rangle$.
$N(\langle 5+\sqrt{35} \rangle)=10$ and so by considering the norm, we conclude that
$$\langle 5+\sqrt{35} \rangle =\langle 2,1+\sqrt{35} \rangle \langle 5,\sqrt{35} \rangle$$

\item Let $\theta=\sqrt[3]{3}$, and so the discriminant of $\mathbb{Z}[\theta]$ is $-2700$. Now, we have:
$$\langle 2,\theta \rangle^3= \langle 2 \rangle \langle 4,2\theta,\theta^2,5\rangle=\langle 2 \rangle$$
$$\langle 5,\theta \rangle^3= \langle 5 \rangle \langle 25,5\theta, \theta^2,2\rangle=\langle 5 \rangle$$
$$\langle 3,\theta-1\rangle^3=\langle 3 \rangle \langle 9,\theta^2-2\theta+1,3\theta-3,3-\theta^2-\theta \rangle =\langle 3\rangle$$
Hence, $2,3$ and $5$ ramify in $K$ and so by Theorem 15.11, $2,3,5|\mathcal{D}_K$. Since
$diss(\mathbb{Z}[\theta])=-2700$, so the only possible discriminant of $\mathcal{O}_K$ is $-2700$ or $-300$ because $2,3,5 |\mathcal{D}_K$. In either case, for any $\alpha \in \mathcal{O}_K$, we have
$3\alpha \in \mathbb{Z}[\theta]$ because the index of $\mathbb{Z}[\theta]$ is either $1$ or $3$.
If the index is $3$, then for any $\alpha \in \mathcal{O}_K$, we have $3\alpha \in \mathbb{Z}[\theta]$,
and so we may write
$$\alpha=a+b\theta+c\theta^2$$
where $a,b,c \in \frac{1}{3}\mathbb{Z}$. But $3$ is a prime, and the index is the product of denominators of
$a,b$ and $c$. Hence two of $a,b$ and $c$ must be integers and only one of them could possibly take the denominator $3$. But none of the following is an algebraic integer:
$$\frac{1}{3},\frac{2}{3},\frac{1}{3}\theta,\frac{2}{3}\theta,\frac{1}{3}\theta^2,\frac{2}{3}\theta^2$$
So the index cannot be $3$ and so is $1$.
\item \begin{enumerate}
\item[(i)] Let $\theta=\sqrt[3]{3}$, Consider the ideal
$$P=\langle 3,\theta \rangle$$
then
$$P^3=\langle 3 \rangle \langle 9,3\theta,\theta^2,1 \rangle=\langle 3 \rangle$$
and so $N(P)^3=N(3)=27$. Then $N(P)=3$ and so $P$ is a prime ideal.\\
\item[(ii)] $p$ ramifies if and only if $p|\mathcal{D}_K$. So if
$$\langle p \rangle =Q^3$$
then $p|\mathcal{D}_K$. Since $\mathbb{Z}[\sqrt[3]{3}] \subseteq \mathcal{O}_K$, so we conclude that
$p|disc(\mathbb{Z}[\sqrt[3]{3}])=-81$. Then $p=3$ since $p$ is a prime. Hence there does not exist any other such prime numbers.
\end{enumerate}
\item By definition,
$$N(\zeta_m)=\prod_{i=1}^\phi(m) \phi_i(\zeta_m)=\prod_{(j,m)=1}e^{\frac{2\pi i}{m}j}$$
Consider now the sum
$$\sum_{(j,m)=1}j$$
It is clear that $(j,m)=1$ if and only if $(m-j,j)=1$. Therefore, we can pair $j$ and $m-j$ in the sum, so
$$m |\sum_{(j,m)=1}j$$
and hence
$$N(\zeta_m)=e^{\sum_{(j,m)=1}j \frac{2\pi i}{m}}=1$$
\item
\begin{enumerate}
\item[(i)] Let $[K:\mathbb{Q}]=n$ and so
$$\langle p \rangle\mathcal{O}_K=P^n$$
where $P$ is a prime ideal in $\mathcal{O}_K$. Let $M$ be an intermediate field and let $[M:\mathbb{Q}]=m$.
Suppose there are distinct prime ideals $Q$ and $R$ in $\mathcal{O}_M$ which contains $P$. Then
let $Q'$ and $R'$ be any ideals in $\mathcal{O}_K$ such that
$$Q' \cap \mathcal{O}_M=Q, R' \cap \mathcal{O}_M=R$$
Then it is clear that $Q',R'$ both divide $\langle p \rangle$ and they are coprime. Indeed, if not, then
there exists $P' \supseteq Q',R'$. Hence $P \cap \mathcal{O}_M \supseteq Q,R$. Therefore, $Q$ and $R$ are not coprime, which is a contradiction. So this shows that if there are two prime ideals containing $p$ in $\mathcal{O}_M$, then there are at least two prime ideals containing $p$ in $\mathcal{O}_K$. Since
$p$ is totally ramified in $\mathcal{O}_K$, so there is only one prime ideal containing $p$ in
$\mathcal{O}_M$. Then we have
$$\langle p \rangle=Q^t$$
for some prime ideal $t$. Then we show $N(Q)=p$ and so that
$$\langle p \rangle=Q^m$$
Indeed, the homomorphism
$$f: \mathcal{O}_M/Q \rightarrow \mathcal{O}_K/P$$
by $f(a+Q)=a+P$ is clearly an injection. But both of them are fields, and
$\card(\mathcal{O}_K/P)=N(P)=p$ which is the base field of characteristic $p$. Therefore, $N(Q)=\card(\mathcal{O}_M/Q)=p$.

\item[(ii)] Let $M=K \cap L$. Since $\langle p \rangle$ is totally ramified in $K$, so it is totally ramified in $M$ by (i) and let $[M:\mathbb{Q}]=n$ so
$$\langle p \rangle O_M=P^n$$
Then the factorisation of $\langle p \rangle$ in $O_L$ is:
$$\langle p \rangle O_M=(P\mathcal{O}_L)^n$$
But whatever the factorisation of $P\mathcal{O}_L$ is, each prime factor must has index $n$ (or at least $n$)
and $\langle p \rangle$ is unramified in $O_L$. This shows that $n$ must be $1$ and so
$$M=\mathbb{Q}$$
\end{enumerate}
\item $disc(\mathbb{Z}[\theta])=-4a^3-27b^2=-23$ which is square free and so we conclude that
$$\mathbb{Z}[\theta]=\mathcal{O}_K$$
Suppose $\sqrt{\theta} \in \mathbb{Q}(\theta)$, then $\sqrt{\theta} \in \mathcal{O}_K$, because clearly it satisfies $x^6-x^2-1=0$.
Thus, there exist $a,b$ and $c \in \mathbb{Z}$ such that
$$\sqrt{\theta}=a+b\theta+c\theta^2$$
which gives
$$c^2 \theta^4+2bc\theta^3+(b^2+2ac)\theta^2+(2ab-1)\theta+a^2=0$$
Use $\theta^3=\theta-1$ we have
$$(b^2+2ac+c^2)\theta^2+(2ab+2bc-2)\theta+a^2-2bc=0$$
This polynomial must be identically zero, otherwise it contradicts that the minimal polynomial of $\theta$ has degree $3$. So
$$b^2+2ac+c^2,ab+bc-1,a^2-2bc=0$$
But $ab+bc-1=0 \Rightarrow (a+c)b=1$ and since $a,b,c$ are all integers. So $b=\pm 1$.
If $b=1$, then $a+c=1$ and $a^2=2c$, which gives
$$c^2-4c+1=0$$ which has no integer solution. If $b=-1$, then
$a+c=-1$ and $a^2=-2c$, so $c^2+4c+1=0$ which again has no integer solution. Therefore, the assumption is not true and we conclude that $\sqrt{\theta} \not \in \mathbb{Q}(\theta)$.
\item
\begin{enumerate}
\item[(i)] $disc(\mathbb{Z}[\theta])=-4a^3-27b^2=-104=-2^2 26$.
But $\mathcal{D}_K \equiv 1$ or $0$ (mod $4$), so it cannot be $-26$. Thus, $\mathcal{D}_K=-104$ and
$\mathcal{O}_K=\mathbb{Z}[\theta]$. Then consider the element $2+\theta \in \mathcal{O}_K$.
We have
$$(2+\theta)(\theta^2-2\theta+3)=\theta^3-\theta+6=2$$
and hence $2$ and $\theta \in \langle 2+\theta \rangle$. Therefore,
$$\langle 2, \theta \rangle=\langle 2+\theta \rangle$$
\item[(ii)] By Kummer-Dedekind, we have
$$\langle 2 \rangle =\langle 2,\theta \rangle \langle 2, \theta-1 \rangle^2$$
and so $N(\langle 2,\theta \rangle)=2$. Thus $N(\langle 2+\theta \rangle)=\pm 2$.
\end{enumerate}
\end{enumerate}
\subsection{Exercises 16}
In the following solutions, let $[1]$ be the identity in the ideal class group.
\begin{enumerate}
\item Let $K=\mathbb{Q}(\sqrt{-30})$ and then $\mathcal{D}_K=-120$. The Minkowski bound is
$$c_K=\frac{4}{\pi}\frac{2!}{2^2}\sqrt{120}<7$$
Consider the primes $p=2,3,5$.
But $2,3,5|-120$, so we have
$$\langle 2 \rangle=P^2,\langle 3 \rangle=Q^2,\langle 5 \rangle=R^2$$
and $N(P)=2,N(Q)=3,N(R)=5$.
Now, consider I=$\langle \sqrt{-30} \rangle$, and $N(I)=30=2\cdot 3 \cdot5$.
Then we conclude that
$$I=PQR \Rightarrow [P][Q][R]=[1]$$
But $[R]^2=[1]$ and so $[P][Q]=[R]$.
Then since $P,Q,R$ are not principal, since we have no element of norm $2,3,5$ and
the ideals with norm less than or equal to $6$ are:
$P,Q,R,PQ$ and so the ideal class group is
$$Cl(K)=\{[P],[Q],[PQ]=[R],[1]\} \cong C_2 \times C_2$$
\item Let $K=\mathbb{Q}(\sqrt{-65})$ and then $\mathcal{D}_K=260$. The Minkowski bound is
$$c_K=\frac{4}{\pi}\frac{2!}{2^2}\sqrt{260}<11$$
Consider the primes $p=2,3,5,7$.
Since $2,5|-260$ so
$$\langle 2 \rangle=P^2, \langle 5 \rangle=R^2$$
where $N(P)=2,N(R)=5$
and $(\frac{-65}{3}=1$, $(\frac{-65}{7})=-1$, so
$$\langle 3 \rangle= Q\bar{Q} \text{ and } \langle 7 \rangle=S$$
where $N(Q)=N(\bar{Q})=3$ and $S$ is prime with $N(S)=49$.
Let $\alpha \in \mathcal{O}_K$, then $\alpha=a+b\sqrt{-65}$ for some $a,b \in \mathbb{Z}$. Then
$N(\alpha)=a^2+65b^2$. Take $a=4,b=1$ then we have
$N(4+\sqrt{-65})=81=3^4$. But it is clear that $4+\sqrt{-65}$ is not divisible by $3$, hence
it is either $Q^4$ or $\bar{Q}^4$. Assume it is $Q^4$. Then
$$[Q]^4=[1]=[\bar{Q}]^4$$
and so $[Q]$ has order $4$ because it is not principal and $[Q]^2$ is not principal as the only principal
ideal with norm $9$ is $\langle 3 \rangle$. Hence $[\bar{Q}]$ also has order $4$.
Then consider $a=5,b=1$ we have $N(5+\sqrt{-65})=90=3^2 \cdot 2 \cdot 5$. It is clear that
$5+\sqrt{-65}$ is not divisible by $3$ and so it is $Q^2PR$ or $\bar{Q}^2 PQ$. Assume it is
$Q^2PR$. Then we have
$$[Q]^2 [P][R]=[1] \Rightarrow [P][Q]^2=[R]$$
because $[R]^2=[1]$. Hence the ideal class group is
$$Cl(K)=\{[P],[Q],[Q]^2,[Q]^3,[P][Q],[P][Q]^2,[P][Q]^3,[1]\} \cong C_4 \times C_2$$
because the ideals with norm less than 11 are:
$$P,P^2,P^3,Q,Q^2,\bar{Q},\bar{Q}^2,R,PR,PQ,P\bar{Q}$$
and each of these belongs to one of the above classes and so we have found them all. Also, we may observe
that the two assumptions we made above does not matter because $Q$ and $\bar{Q}$ are in some sense symmetric,
so we only need to find the order of them.\\
\item Let $f(x)=x^3-4x+2$ and so
$$disc(f)=disc(\mathbb{Z}[\theta])=-4(-4)^3-27(2)^2=148=2^2\cdot 37$$
So the discriminant of $K=\mathbb{Q}(\theta)$ is either $148$ or $37$.
Consider that
$$\langle \theta \rangle^3=\langle \theta^3 \rangle=\langle 2 \rangle \langle 2\theta-1 \rangle$$
and
$$(2\theta-1)(-4\theta^2-2\theta+15)=-8\theta^3+32\theta-15=1$$
and so $\langle 2\theta-1 \rangle=\langle 1 \rangle$. Hence
$$\langle \theta \rangle^3=\langle 2 \rangle$$
Since $2$ ramifies in $K$, so $2|\mathcal{D}_K$ and so $\mathcal{D}_K=148$.
As $disc(f)>0$, so we have three distinct real roots. So we have $n=3,r=3,s=0$.
$$c_K=\frac{3!}{3^3}\sqrt{148}<3$$
so we only need to consider the prime $p=2$.
But $\langle 2 \rangle=\langle \theta \rangle^3$. Since $\langle \theta \rangle$ is principal and so the class group is trivial.\\
\item
The minimal polynomial of $\sqrt[4]{2}$ is $x^4-2$. The conjugates are
$$\sqrt[4]{2},i\sqrt[4]{2},-\sqrt[4]{2},-i\sqrt[4]{2}$$
So $n=4,s=2,r=2$. Since $disc(\mathbb{Z}[\sqrt[4]{2}]=-1024=-32^2$.
So $|\mathcal{D}_K|$ is at most $1024$. Thus
$$c_K=\left(\frac{4}{\pi}\right)^2 \frac{4!}{4^4}\sqrt{|\mathcal{D}_K|} <5$$
Consider the primes $p=2,3$. Write $\theta=\sqrt[4]{2}$. So it is clear that
$$\langle 2 \rangle=\langle \theta \rangle^4=P^4$$
because $\theta^4=2$. Also, the index must be prime to $3$, so by Kummer-Dedekind, the polynomial
$x^4-2$ mod $3$ is irreducible and so $\langle 3 \rangle$ is prime.
But $\langle \theta \rangle$ is also prime. So the class group is trivial.
\item
\begin{enumerate}
\item[(i)] Let $K=\mathbb{Q}(\sqrt{-13}$. We have $n=2,r=0,s=1$ and $\mathcal{D}_K=-52$. So
$$c_K=\left(\frac{4}{\pi}\right) \frac{2!}{2^2}\sqrt{52}<5$$
Consider the primes $p=2,3$. Since $2|-52$, so $2$ ramifies in $K$,
$$\langle 2 \rangle=P^2$$
where $N(P)=2$. As $(\frac{-13}{3})=-1$ so $\langle 3 \rangle$ is prime. It is clear that $P$ is not principal because there exists no element in $\mathcal{O}_K$ with norm $2$. Hence we conclude that
$$Cl(K)=\{[P],[1]\} \cong C_2$$
\item[(ii)] Let
$$\langle x+\sqrt{-13} \rangle=\prod_{i=1}^m P^{a_i}_i\bar{P}^{b_i}_i \prod_{j=1}^n Q^{c_j}_j$$
where $P_i \neq \bar{P}_i$, $Q_j=\bar{Q}_j$.
If $a_i,b_i \neq 0$ for some $i$ then $P_i\bar{P_i}$ is in the factorisation and we know
$P_i\bar{P_i}=\langle p_i \rangle$ for some prime $p_i$. Then
$$p_i |x+\sqrt{-13} \text{ in } \mathcal{O}_K$$
which is impossible. Hence for each $i$, either $a_i$ or $b_i=0$. Assume $b_i=0$, and we collect
the product as
$$\prod_{i=1}^m P^{a_i}_i \prod_{j=1}^n Q^{c_j}_j$$
where the conjugate of $P_i$ never appears.
Then
$$\langle x-\sqrt{-13} \rangle=\prod_{i=1}^m \bar{P}^{a_i}_i \prod_{j=1}^n Q^{c_j}_j$$
and so
$$\langle x^2+13\rangle=\prod_{i=1}^m P^{a_i}_i \bar{P}^{a_i}_i \prod_{j=1}^n Q^{2c_j}_j$$
But this is $\langle y \rangle^3$ and so by unique factorisation we have
$3|a_i,c_j$ for all $i,j$. Hence
$$\langle x+\sqrt{-13} \rangle=I^3$$
for some ideal $3$ because the exponents are all divisible by $3$.
Hence
$$[1]=[\langle x+\sqrt{-13} \rangle]=[I]^3$$
But the ideal class group is $C_2$. Hence $[I]=[1]$ and $I$ is principal.
It is clear that the units in $\mathcal{O}_K$ are $\pm 1$ and so
$$\langle x+\sqrt{-13} \rangle=\langle \left(\pm (a+b\sqrt{-13})\right)^3 \rangle$$
where $a,b \in \mathbb{Z}$.
We take $x+\sqrt{-13}=(a+b\sqrt{-13})^3$ first as the other case is similar.
Then
$$x+\sqrt{-13}=a^3+3a^2b\sqrt{-13}-39ab^2-13b\sqrt{-13}$$
Then compare the coefficient of $\sqrt{-13}$ we have
$$1=b(3a^2-13)$$
Then $b=\pm 1$. (This then includes the other case.)
If $b=1$ then $3a^2-13=1$, which is impossible. If $b=-1$, then $a=\pm 2$.
When $a=2,b=-1$, we have $x=-70,y=17$. When $a=-2,b=-1$, we have $x=70,y=17$.
Hence $x=\pm 70, y=17$ are the only integer solutions.
\end{enumerate}
\item We firstly compute the class group of $K=\mathbb{Q}(\sqrt{-21})$.
We have $n=2,r=0,s=1$, and $\mathcal{D}_K=-84$. Then
$$c_K=\left(\frac{4}{\pi}\right)\frac{2!}{2^2}\sqrt{84}<6$$
Consider the primes $p=2,3,5$.
Since $2,3|-84$ so $2,3$ both ramify in $K$ and
$$\langle 2 \rangle=P^2,\langle 3 \rangle=Q^2$$
where $N(P)=2,N(Q)=3$. Since $(\frac{-84}{5})=1$ so
$$\langle 5 \rangle=RS$$
where $N(R)=N(S)=5$. But
$$N(2 + \sqrt{-21})=25=5^2$$
and $5 \nmid 2+\sqrt{-21}$ so
$$\langle 2+\sqrt{-21} \rangle=R^s \text{ or }S^2$$
but in either case we have
$$[R]=[S]$$
Also, consider the element $\alpha=3+\sqrt{-21}$, $N(\alpha)=30=2\cdot 3 \cdot 5$ and so
$$\langle \alpha \rangle=PQR \text{ or } PQS$$
But since $[R]=[S]$, so we have $[P][Q]=[R]$.
Hence we conclude that
$$Cl(K)=\{[P],[Q],[R]=[P][Q]=[S],[1]\} \cong C_2 \times C_2$$
and $h(K)=4$.

Now by a similar argument as in the previous question, we have
$$\langle x+\sqrt{-21} \rangle=\prod_{i=1}^m P^{a_i}_i \prod_{j=1}^n Q^{c_j}_j$$
where the conjugate of $P_i$ does not appear in the factorisation for each $i$ and $Q_j =\bar{Q}_j$.
Hence
$$\langle x^2+21 \rangle=\prod_{i=1}^m P^{a_i}_i \bar{P}^{a_i}_i \prod_{j=1}^n Q^{2c_j}_j$$
But this is equal to $\langle y \rangle^3$. So each exponent is divisible by $3$ and
$$\langle x+\sqrt{-21} \rangle=I^3$$
and
$$[1]=[\langle x+\sqrt{-21} \rangle]=[I]^3$$
But the class number is $4$. So $[I]=[1]$ and so $[I]$ is principal.
It is clear that the units in $\mathcal{O}_K$ are $\pm 1$. So set
$$x+\sqrt{-21}=(a+b\sqrt{-21})^3=a^3-63ab^2 +3a^2b\sqrt{-21}-21b^3\sqrt{-21}$$
Then $1=(3a^2-21b^2)b$. But the right hand sides is divisible by $3$, which cannot be $1$. Hence we conclude that there exists no integer solutions.\\
\item $m$ is even and so is not $1$ modulo $4$. We use a similar argument as above, so that
$$\langle x+\sqrt{-m} \rangle=I^3$$
Since the class number is coprime to $3$, so $[I]$ has to be $[1]$ and $I$ is principal.
Thus we have
$$x+\sqrt{-m}=\pm(a+b\sqrt{-m})^3$$
because the only units in $\mathcal{O}_K$ are $\pm 1$.
The case when $x+\sqrt{-m}=-(a+b\sqrt{-m})^3$ is covered in the case
when $x+\sqrt{-m}=(a+b\sqrt{-m})^3$ because we just simply take $-a$ and $-b$.
So we have
$$x+\sqrt{-m}=a^3-3ab^2m+3a^2b\sqrt{-m}-mb^3\sqrt{-m}$$
Compare the coefficients of $\sqrt{-m}$, we have
$$1=b(3a^2-mb^2)$$
and so $b= \pm 1$ and either
$$3a^2-m=1$$
or
$$3a^2-m=-1$$
But only one of these can hold. Therefore, we have at most two pairs of $(a,b)$ which satisfies
$$1=b(3a^2-mb^2)$$
and hence at most two pairs of $(x,y)$ because $x$ and $y$ are determined by $a$ and $b$. So we have
at most two integer solutions.\\
\item Since $K$ is quadratic then we have
$$\langle N(I) \rangle=I\bar{I}$$ by Proposition 15.14 and so
$$[1]=[\langle N(I) \rangle]=[I][\bar{I}]=[I]^2$$
because $I^2$ is principal. Hence $[I]=[\bar{I}]$.\\
\item For $d<-4$, we have $w(d)=2$. Then by Theorem 16.28, we have
$$h(K)=\frac{-1}{|d|}\sum_{r=1}^{|d|-1}r\left(\frac{d}{r}\right)$$
As $d$ is odd,
$$\sum_{r=1}^{|d|-1}r\left(\frac{d}{r}\right)=\sum_{1 \le r <\frac{|d|}{2}}\left(2r\left(\frac{d}{2r}\right)
+(|d|-2r)\left(\frac{d}{|d|-2r}\right)\right)$$
Now $(\frac{d}{2r})=(\frac{d}{r})(\frac{d}{2})$ and since $|d|-2r +2r \equiv 0$ (mod $|d|$), then
$$\left(\frac{d}{|d|-2r}\right)=-\left(\frac{d}{2r}\right)$$
by question 11 in exercise 4 because $sgn(d)=-1$. Thus,
$$\sum_{r=1}^{|d|-1}r\left(\frac{d}{r}\right)=\left(\frac{d}{2}\right)\sum_{1 \le r <\frac{|d|}{2}}
(4r-|d|)\left(\frac{d}{r}\right)$$
But we can also write
$$\sum_{r=1}^{|d|-1}r\left(\frac{d}{r}\right)=\sum_{1 \le r <\frac{|d|}{2}}\left(r\left(\frac{d}{r}\right)
+(|d|-r)\left(\frac{d}{|d|-r}\right)\right)$$
and similarly $(\frac{d}{|d|-r})=-(\frac{d}{r})$. Hence
$$\sum_{r=1}^{|d|-1}r\left(\frac{d}{r}\right)=\sum_{1 \le r <\frac{|d|}{2}}(2r-|d|)\left(\frac{d}{r}\right)$$
Let $X=\sum_{r=1}^{|d|-1}r(\frac{d}{r})$, then
$$X-2X\left(\frac{d}{2}\right)=\left(\frac{d}{2}\right)\sum_{1 \le r<\frac{|d|}{2}} |d|\left(\frac{d}{r}\right)$$
and so
$$X=\frac{1}{\left(\frac{d}{2}\right)-2}\sum_{1\le r <\frac{|d|}{2}}|d|\left(\frac{d}{r}\right)$$
because $(\frac{d}{2})=\frac{1}{(\frac{d}{2})}$. Hence
$$h(K)=-\frac{1}{|d|}X=\frac{1}{2-\left(\frac{d}{2}\right)}\sum_{1 \le r <\frac{|d|}{2}}\left(\frac{d}{r}\right)$$
\item Since $p \equiv 3$ (mod $4$), so $-p \equiv 1$ (mod $4$). So the discriminant $d=-p$.
Then by Dirichlet formula, we have
$$h(K)=-\frac{1}{p} \sum_{r=1}^{p-1} r\left(\frac{r}{p}\right)$$
because $(\frac{-p}{r})=(\frac{r}{p})$ by Theorem 4.18(i). Since $p$ is odd, so
$-\frac{1}{p}$ does not change the parity, and so it suffices to show
$$X=\sum_{r=1}^{p-1}r \left(\frac{r}{p}\right)$$
is odd.
But
$$X=\sum_{r=1}^{\frac{p-1}{2}} \left(r\left(\frac{r}{p}\right)+(p-r)\left(\frac{p-r}{p}\right)\right)$$
and since $p \equiv 3$ (mod $4$), so $(\frac{p-r}{p})=-(\frac{r}{p})$. So
$$X=\sum_{r=1}^{\frac{p-1}{2}}\left(2r\left(\frac{r}{p}\right)\right)+\sum_{r=1}^{\frac{p-1}{2}}
\left(p\left(\frac{r}{p}\right)\right)$$
The first term in the sum above is clearly even. Since $p$ is odd, so it suffices to show that
$$\sum_{r=1}^{\frac{p-1}{2}} \left(\frac{r}{p}\right)$$
is odd. But $(\frac{r}{p})$ is either $-1$ or $1$. Let $m$ be the number of $r$ such that $(\frac{r}{p})=1$,
and $n$ be the number of $r$ such that $(\frac{r}{p})=-1$.
Then $n+m=\frac{p-1}{2}$, which is odd because $p \equiv 3$ (mod $4$).
Hence $n-m=n+m-2m$ is also odd and so the sum
$$\sum_{r=1}^{\frac{p-1}{2}} \left(\frac{r}{p}\right)$$
is odd.
\item Consider the factorisation of $\langle 2 \rangle$. It is clear that since $p \equiv 3$ (mod $4$), so
$$\langle 2 \rangle=\langle 2,1+\sqrt{p} \rangle^2$$
and so
$$[1]=[\langle 2 \rangle]=[\langle 2,1+\sqrt{p}\rangle]^2$$
But the class number is odd, so we have
$$[1]=[\langle 2,1+\sqrt{p} \rangle]$$
and so $\langle 2,1+\sqrt{p}\rangle$ is principal.
So there exists $\alpha=a+b\sqrt{p}$ where $a,b \in \mathbb{Z}$ such that
$$\langle \alpha \rangle =\langle 2,1+\sqrt{p} \rangle$$
and so
$$N(\langle \alpha \rangle)=N(\langle 2,1+\sqrt{p} \rangle)=2$$
So $N(\alpha)=2$ or $-2$. So
$$a^2-pb^2=2 \text{ or } a^2-pb^2=-2$$
But $a^2-pb^2=2 \Rightarrow (\frac{2}{p})=1$ and $a^2-pb^2=-2 \Rightarrow (\frac{-2}{p})=1$.
Since one of the above holds, and so we conclude that there exist $a,b \in \mathbb{Z}$ such that
$$a^2-pb^2=(-1)^{\frac{p+1}{4}}2$$
\item The Minkowski bound $c_K \ge 1$, so
$$c_K=\left(\frac{4}{\pi}\right)^s \frac{n!}{n^n} \sqrt{|\mathcal{D}_K|} \ge 1$$
and so
$$|\mathcal{D}_K| \ge e^{2n-\frac{\theta}{6n}}\frac{\left(\frac{\pi}{4}\right)^{2s}}{2\pi n}$$
Hence since the exponential dominates and $s<n$ and $\frac{e\pi}{4}>1$, so the discriminant tends to infinity
as $n$ tends to infinity. 
\item The class number is $3$ so the usual trick does not work here. We need a slight modification. If
$$x^2+31=y^3$$
then
$$x^2-1 \equiv y^3~(\text{mod } 4)$$
Suppose $x$ is even then $y \equiv 3$ (mod $4$). Since $y^3-27=(y+3)(y^2+3y+9)$, and
$y^2+3y+9 \equiv 3$ (mod $4$), so there exists a prime factor $p$ of $y^2+3y+9$ with
$p \equiv 3$ (mod $4$).  Therefore,
$$x^2+4=y^3-27 \equiv 0~(\text{mod } p)$$
and so $-4$ is a square modulo $p$, which is impossible.

Now $x$ is odd and so $y$ is even. Then, since $-31 \equiv 1$ (mod $4$), so
$\mathcal{O}_K=\\mathbb{Z}[\frac{1+\sqrt{-31}}{2}]$.
Hence we consider the factorisation
$$\langle \frac{x+\sqrt{-31}}{2} \rangle \langle \frac{x-\sqrt{-31}}{2} \rangle=\langle \frac{y}{2} \rangle^3
\langle 2 \rangle$$
Since $-31 \equiv 1$ (mod $8$) and so $\langle 2 \rangle=P\bar{P}$ for some prime ideal $P$.
Let
$$\langle \frac{x+\sqrt{-31}}{2} \rangle=\prod_{i=1}^m P^{a_i}_i\bar{P}^{b_i}_i \prod_{j=1}^n Q^{c_j}_j$$
where $\bar{P_i} \neq P_i, \bar{Q_j}=Q_j$. Clearly for each $i$, either $a_i$ or $b_i=0$, otherwise we have a
prime which divides $\frac{x+\sqrt{-31}}{2}$. Assume $b_i=0$ for all $i$ and recollect those prime ideals.
Then
$$\langle \frac{x-\sqrt{-31}}{2} \rangle=\prod_{i=1}^m \bar{P}^{a_i}_i \prod_{j=1}^n Q^{c_j}_j$$
and so use
$$\langle \frac{x+\sqrt{-31}}{2} \rangle \langle \frac{x-\sqrt{-31}}{2} \rangle =P\bar{P}\langle \frac{y}{2}\rangle^3$$
we have
$$\prod_{i=1}^m P^{a_i}_i \bar{P}^{a_i}_i \prod_{j=1}^n Q^{2c_j}_j=P\bar{P} \langle \frac{y}{2}\rangle^3$$
By unique factorisation, some $P_i$ is $P$, and we may assume $P=P_1$. So we cancel $P\bar{P}$ on both sides,
we have the indices
$$3|a_1-1, 3|a_i \text{ if } i \ge 2 \text{ and } 3|c_j$$
Then we have
$$\langle \frac{x+\sqrt{-31}}{2} \rangle=P I^3 \text{ or } \bar{P} I^3$$
for some ideal $I$. But in either case, it follows that
$$[1]=[\langle \frac{x+\sqrt{-31}}{2} \rangle]=[P] [I]^3$$
Since the class number $h(K)=3$, then $[I]^3=[1]$, and so $P$ is principal. But
this is impossible because the norm of $P$ is $2$. Hence we have no integer solutions.
\item $1175=47 \cdot 5^2$. Hence we may consider the class group of $\sqrt{-47}$. We have
$n=2,r=0,s=1$ and $\mathcal{D}_K=-47$ since $-47 \equiv 1$ (mod $4$).
The Minkowksi bound
$$c_K=\frac{4}{\pi} \frac{2!}{2^2} \sqrt{|\mathcal{D}_K|}<5$$
Consider the primes $p=2,3$. $-47 \equiv 1$ (mod $8$) and so
$$\langle 2 \rangle=P\bar{P}$$ where $N(P)=N(\bar{P})=2$.
$(\frac{-47}{3})=1$, so
$$\langle 3 \rangle=Q\bar{Q}$$ where $N(Q)=N(\bar{Q})=3$.
Consider the element $\alpha=\frac{1+\sqrt{-47}}{2} \in \mathcal{O}_K$. The norm
$N(\alpha)=12=2^2 \cdot 3$. Since $2 \nmid \alpha$, so
$$\langle \alpha \rangle=P^2Q \text{ or } \bar{P}^2Q \text{ or }P^2\bar{Q} \text{ or }\bar{P}^2\bar{Q}$$
But each case gives the same result. Assume it is $P^2Q$, then we have
$$[1]=[\langle \alpha \rangle]=[P]^2[Q]$$
Hence, $[P]^2=[\bar{Q}]$.
Consider the element $\beta=\frac{9+\sqrt{-47}}{2}$, and $N(\beta)=32=2^5$.
Since $2 \nmid \beta$, then it is clear that
$$\langle \beta \rangle=P^5 \text{ or } \bar{P}^5$$
But in either case it gives that the order of $[P]$ is $5$ since $5$ is a prime number.
Then $[P]^3=[Q]$ and so the class group is
$$Cl(K)=\{[P],[P]^2=[\bar{Q}],[P]^3=[Q],[P]^4=[\bar{P}],[P]^5=[1]\} \cong C_5$$

From the equation $x^2+1175=4y^3$, it is clear that $x$ is odd. Suppose $x$ is divisible by $5$, then $y$ is divisible by $5$ and let $x=5u,y=5v$, we have
$$u^2+47=20v^3$$
and so reduce both sides modulo $4$, we have
$$u^2 \equiv 1~(\text{mod } 4)$$
which is impossible.
Hence $5 \nmid x$, and so we have
$$\langle \frac{x+5\sqrt{-47}}{2} \rangle \langle \frac{x-5\sqrt{-57}}{2}\rangle=\langle y \rangle^3$$
We may write the factorisation of $\langle \frac{5+\sqrt{-47}}{2} \rangle$ as
$$\prod_{i=1}^m P^{a_i}_i \prod_{j=1}^n Q^{c_j}_j$$
where $P_i \neq \bar{P_i}$ and $Q_j=\bar{Q}_j$. The conjugate of $P_i$ does not appear for each $i$ because if
so then $P_i\bar{P_i}=\langle p_i \rangle$ for some prime $p_i$, and so
$$p_i \big|\frac{x+5\sqrt{-47}}{2}$$
The only possible such $p_i$ is $5$, but then $5|x$, which is impossible by above. Hence
$$\langle \frac{x-5\sqrt{-47}}{2} \rangle =\prod_{i=1}^m \bar{P}^{a_i}_i \prod_{j=1}^n Q^{c_j}_j$$
and so
$$\prod_{i=1}^m P^{a_i}_i \bar{P}^{a_i}_i \prod_{j=1}^n Q^{2c_j}_j =\langle y \rangle^3$$
and so each index is divisible by $3$. Hence
$$\langle \frac{x+5\sqrt{-47}}{2} \rangle=I^3$$
for some $I$, and since the class number is $5$, which is prime to $3$, so $I$ is principal, and
$$\frac{x+5\sqrt{-47}}{2} =\pm \left(\frac{a+b\sqrt{-47}}{2}\right)^3$$
Take the plus sign we have
$$4x+20\sqrt{-47}=a^3-141ab^2+3ab^2\sqrt{-47}-47b^3\sqrt{-47}$$
Compare the coefficient of $\sqrt{-47}$, we have
$$20=b(3a^2-47b^2)$$
Reduce the above modulo $3$, we have $b^3 \equiv 2$ (mod $3$) and so
$b \equiv 2$ (mod $3$). Then check all possible such $b$ which divides $20$,
the only solution is given by
$$b=-1,a=\pm 3$$
Hence $4x=a^3-141ab^2$, and so
$$x=\pm 99,y=14$$
are the only integer solutions.
\item
\begin{enumerate}
\item[(i)] The first part is clear as $(\frac{p}{q})=(\frac{q}{p})$ so at least one of them is $1$ modulo $4$. Since $(\frac{p}{q})=1$, so there exists $u$ such that
    $$u^2 \equiv p~(\text{mod } q)$$
    If $u$ is even, then take $u+q$ so that $u+q$ is odd and
    $$(u+q)^2 \equiv p~(\text{mod } q)$$
    and
    $$(u+q)^2 \equiv 1~(\text{mod } 4) \equiv p~(\text{mod } 4$$
    Hence by Chinese Remainder Theorem, we have
    $$(u+q)^2 \equiv p~(\text{mod }4q)$$
    Further, if we have some $u$ such that
    $$u^2 \equiv p~(\text{mod }4q)$$
    then let $U=u+2qk$ for some integer $k$, then
    $$U^2 \equiv p~(\text{mod }4q)$$
    Then pick a suitable $k$ such that $p|U=u+2qk$, which is always possible because $(p,q)=1$.
    Then such $U$ have
    $$U^2 \equiv p~(\text{mod }4q) \text{ and } p|U$$
    Similarly, there exists $v$ such that
    $$v^2 \equiv q~(\text{mod } p) \text{ and } q|v$$
\item[(ii)] Consider firstly the map
$$f:\mathbb{Z}^3 \rightarrow \mathbb{Z}$$
by
$$f(x,y,z)=\bar{z}$$
where $\bar{z}= z$ (mod $2$). Hence by the first isomorphism theorem,
$$\Lambda_1=\{(x,y,z)\in \mathbb{Z}^3: Z \equiv 0~(\text{mod }2)\}$$ is
a submodule of $\mathbb{Z}^3$.
Now define the map
$$g:\Lambda_1 \rightarrow \mathbb{Z}$$
by
$$g(x,y,z)=\overline{x-uy-vz}$$
where $\overline{x-uy-vz} = x-uy-vz$ (mod $2pq$). Then by the first isomorphism,
$\Lambda$ is a submodule of $\Lambda_1$ and hence a submodule of $\mathbb{Z}^3$. Hence there exists
a $\mathbb{Z}$-basis $\{e_1,e_2,e_3\}$ for $\mathbb{Z}^3$ and $r_1,r_2,r_2 \in \mathbb{Z}$ such that
$\{r_1e_1,r_2e_2,r_3e_3\}$ is a basis for $\Lambda$. So it is a lattice.

Now use (i), we may write
$$u=p(4aq+1),v=q(bp+1)$$
for some $a,b \in \mathbb{Z}$, and let $x=uy+vz+2pqk$, then
$$x^2-py^2-qz^2=(u^2-p)y^2+(v^2-q)z^2+2uvyz+4pq(kuy+kvz+pqk^2)$$
Hence
\begin{eqnarray*}
x^2-py^2-qz^2 &\equiv& (u^2-p)y^2+(v^2-q)z^2+2uvyz \text{mod } 4pq)\\
&\equiv& 4apqy^2+bpqz^2+2pqyz(4aq+1)(bp+1)~(\text{mod }4pq)\\
&\equiv& 0~(\text{mod }4pq)
\end{eqnarray*}
because $z$ is even and so each term is $0$ modulo $4pq$.\\
\item[(iii)] Let $x=2\sqrt{pq}r_1,y=2\sqrt{p}r_2,z=2\sqrt{p}r_3$. Then $dxdydz=8pqdr_1dr_2dr_3$, where
$$r^2_1+r^2_2+r^2_3<1$$
Hence the volume of the ellipsoid is
$$V(X)=\frac{32\pi pq}{3}=2^3 \frac{4\pi pq}{3}$$
Consider the covolume of $\Lambda$. Let $\{e_1,e_2,e_3\}$ be a basis for $\mathbb{Z}^3$ and $r_1,r_2,r_3 \in\mathbb{Z}$ such that
$\{r_1e_1,r_2e_2,r_3e_3\}$ is a basis for $\Lambda$. Then the covolume is the determinant of the matrix,
whose rows are $r_1e_1,r_2e_2,r_3e_3$, which is $|r_1r_2r_3|d$, where $d$ is the determinant of the
matrix whose rows are $e_1,e_2,e_3$ and hence $d=1$.
Now $|r_1r_2r_3|$ is the index of $\Lambda$ in $\mathbb{Z}^3$. The index of $\Lambda_1$ in $\mathbb{Z}^3$ is clearly $2$. Now the map $g$ defined in (ii) is surjective, (simply let $y,z=0$) and so the index of
$\Lambda$ in $\Lambda_1$ is $2pq$ hence the index $|r_1r_2r_3|=2pq \cdot 2=4pq$. Hence, the covolume
$$cov(\Lambda)=4pq$$
Then since $\frac{32\pi pq}{3} >2^3 \cdot 4pq$, by Minkowski's convex body theorem (the second case in the theorem as the $X$ in the question is not closed), there exists $(x,y,z) \in X \cap \Lambda$.
We have
$$x^2-py^2-qz^2 \equiv 0~(\text{mod }4pq)$$
Suppose $$x^2-py^2-qz^2 \ge 4pq$$ then $x^2+py^2+qz^2 \ge 4pq$, contradicting $(x,y,z) \in X$.
Suppose $$x^2-py^2-qz^2 \le -4pq \iff -x^2+py^2+qz^2 \ge 4pq$$
then $x^2+py^2+qz^2 \ge 4pq$, again contradicting $(x,y,z) \in X$.
Therefore,
$$-4pq<x^2-py^2-qz^2<4pq \text{ and } x^2-py^2-qz^2 \equiv 0~(\text{mod } 4pq)$$
so
$$x^2-py^2-qz^2=0$$
\end{enumerate}
\end{enumerate}
\subsection{Exercise 17}
\begin{enumerate}
\item The fundamental unit in a quadratic field can be found by using the continued fraction because we have proved that any solution to
    $$x^2-my^2=\pm 1$$
    is the convergent $x=p_n,y=q_n$ for the continued fraction of $\sqrt{m}$. Thus, the first such pair of convergent gives the smallest solution, and so is the fundamental unit. Thus, $8+3\sqrt{7}$ is the fundamental unit.

    It is a simple matter to check that the class group is a trivial group. Thus, we have
    $$\langle x+\sqrt{7}y \rangle \langle x-\sqrt{7}y \rangle=\langle 2 \rangle=P\bar{P}$$
    where we use Kummer-Dedekind to conclude $P=\langle 2,1+\sqrt{7} \rangle=\langle 3+\sqrt{7} \rangle$.
    So
    $$\langle x+\sqrt{7}y \rangle=P \text{ or }P\bar{P}$$
    and so
    $$x+\sqrt{7}y=u(3\pm \sqrt{7})$$
    Now by Dirichlet's theorem, $u=\pm (8+3\sqrt{7})^n$ and so the solutions are given by
    $$x+\sqrt{-7}y=\pm (8+3\sqrt{7})^n (3 \pm \sqrt{7})$$
\item It is clear that $\alpha=5+\sqrt{26}$ is a fundamental unit. By Kummer-Dedekind, we have
$$\langle 2 \rangle=\langle 2,\sqrt{26}\rangle^2=P^2, \langle 5 \rangle=\langle 5,1+\sqrt{26}\rangle \langle 5,1-\sqrt{-26}\rangle=Q\bar{Q}$$
Hence, from
$$\langle x+\sqrt{26} \rangle \langle x-\sqrt{26} \rangle=\langle 2 \rangle \langle 5 \rangle$$
we deduce that
$$\langle x+\sqrt{-26} \rangle=PQ \text{ or }P\bar{Q}$$
It is easy to check that
$$PQ=\langle \alpha+1 \rangle \text{ and } P\bar{Q}=\langle \alpha-1 \rangle$$
Hence the solutions are given by
$$x+\sqrt{26}=\pm \alpha^n(\alpha \pm 1)$$
\item \begin{enumerate}
\item[(i)] Let
\begin{eqnarray*}
f(x)&=&x^2+4-\sin^2{\theta}(x-2\cos{\theta})^2\\
=x^2\cos^2{\theta}+4\sin^2{\theta}\cos{\theta}x-4\cos^2{\theta}\sin^2{\theta}+4
\end{eqnarray*}
If $\cos{\theta}=0$, then $f(x)=4>0$. If not, then the at minimum point, $x=2\frac{\sin^2{\theta}}{\cos{\theta}}$. We have the minimum,
$$4-4\cos^2{\theta}\sin^2{\theta}-4\sin^4{\theta}=4\cos^2{\theta}>0$$
Hence $f(x)>0$ for all $x \in \mathbb{R}$ and $\theta \in \mathbb{R}$.\\
\item[(ii)] The discriminant is
\begin{equation} \begin{vmatrix} 1&u&u^2\\1&re^{i\theta}&r^2e^{2i\theta}\\1&ue^{-i\theta}&ue^{-2i\theta}
\end{vmatrix}^2 \end{equation}
Also the norm of $u$ is $\pm 1$, and so
$$ur^2=\pm 1$$
Since $u$ and $r>0$, os $u=\frac{1}{r^2}$.
Hence, we conclude that the discriminant is
$$-4\sin^2{\theta}(r^3+r^{-3}-2\cos{\theta})^2$$
Now by (i), we have
$$4\sin^2{\theta}(r^3+r^{-3}-2\cos{\theta})^2<4((r^3+r^{-3})^2+4)=4(u^3+u^{-3}+6)$$
\item[(iii)] Now as $u \in \mathcal{O}_K$, so
$$|\mathcal{D}_K| \le |D(\mathbb{Z}[u])| <4(u^3+u^{-3}+6)$$
Hence,
$$u^6-\left(\frac{|\mathcal{D}_K|}{4}-6\right)u^3+1>0$$
Now complete the square, we have
$$\left(u^3-\left(\frac{|\mathcal{D}_K|}{8}-3\right)\right)^2>\left(\frac{|\mathcal{D}_K|}{8}-3\right)^2-1$$
Now, as $|\mathcal{D}_K|>32$, so $\frac{|\mathcal{D}_K|}{8}-3>1$, and so
$$\left(\frac{|\mathcal{D}_K|}{8}-3\right)^2 >\left(\frac{\mathcal{D}_K}{8}-\frac{15}{4}\right)^2$$
Therefore, we have
$$\left|u^3-\left(\frac{|\mathcal{D}_K|}{8}-3\right)\right|>\frac{|\mathcal{D}_K|}{8}-\frac{15}{4}$$
If $$u^3-\left(\frac{|\mathcal{D}_K|}{8}-3\right)<0$$ then we end up with $u^3<\frac{15}{4}-3<1$, which is a contradiction.
Hence $$u^3-\left(\frac{|\mathcal{D}_K|}{8}-3\right) \ge 0$$ and so
$$u^3>\frac{|\mathcal{D}_K|-27}{4}$$
\end{enumerate}
\item We have seen in the previous exercise that the ring of integers $\mathcal{O}_K=\mathbb{Z}[\theta]$, and the 
discriminant is $148>0$. Hence all roots of $x^3-4x+2$ are real and so $r=3,s=0$ and $r+s-1=2$.
Thus, we may apply Kummer-Dedekind to factorises the prime numbers in $\mathcal{O}_K$.
It is clear that 
$$\langle 2\rangle=\langle \theta \rangle^3$$
$$\langle 3 \rangle \text{ is prime}$$
$$\langle 5 \rangle=P_5Q_5 \text{ where } N(P_5)=5,N(Q_5)=25$$
$$\langle 7 \rangle \text{ is prime}$$
$$\langle 11 \rangle \text{ is prime}$$
$$\langle 13 \rangle=P_{13}Q_{13} \text{ where } N(P_{13})=13,N(Q_{13})=169$$
Now by question 16 of exercise 14, we have
$$\zeta_K(s)=\prod_{P} \frac{1}{1-(N(P))^{-s}}$$
where the product is over all prime ideals. 
The first few terms of the $\zeta$-function gives
$$\frac{1}{1-2^{-s}}\frac{1}{1-3^{-s}}\frac{1}{1-5^{-s}}\cdots$$
and the first few terms of the Dedekind $\zeta$-function is, by above
$$\frac{1}{1-2^{-s}}\frac{1}{1-27^{-s}}\frac{1}{1-5^{-s}}\frac{1}{1-25^{-s}}\cdots$$
Since the limit $(s-1)\zeta(s)$ is $1$ when $s \to 1^+$ and so the limit
$(s-1)\zeta_K(s)$ is the quotient of the Dedekind $\zeta$-function to the (usual) $\zeta$-function, and so
it is roughly:
$$\frac{1-3^{-1}}{1-27^{-1}}\frac{1}{1-25^{-1}}\frac{1-7^{-1}}{1-343^{-1}}\frac{1-11^{-1}}{1-1331^{-1}}
\frac{1}{1-169^{-1}}$$
which is roughly about $0.56736$. This gives an estimate of the limit $(s-1)\zeta_K(s)$ because
the later terms contain large denominator and so the quotient is very small. 

Now let the matrix $E$ be: 
\begin{equation*} E=\begin{pmatrix} \log{|\theta-1|}&\log{|\theta'-1|}\\ \log{|2\theta-1|}&\log{|2\theta'-1|}
\end{pmatrix} \end{equation*}
Where $\theta'$ is another root of $\theta$. Let $\theta$ be any root. Take one of the root, say 
$-2.2143$ and let $\theta'=1.6751$ (worked out by computer). Then
$$|\det{E}|=1.6624$$
Now use the approximation of the limit 
$$\lim_{s \to 1^+}(s-1)\zeta_K(s)=0.56736$$
and the class number is $2$, the number of roots of unity is $2$, we then estimate the regulator by the
class number formula, which gives
$$R(K)=\frac{0.56736 \cdot 2 \cdot \sqrt{148}}{8}=1.7255$$
Now if $\{\theta-1,2\theta-1\}$ is not a fundamental system of units, then the determinant of $E$ is at least
twice the regulator, which is impossible. Hence we conclude that $\{\theta-1,2\theta-1\}$ is a 
fundamental system of units.
\item Write $\alpha=\sqrt[3]{2}$. The usual trick of these question is to use the result of question $3$ and show that $1<1+\alpha+\alpha^2<u^2$ where $u$ is the fundamental unit and hence conclude that $u=1+\alpha+\alpha^2$.
    It is clear that
    $$(1+\alpha+\alpha^2)(\alpha-1)=1$$
    so that $1+\alpha+\alpha^2$ is a unit. Also, the discriminant is $-108$ and so we have one real roots and two non-real roots. Thus, $r+s-1=1$. By question $3$, we have
    $$u^3>\frac{108-27}{4}=\frac{81}{4}>20$$
    and so $u^2>20^{\frac{2}{3}}$. Since
    $$1<v=1+\alpha+\alpha^2<1+2+4=7<20^{\frac{2}{3}}$$
    Hence
    $$1<v<u^2$$
    But $u$ is the fundamental unit, and so $v=u^n$. Then $n=1$ and so $u=v=1+\alpha+\alpha^2$.
\item Let
    $$k=\prod_{i=1}^n p^{e_i}_i \prod_{j=1}^m q_j$$
    where $p_i,q_j$ are distinct prime numbers and $e_i \ge 2$. Then
    $$\phi(k)=\prod_{i=1}^n p^{e_i-1}_i(p_i-1) \prod_{j=1}^m (q_j-1)$$
    and so
    $$\frac{2(\phi(k))^2}{k}=\left(\prod_{i=1}^n p^{2e_i-3}(p_i-1)\right)\left(\prod_{j=1}^m 2\frac{(q_j-1)^2}{q_j}\right)=AB$$
    It is clear that since $e_i \ge 2$, so $A \ge 1$. For $B$, if $q_j \ge 2$, then
    $\frac{(q_j-1)^2}{q_j}\ge 1$, and if $q_j=1$, then $2\frac{(q_j-1)^2}{q_j} \ge 1$. Hence
    $$2\phi(k)^2 \ge k$$
    and since $\phi(k) \le n$, so $k \le 2n^2$.

    Let $K$ be a number field with $[K:\mathbb{Q}]=n$. If $\zeta_k \in \mathcal{O}_K$, then
    $\mathbb{Q}(\zeta_k) \subseteq K$ and so
    $$\phi(k)=[\mathbb{Q}(\zeta_k):\mathbb{Q}] \le [K:\mathbb{Q}]=n$$
    and so $k \le 2n^2$. So it contains at most $\zeta_1,\ldots,\zeta_{2n^2}$.\\
\item  Let $K$ be a number field with $[K:\mathbb{Q}]=n$ where $n$ is odd. If $\zeta_k \in \mathcal{O}_K$,
    then $\mathbb{Q}(\zeta_k) \subseteq K$ and so by Tower Law:
    $$n=[K:\mathbb{Q}]=[K:\mathbb{Q}(\zeta_k)][\mathbb{Q}(\zeta_k):\mathbb{Q}]=[K:\mathbb{Q}(\zeta_k)]\phi(k)$$
    and so $\phi(k)|n$. If $k \ge 3$, let
    $$k=\prod_{i=1}^n p^{e_i}_i$$ then
    $$\phi(k)=\prod_{i=1}^n p^{e_i-1}_i(p_i-1)$$
    It is clear that when $k \ge 3$, $\phi(k)$ is even. Hence $\phi(k)|n$ can hold if and only if $k=2$ because
    $n$ is odd and when $k=2$, $\zeta_2=-1$.
\item 
\begin{enumerate}
\item[(i)] $\zeta_{12}$ satisfies $x^12-1=0$ and so it lies in $\mathcal{O}_K$ if it lies in $K$. Also
$$\zeta_{12}=e^{\frac{\pi i}{6}}=\frac{1}{2}+\frac{i\sqrt{3}}{2} \in K$$
\item[(ii)] If $\zeta_k \in K$, then $\mathbb{Q}(\zeta_k) \subseteq K$ and so
$\phi(k)|4$ by Tower Law. If $\phi(k)=1$, then $k=2$.
If $\phi(k)=2$, then let $k=4,6$. If $\phi(k)=4$, then let
$$k=\prod_{i=1}^m p^{e_i}_i$$
and so
$$4=\prod_{i=1}^m p^{e_i-1}_i(p_i-1)$$
$p_i-1|4$ so $p_i=2,3,5$. Also we cannot have three distinct prime factors, so we conclude that
$k=5,8,10,12$ and so the largest one is $12$.
\end{enumerate}
\item It is easy to check they are independent (either by direct argument or calculate the discriminant of the 
matrix which consists the logarithm of the absolute value of the conjugates). So for each of them, in this case, it
might be easier to check that for each of them, it is not a positive power of some other units in $\mathcal{O}_K$.
For example, if $\sqrt{2}+\sqrt{3}=u^n$ for some $n \ge 2$ and some unit $u$. Then let
$$\alpha^n=(a+b\sqrt{2}+c\sqrt{3}+d\sqrt{6})^n=\sqrt{2}+\sqrt{3}$$
where $a,b,c,d \in \mathbb{Q}$. Since the minimal polynomial of $\sqrt{2}+3$ is 
$$x^4-10x^2+1=0$$
and so $\alpha$ satisfies the equation
$$x^{4n}-10x^{2n}+1=0$$
\item Let $f(x)=x^3-x^2+x-2$. Then $f(1.3)<0$ and $f(1.4)>0$ so $1.3 < \theta <1.4$. Moreover,
let $y=x-\frac{1}{3}$ and so $f(y)=y^3+\frac{2}{3}y-\frac{47}{27}$. So the discriminant is
$$disc(f)=-4(\frac{2}{3})^3-27(\frac{47}{27})^2=-83<0$$
so it has only one real root. Then $r+s-1=1$ and so we have a fundamental unit. Since $\theta^3-\theta^2+\theta-2=0$, so we have
$$(1+\theta^2)(\theta-1)=1$$
and so $1+\theta^2$ is a unit.
By question $3$, if $u$ is the fundamental unit, then
$$u^3>\frac{83-27}{4}=14$$
and so $u^2>14^{\frac{2}{3}}$.
It is clear that $1+\theta^2<5$ and $5^3<14^2$, so
$$1<1+\theta^2<u^2$$
and hence $1+\theta^2=u$.
\item Consider the norm of $\alpha$, $N_{K/\mathbb{Q}}(\alpha)$. Let $f(x)$ be the minimal polynomial of $\alpha$.
If $\alpha_k=a+bi$ is a non-real conjugate of $\alpha$ then $a-bi$ is also a conjugate of $\alpha$ because it is a root of $f(x)$ and hence
$(a+bi)(a-bi)=a^2+b^2>0$. Therefore, if $n=r+2s$, then
$$N(\alpha)=\prod_{i=1}^r \alpha_i \prod_{j=1}^s |\alpha_j|^2$$
It is clear that each $|\alpha_j|^2>0$, and so we focus on the first term. Suppose we have no real conjugate, then
we are done. Suppose we have one, say $\beta$, then let $L=\mathbb{Q}(\zeta_k)$ where $\zeta_k \in K$.
Let $g(x)$ be the minimal polynomial of $\beta$ in $L$.
If the coefficients of $g(x)$ are not real.
Let $h(x)=g(x)-\overline{g(x)}$, then $h(\beta)=0$ but $h(x) \not \equiv 0$. Hence $\beta$ satisfies a
polynomial of smaller degree, which is a contradiction. Hence $g(x)$ is a real polynomial, and in particular
the constant term is real. Hence
$$a=N_{K/L}(\beta) \in \mathbb{R}$$
Now
$$N_{K/\mathbb{Q}}(\beta)=N_{L/\mathbb{Q}} N_{K/L}(\beta)$$
Let $\sigma$ be an element in $Aut_{\mathbb{Q}(L)}$. Since $\sigma$ send $\zeta_k$ to $\zeta^j_k$ for some $j$
which is prime to $k$, so $\sigma$ sends $\zeta^{-1}_k$ to $\zeta^{-j}_k$. This implies that
$\sigma$ fixes every real number in $L$ (in fact, we can see from this argument that $a$ must already be an integer.) and hence
$$N_{L/\mathbb{Q}}(a)=a^{\phi(k)}$$
But $\phi(k)$ is even for any $k \ge 3$ and since $N(\alpha)=N(\beta)$, we have
$$N(\alpha)=N(\beta)=a^{\phi(k)}>0$$
\item Suppose $y$ is even, then consider the congruence modulo $8$, we have
$$x^2 \equiv 5~(\text{mod } 8)$$
which is impossible. Hence $y$ is odd and $x$ is even. Then consider the congruence modulo $4$,
we have
$$x^2 \equiv 1+y^3~(\text{mod } 4)$$
and so
$$y \equiv 3~(\text{mod } 4)$$
We will treat the cases $y \equiv 3$ (mod $8$) and $y \equiv 3$ (mod $8$) separately.
If $y \equiv 3$ (mod $8$), then
$$x^2-72=y^3-27=(y-3)(y^2+3y+9)$$
It is clear that $y^2+3y+9 \equiv 3$ (mod $8$) and so there exists a prime factor $p$ of $y^2+3y+9$ such that
$$p \equiv \pm 3~(\text{mod }8)$$
Hence
$$x^2 \equiv 72~(\text{mod } p)$$
and so
$$\left(\frac{72}{p}\right)=\left(\frac{2}{p}\right)=1$$
which is impossible.

If $y \equiv 7$ (mod $8$). Then
$$x^2-18=y^3+27=(y+3)(y^2-3y+9)$$
Then $y^2-3y+9 \equiv -3$ (mod $8$), and so there exists a prime factor $p$ of $y^2-3y+9$ such that
$$p \equiv \pm 3~(\text{mod } 8)$$
Hence
$$x^2 \equiv 18~(\text{mod }p)$$
and so
$$\left(\frac{18}{p}\right)=\left(\frac{2}{p}\right)=1$$
which is impossible. Hence we have no integer solutions.\\
\item If $x$ is even then $y$ is even, and so $4|x,y$, but $4 \nmid 46$ which is a contradiction.
Hence $x$ is odd and $y$ is odd. Consider the congruence modulo $8$, we have
$y \equiv 3$ (mod $8$). Then we write
$$x^2+18=y^3+64=(y+4)(y^2-4y+16)$$
Then (mod $8$) and $y^2-4y+16 \equiv -3$ (mod $8$). We have at least one of the prime factor $p$ which is
$5$ or $7$ (mod $8$). Then
$$x^2+18 \equiv 0~(\text{mod } p)$$
and so
$$\left(\frac{-18}{p}\right)=\left(\frac{-2}{p}\right)=1$$
which is impossible. Hence it has no integer solutions.

We have
$$x^2=y^3+m^3-2n^2$$
If $x$ is even, then $y$ is even and so $4|x^2,y^3,m^3$, but $4 \nmid 2n^2$ because $n$ is odd.
Hence $x$ and $y$ are both odd. Since $x^2 \equiv 1$ (mod $8$), $8|m^3$ and then reduce the above modulo $8$,
and $2n^2 \equiv 2$ (mod $8$), so we conclude that
$$y \equiv y^3 \equiv 3~(\text{mod } 8)$$
Now write
$$x^2+2n^2=y^3+m^3=(y+m)(y^2-my+m^2)$$
and $y^2-my+m^2 \equiv -3$ (mod $8$), so we have some prime factor $p$ which is $5$ or $7$ (mod $8$). Then
$$x^2+2n^2 \equiv 0~(\text{mod } p)$$
Since any factor dividing $n$ is $1$ or $3$ mod $8$, so $p \nmid n$ and so
$$\left(\frac{-2n^2}{p}\right)=\left(\frac{-2}{p}\right)=1$$
which is impossible. Hence we have no integer solutions.
\item \begin{enumerate}
\item[(i)] Let $(\frac{3}{p})=1$, then if $p \equiv 1$ (mod $4$), then
$(\frac{p}{3})=1$ and so $p \equiv 1$ mod $3$, which gives
$$p \equiv 1~(\text{mod } 12)$$
If $p \equiv -1$ (mod $4$), then
$(\frac{p}{3})=-1$ and so $p \equiv -1$ mod $3$, which gives
$$p \equiv -1~(\text{mod } 12)$$
Hence,
$$p \equiv \pm 1~(\text{mod }12)$$
\item[(ii)] If $y$ is even, then $x$ is odd, since $m^3 \equiv 3$ (mod $4$), and $n^2 \equiv 0$ (mod $4$), then
consider the congruence modulo $4$, we have
$$1 \equiv 3~(\text{mod } 4)$$
which is impossible. Hence $y$ is odd and $x$ is even. By considering the congruence modulo $4$, we have
$y \equiv 1$ (mod $4$). Then write
$$x^2=(y+m)(y^2-my+m^2)+3n^2$$
It is clear that $y^2-my+m^2 \equiv 3$ (mod $4$). Now since $y^2-my+m^2$ is symmetric in $y,m$ we consider
$$(y,m) \equiv (0,0),(1,0),(1,1),(2,0),(2,1),(2,2)~(\text{mod } 3)$$
and then we have $y^2-my+m^2 \equiv 0$ or $1$ (mod $3$). But when $y^2-my+m^2 \equiv 0$ (mod $3$),
we have $(y,m)=(0,0),(2,1)$ which gives $3|y+m$, and so
$$9|y^3+m^3$$
This implies that $9|x^2-3n^2$ and hence $3|x$ and so $9|x^2$. But
the only prime factors of $n$ are $\pm 1$ (mod $12$) and so $3 \nmid n$. Therefore,
$9\nmid x^2-3n^2$, which is a contradiction. Hence we conclude that
$$y^2-my+m^2 \equiv 1~(\text{mod } 3)$$
and so
$$y^2-my+m^2 \equiv -5~(\text{mod }12)$$
This shows that we must have some prime factor $p$ of $y^2-my+m^2$ such that
$$p \equiv \pm 5~(\text{mod }12)$$
Hence, $p \nmid n$, and
$$x^2 \equiv 3n^2~(\text{mod }p)$$
and so
$$\left(\frac{3n^2}{p}\right)=\left(\frac{3}{p}\right)=1$$
which is impossible by (i). Therefore we have no integer solutions.
\end{enumerate}
\item We have $n=2,r=2,s=0,\mathcal{D}_K=28$ and so the Minkowski bound
$$c_K=\sqrt{7}<3$$
Since $2|28$ so $2$ ramifies and so
$$\langle 2 \rangle=P^2=\langle 2,\sqrt{7}\rangle^2$$
So the class number is either $1$ or $2$, which is prime to $3$. Hence, by the usual argument, we have
$$x+\sqrt{7}=\pm u^n(a+b\sqrt{7})^3$$
where $u$ is the fundamental unit and $n \in \mathbb{Z}$. It is clear that $u=8+3\sqrt{7}$. Now, since $(-1)=(-1)^3$, and
we absorb the cubic factors in to $(a+b\sqrt{7})^3$, so we have
$$x+\sqrt{7}=u^n(a+b\sqrt{7})^3$$
where $u=8+3\sqrt{7}$ and $n=0,\pm 1$.

When $n=0$, compare the coefficient of $\sqrt{7}$, we have
$$1=b(3a^2+7b^2)$$
which is impossible. So we have no integer solutions when $n=0$.

When $n=\pm 1$, compare the coefficients of $\sqrt{7}$, we have
$$1=8b(3a^2+7b^2)\pm 3(a^3+21b^2)$$
Consider the congruence modulo $3$, we have
$$1 \equiv 2b^3~(\text{mod }3)$$
and so
$$b \equiv -1~(\text{mod }3)$$
Then
$$b^3 \equiv -1~(\text{mod }9)$$
Thus, consider the congruence modulo $9$, we have
$$1 \equiv 3a^2+7 \pm 3a^3~(\text{mod } 9)$$
If $3|a$, then $9|a^2,3a^3$, which then implies that $0 \equiv 6$ (mod $9$), and this is impossible.
So $a^2 \equiv 1$ (mod $3$) and so $3a^2 \equiv 3$ (mod $9$) and $3a^3 \equiv 3a$ (mod $9$), so we have
$$1 \equiv 1 \pm 3a~(\text{mod } 9)$$
which is impossible.
\item We have $n=r=2,s=0, \mathcal{D}_K=24$ and so the Minkowski bound
$$c_K=\frac{1}{2}\sqrt{24}=\sqrt{6}<3$$
Since $2|24$ and so $2$ ramifies. Hence the class number is either $1$ or $2$, which is prime to $3$.
The fundamental unit is $u=5+2\sqrt{6}$ and so we may write
$$x+\sqrt{6}=u^n (a+b\sqrt{6})^3$$
where $n=0,\pm 1$ and $a,b \in \mathbb{Z}$.
When $n=0$, compare the coefficient of $\sqrt{6}$, we have
$$1=b(3a^2+6b^2)$$
which has no integer solution as $3|3a^2+6b^2$ but $3 \nmid 1$.

When $n=\pm 1$, compare the coefficient of $\sqrt{6}$, we have
$$1=5b(3a^2+6b^2) \pm 2(a^3+18ab^2)$$
Consider the congruence modulo $3$, it is clear that $3 \nmid a$. So $a^2 \equiv 1$ (mod $3$) and
$3a^2 \equiv 3$ (mod $9$). Also, if $3|b$, then consider the congruence modulo $9$, we have
$$1=\pm 2a^3~(\text{mod } 9)$$
But $a^3 \equiv 0,1,8$ (mod $9$) and so $\pm 2a^3$ can only be $0,2,7$ modulo $9$, which is a contradiction.
So $3 \nmid b$ and so $3b^2 \equiv 3$ (mod $9$). Therefore, reduce the above modulo $9$, we have
$$1 \equiv 6b+3b \pm 2a^3 \equiv \pm 2a^3~(\text{ mod} 9)$$
which again is a contradiction. Therefore, the equation has no integer solutions.
\item
\begin{enumerate}
\item[(i)] Suppose $2|Y$, then $2|y^3+54$ and so $2|y$. Consider the congruence modulo $8$, since $8|y^3$,
and $54 \equiv 6$ (mod $8$), we have
    $$x^2+3 \equiv 6~(\text{mod }8) \Rightarrow x^2 \equiv 3~(\text{mod }8)$$
    which is impossible.
\item[(ii)] Suppose $3|Y$, then $3|y^3+54$ and so $3|y$. Then $27|y^3+54$ so $27|X=x^2+3$, which is impossible.
\item[(iii)] Now pick any prime number $p>3$. Suppose $p|Y=y^3+54$, then $p|x^2+3$ and so
$$\left(\frac{-3}{p}\right)=1$$
because $p>3$. This implies that
$$p \equiv 1~(\text{mod }3)$$
Then $X=Y=y^3+54=x^2+3$ is a product of prime numbers which are congruent to $1$ modulo $3$. Then there exists
$t$ such that
$$3t \equiv 1~(\text{mod } p)$$
Further, since $X=Y$ and so $p|x^2+3$, therefore, multiply both sides by $t^3$ we have
$$t^3y^3+54t^3=x^2t^3+3t^3$$
and since $54=2 \cdot 3^3$, reduce the above modulo $p$, we have
$$(-ty)^3 \equiv 2~(\text{mod }p)$$
and so by the hint we have $u$ and $v$ such that
$$p=u^2+27v^2$$
It is also clear that
$$(u^2_1+27v^2_1)(u^2_2+27v^2_2)=(u_1u_2-27v_1v_2)^2+27(u_1v_2+u_2v_1)^2$$
Hence as $X,Y$ are products of such $p$, we conclude that there exists integers $u$ and $v$ such that
$$X=Y=u^2+27v^2$$
Let $\omega$ be the third root of unit. Then since $\mathbb{Z}[\omega]$ is a unique factorisation domain, and
$x^2+3=u^2+27v^2$, we have
$$\pm(x+\sqrt{-3})\omega^n=u+3v\sqrt{-3}$$
where $n=0,1,-1$ as $\omega,\omega^{-1},1$ are the only units in $\mathbb{Z}[\omega]$.
(By Dirichlet's unit theorem, $r=0,s=1$ and so the only units in $\mathbb{Z}[\omega]$ are the roots of unity.)
If $n=0$, then $3v=1$ which is impossible. If $n=\pm 1$, then $\omega=\frac{-1+i\sqrt{3}}{2}$ and so
we have
$$\pm \frac{1}{2}\left(-x \mp 3 + (\pm x-1)\sqrt{-3}\right)$$
Then compare the coefficient of $\sqrt{-3}$, we conclude that $\pm x-1$ is even and so $x+1$ is even.
Then $x$ is odd and so $X=x^2+3$ is even. But $X=Y$, so $Y$ is even, which is impossible by the first part.
Therefore, we have no prime factors $p$ of $Y$ such that $p>3$.

Hence, combine all three parts, we conclude that we have no prime factors of $Y=y^3+54$ and so
$Y=\pm 1$, but this is clearly impossible. Therefore, we have no integer solutions.
\end{enumerate}
\end{enumerate}
