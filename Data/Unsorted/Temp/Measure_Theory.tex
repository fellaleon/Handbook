
\subsection{Bernoulli shifts}

%Let $m$ be a probability measure on $\R$. In previous section, we constructed a probability space $(\Omega, \sF, \pro)$ on which there exists a sequence of independent random variables $(Y_n : n \in \N)$, all having distribution $m$. 

Consider now the infinite product space
\be
E = \R^\N = \{x = (x_n : n \in \N) : x_n \in \R \text{ for all }n\}
\ee
and the $\sigma$-algebra $\sE$ on $E$ generated by the coordinate maps $X_n(x) = x_n$
\be
\sE = \sigma(X_n : n \in \N).
\ee


Note that $\sE$ is also generated by the $\pi$-system
\be
\sA = \left\{ \prod^n_{n \in\N} A_n : A_n \in \sB \text{ for all }n,\ A_n = \R \text{ for sufficiently large }n\right\}.
\ee

Define $Y : \Omega \to E$ by $Y (\omega) = (Y_n(\omega) : n \in \N)$. Then $Y$ is measurable and the image measure $\mu = P \circ Y^{-1}$ satisfies, for $A = \prod_{n\in\N} A_n \in \sA$,
\be
\mu(A) = \prod_{n\in \N} m(A_n).
\ee

By uniqueness of extension, $\mu$ is the unique measure on $\sE$ having this property. Note that, under the probability measure $\mu$, the coordinate maps $(X_n : n \in \N)$ are themselves a sequence of independent random variables with law $m$. The probability space $(E, \sE, \mu)$ is called the canonical model for such sequences. Define the shift map $\theta : E \to E$ by
\be
\theta(x_1, x_2, \dots ) = (x_2, x_3, \dots ).
\ee

\begin{theorem}\label{thm:measure_preserving}
The shift map is an ergodic measure-preserving transformation.
\end{theorem}
\begin{proof}[\bf Proof]
The details of showing that $\theta$ is measurable and measure-preserving are left as an exercise. To see that $\theta$ is ergodic, we recall the definition of the tail $\sigma$-algebras 
\be
\sT_n = \sigma(X_m : m \geq n + 1), \quad\quad \sT = \bigcap_n \sT_n.
\ee

For $A =\prod_{n\in\N} A_n \in \sA$ we have
\be
\theta^{-n}(A) = \{X_{n+k} \in A_k \text{ for all }k\} \in \sT_n.
\ee

Since $\sT_n$ is a $\sigma$-algebra, it follows that $\theta^{-n}(A) \in \sT_n$ for all $A \in \sE$, so $\sE_\theta \subseteq \sT$. Hence $\theta$ is ergodic by Kolmogorov's zero-one law.
\end{proof}
