\chapter{Linear Analysis}

\section{Normed Space}

\subsection{Normed space}

\begin{definition}[normed space\index{normed space}]\label{def:normed_space}
A normed (vector) space over field $\F$\footnote{Usually, we take $\F = \R,\C$} is a pair $(X,\dabs{\cdot})$ where $X$ is a vector space over $\F$ and $\dabs{\cdot} : X\to [0,\infty), x\mapsto \dabs{x}$ such
that:
\ben
\item [(i)] $\dabs{x}\geq 0,\forall x\in X$.
\item [(ii)] $\dabs{x} = 0 \ \lra\ x = 0$.
\item [(iii)] $\dabs{\lm x} = \abs{\lm}\dabs{x}$, $\forall \lm \in \F$, $\forall x\in X$.
\item [(iv)] $\dabs{x+y} \leq \dabs{x} + \dabs{y}$, $\forall x,y\in X$.
\een

$\dabs{\cdot}$ is called a norm\index{norm} on $X$ (see Definition \ref{def:norm_vector_space}).
\end{definition}



\begin{remark}\label{rem:normed_space_is_metric_space}
Note that every normed space is a metric space. Recalling the definition of metric space (Definition \ref{def:metric_space}), we can define a metric $d(x,y) = \dabs{x-y}$ on $X$.

Conversely, if $X$ is a vector space over $\F$ and $d$ is a metric on $X$ with
\be
d(\lm x, \lm y) = \abs{\lm} d(x,y),\qquad d(x+z,y+z) = d(x,y),\qquad \forall \lm \in \F,x,y,z\in X,
\ee
then $\dabs{x} = d(x,0)$ is a norm on $X$. So metric space is not normed space in general case.
\end{remark}


\begin{proposition}
For normed vector space $(X,\dabs{\cdot})$, we have for $x,y\in X$,
\be
\abs{\dabs{x}-\dabs{y}} \leq \dabs{x-y}.
\ee
\end{proposition}

\begin{remark}
For metric space, the triangle inequality is
\be
\abs{d(y,x) - d(x,z)} \leq d(y,z).
\ee
\end{remark}

\begin{proof}[\bf Proof]
First, by the triangle inequality we have
\be
\dabs{x} = \dabs{(x-y)+y} \leq \dabs{x-y} + \dabs{y} \ \ra\ \dabs{x} -\dabs{y} \leq \dabs{x-y}.
\ee

Then switch $x$ and $y$ we have
\be
\dabs{y} - \dabs{x} \leq \dabs{x-y} \ \ra\ \dabs{x} -\dabs{y} \geq -\dabs{x-y}.
\ee

Therefore,
\be
-\dabs{x-y} \leq \dabs{x} -\dabs{y} \leq \dabs{x-y} \ \ra\ \abs{\dabs{x} -\dabs{y}} \leq \dabs{x-y}.
\ee
\end{proof}


\subsection{Typical spaces}

Let us give a list of examples of normed vector spaces.

\begin{example}
\ben
\item [(i)] The $n$-dimensional Euclidean space\footnote{definition needed.}. The vector space $X$ is $\R^n$ or $\C^n$ and the norm is 
\be
\dabs{x} := \bb{\sum^n_{i=1}\abs{x_i}^2}^{1/2},\qquad x=(x_1,\dots, x_n)\in X.
\ee

The former is a real Euclidean space, the latter is a complex one.

\item [(ii)] Let $X$ be any set and $\F_b(X)$ be the vector space of all bounded scalar-valued functions ($f: X \to \F$) of $X$. The norm is defined by
\be
\dabs{f} := \sup_{x\in X}\abs{f(x)},\qquad f\in \F_b(X).
\ee

This norm is the uniform or supremum norm.

\item [(iii)] Let $L$ be a topological space and $X = C_b(L)$ be the vector space of all bounded continuous scalar-valued functions on $L$ and set
\be
\dabs{f} := \sup_{t\in L}\abs{f(t)},\qquad f\in C_b(L).
\ee

\item [(iv)] Consider the speical case of (ii) and (iii). Let $K$ be a compact Hausdorff space and let $C(K)$ be the space of continuous functions on $K$, with the supremum norm
\be
\dabs{f} = \dabs{f}_\infty = \sup_{x\in K}\abs{f(x)}.
\ee

Since $K$ is compact, $\abs{f(x)}$ is bounded on $K$ and attains its supremum.

\item [(v)] Let $X$ be $\R^n$ or $\C^n$ and set
\be
\dabs{x}_1 = \sum^n_{i=1}\abs{x_i} .
\ee

The space is $\ell^n_1(X)$ space and the norm is the $\ell_1$ norm. Also,
\be
\dabs{x}_\infty = \max_{1\leq i \leq n}\abs{x_i} .
\ee
is a norm, the $\ell_\infty$ norm. The space is $\ell_\infty^n(X)$ space.

\item [(vi)]

\een
\end{example}


\subsection{Formulas}

\begin{theorem}[arithmetic-geometric mean inequality\index{arithmetic-geometric mean inequality}]\label{thm:arithmetic_geometric_mean_inequality}
For positive real numbers $x_1, \dots, x_n$, 
\be 
\bb{\prod^n_{i=1} x_i}^{1/n} \leq \frac 1n \sum^n_{i=1} x_i. 
\ee

The equality holds only if $x_1 = x_2 = \dots = x_n$.
\end{theorem}

\begin{proof}[\bf Proof]
See Corollary \ref{cor:arithmetic_geometric_mean_inequality_probability_proof} for probability proof.
\end{proof}


\subsection{Separable normed space}


\begin{definition}[separable normed space\index{separable!normed space}]\label{def:separable_normed_space}
A normed space $(X,\dabs{\cdot})$ is called separable if there exist countably many dense subsets of $X$.
\end{definition}

\footnote{there is different definition for separable space.}

\begin{example}
$\forall p\in [1,\infty)$, $\ell_p(\F)$ is separable. However, $\ell_\infty(\F)$ is not separable\footnote{proof needed.}.
\end{example}


\section{Hahn-Banach Theorem}





\section{Banach Space}



\begin{definition}[Banach space\index{Banach space}]\label{def:banach_space}
\footnote{definition needed.}
\end{definition}

\begin{theorem}[Stone-Weierstrass theorem\index{Stone-Weierstrass theorem}]\label{thm:stone_Weierstrass_separates_points}
Suppose $X$ is a compact Hausdorff space and $\sA$ is a subalgebra of $C(X)$ which contains a non-zero constant function. Then $\sA$ is dense in $C(X)$ if and only if it separates points, i.e.,
$\forall x\in X$, there exist elements $f,g\in \sA$ such that $f(x) \neq g(x)$.
\end{theorem}

\begin{proof}[\bf Proof]
see \cite{Rogers_1994}.II80
\end{proof}


\begin{corollary}[Weierstrass' approximation theorem\index{Weierstrass' approximation theorem}]\label{cor:weierstrass_approximation}
Since the polynomials on $[a,b]$ form a subalgebra of $C[a,b]$\footnote{Note that the coefficients of the polynomial can be outside of $[a,b]$}, $\sA$, which contains the constants and separates
points, then $\sA$ is dense.
\end{corollary}

\begin{proof}[\bf Proof]
\footnote{see Rogers' book}
\end{proof}


\section{Hilbert Space}

\begin{definition}[orthogonal complement\index{orthogonal complement}]
\footnote{details needed.}
\end{definition}

\begin{proposition}
For infinite dimensional Hilbert space $X$ with linear subspace (vector space) W, we have 
\be
\bb{W^\perp }^\perp = \ol{W}.
\ee 

Furthermore, if $W$ is closed, we have 
\be
\bb{W^\perp }^\perp = W.
\ee
\end{proposition}

\begin{proof}[\bf Proof]
\footnote{proof needed. see wiki, orthogonal complement}
\end{proof}

\subsection{Adjoint operator}

\begin{definition}[adjoint operator\index{adjoint operator}]
Let $\sL$ be operator for all functions $f$ and $g$ in the Hilbert space with inner product. Then $\sL^*$ is defined as its adjoint operator if
\be
\inner{f}{\sL g} = \inner{\sL^*f}{g}.
\ee
\end{definition}

\begin{definition}[self-adjoint operator\index{self-adjoint operator}]
Let $\sL$ be operator in the Hilbert space and its adjoint operator is $\sL^*$. Then we call $\sL$ self-adjoint if $\sL = \sL^*$.
\end{definition}


\section{Summary}

normed space $\subseteq$ metric space (see Remark \ref{rem:normed_space_is_metric_space}).

normed space $\subseteq$ topological space\footnote{with respect to the norm topology}.

topological vector space $\subseteq$ Hausdorff.

complete metric space $\subseteq$ Baire space.
