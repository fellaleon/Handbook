\chapter{Calculus}

\section{Integrals}

\subsection{Integrals related to exponential function}

\begin{proposition}\label{pro:integral_exponential_trigonometric_function}%{pro:integral_exponential_cosine}
For $a,b,y,z\in\R$, Then
\be
\int^y_z \exp\brb{ax}\cos(bx)dx = \left. \frac {1}{a^2 + b^2}\brb{b \exp\brb{ax}\sin(bx) + a \exp\brb{ax} \cos(bx)}\right|^y_z,
\ee
\be
\int^y_z \exp\brb{ax}\sin(bx)dx = \left. \frac {1}{a^2 + b^2}\brb{-b \exp\brb{ax}\cos(bx) + a \exp\brb{ax} \sin(bx)}\right|^y_z,
\ee
\end{proposition}

\begin{proof}[\bf Proof]
Let $I = \int^y_z \exp\brb{ax}\cos(bx)dx$. Then by integral by part, we have \beast I & = & \frac 1b \int^y_z \exp\brb{ax}d\sin(bx) = \frac 1b\brb{\left.\exp\brb{ax}\sin(bx) \right|^y_z - a\int^y_z \exp\brb{ax}\sin(bx) dx} \\
& = & \frac 1b\brb{\left.\exp\brb{ax}\sin(bx) \right|^y_z + \frac ab \int^y_z \exp\brb{ax}d \cos(bx)} \\
& = & \frac 1b\brb{\left.\exp\brb{ax}\sin(bx) \right|^y_z + \frac ab \brb{\left.\exp\brb{ax} \cos(bx)\right|^y_z - a\int^y_z \exp\brb{ax} \cos(bx)dx}}\\
& = & \left.\frac 1b \exp\brb{ax}\sin(bx) + \frac a{b^2} \exp\brb{ax} \cos(bx)\right|^y_z - \frac {a^2}{b^2} I
\eeast

So we have
\be
\frac {a^2 + b^2}{b^2} I =  \left.\frac 1b \exp\brb{ax}\sin(bx) + \frac a{b^2} \exp\brb{ax} \cos(bx)\right|^y_z\ee which implies \be I = \left. \frac {1}{a^2 + b^2}\brb{b \exp\brb{ax}\sin(bx) + a \exp\brb{ax}
\cos(bx)}\right|^y_z
\ee


Let $I = \int^y_z \exp\brb{ax}\sin(bx)dx$. Then by integral by part, we have
\beast
I & = & -\frac 1b \int^y_z \exp\brb{ax}d\cos(bx) = -\frac 1b\brb{\left.\exp\brb{ax}\cos(bx) \right|^y_z - a\int^y_z \exp\brb{ax}\cos(bx) dx} \\
& = & -\frac 1b\brb{\left.\exp\brb{ax}\cos(bx) \right|^y_z - \frac ab \int^y_z \exp\brb{ax}d \sin(bx)} \\
& = & -\frac 1b\brb{\left.\exp\brb{ax}\cos(bx) \right|^y_z - \frac ab \brb{\left.\exp\brb{ax} \sin(bx)\right|^y_z - a\int^y_z \exp\brb{ax} \sin(bx)dx}}\\
& = & -\left.\frac 1b \exp\brb{ax}\cos(bx) + \frac a{b^2} \exp\brb{ax} \sin(bx)\right|^y_z - \frac {a^2}{b^2} I
\eeast

So we have
\be
\frac {a^2 + b^2}{b^2} I =  -\left.\frac 1b \exp\brb{ax}\cos(bx) + \frac a{b^2} \exp\brb{ax} \sin(bx)\right|^y_z
\ee
which implies
\be
I = \left. \frac {1}{a^2 + b^2}\brb{-b \exp\brb{ax}\cos(bx) + a \exp\brb{ax}\sin(bx)}\right|^y_z
\ee
\end{proof}


\subsection{The integrals related to logarithm function}

\begin{proposition}\label{pro:integral_x_lnx_product}
For $m,n\in \N$, we have
\be
\int^1_0 x^m\brb{\log x}^n dx = \frac{(-1)^n n!}{(m+1)^{n+1}}.
\ee
\end{proposition}

\begin{proof}[\bf Proof]
Integrating by parts,
\beast
\int^1_0 x^m \brb{\log x}^n dx & = & \left.\frac{1}{m+1} x^{m+1}\brb{\log x}^n \right|^1_0 - \frac{n}{m+1}\int^1_0 x^{m} \brb{\log x}^{n-1} dx \\
& = &  \dots = \frac{(-1)^n n!}{(m+1)^n} \int^1_0 x^{m} dx = \frac{(-1)^n n!}{(m+1)^{n+1}}.
\eeast

Note that $x^{m+1} \brb{\log x}^n \to 0$ as $x\to 0$ by L'Hopital's rule.
\end{proof}

\begin{proposition}
For $n\in \Z^+$, we have
\be
\int^1_0 \frac{\brb{\log x}^n}{(1-x)^2} dx = (-1)^n \zeta(n)\Gamma(n+1).
\ee
\end{proposition}

\begin{proof}[\bf Proof]
Apply Proposition \ref{pro:geometric_series_sum} and Proposition \ref{pro:integral_x_lnx_product},
\beast
\int^1_0 \frac{\brb{\log x}^n}{(1-x)^2}dx & = & \int^1_0 \sum^\infty_{k=1} k x^{k-1}\brb{\log x}^n dx =\sum^\infty_{k=1} k \int^1_0 x^{k-1}\brb{\log x}^n dx = \sum^\infty_{k=1} k \frac{(-1)^n n!}{k^{n+1}} \\
& = & (-1)^n n! \sum^\infty_{k=1} \frac{1 }{k^n} = (-1)^n n!\zeta(n) = (-1)^n \zeta(n)\Gamma(n+1).
\eeast
\end{proof}


\begin{proposition}
For $n\in\Z^+$, The integral
\be
\int^1_0 \frac{\brb{\log x}^n}{1+x}dx = (-1)^n \brb{1 - 2^{-n}}\zeta(n+1)\Gamma(n+1)
\ee
where $\zeta(\cdot)$ is the zeta function and $\Gamma(\cdot)$ is the gamma function.
\end{proposition}

\begin{proof}[\bf Proof]
For $\abs{x}<1$, we have by Proposition \ref{pro:geometric_series_sum}
\be
\sum^\infty_{k=0} (-x)^{k}  = \frac{1}{1+x}.
\ee

Then by Fubini theorem (Theorem \ref{thm:fubini}) and Proposition \ref{pro:integral_x_lnx_product}
\beast
\int^1_0 \frac{\brb{\log x}^n}{1+x}dx & = & \int^1_0 \sum^\infty_{k=0} (-x)^{k}\brb{\log x}^n dx = \sum^\infty_{k=0} (-1)^{k}\int^1_0 x^{k}\brb{\log x}^n dx = \sum^\infty_{k=0} (-1)^{k}\frac{(-1)^n n!}{\brb{k+1}^{n+1}} \\
& = & (-1)^n n! \sum^\infty_{k=1} (-1)^{k-1} \frac{1 }{k^{n+1}} = (-1)^{n} \brb{1 - 2^{-n}}\zeta(n+1)\Gamma(n+1)
\eeast
as required.
\end{proof}

\begin{example}
\beast
\int^1_0 \frac{\log x}{1+x} dx & = & -\brb{1 - \frac 12}\zeta(2)\Gamma(2) = - \frac{\pi^2}{12},\\
\int^1_0 \frac{\brb{\log x}^2}{1+x} dx & = & \brb{1 - \frac 14}\zeta(3)\Gamma(3) = \frac 34 \cdot \zeta(3) \cdot 2 = \frac {3\zeta (3)}{2}.
\eeast
\end{example}

\begin{proposition}\label{pro:integral_ln_even_power_over_one_plus_x_square}
\be
\int^{\infty}_0 \frac{\brb{\log x}^2}{(1+x)^2} dx = \frac {\pi^2}3,\qquad \int^{\infty}_0 \frac{\brb{\log x}^4}{(1+x)^2} dx = \frac {7\pi^4}{15}
\ee
\end{proposition}

\begin{proof}[\bf Proof]
Since the functions have pole -1 of order 2, we have that for keyhole contour $C$\footnote{details needed.},
\beast
\int_C \frac{\brb{\Log z}^m}{(1+z)^2}dz & = & 2\pi i \res\bsb{\frac{\brb{\Log z}^m}{(1+z)^2},-1} = 2\pi i \lim_{z\to -1}\frac{d}{dz}\brb{\Log z}^m \\
& = & 2\pi i \lim_{z\to -1}\frac{m}{z}\brb{\Log z}^{m-1} = -2\pi i m\brb{\ln\abs{-1}+\pi i}^{m-1} = -2m\pi^mi^m.
\eeast

Then set $m=3$, we have that (the integrals along $R$ and around zero are zeros)
\beast
6\pi^3i & = & \int_C \frac{\brb{\Log z}^3}{(1+z)^2}dz = \int^R_0 \frac{\brb{\Log (x+i\ve)}^3}{(1+x+i\ve)^2}dz + \int^0_R \frac{\brb{\Log (x-i\ve)}^3}{(1+x-i\ve)^2}dz \\
& \to & \int^\infty_0 \frac{\brb{\log x}^3 - \brb{\log x + 2\pi i}^3}{(1+x)^2} dx = \int^\infty_0 \frac{-6\pi i \brb{\log x}^2 + 12\pi^2 \log x + 8\pi^3 i}{(1+x)^2} dx.
\eeast

Then for the imaginary part, we have
\be
6\pi^3 = -6\pi \int^\infty_0 \frac{ \brb{\log x}^2}{(1+x)^2} dx + 8\pi^3 \int^\infty_0 \frac{ 1}{(1+x)^2} dx = -6\pi \int^\infty_0 \frac{ \brb{\log x}^2}{(1+x)^2} dx +8\pi^3.
\ee

Therefore, we have
\be
\int^\infty_0 \frac{ \brb{\log x}^2}{(1+x)^2} = \frac{2\pi^3}{6\pi} = \frac{\pi^2}{3}.
\ee

For $m=5$, we have that
\beast
-10\pi^5i & = & \int_C \frac{\brb{\Log z}^5}{(1+z)^2}dz = \int^R_0 \frac{\brb{\Log (x+i\ve)}^5}{(1+x+i\ve)^2}dz + \int^0_R \frac{\brb{\Log (x-i\ve)}^5}{(1+x-i\ve)^2}dz \\
\ra\   \int^\infty_0 \frac{\brb{\log x}^5 - \brb{\log x + 2\pi i}^5}{(1+x)^2} dx & = & \int^\infty_0 \frac{-10\pi i \brb{\log x}^4 + 40\pi^2 \brb{\log x}^3 + 80\pi^3 i \brb{\log x}^2 - 80\pi^4\brb{\log x} - 32\pi^5i  }{(1+x)^2} dx.
\eeast

Then for the imaginary part, we have
\be
-10\pi^5 = -10\pi  \int^{\infty}_0 \frac{\brb{\log x}^4}{(1+x)^2} dx + 80\pi^3 \frac{\pi^2}{3} - 32\pi^5.
\ee

Therefore,
\be
\int^{\infty}_0 \frac{\brb{\log x}^4}{(1+x)^2} dx = \frac{14\pi^5}{3\cdot 10\pi} = \frac{7\pi^4}{15} .
\ee
\end{proof}



%\begin{proposition}%\label{pro:integral_ln_odd_power_over_one_plus_x_square}%
%\be
%\int^1_0 \frac{\ln x}{(1+x)^2} dx = -\ln 2,\qquad \int^1_0 \frac{\brb{\ln x}^3}{(1+x)^2} dx = -\frac {9\zeta(3)}{2},\qquad \int^1_0 \frac{\brb{\ln x}^5}{(1+x)^2} dx = -\frac {225\zeta(5)}{2}.
%\ee
%\end{proposition}

%\begin{proof}[\bf Proof]
%By integral by parts,
%\be
%\int^1_0 \frac{\ln x}{(1+x)^2}dx = -\int^{\infty}_1 \frac{\ln x}{(1+x)^2}dx = \left.\frac{\ln x}{1+x}\right|^{\infty}_1  - \int^{\infty}_1 \frac{1}{x(1+x)}dx = -\left.\ln\brb{\frac{x}{x+1}}\right|^{\infty}_1 = -\ln 2.
%\ee

%\footnote{proof needed.}
%\end{proof}



%Similarly, we have



\begin{proposition}
For $n\in\Z^+$, The integral
\be
\int^{\infty}_1 \frac{\brb{\log x}^n}{(1+x)^2}dx = (-1)^n\int^1_0  \frac{\brb{\log x}^n}{(1+x)^2}dx.
\ee

In particular,
\be
\int^{\infty}_1 \frac{\log x}{(1+x)^2}dx = -\int^1_0  \frac{\log x}{(1+x)^2}dx = \ln 2.
\ee

For $n\in \bra{2,3,\dots}$, we have
\beast
\int^{\infty}_1 \frac{\brb{\log x}^n}{(1+x)^2}dx = (-1)^n\int^1_0  \frac{\brb{\log x}^n}{(1+x)^2}dx & = & n!\brb{1 - 2^{-(n-1)}}\zeta(n) \\
 & = & \brb{1 - 2^{-(n-1)}}\zeta(n)\Gamma(n+1)
\eeast
where $\zeta(\cdot)$ is the zeta function and $\Gamma(\cdot)$ is the gamma function.
\end{proposition}

\begin{proof}[\bf Proof]
First we use substitution $t = 1/x$, then
\be
\int^{\infty}_1 \frac{\brb{\log x}^n}{(1+x)^2}dx = \int^{0}_1 \frac{\brb{-\log t}^n}{-t^2(1+1/t)^2}dt = (-1)^n\int^1_0  \frac{\brb{\log t}^n}{(1+t)^2}dt.
\ee

For $n=1$, integrating by parts we have
\beast
\int^\infty_1  \frac{\log x}{(1+x)^2}dx & = & -\int^\infty_1 \log x d \brb{\frac{1}{1+x}} = -\left. \frac{\log x}{1+x}\right|^\infty_1 + \int^\infty_1  \frac{1}{x(1+x)}d x \\
& = &  \int^\infty_1  \frac{1}{x(1+x)}d x  = \left.\log\brb{\frac{x}{1+x}}\right|^\infty_1 = 0 - \ln(1/2) = \ln 2.
\eeast

For $n\geq 2$, we have that for $\abs{x}<1$, Proposition \ref{pro:geometric_series_sum},
\be
\sum^\infty_{k=1} k (-x)^{k-1}  = \frac{1}{(1+x)^2}.
\ee

Then by Fubini theorem (Theorem \ref{thm:fubini}) and Proposition \ref{pro:integral_x_lnx_product}
\beast
\int^\infty_1 \frac{\brb{\log x}^n}{(1+x)^2}dx & = & (-1)^n\int^1_0 \frac{\brb{\log x}^n}{(1+x)^2}dx  = (-1)^n\int^1_0 \sum^\infty_{k=1} k(-x)^{k-1}\brb{\log x}^n dx \\
& = & (-1)^n\sum^\infty_{k=1} k(-1)^{k-1}\int^1_0 x^{k-1}\brb{\log x}^n dx = (-1)^n\sum^\infty_{k=1} k(-1)^{k-1} \frac{(-1)^n n!}{k^{n+1}} \\
& = & n! \sum^\infty_{k=1} (-1)^{k-1} \frac{1 }{k^n} = n!\brb{1 - 2^{-(n-1)}}\zeta(n)
\eeast
as required.
\end{proof}


\begin{example}
For $n=2$, we have that $\zeta(n) = \zeta(2) = \pi^2 /6$,
\be
\int^\infty_1 \frac{\brb{\log x}^2}{(1+x)^2}dx = \int^1_0 \frac{\brb{\log x}^2}{(1+x)^2}dx = 2!\brb{1 - \frac 12}\zeta(2) = \frac{\pi^2}6 \ \ra\ \int^\infty_0 \frac{\brb{\log x}^2}{(1+x)^2}dx = \frac{\pi^2}3
\ee
which is consistent with Proposition \ref{pro:integral_ln_even_power_over_one_plus_x_square} approached by other method.
\end{example}

\begin{example}
For $n=3$, we have that
\be
\int^{\infty}_1 \frac{\brb{\log x}^3}{(1+x)^2}dx = 3!\brb{1 - \frac 14}\zeta(3) = \frac{9\zeta(3) }{2}\approx 5.4093.
\ee
\end{example}

Similarly, we have
\begin{proposition}
\be
\int^1_0 \frac{\brb{\log x}^n}{(1+x)^3} dx = \frac 12 (-1)^n n! \brb{\brb{1-2^{-(n-2)}}\zeta(n-1) + \brb{1-2^{-(n-1)}}\zeta(n)}.
\ee
\end{proposition}

\begin{example}
\be
\int^1_0 \frac{\brb{\log x}^3}{(1+x)^3} dx = -3\brb{\frac 12\zeta(2) + \frac 34 \zeta(3)} = -\frac{\pi^2}{4} - \frac {9\zeta(3)}{4}.
\ee

\be
\int^1_0 \frac{\brb{\log x}^4}{(1+x)^3} dx = 12 \brb{\frac 34\zeta(3) + \frac 78 \zeta(4)} = 9\zeta(3) + \frac{7\pi^4}{60} .
\ee
\end{example}

\begin{proposition}
\be
\int^1_0 \frac{\brb{\log x}^n}{(1+x)^4} dx = \frac 16 (-1)^n n! \brb{\brb{1-2^{-(n-3)}}\zeta(n-2) + 3\brb{1-2^{-(n-2)}}\zeta(n-1) + 2\brb{1-2^{-(n-1)}}\zeta(n)}.
\ee
\end{proposition}

\begin{example}
\be
\int^1_0 \frac{\brb{\log x}^4}{(1+x)^4} dx = 4 \brb{\frac 12\zeta(2) + 3\frac 34\zeta(3) + 2\frac 78 \zeta(4)} = \frac{\pi^2}{3} + 9\zeta(3) + \frac{7\pi^4}{90} .
\ee
\end{example}

\begin{proposition}\label{pro:logarithm_integral_x_divided_1-x}
\be
\int^1_0 \bsb{\log\brb{\frac x{1-x}}}^n dx = \brb{2-2^{2-n}} \zeta(n)\Gamma(n+1)=  \pi^n \brb{2^n-2} \cdot\abs{B_n}
\ee
where $B_n$ are the Bernoulli numbers of first kind.
\end{proposition}

\begin{proof}[\bf Proof]
Let $t = \frac x{1-x}$, we have $x = \frac{t}{1+t}$,
\be
\int^1_0 \bsb{\log\brb{\frac x{1-x}}}^n dx = \int^\infty_0 \brb{\log t}^n d\brb{\frac{t}{1+t}} = \int^\infty_0 \frac{\brb{\log t}^n}{(1+t)^2} dt .
\ee

If $n$ is odd, the result is zero. If $n$ is even, we have
\be
\int^1_0 \bsb{\log\brb{\frac x{1-x}}}^n dx = \brb{2-2^{2-n}} \zeta(n)\Gamma(n+1).
\ee

Then by Proposition \ref{pro:zeta_function_first_bernoulli_number}, $\abs{B_{n}} = \frac{2n!}{(2\pi)^{n}}\zeta(n)$ and
\be
\int^1_0 \bsb{\log\brb{\frac x{1-x}}}^n dx = \brb{2-2^{2-n}} \abs{B_{n}} \frac{(2\pi)^{n}}{2n!} \Gamma(n+1) = \pi^n \brb{2^n-2} \cdot\abs{B_n}.
\ee
\end{proof}


\subsection{Integral related to gamma function}

\begin{proposition}[Raabe, 1840]% proved that
\be
\int^{a+1}_a \log\Gamma(z)dz = \frac 12 \ln(2\pi) + a\ln a - a, \quad a>0.
\ee

In particular, if $a=0$ then
\be
\int^1_0 \log \Gamma(z)dz = \frac 12\ln (2\pi).
\ee
\end{proposition}

\begin{proof}[\bf Proof]
\footnote{proof needed.}
\end{proof}

\subsection{Integral related to Euler's constant}%$\gamma$}

\begin{proposition}\label{pro:gamma_integral_equivalent_forms}
\beast
\gamma = \int^\infty_0 \brb{\frac 1{e^x -1} - \frac 1{xe^x}} dx = \int^1_0 \brb{\frac 1{\log x} + \frac 1{1-x}}dx = \int^\infty_0 \brb{\frac{1}{1+x^k} - e^{-x}} \frac{dx}{x} ,\ k>0.
\eeast
\end{proposition}

\begin{proof}[\bf Proof]
\footnote{proof needed.}
\end{proof}

\begin{proposition}\label{pro:gamma_integral_exp_x_power_of_ln_x}
\be
\gamma = -\Gamma'(1) =  -\int^\infty_0 e^{-x}\log x dx = -4\int^\infty_0e^{-x^2} x\log xdx = -\int^1_0 \log\brb{\log \brb{\frac 1x}} dx ,
\ee

\be
\int^\infty_0 e^{-x }\brb{\log x}^2 dx = \Gamma''(1) = \gamma^2 + \frac{\pi^2}6,\qquad \int^\infty_0 e^{-x }\brb{\log x}^3 dx = \Gamma'''(1) = -\gamma^3 - \frac {\gamma \pi^2}2 - 2\zeta(3),
\ee

\be
\int^\infty_0 e^{-x }\brb{\log x}^4 dx = \Gamma''''(1) = \gamma^4 + \pi^2 \gamma^2 + 8 \gamma \zeta(3) + \frac{3\pi^4}{20}.
\ee
\end{proposition}

\begin{proof}[\bf Proof]
Direct result of Proposition \ref{pro:derivatives_of_gamma_function_integral} and \ref{pro:derivatives_of_gamma_1234}.%\footnote{proof needed. see Whittaker, E. T. and Watson, G. N. A Course in Modern Analysis, 4th ed. Cambridge, England: Cambridge University Press, pp. 235-236, 246, and 271, 1990.}\footnote{also see wiki gamma function}
\end{proof}

\begin{proposition}\label{pro:gamma_integral_exp_x_square_ln_x}
\be
\int^\infty_0 e^{-x^2 }\log x dx = -\frac{\brb{\gamma + 2\ln 2}\sqrt{\pi}}{4}
\ee
\end{proposition}

\begin{proof}[\bf Proof]
\footnote{proof needed. This integral is actually equal to $\Gamma'\brb{\frac 12}$. see Proposition \ref{pro:derivatives_of_gamma_function_integral}}
\end{proof}

\begin{proposition}
\beast
\gamma & = & \int^1_0\int^1_0 \frac{x-1}{(1-xy)\log (xy)}dxdy = \sum^\infty_{n=1}\brb{\frac 1n - \ln \brb{\frac{n+1}n}} ,\\
\ln \brb{\frac 4{\pi}} & = & \int^1_0\int^1_0 \frac{x-1}{(1+xy)\log (xy)}dxdy = \sum^\infty_{n=1}\brb{(-1)^{n-1}\brb{\frac 1n - \ln \brb{\frac{n+1}n}}}.
\eeast
\end{proposition}

\begin{remark}
It shows that $\ln \brb{\frac 4{\pi}}$ may be thought of as an `alternative Euler constant'.
\end{remark}

\begin{proof}[\bf Proof]
\footnote{proof needed.}
\end{proof}


\section{Polar Coordinate}


\section{Cylindrical Coordinates}


\section{Spherical Coordinates}

\subsection{Integration and differentiation in spherical coordinates}

The surface element spanning from $\theta$ to $\theta + d\theta$ and $\phi$ to $\phi + d\phi$ on a spherical surface at (constant) radius $r$ is

\be
d S_{r}=r^{2}\sin \phi \,d\theta \,d \phi
\ee
as the area of small square on surface is $(r\sin \phi d\theta)\times (r d\phi) = r^2\sin \phi d\theta d \phi $ .

Thus the differential solid angle is
\be
d\Omega ={\frac {d S_{r}}{r^{2}}}=\sin \phi \,d\theta \,d \phi .
\ee

The surface element in a surface of polar angle $\phi$ constant (a cone with vertex the origin) is
\be
d S_{\phi }=r\sin \phi \,d\theta \,d r.
\ee

The surface element in a surface of azimuth $\theta$ constant (a vertical half-plane) is
\be
d S_{\theta }=r\,d r\,d\phi .
\ee

\subsection{Applications}

\begin{theorem}
The area of the sphere with radius $r$ is $4\pi r^2$.
\end{theorem}

\begin{proof}[\bf Proof]
The area of the sphere can be consider as %the cumulation of the circumference of circular sections of the sphere. That is.
%\be
%\int^{\pi}_0 \pi \ell_{r\sin\theta} d\theta = \int^{\pi}_0 \pi r\sin\theta d\theta = -2r\int^{\pi/2}_0 d \sin\theta
%\ee
\be
\int^{\pi}_0 \int^{2\pi}_0 r^{2}\sin \phi \,d\theta \,d \phi = 2\pi r^2 \int^{\pi}_0 \sin\phi d\phi = 2\pi r^2 \left.(-\cos\phi)\right|^{\pi}_0 = 4\pi r^2.
\ee
\end{proof}

\begin{theorem}\label{thm:spherical_area_between_parallel_planes_proportional_to_height}
Let $S$ be a sphere with radius $r$. Then the spherical area between to parallel planes is proportional to the height of the zone (i.e., the distance between to planes).
\end{theorem}

\begin{proof}[\bf Proof]
Let two parallel planes have polar angle $\phi_1<\phi_2$. Thus the height of the zone between two planes is $r\brb{\cos\phi_1 - \cos\phi_2}$. Thus, the area is
\be
\int^{\phi_2}_{\phi_1} \int^{2\pi}_0 r^2\sin \phi \,d\theta \,d \phi= 2r^2  \int^{\phi_2}_{\phi_1}  \sin\phi d\phi = 2r^2 \left.\brb{-\cos \phi}\right|^{\phi_2}_{\phi_1} = 2r^2 \brb{\cos \phi_1 - \cos\phi_2}
\ee
as required.
\end{proof}



\section{Fractional Calculus}

\subsection{Definition of fractional calculus}

\section{Problems}

\begin{problem}
Let $f$ be continuously differentiable on $[0,a]$ with $f(0) = 0$. Then show that
\be
\int^a_0 \abs{f(x)f'(x)}dx \leq \frac a2 \int^a_0 \brb{f'(x)}^2 dx.
\ee
\end{problem}

\begin{solution}[\bf Solution.]
{\it Approach 1.} By AM-GM theorem\footnote{theorem needed.} and the fact $f(x) = f(x) -f(0)= \int^x_0 f'(y)dy$
\beast
\int^a_0 \abs{f(x)f'(x)}dx & = & \int^a_0 \abs{f'(x)}\abs{ \int^x_0 f'(y)dy }dx \leq \int^a_0 \int^x_0 \abs{f'(x) f'(y) }dy dx \\
& \leq & \frac 12 \int^a_0 \int^x_0 \brb{f'(x)}^2 dy dx + \frac 12 \int^a_0 \int^x_0 \brb{f'(y)}^2 dy dx = \frac 12 \int^a_0  x \brb{f'(x)}^2 dx + \frac 12 \int^a_0 \brb{f'(y)}^2 \int^a_y dx dy \\
& = & \frac 12 \int^a_0  x \brb{f'(x)}^2 dx + \frac 12 \int^a_0 (a-x)\brb{f'(x)}^2 dx = \frac a2 \int^a_0  \brb{f'(x)}^2 dx.
\eeast

{\it Approach 2.} , we have
\beast
\int^a_0 \abs{f(x)f'(x)}dx & \leq & \int^a_0 \int^x_0 \abs{f'(x) f'(y) }dy dx \\
& \leq & \frac 12 \brb{\int^a_0 \int^x_0 \abs{f'(x) f'(y) }dy dx + \int^a_0 \int^a_y \abs{f'(x) f'(y) } dx dy} = \frac 12 \brb{\int^a_0  \abs{f'(x)} dx }^2 \\
& \leq & \frac 12 \int^a_0 1 dx \int^a_0 \brb{f'(x)}^2 dx = \frac a2 \int^a_0  \brb{f'(x)}^2 dx
\eeast
by Cauchy-Schwartz inequality\footnote{theorem needed.}.
\end{solution}

\begin{remark}
For $f(x) = x$, $f'(x) = 1$, then
\be
\int^a_0 \abs{f(x)f'(x)}dx = \int^a_0 x dx = \frac 12 a^2 \leq \frac a2 \int^a_0 1 dx = \frac a2 \int^a_0  \brb{f'(x)}^2 dx
\ee

For $f(x) = x^2$, $f'(x) = 2x$, then
\be
\int^a_0 \abs{f(x)f'(x)}dx = \int^a_0 2x^3 dx = \frac 12 a^4 \leq \frac 23a^4 = \frac a2 \int^a_0 4x^2 dx = \frac a2 \int^a_0  \brb{f'(x)}^2 dx
\ee
\end{remark}
