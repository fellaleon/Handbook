\chapter{Fourier Analysis}


%\section{Waves in strings}
%It is said that Pythagoras was the first to realise that the notes emitted by struck strings of lengths $l$, $l/2$, $l/3$ and so on formed particularly attractive harmonies for the human ear. From
%this he concluded, it is said, that all is number and the universe is best understood in terms of mathematics --- one of the most outrageous and most important leaps of faith in human history.
%
%Two millennia later the new theory of mechanics and the new method of mechanics enabled mathematicians to write down a model for a vibrating string. Our discussion will be exploratory with no attempt at rigour. Suppose that the string is in tension $T$ and has constant density $\rho$. If the graph of the position of the string at time $t$ is given by $y=Y(x,t)$ where $Y(x,t)$ is always very small then, working to the first order in $\delta x$, the portion of the string between $x$ and $x+\delta x$ experiences a force parallel to the $y$-axis of
%\be
%T\left(\frac{\partial Y}{\partial x}(x+\delta x,t) -\frac{\partial Y}{\partial x}(x,t)\right) =T\delta x\frac{\partial^{2} Y}{\partial x^{2}}.
%\ee
%
%Applying Newton's second law\footnote{theorem needed here.}, we obtain (still working to first order)
%\be
%\rho\delta x\frac{\partial^{2} Y}{\partial t^{2}} =T\delta x\frac{\partial^{2} Y}{\partial x^{2}}.
%\ee
%
%Thus we have the exact equation
%\be
%\rho\frac{\partial^{2} Y}{\partial t^{2}} =T\frac{\partial^{2} Y}{\partial x^{2}}.
%\ee
%
%For reasons which will become apparent later, it is usual to write $c$ for the positive square root of $T/\rho$ giving our equation in the form
%\be
%\frac{\partial^{2} Y}{\partial t^{2}} =c^{2}\frac{\partial^{2} Y}{\partial x^{2}}.
%\ee
%
%This equation is often called `the wave equation'.



\section{Fourier Series}


\footnote{need sine cosine expression of fourier series here. Note that period is $\sL = 2L$}

\begin{corollary}\label{cor:fourier_series_linear}
Assume $x\in [-L,L]$. Then
\be
x  = 2L\sum^\infty_{n=1} \frac {(-1)^{n-1}}{n\pi}\sin \bb{\frac{n\pi x}{L}}.
\ee
\end{corollary}

\begin{proof}[\bf Proof]
Since $x$ is an odd function, we have\footnote{need theorem here}, $a_n = 0$ and
\be
b_n = \frac 2L \int^L_0 x\sin\bb{\frac{n\pi x}{L}}dx = \frac {2L}{Ln\pi} \left.x\cos\bb{\frac{n\pi x}{L}}\right|^0_L = \frac{2L(-1)^{n-1}}{n\pi},
\ee
as required.
\end{proof}

\begin{corollary}\label{cor:fourier_series_cube}
Assume $x\in [-L,L]$. Then
\be
x^3  = 2L^3\sum^\infty_{n=1} (-1)^{n-1} \bb{\frac 1{n\pi} - \frac{6}{n^3\pi^3}}\sin \bb{\frac{n\pi x}{L}} = L^2 x - 12L^3\sum^\infty_{n=1} \frac{(-1)^{n-1} }{n^3\pi^3}\sin \bb{\frac{n\pi x}{L}}
\ee
\end{corollary}

\begin{proof}[\bf Proof]
Since $x^3$ is an odd function, we have\footnote{need theorem here}, $a_n = 0$ and
\beast
b_n & = & \frac 2L \int^L_0 x^3\sin\bb{\frac{n\pi x}{L}}dx = \frac {2L}{Ln\pi} \bb{\left.x^3\cos\bb{\frac{n\pi x}{L}}\right|^0_L + \int^L_0 3x^2\cos\bb{\frac{n\pi x}{L}}dx} \\
& = & \frac {2}{n\pi} \bb{L^3(-1)^{n-1} + \frac{3L}{n\pi}\bb{\left.x^2\sin\bb{\frac{n\pi x}{L}}\right|^0_L  - \int^L_0 2x \sin\bb{\frac{n\pi x}{L}}dx}} \\
& = & \frac {2}{n\pi} \bb{L^3(-1)^{n-1} - \frac{6L^2}{n^2\pi^2}\bb{\left.x\cos\bb{\frac{n\pi x}{L}}\right|^0_L}} = \frac {2}{n\pi} \bb{L^3(-1)^{n-1} + \frac{6L^2}{n^2\pi^2}L(-1)^n} \\
& = & 2L^3(-1)^{n-1} \bb{\frac 1{n\pi} - \frac{6}{n^3\pi^3}}.
\eeast
as required. The second equation is obtained by Corollary \ref{cor:fourier_series_linear}.
\end{proof}

\begin{corollary}\label{cor:fourier_series_power_5}
Assume $x\in [-L,L]$. Then
\be
x^5  = 2L^5\sum^\infty_{n=1} (-1)^{n-1} \bb{\frac{1}{n\pi}-\frac{20}{n^3\pi^3} + \frac{120}{n^5\pi^5}}  \sin \bb{\frac{n\pi x}{L}} = \frac {10}3L^2 x^3 -\frac 73 L^4 x + 240L^5\sum^\infty_{n=1} \frac{(-1)^{n-1} }{n^5\pi^5}\sin \bb{\frac{n\pi x}{L}}
\ee
\end{corollary}

\begin{proof}[\bf Proof]
Since $x^5$ is an odd function, we have\footnote{need theorem here}, $a_n = 0$ and
\beast
b_n & = & \frac 2L \int^L_0 x^5\sin\bb{\frac{n\pi x}{L}}dx = \frac {2L}{Ln\pi} \bb{\left.x^5\cos\bb{\frac{n\pi x}{L}}\right|^0_L + \int^L_0 5x^4\cos\bb{\frac{n\pi x}{L}}dx} \\
& = & \frac {2}{n\pi} \bb{L^5(-1)^{n-1} - \frac{20L}{n\pi}\int^L_0 x^3 \sin\bb{\frac{n\pi x}{L}}dx} = \frac {2}{n\pi} \bb{L^5(-1)^{n-1} - \frac{20L^2}{n^2\pi^2}\bb{\left.x^3\cos\bb{\frac{n\pi x}{L}}\right|^0_L+ \int^L_0 3x^2\cos\bb{\frac{n\pi x}{L}}dx}} \\
& = & \frac {2}{n\pi} \bb{L^5(-1)^{n-1}\bb{1-\frac{20}{n^2\pi^2}} + \frac{60L^3}{n^3\pi^3}\int^L_0 2x\sin\bb{\frac{n\pi x}{L}}dx} = \frac {2L^5(-1)^{n-1}}{n\pi} \bb{1-\frac{20}{n^2\pi^2} + \frac{120}{n^4\pi^4}}
\eeast
as required. The second equation is obtained by Corollary \ref{cor:fourier_series_linear} and Corollary \ref{cor:fourier_series_cube}.
\end{proof}



%\begin{proposition}
%\be

%\ee
%\end{proposition}

%\begin{proof}[\bf Proof]
%\footnote{this can be calculated by $a_n = \frac 2L\int^L_0 1\sin \bb{\frac{n\pi x}{L}} dx$}
%\end{proof}

\begin{proposition}\label{pro:alternative_odd_series}
\be
\sum^\infty_{n=0} \frac{(-1)^{n}}{2n+1} = 1-\frac 13 + \frac 15 - \frac 17 + \dots = \frac{\pi}{4}.\qquad (\text{Leibniz formula for }\pi)
\ee

\be
\sum^\infty_{n=0} \frac{(-1)^{n}}{(2n+1)^3} = 1-\frac 1{3^3} + \frac 1{5^3} - \frac 1{7^3} + \dots = \frac{\pi^3}{32}.
\ee

\be
\sum^\infty_{n=0} \frac{(-1)^{n}}{(2n+1)^5} = 1-\frac 1{3^5} + \frac 1{5^5} - \frac 1{7^5} + \dots = \frac{5\pi^5}{1536}.
\ee

\be
\sum^\infty_{n=0} \frac{(-1)^{n}}{(2n+1)^7} = 1-\frac 1{3^7} + \frac 1{5^7} - \frac 1{7^7} + \dots = \frac{61\pi^7}{184320}.
\ee
\end{proposition}

\begin{proof}[\bf Proof]
These are easily implied by letting $x = L/2$ in Proposition \ref{cor:fourier_series_linear}, \ref{cor:fourier_series_cube} and \ref{cor:fourier_series_power_5}\footnote{checking needed for the last equation}.
\end{proof}


\begin{proposition}\label{pro:sum_inverse_even_power}
\be
\sum^\infty_{n=1} \frac 1{n^2} = \frac {\pi^2}6,\qquad \sum^\infty_{n=1}\frac 1{n^4} = \frac {\pi^4}{90},\qquad \sum^\infty_{n=1} \frac 1{n^6} = \frac {\pi^6}{945}.
\ee
\end{proposition}

\begin{proof}[\bf Proof]
\footnote{proof needed.}
\end{proof}

\begin{corollary}\label{cor:sum_inverse_square_odd}
\be
\sum^\infty_{n=1} \frac 1{(2n-1)^2} = \frac {\pi^2}8, \qquad \sum^\infty_{n=1} \frac 1{(2n-1)^4} = \frac {\pi^4}{96}, \qquad \sum^\infty_{n=1} \frac 1{(2n-1)^6} = \frac {\pi^6}{960}.
\ee
\end{corollary}

\begin{proof}[\bf Proof]
We have that
\be
\sum^\infty_{n=1} \frac 1{(2n-1)^2} = \sum^\infty_{n=1} \frac 1{n^2} - \sum^\infty_{n=1} \frac 1{(2n)^2} = \frac 34 \sum^\infty_{n=1} \frac 1{n^2} = \frac{\pi^2}{8}
\ee
by Proposition \ref{pro:sum_inverse_even_power}. We can apply the similar method for the other results.
\end{proof}

\begin{proposition}
\be
\sum^\infty_{n=1} (-1)^{n-1}\frac 1{n^2} = \frac {\pi^2}{12},\qquad \sum^\infty_{n=1}(-1)^{n-1}\frac 1{n^4} = \frac {7\pi^4}{720},\qquad \sum^\infty_{n=1} (-1)^{n-1}\frac 1{n^6} = \frac {31\pi^6}{30240}.
\ee
\end{proposition}

\begin{proof}[\bf Proof]
\footnote{proof needed.}
\end{proof}


\section{Fourier Transform}

\subsection{Definitions of Fourier transform}

\begin{definition}[Fourier transform, unitary ordinary frequency\index{Fourier transform!unitary ordinary frequency}]\label{def:fourier_transform_unitary_ordinary_frequency}
The Fourier transform $\wh{f}$ of an integrable function $f : \R \to \C$\footnote{Kaiser 1994, p. 29, Rahman 2011, p. 11} is
\be
\wh{f}(\xi) = \int_{-\infty}^\infty f(x)\ \exp\bb{- 2\pi i x \xi}dx, \qquad  \forall \xi \in \R.
\ee
\end{definition}

\begin{remark}
When the independent variable $x$ represents time (with unit of seconds), the transform variable $\xi$ represents frequency (in hertz).

There are other version of Fourier transform, i.e., Definition \ref{def:fourier_transform_borel}.
\end{remark}

\begin{theorem}[Fourier inversion theorem\index{Fourier inverse transform!unitary ordinary frequency}]
Under suitable conditions\footnote{details needed.}, $f$ is determined by $\wh{f}$ via the inverse transform:
\be
f(x) = \int_{-\infty}^\infty \wh{f}(\xi) \exp\bb{2 \pi i \xi x}\,d\xi,  \qquad \forall x\in \R.
\ee
\end{theorem}

\begin{remark}
The statement that $f$ can be reconstructed from $\wh{f}$ is known as the Fourier inversion theorem, and was first introduced in Fourier's Analytical Theory of Heat (Fourier 1822, p. 525), (Fourier \& Freeman 1878, p. 408), although what would be considered a proof by modern standards was not given until much later (Titchmarsh 1948, p. 1). The functions $f$ and $\wh{f}$ often are referred to as a Fourier integral pair or Fourier transform pair (Rahman 2011, p. 10).
\end{remark}

\begin{proof}[\bf Proof]
\footnote{proof needed.}
\end{proof}

\subsection{Fourier transform of functions}

\begin{proposition}\label{pro:fourier_transform_exponential_minus_quadratic}
For $f(x) = \exp\bb{-a x^2}$ with $\Re(a) >0$
\be
\wh{f}(\xi) = \sqrt{\frac{\pi}a} \exp\bb{-\frac{\pi^2 \xi^2}a},\qquad \wh{f}(\omega) = \frac{1}{\sqrt{2 a}} \exp\bb{-\frac{\omega^2}{4a}},\qquad \wh{f}(\nu) = \sqrt{\frac{\pi}{a}}\cdot \exp\bb{-\frac{\nu^2}{4 a}}.
\ee
\end{proposition}

\begin{proof}[\bf Proof]
\footnote{proof needed.}
\end{proof}


\begin{proposition}\label{pro:fourier_transform_exponential_absolute}
For $f(x) = \exp\bb{-a \abs{x}}$ with $a>0$,
\be
\wh{f}(\xi) = \frac{2 a}{a^2 + 4 \pi^2 \xi^2} ,\qquad \wh{f}(\omega) = \sqrt{\frac{2}{\pi}} \cdot \frac{a}{a^2 + \omega^2},\qquad \wh{f}(\nu) = \frac{2a}{a^2 + \nu^2}.
\ee

That is, the Fourier transform of a decaying exponential function is a Lorentzian function.
\end{proposition}

\begin{proof}[\bf Proof]
\footnote{proof needed.}
\end{proof}


\subsection{Poisson summation formula}

%\be
%\underbrace{\sum_{n=-\infty}^{\infty} s(t + nP)}_{S_P(t)} = \sum_{k=-\infty}^{\infty} \underbrace{\frac{1}{P}\cdot \hat s\left(\frac{k}{P}\right)}_{S[k]}\ e^{i 2\pi \frac{k}{P} t }
%\ee

\begin{theorem}[Poisson summation formula\index{Poisson summation formula}]\label{thm:poisson_summation_formula}
For function $f(x)$\footnote{condition needed},
\be
\sum_{n=-\infty}^{\infty} f(x + ny) = \sum_{k=-\infty}^{\infty} \frac{1}{y}\cdot \wh{f}\left(\frac{k}{y}\right)\ \exp\bb{\frac{2\pi i kx} y}.
\ee
where
\be
\wh{f}(\xi) = \int_{-\infty}^{\infty} f(x) e^{-2\pi i\xi x}  dx.
\ee
is Fourier transform of unitary, ordinary frequency.
\end{theorem}

\begin{proof}[\bf Proof]
\footnote{proof need. (Pinsky 2002; Zygmund 1968), Pinsky, M. (2002), Introduction to Fourier Analysis and Wavelets., Brooks Cole., Zygmund, Antoni (1968), Trigonometric series (2nd ed.), Cambridge University Press (published 1988), ISBN 978-0-521-35885-9.}
\end{proof}
