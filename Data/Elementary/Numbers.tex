
\chapter{Numbers}\label{cha:numbers}

\section{Natual Numbers}

\subsection{Peano axioms and natural numbers}

First we recall the definition of natural number from Tao \cite{Tao_2006_1}.$P_{17-50}$\footnote{need to introduce 0 and increment operation ++ first}.

%\begin{definition}[natural number\footnote{this is informal}]
%A natural number is any element of the set
%\be
%\N := \bra{0,1,2,3,4,\dots},
%\ee
%which is the set o fall the numbers created by starting with 0 and then counting forward indefinitely. We call $\N$ the set of natural numbers.

%Natural numbers are sometimes also known as whole numbers.
%\end{definition}

%\begin{axiom}[Peano axioms]\label{axiom:peano_axioms}
%\end{axiom}


\begin{axiom}[natural numbers\index{natural numbers}]\label{axm:natural_number}
There exists a set $\N$, whose elements are called natural numbers, as well as an object 0 in $\N$, and object $n++$ assigned to every natural number $n\in\N$, such that the following axioms (Peano axioms\index{Peano axioms}) hold:
\ben
\item [(i)] 0 is a natural number.
\item [(ii)] If $n$ is a natural number, then $n++$ is also a natural number.
\item [(iii)] 0 is not the successor of any natural number; i.e., we have $n++\neq 0$ for every natural number $n$.
\item [(iv)] Different natural numbers must have different successors; i.e., if $n,m$ are natural numbers and $n\neq m$, then $n++ \neq m++$. Equivalently, if $n++ = m++$, then we must have $n=m$.
\item [(v)] (principle of mathematical induction). Let $P(n)$ be any property pertaining to a natural number $n$. Suppose that $P(0)$ is true, and suppose that whenever $P(n)$ is true, $P(n++)$ is also true. Then $P(n)$ is true for every natural number $n$.
\een% (\footnote{need five axioms})
\end{axiom}

\begin{remark}
\footnote{explain the axioms}
\end{remark}

\begin{remark}
One interesting feature about the natural numbers is that while each individual natural number is finite\footnote{need definition}, the set of natural numbers is infinite (Tao \cite{Tao_2006_1}.$P_{24}$).%
\end{remark}

\subsection{Addition of natural numbers}

\begin{definition}[addition of natural numbers\index{addition!natural number}]
Let $m$ be a natural number. To add zero to $m$, we define $0+m := m$. Now suppose inductively that we have defined how to add $n$ to $m$. Then we can add $n++$ to $m$ by defining $(n++)+m := (n+m)++$.
\end{definition}

\begin{lemma}
For any natural number $n$, $n+0 = n$.
\end{lemma}

\begin{proof}[\bf Proof]
\footnote{need proof}
\end{proof}

\begin{lemma}
For any natural numbers $n$ and $m$, $n+(m++) = (n+m)++$.
\end{lemma}

\begin{proof}[\bf Proof]
\footnote{need proof}
\end{proof}

\begin{proposition}[addition is commutative]\label{pro:addition_is_commutative_natural_number}
For any natural numbers $n$ and $m$, $n+m = m+n$.
\end{proposition}

\begin{proof}[\bf Proof]
\footnote{need proof}
\end{proof}

\begin{proposition}[addition is associative]
For any natural numbers $a,b,c$, we have $(a+b) +c = a + (b+c)$..
\end{proposition}

\begin{proof}[\bf Proof]
\footnote{need proof}
\end{proof}

\begin{proposition}[cancellation law\index{cancellation law!addition of natural number}]
Let $a,b,c$ be natural numbers such that $a+b = a+c$. Then we have $b=c$.
\end{proposition}

\begin{proof}[\bf Proof]
\footnote{need proof}
\end{proof}

\begin{definition}[positive natural numbers\index{positive!natural number}]
A natural number $n$ is said to be positive iff it is not equal to 0.
\end{definition}

\begin{proposition}
If natural number $a$ is positive and $b$ is a natural number, then $a+b$ is positive. Hence $b+a$ is also positive by Proposition \ref{pro:addition_is_commutative_natural_number}.
\end{proposition}

\begin{proof}[\bf Proof]
\footnote{need proof}
\end{proof}

\begin{corollary}
If $a$ and $b$ are natural numbers such that $a+b = 0$, then $a=0$ and $b=0$
\end{corollary}

\begin{proof}[\bf Proof]
\footnote{need proof}
\end{proof}

\begin{lemma}
Let $a$ be a positive number. Then there exists exactly one natural number $b$ such that $b++ = a$.
\end{lemma}

\begin{proof}[\bf Proof]
\footnote{need proof}
\end{proof}

\begin{definition}[ordering of the natural numbers]
Let $n$ and $m$ be natural numbers. We say that $n$ is greater than or equal to $m$, and write $n\geq m$ or $m\leq n$, iff we have $n=m+a$ for some natrual number $a$. We say that $n$ is strictly greater than $m$, and write $n>m$ or $m<n$, iff $n\geq m$ and $n\neq m$.
\end{definition}

\begin{proposition}[basic properties of order for natural numbers]
Let $a,b,c$ be natural numbers. Then
\ben
\item [(i)] (order is reflexive). $a\geq a$.
\item [(ii)] (order is transitive). If $a\geq b$ and $b\geq c$, then $a\geq c$.
\item [(iii)] (order is anti-symmetric). If $a\geq b$ and $b\geq a$, then $a=b$.
\item [(iv)] (addition preserves order). $a\geq b$ if and only if $a+c \geq b+c$.
\item [(v)] $a<b$ if and only if $a++ \leq b$.
\item [(vi)] $a<b$ if and only if $b=a+d$ for some positive natural number $d$.
\een
\end{proposition}

\begin{proof}[\bf Proof]
\footnote{need proof}
\end{proof}

\begin{proposition}[trichotomy of order for natural numbers]
Let $a$ and $b$ be natural numbers. Then exactly one of the following statement is true: $a<b$, $a=b$, or $a>b$.
\end{proposition}

\begin{proof}[\bf Proof]
\footnote{need proof}
\end{proof}

\begin{proposition}[strong principle of induction]
Let $m_0$ be natural number, and let $P(m)$ be a property pertaining to an arbitrary natural number $m$. Suppose that for each $m\geq m_0$, we have the following implication: if $P(m')$ is true for all natural numbers $m_0\leq m'\leq m$, then $P(m)$ is also true. In particular, this means that $P(m_0)$ is true, since in this case the hypothesis is vacuous. Then we can conclude that $P(m)$ is true for all natural numbers $m\geq m_0$.
\end{proposition}

\begin{proof}[\bf Proof]
\footnote{need proof}
\end{proof}

\subsection{Multiplication of natural numbers}

\begin{definition}[multiplication of natural numbers\index{multiplication!natural number}]
Let $m$ be a natural number. To multiply zero to $m$, we define $0\times m:=0$. Now suppose inductively that we have defined how to multiply $n \to m$. Then we can multiply $n++$ to $m$ by defining $(n++)\times m := (n\times m)+m$.
\end{definition}

\begin{remark}
We can abbreviate $n\times m$ as $nm$.
\end{remark}

\begin{lemma}[multiplication is commutative]
Let $n,m$ be natural numbers. Then $n\times m = m \times n$.
\end{lemma}

\begin{remark}
Therefore, we have no confusion on $abc$ $(a\times b\times c)$ where $a,b,c$ are natural numbers.
\end{remark}


\begin{proof}[\bf Proof]
\footnote{need proof}
\end{proof}

\begin{lemma}[natural numbers have no zero divisors]
Let $n,m$ be natural numbers. Then $n\times m = 0$ if and only if at least one of $n,m$ is equal to zero. In particular, if $n$ and $m$ are both positive, then $nm$ is positive.
\end{lemma}

\begin{proof}[\bf Proof]
\footnote{need proof}
\end{proof}

\begin{proposition}[distributive law\index{distributive law!natural number}]
For any natural numbers $a,b,c$, we have
\be
a(b+c) = ab+ ac,\quad (b+c)a = ba + ca.
\ee
\end{proposition}

\begin{proof}[\bf Proof]
\footnote{need proof}
\end{proof}

\begin{proposition}[multiplication is associative]
For any natural numbers $a,b,c$, we have $(a\times b)\times c = a\times (b\times c)$.
\end{proposition}

\begin{proof}[\bf Proof]
\footnote{need proof}
\end{proof}

\begin{proposition}[multiplication preserves order]
If $a,b$ are natural numbers, such that $a<b$, and $c$ is positive, then $ac < bc$.
\end{proposition}

\begin{proof}[\bf Proof]
\footnote{need proof}
\end{proof}

\begin{corollary}[cancellation law\index{cancellation law!multiplication of natural number}]
Let $a,b,c$ be natural numbers, such that $ac = bc$ and $c$ is non-zero. Then $a = b$.
\end{corollary}

\begin{proof}[\bf Proof]
\footnote{need proof}
\end{proof}

\begin{proposition}[Euclidean algorithm\index{Euclidean algorithm!natural numbers}]
Let $n$ be a natural number, and let $q$ be a positive natural number. Then there exist natural numbers $m,r$ such that $0\leq r<q$ and $n = mq +r$.
\end{proposition}

\begin{proof}[\bf Proof]
\footnote{need proof}
\end{proof}

%\subsection{Exponentiation of natural numbers}

%\begin{definition}[exponentiation of natural numbers\index{exponentiation!natural number}]
%Let $m$ be a natural number. To raise $m$ to the power 0, we define $m^0 := 1$. Now suppose recursively that $m^n$ has been defined for some natural number $n$, then we define $m^{n++}:=m^n \times m$.
%\end{definition}

\section{Integers and Rational Numbers}

\subsection{The integers}

\begin{definition}[intergers\index{integer}]\label{def:integer}
An integer is an expression\footnote{In the language of set theory, what we are doing here is starting with the space $\N\times \N$ of ordered pairs $(a,b)$ of natural numbers. Then we place an equivalence relation $\sim$ on these pairs by declaring $(a,b) \sim (c,d)$ iff $a+d = c+b$. The set-theoretic interpretation of the symbol $a\text{---}b$ is that it is the space of all pairs equivalent to $(a,b)$: $a\text{---}b:=\bra{(c,d)\in \N\times\N:(a,b)\sim (c,d)}$.} of the form $a\text{---}b$\footnote{need to check this is an equivalence relation}, where $a$ and $b$ are natural numbers. Two integers are considered to be equal, $a\text{---}b = c\text{---}d$, if and only if $a+d = c+b$. We let $\Z$ denote the set of all integers.
\end{definition}

\begin{definition}
The sum of two integers, $(a\text{---}b)+(c\text{---}d)$, is defined by the formula
\be
(a\text{---}b)+ (c\text{---}d):= (a+c)\text{---}(b+d).
\ee

The product of two integers, $(a\text{---}b)\times (c\text{---}d)$, is defined by
\be
(a\text{---}b)\times (c\text{---}d) := (ac+bd)\text{---}(ad+bc).
\ee
\end{definition}

\begin{lemma}[addition and multiplication of integers are well-defined]
Let $a,a',b,b',c,d$ be natural numbers. If $(a\text{---}b) = (a'\text{---}b')$, then $(a\text{---}b)+(c\text{---}d) = (a'\text{---}b')+(c\text{---}d)$ and $(a\text{---}b)\times (c\text{---}d) = (a'\text{---}b')\times (c\text{---}d)$, and also $(c\text{---}d)+(a\text{---}b) = (c\text{---}d)+ (a'\text{---}b')$ and $(c\text{---}d)\times (a\text{---}b) = (c\text{---}d)\times (a'\text{---}b')$.

Thus addition and multiplication are well-defined operations (equal inputs give equal outputs).
\end{lemma}

\begin{proof}[\bf Proof]
\footnote{need proof}
\end{proof}

\begin{definition}[negation of integers]
If $(a\text{---}b)$ is an integer, we define the negation $-(a\text{---}b)$ to be the integer $(b\text{---}a)$. In particular if $n=n\text{---}0$ is a positive natural number, we can define its negation $-n = 0\text{---}n$.
\end{definition}

\begin{lemma}
Let $x$ be an integer. Then exactly one of the following three statements is true:
\ben
\item [(i)] $x$ is zero;
\item [(ii)] $x$ is equal to a positive natural number $n$;
\item [(iii)] $x$ is negation $-n$ of a positive natural number $n$.
\een
\end{lemma}

\begin{proof}[\bf Proof]
\footnote{need proof}
\end{proof}

\begin{proposition}[laws of algebra for integers]
Let $x,y,z$ be integers. Then we have
\ben
\item [(i)] $x+y = y+x$.
\item [(ii)] $(x+y) + z = x+(y+z)$.
\item [(iii)] $x+0=0+x = x$.
\item [(iv)] $x+(-x) = (-x)+x = 0$.
\item [(v)] $xy = yx$.
\item [(vi)] $(xy)z = x(yz)$.
\item [(vii)] $x1 = 1x = x$.
\item [(viii)] $x(y+z) = xy + xz$.
\item [(ix)] $(y+z)x = yx + zx$.
\een
\end{proposition}

\begin{proof}[\bf Proof]
\footnote{need proof}
\end{proof}

\begin{definition}[subtraction]
We define the operation of subtraction $x-y$ of two integers by the formula
\be
x-y := x+(-y).
\ee
\end{definition}

\begin{proposition}[integers have no zero divisors]
Let $a$ and $b$ be integers such that $ab = 0$. Then either $a=0$ or $b=0$ (or both).
\end{proposition}

\begin{proof}[\bf Proof]
\footnote{need proof}
\end{proof}

\begin{corollary}[cancellation law for integers\index{cancellation law!integers}]
If $a,b,c$ are integers such that $ac = bc$ and $c$ is non-zero, then $a=b$.
\end{corollary}

\begin{proof}[\bf Proof]
\footnote{need proof}
\end{proof}

\begin{definition}[ordering for the integers]
Let $n$ and $m$ be integers. We say that $n$ is greater than or equal to $m$, and write $n\geq m$ or $m\leq n$, iff we have $n=m+a$ for some natural number $a$. We say that $n$ is strictly greater than $m$, and write $n>m$ or $m<n$, iff $n\geq m$ and $n\neq m$.
\end{definition}

\begin{proposition}[properties of order for integers]
Let $a,b,c$ be integers.
\ben
\item [(i)] $a>b$ if and only if $a-b$ is a positive natural number.
\item [(ii)] (addition preserves order). If $a>b$, then $a+c>b+c$.
\item [(iii)] (positive multiplication preserves order). If $a>b$ and $c$ is positive, then $ac >bc$.
\item [(iv)] (negation reverses order). If $a>b$, then $-a<-b$.
\item [(v)] (order is transitive). If $a>b$ and $b>c$, then $a>c$.
\item [(vi)] (order trichotomy). Exactly one of the statements is true: $a>b$, $a<b$, or $a=b$.
\een
\end{proposition}

\begin{proof}[\bf Proof]
\footnote{need proof}
\end{proof}

\subsection{Rational numbers}

\begin{definition}[rational numbers\index{rational numbers}]
A rational number is an expression of the form $a//b$, where $a$ and $b$ are integers and $b$ is non-zero; $a//0$ is not considered to be a rational number.

Two rational numbers are considered to be equal, $a//b = c//d$, if and only if $ad = cb$.

The set of all rational numbers is denoted $\Q$.
\end{definition}

\begin{definition}
If $a//b$ and $c//d$ are rational numbers, we define their sum
\be
(a//b) + (c//d) := (ad+bc)//(bd),
\ee
their product
\be
(a//b) \times (c//d) := (ac)//(bd)
\ee
and the negation
\be
-(a//b) := (-a)//b.
\ee
\end{definition}

\begin{lemma}
The sum, product, and negation operations on rational numbers are well-defined, in the sense that if one replaces $a//b$ with another rational number $a'//b'$ which is equal to $a//b$, then the output of the above operations remains unchanged, and similarly for $c//d$.
\end{lemma}

\begin{proof}[\bf Proof]
\footnote{need proof}
\end{proof}

\begin{definition}[reciprocal\index{reciprocal!rational numbers}]
If $x=a//b$ is non-zero rational (so that $a,b\neq 0$) then we define the reciprocal $x^{-1}$ of $x$ to be the rational number $x^{-1}:=b//a$.
\end{definition}

\begin{proposition}[laws of algebra for rationals]
Let $x,y,z$ be rationals. Then the following laws of algebra hold:
\ben
\item [(i)] $x+y = y+x$.
\item [(ii)] $(x+y) + z = x+(y+z)$.
\item [(iii)] $x+0=0+x = x$.
\item [(iv)] $x+(-x) = (-x)+x = 0$.
\item [(v)] $xy = yx$.
\item [(vi)] $(xy)z = x(yz)$.
\item [(vii)] $x1 = 1x = x$.
\item [(viii)] $x(y+z) = xy + xz$.
\item [(ix)] $(y+z)x = yx + zx$.
\een

If $x$ is non-zero, we also have
\ben
\item [(x)] $xx^{-1} = x^{-1}x = 1$.
\een
\end{proposition}

\begin{proof}[\bf Proof]
\footnote{need proof}
\end{proof}

\begin{definition}[division\index{division!rational numbers} (quotient\index{quotient!rational numbers})]
We define the division (quotient) $x/y$ of two rational numbers $x$ and $y$, provided that $y$ is non-zero, by the formula
\be
x/y := x\times y^{-1}.
\ee
\end{definition}

\begin{definition}[positive rational numbers\index{positive!rational numbers}, negative rational numbers\index{negative!rational numbers}]
A rational number $x$ is said to be positive iff we have $x = a/b$ for some positive integers $a$ and $b$. It is said to be negative iff we have $x=-y$ for some positive rational $y$ (i.e., $x= (-a)/b$ for some positive integers $a$ and $b$).
\end{definition}

\begin{lemma}[trichotomy of rationals]
Let $x$ be rational number. Then exactly one of the following three statements is ture:
\ben
\item [(i)] $x$ is equal to 0.
\item [(ii)] $x$ is a positive rational number.
\item [(iii)] $x$ is a negative rational number.
\een
\end{lemma}

\begin{definition}[ordering of the rational numbers]
Let $x$ and $y$ be rational numbers. We say that $x>y$ iff $x-y$ is positive rational number, and $x<y$ iff $x-y$ is a negative rational number. We write $x\geq y$ iff either $x>y$ or $x=y$, and similarly define $x\leq y$.
\end{definition}

\begin{proposition}[basic properties of order on the rationals]
Let $x,y,z$ be rational numbers. Then the following properties hold.
\ben
\item [(i)] (order trichotomy). Exactly one of the three statements $x=y$, $x<y$, or $x>y$ is true.
\item [(ii)] (order is anti-symmetric). One has $x<y$ if and only if $y>x$.
\item [(iii)] (order is transitive). If $x<y$ and $y<z$, then $x<z$.
\item [(iv)] (addition preserves order). If $x<y$, then $x+z < y+z$.
\item [(v)] (positive multiplication preserves order). If $x<y$ and $z$ is positive, then $xz < yz$.
\een
\end{proposition}

\begin{remark}%The rationals $\Q$ form an ordered field.
Note that (v) only works when $z$ is positive\footnote{need proof}.
\end{remark}

%\begin{remark}\end{remark}

\subsection{Absolute value and exponentiation}

\begin{definition}[absolute value\index{absolute value!rational numbers}]
If $x$ is a rational number, then absolute value $\abs{x}$ of $x$ is defined as follows. If $x$ is positive, then $\abs{x} := x$. If $x$ is negative, then $\abs{x}:=-x$. If $x$ is zero, then $\abs{x} = 0$.
\end{definition}

\begin{definition}[distance\index{distance!rational number}]
Let $x$ and $y$ be rational numbers. Then quantity $\abs{x-y}$ is called the distance between $x$ and $y$ and is sometimes denoted $d(x,y)$, thus $d(x,y):=\abs{x-y}$. For instance, $d(3,5) = d(5,3) = 2$.
\end{definition}

\begin{proposition}[basic properties of absolute value and distance]
Let $x,y,z$ be rational numbers.
\ben
\item [(i)] (non-degeneracy of absolute value). We have $\abs{x}\geq 0$. Also, $\abs{x} = 0$ if and only if $x$ is 0.
\item [(ii)] (triangle inequality for absolute value). We have $\abs{x+y} \leq \abs{x} + \abs{y}$.
\item [(iii)] We have the inequalities $-y \leq x\leq y$ if and only if $y\geq \abs{x}$. In particular, we have $-\abs{x}\leq x\leq \abs{x}$.
\item [(iv)] (multiplicativity of absolute value). We have $\abs{xy} = \abs{x}\abs{y}$. In particular, $\abs{-x} = \abs{x}$.
\item [(v)] (non-degeneracy of distance). We have $d(x,y) \geq 0$. Also, $d(x,y) = 0$ if and only if $x=y$.
\item [(vi)] (symmetry of distance). $d(x,y) = d(y,x)$.
\item [(vii)] (triangle inequality for distance). $d(x,z) \leq d(x,y) + d(y,z)$.
\een
\end{proposition}

\begin{definition}[$\ve$-closeness]
Let $\ve >0$\footnote{rational?}, and $x,y$ be rational numbers. We say that $y$ is $\ve$-close to $x$ iff we have $d(y,x) \leq \ve$.
\end{definition}

\begin{remark}
This definition is not standard in mathematics textbook; we will use it as `scaffolding' to construct the more important notions of limits (and of Cauchy sequences)\footnote{give link later}, and once we have those more advanced notions we will discard the notion of $\ve$-close.
\end{remark}

\begin{proposition}
Let $x,y,z,w$ be rational numbers. Also, let $\ve,\delta >0$\footnote{rational?}
\ben
\item [(i)] If $x=y$, then $x$ is $\ve$-close to $y$ for every $\ve>0$, then we have $x=y$.
\item [(ii)] If $x$ is $\ve$-close to $y$, then $y$ is $\ve$-close to $x$.
\item [(iii)] If $x$ is $\ve$-close to $y$, and $y$ is $\delta$-close to $z$, then $x$ and $z$ are $(\ve + \delta)$-close.
\item [(iv)] If $x$ and $y$ are $\ve$-close, and $z$ and $w$ are $\delta$-close, then $x+z$ and $y+w$ are $(\ve+\delta)$-close, and $x-z$ and $y-w$ are also $(\ve+\delta)$-close.
\item [(v)] If $x$ and $y$ are $\ve$-close, they are also $\ve'$-close for every $\ve'>\ve$.
\item [(vi)] If $y$ and $z$ are both $\ve$-close to $x$, and $w$ is between $y$ and $z$ (i.e., $y\leq w\leq z$ or $z\leq w\leq y$), then $w$ is also $\ve$-close to $x$.
\item [(vii)] If $x$ and $y$ are $\ve$-close, and $z$ is non-zero, then $xz$ and $yz$ are $\ve\abs{z}$-close.
\item [(viii)] If $x$ and $y$ are $\ve$-close, and $z$ and $w$ are $\delta$-close, then $xz$ and $yw$ are $(\ve \abs{z} + \delta \abs{x} + \ve\delta)$-close.
\een
\end{proposition}

\begin{definition}[exponentiation to a natural number\index{exponentiation!natural number}]\label{def:exponentiation_natural_number}
Let $x$ be a rational number. To raise $x$ to the power 0, we define $x^0 := 1$. Now suppose inductively that $x^n$ has been defined for some natural number $n$, then we define $x^{n+1} := x^n \times x$.
\end{definition}

\begin{proposition}[properties of exponentiation, I]
Let $x,y$ be rational numbers, and let $n,m$ be natural numbers.
\ben
\item [(i)] We have $x^nx^m = x^{n+m}$, $(x^n)^m = x^{nm}$ and $(xy)^n = x^ny^n$.
\item [(ii)] We have $x^n = 0$ if and only if $x=0$.
\item [(iii)] If $x\geq y \geq 0$, then $x^n \geq y^n \geq 0$. If $x> y \geq 0$ and $n>0$, then $x^n > y^n \geq 0$.
\item [(iv)] We have $\abs{x^n} = \abs{x}^n$.
\een
\end{proposition}

\begin{definition}[exponentiation to a negative integer\index{exponentiation!negative integer}]\label{def:exponentiation_negative_integer}
Let $x$ be a non-zero rational number. Then for any negative integer $-n$, we define $x^{-n} := 1/x^n $.
\end{definition}

\begin{proposition}[properties of exponentiation, II]
Let $x,y$ be non-zero rational numbers, and let $n,m$ be integers.
\ben
\item [(i)] We have $x^nx^m = x^{n+m}$, $(x^n)^m = x^{nm}$ and $(xy)^n = x^ny^n$.
\item [(ii)] If $x\geq y > 0$, then $x^n \geq y^n > 0$ if $n$ is positive, and $0<x^n \leq y^n$ if $n$ is negative.
\item [(iii)] If $x,y >0$, $n\neq 0$, and $x^n = y^n$, then $x=y$.
\item [(iv)] We have $\abs{x^n} = \abs{x}^n$.
\een
\end{proposition}

\subsection{Gaps in the rational numbers}

\begin{proposition}[interspersing of integers by rationals]
Let $x$ be a rational number. Then there exists an integer $n$ such that $n\leq x< n+1$. In fact, this integer is unique (i.e., for each $x$ there is only one $n$ for which $n\leq x<n+1$). In particular, there exists a natural number $N$ such that $N>x$ (i.e., there is no such thing as a rational number which is larger than all the natural numbers).

The integer $n$ for which $n\leq x < n+1$ is sometimes referred to as the integer part\index{integer part!rational number} of $x$ and is sometimes denoted $n=\floor{x}$.
\end{proposition}

\begin{proposition}[interspering of rationals by rationalys]
If $x$ and $y$ are two rationals such that $x<y$, then exists a third rational $z$ such that $x<z<y$.
\end{proposition}

\begin{proposition}
There does not exist any rational number $x$ for which $x^2 = 2$.
\end{proposition}

On the other hand, we can get rational numbers which are arbitrarily close to a square root of 2.

\begin{proposition}
For every rational number $\ve>0$, there exists a non-negative rational number $x$ such that $x^2 <2<(x+\ve)^2$.
\end{proposition}

\section{Real Numbers}

What is the difference between $\R$ and $\Q$ which makes calculus work on one even though it fails on the other? Both are `ordered fields', that is, both support operations of `addition' and `multiplication' together with a relation `greater than' (`order') with the properties that we expect.

To state the difference we need the following definition (by recalling definition of sequence\footnote{need link})


\subsection{Absolute value}

\begin{proposition}\label{pro:real_number_inequality_absolute_value}
For space $X\subseteq \R$, we have for $x,y\in X$,
\ben
\item [(i)] $\abs{x+y} \leq \abs{x} + \abs{y}$.
\item [(ii)] $\abs{\abs{x}-\abs{y}} \leq \abs{x-y}$.
\een
\end{proposition}

\begin{proof}[\bf Proof]
\ben
\item [(i)] Since $xy\leq \abs{xy} = \abs{x}\abs{y}$, we have
\be
\abs{x+y}^2 = x^2 + 2xy + y^2 \leq x^2 + 2\abs{x}\abs{y} + y^2 = \bb{\abs{x}+\abs{y}}^2
\ee
which implies the required result.

\item [(ii)] By (i), we have
\be
\abs{x} = \abs{(x-y)+y} \leq \abs{x-y} + \abs{y} \ \ra\ \abs{x} -\abs{y} \leq \abs{x-y}.
\ee

Then switch $x$ and $y$ we have
\be
\abs{y} - \abs{x} \leq \abs{x-y} \ \ra\ \abs{x} -\abs{y} \geq -\abs{x-y}.
\ee

Therefore,
\be
-\abs{x-y} \leq \abs{x} -\abs{y} \leq \abs{x-y} \ \ra\ \abs{\abs{x} -\abs{y}} \leq \abs{x-y}.
\ee
\een
\end{proof}

\subsection{Floor function, ceiling function and sawtooth function}

\begin{definition}[floor function\index{floor function}, ceiling function\index{ceiling function}, sawtooth function\index{sawtooth function}]
Let $x$ be a real number. Then floor function and ceiling function of $x$ are defined by
\be
\floor{x} = \max\bra{n\in \Z:n\leq x},\qquad \ceil{x} = \min\bra{n\in\Z: n\geq x}.
\ee

The sawtooth function it the fractional part of $x$, written by $\bra{x}$ is defined to be $x-\floor{x}$.
\end{definition}



\subsection{Equivalent Cauchy sequences}

\subsection{The construction of the real numbers}

\subsection{Ordering the reals}

\section{Summary}

existsence of least upper bound (Theorem 5.5.9, Dedekind completeness) $\lra$ Archimedean property (Corollary 5.4.13 in Tao's book) + Cauchy completeness (Cauchy sequence converges)











