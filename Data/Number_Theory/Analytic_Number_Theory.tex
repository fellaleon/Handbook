\chapter{Analytic Number Theory}


\section{Basic functions in number theory}

\subsection{Greatest common divisor}

\begin{definition}[greatest common divisor (gcd)]
\footnote{definition needed.}
\end{definition}

\begin{proposition}
Let $a,b$ be two positive integers, then
\be
\gcd\bb{a,b} = \frac 1a \sum^{a-1}_{m=0}\sum^{a-1}_{n=0} \exp\bb{\frac{2\pi i mn b}{a}}.
\ee
\end{proposition}

\begin{theorem}
\be
\gcd(k,n) = \sum^n_{m=1} \exp\bb{\frac{2\pi i km}n}\sum_{d\mid n}\frac{c_d(m)}{d}.
\ee
\end{theorem}

\subsection{Gcd-sum function}

\begin{definition}[gcd-sum function]
\be
g(n) = \sum^n_{k=1}\gcd(k,n).
\ee
\end{definition}

\begin{theorem}
For every prime number $p$ and positive integer $\alpha\geq 1$,
\be
g\bb{p^{\alpha}} = \bb{\alpha+1}p^{\alpha} - \alpha p^{\alpha -1}.
\ee
\end{theorem}

\begin{proof}[\bf Proof]
\footnote{proof needed.}
\end{proof}


\section{Orders of Arithmetical Functions}

\begin{definition}[prime count function]
$\pi(x)$\footnote{definition needed.}
\end{definition}

\section{Distribution of Primes}

\subsection{Sum and product of primes}

\begin{theorem}
For all primes $p$, 
\be
\sum_p p^{-1},\quad \prod_p\frac 1{1-p^{-1}}\quad  \text{ are both divergent.}
\ee
\end{theorem}

\begin{proof}[\bf Proof]
\footnote{proof needed.}
\end{proof}